\chapter{Basic Topology}\label{chap:basic-topology}
This chapter discusses basic notions of point set topology, which focuses on the metric space and its related structures. Then we introduce compactness and prove three major results (\cref{thrm:cantor-intersection}, \cref{thrm:heine-borel}, \cref{thrm:bolzano-weierstrass}). We also briefly talk about perfect sets, and connectedness of sets.

\section{Metric Space}
\subsection{Definitions and Examples}
\begin{definition}[Metric space]
A \vocab{metric space}\index{metric space} is a set $X$ with an associated \emph{metric} $d:X\times X\to\RR$, which satisfies the following properties for all $p,q\in X$:
\begin{enumerate}[label=(\roman*)]
\item Positive definitiveness: $d(p,q)\ge0$, where equality holds if and only if $p=q$;
\item Symmetry: $d(p,q)=d(q,p)$;
\item Triangle inequality: $d(p,q)\le d(p,r)+d(r,q)$ for any $r\in X$.
\end{enumerate}
\end{definition}

For the rest of the chapter, $X$ is taken to be a metric space, unless specified otherwise.

\begin{example}[Metrics on $\RR^n$]
Each of the following functions define metrics on $\RR^n$.
\begin{align*}
d_1(x,y)&=\sum_{i=1}^{n}|x_i-y_i|;\\
d_2(x,y)&=\sqrt{\sum_{i=1}^{n}(x_i-y_i)}\\
d_\infty(x,y)&=\max_{i\in\{1,2,\dots,n\}}|x_i-y_i|.
\end{align*}
These are called the $\ell^1$-, $\ell^2$- (or Euclidean) and $\ell^\infty$-distances respectively.

The proof that each of $d_1$, $d_2$, $d_\infty$ is a metric is mostly very routine, with the exception of proving that $d_2$, the Euclidean distance, satisfies the triangle inequality. To establish this, recall that the Euclidean norm $\norm{x}_2$ of a vector $x=(x_1,\dots,x_n)\in\RR^n$ is
\[\norm{x}_2\coloneqq\brac{\sum_{i=1}^n{x_i}^2}^\frac{1}{2}=\langle x,x\rangle^\frac{1}{2},\]
where the inner product is given by
\[\langle x,y\rangle\coloneqq\sum_{i=1}^{n}x_i y_i.\]
Then $d_2(x,y)=\norm{x-y}_2$, and so the triangle inequality is the statement that
\[\norm{w-y}_2\le\norm{w-x}_2+\norm{x-y}_2.\]
This follows immediately by taking $u=w-x$ and $v=x-y$ in the following lemma.

\begin{lemma}
If $u,v\in\RR^n$ then $\norm{u+v}_2\le\norm{u}_2+\norm{v}_2$.
\end{lemma}

\begin{proof}
Since $\norm{u}_2\ge0$ for all $u\in\RR^n$, squaring both sides of the desired inequality gives
\[{\norm{u+v}_2}^2\le{\norm{u}_2}^2+2\norm{u}_2\norm{v}_2+{\norm{v}_2}^2.\]
But since
\[{\norm{u+v}_2}^2=\langle u+v,u+v\rangle={\norm{u}_2}^2+2\langle u,v\rangle+\norm{v}_2^2,\]
this inequality is immediate from the Cauchy--Schwarz inequality, that is to say the inequality
\[|\langle u,v\rangle|\le\norm{u}_2\norm{v}_2.\]
\end{proof}
\end{example}

The following are a few interesting examples of metrics.

\begin{example}[Discrete metric]
The \textbf{discrete metric} on an arbitrary set $X$ is defined as follows:
\[d(x,y)=\begin{cases}
1&\text{if }x\neq y,\\
0&\text{if }x=y.
\end{cases}\]
\end{example}

\begin{example}[2-adic metric]
On $\ZZ$, define $d(x,y)$ to be $2^{-m}$, where $2^m$ is the largest power of two dividing $x-y$. The triangle inequality holds in the following stronger form, known as the ultrametric property:
\[d(x,z)\le\max\{d(x,y),d(y,z)\}.\]
Indeed, this is just a rephrasing of the statement that if $2^m$ divides both $x-y$ and $y-z$, then $2^m$ divides $x-z$.

This metric is very unlike the usual distance. For example, $d(999,1000) = 1$, whilst $d(0,1000)=\frac{1}{8}$.

The role of $2$ can be replaced by any other prime $p$, and the metric may also be
extended in a natural way to the rationals $\QQ$.
\end{example}

\begin{example}[Path metric]
Let $G$ be a graph, that is to say a finite set of vertices $V$ joined by edges. Suppose that $G$ is connected, that is to say that there is a path joining any pair of distinct vertices. Define a distance $d$ as follows: $d(v,v)=0$, and $d(v,w)$ is the length of the shortest path from $v$ to $w$. Then $d$ is a metric on $V$, as can be easily checked.
\end{example}

\begin{example}[Word metric]
Let $G$ be a group, and suppose that it is generated by elements $a$, $b$ and their inverses. Define a distance on $G$ as follows: $d(v,w)$ is the minimal $k$ such that $v=wg_1\cdots g_k$, where $g_i\in\{a,b,a^{-1},b^{-1}\}$ for all $i$.
\end{example}

\begin{example}[Hamming distance]
Let $X=\{0,1\}^n$ (the boolean cube), the set of all strings of $n$ zeroes and ones. Define $d(x,y)$ to be the number of coordinates
in which $x$ and $y$ differ.
\end{example}

\begin{example}[Projective space]
Consider the set $P(\RR^n)$ of one-dimensional subspaces of $\RR^n$, that is to say lines through the origin. One way to define a distance on this set is to take, for lines $L_1,L_2$, the distance between $L_1$ and $L_2$ to be
\[d(L_1,L_2)=\sqrt{1-\frac{|\langle v,w\rangle|^2}{\norm{v}^2\norm{w}^2}},\]
where $v$ and $w$ are any non-zero vectors in $L_1$ and $L_2$ respectively.

When $n=2$, the distance between two lines is $\sin\theta$ where $\theta$ is the angle between those lines.
\end{example}

\begin{comment}
\subsection{Norms}
\begin{definition}[Norms]
Let $V$ be any vector space (over the reals). A function $\norm{\cdot}:V\to[0,\infty)$ is called a \vocab{norm}\index{norm} if it satisfies the following properties:
\begin{enumerate}[label=(\roman*)]
\item $\norm{x}=0$ if and only if $x=0$;
\item $\norm{\lambda x}=|\lambda|\norm{x}$ for all $\lambda\in\RR$, $x\in V$;
\item $\norm{x+y}\le\norm{x}+\norm{y}$ for all $x,y\in V$.
\end{enumerate}
\end{definition}

Given a norm, it is very easy to check that $d(x,y)\coloneqq\norm{x-y}$ defines a metric on $V$. Indeed, we have already seen that when $V=\RR^n$, $\norm{\cdot}_2$ is a norm (and so the name ``Euclidean norm'' is appropriate) and we defined $d_2(x,y)=\norm{x-y}_2$. The other metrics on $\RR^n$ also come from norms: $d_1$ comes from the $\ell^1$-norm
\[\norm{x}_1\coloneqq\sum_{i=1}^n|x_i|,\]
whilst $d_\infty$ comes from the $\ell^\infty$-norm
\[\norm{x}_\infty\coloneqq\max_{i=1,\dots,n}|x_i|.\]
More generally, the family of $\ell^p$-norms are given by
\[\norm{x}_p\coloneqq\brac{\sum_{i=1}^n|x_i|^p}^\frac{1}{p}.\]

The principle of turning norms into metrics is important enough that we state it as a lemma in its own right.

\begin{lemma}
Let $V$ be a vector space over $\RR$, let $\norm{\cdot}$ be a norm on $V$. Define $d:V\times V\to[0,\infty)$ by $d(x,y)\coloneqq\norm{x-y}$. Then $(V,d)$ is a metric space.
\end{lemma}

\begin{proof}
Simply verify the three axioms for a metric space, which directly correspond to the three axioms for a norm.
\end{proof}

\begin{remark}
The converse is very far from true. For instance, the discrete metric does not arise from a norm. All metrics arising from a norm have the \emph{translation invariance property} $d(x+z,y+z)=d(x,y)$, as well as the \emph{scalar invariance} $d(\lambda x,\lambda y)=|\lambda|d(x,y)$, neither of which are properties of arbitrary metrics.

Conversely one can show that a metric with these two additional properties does come from a norm, an exercise we leave to the reader. [Hint: the norm must arise as $\norm{v}=d(v,0)$.]
\end{remark}

We call a vector space endowed with a norm $\norm{\cdot}$ a \vocab{normed space}. Whenever we talk about normed spaces it is understood that we are also thinking of them as metric spaces, with the metric being defined by $d(v,w)=\norm{v-w}$.

Note that we do not assume that the underlying vector space $V$ is finite dimensional. Here are some examples which are not finite-dimensional (whilst we do not prove that they are not finite-dimensional here, it is not hard to do so and we suggest this as an exercise).

\begin{example}[$\ell^p$ spaces]
Let
\begin{align*}
\ell_1&=\crbrac{(x_n)_{n=1}^\infty\:\bigg|\:\sum_{n\ge1}|x_n|<\infty},\\
\ell_2&=\crbrac{(x_n)_{n=1}^\infty\:\bigg|\:\sum_{n\ge1}x_n^2<\infty},\\
\ell_\infty=&\crbrac{(x_n)_{n=1}^\infty\:\bigg|\:\sup_{n\in\NN}|x_n|<\infty}.
\end{align*}
The sets $\ell_1$, $\ell_2$, $\ell_\infty$ are all real vector spaces, and moreover
\begin{align*}
\norm{(x_n)}_1&=\sum_{n\ge1}|x_n|,\\
\norm{(x_n)}_2&=\brac{\sum_{n\ge1}x_n^2}^\frac{1}{2},\\
\norm{(x_n)}_\infty&=\sup_{n\in\NN}|x_n|
\end{align*}
define norms on $\ell_1$, $\ell_2$ and $\ell_\infty$ respectively.

Note that $\ell_2$ is in fact an inner product space where
\[\langle(x_n),(y_n)\rangle=\sum_{n\ge1}x_ny_n,\]
(the fact that the right-hand side converges if $(x_n)$ and $(y_n)$ are in $\ell_2$ follows from the Cauchy--Schwarz inequality). The space $\ell^2$ is known as \vocab{Hilbert space}.
\end{example}
\end{comment}

A metric space $(X,d)$ naturally induces a metric on any of its subsets.

\begin{definition}[Subspace]
Suppose $(X,d)$ is a metric space, $Y\subset X$. Then the restriction of $d$ to $Y\times Y$ gives $Y$ a metric so that $(Y,d_{Y\times Y})$ is a metric space. We call $Y$ equipped with this metric a \vocab{subspace}.
\end{definition}

\begin{example}
Subspaces of $\RR$ include $[0,1]$, $\QQ$, $\ZZ$.
\end{example}

\begin{proposition}[Product space]
If $(X,d_X)$ and $(Y,d_Y)$ are metric spaces, set
\[d_{X\times Y}\brac{(x_1,y_1),(x_2,y_2)}=\sqrt{d_X(x_1,x_2)^2+d_Y(y_1,y_2)^2.}\]
for $x_1,x_2\in X$, $y_1,y_2\in Y$. Then $d_{X\times Y}$ gives a metric on $X\times Y$; we call $X\times Y$ the \emph{product space}.
\end{proposition}

\begin{proof}
Reflexivity and symmetry are obvious. Less clear is the triangle inequality. We need to prove that
\begin{equation*}\tag{1}
\begin{split}
&\sqrt{d_X(x_1,x_3)^2+d_Y(y_1,y_3)^2}+\sqrt{d_X(x_3,x_2)^2+d_Y(y_3,y_2)^2}\\
&\ge\sqrt{d_X(x_1,x_2)^2+d_Y(y_1,y_2)^2}
\end{split}
\end{equation*}
Write $a_1=d_X(x_2,x_3)$, $a_2=d_X(x_1,x_3)$, $a_3=d_X(x_1,x_2)$ and similarly $b_1=d_Y(y_2,y_3)$, $b_2=d_Y(y_1,y_3)$ and $b_3=d_Y(y_1,y_2)$. Thus we want to show
\begin{equation*}\tag{2}
\sqrt{{a_2}^2+{b_2}^2}+\sqrt{{a_1}^2+{b_1}^2}\ge\sqrt{{a_3}^2+{b_3}^2}.
\end{equation*}
To prove this, note that from the triangle inequality we have $a_1+a_2\ge a_3$, $b_1+b_2\ge b_3$. Squaring and adding gives
\[{a_1}^2+{b_1}^2+{a_2}^2+{b_2}^2+2(a_1a_2+b_1b_2)\ge {a_3}^2+{b_3}^2.\]
By Cauchy--Schwarz,
\[a_1a_2+b_1b_2\le\sqrt{{a_1}^2+{b_1}^2}\sqrt{{a_2}^2+{b_2}^2}.\]
Substituting this into the previous line gives precisely the square of (2), and (1) follows.
\end{proof}

\subsection{Balls and Boundedness}
\begin{definition}[Balls]\index{balls}
The \vocab{open ball}\index{balls!open ball} centred at $x\in X$ with radius $r>0$ is defined to be the set
\[B_r(x)\coloneqq\{y\in X\mid d(x,y)<r\}.\]
Similarly the \vocab{closed ball}\index{balls!closed ball} centred at $x$ with radius $r$ is
\[\overline{B}_r(x)\coloneqq\{y\in X\mid d(x,y)\le r\}.\]
The \vocab{punctured ball}\index{balls!punctured ball} is the open ball excluding its centre:
\[B_r(x)\setminus\{x\}=\{y\in X\mid 0<d(x,y)<r\}.\]
\end{definition}

\begin{example}
Considering $\RR^3$ with the Euclidean metric, $B_1(0)$ really is what we understand geometrically as a ball (minus its boundary, the unit sphere), whilst $\overline{B}_1(0)$ contains the unit sphere and everything inside it.
\end{example}

\begin{remark}
We caution that this intuitive picture of the closed ball being the open ball ``together with its boundary'' is totally misleading in general. For instance, in the discrete metric on a set $X$, the open ball $B_1(a)$ contains only the point $a$, whereas the closed ball $\overline{B}_1(a)$ is the whole of $X$.
\end{remark}

\begin{definition}[Bounded]
$E\subset X$ is said to be \vocab{bounded}\index{boundedness} if $E$ is contained in some open ball; that is, there exists $M\in\RR$ and $q\in X$ such that $d(p,q)<M$ for all $p\in E$.
\end{definition}

\begin{proposition}
Let $E\subset X$. Then the following are equivalent:
\begin{enumerate}[label=(\roman*)]
\item $E$ is bounded;
\item $E$ is contained in some closed ball;
\item The set $\{d(x,y)\mid x,y\in E\}$ is a bounded subset of $\RR$.
\end{enumerate}
\end{proposition}

\begin{proof} \

(1)$\implies$(2) This is obvious.

(2)$\implies$(3) This follows immediately from the triangle inequality.

(3)$\implies$(1) Suppose $E$ satisfies (iii), then there exists $r\in\RR$ such that $d(x,y)\le r$ for all $x,y\in E$. If $E=\emptyset$, then $E$ is certainly bounded. Otherwise, let $p\in E$ be an arbitrary point. Then $E\subset B_{r+1}(p)$.
\end{proof}

\subsection{Open and Closed Sets}
\begin{definition}[Neighbourhood]
$N\subset X$ is called a \vocab{neighbourhood}\index{neighbourhood} of $p\in X$ if $B_\delta(p)\subset N$ for some $\delta>0$.
\end{definition}

\begin{definition}[Open set]
$E\subset X$ is \vocab{open}\index{open set} (in $X$) if it is a neighbourhood of each of its elements; that is, for all $x\in E$, $B_\delta(x)\subset E$ for some $\delta>0$.
\end{definition}

\begin{proposition}
Any open ball is open.
\end{proposition}

\begin{proof}
Let $B_r(x)$ be an open ball. Then for any point $y\in B_r(x)$, there is $d(y,x)<r$. Take $\delta=r-d(y,x)$, which is positive.

Consider the ball $B_\delta(y)$. We shall show it lives in $B_r(x)$. For this, take any point $z\in B_\delta(y)$. By the triangle inequality, we have
\begin{align*}
d(z,x)&\le d(z,y)+d(y,x)\\
&<\delta+d(y,x)\\
&=r.
\end{align*}
and so $z\in B_r(x)$. Since for all $y\in B_r(x)$ there exists $\delta>0$ such that $B_\delta(y)\subset B_r(x)$, we have that $B_r(x)$ is open.
\end{proof}

\begin{proposition}\label{prop:open-set-properties}
\begin{enumerate}[label=(\roman*)]
\item Both $\emptyset$ and $X$ are open.
\item For any indexing set $I$ and collection of open sets $\{E_i\mid i\in I\}$, $\bigcup_{i\in I}E_i$ is open.
\item For any \emph{finite} indexing set $I$ and collection of open sets $\{E_i\mid i\in I\}$, $\bigcap_{i\in I}E_i$ is open.
\end{enumerate}
\end{proposition}

\begin{proof} \
\begin{enumerate}[label=(\roman*)]
\item Obvious by definition.
\item If $ x\in\bigcup_{i\in I}E_i$ then there is some $i\in I$ with $x\in E_i$. Since $E_i$ is open, there exists $\delta>0$ such that $B_\delta(x)\subset E_i$ and hence $ B_\delta(x)\in\bigcup_{i\in I}E_i$.
\item Suppose that $I$ is finite and that $ x\in\bigcap_{i\in I}E_i$. For each $i\in I$, we have $x\in E_i$ and so there exists $\delta_i$ such that $B_{\delta_i}(x)\subset E_i$. Set $\delta=\min_{i\in I}\delta_i$, then $\delta>0$ (here it is, of course, crucial that $I$ be finite), and $B_\delta(x)\subset B_{\delta_i}(x)\subset E_i$ for all $i$. Therefore $ B_\delta(x)\subset\bigcap_{i\in I}E_i$.
\end{enumerate}
\end{proof}

\begin{remark}
(1) is in fact a special case of (2) and (3), taking $I$ to be the empty set.
\end{remark}

\begin{remark}
It is extremely important to note that, whilst the indexing set $I$ in (2) can be arbitrary, the indexing set in (3) must be finite. In general, an arbitrary intersection of open sets is not open; for instance, the intervals $E_i=\brac{-\frac{1}{i},\frac{1}{i}}$ are all open in $\RR$, but their intersection $\bigcap_{i=1}^\infty E_i=\{0\}$, which is not an open set.
\end{remark}

\begin{proposition}\label{prop:open-subspace-cap}
Suppose $Y$ is a subspace of $X$. $E\subset Y$ is open relative to $Y$ if and only if $E=Y\cap G$ for some open subset $G$ of $X$.
\end{proposition}

\begin{proof} \

\fbox{$\implies$} Suppose $E$ is open relative to $Y$. Then for each $p\in E$ there exists $r_p>0$ such that the conditions $d(p,q)<r_p$, $q\in Y$ imply $q\in E$.

For each $p\in E$, let the open ball
\[V_p=\{q\in X\mid d(p,q)<r_p\},\]
and define
\[G=\bigcup_{p\in E}V_p.\]
Since $G$ is an intersection of open balls and open balls are open sets, by \cref{prop:open-set-properties}, $G$ is an open subset of $X$. Since $p\in V_p$ for all $p\in E$, it is clear that $E\subset G\cap Y$.

To show the opposite containment, by our choice of $V_p$, we have $V_p\cap Y\subset E$ for every $p\in E$, so that $G\cap Y\subset E$. Hence $E=G\cap Y$.

\fbox{$\impliedby$} Conversely, if $G$ is open in $X$ and $E=G\cap Y$, every $p\in E$ has a neighbourhood $V_p\cap Y\subset E$. Hence $E$ is open relative to $Y$.
\end{proof}

The complement of an open set is a closed set.

\begin{definition}[Closed set]
$E\subset X$ is \vocab{closed}\index{closed set} if its complement $E^c=X\setminus E$ is open.
\end{definition}

%\begin{example}
%The closed interval $[a,b]$, $a\le b$ is closed in $\RR$.
%\end{example}

\begin{proposition}
Any closed ball is closed.
\end{proposition}

\begin{proof}
To prove that $\overline{B}_r(x)=\{y\in X\mid d(x,y)\le r\}$ is closed, we need to show that its complement $\overline{B}_r(x)^c=\{y\in X\mid d(x,y)>r\}$ is open. To do so, we need to show that for all $z\in\overline{B}_r(x)^c$, there exists $\delta>0$ such that $B_\delta(z)\subset\overline{B}_r(x)^c$.

Take $\delta>0$ such that $r+\delta<d(x,z)$; that is, $\delta<d(x,z)-r$.

Pick $y\in B_\delta(z)$. Then $d(y,z)<\delta$. But $r+d(y,z)<d(x,z)$ so $r<d(x,z)-d(y,z)\le d(x,y)$ by triangle inequality. Hence we have $r<d(x,y)$, thus $y\in\overline{B}_r(x)^c$ and so $B_\delta(z)\subset\overline{B}_r(x)^c$. Therefore $\overline{B}_r(x)^c$ is open, so $\overline{B}_r(x)$ is closed.
\end{proof}

\begin{proposition}\label{prop:closed-set-properties}
\begin{enumerate}[label=(\roman*)]
\item Both $\emptyset$ and $X$ are closed.
\item For any indexing set $I$ and collection of closed sets $\{F_i\mid i\in I\}$, $\bigcap_{i\in I}F_i$ is closed.
\item For any \emph{finite} indexing set $I$ and collection of closed sets $\{F_i\mid i\in I\}$, $\bigcup_{i\in I}F_i$ is closed.
\end{enumerate}
\end{proposition}

\begin{proof}
From \cref{prop:open-set-properties}, simply take complements and apply de Morgan's laws.
\end{proof}

\begin{remark}
As above, the indexing set in (3) must be finite; for instane, the closed intervals $F_i=\sqbrac{-1+\frac{1}{n},1-\frac{1}{n}}$ are all closed in $\RR$, but their union $\bigcup_{i=1}^\infty F_i=(-1,1)$ is open, not closed.
\end{remark}

\subsection{Interiors, Closures, Limit Points}
\begin{definition}
The \vocab{interior}\index{interior} of $E\subset X$, denoted by $E^\circ$, is defined to be the union of all open subsets of $X$ contained in $E$.

The \vocab{closure}\index{closure} of $E$, denoted by $\overline{E}$, is defined to be the intersection of all closed subsets of $X$ containing $E$.

The set $\overline{E}\setminus E^\circ$ is known as the \vocab{boundary}\index{boundary} of $E$, denoted by $\partial E$. $p$ is a \vocab{boundary point}\index{boundary point} of $E$ if $p\in\partial E$.

A set $E\subset X$ is said to be \vocab{dense}\index{dense} if $\overline{E}=X$.
\end{definition}

Since an arbitrary union of open sets is open, $E^\circ$ is itself an open set, and it is clearly the unique largest open subset of $X$ contained in $E$. If $E$ is itself open then evidently $E=E^\circ$.

Since an arbitrary intersection of closed sets is closed, $\overline{E}$ is the unique smallest closed subset of $X$ containing $E$. If $E$ is itself closed then evidently $E=\overline{E}$.

If $x\in E^\circ$ we say that $x$ is an \vocab{interior point} of $E$. One can also phrase this in terms of neighbourhoods: the interior of $E$ is the set of all points in $E$ for which $E$ is a neighbourhood.

\begin{example}
Consider the closed interval $[a,b]$ in $\RR$; its interior is just the open interval $(a,b)$.

The rationals $\QQ$ are a dense subset of $\RR$.
\end{example}

Let us give a couple of simple characterisations of the closure of a set.

\begin{proposition}
Suppose $E\subset X$. $p\in\overline{E}$ if and only if every open ball $B_\delta(p)$ contains a point of $E$.
\end{proposition}

\begin{proof} \

\fbox{$\implies$} Suppose that $p\in\overline{E}$. Suppose, for a contradiction, that there exists some open ball $B_\delta(p)$ that does not meet $E$, then $B_\delta(p)^c$ is a closed set containing $E$. Therefore $B_\delta(p)^c$ contains $\overline{E}$, and hence it contains $p$, which is obviously nonsense.

\fbox{$\impliedby$} Suppose that every ball $B_\delta(p)$ meets $E$. Suppose, for a contradiction, that $p\notin\overline{E}$. Then since $\overline{E}^c$ is open, there is a ball $B_\delta(p)$ contained in $\overline{E}^c$, and hence in $E^c$, contrary to assumption.
\end{proof}

\begin{remark}
A particular consequence of this is that $E\subset X$ is dense if and only if it meets every open set in $X$.
\end{remark}

We now introduce the notion of limit points.

\begin{definition}[Limit point]
$p\in E$ is a \vocab{limit point}\index{limit point} (or \emph{accumulation point}) of $E$ if every neighbourhood of $p$ contains some $q\neq p$ such that $q\in E$.

The \vocab{induced set}\index{induced set} of $E$, denoted by $E^\prime$, is the set of all limit points of $E$ in $X$.
\end{definition}

\begin{example}
Consider the metric space $\RR$, $a$ and $b$ are limit points $(a,b]$. The limit point set of $(a,b]$ is $[a,b]$, which is also the closure $(a,b]$.

Consider the metric space $\RR^2$. The limit point set of any open ball $B_r(x)$ is the closed ball $\bar{B}_r(x)$, which is also the closure of $B_r(x)$.

Consider $\QQ\subset\RR$. $\QQ^\prime=\bar{\QQ}=\RR$.
\end{example}

Note that we do not necessarily have $E\subset E^\prime$, that is to say it is quite possible for a point $p\in E$ not to be a limit point of $E$. This occurs if there exists some ball $B_\delta(p)$ such that $B_\delta(p)\cap E=\{p\}$; in this case we say that $p$ is an \vocab{isolated point} of $E$.

\begin{proposition}
If $p$ is a limit point of $E$, then every ball of $p$ contains infinitely many points of $E$.
\end{proposition}

\begin{proof}
Prove by contradiction. Suppose there exists $B_r(p)$ which contains only a finite number of points of $E$: $q_1,\dots,q_n$, where $q_i\neq p$ for $i=1,\dots,n$. Define
\[r=\min\{d(p,q_1),\dots,d(p,q_n)\}.\]
The minimum of a finite set of positive numbers is clearly positive, so that $r>0$.

$B_r(p)$ contains no point $q\in E,q\neq p$ so that $p$ is not a limit point of $E$, a contradiction.
\end{proof}

\begin{corollary}
A finite point set has no limit points.
\end{corollary}

\begin{proposition}
Suppose $E\subset X$. $E^\prime$ is a closed subset of $X$.
\end{proposition}

\begin{proof}
To prove that $E^\prime$ is closed, we need to show that the complement $(E^\prime)^c$ is open.

Suppose $p\in (E^\prime)^c$. Then exists a ball $B_\epsilon(p)$ whose intersection with $E$ is either empty or $\{p\}$. We claim that $B_\frac{\epsilon}{2}(p)\subset (E^\prime)^c$. Let $q\in B_\frac{\epsilon}{2}(p)$. If $q=p$, then clearly $q\in (E^\prime)^c$. If $q\neq p$, there is some ball about $q$ which is contained in $B_\epsilon(p)$, but does not contain $p$: the ball $B_\delta(q)$ where $\delta=\min\brac{\frac{\epsilon}{2},d(p,q)}$ has this property. This ball meets $E$ in the empty set, and so $q\in (E^\prime)^c$ in this case too.
\end{proof}

\begin{proposition}
Suppose $E\subset X$. Then $\overline{E}=E\cup E^\prime$.
\end{proposition}

\begin{proof}
We first show the containment $E\cup E^\prime\subset\overline{E}$. Obviously $E\subset\overline{E}$, so we need only show that $E^\prime\subset\overline{E}$. Suppose $p\in\overline{E}^c$. Since $\overline{E}^c$ is open, there is some ball $B_\epsilon(p)$ which lies in $\overline{E}^c$, and hence also in $E^c$, and therefore a cannot be a limit point of $E$. This concludes the proof of this direction.

Now we look at the opposite containment $\overline{E}\subset E\cup E^\prime$. If $p\in\overline{E}$, we saw in Lemma 5.1.5 that there is a sequence  $(x_n)$ of elements of $E$ with $x_n\to p$. If $x_n=p$ for some $n$ then we are done, since this implies that $p\in E$. Suppose, then, that $x_n\neq p$ for all $n$. Let $\epsilon>0$ be given, for sufficiently large $n$, all the $x_n$ are elements of $B_\epsilon(p)\setminus\{p\}$, and they all lie in $E$. It follows that $p$ is a limit point of $E$, and so we are done in this case also.
\end{proof}

\begin{proposition}
Suppose $E\subsetneq\RR$, $E\neq\emptyset$ be bounded above. Let $y=\sup E$. Then $y\in\overline{E}$. Hence $y\in E$ if $E$ is closed.
\end{proposition}

\begin{proof}
If $y\in E$ then $y\in\overline{E}$. Assume $y\notin E$. For every $h>0$ there exists then a point $x\in E$ such that $y-h<x<y$, for otherwise $y-h$ would be an upper bound of $E$. Thus $y$ is a limit point of $E$. Hence $y\in\overline{E}$.
\end{proof}
\pagebreak

%%%%%%%%%%%%%%%%%%%%%%%%%%%%%%

\begin{comment}
\item A point $x$ is an \vocab{exterior point} of $A$ if it is an interior point of $A^c$.
\item $E$ is compact if it is a bounded closed set.

\begin{proposition}
The set of exterior points, $(A^c)^\circ$ is the same as $(\bar{A})^c$.
\end{proposition}

\begin{proof}
\begin{align*}
x \in (A^c)^\circ 
&\iff \exists \epsilon>0 \text{ such that } B(x,\epsilon) \subset A^c \\
&\iff B(x,\epsilon) \cap A = \emptyset \\
&\iff x \notin A \text{ and } B_0(x,\epsilon) \cap A=\emptyset \\
&\iff x \notin A \cup A^\prime = \bar A \\
&\iff x \in (\bar A^c)
\end{align*}
\end{proof}

\begin{proposition}
\begin{enumerate}[label=(\roman*)]
\item $A^\prime$ is closed.
\item $\bar{A}$ is closed, i.e. $\bar{\bar{A}}=\bar{A}$
\end{enumerate}
\end{proposition}

\begin{proof} \
\begin{enumerate}[label=(\roman*)]
\item In order to show that $A^\prime$ is closed, we need to show that if $x$ is a limit point of $A^\prime$, then $x\in A^\prime$, i.e. $x$ is a limit point of $A$.

So we need to show that limit points of $A^\prime$ are always limit points of $A$: 
Let $x$ be a limit point of $A^\prime$, then for all $\epsilon>0$, $B_0(x,\epsilon/2)$ intersects with $A^\prime$ and we may pick $y \in B_0(x,\epsilon/2)\cap A^\prime$

Now here's the tricky part
Since $y \in A^\prime$, y is a limit point of $A$, hence $B_0(y,|y-x|)$ intersects with $A$ and thus we may pick $z \in B_0(y,|y-x|)\cap A$.

We show that $z \in B_0(x,\epsilon)$:
\[ |z-x|\le|z-y|+|y-x|<2|y-x|<\epsilon, \]
hence $z \in B(x,\epsilon)$.
\[ |z-y|<|x-y|, \]
hence $z \neq x$

$\therefore\:z \in B_0(x,\epsilon)$

\item 
\end{enumerate}
\end{proof}


%%%%%%%%%%%%%%%%%%%%%%%%

\begin{theorem}[Cantor's Intersection Theorem]
Given a decreasing sequence of compact sets $A_1\supset A_2 \supset \cdots$, there exists a point $x\in\RR^n$ such that $x$ belongs to all $A_i$. In other words, $\bigcap_{i=1}^\infty A_i\neq\emptyset$. Moreover, if for all $i\in\NN$ we have $\diam A_{i+1}\le c\cdot\diam A_k$ for some constant $c<1$, then such a point must be unique, i.e. $\bigcap_{i=1}^\infty A_k=\{x\}$ for some $x\in\RR^n$.
\end{theorem}

\begin{theorem}[Heine--Borel Theorem]
A set $A\subset\RR^n$ is compact if and only if every open covering has a finite subcover, i.e. for any family of open sets $\mathcal{U}=\{U_i\}_{i\in I}$ satisfying $A\subset\bigcup_{i\in I}U_i$, there exists $\{U_1,\dots,U_n\}\subset\mathcal{U}$ such that $A\subset\bigcup_{i=1}^n U_i$.
\end{theorem}

\begin{theorem}[Bolzano--Weierstrass Theorem]
Infinite bounded sets in $\RR^n$ must contain limit points.
\end{theorem}

We will follow a very specific sequence of steps to prove these three theorems:
\begin{enumerate}[label=(\alph*)]
\item Cantor Intersection for $n=1$
\item Bolzano--Weierstrass for $n=1$
\item Bolzano--Weierstrass for general $n$
\item Cantor Intersection for general $n$
\item Heine--Borel for general $n$
\end{enumerate}

\begin{proof} \
\begin{enumerate}[label=(\alph*)]
\item Suppose that there is a decreasing sequence of compact sets $A_1, A_2, \dots$ in the real numbers

Since $A_k$ are bounded, we may let $a_k=\inf A_k$
Also since $A_k$ are closed, $a_k \in A_k$

Note that since $A_k$ is a decreasing sequence of sets we have $a_1\le a_2\le\dots$

Also, whenever we have $n>k$, we have $a_n \in A_n$, but $A_n \subset A_k$ and thus $a_n \in A_k$.

Let $b_1=\sup A_1$, then $a_k \in A_1$ and thus $a_k\le b_1$ for all $k$.

This tells us that the sequence $\{a_k\}$ is bounded above, and thus we may let $a=\sup a_k$.

Our goal is to show that the number $a$ appears in all $A_k$, thus showing that the entire intersection $\bigcap A_k$ contains $a$ and thus must be non-empty.

Now we split this in two cases, which asks whether a is simply made from isolated points, or if it is actually some nontrivial point obtained from the boundaries of $A_k$

\textbf{Case 1:} $a_k=a$ for some $k$
In this case we see that $a_k\le a_n\le a$ for all $n>k$ and thus $a_n=a$ in this case, therefore a is an element in $A_n$ for all $n$

In this case you can imagine that there is a possibility where a is an isolated minimum point of $A_n$ which stays there forever in the decreasing sequence of sets

\textbf{Case 2:} $a_k<a$ for all $k$; in this case we see that $a$ is the limit point of the increasing sequence $\{a_k\}$

Exercise 1: Show that $a$ is a limit point of each $A_k$.

Note that $a_n$ is in $A_k$ for each $n>k$, and since $a=\sup\{a_k\}$ where $a_k$ is increasing, we can actually show that a is a limit point of $\{a_n \mid n \le k\}$:
For every $\epsilon>0$, we pick $n_0$ such that $0 < a-a_{n_0} < \epsilon$
Pick $n\prime > \max\{k,n_0\}$, then $a_{n^\prime} \ge a_{n_0}$ and so
\[ 0<a-a_n\prime \le a_{n_0} < \epsilon \]
This shows that there exists $a_n^\prime$ in $B_0(a,\epsilon) \cap \{a_n \mid n>k\}$ for all $\epsilon$, and so $a$ is a limit point of $\{a_n \mid n>k\}$.

Now since $\{a_n|n \ge k\}$ is a subset of $A_k$ we also see that a is a limit point of $A_k$
Finally, since $A_k$ is closed, we conclude that $a$ is in $A_k$ for all $k$, and we are done

Wait hold on, I forgot about the second part

Now we consider a decreasing sequence of compact sets $A_1, A_2, \dots$ such that $\diam A_{k+1} \le c \diam A_k$ for $c<1$.

Suppose otherwise that there exists $x, y$ in $\bigcap A_k$

You can imagine that this will form a fixed distance between two points, and thus there is a constant positive lower bound for the diameters:
\[ \diam A_k \ge |x-y| > 0 \forall k \]

But this cannot be true because $\diam A_{k+1} \le c \diam A_k$ and so the diameter is controlled by a decreasing geometric sequence:
\[ \diam A_{k+1} \le c^k \diam A_1 \]

So we can simply pick a natural number $k$ such that
\[ k > \log_c \frac{|x-y|}{\diam A_1} \]

\item We consider an infinite bounded set $A$ in the real numbers. Since $A$ is bounded, we can pick a closed interval $[a_1,b_1]$ containing $A$.

We then perform a series of binary cuts: Consider the two halves of $[a_1,b_1]$. We know that at least one of these two must contain infinitely many elements in $A$, otherwise $A$ cannot be infinite. We pick this half of the interval and denote it by $[a_2,b_2]$. We continue this to pick a decreasing sequence of closed intervals $[a_n,b_n]$.

Now $\diam [a_{n+1},b_{n+1}] = \frac{1}{2} \diam [a_n,b_n]$, so by the Cantor Intersection Theorem, there exists a unique real number $c$ in the intersection $\bigcap[a_n,b_n]$.

We show that this $c$ is in fact a limit point of $A$.

For any $\epsilon>0$, we need to show that $B_0(c,\epsilon) \cap A \neq \emptyset$, i.e. we need to find an element $x \neq c$ in $A$ that is less than $\epsilon$ apart from $c$.

We then realize that we can simply exploit the decreasing sequence $[a_n,b_n]$
Since $\diam [a_n,b_n]$ is controlled by a decreasing sequence:
\[ \diam [a_{n+1},b_{n+1}] \le 1/2^n \diam [a_1,b_1] \]
We take a sufficiently large n so that $b_n-a_n<\epsilon$
Since $c$ is in $[a_n,b_n]$, for all $x$ in $[a_n,b_n]$ we have $|x-c|\le b_n-a_n<\epsilon$ and therefore $[a_n,b_n]$ is within $B(c,\epsilon)$.

Here's the funny part: $[a_n,b_n]$ contains infinitely many elements of $A$, so it must contain at least one element in A that is not $c$.

Therefore this element $x \neq c$ is in $B_0(c,\epsilon)$.

\item Now we have an infinte bounded set $A$ in $\RR^n$

The idea here is to consecutively come up with better and better sequences of points in $A$. We denote $x_i$ to be the $i$-th coordinate in $\RR^n$.

Our first wish is to pick some elements in $A$ so that they sort of converge at $x_1$.

Because such considerations of 'restricting to a single coordinate' is important here, we define the projection map to the $i$-th coordinate by
\[ f_i(x_1,\dots,x_n)=x_i \]

So, we look at $f_i(A)$ and try to apply BW for the case where $n=1$.

However, the problem is that $f_i(A)$ need not be infinite. For example, the set $\{(0,0),(0,1),(0,2),\dots\}$ projected onto the first coordinate is simply $\{0\}$.

This forces us to consider two cases

Exercise 2: Show that $f_i(A)$ is bounded
This is simple
1. $f_1(A)$ is infinite, then we can apply BW(n=1) to find a real number $c_1$ which is a limit point in $f_1(A)$

Here we can construct a sequence of points 
\[ \{x^{(1),1},x^{(1),2},...\} \]
so that their first coordinates satisfy
\[ |x^{(1),n}_1-c_1| < 1/n \]
for all natural number n
(I know this notation is cumbersome but the problem is that we need multiple sequences for this proof)

2. $f_1(A)$ is finite, then by the Pigeonhole Principle there exists a real number $c_1$ such that its preimage $f_1^{-1}(c_1)$ in $A$ is infinite

In this case we can randomly pick a sequence $\{x^{(1),1},x^{(1),2},\dots\}$ in $A$ so that their first coordinate is equal to $c_1$

I forgot to mention something that is implied, but we actually do have the need to vocabasize that the sequence $\{x^{(1),1},x^{(1),2},\dots\}$ can be chosen to contain mutually distinct entries

Now that we have a sequence that behaves nice on the first coordinate, we may then move on to the second coordinate

Let $A_1=\{x^{(1),1},x^{(1),2},\dots\}$
We again consider $f_2(A_1)$ in two cases, infinite or finite

In any case, we are able to find a subsequence $\{x^{(2),1},x^{(2),2},\dots\}$, where
$x^{(2),k}=x^{(1),n_k}$ for some strictly increasing sequence of natural numbers $n_k$

So that, for the limit point/point with infinite preimage $c_2$, this sequence satisfies
\[ |f_2(x^{(2),n})-c_2| < \frac{1}{n} \]
Note that the property we have for the second case (we in fact have $f_2(x^{(2),n})=c_2$) is just a better version of this.

Now, take note that picking this subsequence does no harm whatsoever towards the first coordinate (if anything it would turn out to be better) since
\[ |f_1(x^{(2),k})-c_1| = |f_1(x^{(1),n_k}-c_1| < \frac{1}{n_k} \le \frac{1}{k} \]
($n_1<\dots<n_k$ is a strictly increasing sequence of natural numbers so $n_k \ge k$)

This continues on until we obtain a sequence of points $\{x^{(n),1},x^{(n),2},\dots\}$ in $A$ so that
\[ |f_i(x^{(n),k}-c_i|<\frac{1}{k} \quad \forall i,k \]

As we can see, the point $c=(c_1,\dots,c_n)$ is in fact a limit point of $A$ as we can always choose a big enough $k$ so that $x^{(n),k}$ is in $B(c,\epsilon) \cap A$.

Since $\{x^{(n),k}\}$ was always chosen to be a sequence of distinct entries, there is no danger for this sequence to always be c, and so c must be a limit point of $A$.

\item We may now return to the general case of Cantor.

Suppose that there is a sequence of decreasing compact sets $A_1,A_2,\dots$ in $\RR^n$. 
Note that every point is contained in $A_1$, so boundedness will never be an issue here.

Since $A_k$ are all nonempty, we can simply pick any element $a_k$ from $A_k$.

For the uncannily specific case that there are only finitely many $\{a_k\}$ chosen, we simply note that, again by Pigeonhole Principle, one of the $a_k$ appears infinitely often; thus for each $A_n$ we simply pick $n_k>n$ so that $A_{n_k}$ contains $a_k$, then $a_k$ is in $A_{n_k}$ which is a subset of $A_n$.

Otherwise, we can then note that $\{a_k\}$ is an infinite bounded set of points, so there must exist a limit point a of $\{a_k\}$.

We can now see that $a$ is always an element of $A_k$:
Using the same technique as Exercise 1, we see that a is a limit point of $\{a_n \mid n>k\}$ and so is a limit point of $A_k$, therefore a is in $A_k$ as $A_k$ is closed.

This proves the first part of the statement
The second part is completely identical to the second part of the $n=1$ case so we don't need to waste our time there either

\item We now consider a compact set A with some open covering $\mathcal{U}$.

This theorem is proved by contradiction: 
Suppose otherwise that set $A$ cannot be covered by any finite collection of open sets in $\mathcal{U}$

Since $A$ is compact, we may enclose it in a closed cube $Q_1$ (whose edges are parallel to the axes)

Now, for each step, we partition $Q$ into $2^n$ cubes by cutting it in half from each direction.

Then, starting from $Q_1$, there must exist one of these smaller cubes, denoted by $Q_2$, such that $A \cap Q_2$ cannot be covered by a finite collection of open sets in $\mathcal{U}$. 
Otherwise, if each $A \cap Q$ has a finite cover, then we simply collect all of these open sets together to form a finite cover of $A$, which violates our assumption.

We continue on to partition $Q_n$ and pick $Q_{n+1}$ so that $A_{n+1}$ has no finite cover (denote $A_n = A \cap Q_n$).

Note that $A$ and $Q_n$ are both compact, so $A_n$ is compact
Also we see that there is a decreasing sequence $A_1,A_2,\dots$
(we can't exactly obtain a relation between $\diam A_n$ and $\diam A_{n+1}$ here)

By Cantor Intersection Theorem we can always find a point $x$ in $A$ located in the intersection $\bigcap A_k$.

Now, since $\mathcal{U}$ is an open covering of $A$, there exists an open set $U$ in $\mathcal{U}$ such that $x\in U$.

The final key step is to exploit the sequence of decreasing cubes $Q_n$. So even though there isn't a clear cut way to control the sizes of $\diam A_n$, we do in fact have the property that $\diam Q_{n+1} = \frac{1}{2^n} \diam Q_1$.

Therefore, by picking a sufficiently large $n$, we can obtain $Q_n$ that is contained in $U$.

But this is a contradiction. 
This is because we've specifically chosen the sequence $A_n$ to be sets that do not possess any finite cover $\{U_1,...,U_n\}$ in $\mathcal{U}$. But here $A_n$ simply would have a one-element cover $\{U\}$.

This completes our proof.
\end{enumerate}
\end{proof}
%https://www.maths.usyd.edu.au/u/bobh/UoS/MATH3901/00met21.pdf
\end{comment}

\section{Compactness}
The following is a useful analogy to visualise the concept of compactness:
\begin{mdframed}
Compactness is like a well-contained space where nothing ``escapes'' or goes off to infinity.

An open cover is a collection of open sets that completely cover the compact set (think of a bunch of overlapping circles covering a shape).

The key feature of compact sets is that from any open cover, you can always select a finite number of sets from the cover that still manage to cover the entire space.
\end{mdframed}

\begin{definition}
Let $\mathcal{U}=\{U_i\mid i\in I\}$ be a collection of open subsets of $X$. We say that $\mathcal{U}$ is an \vocab{open cover}\index{open cover} of a set $K$ if
\[K\subset\bigcup_{i\in I}U_i.\]
If $I^\prime\subset I$ and $K\subset\bigcup_{i\in I^\prime}U_i$, we say that $\{U_i\mid i\in I^\prime\}$ is a \textbf{subcover} of $\mathcal{U}$. If moreover, $I^\prime$ is finite, then it is called a \textbf{finite subcover}.
\end{definition}

\begin{definition}[Compactness]
$K\subset X$ is said to be \vocab{compact}\index{compactness} if every open cover of $K$ contains a finite subcover.
\end{definition}

That is, if $\mathcal{U}=\{U_i\mid i\in I\}$ is an open cover of $K$, then there are finitely many indices $i_1,\dots,i_n$ such that
\[K\subset \bigcup_{k=1}^{n}U_{i_k}.\]

\begin{exercise}
Every finite set is compact.
\end{exercise}

\begin{solution}
Let $E$ be finite. Let $\mathcal{U}=\{U_i\mid i\in I\}$ be an open cover of $E$, then we have that $E\subset\bigcup_{i\in I}$.

For each point $x\in E$, take $i_x$ such that $x\in U_{i_x}$. Let $\mathcal{V}=\{U_{i_x}\mid x\in E\}$. By construction, since $x\in\mathcal{V}$ for all $x\in E$, $E\subset\mathcal{V}$ so $\mathcal{V}$ is an open cover of $E$. Since there are finitely many $x$, $\mathcal{V}$ is thus a finite subcover of $E$, and hence $E$ is compact.
\end{solution}

\begin{proposition}
Suppose $Y$ is a subspace of $X$, and $K\subset Y$. Then $K$ is compact relative to $X$ if and only if $K$ is compact relative to $Y$.
\end{proposition}

\begin{proof} \

\fbox{$\implies$} Suppose $K$ is compact relative to $X$. Let $\mathcal{U}$ be an open cover of $K$; that is, $\mathcal{U}=\{U_i\mid i\in I\}$ is a collection of sets open relative to $Y$, such that $K\subset\bigcup_{i\in I}U_i$. 
By \cref{prop:open-subspace-cap}, for all $i\in I$, there exist $V_i$ open relative to $X$ such that $U_i=Y\cap V_i$. Since $K$ is compact relative to $X$, we have
\begin{equation*}\tag{1}
K\subset\bigcup_{k=1}^{n}V_{i_k}
\end{equation*}
for some choice of finitely many indices $i_1,\dots,i_n$. Since $K\subset Y$, (1) implies that
\begin{equation*}\tag{2}
K\subset\bigcup_{k=1}^{n}U_{i_k}.
\end{equation*}
This proves that $K$ is compact relative to $Y$.

\fbox{$\impliedby$} Suppose $K$ is compact relative to $Y$, let $\mathcal{V}=\{V_i\mid i\in I\}$ be a collection of open subsets of $X$ which covers $K$, and put $U_i=Y\cap V_i$. Then (2) will hold for some choice of $i_1,\dots,i_n$; and since $U_i\subset V_i$, (2) implies (1).
\end{proof}

\begin{proposition}
Compact subsets of metric spaces are closed.
\end{proposition}

\begin{proof}
Let $K\subset X$ be compact. To prove that $K$ is closed, we need to show that $K^c$ is open.

Suppose $p\in X$, $p\neq K$. If $q\in K$, let $V_q$ and $W_q$ be neighbourhoods of $p$ and $q$ respectively, of radius less than $\frac{1}{2}d(p,q)$. Since $K$ is compact, there exists finite many points $q_1,\dots,q_n\in K$ such that
\[K\subset\bigcup_{k=1}^{n}W_{q_k}=W.\]
If $V=\bigcap_{k=1}^{n}V_{q_k}$, then $V$ is a neighbourhood of $p$ which does not intersect $W$. Hence $V\subset K^c$, so $p$ is an interior point of $K^c$. The theorem follows.
\end{proof}

\begin{proposition}
Closed subsets of compact sets are compact.
\end{proposition}

\begin{proof}
Suppose $F\subset K\subset X$, $F$ is closed (relative to $X$), and $K$ is compact.

Let $\mathcal{V}=\{V_i\mid i\in I\}$ be an open cover of $F$. If $F^c$ is adjoined to $\mathcal{V}$, we obtain an open cover $\Omega$ of $K$. Since $K$ is compact, there is a finite subcollection $\Phi$ of $\Omega$ which covers $K$, and hence $F$. If $F^c$ is a member of $\Phi$, we may remove it from $\Phi$ and still retain an open cover of $F$. We have thus shown that a finite subcollection of $\mathcal{V}$ covers $F$.
\end{proof}

\begin{corollary}
If $F$ is closed and $K$ is compact, then $F\cap K$ is compact.
\end{corollary}

\begin{proposition}
If $E$ is an infinite subset of a compact set $K$, then $E$ has a limit point in $K$.
\end{proposition}

\begin{proposition}
If $(I_n)$ is a sequence of intervals in $\RR$ such that $I_i\supset I_{i+1}$, then $\bigcap_{i=1}^{\infty}I_n\neq\emptyset$.
\end{proposition}

\begin{proposition}
If $(I_n)$ is a sequence of $k$-cells such that $I_n\supset I_{n+1}$, then $\bigcap_{n=1}^{\infty}\neq\emptyset$.
\end{proposition}

\begin{proof}
Let $I_n$ consist of all points $\vb{x}=(x_1,\dots,x_k)$ such that
\end{proof}

\begin{proposition}
Every $k$-cell is compact.
\end{proposition}

\begin{theorem}[Cantor's Intersection Theorem]\label{thrm:cantor-intersection}
Given a decreasing sequence of compact sets $A_1\supset A_2 \supset \cdots$, there exists a point $x\in\RR^n$ such that $x$ belongs to all $A_i$. In other words, $\bigcap_{i=1}^\infty A_i\neq\emptyset$. Moreover, if for all $i\in\NN$ we have $\diam A_{i+1}\le c\cdot\diam A_k$ for some constant $c<1$, then such a point must be unique, i.e. $\bigcap_{i=1}^\infty A_k=\{x\}$ for some $x\in\RR^n$.
\end{theorem}

\begin{proposition}
If $E\subset\RR^n$ has one of the following three properties, then it has the other two:
\begin{enumerate}[label=(\roman*)]
\item $E$ is closed and bounded.
\item $E$ is compact.
\item Every infinite subset of $E$ has a limit point in $E$.
\end{enumerate}
\end{proposition}

\begin{theorem}[Heine--Borel Theorem]\label{thrm:heine-borel}
$E\subset\RR^n$ is compact if and only if $E$ is closed and bounded.
\end{theorem}

\begin{proof}

\end{proof}

\begin{theorem}[Bolzano--Weierstrass Theorem]\label{thrm:bolzano-weierstrass}
Every bounded infinite subset of $\RR^n$ has a limit point in $\RR^n$.
\end{theorem}

\begin{proof}

\end{proof}

\begin{comment}
sequential compactness
A set $K$ is compact if and only if every sequence of points in $K$ has a subsequence that converges to a point in $K$.

Any continuous function defined on a compact set is bounded.

extreme value theorem
\end{comment}
\pagebreak

\section{Perfect Sets}
\begin{definition}[Perfect set]
$E$ is \vocab{perfect}\index{perfect set} if $E$ is closed and if every point of $E$ is a limit point of $E$.
\end{definition}

\begin{proposition}
Let $P$ be a non-empty perfect set in $\RR^n$. Then $P$ is uncountable.
\end{proposition}

\begin{corollary}
Every interval $[a,b]$ is uncountable. In particular, $\RR$ is uncountable.
\end{corollary}

The set which we are now going to construct shows that there exist perfect sets in $\RR$ which contain no segment.

Let
\[E_0=[0,1].\]
Remove the segment $\brac{\frac{1}{3},\frac{2}{3}}$ to give
\[E_1=\sqbrac{0,\frac{1}{3}}\cup\sqbrac{\frac{2}{3},1}.\]
Remove the middle thirds of these intervals to give
\[E_2=\sqbrac{0,\frac{1}{9}}\cup\sqbrac{\frac{2}{9},\frac{3}{9}}\cup\sqbrac{\frac{6}{9},\frac{7}{9}}\cup\sqbrac{\frac{8}{9},1}.\]
Repeating this process, we obtain a monotonically decreasing sequence of compact sets $(E_n)$, where $E_1\supset E_2\supset\cdots$ and $E_n$ is the union of $2^n$ intervals, each of length $3^{-n}$.

The \vocab{Cantor set} is defined as
\[P\coloneqq\bigcap_{n=1}^{\infty}E_n.\]

\begin{proposition}
$P$ is compact.
\end{proposition}

\begin{proposition}
$P$ is not empty.
\end{proposition}

\begin{proof}
This follows from Theorem 2.36.
\end{proof}

\begin{proposition}
$P$ contains no segment.
\end{proposition}

\begin{proof}
No segment of the form
\[\brac{\frac{3k+1}{3^m},\frac{3k+2}{3^m}},\]
where $k,m\in\ZZ^+$, has a point in common with $P$. Since every segment $(\alpha,\beta)$ contains a segment of the above form, if
\[3^{-m}<\frac{\beta-\alpha}{6},\]
$P$ contains no segment.
\end{proof}

\begin{proposition}
$P$ is perfect.
\end{proposition}

\begin{proof}
To show that $P$ is perfect, it is enough to show that $P$ contains no isolated point. Let $x\in P$, and let $S$ be any segment containing $x$. Let $I_n$ be that interval of $E_n$ which contains $x$. Choose $n$ large enough, so that $I_n\subset S$. Let $x_n$ be an endpoint of $I_n$, such that $x_n\neq x$.

It follows from the construction of $P$ that $x_n\in P$. Hence $x$ is a limit point of $P$, and $P$ is perfect.
\end{proof}
\pagebreak

\section{Connectedness}
\begin{definition}[Connectedness]
$A$ and $B$ are said to be \vocab{separated} if $A\cap\overline{B}=\emptyset$ and $\bar{A}\cap B=\emptyset$; that is, no point of $A$ lies in the closure of $B$ and no point of $B$ lies in the closure of $A$.

$E\subset X$ is said to be \vocab{connected}\index{connectedness} if $E$ is not a union of two non-empty separated sets. 
\end{definition}

\begin{remark}
Separated sets are of course disjoint, but disjoint sets need not be separated. For example, the interval $[0,1]$ and the segment $(1,2)$ are not separated, since $1$ is a limit point of $(1,2)$. However, the segments $(0,1)$ and $(1,2)$ are separated.
\end{remark}

The connected subsets of the line have a particularly simple structure: 

\begin{proposition}
$E\subset\RR^1$ is connected if and only if it has the following property: if $x,y\in E$ and $x<z<y$, then $z\in E$.
\end{proposition}

\begin{proof} \

\fbox{$\impliedby$} If there exists $x,y\in E$ and some $z\in(x,y)$ such that $z\notin E$, then $E=A_z\cup B_z$ where
\[ A_z=E\cap(-\infty,z), \quad B_z=E\cap(z,\infty). \]
Since $x\in A_z$ and $y\in B_z$, $A$ and $B$ are non-empty. Since $A_z\subset(-\infty,z)$ and $B_z\subset(z,\infty)$, they are separated. Hence $E$ is not connected.

\fbox{$\implies$} Suppose $E$ is not connecetd. Then there are non-empty separated sets $A$ and $B$ such that $A\cup B=E$. Pick $x\in A$, $y\in B$, and WLOG assume that $x<y$. Define
\[z\coloneqq\sup(A\cap[x,y].)\]
By 
\end{proof}

\begin{definition}
We say that a metric space is \vocab{disconnected} if we can write it as the disjoint union of two nonempty open sets. We say that a space is \vocab{connected} if it is not disconnected.
\end{definition}

If $X$ is written as a disjoint union of two nonempty open sets $U$ and $V$ then we say that these sets \vocab{disconnect} $X$.

\begin{example}
If $X=[0,1]\cup[2,3]\subset\RR$ then we have seen that both $[0,1]$ and $[2,3]$ are open in $X$. Since $X$ is their disjoint union, $X$ is disconnected.
\end{example}

The following lemma gives some equivalent ways to formulate the concept of connected space.

\begin{lemma}
The following are equivalent:
\begin{enumerate}[label=(\roman*)]
\item $X$ is connected.
\item If $f:X\to\{0,1\}$ is a continuous function then $f$ is constant.
\item The only subsets of $X$ which are both open and closed are $X$ and $\emptyset$.
\end{enumerate}
(Here the set $\{0,1\}$ is viewed as a metric space via its embedding in $\RR$, or equivalently with the discrete metric.)
\end{lemma}

\begin{proof}

\end{proof}

Frequently one has a metric space $X$ and a subset $E$ of it whose connectedness or otherwise one wishes to ascertain. To this end, it is useful to record the following lemma.

\begin{lemma}
Let $E\subset X$, considered as a metric space with the metric induced from $X$. Then $E$ is connected if and only if the following is true: if $U,V$ are open subsets of $X$, and $U\cap V\cap E=\emptyset$, then $E\subset U\cup V$ implies either $E\subset U$ or $E\subset V$.
\end{lemma}

\begin{proof}

\end{proof}

We now turn to some basic properties of the notion of connectedness. These broadly conform with one's intuition about how connected sets should behave.

\begin{lemma}[Sunflower lemma]
Let $\{E_i\mid i\in I\}$ be a collection of connected subsets of $X$ such that $\bigcap_{i\in I}E_i\neq\emptyset$. Then $\bigcup_{i\in I}E_i$ is connected.
\end{lemma}

\begin{proof}

\end{proof}


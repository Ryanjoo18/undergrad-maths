\chapter{Riemann--Stieltjes Integral}\label{chap:rs-integration}
\section{Definition of Riemann--Stieltjes Integral}
A \vocab{partition} $P$ of a closed interval $[a,b]\subset\RR$ is a finite set of points $x_0,x_1,\dots,x_n$ where
\[a=x_0\le x_1\le\cdots\le x_{n-1}\le x_n=b.\]
Let $f:[a,b]\to\RR$ be bounded, and $\alpha$ be an increasing function over $[a,b]$. Denote by
\begin{align*}
M_i&=\sup_{[x_{i-1},x_i]}f(x),\\
m_i&=\inf_{[x_{i-1},x_i]}f(x),
\end{align*}
and by
\[\Delta\alpha_i=\alpha(x_i)-\alpha(x_{i-1}).\]
The \vocab{upper sum} of $f$ with respect to the partition $P$ and $\alpha$ is
\[U(f,\alpha;P)=\sum_{i=1}^n M_i \Delta \alpha_i\]
and the \vocab{lower sum} of $f$ with respect to the partition $P$ and $\alpha$ is
\[ L(f,\alpha;P)=\sum_{i=1}^n m_i \Delta \alpha_i. \]
Define the upper Riemann--Stieltjes integral as
\[\upperint_a^bf(x)\dd{\alpha(x)}\coloneqq\inf_P U(f,\alpha;P)\]
and the lower Riemann--Stieltjes integral as
\[\lowerint_a^bf(x)\dd{\alpha(x)}\coloneqq\sup_P L(f,\alpha;P).\]
It is easy to see from definition that
\[ \lowerint_a^bf(x)\dd{\alpha(x)}\le\upperint_a^bf(x)\dd{\alpha(x)}. \]

\begin{definition}[Riemann--Stieltjes integrability]
A function $f$ is \vocab{Riemann--Stieltjes integrable}\index{Riemann--Stieltjes integrability} with respect to $\alpha$ over $[a,b]$, if
\[\lowerint_a^bf(x)\dd{\alpha(x)}=\upperint_a^bf(x)\dd{\alpha(x)}.\]
\end{definition}

\begin{notation}
$\displaystyle\int_a^bf(x)\dd{\alpha(x)}$ denotes the common value, which is called the Riemann--Stieltjes of $f$ with respect to $\alpha$ over $[a,b]$.
\end{notation}

\begin{notation}
$\mathcal{R}_\alpha[a,b]$ denotes the set of Riemann--Stieljes integrable functions with respect to $\alpha$ over $[a,b]$.
\end{notation}

In particular, when $\alpha(x)=x$, we call the corresponding Riemann--Stieljes integration the \emph{Riemann integration}, and use $\mathcal{R}[a,b]$ to denote the set of Riemann integrable functions.

\begin{definition}[Refinement]
The partition $P^\prime$ is a \vocab{refinement} of $P$ if $P^\prime\supset P$. Given two partitions $P_1$ and $P_2$, we say that $P^\prime$ is their \vocab{common refinement} if $P^\prime=P_1\cup P_2$.
\end{definition}

Intuitively, a refinement will give a better estimation than the original partition, so the upper and lower sums of a refinement should be more restrictive. We will now show this.

\begin{lemma}\label{lemma:int-refinement}
If $P^\prime$ is a refinement of $P$, then
\begin{enumerate}[label=(\roman*)]
\item $L(f,\alpha;P)\le L(f,\alpha;P^\prime)$
\item $U(f,\alpha;P^\prime)\le U(f,\alpha;P)$
\end{enumerate}
\end{lemma}

\begin{proof} \
\begin{enumerate}[label=(\roman*)]
\item Suppose first that $P^\prime$ contains just one point more than $P$ Let this extra point be $x^\prime$, and suppose $x_{i-1}<x^\prime<x_i$ for some $i$, where $x_{i-1},x_i\in P$. Put
\[w_1=\inf_{x\in[x_{i-1},x^\prime]}f(x)\]
and
\[w_2=\inf_{x\in[x^\prime,x_i]}f(x).\]
Let, as before,
\[m_i=\inf_{x\in[x_{i-1},x_i]}f(x).\]
Clearly $w_1\ge m_i$ and $w_2\ge m_i$. Hence
\begin{align*}
&L(f,\alpha;P^\prime)-L(f,\alpha;P)\\
&=w_1[\alpha(x^\prime)-\alpha(x_{i-1})]+w_2[\alpha(x_i)-\alpha(x^\prime)]-m_i[\alpha(x_i)-\alpha(x_{i-1})]\\
&=(w_1-m_i)[\alpha(x^\prime)-\alpha(x_{i-1})]+(w_2-m_i)[\alpha(x_i)-\alpha(x^\prime)]\ge0.
\end{align*}
If $P^\prime$ contains $k$ more points than $P$, we repeat this reasoning $k$ times.

\item Analogous to the proof of (i).
\end{enumerate}
\end{proof}

One would expect the lower RS integral to be less than or equal to the upper RS integral. We now show this.

\begin{lemma}\label{lemma:int-upper-lower}
\[\lowerint_a^bf\dd{\alpha}\le\upperint_a^bf\dd{\alpha}.\]
\end{lemma}

\begin{proof}
Let $P^\prime$ be the common refinement of partitions $P_1$ and $P_2$. By \cref{lemma:int-refinement},
\[L(f,\alpha;P_1)\le L(f,\alpha;P^\prime)\le U(f,\alpha;P^\prime)\le U(f,\alpha;P_2)\]
and so
\[L(f,\alpha;P_1)\le U(f,\alpha;P_2).\]
Fix $P_2$ and take sup over all $P_1$ gives
\[\lowerint f\dd{\alpha}\le U(f,\alpha;P_2).\]
Then take inf over all $P_2$, which gives the desired result.
\end{proof}

Now we discuss integrability conditions for $f$.

\begin{theorem}
$f\in \mathcal{R}_\alpha[a,b]$ if and only if
\[\forall\epsilon>0,\quad\exists P,\quad U(f,\alpha;P)-L(f,\alpha;P)<\epsilon.\]
\end{theorem}

\begin{proof} \

\fbox{$\implies$} Suppose $f\in \mathcal{R}_\alpha[a,b]$. Let $\epsilon>0$ be given. Then there exists partitions $P_1$ and $P_2$ such that


\fbox{$\impliedby$} For every $P$, from \cref{lemma:int-upper-lower} we have
\[L(f,\alpha;P)\le\lowerint f\dd{\alpha}\le\upperint f\dd{\alpha}\le U(f,\alpha;P).\]

\end{proof}

\begin{example}[Dirichlet function]
The Dirichlet function is defined over $\RR$ by
\[f(x)=\begin{cases}
1&x\in\QQ \\
0&x\in\RR\setminus\QQ
\end{cases}\]
We try to calculate the two on the interval $[0,1]$.

The Dirichlet function is pathological because for each subinterval $[x_{i-1},x_i]$, the supremum is always $1$ and the infimum is always $0$.

So no matter what partition we use, $U(f,P)$ is always $1$ whereas $L(f,P)$ is always $0$. This means that $U(f)=1$ and $L(f)=0$, so there are two different values for ``the integral of $f$''.

This is like the case where we try to find the limit of the Dirichlet function where $x$ is approaching any given real number $r$, there exists two sequences approaching $r$ whose image approaches two different values.
\end{example}

\begin{example}[Heaviside step function]
The Heaviside step function $H$ is a real-valued function defined by
\[H(x)=\begin{cases}
0&x\le0\\
1&x>0
\end{cases}\]

\begin{proposition*}
$f$ bounded on $[a,b]$, $f$ continuous at $s\in(a,b)$. Let $\alpha(x)=H(x-s)$, then
\[\int_a^b f\dd{\alpha}=f(s).\]
\end{proposition*}

\begin{proposition*}
Suppose $c_n\ge0$ for $n=1,2,\dots$, $\sum c_n$ converges, $(s_n)$ is a sequence of distinct points in $(a,b)$, and
\[\alpha(x)=\sum_{n=1}^{\infty}c_n I(x-s_n).\]
Let $f$ be continuous on $[a,b]$. Then
\[\int_a^b f\dd{\alpha}=\sum_{n=1}^{\infty}c_n f(s_n).\]
\end{proposition*}
\end{example}

\begin{proposition} \
\begin{enumerate}[label=(\roman*)]
\item For all $\epsilon>0$, if there exists $P$ such that $U(f,\alpha;P)-L(f,\alpha;P)<\epsilon$, then $U(f,\alpha;P)-L(f,\alpha;P^\prime)<\epsilon$ where $P^\prime$ is a refinement of $P$.
\item 
\item 
\end{enumerate}
\end{proposition}

\begin{proof}

\end{proof}

\begin{lemma}[Continuity implies integrability]
If $f$ is continuous on $[a,b]$, then $f\in \mathcal{R}_\alpha[a,b]$.
\end{lemma}

\begin{proof}
Let $\epsilon>0$ be given. Choose $\eta>0$ such that
\[[\alpha(b)-\alpha(a)]\eta<\epsilon.\]
Since $f$ is uniformly continuous on $[a,b]$ (Theorem 4.19), there exists $\delta>0$ such that
\[\absolute{f(x)-f(t)}<\eta\]
if $x\in[a,b]$, $t\in[a,b]$, $|x-t|<\delta$.

If $P$ is any partition of $[a,b]$ such that $\Delta x_i<\delta$ for all $i$, then (16) implies that
\[M_i-m_i\le\eta\]
and therefore
\begin{align*}
U(f,\alpha;P)-L(f,\alpha;P)
&=\sum_{i-1}^{n}(M_i-m_i)\Delta\alpha_i\\
&\le\eta\sum_{i=1}^{n}\Delta\alpha_i\\
&=\eta[\alpha(b)-\alpha(a)]\\
&<\epsilon.
\end{align*}
By Theorem 6.6, $f\in\mathcal{R}_\alpha[a,b]$.
\end{proof}

\begin{proposition}
If $f$ is monotonic on $[a,b]$, and if $\alpha$ is continuous on $[a,b]$, then $f\in\mathcal{R}_\alpha[a,b]$.
\end{proposition}

\begin{proposition}
Suppose $f$ is bounded on $[a,b]$, $f$ has only finitely many points of discontinuity on $[a,b]$, and $\alpha$ is continuous at every point at which $f$ is discontinuous. Then $f\in \mathcal{R}_\alpha[a,b]$.
\end{proposition}

\begin{proposition}
$f\in \mathcal{R}_\alpha[a,b]$, $m\le f\le M$, and $\phi$ is uniformly continuous on $[m,M]$. Then
\[\phi\circ f\in \mathcal{R}_\alpha[a,b].\]
\end{proposition}

\begin{proof}
Choose $\epsilon>0$. Since $\phi$ is uniformly continuous on $[m,M]$, there exists $\delta>0$ such that $\delta<\epsilon$ and $|\phi(s)-\phi(t)|<\epsilon$ if $|s-t|<\le\delta$ and $s,t\in[m,M]$.

Since $f\in \mathcal{R}_\alpha[a,b]$, there exists a partition $P=\{x_0,x_1,\dots,x_n\}$ of $[a,b]$ such that
\[U(f,\alpha;P)-L(f,\alpha;P)<\delta^2.\]

\end{proof}

\section{Properties of the Integral}
\begin{theorem} \
\begin{enumerate}[label=(\roman*)]
\item If $f_1,f_2\in \mathcal{R}_\alpha[a,b]$, then 
\[ f_1+f_2\in \mathcal{R}_\alpha[a,b]; \]
$cf\in \mathcal{R}_\alpha[a,b]$ for every $c\in\RR$, and
\[ \int_a^b(f_1+f_2)\dd{\alpha}=\int_a^bf_1\dd{\alpha}+\int_a^bf_2\dd{\alpha}, \]
\[ \int_a^b(cf)\dd{\alpha}=c\int_a^bf\dd{\alpha}. \]

\item If $f_1,f_2\in \mathcal{R}_\alpha[a,b]$ and $f_1\le f_2$, then
\[ \int_a^bf_1\dd{\alpha}\le\int_a^bf_2\dd{\alpha}. \]

\item If $f\in \mathcal{R}_\alpha[a,b]$ and $c\in[a,b]$, then $f\in \mathcal{R}_\alpha[a,c]$ and $f\in \mathcal{R}_\alpha[c,b]$, and
\[ \int_a^bf\dd{\alpha}=\int_a^c\dd{\alpha}+\int_c^b\dd{\alpha}. \]

\item If $f\in \mathcal{R}_\alpha[a,b]$ and $|f|\le M$, then
\[ \absolute{\int_a^bf\dd{\alpha}}\le M\sqbrac{\alpha(b)-\alpha(a)}. \]

\item If $f\in R_{\alpha_1}[a,b]$ and $f\in R_{\alpha_2}[a,b]$, then $f\in R_{\alpha_1+\alpha_2}[a,b]$ and
\[ \int_a^bf\dd{(\alpha_1+\alpha_2)}=\int_a^bf\dd{\alpha_1}+\int_a^bf\dd{\alpha_2}; \]
if $f\in \mathcal{R}_\alpha[a,b]$ and $c$ is a positive constant, then $f\in R_{c\alpha}[a,b]$ and
\[ \int_a^bf\dd{(c\alpha)}=c\int_a^bf\dd{\alpha}. \]

\item If $f\in \mathcal{R}_\alpha[a,b]$ and $g\in \mathcal{R}_\alpha[a,b]$, then $fg\in \mathcal{R}_\alpha[a,b]$.
\end{enumerate}
\end{theorem}

\begin{proof} \
\begin{enumerate}[label=(\roman*)]
\item If $f=f_1+f_2$ and $P$ is any partition of $[a,b]$, we have
\begin{align*}
L(f_1,\alpha;P)+L(f_2,\alpha;P)&\le L(f,\alpha;P)\\
&\le U(f,\alpha;P)\\
&\le U(f_1,\alpha;P)+U(f_2,\alpha;P).
\end{align*}

If $f_1\in \mathcal{R}_\alpha[a,b]$ and $f_2\in \mathcal{R}_\alpha[a,b]$, let $\epsilon>0$ be given. There are partitions $P_1$ and $P_2$ such that


\item 
\item 
\item 
\item 
\item 
\end{enumerate}
\end{proof}

\begin{theorem}[Triangle inequality]
$f\in \mathcal{R}_\alpha[a,b]$, then $|f|\in \mathcal{R}_\alpha[a,b]$,
\[ \absolute{\int_a^bf\dd{\alpha}}\le\int_a^b|f|\dd{\alpha}. \]
\end{theorem}

\begin{proof}

\end{proof}

6.14 6.15
Heaviside step function

6.16 corollary
for intinite sum, need $\sum c_n$ to converge
(23) comparison test

\begin{proposition}[Integration by substitution]
Assume $\alpha$ increases monotonically, $\alpha^\prime\in R[a,b]$. Let $f$ be a bounded real function on $[a,b]$, then
\[f\in \mathcal{R}_\alpha[a,b]\iff f\alpha^\prime\in R[a,b].\]
\end{proposition}

\begin{proposition}[Change of variables]
Suppose $\phi:[A,B]\to[a,b]$ is a strictly increasing continuous function. Suppose $\alpha$ is monotonically increasing on $[a,b]$, $f\in \mathcal{R}_\alpha[a,b]$. Define $\beta$ and $g$ on $[A,B]$ by
\[\beta(y)=\alpha(\phi(y)),\quad g(y)=f(\phi(y)).\]
Then $g\in R(\beta)$ and
\[\int_A^B g\dd{\beta}=\int_a^b f\dd{\alpha}.\]
\end{proposition}

\section{Integration and Differentiation}
We shall show that integration and differentiation are, in a certain sense, inverse operations.

\begin{lemma}
$f\in \mathcal{R}_\alpha[a,b]$. For $x\in [a,b]$, put
\[F(x)=\int_a^x f(t)\dd{t}.\]
Then $F$ is continuous on $[a,b]$; furthermore, if $f$ is continuous at $x_0\in[a,b]$, then $F$ is differentiable at $x_0$, and
\[F^\prime(x_0)=f(x_0).\]
\end{lemma}

\begin{theorem}[Fundamental Theorem of Calculus]
$f\in \mathcal{R}_\alpha[a,b]$, there is a differentiable function $F$ on $[a,b]$ such that $F^\prime=f$, then
\begin{equation}
\int_a^b f(x)\dd{x}=F(b)-F(a).
\end{equation}
\end{theorem}

\begin{theorem}[Integration by parts]
Suppose $F$ and $G$ are differentiable functions on $[a,b]$, $F^\prime=f\in R$, $G^\prime=g\in R$. Then
\begin{equation}
\int_a^b F(x)g(x)\dd{x}=F(b)G(b)-F(a)G(a)-\int_a^b f(x)G(x)\dd{x}.
\end{equation}
\end{theorem}
\chapter{Sequence and Series of Functions}\label{chap:func-seq-series}
\section{Uniform Convergence}
\begin{definition}[Pointwise convergence]
Suppose $(f_n)$ is a sequence of functions defined on a set $E$, and suppose that $\brac{f_n(x)}$ converges for every $x\in E$. We can then define a function $f$ by
\[f(x)=\lim_{n\to\infty}f_n(x)\quad(\forall x\in E)\]
We say that $(f_n)$ \vocab{converges pointwise}\index{pointwise convergence} to $f$ on $E$, denoted by $f_n\to f$, if
\[\forall\epsilon>0,\quad\forall x\in E,\quad\exists N\in\NN,\quad\forall n>N,\quad\absolute{f_n(x)-f(x)}<\epsilon.\]
$f$ is called the \textbf{limit} or limit function of $(f_n)$.

Similarly, if $\sum f_n(x)$ converges for every $x\in E$, and if we define
\[f(x)=\sum_{n=1}^\infty f_n(x)\quad(\forall x\in E)\]
the function $f$ is called the \textbf{sum of the series} $\sum f_n$.
\end{definition}

Most properties are not preserved by pointwise continuity; that is, $f$ does not inherit most properties of $f_n$.

\begin{example}[$f_n$ continuous, $f$ discontinuous]
Let $f_n(x)=x^n$ for $x\in[0,1]$. Then
\[f(x)=\lim_{n\to\infty}f_n(x)=\begin{cases}
0&\text{if }x\in(0,1]\\
1&\text{if }x=1
\end{cases}\]
and so the limit function $f(x)$ is discontinuous.
\end{example}

\begin{example}[$f_n$ integrable, $f$ not integrable]
Recall that the Dirichlet function
\[D(x)=\begin{cases}
1&\text{if }x\in\QQ\\
0&\text{if }x\in\RR\setminus\QQ
\end{cases}\]
is not integrable.

\begin{proof}
Consider the interval $[0,1]$. We partition $P:0=x_0<x_1<\cdots<x_n=1$. The sum is given by $\sum_{i=1}^n D(t_i)\Delta x_i$. Then
\[M_i=\max_{t\in[x_{i-1},x_i]}D(t)=1\implies U(D;P)=1\quad\forall P\]
and
\[m_i=\min_{t\in[x_{i-1},x_i]}D(t)=0\implies L(D;P)=0\quad\forall P.\]
Hence 
\[\upperint_0^1 D(x)\dd{x}=1,\quad\lowerint_0^1 D(x)\dd{x}=0\]
so $\upperint_0^1 D(x)\dd{x}\neq\lowerint_0^1 D(x)\dd{x}$, and thus $D(x)$ is not integrable.
\end{proof}

We define a sequence of functions as follows:
\[D_n(x)=\begin{cases}
1&\text{if }x=\frac{p}{q},p\in\ZZ,q\in\ZZ\setminus\{0\},|q|\le n\\
0&\text{if otherwise}
\end{cases}\]

\end{example}

\begin{definition}[Uniform convergence]
We say that $(f_n)$ \vocab{uniformly converges}\index{uniform convergence} to $f$ over $E$, denoted by $f_n\rightrightarrows f$, if 
\[\forall\epsilon>0,\quad\exists N\in\NN,\quad\forall x\in E,\quad\forall n>N,\quad\absolute{f_n(x)-f(x)}<\epsilon.\]
For series, we say that the series $\sum f_n(x)$ converges uniformly on $E$ if the sequence of partial sums $(S_n)$ defined by
\[S_n(x)=\sum_{i=1}^{n}f_i(x)\]
converges uniformly on $E$.
\end{definition}

Uniform convergence is stronger than pointwise convergence, since $N$ is uniform (or ``fixed'') for all $x\in E$; for pointwise convergence, the choice of $N$ is determined by $x$.

The Cauchy criterion for uniform convergence is as follows.

\begin{lemma}[Cauchy criterion]
$(f_n)\rightrightarrows f$ if and only if
\[\forall\epsilon>0,\quad\exists N\in\NN,\quad\forall x\in E,\quad\forall n,m\ge N,\quad\absolute{f_n(x)-f_m(x)}\le\epsilon.\]
\end{lemma}

\begin{proof} \

\fbox{$\implies$} Suppose $f_n\rightrightarrows f$ on $E$. Let $\epsilon>0$ be given. Then there exists $N\in\NN$ such that for all $x\in E$, for all $n>N$,
\[\absolute{f_n(x)-f(x)}<\frac{\epsilon}{2}.\]
Then for all $n,m>N$,
\begin{align*}
|f_n(x)-f_m(x)|
&=\absolute{\brac{f_n(x)-f(x)}-\brac{f_m(x)-f(x)}}\\
&\le|f_n(x)-f(x)|+|f_m(x)-f(x)|\\
&<\frac{\epsilon}{2}+\frac{\epsilon}{2}=\epsilon
\end{align*}
by triangle inequality.

\fbox{$\impliedby$} Conversely, suppose the Cauchy condition holds. By Theorem 3.11, the sequence $\brac{f_n(x)}$ converges, for every $x$, to a limit which we may call $f(x)$. Thus $(f_n)\to f$ on $E$. We have to prove that the convergence is uniform.

Let $\epsilon>0$ be given. Choose $N\in\NN$ such that (13) holds. Fix $n$, and let $m\to\infty$ in (13). Since $f_m(x)\to f(x)$ as $m\to\infty$, thus for all $n\ge N$ and for all $x\in E$,
\[\absolute{f_n(x)-f(x)}\le\epsilon,\]
which completes the proof.
\end{proof}

The following criterion is sometimes useful.

\begin{proposition}
Suppose $f_n\to f$ on $E$. Let
\[M_n=\sup_{x\in E}\absolute{f_n(x)-f(x)}.\]
Then $f_n\rightrightarrows f$ on $E$ if and only if $M_n\to0$ as $n\to\infty$.
\end{proposition}

For series, there is a very convenient test for uniform convergence, due to Weierstrass.

\begin{lemma}[Weierstrass M-test]
Suppose $(f_n)$ is a sequence of functions defined on $E$, and suppose there exists $(M_n)\in\RR^+$ such that $|f_n(x)|\le M_n$ for all $n\ge1$ and for all $x\in E$.

Then $\sum f_n$ converges uniformly on $E$ if $\sum M_n$ converges.
\end{lemma}

\section{Uniform Convergence and Continuity}
We now consider properties preserved by uniform convergence.

\begin{proposition}
Suppose $f_n\rightrightarrows f$ on $E$. Let $x\in E$ be a limit point, let
\[\lim_{t\to x}f_n(t)=A_n.\]
Then $(A_n)$ converges, and $\displaystyle\lim_{t\to x}f(t)=\lim_{n\to\infty}A_n$.
\end{proposition}

\begin{proposition}
Let $(f_n)$ be a sequence of continuous functions on $E$, $f_n\rightrightarrows f$. Then $f$ is continuous in $E$.
\end{proposition}

\begin{definition}[Supremum norm]
If $X$ is a metric space, we denote the set of all complex-valued, continuous, bounded functions with domain $X$ by $C(X)$.

If $f\in C(X)$, we define 
\[\norm{f}\coloneqq\sup_{x\in X}|f(x)|,\]
known as the \vocab{suprenum norm} of $f$.
\end{definition}

\begin{lemma}
$\norm{f}$ gives a norm on $C(X)$.
\end{lemma}

\begin{proof}
Check that $\norm{f}$ satisfies the conditions for a norm:
\begin{enumerate}[label=(\roman*)]
\item 
\end{enumerate}
\end{proof}

\begin{proposition}
$\brac{C(X),\norm{\cdot}}$ is a metric space.
\end{proposition}

\section{Uniform Convergence and Integration}
\begin{theorem}
Assume $(f_n)$ is a sequence of functions defined over $[a,b]$ and each $f_n\in R_\alpha[a,b]$. If $f_n\to f$, then $f\in R_\alpha[a,b]$, and
\[ \lim_{n\to\infty}\int_a^bf_n\dd{\alpha}=\int_a^bf\dd{\alpha}. \]
\end{theorem}

\begin{proof}
Define
\end{proof}

\begin{corollary}
Assume $a_n\in R_\alpha[a,b]$ and
\[ f(x)\coloneqq\sum_{n=0}^\infty a_n(x) \]
converges uniformly. Then it follows
\[ \int_a^bf\dd{\alpha}=\sum_{n=0}^\infty a_n\dd{\alpha}. \]
\end{corollary}

\begin{proof}
Consider the sequence of partial sums 
\[ f_n(x)\coloneqq\sum_{k=0}^na_k(x), \quad n=0,1,\dots \]
It follows $f_n\in R_\alpha[a,b]$ and $f_n\rightrightarrows f$. Apply above theorem to $(f_n)$ and the conclusion follows.
\end{proof}

\section{Uniform Convergence and Differentiation}
\begin{theorem}
$(f_n)$ differentiable on $[a,b]$, $\exists x_0\in[a,b]\suchthat f_n(x_0)\to y_0=f(x_0)$ and $f_n^\prime\rightrightarrows f^\prime$. Then $f_n\rightrightarrows f$ on $[a,b]$, and $f$ is differentiable, $f^\prime(x)=\lim_{n\to\infty}f_n^\prime(x)$ for any $x\in[a,b]$.
\end{theorem}

\begin{proof}
$f_n(x_0)\to y_0$ thus
\end{proof}

\section{Stone--Weierstrass Approximation Theorem}
\begin{theorem}[Weierstrass approximation theorem]
If $f$ is a continuous complex function on $[a,b]$, there exists a sequence of polynomials $P_n$ such that $P_n\rightrightarrows f$ on $[a,b]$.

If $f$ is real, then $P_n$ may be taken real.
\end{theorem}
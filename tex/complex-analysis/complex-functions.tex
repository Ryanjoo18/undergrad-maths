\chapter{Complex Functions}\label{chap:complex-functions}
\section{The Complex Plane}
$\CC$ is a metric space, with metric $d(z,w)=|z-w|$. Hence all notions defined for general metric spaces, as outlined in \cref{chap:basic-topology,chap:num-seq-series,chap:real-analysis_continuity,chap:special-functions}, are applicable to $\CC$.

If $z_0\in\CC$ and $r>0$, we define the \vocab{open disc} of radius $r$ centered at $z_0$ to be
\[D_r(z_0)=\{z\in\CC\mid |z-z_0|<r\}.\]
The \vocab{closed disc} $D_r(z_0)$ of radius $r$ centered at $z_0$ is defined by
\[D_r(z_0)=\{z\in\CC\mid |z-z_0|\le r\}.\]
The \vocab{boundary} of either the open or closed disc is the circle
\[C_r(z_0)=\{z\in\CC\mid |z-z_0|=r\}.\]
Since the \vocab{unit disc} (the open disc centered at the origin and of radius $1$) plays an important role, we will often denote it by $\DD$:
\[\DD=\{z\in\CC\mid |z|<1\}.\]

The last notion we need is that of connectedness. An open set $\Omega\subset\CC$ is \vocab{connected} if it is not possible to find two disjoint non-empty open sets $\Omega_1$ and $\Omega_2$ such that 
\[\Omega=\Omega_1\cup\Omega_2.\]
A connected open set in $\CC$ is called a \vocab{region}. Similarly, a closed set $F$ is connected if one cannot write $F=F_1\cup F_2$ where $F_1$ and $F_2$ are disjoint non-empty closed sets.

There is an equivalent definition of connectedness for open sets in terms of curves, which is often useful in practice: an open set $\Omega$ is connected if and only if any two points in $\Omega$ can be joined by a curve $\gamma$ entirely contained in $\Omega$.
\pagebreak

\section{Functions on The Complex Plane}
\subsection{Holomorphic Functions}
We present the complex analogue of differentiability, which, at first glance, seems no different from the real case.

\begin{definition}
Let $\Omega\subset\CC$ be open. We say $f\colon\Omega\to\CC$ is \vocab{holomorphic} at $z_0\in\Omega$ if
\begin{equation}\label{eqn:holomorphic-quotient-limit}
\lim_{z\to z_0}\frac{f(z)-f(z_0)}{z-z_0}
\end{equation}
exists.
\end{definition}

The value of the limit in \eqref{eqn:holomorphic-quotient-limit} is known as the \emph{derivative} of $f$ at $z_0$; we write
\[f^\prime(z_0)=\lim_{z\to z_0}\frac{f(z)-f(z_0)}{z-z_0}.\]

If $z=z_0+h$ for some $h\in\CC$, we can rewrite the above equation as 
\[f^\prime(z_0)=\lim_{h\to 0}\frac{f(z_0+h)-f(z_0)}{h}.\]
It should be emphasised that $h$ is a complex number that may approach $0$ from any direction.

We can also rewrite
\begin{equation}
f(z_0+h)-f(z_0)-f^\prime(z_0)h=h\epsilon(h),
\end{equation}
where $\epsilon(h)\to0$ as $h\to 0$.

If $f$ is holomorphic at every point of $\Omega$, we say $f$ is \emph{holomorphic on} $\Omega$. 
If $C\subset\CC$ is closed, we say f is holomorphic on $C$ if $f$ is holomorphic in some open set containing $C$.
Finally, we say $f$ is \vocab{entire} if $f$ is holomorphic in all of $\CC$.

\begin{lemma}
If $f$ is holomorphic at $z_0$, then $f$ is continuous at $z_0$.
\end{lemma}

\begin{proof}
Suppose $f$ is holomorphic at $z_0$. Then the limit $\displaystyle\lim_{z\to z_0}\frac{f(z)-f(z_0)}{z-z_0}$ exists. Thus
\begin{align*}
\lim_{z\to z_0}f(z)-f(z_0)
&=\lim_{z\to z_0}\frac{f(z)-f(z_0)}{z-z_0}(z-z_0)\\
&=\lim_{z\to z_0}\frac{f(z)-f(z_0)}{z-z_0}\cdot\lim_{z\to z_0}(z-z_0)\\
&=f^\prime(z_0)\cdot0=0.
\end{align*}
Hence $\displaystyle\lim_{z\to z_0}f(z)=f(z_0)$, so $f$ is continuous at $z_0$.
\end{proof}

The following lemma collects the basic facts about holomorphic functions. We omit the proof, which is essentially identical to the real case.

\begin{lemma}
Suppose $\Omega\subset\CC$ is open, and $f,g\colon\Omega\to\CC$ are holomorphic on $\Omega$. 
\begin{enumerate}[label=(\roman*)]
\item $f+g$ is holomorphic on $\Omega$, and\hfill(sums)
\[(f+g)^\prime=f^\prime+g^\prime.\]
\item $fg$ is holomorphic on $\Omega$, and\hfill(products)
\[(fg)^\prime=f^\prime g+f g^\prime.\]
\item $f/g$ is holomorphic on $\Omega$ (provided $g(z)\neq0$), and\hfill(quotients)
\[\brac{\frac{f}{g}}^\prime=\frac{f^\prime g-fg^\prime}{g^2}.\]
\end{enumerate}
\end{lemma}

\begin{lemma}[Chain rule]
Suppose $\Omega$ and $U$ are open subsets of $\CC$, and $f\colon\Omega\to U$ and $g\colon U\to\CC$ are holomorphic. Then
\begin{equation}
(g\circ f)^\prime(z)=g^\prime(f(z))f^\prime(z)\quad(z\in\Omega).
\end{equation}
\end{lemma}

\begin{example}[Polynomials]
$f(z)=z$ is holomorphic on any open set in $\CC$, and $f^\prime(z)=1$. In fact, any polynomial
\[p(z)=a_nz^n+\cdots+a_1z+a_0\]
is holomorphic in the entire complex plane, and
\[p^\prime(z)=na_nz^{n-1}+\cdots+a_1.\]
\end{example}

\begin{example}
$f(z)=\frac{1}{z}$ is holomorphic on any open set in $\CC$ that does not contain the origin, and $f^\prime(z)=-\frac{1}{z^2}$.
\end{example}

\begin{example}
$f(z)=\overline{z}$ is not holomorphic. Indeed, we have
\[\frac{f(z_0+h)-f(z_0)}{h}=\frac{\overline{h}}{h}\]
which has no limit as $h\to 0$, as one can see by first taking $h$ real and then $h$ purely imaginary.
\end{example}

\subsection{Cauchy--Riemann Equations}
To each complex-valued function $f=u+iv$, we associate the mapping $F(x,y)=(u(x,y), v(x,y))$ from $\RR^2$ to $\RR^2$.

Recall from \cref{chap:multivariable-differentiation} that a function $F(x,y)=\brac{u(x, y),v(x,y)}$ is said to be \emph{differentiable} at a point $P_0=(x_0,y_0)$ if there exists a linear transformation $J\colon\RR^2\to\RR^2$ such that
\[\lim_{|h|\to 0}\frac{F(P_0+h)-F(P_0)-J(h)}{|h|}=0.\]
Equivalently, we can write
\[F(P_0+h)-F(P_0)=J(h)+|h|\epsilon(h),\]
with $\norm{\epsilon(h)}\to 0$ as $|h|\to 0$. The linear transformation $J$ is unique and is called the \emph{derivative} of $F$ at $P_0$. 

If $F$ is differentiable, the partial derivatives of $u$ and $v$ exist, and the linear transformation $J$ can be described in the standard basis of $\RR^2$ by the Jacobian matrix of $F$:
\[J=J_F(x,y)=\begin{pmatrix}
\pdv{u}{x}&\pdv{u}{y}\\
\pdv{v}{x}&\pdv{v}{y}
\end{pmatrix}.\]

In the case of complex differentiation, the derivative is a complex number $f^\prime(z_0)$; in the case of real derivatives, it is a matrix. There is, however, a connection between these two notions, which is given in terms of special relations that are satisfied by the entries of the Jacobian matrix, that is, the partials of $u$ and $v$.

\begin{theorem}[Cauchy--Riemann equations]
Suppose $f$ is holomorphic. Then its real and imaginary parts satisfy
\begin{equation}\label{eqn:cauchy-riemann}
\pdv{u}{x}=\pdv{v}{y},\quad\pdv{u}{y}=-\pdv{v}{x}.
\end{equation}
\end{theorem}

\begin{proof}
Consider the limit
\[f^\prime(z_0)=\lim_{h\to0}\frac{f(z_0+h)-f(z_0)}{h}\]
when $h$ is first real, say $h=h_1+ih_2$ with $h_2=0$. Then, if we write $z=x+iy$, $z_0=x_0+iy_0$, and $f(z)=f(x,y)$, we find that
\begin{align*}
f^\prime(z_0)&=\lim_{h_1\to 0}\frac{f(x_0+h_1,y_0)-f(x_0,y_0)}{h_1}\\
&=\pdv{f}{x}(z_0).
\end{align*}
Now taking $h$ purely imaginary, say $h=ih_2$, a similar argument yields
\begin{align*}
f^\prime(z_0)&=\lim_{h_2\to 0}\frac{f(x_0,y_0+h_2)-f(x_0,y_0)}{ih_2}\\
&=\frac{1}{i}\pdv{f}{y}(z_0)
=-i\pdv{f}{y}(z_0).
\end{align*}
Hence
\[\pdv{f}{x}=-i\pdv{f}{y}.\]
Writing $f=u+iv$, we find after separating real and imaginary parts, that the partials of $u$ and $v$ exist, and they satisfy \eqref{eqn:cauchy-riemann}.
\end{proof}

\begin{remark}
This suggests that complex differentiability is a much more rigid property than one might think at first sight; if $f$ is differentiable then these partial derivatives do exist, and moreover they are subject to a constraint.
\end{remark}

We can clarify the situation further by defining two differential operators
\begin{align*}
\pdv{}{z}&=\frac{1}{2}\brac{\pdv{}{x}-i\pdv{}{y}},\\
\pdv{}{\overline{z}}&=\frac{1}{2}\brac{\pdv{}{x}+i\pdv{}{y}}.
\end{align*}

\begin{proposition}
If $f$ is holomorphic at $z_0$, then
\[\pdv{f}{\overline{z}}=0\]
and thus
\[f^\prime(z_0)=\pdv{f}{z}(z_0)=2\pdv{u}{z}(z_0).\]
If we write $F(x,y)=f(z)$, then $F$ is differentiable in the sense of real variables, and
\[\det J_F(x_0,y_0)=|f^\prime(z_0)|^2.\]
\end{proposition}

\begin{proof}
Taking real and imaginary parts, it is easy to see that the Cauchy--Riemann equations are equivalent to $\pdv{f}{\overline{z}}=0$.

Moreover, by our earlier observation,
\[f^\prime(z_0)
=\frac{1}{2}\brac{\pdv{f}{x}(z_0)-i\pdv{f}{y}(z_0)}\\
=\pdv{f}{z}(z_0),\]
and the Cauchy--Riemann equations give
\[\pdv{f}{z}=2\pdv{u}{z}.\]
To prove that $F$ is differentiable it suffices to observe that if $h=(h_1,h_2)$ and $h=h_1+ih_2$, then the Cauchy--Riemann equations imply
\[J_F(x_0,y_0)(h)=\brac{\pdv{u}{x}-i\pdv{u}{y}}(h_1+ih_2)=f^\prime(z_0)h,\]
where we have identified a complex number with the pair of real and imaginary parts. 
After a final application of the Cauchy--Riemann equations, the above results imply that
\[\det J_F(x_0,y_0)
=\pdv{u}{x}\pdv{v}{y}-\pdv{v}{x}\pdv{u}{y}
=\brac{\pdv{u}{x}}^2+\brac{\pdv{u}{y}}^2=\absolute{2\pdv{u}{z}}^2=|f^\prime(z_0)|^2.\]
\end{proof}

So far, we have assumed that $f$ is holomorphic and deduced relations satisfied by its real and imaginary parts. The next result contains an important converse, which completes the circle of ideas presented here.

\begin{proposition}
Suppose $f=u+iv$ is a complex-valued function defined on an open set $\Omega\subset\CC$. If $u,v\colon\RR^2\to\RR$ are continuously differentiable and satisfy the Cauchy--Riemann equations on $\Omega$, then $f$ is holomorphic on $\Omega$ and $f^\prime(z)=\pdv{f}{z}$.
\end{proposition}

\begin{proof}
Write
\[u(x+h_1,y+h_2)-u(x,y)=\pdv{u}{x}h_1+\pdv{u}{y}h_2+|h|\epsilon_1(h)\]
and
\[v(x+h_1,y+h_2)-v(x,y)=\pdv{v}{x}h_1+\pdv{v}{y}h_2+|h|\epsilon_2(h)\]
where the remainders $\epsilon_1(h),\epsilon_2(h)\to 0$ as $|h|\to 0$, and $h=h_1+ih_2$. Using the Cauchy--Riemann equations we find that
\[f(z+h)-f(z)=\brac{\pdv{u}{x}-i\pdv{u}{y}}(h_1+ih_2)+|h|\epsilon(h),\]
where $\epsilon(h)=\epsilon_1(h)+\epsilon_2(h)\to 0$ as $|h|\to 0$. Hence $f$ is holomorphic and
\[f^\prime(z)=2\pdv{u}{z}=\pdv{f}{z}.\] 
\end{proof}

\begin{example}
The function $f(z)=\bar{z}$ is not holomorphic.
\begin{proof}
Let $u,v:\RR^2\to\RR$ be the components of $f$. Then $u(x,y) = x$, $v(x,y)=-y$, and so
\[\pdv{u}{x}=1,\quad\pdv{u}{y}=0,\quad\pdv{v}{x}=0,\quad\pdv{v}{y}=-1.\]
Since $u$ and $v$ do not satisfy the Cauchy--Riemann equations, $f$ is not holomorphic.
\end{proof}
\end{example}

We shall prove later that the derivative of an holomorphic function is itself holomorphic. By this fact, $u$ and $v$ have continuous second partial derivatives. Differentiating \eqref{eqn:cauchy-riemann} again gives
\[\pdv[2]{u}{x}=\pdv{v}{x,y},\quad\pdv[2]{u}{y}=-\pdv{v}{y,x}.\]
Hence
\begin{equation}\label{eqn:harmonic-function-diff}
\pdv[2]{u}{x}+\pdv[2]{u}{y}=0.
\end{equation}
\eqref{eqn:harmonic-function-diff} is called \emph{Laplace's equation}. Solutions to \eqref{eqn:harmonic-function-diff} are said to be \vocab{harmonic}. If two harmonic functions $u$ and $v$ satisfy the Cauchy--Riemann equations \eqref{eqn:cauchy-riemann}, then $v$ is called the \emph{conjugate harmonic function} of $u$.

\begin{proposition}
Let $\Omega$ be either $\CC$ or some open disk. If $u\colon\Omega\to\RR$ is a harmonic function, then $u$ has a harmonic conjugate.
\end{proposition}

\begin{proof}
Let $U=D_R(0)$ where $0<R\le\infty$, and let $u:U\to\RR$ be a harmonic function. We will find a harmonic function $v$ such that $u$ and $v$ satisfy the Cauchy--Riemann equations.

Let
\[v(x,y)=\int_{0}^{y}u_x(x,t)\dd{t}+\phi(x)\]
and determine $\phi$ so that $u_x=-u_y$. Differentiating both sides of this equation with respect to $x$ gives
\begin{align*}
v_x(x,y)&=\int_{0}^{y}u_{xx}(x,t)\dd{t}+\phi^\prime(x)\\
&=-\int_{0}^{y}u_{yy}(x,t)\dd{t}+\phi^\prime(x)\\
&=-u_y(x,y)+u_y(x,0)+\phi^\prime(x)
\end{align*}
So it must be that $\phi^\prime(x)=-u_y(x,0)$. It is easily checked that $u$ and
\[v(x,y)=\int_{0}^{y}u_x(x,t)\dd{t}-\int_{0}^{x}u_y(s,0)\dd{s}\]
so satisfy the Cauchy--Riemann equations.
\end{proof}

\subsection{Power Series}
Recall our discussion of power series in \cref{chap:special-functions}, including Hadamard's formula \eqref{eqn:hadamard-formula-power-series} for the radius of convergence of a power series.

Having defined complex differentiation, we now prove the complex analogue of \ref{prop:real-power-series-derivative}, concerning the derivative of complex power series.

\begin{proposition}\label{prop:power-series-derivative}
The power series $f(z)=\sum_{n=0}^{\infty}a_n z^n$ defines a holomorphic function in its disc of convergence. Then the derivative of $f$ is obtained by differentiating term-by-term the series of $f$:
\begin{equation}
f^\prime(z)=\sum_{n=0}^{\infty}na_n z^{n-1}.
\end{equation}
Moreover, $f^\prime$ has the same radius of convergence as $f$.
\end{proposition}

\begin{corollary}
A power series is infinitely complex differentiable in its disc of convergence, and the higher derivatives are also power series obtained by termwise differentiation.
\end{corollary}

\begin{definition}
We say $f\colon\Omega\subset\CC\to\CC$ is \vocab{analytic} (or have a \emph{power series expansion}) at $z_0\in\Omega$ if there exists a power series $\sum a_n(z-z_0)^n$ centered at $z_0$, with positive radius of convergence, such that
\[f(z)=\sum_{n=0}^{\infty}a_n(z-z_0)^n\quad\text{for all $z$ in a neighbourhood of $z_0$.}\]
If $f$ has a power series expansion at every point in $\Omega$, we say $f$ is \emph{analytic on} $\Omega$.
\end{definition}

\subsubsection{Exponential Function}
We are familiar with the exponential function $e^x$ of a real variable, which has the property that $(e^x)^\prime=e^x$. The complex exponential has the same property.

\begin{lemma}
$\exp^\prime(z)=\exp(z)$ for all $z\in\CC$.
\end{lemma}

\begin{proof}
Using \ref{prop:power-series-derivative}, we calculate the derivative of $\exp(z)$ by differentiating term-by-term:
\[\exp^\prime(z)=(1)^\prime+\sum_{n=1}^\infty\brac{\frac{z^n}{n!}}^\prime=\sum_{n=1}^\infty\frac{nz^{n-1}}{n!}=\sum_{n=1}^\infty\frac{z^{n-1}}{(n-1)!}=\exp(z).\]
\end{proof}

\begin{lemma}[Basic properties of $\exp$] \
\begin{enumerate}[label=(\roman*)]
\item There exists a positive number $\pi$ such that $e^\frac{\pi i}{2}=i$ and such that $e^z=1$ if and only if $\frac{z}{2\pi i}$ is an integer.
\item $\exp$ is a periodic function, with period $2\pi i$. 
\item The mapping $t\mapsto e^{it}$ maps the real axis onto the unit circle.
\item If $w\in\CC$, $w\neq0$, then $w=e^z$ for some $z$.
\end{enumerate}
\end{lemma}

We shall encounter the integral of $(1+x^2)^{-1}$ over the real line. To evaluate it, put $\phi(t)=\frac{\sin t}{\cos t}$ in $\brac{-\frac{\pi}{2},\frac{\pi}{2}}$. By (6), $\phi^\prime=1+\phi^2$. Hence $\phi$ is a monotonically increasing mapping of $\brac{-\frac{\pi}{2},\frac{\pi}{2}}$ onto $(-\infty,\infty)$, and we obtain 
\[\int_{-\infty}^{\infty}\frac{1}{1+x^2}\dd{x}=\int_{-\frac{\pi}{2}}^{\frac{\pi}{2}}\frac{\phi^\prime(t)}{1+\phi^2(t)}\dd{t}=\int_{-\frac{\pi}{2}}^{\frac{\pi}{2}}\dd{t}=\pi.\]
\pagebreak

\section{Integration Along Curves}
\begin{definition}
A \vocab{parametrised curve} is a function $z(t)$ which maps a closed interval $[a,b]\subset\RR$ to $\CC$.

We say that the parametrised curve is \vocab{smooth} if $z^\prime(t)$ exists and is continuous on $[a,b]$, and $z^\prime(t)\neq 0$ for $t\in[a,b]$.
\end{definition}

\begin{remark}
At the endpoints $t=a$ and $t=b$, the quantities $z^\prime(a)$ and $z^\prime(b)$ are interpreted as the one-sided limits
\[z^\prime(a)=\lim_{h\to 0^+}\frac{z(a+h)-z(a)}{h},\quad 
z^\prime(b)=\lim_{h\to 0^-}\frac{z(b+h)-z(b)}{h},\]
which are the right-hand derivative of $z(t)$ at $a$, and the left-hand derivative of $z(t)$ at $b$, respectively.
\end{remark}

Similarly we say that the parametrised curve is \vocab{piecewise-smooth} if $z$ is continuous on $[a,b]$ and if there exist points
\[a=a_0<a_1<\cdots<a_n=b,\]
where $z(t)$ is smooth in the intervals $[a_i,a_{i+1}]$.

We say two parametrisations
\[z\colon[a,b]\to\CC,\quad\overline{z}\colon[c,d]\to\CC\]
are \emph{equivalent} if there exists a continuously differentiable bijection $s\mapsto t(s)$ from $[c,d]$ to $[a,b]$ so that $t^\prime(s)>0$ and
\[\overline{z}(s)=z(t(s)).\]

\begin{remark}
The condition $t^\prime(s)>0$ says precisely that the orientation is preserved: as $s$ travels from $c$ to $d$, then $t(s)$ travels from $a$ to $b$.
\end{remark}

The family of all parametrisations that are equivalent to $z(t)$ determines a smooth curve $\gamma\subset\CC$, namely the image of $[a,b]$ under $z$ with the orientation given by $z$ as $t$ travels from $a$ to $b$. 

The points $z(a)$ and $z(b)$ are called the \emph{end-points} of the curve, and are independent on the parametrisation. 
Since $\gamma$ carries an orientation, it is natural to say that $\gamma$ begins at $z(a)$ and ends at $z(b)$.

A smooth or piecewise-smooth curve is \vocab{closed} if $z(a)=z(b)$ for any of its parametrisations. 

Finally, a smooth or piecewise-smooth curve is \vocab{simple} if it is not self-intersecting, that is, $z(t)\neq z(s)$ unless $s=t$. (Of course, if the curve is closed to begin with, then we say that it is simple whenever $z(t)\neq z(s)$ unless $s=t$, or $s=a$ and $t=b$.)

For brevity, we shall call any piecewise-smooth curve a \emph{curve}, since these will be the objects we shall be primarily concerned with.

A basic example consists of a circle. Consider the circle $C_r(z_0)$ centered at $z_0$ and of radius $r$:
\[C_r(z_0)=\{z\in\CC\mid |z-z_0|=r\}.\]

\begin{definition}[Orientation]
The positive orientation (counterclockwise) is the one that is given by the standard parametrisation
\[z(t)=z_0+re^{it}\quad(0\le t\le 2\pi).\]
The negative orientation (clockwise) is given by
\[z(t)=z_0+re^{-it}\quad(0\le t\le 2\pi).\]
\end{definition}

In the following chapters, we shall denote by $C$ a general positively oriented circle.

\begin{definition}[Integral along curve]
Given a smooth curve $\gamma\subset\CC$ parametrised by $z\colon[a,b]\to\CC$, and $f$ a continuous function on $\gamma$, define the \vocab{integral of $f$ along $\gamma$} by
\begin{equation}
\int_{\gamma}f(z)\dd{z}\colonequals\int_{a}^{b}f\brac{z(t)}z^\prime(t)\dd{t}.
\end{equation}
\end{definition}

In order for this definition to be meaningful, we must show that the integral on the RHS is independent of the parametrisation chosen for $\gamma$.
Suppose $\overline{z}$ is an equivalent parametrisation as above. Then the change of variables formula and the chain rule imply that
\[\int_{a}^{b}f\brac{z(t)}z^\prime(t)\dd{t}
=\int_{c}^{d}f\brac{z(t(s))}z^\prime\brac{t(s)}t^\prime(s)\dd{s}
=\int_{c}^{d}f\brac{\overline{z}(s)}\overline{z}^\prime(s)\dd{s}.\]
This proves that the integral of $f$ over $\gamma$ is well defined.

If $\gamma$ is piecewise smooth, then the integral of $f$ over $\gamma$ is simply the sum of the integrals of $f$ over the smooth parts of $\gamma$, so if $z(t)$ is a piecewise-smooth parametrisation as before, then
\[\int_{\gamma}f(z)\dd{z}=\sum_{i=1}^{n}\int_{a_{i-1}}^{a_i}f\brac{z(t)}z^\prime(t)\dd{t}.\]
By definition, the \vocab{length} of the smooth curve $\gamma$ is
\[\Lambda(\gamma)=\int_{a}^{b}|z^\prime(t)|\dd{t}.\]
Arguing as we just did, it is clear that this definition is also independent of the parametrisation. 
Also, if $\gamma$ is only piecewise-smooth, then its length is the sum of the lengths of its smooth parts.

\begin{lemma}[Basic properties] \
\begin{enumerate}[label=(\roman*)]
\item Linearity: if $\alpha,\beta\in\CC$, then
\[\int_{\gamma}\brac{\alpha f(z)+\beta g(z)}\dd{z}=\alpha\int_{\gamma}f(z)\dd{z}+\beta\int_{\gamma}g(z)\dd{z}.\]
\item If $\gamma^-$ is $\gamma$ with the reverse orientation, then
\[\int_{\gamma}f(z)\dd{z}=-\int_{\gamma^-}f(z)\dd{z}.\]
\item One has the inequality
\[\absolute{\int_{\gamma}f(z)\dd{z}}\le\sup_{z\in\gamma}|f(z)|\cdot\Lambda(\gamma).\]
\end{enumerate}
\end{lemma}

A \vocab{primitive} for $f$ on $\Omega$ is a function $F$ that is holomorphic on $\Omega$ and such that $F^\prime(z)=f(z)$ for all $z\in\Omega$.

\begin{proposition}
If a continuous function $f$ has a primitive $F$ in $\Omega$, and $\gamma$ is a curve in $\Omega$ that begins at $w_1$ and ends at $w_2$, then
\begin{equation}
\int_{\gamma}f(z)\dd{z}=F(w_2)-F(w_1).
\end{equation}
\end{proposition}

\begin{corollary}
If $\gamma$ is a closed curve in an open set $\Omega$, and $f$ is continuous and has a primitive in $\Omega$, then
\[\int_{\gamma}f(z)\dd{z}=0.\]
\end{corollary}

\begin{proof}
This is immediate since the end-points of a closed curve coincide.
\end{proof}

\begin{corollary}
If $f$ is holomorphic in a region $\Omega$ and $f^\prime=0$, then $f$ is constant.
\end{corollary}
\chapter{$L^p$-Spaces}
\section{Convex Functions and inequalities}
\begin{definition}[Convex function]
We say $\phi:(a,b)\to\RR$ is \vocab{convex} if
\begin{equation}\label{eqn:convex-inequality}
\phi\brac{(1-\lambda)x+\lambda y}\le(1-\lambda)\phi(x)+\lambda\phi(y)
\end{equation}
for all $x,y\in(a,b)$ and $0\le\lambda\le1$.
\end{definition}

\todo{figure}

\begin{example}
The exponential function is convex on $(-\infty,\infty)$.
\end{example}

Also, \eqref{eqn:convex-inequality} is equivalent to
\[\frac{\phi(t)-\phi(s)}{t-s}\le\frac{\phi(u)-\phi(t)}{u-t}\]
whenever $a<s<t<u<b$.

The mean value theorem for differentiation, combined with (2), shows immediately that a real differentiable function $\phi:(a,b)\to\RR$ is convex if and only if $a<s<t<b$ implies $\phi^\prime(s)\le\phi^\prime(t)$, i.e., if and only if the derivative $\phi^\prime$ is monotonically increasing.

\begin{lemma}

\end{lemma}

\section{The $L^p$-spaces}


\section{Approximation by Ccontinuous Functions}

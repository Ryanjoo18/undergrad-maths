\begin{center}
\

\vspace{3cm}

{\color{schoolbusyellow}\Huge
\uppercase{Topics in}\\[1em]
\uppercase{Pure Mathematics}}

\vspace{9cm}

{\color{schoolbusyellow}\huge Ryan Joo}
\end{center}
\thispagestyle{empty}
\pagecolor{smalt(darkpowderblue)}
\pagebreak

\pagecolor{white}
\thispagestyle{empty}
\

\vfill

\begin{quote}
\textit{The mathematician does not study mathematics because it is useful; he studies it because he delights in it and he delights in it because it is beautiful.}

\begin{flushright}--- Henri Poincar\'{e} (1854--1912)\\
French mathematician and theoretical physicist\end{flushright}
\end{quote}

\vfill

Copyright \copyright \ 2025 by Ryan Joo.

This book is licensed under the terms of the Creative Commons Attribution-NonCommercial 4.0 International License (\url{https://creativecommons.org/licenses/by-nc/4.0}), which permits any noncommercial use, sharing, adaptation, distribution and reproduction in any medium or format, as long as you give appropriate credit to original author and source, provide a link to the Creative Commons license, and indicate if changes were made. The images or other third party material in this book are included in the book's Creative Commons license, unless indicated otherwise in a credit line to the material. If material is not included in the book's Creative Commons license and your intended use is not permitted by statutory regulation or exceeds the permitted use, you will need to obtain permission directly from the copyright holder.

This is (still!) an incomplete draft. Please send corrections and comments to \url{ryanjooruian18@gmail.com}, or pull-request at \url{https://github.com/Ryanjoo18/undergrad-maths}.

Typeset using \LaTeX.

Last updated \today.
\pagebreak

\frontmatter
\section*{Preface}
\ifprelim
The reader is not assumed to have any mathematical prerequisites, although some experience with proofs may be helpful. \textbf{Preliminary topics} such as logic and methods of proofs (\cref{chap:logic-proofs}), and basic set theory (\cref{chap:set-theory}) are covered in \cref{part:prelim}.
\fi

\ifabsalg
\cref{part:abstract-algebra} covers \textbf{abstract algebra}, which follows \cite{dummit-foote,artin}.
\begin{itemize}
\item \cref{chap:groups} introduces groups.
\end{itemize}
\fi

\iflinalg
\cref{part:linear-algebra} covers \textbf{linear algebra}, which follows \cite{axler}.
\begin{itemize}
\item \cref{chap:vector-spaces} introduces vector spaces, subspaces, span, linear independence, bases and dimension.
\item \cref{chap:linear-maps} concerns linear maps and related concepts.
\end{itemize}
\fi

\ifanalysis
\cref{part:real-analysis} covers \textbf{real analysis}, which follows \cite{rudin,apostol}.
\begin{itemize}
\item \cref{chap:number-systems} introduces the real and complex number systems.
\item \cref{chap:basic-topology} covers basic point-set topology, in the context of metric spaces.
\item \cref{chap:num-seq-series} concerns numerical sequences and series, in particular their convergence.
\item \cref{chap:real-analysis_continuity} covers continuity of functions.
\item \cref{chap:differentiation} covers differentiation.
\item \cref{chap:rs-integration} covers Riemann--Stieljes integration.
\item \cref{chap:func-seq-series} covers sequences and series of functions.
\item \cref{chap:special-functions} covers some special functions, most notably power series and the fourier series.
\end{itemize}
\fi

\iftop
\cref{part:topology} covers \textbf{general topology}, which follows \cite{munkres}.
\fi

For ease of reference, important terms are \vocab{coloured} when first defined, and are included in the glossary; less important terms are \emph{italicised} when first defined, and are not included in the glossary.
\pagebreak

\section*{Note on Problem Solving}
Mathematics is about problem solving. In \cite{polya}, George P\'{o}lya outlined the following problem solving cycle.
\begin{enumerate}
\item \textbf{Understand the problem}

Ask yourself the following questions:
\begin{itemize}
\item Do you understand all the words used in stating the problem?
\item Is it possible to satisfy the condition? Is the condition sufficient to determine the unknown? Or is it insufficient? Or redundant? Or contradictory?
\item What are you asked to find or show? Can you restate the problem in your own words?
\item Draw a figure. Introduce suitable notation.
\item Is there enough information to enable you to find a solution?
\end{itemize}

\item \textbf{Devise a plan}

A partial list of heuristics -- good rules of thumb to solve problems -- is included:
\begin{multicols}{2}
\begin{itemize}
\item Guess and check
\item Look for a pattern
\item Make an orderly list
\item Draw a picture
\item Eliminate possibilities
\item Solve a simpler problem
\item Use symmetry
\item Use a model
\item Consider special cases
\item Work backwards
\item Use direct reasoning
\item Use a formula
\item Solve an equation
\item Be ingenious
\end{itemize}
\end{multicols}

\item \textbf{Execute the plan}

This step is usually easier than devising the plan. In general, all you need is care and patience, given that you have the necessary skills. Persist with the plan that you have chosen. If it continues not to work discard it and choose another. Don't be misled, this is how mathematics is done, even by professionals.

\begin{itemize}
\item Carrying out your plan of the solution, check each step. Can you see clearly that the step is correct? Can you prove that it is correct?
\end{itemize}

\item \textbf{Check and expand}

P\'{o}lya mentions that much can be gained by taking the time to reflect and look back at what you have done, what worked, and what didn't. Doing this will enable you to predict what strategy to use to solve future problems.

Look back reviewing and checking your results. Ask yourself the following questions:
\begin{itemize}
\item Can you check the result? Can you check the argument?
\item Can you derive the solution differently? Can you see it at a glance?
\item Can you use the result, or the method, for some other problem?
\end{itemize}
\end{enumerate}

Building on P\'{o}lya's problem solving strategy, Schoenfeld \cite{schoenfeld} came up with the following framework for problem solving, consisting of four components:
\begin{enumerate}
\item \textbf{Cognitive resources}: the body of facts and procedures at one's disposal.
\item \textbf{Heuristics}: `rules of thumb' for making progress in difficult situations.
\item \textbf{Control}: having to do with the efficiency with which individuals utilise the knowledge at their disposal. Sometimes, this is referred to as metacognition, which can be roughly translated as `thinking about one's own thinking'.
\begin{enumerate}
\item These are questions to ask oneself to monitor one's thinking.
\begin{itemize}
    \item What (exactly) am I doing? [Describe it precisely.] Be clear what I am doing NOW. Why am I doing it? [Tell how it fits into the solution.]
    \item Be clear what I am doing in the context of the BIG picture -- the solution. Be clear what I am going to do NEXT.
\end{itemize}

\item Stop and reassess your options when you
\begin{itemize}
    \item cannot answer the questions satisfactorily [probably you are on the wrong track]; OR
    \item are stuck in what you are doing [the track may not be right or it is right but it is at that moment too difficult for you].
\end{itemize}

\item Decide if you want to
\begin{itemize}
    \item carry on with the plan,
    \item abandon the plan, OR
    \item put on hold and try another plan.
\end{itemize}
\end{enumerate}

\item \textbf{Belief system}: one's perspectives regarding the nature of a discipline and how one goes about working on it.
\end{enumerate}
\pagebreak

\tableofcontents
\pagebreak
%\printglossary[type=\acronymtype]
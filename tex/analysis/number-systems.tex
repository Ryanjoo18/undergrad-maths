\chapter{Real and Complex Number Systems}\label{chap:number-systems}

\begin{summary}
\item Supremum, infimum.
\item Construction and properties of the real field $\RR$.
\item Construction and properties of the complex field $\CC$.
\item Construction and properties of the Euclidean space $\RR^n$.
\end{summary}

\section{Ordered Sets and Boundedness}
\subsection{Definitions}
Let $S$ be a set.
\begin{definition}
An \vocab{order}\index{order} on $S$ is a binary relation $<$ such that
\begin{enumerate}[label=(\roman*)]
\item for all $x,y\in S$, exactly one of $x<y$, $x=y$, or $y<x$ holds;\hfill(trichotomy)
\item if $x,y,z\in S$ are such that $x<y$ and $y<z$, then $x<z$.\hfill(transitivity)
\end{enumerate}
$S$ is an \vocab{ordered set} if it has an order; denote it by $(S,<)$.
\end{definition}

\begin{notation}
We write $x\le y$ if $x<y$ or $x=y$. We define $>$ and $\ge$ in the obvious way.
\end{notation}

\begin{definition}[Boundedness]
Let $E\subset S$, where $S$ is an ordered set.
\begin{enumerate}[label=(\roman*)]
\item $E$ is \vocab{bounded above} if there exists $\beta\in S$ such that $x\le\beta$ for all $x\in E$; we call $\beta$ an \emph{upper bound} of $E$.
\item $E$ is \vocab{bounded below} if there exists $\beta\in S$ such that $x\ge\beta$ for all $x\in E$; we call $\beta$ a \emph{lower bound} of $E$.
\end{enumerate}
$E$ is \vocab{bounded} in $S$ if it is bounded above and below.
\end{definition}

\begin{definition}[Supremum, infimum]
We say $\alpha\in S$ is the \vocab{supremum}\index{supremum} of $E$ if
\begin{enumerate}[label=(\roman*)]
\item $\alpha$ is an upper bound for $E$;
\item if $\beta<\alpha$ then $\beta$ is not an upper bound of $E$, i.e. $\exists x\in S\suchthat x>\beta$ (least upper bound).
\end{enumerate}

Likewise, we say $\alpha\in S$ is the \vocab{infimum}\index{infimum} of $E$ if 
\begin{enumerate}[label=(\roman*)]
\item $\alpha$ is a lower bound for $E$;
\item if $\beta>\alpha$ then $\beta$ is not a lower bound of $E$, i.e. $\exists x\in S\suchthat x<\beta$ (greatest lower bound).
\end{enumerate}
\end{definition}

\begin{remark}
It is not necessary for the supremum and infimum of $E$ to be in $E$.
\end{remark}

\begin{lemma}[Uniqueness of suprenum]
If $E$ has a supremum, then it is unique.
\end{lemma}

\begin{proof}
Suppose $\alpha$ are $\beta$ be suprema of $E$.

Since $\beta$ is a supremum, it is an upper bound for $E$. Since $\alpha$ is a supremum, then it is the \emph{least} upper bound, so $\alpha\le\beta$. Interchanging the roles of $\alpha$ and $\beta$ gives $\beta\le\alpha$. Hence $\alpha=\beta$.
\end{proof}

Since the supremum and infimum are unique, we can give them a notation.

\begin{notation}
Denote the supremum of $E$ by $\sup E$, the infimum by $\inf E$.
\end{notation}

\begin{example}
Let $E=\crbrac{\frac{1}{n}\:\big|\:n\in\NN}$. Then $\sup E=1$, $\inf E=0$.
\begin{proof}
It is clear that $1$ is an upper bound for $E$. Suppose $\beta<1$. Since $1\in E$, evidently $\beta$ is not an upper bound for $E$. Hence $\sup E=1$.

It is clear that $0$ is a lower bound for $E$. Suppose $\beta>0$. Pick $n=\floor{\frac{1}{\beta}}+1$, then $\beta>\frac{1}{n}$, so $\beta$ is not a lower bound for $E$. Hence $\inf E=0$.
\end{proof}
\end{example}
\pagebreak

\subsection{Least-upper-bound Property}
\begin{definition}
An ordered set $S$ has the \vocab{least-upper-bound property} (l.u.b.) if every non-empty subset of $S$ that is bounded above has a supremum in $S$.

We define the \emph{greatest-lower-bound property} similarly.
\end{definition}

\begin{proposition}
Suppose $S$ is an ordered set. If $S$ has the least-upper-bound property, then $S$ has the greatest-lower-bound property.
\end{proposition}

\begin{proof}
Suppose $S$ has the least-upper-bound property. Let non-empty $B\subset S$ be bounded below. We want to show that $\inf B\in S$.

Let $L\subset S$ be the set of all lower bounds of $B$; that is,
\[L=\{y\in S\mid y\le x\forall x\in B\}.\]

Since $B$ is bounded below, $B$ has a lower bound, so $L\neq\emptyset$. Since every $x\in B$ is an upper bound of $L$, $L$ is bounded above. By the least-upper-bound property of $S$, we have that $\sup L\in S$.

\begin{claim}
$\inf B=\sup L$.
\end{claim}

To show that $\sup L=\inf B$ (greatest lower bound), we need to show that (i) $\sup L$ is a lower bound of $B$, (ii) and $\sup L$ is the greatest of the lower bounds.
\begin{enumerate}[label=(\roman*)]
\item Suppose $\gamma<\sup L$, then $\gamma$ is not an upper bound of $L$. Since $B$ is the set of upper bounds of $L$, $\gamma\notin B$. Considering the contrapositive, if $\gamma\in B$, then $\gamma\ge\sup L$. Hence $\sup L$ is a lower bound of $B$, and thus $\sup L\in L$.
\item If $\sup L<\beta$ then $\beta\notin L$, since $\sup L$ is an upper bound of $L$. In other words, $\sup L$ is a lower bound of $B$, but $\beta$ is not if $\beta>\sup L$. This means that $\sup L$ is the greatest of the lower bounds.
\end{enumerate}
Hence $\inf B=\sup L\in S$.
\end{proof}

\begin{corollary}
If $S$ has the greatest-lower-bound property, then it has the least-upper-bound property.

Hence $S$ has the least-upper-bound property if and only if $S$ has the greatest-lower-bound property.
\end{corollary}
\pagebreak

\subsection{Properties of Suprema and Infima}
This section discusses some fundamental properties of the supremum that will be useful in this text. There is a corresponding set of properties of the infimum that the reader should formulate for himself.

The next result shows that a set with a supremum contains numbers arbitrarily close to its supremum.

\begin{lemma}[Approximation property]\label{lemma:sup-approx}
Let $S\subset\RR$ be non-empty, $b=\sup S$. Then for every $a<b$ there exists $x\in S$ such that
\[a<x\le b.\]
\end{lemma}

\begin{proof}
We first show $x\le b$. Since $b=\sup S$ is an upper bound of $S$, $x\le b$ for all $x\in S$.

We now show there exist $x\in S$ such that $a<x$. Suppose otherwise, for a contradiction, that $x\le a$ for every $x\in S$. Then $a$ would be an upper bound for $S$. But since $a<b$ and $b$ is the supremum, this means $a$ is smaller than the least upper bound, a contradiction.
\end{proof}

For the rest of this section, suppose $S$ has the least-upper-bound property.

\begin{lemma}[Additive property]
Given non-empty $A,B\subset S$, let
\[C=\{x+y\mid x\in A,y\in B\}.\]
If each of $A$ and $B$ has a supremum, then $C$ has a supremum, and
\[\sup C=\sup A+\sup B.\]
\end{lemma}

\begin{proof}
Let $a=\sup A$, $b=\sup B$. Let $z\in C$, then $z=x+y$ for some $x\in A$, $y\in B$. Then
\[z=x+y\le a+b,\]
so $a+b$ is an upper bound for $C$. Since $C$ is non-empty and bounded above, by the lub property of $S$, $C$ has a supremum in $S$. 

Let $c=\sup C$. To show that $a+b=c$, we need to show that (i) $a+b\ge c$, and (ii) $a+b\le c$.
\begin{enumerate}[label=(\roman*)]
\item Since $c$ is the \emph{least} upper bound for $C$, and $a+b$ is an upper bound for $C$, we must have that $c\le a+b$.
\item Choose any $\epsilon>0$. By \ref{lemma:sup-approx} there exist $x\in A$ and $y\in B$ such that
\[a-\epsilon<x,\quad b-\epsilon<y.\]
Adding these inequalities gives
\[a+b-2\epsilon<x+y\le c.\]
Thus $a+b<c+2\epsilon$ for every $\epsilon>0$. Hence $a+b\le c$.
\end{enumerate}
\end{proof}

\begin{lemma}[Comparison property]
Let non-empty $A,B\subset S$ such that $a\le b$ for every $a\in A$, $b\in B$. If $B$ has a supremum, then $A$ has a supremum, and
\[\sup A\le\sup B.\]
\end{lemma}

\begin{proof}
Let $\beta=\sup B$. Since $\beta$ is a supremum for $B$, then $b\le\beta$ for all $b\in B$.

Let $a\in A$ and choose any $b \in B$. Since $a \le b$ and $t \le \beta$, $a\le\beta$. Thus $\beta$ is an upper bound for $A$.

Since $A$ is non-empty and bounded above, by the lub property of $S$, $A$ has a supremum in $S$; let $\alpha=\sup A$. Since $\beta$ is an upper bound for $A$, and $\alpha$ is the \emph{least} upper bound for $A$, we have that $\alpha\le\beta$, as desired.
\end{proof}

\begin{lemma}\label{lemma:sup-inf-negative}
Let $B\subset S$ be non-empty and bounded below. Let
\[A=-B\colonequals\{-b\mid b\in B\}.\]
Then $A$ is non-empty and bounded above. Furthermore, $\inf B$ exists, and $\inf B=-\sup A$.
\end{lemma}

\begin{proof}
Since $B$ is non-empty, so is $A$. Since $B$ is bounded below, let $\beta$ be a lower bound for $B$. Then $b\ge\beta$ for all $b\in B$, which implies $-b\le-\beta$ for all $b\in B$. Hence $a\le-\beta$ for all $a\in A$, so $-\beta$ is an upper bound for $A$.

Since $A$ is non-empty and bounded above, by the lub property of $S$, $A$ has a supremum. Then $a\le\sup A$ for all $a\in A$, so $b\ge-\sup A$ for all $b\in B$. Thus $-\sup A$ is a lower bound for $B$.

Also, we saw before that if $\beta$ is a lower bound for $B$ then $-\beta$ is an upper bound for $A$. Then $-\beta \ge \sup A$ (since $\sup A$ is the least upper bound), so $\beta \le -\sup A$. Therefore $-\sup A$ is the greatest lower bound of $B$.
\end{proof}
\pagebreak

\subsection{Ordered Fields}
\begin{definition}[Ordered field]
A field $F$ is an \vocab{ordered field} if there exists an order $<$ on $F$ such that for all $x,y,z\in F$,
\begin{enumerate}[label=(\roman*)]
\item if $y<z$ then $x+y<x+z$;
\item if $x>0$ and $y>0$ then $xy>0$.
\end{enumerate}
\end{definition}

If $x>0$, we call $x$ \emph{positive}; if $x<0$, $x$ is \emph{negative}.

All the familiar rules for working with inequalities apply in every ordered field: Multiplication by positive [negative] quantities preserves [reverses] inequalities, no square is negative, etc. The following result lists some of these. 

\begin{lemma}[Basic properties]\label{lemma:ordered-field-properties}
Let $F$ be an ordered field, $x,y,z\in F$.
\begin{enumerate}[label=(\roman*)]
\item If $x>0$ then $-x<0$, and vice versa.
\item If $x>0$ and $y<z$ then $xy<xz$.
\item If $x<0$ and $y<z$ then $xy>xz$.
\item If $x\neq0$ then $x^2>0$. In particular, $1>0$.
\item If $0<x<y$ then $0<\frac{1}{y}<\frac{1}{x}$.
\end{enumerate}
\end{lemma}

\begin{proof} \
\begin{enumerate}[label=(\roman*)]
\item If $x>0$ then $0=-x+x>-x+0$, so that $-x<0$.

If $x<0$ then $0=-x+x<-x+0$, so that $-x>0$.
\item Since $z>y$, we have $z-y>y-y=0$, so $x(z-y)>0$. Hence
\[xz=x(z-y)+xy>0+xy=xy.\]
\item By (i) and (ii),
\[-[x(z-y)]=(-x)(z-y)>0,\]
so that $x(z-y)<0$. Hence $xz<xy$.
\item If $x>0$, part (ii) of the above definition gives $x^2>0$.

If $x<0$, then $-x>0$ so $(-x)^2>0$. But $x^2=(-x)^2$.

Since $1=1^2$, $1>0$.
\item If $y>0$ and $v\le0$, then $yv\le 0$. But $y\brac{\frac{1}{y}}=1>0$, so $\frac{1}{y}>0$. Likewise, $\frac{1}{x}>0$.

Multiplying both sides of the inequality $x<y$ by the positive quantity $\brac{\frac{1}{x}}\brac{\frac{1}{y}}$, we obtain $\frac{1}{y}<\frac{1}{x}$.
\end{enumerate}
\end{proof}
\pagebreak

\section{Real Numbers}
\subsection{Problems with $\QQ$}
$\QQ$ has some problems, the first of which being \emph{algebraic incompleteness}: there exists equations with coefficients in $\QQ$ but do not have solutions in $\QQ$ (in fact $\RR$ has this problem too, but $\CC$ is algebraically complete, by the fundamental theorem of algebra).

\begin{lemma}
$x^2-2=0$ has no solution in $\QQ$.
\end{lemma}

\begin{proof}
Suppose, for a contradiction, that $x^2-2=0$ has a solution $x=\frac{p}{q}$, $q\neq0$. We also assume $\frac{p}{q}$ is in lowest terms; that is, $p,q$ are coprime. Squaring both sides gives $\frac{p^2}{q^2}=2$, or $p^2=2q^2$. Observe that $p^2$ is even, so $p$ is even; let $p=2m$ for some integer $m$. Then this implies $4m^2=2q^2$, or $2m^2=q^2$. Similarly, $q^2$ is even so $q$ is even.

Since $p$ and $q$ share a common factor of $2$, we have reached a contradiction.
\end{proof}

The second problem is \emph{analytic incompleteness}: there exists a sequence of rational numbers that approach a point that is not in $\QQ$; for example, the sequence
\[1,1.4,1.41,1.414,1.4142,\dots\]
tends to the the irrational number $\sqrt{2}$.

Continuing from the above lemma,
\begin{lemma}
Let
\begin{align*}
A&=\{p\in\QQ\mid p>0,p^2<2\},\\
B&=\{p\in\QQ\mid p>0,p^2>2\}.
\end{align*}
Then $A$ contains no largest number, and $B$ contains no smallest number.
\end{lemma}

\begin{proof}
Prove by construction. We associate with each rational $p>0$ the number
\[q=p-\frac{p^2-2}{p+2}=\frac{2p+2}{p+2}\]
and so
\[q^2-2=\frac{2(p^2-2)}{(p+2)^2}.\]

For any $p\in A$, $q>p$ and $q\in A$ since $q^2<2$, so $A$ has no largest number.

For any $p\in B$, $q<p$ and $q\in B$ since $q^2>2$, so $B$ has no smallest number.
\end{proof}

\begin{proposition}
$\QQ$ does not have the least-upper-bound property.
\end{proposition}

\begin{proof}
In the previous result, note that $B$ is the set of all upper bounds of $A$, and $B$ does not have a smallest element. Hence $A\subset\QQ$ is bounded above but $A$ has no least upper bound in $\QQ$.
\end{proof}

\subsection{Real Field}
The sole objective of this subsection is to prove the following result.

\begin{theorem}[Existence of real field]\label{thrm:existence-real-field}
There exists an ordered field $\RR$ that
\begin{enumerate}[label=(\roman*)]
\item contains $\QQ$ as a subfield, and
\item has the least-upper-bound property (also known as the completeness axiom).
\end{enumerate}
\end{theorem}

We want to construct $\RR$ from $\QQ$; one method to do so is using Dedekind cuts.

\begin{definition}[Dedekind cut]
$\alpha\subset\QQ$ is a \vocab{Dedekind cut}\index{Dedekind cut}, if
\begin{enumerate}[label=(\roman*)]
\item $\alpha\neq\emptyset$, $\alpha\neq\QQ$;\hfill(non-trivial)
\item if $p\in\alpha$, $q\in\QQ$ and $q<p$, then $q\in\alpha$;
\item if $p\in\alpha$, then $p<r$ for some $r\in\alpha$.
\end{enumerate}
\end{definition}

\begin{remark}
Note that (iii) simply says that $\alpha$ has no largest member; (ii) implies two facts which will be used freely:
\begin{itemize}
\item If $p\in\alpha$ and $q\notin\alpha$, then $p<q$.
\item If $r\notin\alpha$ and $r<s$, then $s\notin\alpha$.
\end{itemize}
\end{remark}

\begin{example}
Let $r\in\QQ$ and define
\[ \alpha_r\colonequals\{p\in\QQ\mid p<r\}. \]
We now check that this is indeed a Dedekind cut.
\begin{enumerate}[label=(\roman*)]
\item $p=1+r\notin\alpha_r$ thus $\alpha_r\neq\QQ$. $p=r-1\in\alpha_r$ thus $\alpha_r\neq\emptyset$.

\item Suppose that $q\in\alpha_r$ and $q^\prime<q$. Then $q^\prime<q<r$ which implies that $q^\prime<r$ thus $q^\prime\in\alpha_r$.

\item Suppose that $q\in\alpha_r$. Consider $\dfrac{q+r}{2}\in\QQ$ and $q<\dfrac{q+r}{2}<r$. Thus $\dfrac{q+r}{2}\in\alpha_r$.
\end{enumerate}
\end{example}

This example shows that every rational $r$ corresponds to a Dedekind cut $\alpha_r$.

\begin{example}
$\sqrt[3]{2}$ is not rational, but it is real. $\sqrt[3]{2}$ corresponds to the cut
\[ \alpha=\{p\in\QQ\mid p^3<2\}. \]
\begin{enumerate}[label=(\roman*)]
\item Trivial.
\item If $q<p$, by the monotonicity of the cubic function, this implies that $q^3<p^3<2$ thus $q\in\alpha$.
\item If $p\in\alpha$, consider $\brac{p+\frac{1}{n}}^3<2$.
\end{enumerate}
\end{example}

\begin{definition}
The set of real numbers, denoted by $\RR$, is the set of all Dedekind cuts:
\[\RR\colonequals\{\alpha\subset\QQ\mid\alpha\text{ is a Dedekind cut}\}.\]
\end{definition}

\begin{proposition}
$\RR$ has an order, where $\alpha<\beta$ is defined to mean that $\alpha\subsetneq\beta$.
\end{proposition}

\begin{proof}
Simply check if this is a valid order (by checking for trichotomy and transitivity).
\end{proof}

\begin{proposition}
The ordered set $\RR$ has the least-upper-bound property.
\end{proposition}

\begin{proof}
Let non-empty $A\subset\RR$ be bounded above. Let $\beta\in\RR$ be an upper bound of $A$. We want to show that $A$ has a supremum in $\RR$.

Let
\[\gamma=\bigcup_{\alpha\in A}\alpha.\]
Then $p\in\gamma$ if and only if $p\in\alpha$ for some $\alpha\in A$.

\begin{claim}
$\gamma\in\RR$ and $\gamma=\sup A$.
\end{claim}

We first prove that $\gamma\in\RR$ by checking that it is a Dedekind cut:
\begin{enumerate}[label=(\roman*)]
\item Since $A\neq\emptyset$, there exists $\alpha_0\in A$. Since $\alpha_0\in\RR$, it is a Dedekind cut so $\alpha_0\neq\emptyset$. Since $\alpha_0\subset\gamma$, $\gamma\neq\emptyset$.

Since $\alpha\subset\beta$ for every $\alpha\in A$, the union of $\alpha\in A$ must be a subset of $\beta$; thus $\gamma\subset\beta$. Hence $\gamma\neq\QQ$.

\item Let $p\in\gamma$. Then $p\in\alpha_1$ for some $\alpha_1\in A$. If $q<p$, then $q\in\alpha_1$ (since $\alpha_1$ is a Dedekind cut). Hence $q\in\gamma$.

\item If $r\in\alpha_1$ is so chosen that $r>p$, we see that $r\in\gamma$ (since $\alpha_1\subset\gamma$).
\end{enumerate}

Next we prove that $\gamma=\sup A$, by checking that (i) $\gamma$ is an upper bound of $A$, (ii) $\gamma$ is the \emph{least} of the upper bounds.
\begin{enumerate}[label=(\roman*)]
\item It is clear that $\alpha\le\gamma$ for every $\alpha\in A$.
\item Suppose $\delta<\gamma$. Then there exists $s\in\gamma$ such that $s\notin\delta$. Since $s\in\gamma$, $s\in\alpha$ for some $\alpha\in A$. Hence $\delta<\alpha$, so $\delta$ is not an upper bound of $A$.
\end{enumerate}
\end{proof}

\begin{remark}
The l.u.b.\ property of $\RR$ is also known as the \emph{completeness axiom} of $\RR$.
\end{remark}

We now define operations on $\RR$.

\begin{definition}[Addition]
Given $\alpha,\beta\in\RR$, define addition as
\[\alpha+\beta\colonequals\{r\in\QQ\mid r=a+b,a\in\alpha,b\in\beta\}.\]
\end{definition}

We first check if the above definition makes sense. We want to show that addition on $\RR$ is closed: for all $\alpha,\beta\in\RR$, $\alpha+\beta\in\RR$.

\begin{proof}
We check that $\alpha+\beta$ is a Dedekind cut:
\begin{enumerate}[label=(\roman*)]
\item Since $\alpha\neq\emptyset$ and $\beta\neq\emptyset$, there exists $a\in\alpha$ and $b\in\beta$. Hence $r=a+b\in\alpha+\beta$ so $\alpha+\beta\neq\emptyset$.

Since $\alpha\neq\QQ$ and $\beta\neq\QQ$, there exist $c\neq\alpha$ and $d\neq\beta$. Thus $r^\prime=c+d>a+b$ for any $a\in\alpha,b\in\beta$, so $r^\prime\notin\alpha+\beta$. Hence $\alpha+\beta\neq\QQ$.

\item Suppose that $r\in\alpha+\beta$ and $r^\prime<r$. We want to show that $r^\prime\in\alpha+\beta$.

$r=a+b$ for some $a\in\alpha,b\in\beta$. Then $r^\prime-a<b$. Since $\beta\in\RR$, $r^\prime-a\in\beta$ so $r^\prime-a=b_1$ for some $b_1\in\beta$. Hence $r^\prime=a+b_1\in\alpha+\beta$.

\item Suppose $r\in\alpha+\beta$, so $r=a+b$ for some $a\in\alpha,b\in\beta$. Since $\alpha,\beta$ are Dedekind cuts, there exist $a^\prime\in\alpha,b^\prime\in\beta$ with $a<a^\prime$ and $b<b^\prime$. Then $r=a+b<a^\prime+b^\prime\in\alpha+\beta$. We define $r^\prime=a^\prime+b^\prime\in\alpha+\beta$ with $r<r^\prime$.
\end{enumerate}
\end{proof}

\begin{lemma} \
\begin{enumerate}[label=(\roman*)]
\item Addition on $\RR$ is commutative: $\alpha+\beta=\beta+\alpha$ for all $\alpha,\beta\in\RR$.
\item Addition on $\RR$ is associative: $\alpha+(\beta+\gamma)=(\alpha+\beta)+\gamma$ for all $\alpha,\beta,\gamma\in\RR$.
\item Additive identity: Define $0^*\colonequals\{p\in\QQ\mid p<0\}$. Then $\alpha+0^*=\alpha$ for all $\alpha\in\RR$.
\item Additive inverse: Fix $\alpha\in\RR$, define $\beta=\{p\in\QQ\mid\exists r>0,\:-p-r\notin\alpha\}$. Then $\alpha+\beta=0^*$.
\end{enumerate}
\end{lemma}

\begin{remark}
Recall that to prove that two sets are equal, show double inclusion.
\end{remark}

\begin{proof} \
\begin{enumerate}[label=(\roman*)]
\item We need to show that $\alpha+\beta\subset\beta+\alpha$ and $\beta+\alpha\subset\alpha+\beta$.

Let $r\in\alpha+\beta$. Then $r=a+b$ for $a\in\alpha$ and $b\in\beta$. Thus $r=b+a$ since $+$ is commutative on $\QQ$. Hence $r\in\beta+\alpha$. Therefore $\alpha+\beta\subset\beta+\alpha$.

Similarly, $\beta+\alpha\subset\alpha+\beta$.

Therefore $\alpha+\beta=\beta+\alpha$.

\item Let $r\in\alpha+(\beta+\gamma)$. Then $r=a+(b+c)$ where $a\in\alpha,b\in\beta,c\in\gamma$. Thus $r=(a+b)+c$ by associativity of $+$ on $\QQ$. Therefore $r\in(\alpha+\beta)+\gamma$, hence $\alpha+(\beta+\gamma)\subset(\alpha+\beta)+\gamma$.

Similarly, $(\alpha+\beta)+\gamma\subset\alpha+(\beta+\gamma)$.

\item It is clear that $0^*$ is a Dedekind cut.

Let $r\in\alpha+0^*$. Then $r=a+p$ for some $a\in\alpha,p\in0^*$. Thus $r=a+p<a+0=a$ so $r\in\alpha$. Hence $\alpha+0^*\subset\alpha$.

Let $r\in\alpha$. Then there exists $r^\prime\in\alpha$ where $r^\prime>r$. Thus $r-r^\prime<0$, so $r-r^\prime\in0^*$. We see that $r=r^\prime+(r-r^\prime)$ where $r^\prime\in\alpha$, $r-r^\prime\in0^*$. Hence $\alpha\subset\alpha+0^*$.

\item Fix some $\alpha\in\RR$. We first show that $\beta$ is a Dedekind cut.
\begin{enumerate}[label=(\roman*)]
\item Let $s\notin\alpha$, let $p=-s-1$. Then $-p-1\notin\alpha$. Hence $p\in\beta$, so $\beta\neq\emptyset$.

Let $q\in\alpha$. Then $-q\notin\beta$ so $\beta\neq\QQ$.

\item Let $p\in\beta$. Then there exists $r>0$ such that $-p-r\notin\alpha$. If $q<p$, then $-q-r>-p-r$ so $-q-r\notin\alpha$. Hence $q\in\beta$.
\item Let $t=p+\frac{r}{2}$. Then $t>p$, and $-t-\frac{r}{2}=-p-r\notin\alpha$. Hence $t\in\beta$.
\end{enumerate}

Let $r\in\alpha$, $s\in\beta$. Then $-s\notin\alpha$. This implies $r<-s$ (since $\alpha$ is closed downwards) so $r+s<0$. Hence $\alpha+\beta\subset0^*$.

To prove the opposite inclusion, let $v\in0^*$, and let $w=-\frac{v}{2}$. Then $w>0$. By the Archimedean property on $\QQ$, there exists $n\in\NN$ such that $nw\in\alpha$ but $(n+1)w\notin\alpha$. Let $p=-(n+2)w$. Then
\[-p-w=(n+2)w-w=(n+1)w\notin\alpha\]
so $p\in\beta$. Since $v=nw+p$ where $nw\in\alpha$, $p\in\beta$, $v\in\alpha+\beta$. Hence $0^*\subset\alpha+\beta$.
\end{enumerate}
\end{proof}

\begin{notation}
$\beta$ is denoted by the more familiar notation $-\alpha$.
\end{notation}

\begin{lemma}
If $\alpha,\beta,\gamma\in\RR$ and $\beta<\gamma$, then $\alpha+\beta<\alpha+\gamma$.
\end{lemma}

\begin{proof}

\end{proof}

We say that a Dedekind cut $\alpha$ is \emph{positive} if $0\in\alpha$, and \emph{negative} if $0\notin\alpha$. If $\alpha$ is neither positive nor negative, then $\alpha=0^*$.

Multiplication is a little more bothersome than addition in the present context, since products of negative rationals are positive. For this reason we confine ourselves first to $\RR^+$ (the set of all $\alpha\in\RR$ with $\alpha>0^*$).

\begin{definition}
Given $\alpha,\beta\in\RR^+$, define multiplication as
\[\alpha\beta\colonequals\{p\in\QQ\mid p\le rs,\:r\in\alpha,s\in\beta,\:r,s>0\}.\]
\end{definition}

We also define $1^*\colonequals\{q\in\QQ\mid q<1\}$.

As again, check if the above definition makes sense. We want to show that multiplication on $\RR^+$ is closed: for all $\alpha,\beta\in\RR$, $\alpha\beta\in\RR$.

\begin{proof}
Check that $\alpha\beta$ is a Dedekind cut.
\begin{enumerate}[label=(\roman*)]
\item $\alpha\neq\emptyset$ means there exists $r\in\alpha,r>0$. Similarly, $\beta\neq\emptyset$ means there exists $s\in\beta,s>0$. Then $rs\in\QQ$ and $rs\le rs$, so $rs\in\alpha\beta$. Hence $\alpha\beta\neq\emptyset$.

$\alpha\neq\QQ$ means there exists $r^\prime\notin\alpha$ such that $r^\prime>r$ for all $r\in\alpha$. Similarly $\beta\neq\QQ$ means there exists $s^\prime\in\beta$ such that $s^\prime>s$ for all $s\in\beta$. Then $r^\prime s^\prime>rs$ for all $r\in\alpha,s\in\beta$, so $r^\prime s^\prime\notin\alpha\beta$. Hence $\alpha\beta\neq\QQ$.

\item Let $p\in\alpha\beta$. Then $p\le ab$ for some $a\in\alpha,b\in\beta$, $a,b>0$.

If $q<p$, then $q<p\le ab$ so $q\in\alpha\beta$.

\item Let $p\in\alpha\beta$. Then $p\le ab$ for some $a\in\alpha,b\in\beta$, $a,b>0$. Pick $a^\prime\in\alpha$ and $b^\prime\in\beta$ with $a^\prime>a$ and $b^\prime>b$. Form $a^\prime b^\prime>ab\ge p$, $a^\prime b^\prime\le a^\prime b^\prime$ means $a^\prime b^\prime\in\alpha\cdot\beta$.
\end{enumerate}
\end{proof}

We now complete the definition of multiplication by setting $\alpha0^*=0^*=0^*\alpha$, and by setting
\[\alpha\beta=\begin{cases}
(-\alpha)(-\beta)&a<0^*,\beta<0^*,\\
-[(-\alpha)\beta]&a<0^*,\beta>0^*,\\
-[\alpha(-\beta)]&\alpha>0^*,\beta<0^*.
\end{cases}\]
where we make negative numbers positive, multiply, and then negate them as needed.

\begin{lemma} \
\begin{enumerate}[label=(\roman*)]
\item Multiplication on $\RR$ is commutative: $\alpha\beta=\beta\alpha$ for all $\alpha,\beta\in\RR$.
\item Multiplication on $\RR$ is associative: $(\alpha\beta)\gamma=\alpha(\beta\gamma)$ for all $\alpha,\beta,\gamma\in\RR$.
\item Multiplicative identity: $1\alpha=\alpha$ for all $\alpha\in\RR$.
\item Multiplicative inverse: If $\alpha\in\RR$, $\alpha\neq0^*$, then there exists $\beta\in\RR$ such that $\alpha\beta=1^*$.
\end{enumerate}
\end{lemma}

We associate each $r\in\QQ$ with the set
\[r^*=\{p\in\QQ\mid p<r\}.\]
It is obvious that each $r^*$ is a cut; that is, $r^*\in\RR$.

\begin{proposition}
The replacement of $r\in\QQ$ by the corresponding ``rational cuts'' $r^*\in\RR$ preserves sums, products, and order. That is, for all $r^*,s^*\in\RR$,
\begin{enumerate}[label=(\roman*)]
\item $r^*+s^*=(r+s)^*$;
\item $r^*s^*=(rs)^*$;
\item $r^*<s^*$ if and only if $r<s$.
\end{enumerate}
\end{proposition}

\begin{proof} \
\begin{enumerate}[label=(\roman*)]
\item Let $p\in r^*+s^*$. Then $p=u+v$ for some $u\in r^*$, $v\in s^*$, where $u<r$, $v<s$. Then $p<r+s$. Hence $p\in(r+s)^*$, so $r^*+s^*\subset(r+s)^*$.

Let $p\in(r+s)^*$. Then $p<r+s$. Let $t=\frac{(r+s)-p}{2}$, and let
\[r^\prime=r-t,\quad s^\prime=s-t.\]
Since $t>0$, $r^\prime<r$ so $r^\prime\in r^*$; $s^\prime<s$ so $s^\prime\in s^*$. Then $p=r^\prime+s^\prime$, so $p\in r^*+s^*$. Hence $(r+s)^*\subset r^*+s^*$.

\item 
\item Suppose $r<s$. Then $r\in s^*$, but $r\notin r^*$. Hence $r^*<s^*$.

Conversely, suppose $r^*<s^*$. Then there exists $p\in s^*$ such that $p\in r^*$. Hence $r\le p<s$, so $r<s$.
\end{enumerate}
\end{proof}

This shows that the ordered field $\QQ$ is isomorphic to the ordered field $\QQ^*=\{q^*\mid q\in\QQ\}$ whose elements are rational cuts. It is this identification of $\QQ$ with $\QQ^*$ which allows us to regard $\QQ$ as a subfield of $\RR$.

\begin{remark}
In fact, $\RR$ is the only ordered field with the l.u.b.\ property. Hence any other ordered field with the l.u.b.\ property is isomorphic to $\RR$.
\end{remark}

Therefore we have proven \ref{thrm:existence-real-field}.
\pagebreak

\subsection{Properties of $\RR$}
\begin{proposition}[Archimedean property]\label{prop:R-archimedean}
For any $x\in\RR^+$, $y\in\RR$, there exists $n\in\NN$ such that
\[nx>y.\]
\end{proposition}

\begin{proof}
Suppose, for a contradiction, that $nx\le y$ for all $n\in\NN$. Then $y$ is an upper bound of the set
\[A=\{nx\mid n\in\NN\}.\]
Since $A\subset R$ is non-empty and bounded above, by the l.u.b.\ property of $\RR$, $A$ has a supremum in $\RR$, say $\alpha=\sup A$.

Consider $\alpha-x$. Since $\alpha-x<\alpha=\sup A$, $\alpha-x$ is not an upper bound of $A$. Then $\alpha-x\le n_0x$ for some $n_0\in\NN$; rearranging gives $\alpha\le(n_0+1)x$. This implies that $\alpha$ is not an upper bound of $A$, which contradicts the fact that $\alpha$ is the supremum of $A$.
\end{proof}

\begin{corollary}\label{cor:R-archimedean-1/n}
Let $\epsilon>0$. Then there exists $n\in\NN$ such that $0<\frac{1}{n}<\epsilon$.
\end{corollary}

\begin{proof}
In \ref{prop:R-archimedean}, take $x=\epsilon$ and $y=1$.
\end{proof}

\begin{proposition}[$\QQ$ is dense in $\RR$]
For any $x,y\in\RR$ with $x<y$, there exists $p\in\QQ$ such that 
\[x<p<y.\]
\end{proposition}

\begin{proof}
We prove by construction (construct the required $p$ from the given $x$ and $y$).

Since $x<y$, we have $y-x>0$. By \ref{cor:R-archimedean-1/n}, there exists $n\in\NN$ such that
\[\frac{1}{n}<y-x.\]
Consider the set comprising multiples of $\frac{1}{n}$:
\[E=\crbrac{\frac{k}{n}\:\bigg|\:k\in\NN}.\]
Since $E$ is unbounded, choose the first multiple $m\in\NN$ such that $\frac{m}{n}>x$.

\begin{claim}
$x<\frac{m}{n}<y.$
\end{claim}
It suffices to show that $\frac{m}{n}<y$. If not, then
\[\frac{m-1}{n}<x\quad\text{and}\quad\frac{m}{n}>y,\]
where the first inequality follows from the minimality of $m$. But these two statements combined imply that $\frac{1}{n}>y-x$, a contradiction.
\end{proof}

\begin{proposition}[$\RR$ is closed under taking roots]
For every $x\in\RR^+$ and every $n\in\NN$, there exists a unique $y\in\RR^+$ so that $y^n=x$.
\end{proposition}

We call the number $y$ the positive $n$-th root of $x$, and denote it by $\sqrt[n]{x}$ or $x^\frac{1}{n}$.

\begin{proof}
Let $x\in\RR^+$, fix $n\in\NN$.

\fbox{Existence} Let
\[E=\{t\in\RR^+\mid t^n<x\}.\]

\begin{claim}
$y=\sup E$ satisfies $y^n=x$.
\end{claim}
We first show that $E$ has a supremum.
\begin{enumerate}[label=(\roman*)]
\item Let $t=\dfrac{x}{1+x}$. Then $0\le t<1$, so $t^n\le t<x$ implies $t^n<x$. Hence $t\in E$, so $E\neq\emptyset$.
\item We claim that $1+x$ is an upper bound for $E$.

If $t>1+x$, then $t^n\ge t>x$ implies $t^n>x$, so $t\notin E$. [This is the contrapositive of $t\in E\implies t\le1+x$.] Hence $1+x$ is an upper bound of $E$, so $E$ is bounded above.
\end{enumerate}
Hence $E$ has a supremum; let $y=\sup E$.

To prove that $y^n=x$, we show that both the inequalities $y^n<x$ and $y^n>x$ lead to a contradiction. Consider the identity $b^n-a^n=(b-a)\brac{b^{n-1}+b^{n-2}a+\cdots+a^{n-1}}$. If $0<a<b$, then
\begin{equation*}\tag{1}
b^n-a^n<(b-a)nb^{n-1}.
\end{equation*}
\begin{description}
\item[Case 1: $y^n<x$.] 
\begin{idea}
We can find a \emph{small} $h>0$ such that $(y+h)^n<x$.
\end{idea}
Choose $h$ so that $0<h<1$ and
\[h<\frac{x-y^n}{n(y+1)^{n-1}}.\]
In (1), take $b=y+h$, $a=y$. Then
\begin{align*}
(y+h)^n-y^n
&<hn(y+h)^{n-1}\\
&<hn(y+1)^{n-1}\\
&<\frac{x-y^n}{\cancel{n(y+1)^{n-1}}}\cancel{n(y+1)^{n-1}}\\
&=x-y^n.
\end{align*}
Thus $(y+h)^n<x$, and $y+h\in E$. Since $y+h>y$, this contradicts the fact that $y$ is an upper bound of $E$.

\item[Case 2: $y^n>x$.] 
\begin{idea}
Similarly, we can find a \emph{small} $k>0$ such that $(y-k)^n>x$.
\end{idea}
Let
\[k=\frac{y^n-x}{ny^{n-1}}.\]
Then $0<k<y$, by (1). If $t\ge y-k$,
\begin{align*}
y^n-t^n&\le y^n-(y-k)^n\\
&<kny^{n-1}\\
&=\frac{y^n-x}{\cancel{ny^{n-1}}}\cancel{ny^{n-1}}\\
&=y^n-x.
\end{align*}
Thus $t^n>x$, and $t\notin E$. It follows that $y-k$ is an upper bound of $E$. But $y-k<y$, which contradicts the fact that $y$ is the \emph{least} upper bound of $E$.
\end{description}

\fbox{Uniqueness} Suppose, for a contradiction, that there exist distinct $y_1,y_2$ which are both $n$-th roots of $x$. WLOG assume that $0<y_1<y_2$. Then taking the $n$-th power gives ${y_1}^n<{y_2}^n$. 

Since $y_1$ is a $n$-th root of $x$, then $x={y_1}^n$, so $x<{y_2}^n$ implies $x\neq {y_2}^n$. Hence $y_2$ cannot be a $n$-th root of $x$, a contradiction.
\end{proof}

\begin{corollary}
If $a,b\in\RR^+$ and $n\in\NN$, then
\[(ab)^\frac{1}{n}=a^\frac{1}{n}b^\frac{1}{n}.\]
\end{corollary}

\begin{proof}
Let $\alpha=a^\frac{1}{n}$, $\beta=b^\frac{1}{n}$. Then
\[ab=\alpha^n\beta^n=(\alpha\beta)^n\]
since multiplication is commutative. The uniqueness assertion of the previous result shows that
\[(ab)^\frac{1}{n}=\alpha\beta=a^\frac{1}{n}b^\frac{1}{n}.\]
\end{proof}

\begin{lemma}
If $x\in\RR^+$ and $m,n\in\NN$, then
\[(x^\frac{1}{n})^m=\brac{x^m}^\frac{1}{n}.\]
\end{lemma}

\begin{proof}
Exercise.
\end{proof}

We can now define \emph{rational exponents} $x^r$, where $x>0$ and $r\in\QQ$.
\begin{definition}[Rational exponents]
For $x>0$ and $m,n\in\NN$, define
\[x^\frac{m}{n}\colonequals\brac{x^\frac{1}{n}}^m\quad\text{and}\quad x^{-\frac{m}{n}}\colonequals\frac{1}{x^\frac{m}{n}}.\]
(We also define $x^0=1$.)
\end{definition}

We need to check that the above definition of $x^r$ is well defined. That is, if $m,n,p,q\in\NN$ are such that $\frac{m}{n}=\frac{p}{q}$, then $(x^\frac{1}{n})^m=(x^\frac{1}{q})^p$. 
To see this, note that $mq=np$ and
\[\brac{(x^\frac{1}{n})^m}^q=(x^\frac{1}{n})^{mq}=(x^\frac{1}{n})^{np}=x^p.\]
Thus $(x^\frac{1}{n})^m$ is the $q$-th root of $x^p$, i.e.,
\[(x^\frac{1}{n})^m=(x^p)^\frac{1}{q}.\]

\begin{lemma}[Properties of rational exponents] \
\begin{enumerate}[label=(\roman*)]
\item If $a>0$ and $r,s\in\QQ$, then $a^{r+s}=a^r a^s$ and $(a^r)^s=a^{rs}$.
\item If $0<a<b$ and $r\in\QQ$ with $r>0$, then $a^r<b^r$.
\item If $a>1$, $r,s\in\QQ$ with $r<s$, then $a^r<a^s$.
\end{enumerate}
\end{lemma}

The next result shows that real numbers can be approximated to any desired degree of accuracy by rational numbers with finite decimal representations.

\begin{proposition}
Let $x\ge0$. Then for every integer $n\ge1$ there exists a finite decimal $r_n=a_0.a_1a_2\cdots a_n$ such that
\[r_n\le x<r_n+\frac{1}{10^n}.\]
\end{proposition}

\begin{proof}
We prove by construction (construct the required finite decimal from $x$).

Let
\[S=\{k\in\ZZ\mid k\le x\}.\]
$S$ is non-empty (since $0\in S$), and $S$ is bounded above by $x$. Hence by the lub property of $\RR$, $S$ has a supremum in $\RR$, say $a_0=\sup S$. It is easily verified that $a_0\in S$, so $a_0$ is a non-negative integer. We call $a_0$ the \emph{greatest integer} in $x$, and write $a_0=\floor{x}$. Clearly we have
\[a_0\le x<a_0+1.\]
Now let $a_1=\floor{10(x-a_0)}$. Since $0\le 10(x-a_0)<10$, we have $0\le a_1\le 9$ and
\[a_1\le 10x-10a_0<a_1+1.\]
In other words, $a_1$ is the largest integer satisfying the inequalities
\[a_0+\frac{a_1}{10}\le x<a_0+\frac{a_1+1}{10}.\]
More generally, having chosen $a_1,\dots,a_{n-1}$ with $0\le a_i\le 9$, let $a_n$ be the largest integer satisfying the inequalities
\[a_0+\frac{a_1}{10}+\cdots+\frac{a_n}{10^n}\le a_0+\frac{a_1}{10}+\cdots+\frac{a_n+1}{10^n}.\]
Then $0\le a_n\le 9$ and we have
\[r_n\le x<r_n+\frac{1}{10^n},\]
where $r_n=a_0.a_1a_2\cdots a_n$.
\end{proof}

Furthermore, it is easy to verify that $\displaystyle x=\sup_{n\in\NN}r_n$.
\pagebreak

\subsection{Extended Real Number System}
\begin{definition}[Extended real number system]
The \vocab{extended real number system}\index{extended real number system} is defined to be the union
\[\overline{\RR}\colonequals\RR\cup\{-\infty,+\infty\},\]
where we preserve the original order in $\RR$, and define $-\infty<x<+\infty$ for all $x\in\RR$.
\end{definition}

Defining $\overline{\RR}$ is convenient since the following result holds.

\begin{proposition}
Any non-empty $E\subset\overline{\RR}$ has a supremum and infimum in $\overline{\RR}$.
\end{proposition}

\begin{proof}
If $E$ is bounded above in $\RR$, then by the l.u.b.\ property of $\RR$, it has a supremum in $\RR\subset\overline{\RR}$. If $E$ is not bounded above in $\RR$, then $\sup E=+\infty\in\overline{\RR}$.

Exactly the same remarks apply to lower bounds.
\end{proof}

$\overline{\RR}$ does not form a field, but it is customary to make the following conventions for arithmetic on $\overline{\RR}$:
\begin{enumerate}[label=(\roman*)]
\item If $x\in\RR$ then
\[ x+\infty=+\infty, \quad x-\infty=-\infty, \quad \frac{x}{+\infty}=\frac{x}{-\infty}=0. \]
\item If $x>0$ then
\[x\cdot(+\infty)=+\infty,\quad x\cdot(-\infty)=-\infty.\]
If $x<0$ then
\[x\cdot(+\infty)=-\infty,\quad x\cdot(-\infty)=+\infty.\]
\end{enumerate}
When it is desired to make the distinction between real numbers on the one hand and the symbols $+\infty$ and $-\infty$ on the other quite explicit, the former are called \emph{finite}.
\pagebreak
\section{Complex Field}
\begin{lemma}
Let $(a,b),(c,d)\in\RR^2$. Define addition and multiplication on $\RR^2$ as
\begin{align*}
(a,b)+(c,d)&=(a+c,b+d),\\
(a,b)(c,d)&=(ac-bd,ad+bc).
\end{align*}
Then $\RR^2$ is a field, with additive identity $(0,0)$ and multiplicative identity $(1,0)$.
\end{lemma}

We call this structure $\CC$, the \vocab{complex field}; its elements are called \emph{complex numbers}.

\begin{proof}
Check the field axioms.
\end{proof}

The next result shows that the complex numbers of the form $(a,0)$ have the same arithmetic properties as the corresponding real numbers $a$. We can therefore identify $(a,0)\in\CC$ with $a\in\RR$. This identification implies that $\RR$ is a subfield of $\CC$.

\begin{lemma}
For any $a,b\in\RR$, we have
\begin{align*}
(a,0)+(b,0)&=(a+b,0),\\
(a,0)(b,0)&=(ab,0).
\end{align*}
\end{lemma}

\begin{proof}
Exercise.
\end{proof}

You may have noticed that we have defined the complex numbers without referring to the mysterious square root of $-1$. We now show that the notation $(a,b)$ is equivalent to the more customary $a+bi$.

Define the imaginary number $i\colonequals(0,1)$. See that 
\[i^2=(0,1)(0,1)=(-1,0)=-1.\]

\begin{lemma}
For $a,b\in\RR$, $(a,b)=a+bi$.
\end{lemma}

\begin{proof}
\begin{align*}
a+bi&=(a,0)+(b,0)(0,1)\\
&=(a,0)+(0,b)\\
&=(a,b).
\end{align*}
\end{proof}

For $a,b\in\RR$, we write $z=a+bi$; we call $a$ and $b$ the \emph{real part} and \emph{imaginary part} of $z$ respectively, denoted by $a=\Re(z)$, $b=\Im(z)$; $\overline{z}=a-bi$ is called the \emph{conjugate} of $z$.

\begin{lemma}[Properties of conjugate]
For $z,w\in\CC$,
\begin{enumerate}[label=(\roman*)]
\item $\overline{z+w}=\overline{z}+\overline{w}$
\item $\overline{zw}=\overline{z}\:\overline{w}$
\item $z+\overline{z}=2\Re(z)$, $z-\overline{z}=2i\Im(z)$
\item $z\overline{z}\in\RR$ and $z\overline{z}\ge0$
\end{enumerate}
\end{lemma}

For $z\in\CC$, tbe \emph{absolute value} of $z$ is defined as
\[|z|\colonequals\brac{z\overline{z}}^\frac{1}{2}.\]

\begin{lemma}[Properties of absolute value]
For $z,w\in\CC$,
\begin{enumerate}[label=(\roman*)]
\item $|z|\ge0$
\item $|\overline{z}|=|z|$
\item $|zw|=|z||w|$
\item $|\Re(z)|\le|z|$
\end{enumerate}
\end{lemma}

\begin{proof} \
\begin{enumerate}[label=(\roman*)]
\item The square root is non-negative, by definition.
\item The conjugate of $\overline{z}$ is $z$, and the rest follows by the definition of absolute value.
\item Let $z=a+bi$, $w=c+di$ where $a,b,c,d\in\RR$. Then
\begin{align*}
|zw|^2&=(ac-bd)^2+(ad-bc)^2\\
&=(a^2+b^2)(c^2+d^2)\\
&=|z|^2|w|^2=\brac{|z||w|}^2
\end{align*}
and the desired result follows by taking square roots on both sides.
\item Let $z=a+bi$. Note that $a^2\le a^2+b^2$, hence
\[|\Re(z)|=|a|=\sqrt{a^2}\le\sqrt{a^2+b^2}=|z|.\]
\end{enumerate}
\end{proof}

\begin{proposition}[Triangle inequality]
For $z,w\in\CC$,
\begin{equation}
|z+w|\le|z|+|w|.
\end{equation}
\end{proposition}

\begin{proof}
Let $z,w\in\CC$. Note that the conjugate of $z\overline{w}$ is $\overline{z}w$, so $z\overline{w}+\overline{z}w=2\Re(z\overline{w})$. Hence
\begin{align*}
|z+w|^2&=(z+w)(\overline{z+w})=(z+w)(\overline{z}+\overline{w})\\
&=z\overline{z}+z\overline{w}+\overline{z}w+w\overline{w}\\
&=|z|^2+2\Re(z\overline{w})+|w|^2\\
&\le|z|^2+2|z\overline{w}|+|w|^2\\
&=|z|^2+2|z||w|+|w|^2\\
&=\brac{|z|+|w|}^2
\end{align*}
and taking square roots yields the desired result.
\end{proof}

\begin{corollary}[Generalised triangle inequality]
For $z_1,\dots,z_n\in\CC$,
\[|z_1+\cdots+z_n|\le|z_1|+\cdots+|z_n|.\]
\end{corollary}

\begin{proof}
We have proven the case $n=2$. Assume the statement holds for $n-1$. Then
\[|z_1+\cdots+z_{n-1}+z_n|\le|z_1+\cdots+z_{n-1}|+|z_n|\le|z_1|+\cdots+|z_n|,\]
which establishes the claim by induction.
\end{proof}

\begin{corollary}
For $x,y,z\in\CC$,
\begin{enumerate}[label=(\roman*)]
\item $\absolute{|x|-|y|}\le|x-y|$;
\item $|x-y|\le|x-z|+|z-y|$.
\end{enumerate}
\end{corollary}

\begin{proof} \
\begin{enumerate}[label=(\roman*)]
\item By the triangle inequality,
\[|x|=|(x-y)+y|\le|x-y|+|y|\]
so that
\[|x|-|y|\le|x-y|.\]
Interchanging the roles of $x$ and $y$ in the above gives
\[|y|-|x|\le|x-y|.\]
Hence
\[\absolute{|x|-|y|}\le|x-y|.\]
\item In the triangle inequality, replace $x$ by $x-y$ and $y$ by $y-z$.
\end{enumerate}
\end{proof}

\begin{proposition}[Cauchy--Schwarz inequality]
If $a_1,\dots,a_n,b_1,\dots,b_n\in\CC$, then
\begin{equation}\label{eqn:cauchy-schwarz-inequality}
\absolute{\sum_{i=1}^{n}a_i\overline{b_i}}^2\le\sum_{i=1}^{n}|a_i|^2\sum_{i=1}^{n}|b_i|^2.
\end{equation}
\end{proposition}

\begin{proof}
For simplicity, we shall drop the upper and lower limits of the sums. 
Let
\[A=\sum|a_i|^2,\quad B=\sum|b_i|^2,\quad C=\sum a_i\overline{b_i}.\]
Then \eqref{eqn:cauchy-schwarz-inequality} becomes
\[|C|^2\le AB.\]
If $B=0$, then $b_1=\cdots=b_n=0$, and the conclusion is trivial. Now assume that $B>0$. Then consider the sum
\begin{align*}
\sum |Ba_i-Cb_i|^2
&=\sum (Ba_i-Cb_i)(\overline{Ba_i-Cb_i})\\
&=\sum (Ba_i-Cb_i)(B\overline{a_i}-\overline{Cb_i})\\
&=B^2\sum |a_i|^2-B\overline{C}\sum a_i\overline{b_i}-BC\sum \overline{a_i}b_i+|C|^2\sum |b_i|^2\\
&=B^2A-B|C|^2\\
&=B(AB-|C|^2).
\end{align*}
Each term in $\displaystyle\sum |Ba_i-Cb_i|^2$ is non-negative, so $\displaystyle\sum |Ba_i-Cb_i|^2\ge0$. Thus
\[B(AB-|C|^2)\ge0.\]
Since $B>0$, it follows that $AB-|C|^2\ge0$, or $|C|^2\le AB$. This is the desired inequality.

(when does equality hold?)
\end{proof}

\begin{mdframed}
Define
\[\CC^n=\{(z_1,\dots,z_n)\mid z_i\in\CC\}.\]
We can define an inner product on $\CC^n$: for $\vb{a},\vb{b}\in\CC^n$,
\[\langle\vb{a},\vb{b}\rangle=\sum_{i=1}^{n}a_i\overline{b_i}.\]
We can also define the norm of $\vb{a}\in\CC^n$:
\[|\vb{a}|=\langle\vb{a},\vb{a}\rangle^\frac{1}{2}.\]
\end{mdframed}
\pagebreak

\section{Euclidean Space}
For $n\in\NN$, define
\[\RR^n\colonequals\{(x_1,\dots,x_n)\mid x_i\in\RR\}\]
where $\vb{x}=(x_1,\dots,x_n)$, $x_i$'s are called the coordinates of $\vb{x}$. The elements of $\RR^n$ are called \emph{points}, or \emph{vectors}.

\begin{lemma}
Let $\vb{x}=(x_1,\dots,x_n)$, $\vb{y}=(y_1,\dots,y_n)$. $\RR^n$, with addition and scalar multiplication defined as
\begin{align*}
\vb{x}+\vb{y}&=(x_1+y_1,\dots,x_n+y_n),\\
\alpha\vb{x}&=(\alpha x_1,\dots,\alpha x_n).
\end{align*}
for $\vb{x},\vb{y}\in\RR^n$, $\alpha\in\RR$, is a vector space over $\RR$. Note that the zero element of $\RR^n$ is $\vb{0}=(0,\dots,0)$.
\end{lemma}

\begin{proof}
These two operations satisfy the commutative, associatives, and distributive laws (the proof is trivial, in view of the analagous laws for the real numbers).
\end{proof}

We define the \emph{inner product} of $\vb{x}$ and $\vb{y}$ by
\[\vb{x}\cdot\vb{y}\colonequals\sum_{i=1}^nx_iy_i,\]
and the \emph{norm} of $\vb{x}$ by
\[\norm{\vb{x}}\colonequals\sqrt{\vb{x}\cdot\vb{x}}.\]
The structure now defined (the vector space $\RR^n$ with the above inner product and norm) is called the \vocab{Euclidean $n$-space}.

\begin{lemma}[Basic properties of norm]
Suppose $\vb{x},\vb{y},\vb{z}\in\RR^n$, $\alpha\in\RR$.
\begin{enumerate}[label=(\roman*)]
\item $\norm{\vb{x}}\ge0$, where equality holds if and only if $\vb{x}=\vb{0}$\hfill(positive definiteness)
\item $\norm{\alpha\vb{x}}=|\alpha|\norm{\vb{x}}$\hfill(homogeneity)
\item $\norm{\vb{x}\cdot\vb{y}}\le\norm{\vb{x}}\norm{\vb{y}}$\hfill(Cauchy--Schwarz inequality)
\item $\norm{\vb{x}+\vb{y}}\le\norm{\vb{x}}+\norm{\vb{y}}$\hfill(triangle inequality)
\item $\norm{\vb{x}-\vb{z}}\le\norm{\vb{x}-\vb{y}}+\norm{\vb{y}-\vb{z}}$\hfill(triangle inequality)
\end{enumerate}
\end{lemma}

\begin{proof} \
\begin{enumerate}[label=(\roman*)]
\item Obvious from definition.
\item Obvious from definition.
\item We want to show 
\[\sqrt{\sum_{i=1}^{n}x_iy_i}\le\sqrt{\sum_{i=1}^{n}{x_i}^2}\sqrt{\sum_{i=1}^{n}{y_i}^2},\]
or, squaring both sides,
\[\sum_{i=1}^{n}x_iy_i\le\brac{\sum_{i=1}^{n}{x_i}^2}\brac{\sum_{i=1}^{n}{y_i}^2}.\]
But this is simply the Cauchy--Schwarz inequality \eqref{eqn:cauchy-schwarz-inequality}. 
\item By (iii) we have
\begin{align*}
\norm{\vb{x}+\vb{y}}&=(\vb{x}+\vb{y})\cdot(\vb{x}+\vb{y})\\
&=\vb{x}\cdot\vb{x}+2\vb{x}\cdot\vb{y}+\vb{y}\cdot\vb{y}\\
&\le\norm{\vb{x}}^2+2\norm{\vb{x}}\norm{\vb{y}}+\norm{\vb{y}}^2\\
&=\brac{\norm{\vb{x}}+\norm{\vb{y}}}^2.
\end{align*}
\item This follows directly from (iv) by replacing $\vb{x}$ by $\vb{x}-\vb{y}$, and $\vb{y}$ by $\vb{y}-\vb{z}$.
\end{enumerate}
\end{proof}
\pagebreak

\section*{Exercises}
\begin{exercise}[\cite{rudin} 1.1]
If $r\in\QQ\setminus\{0\}$ and $x\in\RR\setminus\QQ$, prove that $r+x\in\RR\setminus\QQ$ and $rx\in\RR\setminus\QQ$.
\end{exercise}

\begin{solution}
Prove by contradiction. If $r$ and $r+x$ were both rational, then $x=(r+x)-r$ would also be rational. Similarly if $rx$ were rational, then $x=\frac{rx}{r}$ would also be rational.
\end{solution}

\begin{exercise}[\cite{rudin} 1.2]
Prove that there is no rational number whose square is $12$.
\end{exercise}

\begin{solution}
Prove by contradiction.
\end{solution}

\begin{exercise}[\cite{rudin} 1.4]
Let $E$ be a nonempty subset of an ordered set; suppose $\alpha$ is a lower bound of $E$, and $\beta$ is an upper bound of $E$. Prove that $\alpha\le\beta$.
\end{exercise}

\begin{solution}
Since $E$ is non-empty, there exists $x\in E$. By definition of lower and upper bounds, we have $\alpha\le x\le\beta$.
\end{solution}

\begin{exercise}[\cite{rudin} 1.8]
Prove that no order can be defined in $\CC$ that turns it into an ordered field. \emph{Hint}: $-1$ is a square.
\end{exercise}

\begin{solution}
By \ref{lemma:ordered-field-properties}, an order $<$ that makes $\CC$ an ordered field would have to satisfy $-1=i^2>0$, contradicting $1>0$.
\end{solution}

\begin{exercise}[\cite{rudin} 1.9, lexicographic order]
Suppose $z=a+bi$, $w=c+di$. Define an order on $\CC$ as follows:
\[z<w\iff\begin{cases}
a<c,\text{ or}\\
a=c,b<d.
\end{cases}\]
Prove that this turns $\CC$ into an ordered set. Does this ordered set have the least upper bound property?
\end{exercise}

\begin{solution}
We show that this order turns $\CC$ into an ordered set.
\begin{enumerate}[label=(\roman*)]
\item Since the \emph{real} numbers are ordered, we have $a<c$ or $a=c$ or $c<a$. In the first case $z<w$; in the third case $w<z$.

Now consider the second case where $a=c$. We must have $b<d$ or $b=d$ or $d<b$, which correspond to $z<w$, $z=w$, $w<z$ respectively.

Hence we have shown that either $z<w$ or $z=w$ or $w<z$.

\item We now show that if $z<w$ and $w<u$, then $z<u$. Let $u=e+fi$.

Since $z<w$, we have either $a<c$, or $a=c$ and $b<d$. Since $w<u$, we have either $c<f$, or $c=f$ and $d<g$. Hence there are four possible cases:
\begin{itemize}
\item $a<c$ and $c<f$. Then $a<f$ and so $z<u$, as required.
\item $a<c$ and $c=f$, and $d<g$. Again $a<f$, so $z<u$.
\item $a=c$, and $b<d$ and $c<f$. Once again $a<f$ so $z<u$.
\item $a=c$ and $b<d$, and $c=f$ and $d<g$. Then $a=f$ and $b<g$, so $z<u$.
\end{itemize}
\end{enumerate}
\end{solution}

\begin{exercise}[\cite{rudin} 1.10]
Suppose $z=a+bi$, $w=u+iv$, and
\[a=\brac{\frac{|w|+u}{2}}^\frac{1}{2},\quad b=\brac{\frac{|w|-u}{2}}^\frac{1}{2}.\]
Prove that $z^2=w$ if $v\ge0$ and that $\overline{z}^2=w$ if $v\le0$. Conclude that every complex number (with one exception!) has two complex square roots.
\end{exercise}

\begin{solution}
We have
\[a^2-b^2=\frac{|w|+u}{2}-\frac{|w|-u}{2}=u,\]
and
\[2ab=\brac{|w|+u}^\frac{1}{2}\brac{|w|-u}^\frac{1}{2}=\brac{|w|^2-u^2}^\frac{1}{2}=\brac{v^2}^\frac{1}{2}=|v|.\]
Hence if $v\ge0$,
\[z^2=\brac{a^2-b^2}+2abi=u+|v|i=w;\]
if $v\le0$,
\[\overline{z}^2=\brac{a^2-b^2}-2abi=u-|v|i=w.\]
Hence every non-zero $w$ has two square roots $\pm z$ or $\pm\overline{z}$. Of course, $0$ has only one square root, itself.
\end{solution}

\begin{exercise}[\cite{rudin} 1.11]
If $z\in\CC$, prove that there exists $r\ge0$ and $w\in\CC$ with $|w|=1$ such that $z=rw$. Are $w$ and $r$ always uniquely determined by $z$?
\end{exercise}

\begin{solution}
If $z=0$, take $r=0$ and $w=1$; in this case $w$ is not unique.

Otherwise take $r=|z|$ and $w=\frac{z}{|z|}$; these choices are unique, since if $z=rw$, we must have $r=r|w|=|rw|=|z|$ so $w=\frac{z}{r}=\frac{z}{|z|}$ are unique.
\end{solution}
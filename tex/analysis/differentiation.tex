\chapter{Differentiation}\label{chap:differentiation}
\section{The Derivative of A Real Function}
\subsection{Definitions and Properties}
The derivative of a function $f:[a,b]\to\RR$ at a point $x$ can be intuitively thought of as the gradient of the tangent line at $x$.
Consider the gradient of the secant line:
\[\phi(t)=\frac{f(t)-f(x)}{t-x}.\]
Taking the limit $t\to x$, the secant line approaches the tangent line at $x$, and so the value of $\phi(t)$ approaches the gradient of the tangent line.

\begin{figure}[H]
\centering
\begin{tikzpicture}
\pgfplotsset{ticks=none}
\begin{axis}[
    axis lines = left,
    every axis x label/.style={at={(current axis.south east)}, right=4pt},
    every axis y label/.style={at={(current axis.north west)},above=2mm},
    xlabel = \(t\),
    ylabel = {\(f(t)\)},
]

\addplot [
    domain=0:5, 
    samples=100, 
    color=black,
]
{-e^(-x)+1};

\addplot[mark=*] coordinates {(1,0.632)};

\addplot [
    domain=0:2, 
    samples=100, 
    color=black,
]
{0.368*x+0.264};

\end{axis}
\end{tikzpicture}
\end{figure}

\begin{definition}[Derivative]
Suppose $f\colon [a,b]\to\RR$. For any $x\in[a,b]$, if the limit
\begin{equation}\label{eqn:derivative}
\lim_{t\to x}\frac{f(t)-f(x)}{t-x}\quad(a<t<b,t\neq x)
\end{equation}
exists, we call it the \vocab{derivative} of $f$, and denote it by $f^\prime$. 

If $f^\prime$ is defined at $x$, we say that $f$ is \vocab{differentiable} at $x$. If $f^\prime$ is defined at every point of $E\subset[a,b]$, we say that $f$ is \emph{differentiable on} $E$.

We say that $f$ is \emph{continuously differentiable} on $E$ if $f^\prime$ exists at every point of $E$, and $f^\prime$ is continuous on $E$.
\end{definition}

Equivalently, \eqref{eqn:derivative} can be written as
\[\lim_{h\to0}\frac{f(x+h)-f(x)}{h}=f^\prime(x),\]
or,
\[\frac{f(x+h)-f(x)}{h}=f^\prime(x)+\epsilon(h),\]
where $\epsilon(h)\to0$ as $h\to0$. Rearranging gives
\[f(x+h)=f(x)+hf^\prime(x)+h\epsilon(h).\]
Using the small-$o$ notation, we write $o(h)$ for a function that satisfies $o(h)/h\to0$ as $h\to0$. Hence we have
\begin{equation}\label{eqn:derivative-approximation}
f(x+h)=f(x)+hf^\prime(x)+o(h).
\end{equation}
We can interpret \eqref{eqn:derivative-approximation} as an \emph{approximation} of $f(x+h)$:
\[f(x+h)=\underbrace{f(x)+hf^\prime(x)}_\text{linear approximation}+\underbrace{o(h)}_\text{error term}.\]

\begin{lemma}[Differentiability implies continuity]\label{lemma:diff-cont}
If $f\colon [a,b]\to\RR$ is differentiable at $x\in[a,b]$, then $f$ is continuous at $x$.
\end{lemma}

\begin{proof}
Suppose $f\colon [a,b]\to\RR$ is differentiable at $x\in[a,b]$. Then the limit $\displaystyle\lim_{t\to x}\frac{f(t)-f(x)}{t-x}$ exists. Thus by arithmetic properties of limits,
\begin{align*}
\lim_{t\to x}[f(t)-f(x)]
&=\lim_{t\to x}\sqbrac{\frac{f(t)-f(x)}{t-x}\cdot(t-x)}\\
&=\lim_{t\to x}\frac{f(t)-f(x)}{t-x}\cdot\lim_{t\to x}(t-x)\\
&=f^\prime(x)\cdot0=0.
\end{align*}
Since $\displaystyle\lim_{t\to x}f(t)=f(x)$, by \ref{lemma:continuity-limit}, $f$ is continuous at $x$.
\end{proof}

\begin{remark}
The converse is not true; it is easy to construct continuous functions which fail to be differentiable at isolated points.
\end{remark}

\begin{example}[Weierstrass function]
Let $0<a<1$, let $b>1$ be an odd integer, and $ab>1+\frac{3}{2}\pi$. Then the function
\[W(x)=\sum_{n=0}^{\infty}a^n\cos(b^n\pi x)\]
is continuous and nowhere differentiable on $\RR$.
\end{example}

\begin{example}
One family of pathological examples in calculus is functions of the form
\[f(x)=x^p\sin\frac{1}{x}.\]
For $p=1$, the function is continuous and differentiable everywhere other than $x=0$; for $p=2$, the function is differentiable everywhere, but the derivative is discontinuous.
\end{example}

\begin{lemma}[Differentiation rules]
Suppose $f,g:[a,b]\to\RR$ are differentiable at $x\in[a,b]$. Then
\begin{enumerate}[label=(\roman*)]
\item For a constant $\alpha$, $\alpha f$ is differentiable at $x$, and\hfill(scalar multiplication)
\[(\alpha f)^\prime(x)=\alpha f^\prime(x).\]
\item $f+g$ is differentiable at $x$, and\hfill(addition)
\[(f+g)^\prime(x)=f^\prime(x)+g^\prime(x).\]
\item $fg$ is differentiable at $x$, and\hfill(product rule)
\[(fg)^\prime(x)=f^\prime(x)g(x)+f(x)g^\prime(x).\]
\item $f/g$ (when $g(x)\neq0$) is differentiable at $x$, and\hfill(quotient rule)
\[\brac{\frac{f}{g}}^\prime(x)=\frac{f^\prime(x)g(x)-f(x)g^\prime(x)}{g(x)^2}.\]
\end{enumerate}
\end{lemma}

\begin{proof} \
\begin{enumerate}[label=(\roman*)]
\item \[(\alpha f)^\prime(x)=\lim_{t\to x}\frac{(\alpha f)(t)-(\alpha f)(x)}{t-x}=\alpha\lim_{t\to x}\frac{f(t)-f(x)}{t-x}=\alpha f^\prime(x).\]
\item \begin{align*}
(f\pm g)^\prime(x)
&=\lim_{t\to x}\frac{(f+g)(t)-(f+g)(x)}{t-x}\\
&=\lim_{t\to x}\frac{f(t)+g(t)-f(x)-g(x)}{t-x}\\
&=\lim_{t\to x}\frac{f(t)-f(x)}{t-x}+\lim_{t\to x}\frac{g(t)-g(x)}{t-x}\\
&=f^\prime(x)+g^\prime(x)
\end{align*}

\item \begin{align*}
(fg)^\prime(x)
&=\lim_{t\to x}\frac{(fg)(t)-(fg)(x)}{t-x}\\
&=\lim_{t\to x}\frac{f(t)g(t)-f(x)g(x)}{t-x}\\
&=\lim_{t\to x}\frac{\sqbrac{f(t)-f(x)}g(t)+f(x)\sqbrac{g(t)-g(x)}}{t-x}\\
&=\lim_{t\to x}\frac{f(t)-f(x)}{t-x}\cdot g(t)+\lim_{t\to x}f(x)\cdot\frac{g(t)-g(x)}{t-x}\\
&=f^\prime(x)g(x)+f(x)g^\prime(x)
\end{align*}

\item \begin{align*}
\brac{\frac{f}{g}}^\prime(x)
&=\lim_{t\to x}\frac{\brac{\frac{f}{g}}(t)-\brac{\frac{f}{g}}(x)}{t-x}\\
&=\lim_{t\to x}\frac{\frac{f(t)}{g(t)}-\frac{f(x)}{g(x)}}{t-x}\\
&=\lim_{t\to x}\frac{1}{g(t)g(x)}\sqbrac{g(x)\cdot\frac{f(t)-f(x)}{t-x}-f(x)\cdot\frac{g(t)-g(x)}{t-x}}\\
&=\frac{f^\prime(x)g(x)-f(x)g^\prime(x)}{g(x)^2}
\end{align*}
\end{enumerate}
\end{proof}

By induction, we can obtain the following extensions of the differentiation rules.

\begin{corollary*}
Suppose $f_1,f_2,\dots,f_n:[a,b]\to\RR$ are differentiable at $x\in[a,b]$. Then
\begin{enumerate}[label=(\roman*)]
\item $f_1+f_2+\cdots+f_n$ is differentiable at $x$, and
\[(f_1+f_2+\cdots+f_n)^\prime(x)={f_1}^\prime(x)+{f_2}^\prime(x)+\cdots+{f_n}^\prime(x).\]
\item $f_1f_2\cdots f_n$ is differentiable at $x$, and
\begin{align*}
(f_1f_2\cdots f_n)^\prime(x)
=&{f_1}^\prime(x)f_2(x)\cdots f_n(x)+f_1(x){f_2}^\prime(x)\cdots f_n(x)\\
&+\cdots+f_1(x)f_2(x)\cdots {f_n}^\prime(x).
\end{align*}
\end{enumerate}
\end{corollary*}

The next result concerns the derivative of composition of functions.

\begin{lemma}[Chain rule]\label{lemma:chain-rule}
Suppose $f$ is continuous on $[a,b]$, $f^\prime(x)$ exists at $x\in[a,b]$, $g$ is defined on $I$ that contains $f([a,b])$, and $g$ is differentiable at $f(x)$. Then $h=g\circ f$ is differentiable at $x$, and
\begin{equation}
h^\prime(x)=g^\prime\brac{f(x)}f^\prime(x).
\end{equation}
\end{lemma}

\begin{proof}
By the definition of the derivative, we have
\begin{equation*}\tag{1}
f(t)-f(x)=(t-x)[f^\prime(x)+u(t)]
\end{equation*}
\begin{equation*}\tag{2}
g(s)-g(f(x))=(s-f(x))[g^\prime(f(x))+v(s)]
\end{equation*}
where $t\in[a,b]$, $s\in I$, $\displaystyle\lim_{t\to x}u(t)=0$, $\displaystyle\lim_{s\to f(x)}v(s)=0$. ($u(t)$ and $v(s)$ can be viewed as some small error terms which eventually go to $0$.) Using first (2) and then (1), we obtain
\begin{align*}
h(t)-h(x)
&=g\brac{f(t)}-g\brac{f(x)}\\
&=[f(t)-f(x)]\cdot[g^\prime(f(x))+v(s)]\\
&=(t-x)[f^\prime(x)+u(t)][g^\prime(f(x))+v(s)],
\end{align*}
or, if $t\neq x$,
\[\frac{h(t)-h(x)}{t-x}=[g^\prime(f(x))+v(s)][f^\prime(x)+u(t)].\]
Taking limits $t\to x$, we see that $u(t)$ and $v(s)$ eventually go to $0$, so
\[h^\prime(x)=\lim_{t\to x}\frac{h(t)-h(x)}{t-x}=g^\prime\brac{f(x)}f^\prime(x)\]
as desired.
\end{proof}

Later on when we talk about properties of differentiation such as the intermediate value theorems, we usually have the following requirement on the function:
\begin{quote}
$f$ is continuous on $[a,b]$, differentiable on $(a,b)$.
\end{quote}
\pagebreak

\subsection{Derivatives of Higher Order}
If $f$ has a derivative $f^\prime$ on an interval, and if $f^\prime$ is itself differentiable, we denote the derivative of $f^\prime$ by $f^{\prime\prime}$, and call $f^{\prime\prime}$ the \emph{second derivative} of $f$. Continuing in this manner, we obtain functions
\[f,f^\prime,f^{\prime\prime},f^{(3)},f^{(4)},\dots,f^{(n)},\]
each of which is the derivative of the preceding one. $f^{(n)}$ is called the $n$-th derivative (or the derivative or order $n$) of $f$.

\begin{notation}
$\mathcal{C}_1[a,b]$ denotes the set of differentiable functions over $[a,b]$ whose derivative is continuous. More generally, $\mathcal{C}_n[a,b]$ denotes the set of functions whose $n$-th derivative is continuous. In particular, $\mathcal{C}_0[a,b]$ is the set of continuous functions over $[a,b]$.
\end{notation}
\pagebreak

\section{Mean Value Theorems}
Let $(X,d)$ be a metric space.

\begin{definition}[Local maximum and minimum]
We say $f\colon X\to\RR$ has
\begin{enumerate}[label=(\roman*)]
\item a \vocab{local maximum} at $x\in X$ if there exists $\delta>0$ such that $f(x)\ge f(t)$ for all $t\in B_\delta(x)$;
\item a \vocab{local minimum} at $x\in X$ if there exists $\delta>0$ such that $f(x)\le f(t)$ for all $t\in B_\delta(x)$.
\end{enumerate}
\end{definition}

\begin{figure}[H]
\centering
\begin{tikzpicture}
\pgfplotsset{ticks=none}
\begin{axis}[
    axis lines = left,
    every axis x label/.style={at={(current axis.south east)}, right=4pt},
    every axis y label/.style={at={(current axis.north west)},above=2mm},
    xlabel = \(t\),
    ylabel = {\(f(t)\)},
]

\addplot[
   domain=0:360,
   samples=181, % with domain=0:360 and 181 samples you get a sample every 2 degrees
   ] {sin(x)};

\addplot[mark=*] coordinates {(90,1)};

\addplot[mark=*] coordinates {(270,-1)};

\end{axis}
\end{tikzpicture}
\end{figure}

Our next result is the basis of many applications of differentiation. 

\begin{lemma}[Fermat's theorem]
Suppose $f\colon [a,b]\to\RR$. If $f$ has a local maximum or minimum at $x\in(a,b)$, and if $f^\prime(x)$ exists, then
\[f^\prime(x)=0.\]
\end{lemma}

\begin{proof}
We prove the case for local maxima; the proof for the case for local minima is similar.

Since $x$ is a local maximum, choose $\delta>0$ such that
\[a<x-\delta<x<x+\delta<b,\]
and $f(x)\ge f(t)$ for all $t\in(x-\delta,x+\delta)$. Then
\[\frac{f(t)-f(x)}{t-x}
\begin{cases}
\ge0&(x-\delta<t<x)\\
\le0&(x<t<x+\delta)
\end{cases}
\]
Letting $t\to x$, we obtain
\[f^\prime(x)
\begin{cases}
\ge0&(x-\delta<t<x)\\
\le0&(x<t<x+\delta)
\end{cases}\]
Hence we must have $f^\prime(x)=0$.
\end{proof}

\begin{theorem}[Rolle's theorem]\label{thrm:rolle}
Suppose $f$ is continuous on $[a,b]$ and differentiable in $(a,b)$. If $f(a)=f(b)$, then there exists $c\in(a,b)$ such that 
\[f^\prime(c)=0.\]
\end{theorem}

The idea is to show that $f$ has a local maximum/minimum, then by Fermat's theorem this will then be the stationary point that we're trying to find.

\begin{proof}
Since $f$ is continuous on $[a,b]$, by the extreme value theorem (\ref{thrm:extreme-value}), $f$ attains its maximum $M$ and minimum $m$.
\begin{itemize}
\item If $M$ and $m$ both equal $f(a)=f(b)$, then $f$ is simply a constant function; hence $f^\prime(x)=0$ for all $x\in[a,b]$.

\item Otherwise, $f$ has a maximum/minimum that does not equal $f(a)=f(b)$. Then there exists $c\in(a,b)$ such that $f(c)$ is a local maximum/minimum. Since $f$ is differentiable on $(a,b)$, $f^\prime(c)$ exists, so by Fermat's theorem, $f^\prime(c)=0$.
\end{itemize}
\end{proof}

\begin{theorem}[Generalised mean value theorem]\label{thrm:generalised-mvt}
Suppose $f$ and $g$ are continuous on $[a,b]$ and differentiable in $(a,b)$. Then there exists $c\in(a,b)$ such that
\begin{equation}
[f(b)-f(a)]g^\prime(c)=[g(b)-g(a)]f^\prime(c).
\end{equation}
\end{theorem}

\begin{proof}
For $t\in[a,b]$, consider the \emph{auxilliary function}
\[h(t)=[f(b)-f(a)]g(t)-[g(b)-g(a)]f(t).\]
Then $h$ is continuous on $[a,b]$, and $h$ is differentiable on $(a,b)$. Moreover,
\[h(a)=f(b)g(a)-f(a)g(b)=h(b).\]
By Rolle's theorem, there exists $c\in(a,b)$ such that $h^\prime(c)=0$; that is,
\[[f(b)-f(a)]g^\prime(c)=[g(b)-g(a)]f^\prime(c)\]
as desired.
\end{proof}

\begin{theorem}[Mean value theorem]\label{thrm:mvt}
Suppose $f$ is continuous on $[a,b]$ and differentiable in $(a,b)$. Then there exists $c\in(a,b)$ such that
\begin{equation}
f(b)-f(a)=f^\prime(c)(b-a).
\end{equation}
\end{theorem}

\begin{proof}
Take $g(x)=x$ in \ref{thrm:generalised-mvt}.
\end{proof}

\begin{lemma}
Suppose $f$ is differentiable in $(a,b)$.
\begin{enumerate}[label=(\roman*)]
\item If $f^\prime(x)\ge0$ for all $x\in(a,b)$, then $f$ is monotonically increasing.
\item If $f^\prime(x)=0$ for all $x\in(a,b)$, then $f$ is constant.
\item If $f^\prime(x)\le0$ for all $x\in(a,b)$, then $f$ is monotonically decreasing.
\end{enumerate}
\end{lemma}

\begin{proof}
All conclusions can be read off from the equation
\[f^\prime(x)=\frac{f(x_2)-f(x_1)}{x_2-x_1},\]
which is valid, for each pair of numbers $x_1,x_2$ in $(a,b)$, for some $x$ between
$x_1$ and $x_2$.
\end{proof}

\begin{theorem}[Generalised Rolle's theorem]
Suppose $f\in\mathcal{C}[a,b]$ is $n$ times differentiable on $(a,b)$. If $f(x)=0$ at the $n+1$ distinct numbers $a\le x_0<x_1<\cdots<x_n\le b$, then there exists $c\in(x_0,x_n)$, and hence in $(a,b)$, such that
\[f^{(n)}(c)=0.\]
\end{theorem}
\pagebreak

\section{Continuity of Derivatives}
The following result implies some sort of a ``intermediate value'' property of derivatives that is similar to continuous functions.

\begin{theorem}[Darboux's theorem]
Suppose $f$ is differentiable on $[a,b]$, and suppose $f^\prime(a)<c<f^\prime(b)$. Then there exists $x\in(a,b)$ such that $f^\prime(x)=c$.
\end{theorem}

\begin{proof}
For $t\in(a,b)$, consider the auxilliary function
\[g(t)=f(t)-ct.\]
Then
\[g^\prime(a)=f^\prime(a)-c<0,\]
so there exists $t_1\in(a,b)$ such that $g(t_1)<g(a)$. Similarly,
\[g^\prime(b)=f^\prime(b)-c>0,\]
so there exists $t_2\in(a,b)$ such that $g(t_2)<g(b)$.

By the extreme value theorem, $g$ attains its minimum on $[a,b]$. From above, $g(a)$ and $g(b)$ cannot be minimums, so $g$ attains its minimum at $x\in(a,b)$. By Fermat's theorem, $g^\prime(x)=0$. Hence $f^\prime(x)=c$, as desired.
\end{proof}

\begin{corollary}
If $f$ is differentiable on $[a,b]$, then $f^\prime$ cannot have any simple discontinuities on $[a,b]$.
\end{corollary}
\pagebreak

\section{L'Hopital's Rule}
The following result is frequently used in the evaluation of limits.

\begin{lemma}[L'Hopital's rule]
Suppose $f$ and $g$ are differentiable over $(a,b)$, with $g^\prime(x)\neq0$ for all $x\in(a,b)$, where $-\infty\le a<b\le+\infty$. 
If either
\begin{enumerate}[label=(\roman*)]
\item $\displaystyle\lim_{x\to a}f(x)=0$ and $\displaystyle\lim_{x\to a}g(x)=0$; or
\item $\displaystyle\lim_{x\to a}g(x)=+\infty$,
\end{enumerate}
and
\[\lim_{x\to a}\frac{f^\prime(x)}{g^\prime(x)}=A,\]
then
\[\lim_{x\to a}\frac{f(x)}{g(x)}=A.\]
\end{lemma}

The analogous statement is of course also true if $x>b$, or if $g(x)\to-\infty$ in (ii).

Note that we now use the limit concept in the extended sense of \cref{defn:limit-function-infinite}.

\begin{proof}
We first consider the case in which $-\infty\le A<+\infty$. Choose $q\in\RR$ such that $A<q$, and choose $r\in\RR$ such that $A<r<q$.
By (13) there exists $c\in(a,b)$ such that $a<x<c$ implies
\[\frac{f^\prime(x)}{g^\prime(x)}<r.\]
If $a<x<y<c$, then by the generalised mean value theorem (\ref{thrm:generalised-mvt}), there exists $t\in(x,y)$ such that
\[\frac{f(x)-f(y)}{g(x)-g(y)}=\frac{f^\prime(t)}{g^\prime(t)}<r.\]
\begin{enumerate}[label=(\roman*)]
\item Suppose $\displaystyle\lim_{x\to a}f(x)=0$ and $\displaystyle\lim_{x\to a}g(x)=0$. Let $x\to a$ in (18), we see that
\[\frac{f(y)}{g(y)}\le r<q\quad(a<y<c).\]

\item Next, suppose $\displaystyle\lim_{x\to a}g(x)=+\infty$. 
Keeping $y$ fixed in (18), we can choose a point $c_1\in(a,y)$ such that $g(x)>g(y)$ and $g(x)>0$ if $a<x<c_1$. 
Multiplying (18) by $[g(x)-g(y)]/g(x)$, we obtain
\[\frac{f(x)}{g(x)}<r-r\frac{g(y)}{g(x)}+\frac{f(y)}{g(x)}\quad(a<x<c_1).\]
If we let $x\to a$ in (20), (15) shows that there exists $c_2\in(a,c_1)$ such that
\[\frac{f(x)}{g(x)}<q\quad(a<x<c_2).\]
Summing up, (19) and (21) show that for any $q$, subject only to the condition $A<q$, there is a point $c_2$ such that $f(x)/g(x)<q$ if $a<x<c_2$.
\end{enumerate}

In the same manner, if $-\infty<A\le+\infty$, and $p$ is chosen so that $p<A$, we can find a point $c_3$ such that
\[p<\frac{f(x)}{g(x)}\quad(a<x<c_3),\]
and (16) follows from these two statements.
\todo{to review proof}
\end{proof}
\pagebreak

\section{Taylor's Theorem}
\begin{theorem}[Taylor's theorem]
Suppose $f\colon [a,b]\to\RR$, $f^{(n-1)}$ is continuous on $[a,b]$, $f^{(n)}$ exists on $(a,b)$. Assume that $c\in[a,b]$. Let the \emph{Taylor polynomial} of degree $n-1$ of $f$ at $x=c$ be
\begin{align*}
P_{n-1}(x)&=\sum_{k=0}^{n-1}\frac{f^{(k)}(c)}{k!}(x-c)^k\\
&=f(c)+f^\prime(c)(x-c)+\frac{f^{\prime\prime}(c)}{2!}(x-c)^2+\cdots+\frac{f^{(n-1)}(c)}{(n-1)!}(x-c)^{n-1}.
\end{align*}
Then for every $x\in[a,b]$, $x\neq c$, there exists $z_x$ between $x$ and $c$ such that
\begin{equation}\label{eqn:taylor-thrm}
f(x)=P_{n-1}(x)+\frac{f^{(n)}(z_x)}{n!}(x-c)^n.
\end{equation}
\end{theorem}

For $n=1$, this is just the mean value theorem. In general, the theorem shows that $f$ can be approximated by a polynomial of degree $n-1$, and that \eqref{eqn:taylor-thrm} allows us to accurately estimate the error.

\begin{proof}
Let $M$ be the number defined by
\[f(x)=P_{n-1}(x)+M(x-c)^n.\]
We claim that $n!M=f^{(n)}(z_x)$ for some $z_x$ between $x$ and $c$.

For all $x\in[a,b]$, let
\[g(x)=f(x)-P_{n-1}(x)-M(x-c)^n.\]
Then for all $x\in(a,b)$,
\[g^{(n)}(x)=f^{(n)}(x)-n!M.\]
Hence our proof will be complete if we can show that $g^{(n)}(z_x)=0$ for some $z_x$ between $c$ and $x$.

Since $P_{n-1}^{(k)}(c)=f^{(k)}(c)$ for $k=0,\dots,n-1$, we have
\[g(c)=g^\prime(c)=\cdots=g^{(n-1)}(c)=0.\]

By our choice of $M$, we have that $g(x)=0$. By the mean value theorem, there exists $x_1$ between $x$ and $c$ such that $g^\prime(x_1)=0$. Since $g^\prime(c)=0$, we conclude similarly that $g^{\prime\prime}(x_2)=0$ for some $x_2$ between $x_1$ and $c$. After $n$ steps we arrive at the conclusion that $g^{(n)}(x_n)=0$ for some $x_n$ between $x_{n-1}$ and $c$, that is, between $x$ and $c$.
\end{proof}
\pagebreak

\section{Differentiation of Vector-valued Functions}
Definition 5.1 applies without any change to complex functions
$f$ defined on $[a, b]$, and Theorems 5.2 and 5.3, as well as their proofs, remain valid. 
If $f_1$ and $f_2$ are the real and imaginary parts of $f$, that is, if
\[f(t)=f_1(t)+if_2(t)\]
for $a\le t\le b$, where $f_1(t)$ and $f_2(t)$ are real, then we clearly have
\[f^\prime(x)=f_1^\prime(x)+if_2^\prime(x);\]
also, $f$ is differentiable at $x$ if and only if both $f_1$ and $f_2$ are differentiable at $x$.

Passing to vector-valued functions $\vb{f}:[a,b]\to\RR^k$, we may still apply Definition 5.1 to define $\vb{f}^\prime(x)$. The term $\phi(t)$ in (1) is now, for each $t$, a point in $\RR^k$, and the limit in (2) is taken with respect to the norm of $\RR^k$. In other words, $\vb{f}^\prime(x)$ is that point of $\RR^k$ (if there is one) for which
\[\lim_{t\to x}\absolute{\frac{\vb{f}(t)-\vb{f}(x)}{t-x}-\vb{f}^\prime(x)}=0,\]
and $\vb{f}^\prime$ is again a function with values in $\RR^k$.

If $f_1,\dots,f_k$ are the components of $\vb{f}$, as defined in Theorem 4.10, then
\[\vb{f}^\prime=\brac{f_1,\dots,f_k},\]
and $\vb{f}$ is differentiable at a point $x$ if and only if each of the functions $f_1,\dots,f_k$ is differentiable at $x$.

Theorem 5.2 is true in this context as well, and so is Theorem 5.3(a) and (b), if $fg$ is replaced by the inner product $\vb{f}\cdot\vb{g}$ (see Definition 4.3).

When we turn to the mean value theorem, however, and to one of its consequences, namely, L'Hospital's rule, the situation changes. The next two examples will show that each of these results fails to be true for complex-valued functions.

\begin{example}
Define, for real $x$,
\[f(x)\colonequals e^{ix}.\]
Then $f(x)=\cos x+i\sin x$, so
\[f(2\pi)-f(0)=1-1=0,\]
but $f^\prime(x)=ie^{ix}$, so $|f^\prime(x)|=1$ for all real $x$.

Hence the mean value theorem fails to hold in this case.
\end{example}

\begin{example}
On $(0,1)$ define
\[f(x)\colonequals x,\quad g(x)\colonequals x+x^2 e^{i/x^2}.\]
Since $|e^{it}=1|$, we see that
\[\lim_{x\to0}\frac{f(x)}{g(x)}=1.\]
Next, 
\[g^\prime(x)=1+\brac{2x-\frac{2i}{x}}e^{i/x^2}\quad(0<x<1),\]
so that
\[|g^\prime(x)|\ge\absolute{2x-\frac{2i}{x}}-1\ge\frac{2}{x}-1.\]
Hence
\[\absolute{\frac{f^\prime(x)}{g^\prime(x)}}=\frac{1}{|g^\prime(x)|}\le\frac{x}{2-x}\]
and so
\[\lim_{x\to0}\frac{f^\prime(x)}{g^\prime(x)}=0.\]
By (36) and ( 40), L'Hospital's rule fails in this case. Note also that $g^\prime(x)\neq0$ on $(0,1)$, by (38). 
\end{example}

However, there is a consequence of the mean value theorem which, for purposes of applications, is almost as useful as Theorem 5.10, and which remains true for vector-valued functions: From Theorem 5.10 it follows that 
\[|f(b)-f(a)|\le(b-a)\sup_{x\in[a,b]}|f^\prime(x)|.\]

\begin{theorem}\label{thrm:mvt-vector-valued-functions}
Suppose $\vb{f}:[a,b]\to\RR^k$ is continuous on $[a,b]$ and differentiable in $(a,b)$. Then there exists $x\in(a,b)$ such that
\begin{equation}
\norm{\vb{f}(b)-\vb{f}(a)}\le(b-a)\norm{\vb{f}^\prime(x)}.
\end{equation}
\end{theorem}

\begin{proof}
Put $\vb{z}=\vb{f}(b)-\vb{f}(a)$, and define
\[\phi(t)=\vb{z}\cdot\vb{f}(t)\quad(a\le t\le b).\]
Then $\phi$ is a real-valued continuous function on $[a,b]$ which is differentiable in $(a,b)$. By the mean value theorem (\ref{thrm:mvt}), there exists $x\in(a,b)$ such that
\[\phi(b)-\phi(a)=(b-a)\phi^\prime(x)=(b-a)\vb{z}\cdot\vb{f}^\prime(x).\]

On the other hand,
\begin{align*}
\phi(b)-\phi(a)
&=\vb{z}\cdot\vb{f}(b)-\vb{z}\cdot\vb{f}(a)\\
&=\vb{z}\cdot\brac{\vb{f}(b)-\vb{f}(a)}\\
&=\vb{z}\cdot\vb{z}=\norm{\vb{z}}^2.
\end{align*}

By the Cauchy--Schwarz inequality, we obtain
\[\norm{\vb{z}}^2=(b-a)\norm{\vb{z}\cdot\vb{f}^\prime(x)}\le(b-a)\norm{\vb{z}}\:\norm{\vb{f}^\prime(x)}.\]
Hence $\norm{\vb{z}}\le(b-a)\norm{\vb{f}^\prime(x)}$, which is the desired conclusion.
\end{proof}
\pagebreak

\section*{Exercises}
\begin{exercise}
Let $f$ and $g$ be continuous on $[a,b]$ and differentiable on $(a,b)$. If $f^\prime(x)=g^\prime(x)$, then $f(x)=g(x)+C$.
\end{exercise}

\begin{exercise}
Given that $f(x)=x^\alpha$ where $0<\alpha<1$. Prove that $f$ is uniformly continuous on $[0,+\infty)$.
\end{exercise}

\begin{exercise}
Let $f$ be continuous on $[0,1]$ and differentiable on $(0,1)$ where $f(0)=f(1)=0$. Prove that there exists $c\in(0,1)$ such that
\[ f(x)+f^\prime(x)=0. \]
\end{exercise}
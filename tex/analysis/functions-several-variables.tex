\chapter{Functions of Several Variables}
\section{Linear Transformations}


\section{Differentiation}
Recall that for $f:(a,b)\to\RR$, we defined the derivative as
\[\lim_{h\to0}\frac{f(x+h)-f(x)}{h}.\]
In other words, there exists a number $a$ such that
\[\lim_{h\to0}\absolute{\frac{f(x+h)-f(x)}{h}-a}=\lim_{h\to0}\absolute{\frac{f(x+h)-f(x)-ah}{h}}=\lim_{h\to0}\frac{|f(x+h)-f(x)-ah|}{|h|}=0.\]
Multiplying by $a$ is a linear map in one dimension; that is, $a\in\mathcal{L}(\RR,\RR)$. Hence we can use this definition to extend differentiation to more variables.

\begin{definition}
Let $\vb{f}:U\subset\RR^n\to\RR^m$, where $U$ is an open subset of $\RR^n$. Then $\vb{f}$ is \vocab{differentiable} at $\vb{x}\in U$ if there exists $A\in\mathcal{L}(\RR^n,\RR^m)$ such that
\[\lim_{\vb{h}\to\vb{0}}\frac{\norm{\vb{f}(\vb{x}+\vb{h})-\vb{f}(\vb{x})-A\vb{h}}}{\norm{\vb{h}}}=0.\]
Then we write $\vb{f}^\prime(\vb{x})=A$, and say that $A$ is the \emph{derivative} of $\vb{f}$ at $\vb{x}$.

If $\vb{f}$ is differentiable at every $\vb{x}\in U$, we say that $\vb{f}$ is \emph{differentiable on} $U$. 
\end{definition}

Note that the derivative is a function from $U$ to $\mathcal{L}(\RR^n,\RR^m)$.

We now show that the above derivative is in fact unique.

\begin{lemma}
Let $\vb{f}:U\subset\RR^n\to\RR^m$, where $U$ is an open subset of $\RR^n$. Suppose $\vb{x}\in U$, and there exist $A,B\in\mathcal{L}(\RR^n,\RR^m)$ such that
\[\lim_{\vb{h}\to\vb{0}}\frac{\norm{\vb{f}(\vb{x}+\vb{h})-\vb{f}(\vb{x})-A\vb{h}}}{\norm{\vb{h}}}=0\quad\text{and}\quad\lim_{\vb{h}\to\vb{0}}\frac{\norm{\vb{f}(\vb{x}+\vb{h})-\vb{f}(\vb{x})-B\vb{h}}}{\norm{\vb{h}}}=0.\]
Then $A=B$.
\end{lemma}

\begin{proof}
Write
\begin{align*}
\frac{\norm{(A-B)\vb{h}}}{\norm{\vb{h}}}
&=\frac{\norm{\brac{\vb{f}(\vb{x}+\vb{h})-\vb{f}(\vb{x})-A\vb{h}}-\brac{\vb{f}(\vb{x}+\vb{h})-\vb{f}(\vb{x})-B\vb{h}}}}{\norm{\vb{h}}}\\
&\le\frac{\norm{\vb{f}(\vb{x}+\vb{h})-\vb{f}(\vb{x})-A\vb{h}}}{\norm{\vb{h}}}+\frac{\norm{\vb{f}(\vb{x}+\vb{h})-\vb{f}(\vb{x})-B\vb{h}}}{\norm{\vb{h}}}.
\end{align*}
Taking the limit $\vb{h}\to\vb{0}$, we must have $\displaystyle\lim_{\vb{h}\to\vb{0}}\frac{\norm{(A-B)\vb{h}}}{\norm{\vb{h}}}=0$. For fixed $\vb{h}\neq\vb{0}$, it follows that
\[\lim_{t\to0}\frac{\norm{(A-B)(t\vb{h})}}{\norm{t\vb{h}}}=0.\]
The linearity of $A-B$ shows that the LHS of the above equation is independent of $t$. Thus $(A-B)\vb{h}=0$ for every $\vb{h}\in\RR^n$. Hence $A-B=0$, therefore $A=B$. 
\end{proof}

Another way to write the differentiability is to write
\[\vb{f}(\vb{x}+\vb{h})-\vb{f}(\vb{x})=\vb{f}^\prime(\vb{x})\vb{h}+\vb{r}(\vb{h})\]
where the remainder $\vb{r}(\vb{h})$ satisfies $\displaystyle\lim_{\vb{h}\to\vb{0}}\frac{\norm{\vb{r}(\vb{h})}}{\norm{\vb{h}}}=0$.

\begin{example}
If $A\in\mathcal{L}(\RR^n,\RR^m)$ and $\vb{x}\in\RR^n$, then
\[A^\prime(\vb{x})=A.\]
\begin{proof}

\end{proof}
\end{example}

\begin{lemma}[Differentiability implies continuity]
Let $\vb{f}:U\subset\RR^n\to\RR^m$, where $U$ is an open subset of $\RR^n$. If $\vb{f}$ is differentiable at $\vb{x}\in U$, then $\vb{f}$ is continuous at $\vb{x}$.
\end{lemma}

\begin{proof}

\end{proof}

We now extend the chain rule to the present situation. 

\begin{lemma}[Chain rule]
Let $\vb{f}:U\subset\RR^n\to\RR^m$, where $U$ is an open subset of $\RR^n$, and $\vb{f}$ is differentiable at $\vb{x}\in U$. Let $V\subset\RR^m$ be open, $f(U)\subset V$, and let $g:V\to\RR^k$ be differentiable at $\vb{f}(\vb{x})$.

Then $\vb{F}=\vb{g}\circ\vb{f}$ is differentiable at $\vb{x}$, and
\[\vb{F}^\prime(\vb{x})=\vb{g}^\prime\brac{\vb{f}(\vb{x})}\vb{f}^\prime(\vb{x}).\]
\end{lemma}

There is another way to generalise the derivative from one dimension. We can simply hold all but one variables constant and take the regular derivative.

Let $\{\vb{e}_1,\dots,\vb{e}_n\}$ and $\{\vb{u}_1,\dots,\vb{u}_m\}$ be the standard bases of $\RR^n$ and $\RR^m$.

\begin{definition}[Partial derivative]
Let $\vb{f}:U\subset\RR^n\to\RR^m$, where $U$ is an open subset of $\RR^n$. The \emph{components} of $\vb{f}$ are the real functions $f_1,\dots,f_m$ defined by
\[\vb{f}(\vb{x})=\sum_{i=1}^{m}f_i(\vb{x})\vb{u}_i\quad(\vb{x}\in U),\]
or equivalently, $f_i(\vb{x})=\vb{f}(\vb{x})\cdot\vb{u}_i$ for $i=1,\dots,n$.

For $\vb{x}\in U$, $i=1,\dots,m$, $j=1,\dots,n$, we define the \vocab{partial derivative}
\[\pdv{f_i}{x_j}(\vb{x})=\lim_{t\to0}\frac{f_i(\vb{x}+t\vb{e}_j)-f_i(\vb{x})}{t},\]
provided the limit exists.
\end{definition}

Partial derivatives are easier to compute with all the machinery of calculus, and they provide a way to compute the total derivative of a function.

\begin{proposition}
Let $\vb{f}:U\subset\RR^n\to\RR^m$, where $U$ is an open subset of $\RR^n$. If $\vb{f}$ is differentiable at $\vb{x}\in U$, then the partial derivatives $\pdv{f_i}{x_j}$ exist, and
\[\vb{f}^\prime(\vb{x})\vb{e}_j=\sum_{i=1}^{m}\pdv{f_i}{x_j}(\vb{x})\vb{u}_i\quad(j=1,\dots,n).\]
\end{proposition}

\begin{definition}[Gradient]
Let $f:U\subset\RR^n\to\RR$, where $U$ is an open subset of $\RR^n$, and $f$ is a differentiable function. We call the following vector the \vocab{gradient}:
\[(\nabla f)(\vb{x})\coloneqq\sum_{j=1}^{n}\pdv{f}{x_j}(\vb{x})\vb{e}_j.\]
\end{definition}

Let us prove a ``mean value theorem'' for vector-valued functions.

\begin{theorem}
If $\phi:[a,b]\to\RR^n$ is differentiable on $(a,b)$ and continuous on $[a,b]$, then there exists $t\in[a,b]$ such that
\begin{equation}
\norm{\phi(b)-\phi(a)}\le(b-a)\norm{\phi^\prime(t)}.
\end{equation}
\end{theorem}

\begin{definition}
$\vb{f}:U\subset\RR^n\to\RR^m$ is \vocab{continuously differentiable} if $\vb{f}$ is differentiable, and $\vb{f}^\prime:U\to\mathcal{L}(\RR^n,\RR^m)$ is continuous.
\end{definition}

\begin{proposition}
Let $\vb{f}:U\subset\RR^n\to\RR^m$, where $U$ is an open subset of $\RR^n$. Then $\vb{f}$ is continuously differentiable if and only if all the partial derivatives $\pdv{f_i}{x_j}$ exist and are continuous on $U$.
\end{proposition}

\subsection{Inverse Function Theorem}
The idea of a derivative is that if a function is differentiable, then it locally ``behaves like'' the derivative (which is a linear function). So for example, if a function is differentiable and the derivative is invertible, the function is (locally) invertible.

\subsection{Implicit Function Theorem}

\section{Derivatives of Higher Order}
\section{Differentiation of Integrals}

\pagebreak

\section*{Exercises}
\addcontentsline{toc}{section}{Exercises}
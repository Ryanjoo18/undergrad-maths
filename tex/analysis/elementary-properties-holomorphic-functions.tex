\chapter{Elementary Properties of Holomorphic Functions}\label{chap:elementary-properties-holomorphic-functions}
\section{Complex Differentiation}
We now study complex functions defined in subsets of the complex plane. It will be convenient to adopt some standard notations.

The \emph{open disc} centred at $a\in\CC$ with radius $r>0$ is
\[D_r(a)=\{z\in\CC\mid |z-a|<r\}.\]
Its closure is denoted by $\overline{D}_r(a)$. The \emph{punctured disc} centred at $a$ with radius $r$ is
\[D_r^\prime(a)=\{z\in\CC\mid 0<|z-a|<r\}.\]

A \emph{region} is a non-empty connected open subset of $\CC$. Since each open set $\Omega$ in the plane is a union of discs, and since all discs are connected, each component of $\Omega$ is open. Every plane open set is thus a union of disjoint regions. The letter $\Omega$ will from now on denote a plane open set. 

\begin{definition}
Let $f:\Omega\to\CC$. If $z_0\in\Omega$ and
\[\lim_{z\to z_0}\frac{f(z)-f(z_0)}{z-z_0}\]
exists, we denote this limit by $f^\prime(z_0)$ and call it the \emph{derivative} of $f$ at $z_0$. If $f^\prime(z_0)$ exists for every $z_0\in\Omega$, we say that $f$ is \emph{holomorphic} (or \emph{analytic}) in $\Omega$. 
\end{definition}

To be explicit, $f^\prime(z_0)$ exists if
\[\forall\epsilon>0,\quad\exists\delta>0,\quad\forall z\in D_\delta^\prime(z_0),\quad\absolute{\frac{f(z)-f(z_0)}{z-z_0}-f^\prime(z_0)}<\epsilon.\]

\begin{notation}
The class of all holomorphic functions in $\Omega$ will be denoted by $\mathcal{H}(\Omega)$.
\end{notation}

\begin{lemma}
If $f,g\in\mathcal{H}(\Omega)$, then $\mathcal{H}(\Omega)$ is a ring:
\begin{enumerate}[label=(\roman*)]
\item $f+g\in\mathcal{H}(\Omega)$;
\item $fg\in\mathcal{H}(\Omega)$.
\end{enumerate}
\end{lemma}

\begin{lemma}[Chain rule]

\end{lemma}

\begin{example}[Polynomials]

\end{example}

The next result concerns complex power series.
\begin{lemma}
To each power series
\[\sum_{n=0}^{\infty}c_n(z-a)^n\]
there exists $R\in[0,\infty]$ such that the series
\begin{enumerate}[label=(\roman*)]
\item converges absolutely and uniformly in $\overline{D}_r(a)$ for every $r<R$;
\item diverges if $z\notin\overline{D}_r(a)$.
\end{enumerate}
The \emph{radius of convergence} $R$ is given by the root test:
\[\frac{1}{R}=\limsup_{n\to\infty}\sqrt[n]{|c_n|}.\]
\end{lemma}

\section{Integration Over Paths}
\section{Local Cauchy Theorem}
\section{Power Series Representation}
\section{Open Mapping Theorem}
\section{Global Cauchy Theorem}
\section{Calculus of Residues}
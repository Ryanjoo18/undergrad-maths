\chapter{Finite-Dimensional Vector Spaces}\label{chap:finite-dim-vector-spaces}
Key concepts in this chapter include linear combinations, span, linear independence, bases and dimension.

\section{Span and Linear Independence}
\begin{definition}[Linear combination]
$v$ is a \vocab{linear combination}\index{linear combination} of vectors $v_1,\dots,v_n\in V$ if there exists $a_1,\dots,a_n\in\FF$ such that
\[v=a_1v_1+\cdots+a_nv_n.\]
\end{definition}

\begin{definition}[Span]
The \vocab{span}\index{span} of $\{v_1,\dots,v_n\}$ is the set of all linear combinations of $v_1,\dots,v_n$:
\[\spn(v_1,\dots,v_n)\coloneqq\{a_1v_1+\cdots+a_nv_n\mid a_i\in\FF\}.\]
The span of the empty list $(\:)$ is defined to be $\{\vb{0}\}$.

We say that $v_1,\dots,v_n$ \textbf{spans} $V$ if $\spn(v_1,\dots,v_n)=V$.
\end{definition}

\begin{proposition}
$\spn(v_1,\dots,v_n)$ in $V$ is the smallest subspace of $V$ containing $v_1,\dots,v_n$.
\end{proposition}

\begin{proof}
First we show that $\spn(v_1,\dots,v_n)\le V$, using \cref{lemma:subspace-conditions}.
\begin{enumerate}[label=(\roman*)]
\item $\vb{0}=0v_1+\cdots+0v_n\in\spn(v_1,\dots,v_n)$
\item $(a_1v_1+\cdots+a_nv_n)+(c_1v_1+\cdots+c_nv_n)=(a_1+c_1)v_1+\cdots+(a_n+c_n)v_n\in\spn(v_1,\dots,v_n)$, so $\spn(v_1,\dots,v_n)$ is closed under addition.
\item $\lambda(a_1v_1+a_nv_n)=(\lambda a_1)v_1+\cdots+(\lambda a_n)v_n\in\spn(v_1,\dots,v_n)$, so $\spn(v_1,\dots,v_n)$ is closed under scalar multiplication.
\end{enumerate}

Let $M$ be the smallest vector subspace of $V$ containing $v_1,\dots,v_n$. We claim that $M=\spn(v_1,\dots,v_n)$. To show this, we show that (i) $M\subset\spn(v_1,\dots,v_n)$ and (ii) $M\supset\spn(v_1,\dots,v_n)$.
\begin{enumerate}[label=(\roman*)]
\item Each $v_i$ is a linear combination of $v_1,\dots,v_n$, as
\[v_i=0\cdot v_1+\cdots+0\cdot v_{i-1}+1\cdot v_i+0\cdot v_{i+1}+\cdots+0\cdot v_n,\]
so by the definition of the span as the collection of all linear combinations of $v_1,\dots,v_n$, we have that $v_i\in\spn(v_1,\dots,v_n)$. But $M$ is the smallest vector subspace containing $v_1,\dots,v_n$, so
\[M\subset\spn(v_1,\dots,v_n).\]
\item Since $v_i\in M$ ($1\le i\le n$) and $M$ is a vector subspace (closed under addition and scalar multiplication), it follows that
\[a_1v_1+\dots+a_nv_n\in M\]
for all $a_i\in\FF$ (i.e. $M$ contains all linear combinations of $v_1,\dots,v_n$). So
\[\spn(v_1,\dots,v_n)\subset M.\]
\end{enumerate}
\end{proof}

\begin{definition}[Finite-dimensional vector space]
$V$ is \vocab{finite-dimensional}\index{finite-dimensional} if there exists some list of vector $(v_1,\dots,v_n)$ that spans $V$; otherwise, it is \textbf{infinite-dimensional}.
\end{definition}

\begin{remark}
Recall that by defnition every list of vectors has finite length.
\end{remark}

\begin{remark}
From this definition, infinite-dimensionality is the negation of finite-dimensionality (i.e. \emph{not} finite-dimensional). Hence to prove that a vector space is infinite-dimensional, we prove by contradiction; that is, first assume that the vector space is finite-dimensional, then try to come to a contradiction.
\end{remark}

\begin{exercise}
For positive integer $n$, $\FF^n$ is finite-dimensional.
\end{exercise}

\begin{proof}
Suppose $(x_1,x_2,\dots,x_n)\in\FF^n$, then
\[(x_1,x_2,\dots,x_n)=x_1(1,0,\dots,0)+x_2(0,1,\dots,0)+\cdots+x_n(0,0,\dots,1)\]
so
\[(x_1,\dots,x_n)\in\spn\brac{(1,0,\dots,0),(0,1,\dots,0),\dots,(0,\dots,0,1)}.\]
The vectors $(1,0,\dots,0),(0,1,\dots,0),\dots,(0,\dots,0,1)$ spans $\FF^n$, so $\FF^n$ is finite-dimensional.
\end{proof}

\begin{definition}[Linear independence]
A list of vectors $v_1,\dots,v_n$ is \vocab{linearly independent}\index{linear independence} in $V$ if the only choice of $a_1,\dots,a_n\in\FF$ that makes
\[a_1v_1+\cdots+a_nv_n=\vb{0}\]
is $a_1=\cdots=a_n=0$; otherwise, it is \textbf{linearly dependent}.
\end{definition}

\cref{lemma:linear-dependence} will often be useful; it states that given a linearly dependent list of vectors, one of the vectors is in the span of the previous ones and furthermore we can throw out that vector without changing the span of the original list.

\begin{lemma}[Linear dependence lemma]\label{lemma:linear-dependence}
Suppose $(v_1,\dots,v_n)$ is linearly dependent in $V$. Then there exists $v_k$ such that the following hold:
\begin{enumerate}[label=(\roman*)]
\item $v_k\in\spn\{v_1,\dots,v_{k-1}\}$
\item $\spn\{v_1,\dots,v_{k-1},v_{k+1},\dots,v_n\}=\spn(v_1,\dots,v_n)$
\end{enumerate}
\end{lemma}

\begin{proof}
Since $\{v_1,\dots,v_n\}$ is linearly dependent, there exists $a_1,\dots,a_n\in\FF$, not all $0$, such that
\[a_1v_1+\cdots+a_nv_n=0.\]
Let $k=\max\{1,\dots,n\}$ such that $a_k\neq0$. Then
\[v_k=-\frac{a_1}{a_k}v_1-\cdots-\frac{a_{k-1}}{a_k}v_{k-1},\]
proving (i).

To prove (ii), suppose $u\in\spn(v_1,\dots,v_n)$. Then there exists $c_1,\dots,c_n\in\FF$ such that
\[u=c_1v_1+\cdots+c_nv_n.\]

\end{proof}

\cref{prop:length-linind-span} says that no linearly independent list in $V$ is longer than a spanning list in $V$.

\begin{proposition}\label{prop:length-linind-span}
In a finite-dimensional vector space, the length of every linearly independent list of vectors is less than or equal to the length of every spanning list of vectors.
\end{proposition}

\begin{proof}
Suppose $\{u_1,\dots,u_m\}$ linearly independent in $V$, $\{w_1,\dots,w_n\}$ spans $V$. We want to show $m\le n$. We do so through the following steps:
\begin{itemize}
\item[Step 1] Adjoin $u_1$ at the beginning of $\{w_1,\dots,w_n\}$. Then $\{u_1,w_1,\dots,w_n\}$ is linearly dependent.

By linear dependence lemma, one of the vectors in $\{u_1,w_1,\dots,w_n\}$ is a linear combination of the previous vectors. Since $\{u_1,\dots,u_m\}$ is linearly independent, $u_1\neq\vb{0}$, so 
\end{itemize}
\end{proof}

\section{Bases}
\begin{definition}[Basis]
$(v_1\dots,v_n)$ is a \vocab{basis}\index{basis} of $V$ if it is
\begin{enumerate}[label=(\roman*)]
\item linearly independent;
\item spans $V$.
\end{enumerate}
\end{definition}

\begin{example}[Standard basis]
Let $\vb{e}_i=(0,\dots,0,1,0,\dots,0)$ where the $i$-th coordinate is $1$. $(\vb{e}_1,\dots,\vb{e}_n)$ is a basis of $\FF^n$, known as the \textbf{standard basis} of $\FF^n$.
\end{example}

\begin{lemma}[Criterion for basis]\label{lemma:basis-criterion}
The following are equivalent:
\begin{enumerate}[label=(\roman*)]
\item $\{v_1,\dots,v_n\}$ is a basis of $V$.
\item Every $v\in V$ is uniquely expressed as a linear combination of $v_1,\dots,v_n$.
\item $v_i\neq0$, $V=Fv_1\oplus\cdots\oplus Fv_n$.
\end{enumerate}
\end{lemma}

\begin{proof}

\end{proof}

\begin{lemma}\label{lemma:reduce-spanninglist-basis}
Every spanning list in a vector space can be reduced to a basis of the vector space.
\end{lemma}

\begin{proof}
Suppose $B=\{v_1,\dots,v_n\}$ spans $V$. We want to remove some vectors from $B$ so that the remaining vectors form a basis of $V$. We do this through the multistep process described below.

\begin{enumerate}
\item[Step 1] If $v_1=\vb{0}$, delete $v_1$ from $B$. If $v_1\neq\vb{0}$, leave $B$ unchanged.
\item[Step $k$] If $v_k\in\spn\{v_1,\dots,v_{k-1}\}$, delete $v_k$ from $B$. If $v_k\notin\spn\{v_1,\dots,v_{k-1}\}$, leave $B$ unchanged.
\end{enumerate}

Stop the process after step $n$.

Since our original list spanned $V$ and we have discarded only vectors that were already in the span of the previous vectors, the resulting list $B$ spans $V$ because

The process ensures that no vector in $B$ is in the span of the previous ones. By linear dependence lemma, $B$ is linearly independent.

Since $B$ is linearly independent and spans $V$, $B$ is a basis of $V$.
\end{proof}

\begin{proposition}
Every finite-dimensional vector space has a basis.
\end{proposition}

\begin{proof}
By definition, a finite-dimensional vector space has a spanning list. By \cref{lemma:reduce-spanninglist-basis}, the spanning list can be reduced to a basis.
\end{proof}

\begin{lemma}\label{lemma:extend-linind-basis}
Every linearly independent list of vectors in a finite-dimensional vector space can be extended to a basis of the vector space.
\end{lemma}

\begin{proof}
Suppose $\{u_1,\dots,u_m\}$ linearly independent in $V$, $\{w_1,\dots,w_n\}$ spans $V$. Thus
\[\{u_1,\dots,u_m,w_1,\dots,w_n\}\]
spans $V$. By \cref{lemma:reduce-spanninglist-basis}, we can reduce this list to a basis of $V$ consisting $u_1,\dots,u_m$ (since $\{u_1,\dots,u_m\}$ linearly independent) and some of the $w$'s.
\end{proof}

\begin{proposition}
Suppose $U$ is a subspace of $V$. Then there exists a subspace $W$ of $V$ such that $V=U\oplus W$.
\end{proposition}

\begin{proof}

\end{proof}

\begin{proposition}\label{prop:bases-same-length}
Any two bases of a finite-dimensional vector space have the same length.
\end{proposition}

\begin{proof}
Suppose $V$ is finite-dimensional. Let $B_1$ and $B_2$ be two bases of $V$. Then $B_1$ is linearly independent in $V$ and $B_2$ spans $V$, so by \cref{prop:length-linind-span}, the length of $B_1$ is at most the length of $B_2$.

Similarly, $B_2$ is linearly independent in $V$ and $B_1$ spans $V$, so the length of $B_2$ is at most the length of $B_1$.

Hence the length of $B_1$ equals the length of $B_2$, as desired.
\end{proof}

\section{Dimension}
By \cref{prop:bases-same-length}, since any two bases of a fnite-dimensional vector space have the same length, we can formally define the dimension of such spaces.

\begin{definition}[Dimension]
The \vocab{dimension}\index{dimension} of $V$ is the length of any basis of $V$, denoted by $\dim V$.
\end{definition}

\begin{proposition}
Suppose $V$ is finite-dimensional, $U\le V$. Then $\dim U\le\dim V$.
\end{proposition}

\begin{proof}
Think of a basis of $U$ as a linearly independent list in $V$, and think of a basis of $V$ as a spanning list in $U$. Now use 2.22 to conclude that $\dim U\le\dim V$.
\end{proof}

\begin{proposition}
Suppose $V$ is finite-dimensional. Then every linearly independent list of vectors in $V$ with length $\dim V$ is a basis of $V$.
\end{proposition}

\begin{proof}
Suppose $\dim V=n$ and $(v_1,\dots,v_n)$ is linearly independent in $V$. By 2.32, the list $(v_1,\dots,v_n)$ can be extended to a basis of $V$. However, every basis of $V$ has length $n$, which means that no elements are adjoined to $(v_1,\dots,v_n)$. Hence $(v_1,\dots,v_n)$ is a basis of $V$, as desired.
\end{proof}

\begin{proposition}
Suppose $V$ is finite-dimensional, $U\le V$ and $\dim U=\dim V$. Then $U=V$.
\end{proposition}

\begin{proof}
Let $(u_1,\dots,u_n)$ be a basis of $U$. Then $\dim U=n$ so $\dim V=n$. Thus $(u_1,\dots,u_n)$ is a linearly indepdent list of vectors in $V$ (because it is a basis of $U$) of length $\dim V$. From 2.38, we see that $(u_1,\dots,u_n)$ is a basis of $V$. In particular every vector in $V$ is a linear combination of $u_1,\dots,u_n$. Thus $U=V$.
\end{proof}

\begin{proposition}
Suppose $V$ is finite-dimensional. Then every spanning list of vectors in $V$ with length $\dim V$ is a basis of $V$.
\end{proposition}

\begin{proof}
Suppose $\dim V=n$ and $(v_1,\dots,v_n)$ spans $V$. By 2.30, $(v_1,\dots,v_n)$ can be reduced to a basis of $V$. However, every basis of $V$ has length $n$, which means that no elements are deleted from $(v_1,\dots,v_n)$. Hence $(v_1,\dots,v_n)$ is a basis of $V$, as desired.
\end{proof}

\begin{lemma}[Dimension of a sum]
Suppose $U_1,U_2\le V$. Then
\[\dim(U_1+U_2)=\dim U_1+\dim U_2-\dim(U_1\cap U_2).\]
\end{lemma}
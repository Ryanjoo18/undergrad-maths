\chapter{Polynomials}
\section{Definitions}
\begin{definition}[Polynomial]
$p:\FF\to\FF$ is a \vocab{polynomial}\index{polynomial} with coefficients in $\FF$ if there exist $a_i\in\FF$ such that
\[p(z)=a_0+a_1z+\cdots+a_nz^n\quad(z\in\FF)\]
\end{definition}

\begin{notation}
The set of polynomials with coefficients in $\FF$ is denoted by $\FF[z]$.
\end{notation}

\begin{proposition}
With the usual operations of addition and scalar multiplication, $\FF[z]$ is a vector space over $\FF$, as you should verify. Hence $\FF[z]$ is a subspace of $\FF^\FF$ (vector space of functions from $\FF$ to $\FF$).
\end{proposition}

\begin{definition}[Degree]
A polynomial $p\in\FF[z]$ is has \vocab{degree}\index{polynomial!degree} $n$, denoted by $\deg p=n$, if there exist scalars $a_0,a_1,\dots,a_n\in\FF$ with $a_n\neq0$ such that $p(z)=a_0+a_1z+\cdots+a_nz^n$ for all $z\in\FF$.
\end{definition}

\begin{notation}
For non-negative integer $n$, $\FF_n[z]$ denotes the set of polynomials with coefficients in $\FF$ and degree at most $n$.
\end{notation}

\begin{proposition}
For non-negative integer $n$, $\FF_n[z]$ is finite-dimensional.
\end{proposition}

\begin{proof}
$\FF_n[z]=\spn(1,z,z^2,\dots,z^n)$ [here we slightly abuse notation by letting $z^k$ denote a function]. 
\end{proof}

\begin{proposition}
$\FF[z]$ is infinite-dimensional.
\end{proposition}

\begin{proof}
Consider any list of elements of $\FF[z]$. Let $n$ denote the highest degree of the polynomials in this list. Then every polynomial in the span of this list has degree at most $n$. Thus $z^{n+1}$ is not in the span of our list. Hence no list spans $\FF[z]$. Thus $\FF[z]$ is infnite-dimensional.
\end{proof}
\pagebreak

\section{Zeros of Polynomials}
\begin{definition}[Zero of polynomial]
$\lambda\in\FF$ is called a \vocab{zero}\index{polynomial!zero} (or \emph{root}) of a polynomial $p\in\FF[z]$ if
\[p(\lambda)=0.\]
\end{definition}

The next result is the key tool that we will use to show that the degree of a polynomial is unique.

\begin{lemma}[Factor theorem]\label{lemma:factor-thm}
Suppose $n\in\ZZ^+$, $p\in\FF_n[z]$. Suppose $\lambda\in\FF$, then $p(\lambda)=0$ if and only if there exists $q\in\FF_{n-1}[z]$ such that
\[p(z)=(z-\lambda)q(z)\quad(\forall z\in\FF)\]
\end{lemma}

\begin{proof} \

\fbox{$\implies$} Suppose $p(\lambda)=0$. Let $a_0,a_1,\dots,a_n\in\FF$ be such that
\[p(z)=a_nz^n+\cdots+a_1z+a_0\quad(\forall z\in\FF)\]
Then for all $z\in\FF$,
\begin{align*}
p(z)&=p(z)-p(\lambda)\\
&=\brac{a_nz^n+\cdots+a_1z+a_0}-\brac{a_n\lambda^n+\cdots+a_1\lambda+a_0}\\
&=a_n\brac{z^n-\lambda^n}+\cdots+a_1(z-\lambda).
\end{align*}
Note that for each $k=1,\dots,n$, we can factorise
\[z^k-\lambda^k=(z-\lambda)\brac{z^{k-1}+z^{k-2}\lambda+\cdots+\lambda^{k-1}}.\]
Thus $p$ equals $z-\lambda$ times some polynomial of degree $n-1$, as desired.

\fbox{$\impliedby$} Now suppose that there exists a polynomial $q\in\FF[z]$ such that
\[p(z)=(z-\lambda)q(z)\quad(\forall z\in\FF)\]
Then
\[p(\lambda)=(\lambda-\lambda)q(\lambda)=0,\]
as desired.
\end{proof}

Now we can prove that the degree of a polynomials determines how many zeros it has.

\begin{proposition}
Suppose $n\in\ZZ^+$, $p\in\FF_n[z]$. Then $p$ has at most $n$ zeros in $\FF$.
\end{proposition}

\begin{proof}
Prove by induction on $n$.

The desired result holds for $n=1$ because if $a_1\neq0$ then the polynomial $a_0+a_1z$ has only one zero (which equals $-\frac{a_0}{a_1}$).

Now assume the desired result holds for $n-1$. If $p$ has no zeros in $\FF$, then the desired result holds and we are done. Thus suppose $p$ has a zero $\lambda\in\FF$. By \cref{lemma:factor-thm}, there exists $q\in\FF[z]$ of degree $n-1$ such that
\[p(z)=(z-\lambda)q(z)\quad(\forall z\in\FF)\]
By the induction hypothesis, $q$ has at most $n-1$ zeros in $\FF$. The equation above shows that the zeros of $p$ in $\FF$ are exactly the zeros of $q$ in $\FF$ along with $\lambda$. Thus $p$ has at most $n$ zeros in $\FF$.
\end{proof}

The result above implies that the coefficients of a polynomial are uniquely determined (because if a polynomial had two different sets of coefficients, then subtracting the two representations of the polynomial would give a polynomial with some nonzero coefficients but infinitely many zeros). In particular, the degree of a polynomial is uniquely defined.
\pagebreak

\section{Division Algorithm for Polynomials}
\begin{proposition}[Division algorithm]
Suppose $p,s\in\FF[z]$, $s\neq0$. Then there exists unique polynomials $q,r\in\FF[z]$, where $\deg r<\deg s$, such that
\[p=sq+r.\]
\end{proposition}

\begin{proof}
Let $n=\deg p$, $m=\deg s$. If $n<m$, take $q=0$ and $r=p$ to get the desired equation.

Now assume that $n\ge m$.
\end{proof}
\pagebreak

\section{Factorisation of Polynomials over $\CC$}
\begin{theorem}[Fundamental theorem of algebra, first version]
Every non-constant polynomial with complex coefficients has a zero in $\CC$.
\end{theorem}

\begin{theorem}[Fundamental theorem of algebra]
If $p\in\CC[z]$ is a non-constant polynomial, then $p$ has a unique factorisation (except for the order of the factors) of the form
\[p(z)=c(z-\lambda_1)\cdots(z-\lambda_n),\]
where $c,\lambda_1,\dots,\lambda_n\in\CC$.
\end{theorem}
\pagebreak

\section{Factorisation of Polynomials over $\RR$}
A polynomial with real coefficients may have no real zeros. For example, the polynomial $x^2+1$ has no real zeros.

To obtain a factorisation theorem over $\RR$, we will use our factorisation theorem over $\CC$. We begin with the next result.

\begin{proposition}
Suppose $p\in\CC[z]$ is a polynomial with real coefficients. If $\lambda\in\CC$ is a zero of $p$, then so is the conjugate $\overline{\lambda}$.
\end{proposition}

We want a factorisation theorem for polynomials with real coefficients. We begin with the following result.

\begin{lemma}[Factorisation of quadratic polynomial]
Suppose $b,c\in\RR$. Then there is a polynomial factorisation of the form
\[x^2+bx+c=(x-\lambda_1)(x-\lambda_2)\]
with $\lambda_1,\lambda_2\in\RR$ if and only if $b^2\ge 4c$.
\end{lemma}

\begin{theorem}[Factorisation of polynomial over $\RR$]
Suppose $p\in\RR[x]$ is a non-constant polynomial. Then $p$ has a unique factorisation (except for the order of the factors) of the form
\[p(x)=c(x-\lambda_1)\cdots(x-\lambda_n)(x^2+b_1x+c_1)\cdots(x^2+b_Nx+c_N),\]
where $c,\lambda_1,\dots,\lambda_n,b_1,\dots,b_N,c_1,\dots,c_N\in\RR$, with ${b_k}^2<4c_k$ for each $k$.
\end{theorem}
\chapter{Groups}\label{chap:groups}
One of the simplest forms of abstract algebraic systems is a \emph{group}, which is roughly a set of objects and a rule for multiplying them together. 
Groups arise all over mathematics, particularly where there is symmetry.

\section{Groups}
\subsection{Definitions and Properties}
A \vocab{binary operation} on a set $G$ is a map $\ast\colon G\times G\to G$. 

\begin{notation}
For any $a,b\in G$, if the operation is clear, we write $ab$ for the image of $(a,b)$ under $\ast$.
\end{notation}

\begin{definition}[Group]
A \vocab{group}\index{group} $(G,\ast)$ consists of a set $G$ and a binary operation $\ast$ on $G$ satisfying the following properties:
\begin{enumerate}[label=(\roman*)]
\item $a(bc)=(ab)c$ for all $a,b,c\in G$;\hfill(associativity)
\item there exists $e\in G$ such that $ae=ea=a$ for all $a\in G$;\hfill(identity)
\item for all $a\in G$, there exists $c\in G$ such that $ac=ca=e$.\hfill(invertibility)
\end{enumerate}
\end{definition}

\begin{notation}
If the operation is clear, we simply denote a group $(G,\ast)$ by $G$.
\end{notation}

\begin{remark}
When verifying that $G$ is a group we have to check (i), (ii), (iii) above and also that $\ast$ is a binary operation closed in $G$: $ab\in G$ for all $a,b\in G$.
\end{remark}

\begin{notation}
Since $\ast$ is associative, we omit unnecessary parentheses and write $(ab)c=a(bc)=abc$.
\end{notation}

We say $G$ is \vocab{abelian} if the operation is commutative; otherwise, $G$ is \emph{non-abelian}.

\begin{lemma}
A group has a unique identity.
\end{lemma}

\begin{proof}
Suppose that $e$ and $e^\prime$ are identities of $G$. Then
\[e=ee^\prime=e^\prime\]
where the first equality holds since $e^\prime$ is an identity, and the second equality holds since $e$ is an identity.
\end{proof}

\begin{notation}
We denote \emph{the} identity of $G$ as $1_G$, and omit the subscript if there is no ambiguity.
\end{notation}

\begin{lemma}
Each element of a group has a unique inverse.
\end{lemma}

\begin{proof}
Suppose that $b$ and $c$ are both inverses of $a$. Then $ab=1$, $ca=1$, so
\[c=c1=c(ab)=(ca)b=1b=b.\]
\end{proof}

\begin{notation}
We denote \emph{the} inverse of $a\in G$ as $a^{-1}$.
\end{notation}

\begin{lemma}
Let $G$ be a group.
\begin{enumerate}[label=(\roman*)]
\item $(a^{-1})^{-1}=a$ for all $a\in G$.
\item $(ab)^{-1}=b^{-1}a^{-1}$ for all $a,b\in G$.
\item For any $a_1,\dots,a_n\in G$, $a_1\cdots a_n$ is independent of how we arrange the parantheses (generalised associative law).
\end{enumerate}
\end{lemma}

\begin{proof} \
\begin{enumerate}[label=(\roman*)]
\item To show $(a^{-1})^{-1}=a$ is exactly the problem of showing that $a$ is the inverse of $a^{-1}$, which is by definition of the inverse (with the roles of $a$ and $a^{-1}$ interchanged).

\item Let $c=(ab)^{-1}$. Then $(ab)c=1$, or $a(bc)=1$ by associativity, which gives $bc=a^{-1}$. Applying $b^{-1}$ on both sides gives $c=b^{-1}a^{-1}$.

\item Induct on $n$. The result is trivial for $n=1,2,3$. For all $k<n$ assume that any $a_1\cdots a_k$ is independent of parantheses. Then
\[(a_1\cdots a_n)=(a_1\cdots a_k)(a_{k+1}\cdots a_n).\]
By inductive hypothesis, both terms are independent of parentheses since $k,n-k<n$. Hence by induction we are done.
\end{enumerate}
\end{proof}

\begin{lemma}[Cancellation law]
Let $a,b\in G$. Then the equations $ax=b$ and $ya=b$ have unique solutions for $x,y\in G$.
\end{lemma}

This means that we can cancel on the left and right.

\begin{proof}
To solve $ax=b$, apply $a^{-1}$ on both sides to get $x=a^{-1}b$. The uniqueness of $x$ follows because $a^{-1}$ is unique. 

A similar case holds for $ya=b$.
\end{proof}

We now introduce notation for repeated application of the operation on an element.

\begin{notation}
For any $a\in G$, $n\in\NN$, denote $a^n=\underbrace{a\cdot a\cdots a}_\text{$n$ times}$, $a^0=1$, and $a^{-n}=(a^{-1})^n$.
\end{notation}

The usual rules of exponents hold true:
\[a^{m+n}=a^m a^n\]
\[(a^m)^n=a^{mn}\]
\[(a^n)^{-1}=(a^{-1})^n\]

\begin{definition}[Order of a group]
Let $G$ be a group. The \vocab{order} of $G$ is its cardinality $|G|$. We say $G$ is a \emph{finite group} if $|G|<\infty$.
\end{definition}

One way to represent a finite group is by means of a \vocab{Cayley table}. Let $G=\{1,g_2,g_3,\dots,g_n\}$. The Cayley table of $G$ is a square grid which contains all the possible products of two elements from $G$: the product $g_ig_j$ appears in the $i$-th row and $j$-th column.

\begin{remark}
Note that a group is abelian if and only if its Cayley table is symmetric about the main (top-left to bottom-right) diagonal.
\end{remark}

\subsection{Examples}
\begin{example} \
\begin{itemize}
\item $\ZZ$, $\QQ$, $\RR$, $\CC$ are abelian groups under addition.
\item $\QQ^\times=\QQ\setminus\{0\}$, $\RR^\times=\RR\setminus\{0\}$, and $\CC^\times=\CC\setminus\{0\}$ are groups under multiplication.
\item The complex numbers of absolute value 1 form a group under multiplication.
\item $\{1,-1\}$ is a group under multiplication.
\item $\{1,-1,i,-i\}$ is a group under multiplication.
\end{itemize}
\end{example}

\begin{example}[Modular arithmetic]
For $n\in\NN$, the set of (congruence classes of) integers modulo $n$, $\ZZ/n\ZZ$, is an abelian group under addition.

For $n\in\NN$, $(\ZZ/n\ZZ)^\times$ is an abelian group under multiplication.
\end{example}

\begin{example}[Direct product]
Let $G$, $H$ be groups. The cartesian product $G\times H$ is a group under the operation
\[(g_1,h_1)\cdot(g_2,h_2)=(g_1g_2,h_1h_2).\]
We call $G\times H$ the \emph{direct product} of $G$ and $H$.

One may also take a direct product of a finite number of groups. Thus if $G_1,\dots,G_n$, we let
\[\prod_{i=1}^{n}G_i=G_1\times\cdots\times G_n\]
be the set of all $n$-tuples $(x_1,\dots,x_n)$ with $x_i\in G_i$. We define multiplication componentwise, and see at once that $G_1\times\cdots\times G_n$ is a group. If $1_i$ is the identity of $G_i$, then $(1_1,\dots,1_n)$ is the identity of the product.
\end{example}

\begin{example}[Dihedral groups]
An important family of groups is the \vocab{dihedral groups}. For $n\in\NN$, $n\ge3$, let $D_{2n}$ be the set of symmetries of a regular $n$-gon.

Let $r$ be the rotation clockwise about the origin by $\frac{2\pi}{n}$ radians, $s$ be the reflection about the line of symmetry through the first labelled vertex and the origin. (Read from right to left: for instance, $sr$ means do $r$ then $s$.)

Properties of $D_{2n}$:
\begin{itemize}
\item $1,r,r^2,\dots,r^{n-1}$ are all distinct and $r^n=1$, so $|r|=n$.
\item $s^2=1$ since we either reflect or do not reflect, so $|s|=2$.
\item $s\neq r^i$ for any $i$, since the effect of any reflection cannot be obtained from any form of rotation.
\item $sr^i\neq sr^j$ for all $i\neq j$ ($0\le i,j\le n-1$), so
\[D_{2n}=\{1,r,\dots,r^{n-1},s,sr,\dots,sr^{n-1}\}\]
and thus $|D_{2n}|=2n$.
\item $rs=sr^{-1}$
\item $r^is=sr^{-i}$

Proof: From above, this is true for $i=1$. Assume it holds for $k<n$. Then $r^{k+1}s=r(r^ks)=rsr^{-k}$. Then $rs=sr^{-1}$ so $rsr^{-k}=sr^{-1}r^{-k}=sr^{-k-1}$ so we are done.
\end{itemize}
\end{example}

Note that for each $n\in\NN$, the generators of $D_{2n}$ are $r$ and $s$, and we have shown that they satisfy $r^n=1$, $s^2=1$, and $rs=sr^{-1}$; these are called \emph{relations}. Any other equation involving the generators can be derived from these relations.

Any such collection of generators $S$ and relations $R_1,\dots,R_m$ for a group $G$ is called a \emph{presentation}, written
\[G=\langle S\mid R_1,\dots,R_m\rangle.\]

For example,
\[D_{2n}=\langle r,s\mid r^n=s^2=1,rs=sr^{-1}\rangle.\]

\begin{example}[Matrix groups]
For $n\in\NN$, let $GL_n(\FF)$ be the set of all $n\times n$ invertible matrices whose entries are in $\FF$:
\[GL_n(\FF)=\{A\in M_{n\times n}(\FF)\mid\det(A)\neq0\}.\]

We show that $GL_n(\FF)$ is a group under matrix multiplication; $GL_n(\FF)$ is the \vocab{general linear group} of degree $n$. 
Since $\det AB=\det A\det B$, if $\det A\neq0$ and $\det B\neq0$, then $\det AB\neq0$, so $GL_n(\FF)$ is closed under matrix multiplication.
\begin{enumerate}[label=(\roman*)]
\item Matrix multiplication is associative.
\item $\det(A)\neq0$ if and only if $A$ has an inverse matrix, so each $A\in GL_n(\FF)$ has an inverse $A^{-1}\in GL_n(\FF)$ such that
\[AA^{-1}=A^{-1}A=I\]
where $I$ is the $n\times n$ identity matrix.
\item Inverse
\end{enumerate}
\end{example}

\begin{example}[Quaternion group]
The \vocab{Quaternion group} $Q_8$ is defined by
\[Q_8=\{1,-1,i,-i,j,-j,k,-k\}\]
with product $\cdot$ computed as follows:
\begin{itemize}
\item $1\cdot a=a\cdot 1=a$ for all $a\in Q_8$
\item $(-1)\cdot(-1)=1$
\item $(-1)\cdot a=a\cdot(-1)=-a$ for all $a\in Q_8$
\item $i\cdot i=j\cdot j=k\cdot k=-1$
\item $i\cdot j=k$, $j\cdot i=-k$, $j\cdot k=i$, $k\cdot j=-i$, $k\cdot i=j$, $i\cdot k=-j$
\end{itemize}
Note that $Q_8$ is a non-abelian group of order $8$.
\end{example}

\begin{example}[Roots of unity]
Let $n\in\ZZ^+$. Consider the set of roots of unity
\[\mu_n=\{e^\frac{2k\pi i}{n}\mid k=0,\dots,n-1\}.\]
This forms an abelian group under multiplication, of order $n$.
\end{example}

\subsection{Subgroups}
When given a set with certain properties, it is natural to consider its subsets that inherit the same properties.

\begin{definition}[Subgroup]
Let $G$ be a group. We say that a non-empty $H\subset G$ is a \vocab{subgroup}\index{subgroup} of $G$, denoted by $H\le G$, if $H$ is a group under the restricted operation from $G$.
\end{definition}

Every group $G$ has two obvious subgroups: the group $G$ itself, and the \emph{trivial subgroup} $\{1\}$. A subgroup is a \emph{proper subgroup} if it is not one of those two.

\begin{example} \
\begin{itemize}
\item $(\QQ,+)$ is a subgroup of $(\RR,+)$.
\item The group of complex numbers of absolute value 1 is a subgroup of $\CC^\times$, under multiplication.
\item $\{1,-1\}$ is a subgroup of $\{1,-1,i,-i\}$, under multiplication.
\end{itemize}
\end{example}

According to the definition, to prove that $H$ is a subgroup of $G$, we need to make sure $H$ satisfies all group axioms. However, this is often tedious. Instead, there are some simplified criteria to decide whether $H$ is a subgroup.

\begin{lemma}
Let $G$ be a group. Then $H\le G$ if and only if
\begin{enumerate}[label=(\roman*)]
\item $1\in H$;\hfill(identity)
\item $ab\in H$ for all $a,b\in H$;\hfill(closure)
\item $a^{-1}\in H$ for all $a\in H$.\hfill(inverses)
\end{enumerate}
\end{lemma}

Humans are lazy, and the test above is still too complicated. We thus come up with an even simpler test:

\begin{lemma}[Subgroup criterion]
Let $G$ be a group. Then $H\le G$ if and only if
\begin{enumerate}[label=(\roman*)]
\item $H\neq\emptyset$;
\item $ab^{-1}\in H$ for all $a,b\in H$. 
\end{enumerate}
\end{lemma}

\begin{proof} \

\fbox{$\implies$} If $H\le G$, then we are done, by definition of subgroup.

\fbox{$\impliedby$} Check group axioms:
\begin{enumerate}[label=(\roman*)]
\item Since $H\neq\emptyset$, there exists $a\in H$. Then $1=aa^{-1}\in H$.
\item Since $1\in H$ and $a\in H$, then $a^{-1}=1a^{-1}\in H$.
\item For any $a,b\in H$, $a,b^{-1}\in H$, so by (ii), $a(b^{-1})^{-1}=ab\in H$.
\end{enumerate}
\end{proof}

The next result and its corollary show that the intersection of subgroups is a subgroup.

\begin{proposition}
Let $G$ be a group, $H,K\le G$. Then $H\cap K\le G$.
\end{proposition}

\begin{proof}
Apply the subgroup criterion:
\begin{enumerate}[label=(\roman*)]
\item Since $1\in H$ and $1\in K$, then $1\in H\cap K$ so $H\cap K\neq\emptyset$.
\item Let $a,b\in H\cap K$. Then $a,b\in H$ and $a,b\in K$. Since $H,K\le G$, by the subgroup criterion, $ab^{-1}\in H$ and $ab^{-1}\in K$, so $ab^{-1}\in H\cap K$.
\end{enumerate}
\end{proof}

\begin{corollary}
Let $G$ be a group, $\{H_i\mid i\in I\}$ is a collection of subgroups of $G$. Then
\[\bigcap_{i\in I}H_i\le G.\]
\end{corollary}

\begin{proposition}
Let $H,K\le G$. If $H\cup K\le G$, then either $H\subset K$ or $K\subset H$.
\end{proposition}

\begin{proof}
Suppose $H\cup K\le G$. Suppose, for a contradiction, that $H\not\subset K$ and $K\not\subset H$. 

Let $h\in H\setminus K$, $k\in K\setminus H$. Since $H\cup K\le G$, we have $hk\in H\cup K$.
\begin{itemize}
\item Suppose $hk\in H$, and let $h^\prime=hk$. Since $h\in H$ and $H\le G$, we have $h^{-1}\in H$. Thus $h^{-1}h^\prime=h^{-1}hk=k$. But $h^{-1}h^\prime\in H$ and $k\notin H$, which is a contradiction.   
\item Suppose $hk\in K$. Then similarly we will arrive at a contradiction.
\end{itemize}
Therefore, either $H\setminus K=\emptyset$ or $K\setminus H=\emptyset$. Equivalently, $H\subset K$ or $K\subset H$.
\end{proof}

There is a general way of obtaining subgroups from a group. 

\begin{definition}[Subgroup generated by subset]
Let $S\subset G$ be non-empty. Let $H$ be the set of elements of $G$ consisting of all products $x_1\cdots x_n$ such that $x_i\in S$ or $x_i^{-1}\in S$ for each $i$, and also containing the unit element.

We call $H$ the \vocab{subgroup generated} by $S$. We also say that $S$ is a set of \emph{generators} of $H$, and denote
\[H=\angbrac{S}.\]
\end{definition}

\begin{lemma*}
The subgroup generated by a subset is indeed a subgroup.
\end{lemma*}

Thus if elements $\{x_1,\dots,x_n\}$ form a set of generators for $G$, we write
\[G=\angbrac{x_1,\dots,x_n}.\]

\begin{example}
$1$ is a generator for $\ZZ$, since every integer can be written in the form
\[1+1+\cdots+1\]
or
\[-1-1-\cdots-1,\]
or it is the $0$ integer.
\end{example}

\subsection{Cyclic Groups}
We consider the subgroup generated by one element.

\begin{definition}[Cyclic subgroup]
The \vocab{cyclic subgroup} $H$ generated by $a\in G$ is the set of all powers of $a$:
\[H=\angbrac{a}=\{a^n\mid n\in\ZZ\}.\]
We say that $a$ is a \emph{generator} of $H$.

We say $G$ is \vocab{cyclic} if there exists $a\in G$ such that $G=\angbrac{a}$.
\end{definition}

We write $C_n$ for the cyclic group of order $n$:
\[C_n=\angbrac{a\mid a^n=1}.\]

\begin{lemma*}
$H=\angbrac{a}$ is a subgroup of $G$.
\end{lemma*}

\begin{proof} \
\begin{enumerate}[label=(\roman*)]
\item $H$ contains the identity $1=a^0$.
\item Let $a^n,a^m\in H$. Then $a^m a^n=a^{m+n}\in H$.
\item $(a^n)^{-1}=a^{-n}\in H$.
\end{enumerate}
\end{proof}

\begin{example} \
\begin{itemize}
\item $\ZZ$ is cyclic with generator $1$ or $-1$. It is \emph{the} infinite cyclic group.
\item The multiplicative group $\{1,-1\}$ is cyclic with generator $-1$.
\item $\ZZ/n\ZZ$ is cyclic, with all numbers coprime with $n$ as generators.
\item The multiplicative group $\{1,-1\}$ is cyclic of order 2.
\item The complex numbers $\{1,i,-1,-i\}$ form a cyclic group of order 4. The number $i$ is a generator.
\end{itemize}
\end{example}

\begin{remark}
A cyclic subgroup may have more than one generator. For example, if $a$ is a generator, then $a^{-1}$ is also a generator:
\[\{a^n\mid n\in\ZZ\}=\{(a^{-1})^n\mid n\in\ZZ\}.\] 
\end{remark}

\begin{lemma}
Cyclic groups are abelian.
\end{lemma}

\begin{proof}
Let $G$ be a cyclic group. For $a^i,a^j\in G$, we have $a^i a^j=a^{i+j}=a^j a^i$.
\end{proof}

\begin{proposition}
A subgroup of a cyclic group is cyclic.
\end{proposition}

\begin{proof}
Let $a\in G$, $H\le\langle a\rangle$. If $H=\{1\}$ then trivially $H$ is cyclic.

Suppose that $H$ contains some other element $b\neq1$. Then $b=a^n$ for some integer $n$. Since $H$ is a subgroup, $b^{-1}=a^{-n}\in H$. Since either $n$ or $-n$ is positive, we can assume $H$ contains positive powers of $a$ and $n>0$. Let $m$ be the smallest positive integer such that $a^m\in H$ (such an $m$ exist by the well-ordering principle).

\begin{claim}
$h=a^m$ is a generator for $H$.
\end{claim}

We need to show that every $h^\prime\in H$ can be written as a power of $h$. Since $h^\prime\in H$ and $H\le\langle a\rangle$, $h^\prime=a^k$ for some integer $k$. By the division algorithm, there exist integers $q,r$ such that $k=qm+r$ with $0\le r<m$. Hence
\[a^k=a^{qm+r}=(a^m)^q a^r=h^q a^r\]
so $a^r=a^k h^{-q}$. Since $a^k,h^{-q}\in H$, we must have $a^r\in H$. By the minimality of $m$, we must have $m=0$ and so $k=qm$. Hence
\[h^\prime=a^k=a^{qm}=h^q\]
and $H$ is generated by $h$.
\end{proof}

A corollary concerns all the subgroups of $\ZZ$.

\begin{corollary}\label{cor:subgroup-Z}
The subgroups of $\ZZ$ are exactly $n\ZZ$ for $n=0,1,2,\dots$.
\end{corollary}

\subsection{Order}
\begin{definition}[Order]
Let $G$ be a group, $a\in G$. If there is a positive integer $k$ such that $a^k=1$, then the \vocab{order} of $g$ is defined as
\[o(a)\colonequals\min\{m>0\mid a^m=1\}.\]
Otherwise we say that the order of $a$ is infinite.
\end{definition}

We have given two different meanings to the word ``order''. One is the order of a group and the other is the order of an element. Since mathematicians are usually (but not always) sensible, the name wouldn't be used twice if they weren't related. This is explained by the next result.

\begin{lemma}
For $a\in G$, $o(a)=|\angbrac{a}|$.
\end{lemma}

\begin{proof}
We consider the cases where $o(a)$ is finite or infinite.
\begin{description}
\item[Case 1: $o(a)=\infty$.] Then $a^n\neq a^m$ for all $n\neq m$; otherwise $a^{m-n}=1$. Thus $|\angbrac{a}|=\infty=o(a)$.

\item[Case 2: $o(a)<\infty$.] Suppose $o(a)=k$. Thus $a^k=1$. We now claim that $\angbrac{a}=\{1,a^1,a^2,\dots,a^{k-1}\}$. 

Note that $\angbrac{a}$ does not contain higher powers of $a$, since $a^k=1$ so higher powers will loop back to existing elements. There are also no repeating elements in the list provided since $a^m=a^n$ implies $a^{m-n}=1$. Hence $|\angbrac{a}|=k=o(a)$.
\end{description}
\end{proof}

\begin{lemma}
If $a\in G$ and $o(a)$ is finite, then $a^n=1$ if and only if $o(a)\mid n$.
\end{lemma}

\begin{proof} \

\fbox{$\impliedby$} Suppose $o(a)\mid n$. Then $n=ko(a)$ for some $k\in\ZZ$, so
\[a^n=\brac{a^{o(a)}}^k=1^k=1.\]
\fbox{$\implies$} Suppose $a^n=1$. By the division algorithm, there exists integers $q,r$ such that $n=qo(a)+r$, where $0\le r<o(a)$. Then
\[a^r=a^{n-qo(a)}=a^n\brac{a^{o(a)}}^{-q}=1.\]
By the minimality of $o(a)$, we must have $r=0$, and so $n=qo(a)$ implies $o(a)\mid n$.
\end{proof}

\begin{corollary}
Let $G$ be a cyclic group, $a\in G$. Then $a^k=a^m$ if and only if $m\equiv k\pmod{o(a)}$.
\end{corollary}
\pagebreak

\section{Homomorphisms and Isomorphisms}
In this section, we make precise the notion of when two groups ``look the same''; that is, they have the same group-theoretic structure. This is the notion of an \emph{isomorphism} between two groups.

\subsection{Definitions and Properties}
When we talk about functions between groups it makes sense to limit our scope to functions that preserve the group operation (morphisms in the category of groups). More precisely:

\begin{definition}[Homomorphism]
Let $(G,\ast)$ and $(H,\diamond)$ be groups. We say $\phi\colon G\to H$ is a \vocab{homomorphism}\index{homomorphism} if
\[\phi(x\ast y)=\phi(x)\diamond\phi(y)\quad(x,y\in G).\]
\end{definition}

When the group operations for $G$ and $H$ are understood, we omit them and simply write
\[\phi(xy)=\phi(x)\phi(y).\]

\begin{example} \
\begin{itemize}
\item Let $G$ be a commutative group. The map $x\mapsto x^{-1}$ from $G$ into itself is a homomorphism.
\item The map $z\mapsto|z|$ is a homomorphism from $\CC^\times$ to $\RR^+$.
\item The map $x\mapsto e^x$ is a homomorphism from $(\RR,+)$ to $(\RR^+,\times)$. Its inverse map, the logarithm, is also a homomorphism.
\end{itemize}
\end{example}

\begin{lemma}[Basic properties]
Let $\phi\colon G\to H$ be a homomorphism. Let $g\in G$, $n\in\ZZ$. Then
\begin{enumerate}[label=(\roman*)]
\item $\phi(1_G)=1_H$
\item $\phi(g^{-1})=\phi(g)^{-1}$
\item $\phi(g^n)=\phi(g)^n$
\end{enumerate}
\end{lemma}

\begin{proof} \
\begin{enumerate}[label=(\roman*)]
\item $\phi(1_G)=\phi(1_G 1_G)=\phi(1_G)\phi(1_G)$, then apply $\phi(1_G)^{-1}$ to both sides to get $\phi(1_G)=1_H$.

\item $\phi(g)\phi(g^{-1})=\phi(gg^{-1})=\phi(1_G)=1_H$.

\item Note more generally that we can show $\phi(g^n)=(\phi(g))^n$ for $n>0$ by induction. For $n=-k<0$ we have
\[\phi(g^n)=\phi((g^{-1})^k)=\phi(g^{-1})^k=(\phi(g)^{-1})^k=\phi(g)^n.\]
\end{enumerate}
\end{proof}

\begin{lemma}
Let $\phi\colon G\to H$ and $\psi\colon H\to K$ be homomorphisms. Then $\psi\circ\phi$ is a homomorphism.
\end{lemma}

\begin{proof}
We have
\[(\psi\circ\phi)(xy)=\psi\brac{\phi(xy)}=\psi\brac{\phi(x)\phi(y)}=\psi(\phi(x))\psi(\phi(y))=(\psi\circ\phi)(x)(\psi\circ\phi)(y).\]
\end{proof}

Let $\mathrm{Hom}(G,H)$ denote the set of homomorphisms from $G$ to $H$. 
Then $\mathrm{Hom}(G,H)$ is a group under addition. 
\begin{enumerate}[label=(\roman*)]
\item If $\phi,\psi\in\mathrm{Hom}(G,H)$, then for $x,y\in G$,
\begin{align*}
(\phi+\psi)(x+y)
&=\phi(x+y)+\psi(x+y)\\
&=\phi(x)+\psi(x)+\phi(y)+\psi(y)\\
&=(\phi+\psi)(x)+(\phi+\psi)(y),
\end{align*}
so that $f+g$ is a homomorphism.

\item If $\phi,\psi,\gamma\in\mathrm{Hom}(G,H)$, then for all $x\in G$,
\[\brac{(\phi+\psi)+\gamma}(x)=(\phi+\psi)(x)+\gamma(x)=\phi(x)+\psi(x)+\gamma(x),\]
and
\[\brac{\phi+(\psi+\gamma)}(x)=\phi(x)+(\psi+\gamma)(x)=\phi(x)+\phi(x)+\gamma(x).\]
Hence $(\phi+\psi)+\gamma=\phi+(\psi+\gamma)$.

\item The zero map is the identity element of $\mathrm{Hom}(G,H)$.

\item The inverse of $\phi\in\mathrm{Hom}(G,H)$ is $-\phi$ (which is a homomorphism).
\end{enumerate}

\begin{definition}[Isomorphism]
An \vocab{isomorphism}\index{isomorphism} is a bijective homomorphism. 
If there exists an isomorphism $\phi\colon G\to H$, we say $G$ and $H$ are \vocab{isomorphic}, denoted by $G\cong H$.
\end{definition}

An \emph{automorphism} of a group $G$ is an isomorphism from $G$ to $G$; the automorphisms of $G$ form a group $\Aut(G)$ under composition. An \emph{endomorphism} of $G$ is a homomorphism from $G$ to $G$.

\begin{example}
The exponential map $\exp\colon\RR\to\RR^+$ defined by $\exp(x)=e^x$ is an isomorphism from $(\RR,+)$ to $(\RR^+,\times)$.
\begin{enumerate}[label=(\roman*)]
\item $\exp$ is a bijection since it has an inverse function (namely $\ln$).
\item $\exp$ preserves the group operations since $e^{x+y}=e^xe^y$.
\end{enumerate}
Hence $(\RR,+)\cong(\RR^+,\times)$.
\end{example}

\subsection{Kernel and Image}
We introduce two important groups related to every homomorphism.

\begin{definition}[Kernel]
Let $\phi\colon G\to H$ be a homomorphism. The \vocab{kernel}\index{kernel} of $\phi$ is
\[\ker\phi\colonequals\{g\in G\mid \phi(g)=1_H\}.\]
\end{definition}

\begin{lemma*}
$\ker\phi\triangleleft G$.
\end{lemma*}

\begin{proof}
Apply the subgroup criterion. Since $1_G\in\ker\phi$, $\ker\phi\neq\emptyset$. Let $x,y\in\ker\phi$; that is, $\phi(x)=\phi(y)=1_H$. Then
\[\phi(xy^{-1})=\phi(x)\phi(y)^{-1}=1_H\]
so $xy^{-1}\in\ker\phi$. By the subgroup criterion, $\ker\phi\le G$.

Let $x\in\ker\phi$, $g\in G$. Then
\[\phi(gxg^{-1})=\phi(g)\phi(x)\phi(g^{-1})=1,\]
so $gxg^{-1}\in\ker\phi$. Hence $\ker\phi\triangleleft G$.
\end{proof}

\begin{definition}[Image]
Let $\phi\colon G\to H$ be a homomorphism. 
The \vocab{image}\index{image} of $G$ under $\phi$ is
\[\im\phi\colonequals\phi(G)=\{\phi(g)\mid g\in G\}.\]
\end{definition}

\begin{remark}
$\im\phi$ is the usual set theoretic image of $\phi$.
\end{remark}

\begin{lemma*}
$\im\phi\le H$.
\end{lemma*}

\begin{proof}
Since $\phi(1_G)=1_H$, $1_H\in\im\phi$ so $\im\phi\neq\emptyset$. Let $x,y\in\im\phi$. Then there exists $a,b\in G$ such that $\phi(a)=x$, $\phi(b)=y$. Then
\[xy^{-1}=\phi(a)\phi(b)^{-1}=\phi(ab^{-1})\]
so $xy^{-1}\in\im\phi$. By the subgroup criterion, $\im\phi\le G$.
\end{proof}

The following result is a useful characterisation for injective homomorphisms.

\begin{lemma}
Let $\phi\colon G\to H$ be a homomorphism. Then $\phi$ is injective if and only if $\ker\phi=\{1_G\}$.
\end{lemma}

\begin{proof} \

\fbox{$\implies$} Suppose $\phi$ is injective. Since $\phi(1_G)=1_H$, $1_G\in\ker\phi$ so $\{1_G\}\subset\ker\phi$. 

Conversely, let $x\in\ker\phi$, so $\phi(x)=1_H$. Then $\phi(x)=1_H=\phi(1_G)$, so by injectivity $x=1_G$. Hence $\ker\phi\subset\{1_G\}$, so $\ker\phi=\{1_G\}$.

\fbox{$\impliedby$} Suppose $\ker\phi=\{1_G\}$. Suppose $\phi(a)=\phi(b)$, then $\phi(ab^{-1})=\phi(a)\phi(b^{-1})=\phi(a)\phi(a)^{-1}=1_H$. Hence $ab^{-1}\in\ker\phi=\{1_G\}$, so $ab^{-1}=1_G$ and thus $a=b$. Therefore $\phi$ is injective.
\end{proof}

\begin{lemma}
Let $\phi\colon G\to H$ be an isomorphism. Then its inverse $\phi^{-1}\colon H\to G$ is an isomorphism.
\end{lemma}

\begin{proof}
The inverse of a bijective map is bijective. Hence it suffices to show that $\phi^{-1}(x)\phi^{-1}(y)=\phi^{-1}(xy)$ for all $x,y\in H$.

Let $a=\phi^{-1}(x)$, $b=\phi^{-1}(y)$, $c=\phi^{-1}(xy)$; we will show that $ab=c$. Since $\phi$ is bijective, it suffices to show that $\phi(ab)=\phi(c)$.

Since $\phi$ is a homomorphism,
\[\phi(ab)=\phi(a)\phi(b)=xy=\phi(c).\]
\end{proof}

\subsection{Cosets}
\begin{definition}[Coset]\index{coset}
Let $H\le G$. For $a\in G$, a \vocab{left coset}\index{coset!left coset} and \vocab{right coset}\index{coset!right coset} of $H$ in $G$ are
\begin{align*}
aH&\colonequals\{ah\mid h\in H\}\\
Ha&\colonequals\{ha\mid h\in H\}
\end{align*}
Any element of a coset is called a \emph{representative} for the coset.
\end{definition}

\begin{example}
Consider the subgroup $2\ZZ\le\ZZ$. Then $6+2\ZZ=\{\text{all even numbers}\}=0+2\ZZ$, and $1+2\ZZ=\{\text{all odd numbers}\}=17+2\ZZ$.
\end{example}

\begin{notation}
We denote the set of (left) cosets by $G/H$.
\end{notation}

In what will follow, the analogous results hold similarly for right cosets.

\begin{lemma}
Let $H\le G$. Then $aH=H$ if and only if $a\in H$.
\end{lemma}

\begin{proof} \

\fbox{$\implies$} Suppose $aH=H$. Then $ah\in H$ for some $h\in H$. Let $k=ah$, then $a=kh^{-1}\in H$.

\fbox{$\impliedby$} Let $a\in H$. Then $aH\subset H$.

Since $a^{-1}\in H$, $a^{-1}H\subset H$. Then $H=eH=(aa^{-1})H=a(a^{-1})H\subset aH$. Hence $aH=H$.
\end{proof}

The next result shows when two cosets are equal.

\begin{lemma}
Let $H\le G$, $a,b\in G$. Then $aH=bH$ if and only if $a^{-1}b\in H$.
\end{lemma}

\begin{proof}
\begin{align*}
aH=bH&\iff a^{-1}(aH)=a^{-1}bH\\
&\iff (a^{-1}a)H=(a^{-1}b)H\\
&\iff H=(a^{-1}b)H
\end{align*}
From the previous result, $H=(a^{-1}b)H$ if and only if $a^{-1}b\in H$.
\end{proof}

\begin{proposition}
Let $H\le G$. Then $G/H$ forms a partition of $G$.
\end{proposition}

We need to prove the following.
\begin{enumerate}[label=(\roman*)]
\item For all $a\in G$, $aH\neq\emptyset$.
\item $\bigcup_{a\in G}aH=G$.
\item For every $a,b\in G$, $aH\cap bH=\emptyset$ or $aH=bH$.
\end{enumerate}

\begin{proof} \
\begin{enumerate}[label=(\roman*)]
\item Since $H\le G$, $1\in H$. Thus for all $a\in G$, $a=a1\in aH$ so $aH\neq\emptyset$.
\item For all $a\in G$, $aH\subset G$, then $\bigcup_{a\in G}aH\subset G$. Note that $a\in G$ implies $a=ae\in aH$, and so $G=\bigcup_{a\in G}g\subset\bigcup_{a\in G}aH$. By double inclusion we are done.
\item If $aH\cap bH=\emptyset$, then we are done. If $aH\cap bH\neq\emptyset$ we need to show $aH=bH$. Let $x\in G$ such that $x\in aH\cap bH$. Then $x=ah_1=bh_2$ for $h_1,h_2\in H$ so $h_1=a^{-1}bh_2$. Notice that $a^{-1}b=h_1h_2^{-1}\in H$ and thus $aH=bH$.
\end{enumerate}
\end{proof}

The next result shows that the left cosets of $H$ partition $G$ into equal-sized parts.

\begin{lemma}
The cosets of $H$ in $G$ are the same size as $H$; that is, for all $a\in G$, $|aH|=|H|$.
\end{lemma}

\begin{proof}
Consider the mapping
\begin{align*}
f\colon H&\to aH\\
h&\mapsto ah 
\end{align*}
We will show that $f$ is bijective. 
\begin{itemize}
\item Let $h_1,h_2\in H$, then
\begin{align*}
f(h_1)=f(h_2)
&\implies ah_1=ah_2\\
&\implies a^{-1}ah_1=a^{-1}ah_2\\
&\implies h_1=h_2
\end{align*}
so $f$ is injective. 
\item Note that $f$ is surjective by the definition of $aH$. 
\end{itemize}
Since $f$ is bijective, $|H|=|aH|$.
\end{proof}

\subsection{Lagrange's Theorem}
\begin{definition}[Index]
Let $H\le G$. The \vocab{index}\index{index} of $H$ in $G$ is the number of left cosets of $H$ in $G$, denoted by $|G:H|$.
\end{definition}

Then $|G|=|G:1|$; that is, the order of $G$ is the index of the trivial subgroup in $G$.

\begin{theorem}[Lagrange's theorem]
Let $G$ be a finite group, $H\le G$. Then $|H|$ divides $|G|$; in particular,
\begin{equation}\label{eqn:counting-formula}
|G|=|H|\:|G:H|.
\end{equation}
\end{theorem}

\eqref{eqn:counting-formula} is known as the \emph{counting formula}.

\begin{proof}
Suppose that there are $|G:H|$ left cosets in total. Since the left cosets partition $G$, and each coset has size $|H|$, we have
\[|H|\:|G:H|=|G|.\]
\end{proof}

\begin{corollary}
The order of an element of a finite group divides the order of the group.
\end{corollary}

\begin{proof}
Consider the subgroup generated by $a$, which has order $o(a)$. Then by Lagrange's theorem, $o(a)$ divides $|G|$.
\end{proof}

\begin{corollary}
For any finite group $G$ and $a\in G$, $a^{|G|}=1$.
\end{corollary}

\begin{proof}
We know that $|G|=k\:o(a)$ for some $k\in\NN$. Then $a^{|G|}=\brac{a^{o(a)}}^k=1^k=1$.
\end{proof}

A special case of the above result is Fermat's little theorem, by taking $G=(\ZZ/p\ZZ)^\times$.
\begin{quote}
If $p$ is prime, and $a$ is any integer, then
\[a^p\equiv a\pmod p.\]
\end{quote}

\begin{corollary}
A group of prime order is cyclic.
\end{corollary}

\begin{proof}
Let $|G|=p$ be prime. Let $a\in G$, $a\neq1$. We will show that $G=\langle a\rangle$.

Since $o(a)\mid |G|=p$ and $o(a)>1$, we must have $o(a)=p$. Notice that this is also the order of $\langle a\rangle$. Since $G$ has order $p$, thus $\langle a\rangle=G$.
\end{proof}

This corollary classifies groups of prime order $p$. They form one isomorphism class: the class of the cyclic groups of order $p$.

The next result is of great interest in number theory. The \emph{Euler $\phi$-function} $\phi(n)$ is defined for all positive integers as follows:
\[\phi(n)=\begin{cases}
1&(n=1)\\
\text{number of positive integers less than $n$, relatively prime to $n$}&(n>1)
\end{cases}\]

\begin{theorem}[Euler]
If $n$ is a positive integer, and $a$ is coprime to $n$, then
\[a^{\phi(n)}\equiv1\pmod n.\]
\end{theorem}

\subsection{Counting Principle}
We generalise the notion of cosets, as defined earlier.

\begin{definition}
Let $H,K\le G$. Define
\[HK\colonequals\{hk\mid h\in H,k\in K\}.\]
\end{definition}

\begin{lemma}
Let $H,K\le G$. Then $HK\le G$ if and only if $HK=KH$.
\end{lemma}

\begin{proof} \

\fbox{$\impliedby$} Suppose $HK=KH$; that is, if $h\in H$ and $k\in K$, then $hk=k_1h_1$ for some $k_1\in K,h_1\in H$.

We now show that $HK$ is a subgroup of $G$:
\begin{enumerate}[label=(\roman*)]
\item $1\in H$ and $1\in K$, so $1\in HK$.
\item Let $x=hk\in HK$, $y=h^\prime k^\prime\in HK$. then
\[xy=hkh^\prime k^\prime.\]
Note that $kh^\prime\in KH=HK$, so $kh^\prime=h_2k_2$ for some $h_2\in H,k_2\in K$. Then
\[xy=h(h_2k_2)k^\prime=(hh_2) (k_2k^\prime)\in HK.\]
Thus $HK$ is closed.
\item Let $x\in HK$, then $x=hk$ for some $h\in H,k\in K$. Thus
\[x^{-1}=(hk)^{-1}=k^{-1}h^{-1}\in KH=HK,\]
so $x^{-1}\in HK$.
\end{enumerate}

\fbox{$\implies$} Suppose $HK\le G$. 
\begin{itemize}
\item Let $x\in KH$, so $x=kh$ for some $k\in K,h\in H$. Then
\[x=kh=(h^{-1}k^{-1})^{-1}\in HK.\]
Thus $KH\subset HK$.
\item Let $x\in HK$. Since $HK\le G$, $HK$ is closed under inverses, so $x^{-1}=hk\in HK$. Then
\[x=(x^{-1})^{-1}=(hk)^{-1}=k^{-1}h^{-1}\in KH.\]
Thus $HK\subset KH$.
\end{itemize}
Hence $HK=KH$.
\end{proof}

An interesting special case is the situation when $G$ is an abelian group, for in that case trivially $HK=KH$. Thus as a consequence we have the following result.

\begin{corollary}
Let $H,K\le G$, where $G$ is abelian. Then $HK\le G$.
\end{corollary}

\begin{proposition}
If $H,K\le G$ are finite groups, then
\[|HK|=\frac{|H||K|}{|H\cap K|}.\]
\end{proposition}

\begin{proof}
Notice that $HK$ is a union of left cosets of $K$, namely
\[HK=\bigcup_{h\in H}hK.\]

\end{proof}

\subsection{Normal Subgroups, Quotient Groups}
\begin{definition}[Normal subgroup]
Let $G$ be a group. We say $H\le G$ is a \vocab{normal subgroup}\index{normal subgroup} of $G$, denoted by $H\triangleleft G$, if
\[aH=Ha\quad(\forall a\in G)\]
\end{definition}

\begin{remark}
This does \emph{not} mean that $ah=ha$ for all $a\in G$, $h\in H$ or that $G$ is abelian; although we can easily see that all subgroups of abelian groups are normal. In general, a left coset does not equal the right coset.
\end{remark}

\begin{lemma}
The following are equivalent.
\begin{enumerate}[label=(\roman*)]
\item $H\triangleleft G$.
\item $ghg^{-1}\in H$ for all $g\in G$, $h\in H$.
\item $gHg^{-1}=H$ for all $g\in G$.
\end{enumerate}
\end{lemma}

\begin{proof} \

\fbox{(i)$\iff$(ii)} First suppose $aH=Ha$ for all $a\in G$. Let $g\in G$, $x\in H$. Then $gH=Hg$ so $gx=h^\prime g$ for some $h^\prime\in H$. Then $gxg^{-1}=h^\prime gg^{-1}=h^\prime\in H$.

Conversely suppose $ghg^{-1}\in H$ for all $g\in G$, $h\in H$. Fix $g$. Then $ghg^{-1}\in H$ implies $gh\in Hg$ for all $h\in H$. So $gH\subset Hg$. Similarly $gH\supset Hg$, so $gH=Hg$.

\fbox{(i)$\iff$(iii)} $H\triangleleft G$ if and only if for all $g\in G$,
\begin{align*}
gH=Hg&\iff(gH)g^{-1}=(Hg)g^{-1}\\
&\iff gHg^{-1}=H
\end{align*}
\end{proof}

\begin{remark}
We frequently use (ii) to check if a subgroup is a normal subgroup.
\end{remark}

\begin{lemma}\label{lemma:subgroup-of-abelian-group-is-normal}
A subgroup of an abelian group is normal.
\end{lemma}

\begin{proof}
Let $G$ be abelian, $H\le G$. 
For all $g\in G$, $h\in H$, we have $ghg^{-1}=gg^{-1}h=h\in H$. 
Thus $H$ is normal.
\end{proof}

\begin{lemma}
Every subgroup of index $2$ is normal.
\end{lemma}

\begin{proof}
Suppose $H\le G$ has index $2$. Then there are only two possible cosets, namely $H$ and $G\setminus H$.

Since $1H=H1$ and cosets partition $G$, the other left coset and right coset must be $G\setminus H$. Hence all left cosets and right cosets are the same.
\end{proof}

\begin{proposition}
A group of order $6$ is either cyclic or dihedral.
\end{proposition}

\begin{proof}
Let $|G|=6$. We will show that either $G\cong C_6$ or $G\cong D_6$.

By Lagrange's theorem, the possible element orders are $1$, $2$, $3$ and $6$. If there exists $a\in G$ of order $6$, then $G=\angbrac{a}\cong C_6$. 

Otherwise, we can only have elements of orders $2$ and $3$ other than the identity. If $G$ only has elements of order $2$, the order must be a power of $2$ (why), which is not the case.
So there must be an element $r$ of order $3$. So $\angbrac{r}\triangleleft G$ as it has index $2$. Now $G$ must also have an element $s$ of order $2$ (why).

Since $\angbrac{r}$ is normal, we know that $srs^{-1}\in\angbrac{r}$. If $srs^{-1}=1$, then $r=1$, which is not true. If $srs^{-1}=r$, then $sr=rs$ and $sr$ has order $6$ (lcm of the orders of $s$ and $r$), which was ruled out above. Otherwise if $srs^{-1}=r^2=r^{-1}$, then $G$ is dihedral by definition of the dihedral group.
\end{proof}

The (left) cosets of a group form a group, known as the \emph{quotient group}.

\begin{definition}[Quotient group]
Let $G$ be a group, $H\triangleleft G$. Then the \vocab{quotient group}\index{quotient group} of $G$ by $H$ is the set of left cosets of $H$ in $G$:
\[G/H\colonequals\{aH\mid a\in G\}.\]
\end{definition}

\begin{remark}
Quotient groups are not subgroups of $G$; they contain different kinds of elements. For example, $\ZZ/n\ZZ\cong C_n$ are finite, but all subgroups of $\ZZ$ infinite.
\end{remark}

\begin{lemma*}
$G/H$ is a group under the operation $aH\ast bH=(ab)H$.
\end{lemma*}

\begin{proof}
First show that the operation is well-defined; that is, if $aH=a^\prime H$ and $bH=b^\prime H$, we want to show that $aH\ast bH=a^\prime H\ast b^\prime H$. 

We know that $a^\prime=ak_1$ and $b^\prime=bk_2$ for some $k_1,k_2\in H$. Then $a^\prime b^\prime=ak_1bk_2$. We know that $b^{-1}k_1b\in H$. Let $b^{-1}k_1b=k_3$. Then $k_1b=bk_3$. So $a^\prime b^\prime=abk_3k_2\in(ab)H$. So picking a different representative of the coset gives the same product.

If $aH$ and $bH$ are cosets, then $(ab)H$ is also a coset, so the operation is closed.

\begin{enumerate}[label=(\roman*)]
\item For $a,b,c\in G$, by associativity of $G$,
\[(aH)(bHcH)=(aH)(bcH)=a(bc)H=(ab)cH=(aHbH)cH\]
so the operation is associative.
\item The identity is $1H=\{1h\mid h\in H\}=\{h\mid h\in H\}=H$.
\item The inverse of $aH$ is $a^{-1}H$, since
\[(aH)(a^{-1}H)=aa^{-1}H=H\implies(aH)^{-1}=a^{-1}H.\]
\end{enumerate}
\end{proof}

\begin{example}[Modular arithmetic]
Fix $n\in\ZZ^+$. 
Evidently $n\ZZ$ is a subgroup of $\ZZ$.
Then the quotient group $\ZZ/n\ZZ$ consists of cosets of the form
\[n\ZZ,1+n\ZZ,2+n\ZZ,\dots,n-1+n\ZZ.\]
If we consider each coset as an equivalence class, we write
\[\ZZ/n\ZZ=\{[0],[1],\dots,[n-1]\}.\]
Addition on $\ZZ/n\ZZ$ is defined as
\[[x]+[y]=[x+y].\]
\end{example}

The next result concerns the order of the quotient group.

\begin{lemma}
Let $G$ be a finite group, $H\triangleleft G$. Then
\[|G/H|=|G:H|=\frac{|G|}{|H|}.\]
\end{lemma}

\begin{proof}
Since $G/H$ has as its elements the left cosets of $H$ in $G$, and there are precisely $|G:H|$ such cosets, the first equality holds.

The second equality holds by Lagrange's theorem.
\end{proof}

We now define a \emph{canonical} homomorphism (``natural'' map) from a group to its quotient group.

\begin{definition}[Quotient map]
Let $H\triangleleft G$. The \vocab{quotient map}\index{quotient map} is the map
\begin{align*}
\pi\colon G&\to G/H\\
a&\mapsto aH
\end{align*}
\end{definition}

\begin{lemma}
Quotient maps are surjective homomorphisms.
\end{lemma}

\begin{proof}
Let $\pi\colon G\to G/H$ which maps $a\mapsto aH$ be a quotient map.
\begin{itemize}
\item For all $a,b\in G$,
\[\pi(ab)=(ab)H=(aH)(bH)=\pi(a)\pi(b).\]
Thus $\pi$ is a homomorphism.
\item For all $aH\in G/H$, $\pi(a)=aK$. Thus $\pi$ is surjective.
\end{itemize}
\end{proof}

The next result provides a characterisation of normal subgroups.

\begin{lemma}
$H\triangleleft G$ if and only if $H$ is the kernel of some homomorphism.
\end{lemma}

\begin{proof} \

\backward Suppose $H=\ker\phi$ for some homomorphism $\phi\colon G\to G^\prime$. 

Let $g\in G$, $h\in H$. Then
\[\phi(ghg^{-1})=\phi(g)\phi(h)\phi(g)^{-1}=\phi(h)=1.\]
Thus $ghg^{-1}\in\ker\phi=H$.

\forward The kernel of the quotient map is $H$ itself:
\[\ker\pi=\{a\in G\mid aH=H\}=\{a\in G\mid a\in H\}=H.\]
\end{proof}

\subsection{Isomorphism Theorems}
In this section, we will prove several isomorphism theorems.

\begin{theorem}[First isomorphism theorem]
Let $\phi\colon G\to H$ be a homomorphism. Then
\begin{equation}
G/\ker\phi\cong\im\phi.
\end{equation}
\end{theorem}

\begin{proof}
For ease of notation, denote $K=\ker\phi$. 
Consider the mapping
\begin{align*}
\theta\colon G/K&\to\im\phi\\
\forall x\in G,\quad xK&\mapsto\phi(x)
\end{align*}
We claim that $\theta$ is an isomorphism.
\begin{enumerate}
\item We check that $\theta$ is well-defined. Let $x,y\in G$, suppose $xK=yK$. Then
\begin{align*}
&xK=yK\\
\iff& x^{-1}y\in K\\
\iff&\phi(x^{-1}y)=1_H\\
\iff&\phi(x)^{-1}\phi(y)=1_H\\
\iff&\phi(x)=\phi(y)
\end{align*}

\item We show that $\theta$ is a homomorphism: for all $x,y\in G$,
\[\theta\brac{xKyK}=\theta(xyK)=\phi(xy)=\phi(x)\phi(y)=\theta(xK)\theta(yK).\]

\item We show that $\theta$ is bijective:
\begin{itemize}
\item $\theta$ is injective since
\[\theta(xK)=\theta(yK)\implies\phi(x)=\phi(y)\implies xK=yK.\]
\item $\theta$ is surjective, since
\[\im\theta=\{\theta(xK)\mid x\in G\}
=\{\phi(x)\mid x\in G\}
=\im\phi.\]
\end{itemize}
\end{enumerate}
\end{proof}

\begin{corollary}
Any cyclic group is isomorphic to either $\ZZ$ or $\ZZ/n\ZZ$ for some $n\in\NN$.
\end{corollary}

\begin{proof}
Let $G=\angbrac{g}$ for some $g\in G$. Define the mapping
\begin{align*}
\phi\colon\ZZ&\to G\\
m&\mapsto g^m
\end{align*}
We claim that $\phi$ is a surjective homomorphism.
\begin{enumerate}
\item $\phi$ is a homomorphism, since $\phi(m_1+m_2)=g^{m_1+m_2}=g^{m_1}g^{m_2}=\phi(m_1)\phi(m_2)$.
\item $\phi$ is surjective, since $G$ is by definition all $g^m$ for all $m$.
\end{enumerate}

By surjectivity, $\im\phi=G$.
We know that $\ker\phi\triangleleft\ZZ$. 
We have the following possibilities for the kernel:
\begin{description}
\item[Case 1: $\ker\phi=\{1\}$] This implies $\phi$ is injective, so $\phi$ is an isomorphism. Hence $G\cong\ZZ$.
\item[Case 2: $\ker\phi=\ZZ$] By the first isomorphism theorem, $G\cong\ZZ/\ZZ=\{1\}=C_1$.
\item[Case 3: $\ker\phi=n\ZZ$] (Since these are the only remaining proper subgroups of $\ZZ$.) By the first isomorphism theorem, $G\cong\ZZ/n\ZZ$.
\end{description}
\end{proof}

\begin{example}[Circle group]
Consider the subgroup $(\ZZ,+)$ of $(\RR,+)$. 
The quotient group $\RR/\ZZ$ is called the \emph{circle group}.

Define a congruence relation on $\RR$:
\[x\sim y\iff x-y\in\ZZ.\]
If $x\sim y$, we say $x,y\in\RR$ are \emph{congruent} mod $\ZZ$, and denote $x\equiv y\pmod\ZZ$. This congruence is an equivalence relation, and the congruence classes are precisely the cosets of $\ZZ$ in $\RR$.

If $x\equiv y\pmod\ZZ$, then $e^{2\pi ix}=e^{2\pi iy}$, and conversely.
Thus the map
\begin{align*}
f\colon\RR/\ZZ&\to\TT\\
x&\mapsto e^{2\pi ix}
\end{align*}
is an isomorphism, where $\TT=\{z\in\CC\mid|z|=1\}$ is the multiplicative group of complex numbers having absolute value 1.

\begin{remark}
$2\pi x$ can be considered as the angle measured from the positive real axis of $\CC$.
\end{remark}
\end{example}

\begin{example}
Let $\CC^\times$ be the multiplicative group of non-zero complex numbers, and $\RR^+$ the multiplicative group of positive real numbers.
Given a complex number $z\neq0$, we can write
\[z=ru,\]
where $r\in\RR^+$, and $u$ has absolute value 1. (Let $u=z/|z|$.) Such an expression is uniquely determined, and the map
\begin{align*}
f\colon\CC^\times&\to\TT\\
z&\mapsto\frac{z}{|z|}
\end{align*}
is a homomorphism. Since $\ker f=\RR^+$ and $\im f=\TT$ (by surjectivity), by the first isomorphism theorem, $\CC^\times/\RR^+$ is isomorphic to $\TT$.
\end{example}

\begin{theorem}[Second isomorphism theorem]
Let $H\le G$, $K\triangleleft G$. Then
\begin{equation}
HK/K\cong H/(H\cap K).
\end{equation}
\end{theorem}

We first prove a few preliminary results.

\begin{lemma*}
Let $H\le G$, $K\triangleleft G$. Then
\begin{enumerate}[label=(\roman*)]
\item $HK\le G$;
\item $K\triangleleft HK$;
\item $H\cap K\triangleleft H$.
\end{enumerate}
\end{lemma*}

\begin{proof} \
\begin{enumerate}[label=(\roman*)]
\item Since $1\in H$ and $1\in K$, we have $1\in HK$, so $HK\neq\emptyset$.
Let $hk,h^\prime k^\prime\in HK$. Then
\[h^\prime k^\prime(hk)^{-1}
=h^\prime k^\prime k^{-1}h^{-1}
=(\underbrace{h^\prime h^{-1}}_{\in H})(\underbrace{hk^\prime k^{-1}h^{-1}}_{\in K\text{, by normality}})\in HK.\]

By the subgroup criterion, $HK\le G$.

\item 

\item Since the intersection of subgroups is a subgroup, $H\cap K$ is a subgroup of $N$.
It remains to be shown that $H\cap K$ is normal in $H$.

Let $h\in H$, $x\in H\cap K$. We will show that $hxh^{-1}\in H\cap K$. 

$H\le G$ and $h\in H$ imply $h\in G$. Since $K\triangleleft G$, $x\in H\cap K$ and $h\in H$ imply $hxh^{-1}\in H$.
\end{enumerate}
\end{proof}

We are now ready to prove the theorem.

\begin{proof}
Define the map
\begin{align*}
\phi\colon H&\to G/K\\
h&\mapsto hK
\end{align*}
This is easily seen to be a homomorphism. Let $x,y\in H$, then
\[\phi(xy)=(xy)K=(xK)(yK)=\phi(x)\phi(y).\]
The kernel and image of $\phi$ are
\begin{align*}
\ker\phi&=\{h\in H\mid hK=K\}=\{h\in H\mid h\in K\}=H\cap K,\\
\im\phi&=\{\phi(h)\mid h\in H\}=\{hK\mid h\in H\}=HK/K.
\end{align*}
Hence the desired result follows from the first isomorphism theorem.
\end{proof}

\begin{theorem}[Third isomorphism theorem]
Let $H,K\triangleleft G$, $H\le K$. Then
\begin{equation}
(G/H)/(K/H)\cong G/K.
\end{equation}
\end{theorem}

\begin{lemma*}
$K/H\triangleleft G/H$.
\end{lemma*}

\begin{proof}
We first show $K/H\le G/H$:
\begin{enumerate}[label=(\roman*)]
\item The identity of $G/H$ is $H$, which is also the identity of $K/H$, since $1\in K$.
\item Let $aH,bH\in K/H$. Since $ab\in K$ for all $a,b\in K$, we have $(aH)(bH)=(ab)H\in K/H$.
\item Let $aH\in K/H$. Since $a^{-1}\in K$, we have $a^{-1}H\in K/H$.
\end{enumerate}
To show normality, let $gH\in G/H$, $kH\in K/H$. Then $(gH)(kH)(gH)^{-1}=(gkg^{-1})H$. Since $K\triangleleft G$, $gkg^{-1}\in K$. Thus $(gkg^{-1})H\in K/H$.
\end{proof}

We can now prove the theorem.

\begin{proof}
Define the \emph{canonical} homomorphism
\begin{align*}
\phi\colon G/H&\to G/K\\
gH&\mapsto gK
\end{align*}
We claim that $\phi$ is a surjective homomorphism.
\begin{enumerate}
\item We check that $\phi$ is well-defined: If $gH=g^\prime H$, then $g^{-1}g^\prime\in H$. Since $H\subset K$, $g^{-1}g^\prime\in K$. Thus $gK=g^\prime K$.

\item $\phi$ is a homomorphism:
\[\phi(gH\cdot g^\prime H)=\phi(gg^\prime H)=gg^\prime K=(gK)(g^\prime K)=\phi(gH)\phi(g^\prime H).\]

\item $\phi$ is clearly surjective, since any coset $gK$ is the image $\phi(gH)$.
\end{enumerate}

The kernel and image of $\phi$ are
\begin{align*}
\ker\phi&=\{gH\mid gK=K\}=\{gH\mid g\in K\}=K/H,\\
\im\phi&=G/K\quad\text{by surjectivity}.
\end{align*}
Hence the conclusion follows from the first isomorphism theorem.
\end{proof}

We now discuss an isomorphism theorem concerning pre-images of groups.

\begin{theorem}\label{thrm:isomorphism-pre-image}
Let $\phi\colon G\to G^\prime$ be a surjective homomorphism. Let $H^\prime\triangleleft G^\prime$ and $H=\phi^{-1}(H^\prime)$. Then
\begin{equation}
G/H\cong G^\prime/H^\prime.
\end{equation}
\end{theorem}

\begin{lemma*}
$H=\phi^{-1}(H^\prime)\triangleleft G$.
\end{lemma*}

\begin{proof}
We first show $H\le G$.
\begin{enumerate}[label=(\roman*)]
\item Since $H^\prime\le G^\prime$, $1_{G^\prime}\in H^\prime$. Then $\phi(1_G)=1_{G^\prime}\in H^\prime$, so $1_G\in\phi^{-1}(H^\prime)$.
\item Let $a,b\in H$. Then $\phi(a),\phi(b)\in H^\prime$. By closure, $\phi(a)\phi(b)=\phi(ab)\in H^\prime$, so $ab\in\phi^{-1}(H^\prime)$.
\item Let $a\in H$. Then $\phi(a)\in H^\prime$. Since $H^\prime$ is closed under inverses, $\phi(a)^{-1}=\phi(a^{-1})\in H^\prime$, so $a^{-1}\in\phi^{-1}(H^\prime)$.
\end{enumerate}
To show normality, let $g\in G$, $h\in H$, then $\phi(h)\in H^\prime$. Since $\phi(ghg^{-1})=\phi(g)\phi(h)\phi(g)^{-1}$, where $\phi(g)\in G^\prime$ and $\phi(h)\in H^\prime$, by normality $\phi(ghg^{-1})\in H^\prime$. Thus $ghg^{-1}\in\phi^{-1}(H^\prime)$.
\end{proof}

We now prove the theorem.

\begin{proof}
Consider the map
\begin{align*}
\psi\colon G&\to G^\prime/H^\prime\\
g&\mapsto\phi(g)H^\prime
\end{align*}
Note that $\psi$ can also be described as the composite map $\psi=\pi\circ\phi$:
\[G\xrightarrow{\phi}G^\prime\xrightarrow{\pi}G^\prime/H^\prime\]
where $\pi\colon G^\prime\to G^\prime/H^\prime$ is the quotient map.

We claim that $\psi$ is a surjective homomorphism.
\begin{enumerate}
\item The composition of homomorphisms is a homomorphism, so $\psi$ is a homomorphism.
\item The composition of surjective maps is surjective, so $\psi$ is surjective.
\end{enumerate}

The kernel and image of $\psi$ are
\begin{align*}
\ker\psi&=\{g\in G\mid\psi(g)=H^\prime\}=\{g\in G\mid\phi(g)H^\prime=H^\prime\}\\
&=\{g\in G\mid\phi(g)\in H^\prime\}=\phi^{-1}(H^\prime)=H,\\
\im\psi&=G^\prime/H^\prime\quad\text{by surjectivity}.
\end{align*}

Hence the desired conclusion follows from the first isomorphism theorem.
\end{proof}

\begin{theorem}[Fourth isomorphism theorem]
Let $N\triangleleft G$. The canonical projection homomorphism $G\to G/N$ defines a bijective correspondence between the set of subgroups of $G$ containing $N$ and the set of (all) subgroups of $G/N$. 
Under this correspondence normal subgroups correspond to normal subgroups.
\end{theorem}

\subsection{Solvable Groups}
\begin{definition}[Solvable group]
A group $G$ is \vocab{solvable}\index{solvable} if there exists a sequence of normal subgroups (known as a \emph{composition series})
\[G=H_0\triangleright H_1\triangleright\cdots\triangleright H_n=\{1\},\]
such that $H_i/H_{i+1}$ (this quotient is called a \emph{composition factor}) is abelian.
\end{definition}

\begin{proposition}\label{prop:solvable-group-quotient}
Let $K\triangleleft G$. If $K$ and $G/K$ are solvable, then $G$ is solvable. 
\end{proposition}

\begin{proof}
By definition, and the assumption that $K$ is solvable, it suffices to prove the existence of a sequence of normal subgroups
\[G=H_0\triangleright H_1\triangleright\cdots\triangleright H_n=K\]
such that $H_i/H_{i+1}$ is abelian. Let $\bar{G}=G/K$. By assumption, there exists a sequence of normal subgroups
\[\bar{G}=\bar{H}_0\triangleright\bar{H}_1\triangleright\cdots\triangleright\bar{H}_n=\{\bar{1}\}\]
such that $\bar{H}_i/\bar{H}_{i+1}$ is abelian.

Consider the quotient map
\begin{align*}
\pi\colon G&\to\bar{G}\\
g&\mapsto gK
\end{align*}
and let $H_i=\pi^{-1}(\bar{H}_i)$.
\begin{claim}
These $H_i$ comprise a composition series for $G$.
\end{claim}
Since the quotient map $\pi$ is a surjective homomorphism, by \ref{thrm:isomorphism-pre-image}, we have an isomorphism
\[H_i/H_{i+1}\cong\bar{H}_i/\bar{H}_{i+1}\]
and $K=\pi^{-1}(\bar{H}_n)$, so we have found the sequence of subgroups of $G$ as we wanted, proving the result.
\end{proof}

\begin{proposition}
Let $H\le G$. If $G$ is solvable, then $H$ is solvable.
\end{proposition}

\begin{proof}
Since $G$ is solvable, there exists a sequence of normal subgroups
\[G=H_0\triangleright H_1\triangleright\cdots\triangleright H_n=\{1\}\]
such that $H_i/H_{i+1}$ is abelian.
Consider
\[H=H\cap H_0\triangleright\cdots\triangleright H\cap H_n=\{1\}.\]
Now $(H\cap H_i)\cap H_{i+1}=H\cap H_{i+1}$. 
Since $H_i\triangleright H_{i+1}$, this implies $H\cap H_i\triangleright H\cap H_{i+1}$ by looking at the conjugacy relationship. By \ref{thrm:isomorphism-pre-image},
\[(H\cap H_i)/(H\cap H_{i+1})\cong(H\cap H_i)H_{i+1}/H_{i+1}\le H_i/H_{i+1}\]
which is abelian, by the following lemma:

\begin{lemma*}
Let $G$ be abelian, and $\phi\colon G\to G^\prime$ be a surjective homomorphism. Then $G^\prime$ is abelian.
\end{lemma*} 

\begin{proof}
Let $x,y\in G^\prime$. We can write $x=\phi(a)$ and $y=g(b)$, by surjectivity of $\phi$. Thus $xy=\phi(a)\phi(b)=\phi(ab)=\phi(ba)=\phi(b)\phi(a)=yx$. 
\end{proof}
\end{proof}

\begin{proposition}
If $G$ is solvable and $\phi\colon G\to G^\prime$ is a surjective homomorphism, then $G^\prime$ is solvable.
\end{proposition}

\begin{proof}
Since $G$ is solvable, there exists a sequence of normal subgroups
\[G=H_0\triangleright H_1\triangleright\cdots\triangleright H_n=\{1\}\]
such that $H_i/H_{i+1}$ is abelian.
Let $H_i^\prime=\phi(H_i)$. Clearly $H_0^\prime=\phi(G)=G^\prime$ (by surjectivity), and $H_n^\prime=\phi(\{1\})=\{1^\prime\}$. 

Let $y\in H_{i+1}^\prime$, and $y=\phi(a)$ for some $a\in H_{i+1}$. 
Since $H_{i+1}\subset H_i$, $a\in H_i$ and $y\in H_i^\prime$. 
Thus $H_{i+1}^\prime\le H_i^\prime$. 
To show normality, let $x=f(b)\in H_i^\prime$ where $b\in H_i$. 
Then $xyx^{-1}=\phi(bab^{-1})$. Since $H_{i+1}\triangleleft H_i$, $bab^{-1}\in H_i$, so $\phi(bab^{-1})\in H_{i+1}^\prime$. 
Thus $H_{i+1}^\prime\triangleleft H_i^\prime$.

Consider the map
\begin{align*}
\psi\colon H_i/H_{i+1}&\to H_i^\prime/H_{i+1}^\prime\\
hH_{i+1}&\mapsto\phi(h)H_{i+1}^\prime
\end{align*}
We claim that $\psi$ is a surjective homomorphism.
\begin{enumerate}
\item We first check that $\psi$ is well-defined. If $h_1H_{i+1}=h_2H_{i+1}$, then $h_1h_2^{-1}\in H_{i+1}$, so $\phi(h_1)\phi(h_2)^{-1}\in H_{i+1}^\prime$. Thus $\phi(h_1)H_{i+1}^\prime=\phi(h_2)H_{i+1}^\prime$.
\item $\psi$ is a homomorphism, since
\[\psi(h_1H_{i+1}h_2H_{i+1})=\psi(h_1h_2H_{i+1})=\phi(h_1h_2)H_{i+1}^\prime=\phi(h_1)H_{i+1}^\prime\phi(h_2)H_{i+1}^\prime=\psi(h_1H_{i+1})\psi(h_2H_{i+1}).\]
\item If $h^\prime H_{i+1}^\prime\in H_i^\prime/H_{i+1}^\prime$, then since $h^\prime\in H_i^\prime$, $h^\prime=\phi(h)$ for some $h\in H_i$ and $\psi(h H_{i+1})=h^\prime H_{i+1}^\prime$. Thus $\psi$ is surjective.
\end{enumerate}
By the lemma in the proof of the previous result, $H_i^\prime/H_{i+1}^\prime$ is abelian.
\end{proof}

In a sense, the objects having the ``simplest'' structure are the building blocks for the more complicated objects. 
For groups, these are the \emph{simple groups}.
All finite simple groups have been classified under the \href{https://en.wikipedia.org/wiki/ATLAS_of_Finite_Groups}{ATLAS of Finite Groups}.

\begin{definition}[Simple group]
A group is \vocab{simple} if it has no non-trivial proper normal subgroup
\end{definition}

That is, $G$ is simple if the only normal subgroups are $\{1\}$ and $G$.

\begin{example}
For prime $p$, the cyclic group $C_p$ is simple, since it has no proper subgroups at all, let alone normal ones. 
\end{example}

Let G be a finite group. Then one can find a sequence of normal subgroups 
\[G=H_0\triangleright H_1\triangleright\cdots\triangleright H_n=\{1\},\]
such that $H_i/H_{i+1}$ is simple.
(This follows from a previous result and the third isomorphism theorem.)

A group may have more than one composition series. However, the Jordan--H\"{o}lder theorem states that any two composition series of a given group are equivalent. That is, they have the same composition length and the same composition factors, up to permutation and isomorphism.

\begin{theorem}[Jordan--H\"{o}lder theorem]
Let $G$ be a finite group. Consider two composition series
\[G=H_0\triangleright H_1\triangleright\cdots\triangleright H_n=\{1\}\]
\[G=K_0\triangleright K_1\triangleright\cdots\triangleright K_m=\{1\}\]
where each composition factor is simple.
Then $n=m$, and the list of composition factors is unique up to permutation.
\end{theorem}

\begin{comment}
If we have $H\triangleleft G$ with $H\neq G$ or $\{1\}$, then we can ``quotient out'' $G$ into $G/H$. If $G/H$ is not simple, repeat. Then we can write $G$ as an ``inverse quotient'' of simple groups.

In general, simple groups are complicated. However, if we only look at abelian groups, then life is simpler. Note that by commutativity, the normality condition is always trivially satisfied. So any subgroup is normal. Hence an abelian group can be simple only if it has no non-trivial subgroups at all.

\begin{lemma}
An abelian group is simple if and only if it is isomorphic to the cyclic group $C_p$ for some prime number $p$.
\end{lemma}

\begin{proof}
By Lagrange's theorem, any subgroup of $C_p$ has order dividing $|C_p|=p$.
Hence if $p$ is prime, then it has no such divisors, and any subgroup must have order $1$ or $p$, i.e., it is either $\{1\}$ or $C_p$ itself. Hence in particular any normal subgroup must be $\{1\}$ or $C_p$. So it is simple.

Now suppose $G$ is abelian and simple. Let $1\neq g\in G$ be a non-trivial element, and consider $H=\{\dots,g^{-2},g^{-1},1,g,g^2,\dots\}$. Since $G$ is abelian, conjugation does nothing, and every subgroup is normal. So $H$ is a normal subgroup. Since $G$ is simple, $H=\{1\}$ or $H=G$. Since it contains $g\neq1$, it is non-trivial. So we must have $H=G$. So $G$ is cyclic.

If $G$ is infinite cyclic, then it is isomorphic to $\ZZ$. But $\ZZ$ is not simple, since $2\ZZ\triangleleft\ZZ$. So $G$ is a finite cyclic group, i.e., $G\cong C_m$ for some finite $m$.

If $n\mid m$, then $g^{m/n}$ generates a subgroup of $G$ of order $n$. So this is a normal subgroup. Therefore $n$ must be $m$ or $1$. Hence $G$ cannot be simple unless $m$ has no divisors except $1$ and $m$, i.e., $m$ is a prime.
\end{proof}

One reason why simple groups are important is the following:

\begin{theorem}
Let $G$ be any finite group. Then there are subgroups
\[G=H_1\triangleright H_2\triangleright\cdots\triangleright H_n=\{1\}\]
such that $H_i/H_{i+1}$ is simple.
\end{theorem}

\begin{proof}
If $G$ is simple, let $H_2=\{1\}$. Then we are done.

If $G$ is not simple, let $H_2$ be a maximal proper normal subgroup of $G$. We now claim that $G/H_2$ is simple.

If $G/H_2$ is not simple, it contains a proper non-trivial normal subgroup $L\triangleleft G/H_2$ such that $L\neq\{1\}$, $G/H2$. However, there is a correspondence between normal subgroups of $G/H_2$ and normal subgroups of $G$ containing $H_2$. So $L$ must be $K/H_2$ for some $K\triangleleft G$ such that $K\ge H_2$. Moreover, since $L$ is non-trivial and not $G/H_2$, we know $K$ is not $G$ or $H_2$. So $K$ is a larger normal subgroup. Contradiction.

So we have found an $H_2\triangleleft G$ such that $G/H_2$ is simple. Iterating this process on $H_2$ gives the desired result. Note that this process eventually stops, as $H_{i+1}<H_i$, and hence $|H_{i+1}|<|H_i|$, and all these numbers are finite.
\end{proof}
\end{comment}
\pagebreak

\section{Symmetric Group}
Let $S$ be a non-empty set. A bijection $\sigma\colon S\to S$ is called a \emph{permutation} of $S$; the set of permutations of $S$ is denoted by $\Sym(S)$.

\begin{lemma*}
$\Sym(S)$ forms a group under function composition $\circ$.
\end{lemma*}

We call $\Sym(S)$ the \vocab{symmetric group} on $S$.

\begin{proof}
If $\sigma\colon S\to S$ and $\tau\colon S\to S$ are bijections, then the composition $\sigma\circ\tau$ is a bijection from $S$ to $S$. Thus $\circ$ is a binary operation on $\Sym(S)$.
\begin{enumerate}[label=(\roman*)]
\item Composition of functions is associative, so $\circ$ is associative.
\item The identity of $\Sym(S)$ is the identity map.
\item Every bijection has a bijective inverse.
\end{enumerate}
\end{proof}

In the special case where $S=\{1,2,\dots,n\}=J_n$, the symmetric group on $S$ is called the \emph{symmetric group of degree $n$}, and we denote it by $S_n$.

\begin{lemma}
$|S_n|=n!$
\end{lemma}

\begin{proof}
Obvious, since there are $n!$ permutations of $\{1,2,\dots,n\}$.

There are $n$ choices for $\sigma(1)$, $n-1$ choices for $\sigma(2)$, ..., $1$ choice for $\sigma(n)$. Hence $|S_n|=n(n-1)\cdots1=n!$.
\end{proof}

There are two ways to denote a permutation (an element of the symmetric group). The first is the \emph{two row notation}: if $\sigma\in S_n$, we write
\[\sigma=\begin{pmatrix}
1&2&3&\cdots&n\\
\sigma(1)&\sigma(2)&\sigma(3)&\cdots&\sigma(n)
\end{pmatrix}.\]

\begin{lemma}
If $|S|\ge3$, then $\Sym(S)$ is non-abelian.
\end{lemma}

\begin{proof}
$S_3$ consists of
\[\begin{pmatrix}
1&2&3\\
1&2&3
\end{pmatrix},\quad
\begin{pmatrix}
1&2&3\\
2&3&1
\end{pmatrix},\quad
\begin{pmatrix}
1&2&3\\
3&1&2
\end{pmatrix},
\]
\[
\begin{pmatrix}
1&2&3\\
2&1&3
\end{pmatrix},\quad
\begin{pmatrix}
1&2&3\\
3&2&1
\end{pmatrix},\quad
\begin{pmatrix}
1&2&3\\
1&3&2
\end{pmatrix}.
\]
By considering the composition of any two of the above permutations, we see that they do not commute. Thus $S_3$ is not abelian. 

For $n\ge3$, since we can view $S_3$ as a subgroup of $S_n$ by fixing $4,5,6,\dots,n$, it follows that $S_n$ is not abelian.
\end{proof}

\begin{theorem}[Cayley's theorem]
Every finite group is isomorphic to some subgroup of some symmetric group.
\end{theorem}

\begin{proof}
Let $G$ be a finite group.
For $g\in G$, $\sigma_g(h)=gh$ defines a permutation on $G$, and $\sigma_{g_1}\sigma_{g_2}(h)=\sigma_{g_1}(g_2 h)=g_1g_2h=\sigma_{g_1g_2}(h)$.
\end{proof}

A \vocab{transposition} $\tau$ is a permutation which interchanges two numbers and leaves the others fixed, i.e., there exist distinct $i,j\in J_n$ such that $\tau(i)=j$, $\tau(j)=i$, and $\tau(k)=k$ if $k\neq i$, $k\neq j$.

One sees at once that if $\tau$ is a transposition, then $\tau^{-1}=\tau$ and $\tau^2=\id$. 
In particular, the inverse of a transposition is a transposition.
We shall prove that the transpositions generate $S_n$.

\begin{proposition}
Every permutation in $S_n$ can be expressed as a product of transpositions.
\end{proposition}

\begin{proof}
Induct on $n$.
For $n=1$, there is nothing to prove since there is only one element.
Let $n>1$ and assume the assertion proved for $n-1$.

Let $\sigma\in S_n$. Let $\sigma(n)=k$. Let $\tau\in S_n$ be such that $\tau(k)=n$, $\tau(n)=k$. Then $\tau\sigma$ is a permutation such that
\[\tau\sigma(n)=\tau(k)=n.\]
In other words, $\tau\sigma$ leaves $n$ fixed. We may therefore view $\tau\sigma$ as a permutation of $J_{n-1}$, and by induction, there exist transpositions $\tau_1,\dots,\tau_s\in S_{n-1}$, leaving $n$ fixed, such that
\[\tau\sigma=\tau_1\cdots\tau_s.\]
We now write
\[\sigma=\tau^{-1}\tau_1\cdots\tau_s,\]
thereby proving our proposition.
\end{proof}

The two row notation is clumsy to write and wastes a lot of space. Hence we often use the \emph{cycle notation}. Let $a_1,\dots,a_k$ be distinct integers in $J_n$. By the symbol
\[(a_1\cdots a_k)\]
we shall mean the permutation $\sigma$ such that 
\[\sigma(a_1)=a_2,\quad\sigma(a_2)=a_3,\quad\dots,\quad\sigma(a_k)=a_1,\]
and $\sigma$ leaves all other integers fixed. We call $(a_1,a_2,\dots,a_k)$ a \vocab{$k$-cycle}. (Thus a transposition is a $2$-cycle.)

\begin{example}
$(132)$ denotes the permutation $\sigma$ such that $\sigma(1)=3$, $\sigma(3)=2$, $\sigma(2)=1$, and $\sigma$ leaves all other integers fixed.
\end{example}

If $\sigma=(a_1\cdots a_k)$ is a cycle, then one verifies at once that the inverse $\sigma^{-1}$ is also a cycle, and 
\[\sigma^{-1}=(a_k\cdots a_1).\]

A product of cycles is easily determined, as illustrated by the following example.

\begin{example}
$(132)(34)=(2134)$. One sees this using the definition: If $\sigma=(132)$ and $\tau=(34)$. then
\begin{align*}
\sigma(\tau(3))&=\sigma(4)=4,\\
\sigma(\tau(4))&=\sigma(3)=2,\\
\sigma(\tau(2))&=\sigma(2)=1,\\
\sigma(\tau(1))&=\sigma(1)=3.
\end{align*}
\end{example}

Two cycles are said to be \emph{disjoint} if no number appears in both cycles.

\begin{lemma}
Disjoint cycles commute.
\end{lemma}

\begin{proof}
Suppose $\sigma,\tau\in S_n$ are disjoint cycles. We will show that $\sigma(\tau(a))=\tau(\sigma(a))$. 

If $a$ is in neither of $\sigma$ and $\tau$, then $\sigma(\tau(a))=\tau(\sigma(a))=a$. 

Otherwise, WLOG assume that $a$ is in $\tau$ but not in $\sigma$. Then $\tau(a)\in\tau$ and thus $\tau(a)\notin\sigma$. Thus $\sigma(a)=a$ and $\sigma(\tau(a))=\tau(a)$. Hence we have $\sigma(\tau(a))=\tau(\sigma(a))=\tau(a)$.

Therefore $\tau$ and $\sigma$ commute.
\end{proof}

We shall prove that for $n\ge 5$, the group $S_n$ is not solvable. We need some prehminaries. 

\begin{lemma*}
Let $H\triangleleft G$. Then $G/H$ is abelian if and only if $H$ contains all elements of the form $xyx^{-1}y^{-1}$, where $x,y\in G$. 
\end{lemma*}

\begin{proof} \

\forward Consider the quotient map $\pi\colon G\to G/H$.
Suppose $G/H$ is abelian.
For any $x,y\in G$, we have
\[\pi(xyx^{-1}y^{-1})=\pi(x)\pi(y)\pi(x)^{-1}\pi(y)^{-1}=H,\]
since $G/H$ is abelian. Hence $xyx^{-1}y^{-1}\in H$.

\backward Suppose $H$ contains all elements of the form $xyx^{-1}y^{-1}$, where $x,y\in G$.

Let $\bar{x},\bar{y}\in G/H$. Since $\pi$ is surjective, there exist $x,y\in G$ such that $\bar{x}=\pi(x)$, $\bar{y}=\pi(y)$. 

Let $\bar{1}$ denote the identity of $G/H$, and $1$ denote the identity of $G$. 
Then
\[\bar{1}=\pi(1)=\pi(xyx^{-1}y^{-1})=\pi(x)\pi(y)\pi(x)^{-1}\pi(y)^{-1}=\bar{x}\bar{y}\bar{x}^{-1}\bar{y}^{-1}.\]
Multiplying by $\bar{y}$ and then $\bar{x}$ on the right, we find
\[\bar{y}\bar{x}=\bar{x}\bar{y}.\]
Hence $G/H$ is abelian.
\end{proof}

\begin{theorem}
If $n\ge 5$, then $S_n$ is not solvable.
\end{theorem}

\begin{proof}
We need the following result.
\begin{lemma*}
Let $N\triangleleft H\le S_n$. If $H$ contains every 3-cycle and $H/N$ is abelian, then $N$ contains every 3-cycle.
\end{lemma*}

\begin{proof}
Let $i,j,k,r,s$ be distinct integers in $\{1,\dots,n\}$, and let
\[\sigma=(ijk),\quad\tau=(krs).\]
Then
\begin{align*}
\sigma\tau\sigma^{-1}\tau^{-1}
&=(ijk)(krs)(kji)(srk)\\
&=(rki).
\end{align*}
Since $H$ contains every 3-cycle, $\sigma,\tau\in H$. Since $H/N$ is abelian, by the above lemma, $N$ contains all elements of the form $\sigma\tau\sigma^{-1}\tau^{-1}$. Thus $\sigma\tau\sigma^{-1}\tau^{-1}\in N$. 

Since the choice of $i,j,k,r,s$ was arbitrary, this implies $\sigma\tau\sigma^{-1}\tau^{-1}$ is an arbitrary 3-cycle. Hence $N$ contains every 3-cycle.
\end{proof}

$S_n$ contains all 3-cycles. Thus by induction on the previous lemma, a composition series
\[S_n=H_0\triangleright H_1\triangleright H_2\triangleright\cdots\triangleright H_n\]
must have $H_n$ containing all 3-cycles; thus $H_n\neq\{1\}$ (since the trivial subgroup does not contain any 3-cycles).
\end{proof}
\pagebreak

\section{Group Actions}
\subsection{Group Acting on Sets}
We move now, from thinking of groups in their own right, to thinking of how groups can move sets around; for example, $S_n$ permutes $\{1,2,\dots,n\}$, and matrix groups move vectors.
This leads to the notion of a \emph{group action}.

\begin{definition}[Group action]
Let $G$ be a group, $S$ be a set. An \vocab{action}\index{group action} of $G$ on $S$ is a map $G\times S\to S$ satisfying the following properties:
\begin{enumerate}[label=(\roman*)]
\item $g(hs)=(gh)s$ for all $g,h\in G$, $s\in S$;\hfill(associativity)
\item $1s=s$ for all $s\in S$.\hfill(identity)
\end{enumerate}
\end{definition}

\begin{notation}
If the group action $\cdot\colon G\times S\to S$ is not clear from context, we write $g\cdot s$ instead of $gs$.
\end{notation}

Intuitively, an action of $G$ on $S$ means that every element $g\in G$ acts as a permutation on $S$ in a manner consistent with the group operations in $G$.

There is another way of defining group actions, which is arguably a better way of thinking about group actions.

\begin{lemma}
An action of $G$ on $S$ is equivalent to a homomorphism $\phi\colon G\to\Sym(S)$.
\end{lemma}

Note that the statement by itself is useless, since it does not tell us how to translate between the homomorphism and a group action. The important part is the proof.

\begin{proof}
Let $(G,\ast)$ be a group.
Let $\cdot\colon G\times S\to S$ be an action. Define
\begin{align*}
\phi\colon G&\to\Sym(S)\\
g&\mapsto(g\cdot\ast\colon S\to S)
\end{align*}
This is indeed a permutation
\end{proof}

\begin{example} \
\begin{itemize}
\item The \emph{trivial action} is $gs=s$.
\item $S_n$ acts on $\{1,\dots,n\}$ by permutation.
\item $D_{2n}$ acts on the vertices of a regular $n$-gon (or the set $\{1,\dots,n\}$).
\end{itemize}
\end{example}

\begin{definition}[Kernel of action]
Let $G$ act on $S$. The \vocab{kernel} of the action is
\[\{g\in G\mid gs=s\:\forall s\in S\}.\]
\end{definition}

We say an action is \emph{faithful} if the kernel is just $\{1\}$.

\subsection{Orbits and Stabilisers}
\begin{definition}[Orbit]
Let $G$ act on $S$. The \vocab{orbit} of $s\in S$ is
\[Gs\colonequals\{gs\mid g\in G\}.\]
\end{definition}

Intuitively, it is the elements that $s$ can possibly get mapped to.
An element $x\in G_s$ in an orbit is called a \emph{representative} of the orbit, and we say that $x$ \emph{represents} the orbit.

In what follows, the formalism of orbits is similar to the formalism of cosets.

\begin{lemma}
If $s\in G$, then $Gs=G$.
\end{lemma}

\begin{proof}
Suppose $s\in G$. Then
\begin{align*}
x\in Gs
&\iff x=gs,\:g\in G\\
&\iff x\in G
\end{align*}
\end{proof}

The next result states when two orbits are equal.

\begin{lemma}
If $t\in Gs$, then $Gt=Gs$.
\end{lemma}

\begin{proof}
Suppose $t\in G_s$. Then $t=gs$ for some $g\in G$. Then
\[Gt=G(gs)=\{h(gs)\mid h\in G\}=\{(hg)s\mid h\in G\}=(Gg)s=Gs.\]
\end{proof}

\begin{lemma}
Suppose the group $G$ acts on $S$. Then the orbits of the action partition $S$.
\end{lemma}

We need to prove the following:
\begin{enumerate}[label=(\roman*)]
\item The orbits are non-empty.
\item The union of orbits is $S$.
\item Two orbits are either equal or disjoint.
\end{enumerate}

\begin{proof} \
\begin{enumerate}[label=(\roman*)]
\item For every $s\in S$, $1s=s$ so $s\in Gs$. Hence every $s$ is in some orbit.

\item 

\item Let $x\in Gs$ and $x\in Gt$. Then $x=g_1 s=g_2 t$ for some $g_1,g_2\in G$. 
Thus
\[Gs=Gg_1s=Gg_2t=Gt.\]
\end{enumerate}
\end{proof}

Hence $S$ is a disjoint union of the distinct orbits, and we can write
\[S=\bigcup_{i\in I}Gs_i\]
where $I$ is some indexing set, and the $s_i$ represent distinct orbits.

We say an action $G$ on $S$ is \emph{transitive}, if for all $s\in S$, $Gs=S$. This means that we can reach any element from any element.

\begin{example}[Left regular action]
Any group $G$ acts on itself by left multiplication:
\[g\cdot s=gs.\]
This action is faithful and transitive.
\begin{proof}
If $g,s\in G$, then $g\cdot s=gs\in G$, so the operation is closed. We now show this is an action.
\begin{enumerate}[label=(\roman*)]
\item For all $a\in G$, $1\cdot a=1a=a$ by definition of a group.
\item For all $g,h\in G$ and $a\in G$, $g(ha)=(gh)a$ by associativity.
\end{enumerate} 

To show that it is faithful, we want to show that for all $a\in A$, $g\cdot a=a$ implies $g=1$; but this follows directly from the uniqueness of identity of the group $G$.

To show that it is transitive, for all $x,y\in G$, then $(yx^{-1})\cdot x=y$. Thus any $x$ can be sent to any $y$.
\end{proof}
\end{example}

\begin{definition}[Stabiliser]
Let $G$ act on $S$. The \vocab{stabiliser} of $s\in S$ is
\[G_s\colonequals\{g\in G\mid gs=s\}.\]
\end{definition}

Intuitively, it is the elements in $G$ that leave $s$ unchanged.

\begin{lemma*}
$G_s\le G$.
\end{lemma*}

\begin{proof}
Apply the subgroup criterion.
\begin{enumerate}[label=(\roman*)]
\item By definition, $1s=s$, so $1\in G_s$. Thus $G_s$ is non-empty. 
\item Let $g,h\in G_s$. Then $(gh^{-1})s=g(h^{-1} s)=gs=s$. Thus $gh^{-1}\in G_s$.
\end{enumerate}
\end{proof}

Since $G_s$ is a subgroup of $G$, we can consider cosets of $G_s$ in $G$.

\begin{lemma}
Let $G$ act on $S$. If $g,h\in G$ are in the same coset of $G_s$, then $gs=hs$.
\end{lemma}

\begin{proof}
If $g,h\in G$ are in the same coset of $G_s$, then we can write $h=gk$ for some $k\in G_s$. Then
\[hs=(gk)s=g(ks)=gs.\]
\end{proof}

\begin{theorem}[Orbit--stabiliser theorem]
Let $G$ act on $S$, and let $s\in S$. Then there exists a bijection between $Gs$ and cosets of $G_s$ in $G$. In particular, if $G$ is finite, then
\begin{equation}
|G|=|Gs|\:|G_s|.
\end{equation}
\end{theorem}

\begin{proof}
We biject the cosets of $G_s$ in $G$ with elements in the $Gs$. 
Consider the mapping
\begin{align*}
\theta\colon G/G_s&\to Gs\\
gG_s&\mapsto gs
\end{align*}
We claim that $\theta$ is a bijection.
\begin{enumerate}
\item We check that $\theta$ is well-defined: Suppose $gG_s=hG_s$. Then $h=gk$ for some $k\in G_s$. Thus $\theta(hG_s)=hs=(gk)s=g(ks)=gs=\theta(gG_s)$.

\item $\theta$ is surjective: Let $x\in Gs$. Then there exists $g\in G$ such that $x=gs$. Thus $\theta(gG_s)=gs=x$. 

\item $\theta$ is injective: Suppose $gs=hs$. Then $h^{-1}gs=s$, so $h^{-1}g\in G_s$. Thus $h^{-1}gG_s=G_s$, which implies $gG_s=hG_s$.
\end{enumerate}

Since $\theta$ is a bijection, we have
\[|G/G_s|=|Gs|.\]
Then the result follows from Lagrange's theorem.
\end{proof}

An immediate corollary is a formula for the size of an orbit:
\[|Gs|=|G:G_s|.\]
Suppose $S$ is a finite set. Then we get a decomposition of the order of $S$ as a sum or orders of orbits, which we call the \emph{orbit decomposition formula}:
\begin{equation}
|S|=\sum_{i=1}^{n}|G:G_{s_i}|.
\end{equation}

An important application of the orbit--stabiliser theorem is determining group sizes. 
To find the order of the symmetry group of, say, a pyramid, we find something for it to act on, pick a favorite element, and find the orbit and stabiliser sizes.

\begin{example}
Suppose we want to know how big $D_{2n}$ is. $D_{2n}$ acts on the vertices $\{1,2,\dots,n\}$ transitively. Since
\begin{align*}
|\orb(1)|&=n\\
\stab(1)&=\{e,\text{reflection in the line through 1}\} 
\end{align*}
we have that $|D_{2n}|=|\orb(1)|\:|\stab(1)|=2n$.
\end{example}

\subsection{More Actions}
Given any group $G$, there are a few important actions we can define. In particular, we will define the \emph{conjugation action}, which is a very important concept on its own. 

\begin{definition}[Conjugation of element]
The \vocab{conjugation}\index{conjugation} of $a\in G$ by $b\in G$ is
\[bab^{-1}\in G.\]
Two elements $a,b\in G$ are \emph{conjugate}\index{conjugate} if there exists $g\in G$ such that $b=gag^{-1}$.
\end{definition}

\begin{lemma}
Conjugation is an equivalence relation.
\end{lemma}

\begin{proof}
There are three properties to check:
\begin{enumerate}[label=(\roman*)]
\item Since $a=1a1^{-1}$, $a$ is conjugate to $a$. (Reflexivity)
\item If $a$ is conjugate to $b$, then $a=gbg^{-1}$ for some $g\in G$, so $b=g^{-1}ag$. (Symmetry)
\item Suppose $a$ is conjugate to $b$, and $b$ is conjugate to $c$. Then $a=gbg^{-1}$ for some $g\in G$, and $b=hch^{-1}$ for some $h\in G$. Thus $a=(gh)c(gh)^{-1}$. (Transitivity)
\end{enumerate}
\end{proof}

\begin{lemma}[Conjugation action]
Any group $G$ acts on itself by conjugation:
\[g\cdot h=ghg^{-1}\]
for all $g,h\in G$.
\end{lemma}

\begin{proof}
If $g,h\in G$ then $ghg^{-1}\in G$. We now show that this is an action:
\begin{enumerate}[label=(\roman*)]
\item $1\cdot s=1s1^{-1}=s$.
\item $g\cdot(h\cdot k)=g\cdot(hkh^{-1})=ghkh^{-1}g^{-1}=(gh)k(gh)^{-1}=(gh)\cdot k$.
\end{enumerate}
\end{proof}

We give special names for the orbits and stabilisers of the conjugation action.

\begin{definition}
The \vocab{conjugacy classes} are the orbits of the conjugacy action:
\[\ccl(a)\colonequals\{b\in G\mid\exists g\in G, gag^{-1}=b\}.\]
The \vocab{centralisers}\index{centralisers} are the stabilisers of the conjugation action:
\[C_G(a)\colonequals\{g\in G\mid gag^{-1}=a\}=\{g\in G\mid ga=ag\}.\]
\end{definition}

By the orbit decomposition formula,
\[|G|=\sum_{i=1}^{n}|G:C_G(g_i)|,\]
where the $g_i$ represent distinct centralisers.

The centraliser is defined as the elements that commute with a particular element $h$. For the whole group $G$, we can define the \emph{center}.

\begin{definition}[Center]
The \vocab{center}\index{center} of $G$ is the set of elements which commute with all the elements of $G$:
\[Z(G)\colonequals\{g\in G\mid gh=hg\:\forall h\in G\}.\]
\end{definition}

Suppose $g_1,\dots,g_m$ are representatives of the $m$ conjugacy classes which contain more than one element. 
Note that an element $g\in G$ is in the center of $G$ if and only if the orbit of $g$ is $\{g\}$.  
In general, the order of the orbit of $g$ is equal to the index of the centraliser of $g$.
Then
\begin{equation}
|G|=|Z(G)|+\sum_{i=1}^{m}|G:C_G(g_i)|.
\end{equation}
This is known as the \emph{class equation}.

In many ways, conjugation is related to normal subgroups.

\begin{lemma}
Let $H\triangleleft G$. Then $G$ acts by conjugation on $H$.
\end{lemma}

\begin{proof}
We only have to prove closure since the other properties follow from the conjugation action. However, by definition of a normal subgroup, for every $g\in G$, $h\in H$, we have $ghg^{-1}\in H$. So it is closed.
\end{proof}

\begin{proposition}
Normal subgroups are exactly those subgroups which are unions of conjugacy classes.
\end{proposition}

\begin{proof}
Let $H\triangleleft G$. If $h\in H$, then by definition for every $g\in G$, we get $ghg^{-1}\in H$. So $\ccl(h)\subset H$. So $H$ is the union of the conjugacy classes of all its elements.

Conversely, if $H$ is a union of conjugacy classes and a subgroup of $G$, then for all $h\in H$, $g\in G$, we have $ghg^{-1}\in H$. So $H$ is normal.
\end{proof}

\begin{lemma}
Let $X$ be the set of subgroups of $G$. Then $G$ acts by conjugation on $X$.
\end{lemma}

\begin{proof}
We first show closure. If $H\le G$, we need to show that $gHg^{-1}$ is also a subgroup.
\begin{enumerate}[label=(\roman*)]
\item We know that $1\in H$ and thus $g1g^{-1}=1\in gHg^{-1}$, so $gHg^{-1}$ is non-empty.
\item For any two elements $gag^{-1}$ and $gbg^{-1}\in gHg^{-1}$, $(gag^{-1})(gbg^{-1})^{-1}=g(ab^{-1})g^{-1}\in gHg^{-1}$.
\end{enumerate}

We now show that it is an action.
\begin{enumerate}[label=(\roman*)]
\item $1H1^{-1}=H$.
\item $g_1(g_2 Hg_2^{-1})g_1^{-1}=(g_1g_2)H(g_1g_2)^{-1}$.
\end{enumerate}
\end{proof}

Under this action, normal subgroups have singleton orbits.

\begin{definition}[Normaliser of subgroup]
The \vocab{normaliser} of a subgroup $H$ is the stabiliser of the (group) conjugation action:
\[N_G(H)\colonequals\{g\in G\mid gHg^{-1}=H\}.\]
\end{definition}

We clearly have $H\subset N_G(H)$. It is easy to show that $N_G(H)$ is the largest subgroup of $G$ in which $H$ is a normal subgroup, hence the name.

There is a connection between actions in general and conjugation of subgroups.

\begin{lemma}
Stabilisers of the elements in the same orbit are conjugate, i.e., let $G$ act on $S$ and let $g\in G$, $s\in S$. Then $G_{gs}=g\:G_s\:g^{-1}$.
\end{lemma}

\subsection{Applications}
\begin{theorem}[Cauchy's theorem]
Let $G$ be a finite group and prime $p$ dividing $|G|$. Then $G$ has an element of order $p$ (in fact there must be at least $p-1$ elements of order $p$).
\end{theorem}

\begin{proof}
Let $G$ and $p$ be fixed. Consider $G^p=G\times\cdots\times G$, the set of $p$-tuples of $G$. Let $X\subset G^p$ be
\[X=\{(a_1,\dots,a_p)\in G^p\mid a_1\cdots a_p=1\}.\]
In particular, if an element $b$ has order $p$, then $(b,b,\dots,b)\in X$. In fact, if $(b,b,\dots,b)\in X$ and $b\neq 1$, then $b$ has order $p$, since $p$ is prime.

Now let $H=\angbrac{h\mid h^p=1}\cong C_p$ be a cyclic group of order $p$ with generator $h$. Let $H$ act on $X$ by ``rotation'':
\[h(a_1,a_2\dots,a_p)=(a_2,a_3,\dots,a_p,a_1).\]
For closure, if $a_1\cdots a_p=1$, then $a_1^{-1}=a_2\cdots a_p$. So $a_2\cdots _p a_1=a^{-1}a_1=1$ thus $(a_2,a_3,\dots,a_p,a_1)\in X$.
This is an action:
\begin{enumerate}[label=(\roman*)]
\item $1$ acts as an identity by construction.
\item The associativity condition also works by construction.
\end{enumerate}

As orbits partition $X$, the sum of all orbit sizes must be $|X|$. We know that $|X|=|G|^{p-1}$ since we can freely choose the first $p-1$ entries and the last one must be the inverse of their product. 

Since $p$ divides $|G|$, we see that $p$ also divides $|X|$. We have $|\orb(a1,\dots,a_p)|\:|\stab_H(a_1,\dots,a_p)|=|H|=p$. So all orbits have size $1$ or $p$, and they sum to $|X|=p\times\text{something}$. We know that there is one orbit of size $1$, namely $(1,1,\dots,1)$. So there must be at least $p-1$ other orbits of size $1$ for the sum to be divisible by $p$.

In order to have an orbit of size $1$, they must look like $(a,a,\dots,a)$ for some $a\in G$, which has order $p$.
\end{proof}

\subsection{Sylow Subgroups}
The Sylow theorems are a set of related theorems describing the subgroups of prime power order of a given finite group. They are very powerful, since they can apply to any finite group, and play an important role in the theory of finite groups.

\begin{definition}
Let $p$ be a prime. By a \emph{$p$-group}, we mean a finite group whose order is a power of $p$ (i.e., $p^\alpha$ for some $\alpha\in\NN$).

Let $G$ be a finite group, $H\le G$. We call $H$ a \vocab{$p$-subgroup} of $G$ if $H$ is a $p$-group. 
\end{definition}

\begin{proposition}
Let $G$ be a non-trivial $p$-group. Then
\begin{enumerate}[label=(\roman*)]
\item $Z(G)$ is non-trivial;
\item $G$ is solvable.
\end{enumerate}
\end{proposition}

\begin{proof} \
\begin{enumerate}[label=(\roman*)]
\item By the class equation,
\[|G|=|Z(G)|+\sum|G:G_{x_i}|\]
where the sum is taken over a finite number of elements $x_i$ with $|G:G_{x_i}|\neq1$.

Since $G$ is a $p$-group, it follows that $p$ divides $|G|$ and also $|G:G_{x_i}|$. Hence $p$ divides $|Z(G)|$, so the center $Z(G)$ is not trivial.

\item $|G/Z(G)|$ divides $|G|$ so $G/Z(G)$ is a $p$-group, and by (i), we know that $|G/Z(G)|<|G|$. By induction $G/Z(G)$ is solvable. By \ref{prop:solvable-group-quotient}, it follows that $G$ is solvable.
\end{enumerate}
\end{proof}

\begin{definition}[Sylow $p$-subgroup]
Let $G$ be a finite group, $H\le G$. We say $H$ is a \vocab{Sylow $p$-subgroup}\index{Sylow $p$-subgroup} if $|H|=p^\alpha$ and $p^\alpha$ is the highest power of $p$ dividing $|G|$.
\end{definition}

We shall prove below that such subgroups always exist. For this we need a lemma.

\begin{lemma}
Let $G$ be a finite abelian group, $|G|=m$. Let $p$ be a prime, $p\mid m$. Then there exists $H\le G$ such that $|H|=p$.
\end{lemma}

\begin{proof}

\end{proof}

The frst Sylow theorem indicates existence of Sylow subgroups, the second Sylow theorem indicates that all Sylow subgroups are related by conjugation, and the third provides constraints on the number of such subgroups.

\begin{theorem}[Sylow I]
Let $G$ be a finite group, $p\mid|G|$. Then there exists a $p$-Sylow subgroup of $G$.
\end{theorem}

\begin{proof}

\end{proof}



\begin{theorem}[Sylow II]
Let $G$ be a finite group.

If $H$ is a $p$-subgroup of $G$, then $H$ is contained in some $p$-Sylow subgroup.

All $p$-Sylow subgroups are conjugate. 
\end{theorem}

\begin{theorem}[Sylow III]
The number of $p$-Sylow subgroups of $G$ is $\equiv1\pmod p$. 
\end{theorem}
\pagebreak

\section*{Exercises}
\begin{exercise}
Show that any two cyclic groups of the same order are isomorphic.
\end{exercise}

\begin{solution}
Suppose $\langle x\rangle$ and $\langle y\rangle$ are both cyclic groups of order $n$. We first prove the case where $n<\infty$. We claim that the map $\phi\colon \langle x\rangle\to\langle y\rangle$ which sends $x^k\mapsto y^k$ is an isomorphism.
\begin{lemma*}
Let $G$ be a group, $g\in G$, let $m,n\in\ZZ$. Denote $d=\gcd(m,n)$. If $g^n=1$ and $g^m=1$, then $g^d=1$.
\end{lemma*}
\begin{proof}
By Bezout's lemma, since $d=\gcd(m,n)$, then there exists $q,r\in\ZZ$ such that $qm+rn=d$. Thus
\[g^d=g^{qm+rn}=\brac{g^m}^q\brac{g^n}^r=1.\]
\end{proof}
We first show that $\phi$ is well-defined; that is, $x^r=x^s\implies \phi(x^r)=\phi(x^s)$. Note that $x^{r-s}=e$, so by the above lemma, $n\mid r-s$. Write $r=tn+s$ for some $t\in\ZZ$, so
\[\phi(x^r)=\phi(x^{tn+s})=y^{tn+s}=(y^n)^ty^s=y^s=\phi(x^s).\]

We then show that $\phi$ is a homomorphism:
\[\phi(x^a x^b)=\phi(x^{a+b})=y^{a+b}=y^a y^b=\phi(x^a)\phi(x^b).\]

Finally we show that $\phi$ is bijective. Since the element $y^k$ of $\langle y\rangle$ is in the image of $x^k$ under $\phi$, $\phi$ is surjective. Since both groups have the same finite order, any surjection from one to the other is a bijection. Therefore $\phi$ is an isomorphism.

We now prove the case where $n=\infty$. If $\langle x\rangle$ is an infinite cyclic group, let $\phi\colon\ZZ\to\langle x\rangle$ be defined by $\phi(k)=x^k$. (This map is well-defined since there is no ambiguity in the representation of elements in the domain.)

Since $x^a\neq x^b$ for all distinct $a,b\in\ZZ$, $\phi$ is injective. By definition of a cyclic group, $\phi$ is surjective. As above, the laws of exponents ensure $\phi$ is a homomorphism. Hence $\phi$ is an isomorphism.
\end{solution}

\begin{exercise}
The quotient group of a cyclic group is cyclic.
\end{exercise}

\begin{proof}
Let $G=C_n$ with $H\le C_n$. We know that $H$ is also cyclic; say $C_n=\angbrac{c}$ and $H=\angbrac{c^k}\cong C_\ell$, where $k\ell=n$. We have $C_n/H=\{H,cH,c^2H,\dots,c^{k-1}H\}=\angbrac{cH}\cong C_k$.
\end{proof}
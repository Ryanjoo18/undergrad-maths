\chapter{Groups}\label{chap:groups}
\begin{summary}
\item Group, Abelian group, examples. Subgroups. subgroup generated by a subset of a group. Cyclic subgroups.
\item Cosets and Lagrange's theorem; examples. The
order of an element. Fermat's little theorem.
\item Isomorphisms, examples. Groups of order 8 or less up to isomorphism (stated without
proof). Homomorphisms of groups with motivating examples. Kernels. Images. Normal subgroups. Quotient groups; examples. First Isomorphism Theorem. Cayley's theorem.
\item Group actions; examples. Definition of orbits and stabilizers. Transitivity. Orbits partition the set. Orbit-stabilizer Theorem. Examples and applications including Cauchy's Theorem and to conjugacy classes. Orbit-counting formula.
\end{summary}

\section{Definition and Properties}
\begin{definition}
A \vocab{binary operation} on a set $G$ is a map $\ast:G\times G\to G$. 
\end{definition}

\begin{notation}
For any $a,b\in G$, if the operation is clear, we write $ab$ for the image of $(a,b)$ under $\ast$.
\end{notation}

$\ast$ is \emph{associative} if $(ab)c=a(bc)$ for all $a,b,c\in G$; 
$\ast$ is \emph{commutative} if $ab=ba$ for all $a,b\in G$.

\begin{definition}[Group]
A \vocab{group}\index{group} $(G,\ast)$ consists of a set $G$ and a binary operation $\ast$ on $G$ satisfying the following group axioms:
\begin{enumerate}[label=(\roman*)]
\item Associativity: $a(bc)=(ab)c$ for all $a,b,c\in G$;\
\item Identity: there exists $e\in G$ such that $ae=ea=a$ for all $a\in G$.
\item Invertibility: for all $a\in G$, there exists $c\in G$ such that $ac=ca=e$.
\end{enumerate}

$G$ is \vocab{abelian} if the operation is commutative; it is \emph{non-abelian} if otherwise.
\end{definition}

\begin{remark}
When verifying that $(G,\ast)$ is a group we have to check (i), (ii), (iii) above and also that $\ast$ is a binary operation closed in $G$---that is, $a\ast b\in G$ for all $a,b\in G$.
\end{remark}

\begin{notation}
We simply denote a group $(G,\ast)$ by $G$ if the operation is clear.
\end{notation}

\begin{notation}
Since $\ast$ is associative, we omit unnecessary parentheses and write $(ab)c=a(bc)=abc$.
\end{notation}

\begin{notation}
For any $a\in G$, $n\in\ZZ^+$, denote $a^n=\underbrace{a\cdot a\cdots a}_\text{$n$ times}$.
\end{notation}

\begin{notation}
Denote the additive group $\CC=(\CC,+)$ etc., the multiplicative group $\CC^\times=\CC\setminus\{0\}$ etc., the set of (congruence classes of) integers modulo $n$ under addition as $\ZZ_n$ and under multiplication as $(\ZZ_n)^\times$.
\end{notation}

\begin{example}
\begin{itemize}
\item $\ZZ$, $\QQ$, $\RR$, $\CC$ are groups, with identity $0$ and (additive) inverse $-a$ for all $a$.
\item $\QQ^\times$, $\RR^\times$, $\CC^\times$, $\QQ^+$, $\RR^+$ are groups under $\times$, with identity $1$ and (multiplicative) inverse $\frac{1}{a}$ for all $a$.
\item For $n\in\ZZ^+$, $\ZZ_n$ is an abelian group under $+$.
\item For $n\in\ZZ^+$, $(\ZZ_n)^\times$ is an abelian group under multiplication.
\end{itemize}
\end{example}

\begin{proposition}
Let $G$ be a group.
\begin{enumerate}[label=(\roman*)]
\item The identity of $G$ is unique.
\item For each $a\in G$, $a^{-1}$ is unique.
\item $(a^{-1})^{-1}=a$ for all $a\in G$.
\item $(ab)^{-1}=b^{-1}a^{-1}$.
\item For any $a_1,\dots,a_n\in G$, $a_1\cdots a_n$ is independent of how we arrange the parantheses (generalised associative law).
\end{enumerate}
\end{proposition}

\begin{proof} \
\begin{enumerate}[label=(\roman*)]
\item Suppose that $e$ and $e^\prime$ are identities of $G$. Then
\[e=ee^\prime=e^\prime\]
where the first equality holds as $e^\prime$ is an identity, and the second equality holds as $e$ is an identity. Since $e=e^\prime$, the identity is unique.
\begin{notation}
Denote the identity of $G$ by $1_G$; the subscript may be omitted if there is no ambiguity. 
\end{notation}
\item Suppose that $b$ and $c$ are both inverses of $a$. Then $ab=1$, $ca=1$, so
\[c=c1=c(ab)=(ca)b=1b=b.\]
Hence the inverse is unique.
\begin{notation}
Denote the inverse of $a\in G$ by $a^{-1}$.
\end{notation}
\item To show $(a^{-1})^{-1}=a$ is exactly the problem of showing that $a$ is the inverse of $a^{-1}$, which is by definition of the inverse (with the roles of $a$ and $a^{-1}$ interchanged).
\item Let $c=(ab)^{-1}$. Then $(ab)c=1$, or $a(bc)=1$ by associativity, which gives $bc=a^{-1}$. Applying $b^{-1}$ on both sides gives $c=b^{-1}a^{-1}$.
\item The result is trivial for $n=1,2,3$. For all $k<n$ assume that any $a_1\cdots a_k$ is independent of parantheses. Then
\[(a_1\cdots a_n)=(a_1\cdots a_k)(a_{k+1}\cdots a_n).\]
By inductive hypothesis, both terms are independent of parentheses since $k,n-k<n$. Hence by induction we are done.
\end{enumerate}
\end{proof}

\begin{proposition}[Cancellation law]
Let $a,b\in G$. Then the equations $ax=b$ and $ya=b$ have unique solutions for $x,y\in G$. In particular, we can cancel on the left and right.
\end{proposition}

\begin{proof}
We can solve $ax=b$ by applying $a^{-1}$ to both sides of the equation to get $x=a^{-1}b$. The uniqueness of $x$ follows because $a^{-1}$ is unique. A similar case holds for $ya=b$.
\end{proof}

\begin{definition}[Order of a group]
Let $G$ be a group. The \vocab{order} of $G$ is its cardinality $|G|$. A group $G$ is a \emph{finite group} if $|G|<\infty$.
\end{definition}

One way to represent a finite group is by means of a \vocab{Cayley table}. Let $G=\{1,g_2,g_3,\dots,g_n\}$. The Cayley table of $G$ is a square grid which contains all the possible products of two elements from $G$: the product $g_ig_j$ appears in the $i$-th row and $j$-th column.

\begin{remark}
Note that a group is abelian if and only if its Cayley table is symmetric about the main (top-left to bottom-right) diagonal.
\end{remark}

\subsection{Examples}
\begin{example}[Dihedral groups]
An important family of groups is the \vocab{dihedral groups}. For $n\in\ZZ^+$, $n\ge3$, let $D_{2n}$ be the set of symmetries of a regular $n$-gon.

Let $r$ be the rotation clockwise about the origin by $\frac{2\pi}{n}$ radians, $s$ be the reflection about the line of symmetry through the first labelled vertex and the origin. (Read from right to left: for instance, $sr$ means do $r$ then $s$.)

Properties of $D_{2n}$:
\begin{itemize}
\item $1,r,r^2,\dots,r^{n-1}$ are all distinct and $r^n=1$, so $|r|=n$.
\item $s^2=1$ since we either reflect or do not reflect, so $|s|=2$.
\item $s\neq r^i$ for any $i$, since the effect of any reflection cannot be obtained from any form of rotation.
\item $sr^i\neq sr^j$ for all $i\neq j$ ($0\le i,j\le n-1$), so
\[D_{2n}=\{1,r,\dots,r^{n-1},s,sr,\dots,sr^{n-1}\}\]
and thus $|D_{2n}|=2n$.
\item $rs=sr^{-1}$
\item $r^is=sr^{-i}$

Proof: From above, this is true for $i=1$. Assume it holds for $k<n$. Then $r^{k+1}s=r(r^ks)=rsr^{-k}$. Then $rs=sr^{-1}$ so $rsr^{-k}=sr^{-1}r^{-k}=sr^{-k-1}$ so we are done.
\end{itemize}
\end{example}

Note that for each $n\in\ZZ^+$, the generators of $D_{2n}$ are $r$ and $s$, and we have shown that they satisfy $r^n=1$, $s^2=1$, and $rs=sr^{-1}$; these are called \emph{relations}. Any other equation involving the generators can be derived from these relations.

Any such collection of generators $S$ and relations $R_1,\dots,R_m$ for a group $G$ is called a \emph{presentation}, written
\[G=\langle S\mid R_1,\dots,R_m\rangle.\]

For example,
\[D_{2n}=\langle r,s\mid r^n=s^2=1,rs=sr^{-1}\rangle.\]

\begin{example}[Permutation groups]
Let $S$ be a non-empty set. A bijection $S\to S$ is called a \emph{permutation} of $S$; the set of permutations of $S$ is denoted by $\Sym(S)$.

We now show that $\Sym(S)$ is a group under function composition $\circ$; $(\Sym(S),\circ)$ is the \vocab{symmetric group} on $S$. Note that $\circ$ is a binary operation on $\Sym(S)$ since if $\sigma:S\to S$ and $\tau:S\to S$ are both bijections, then $\sigma\circ\tau$ is also a bijection from $S$ to $S$.
\begin{enumerate}[label=(\roman*)]
\item Function composition is associative so $\circ$ is associative.
\item The identity of $\Sym(S)$ is the identity map $1$, defined by $1(a)=a$ for all $a\in S$.
\item For every permutation $\sigma$, $\sigma$ is bijective and thus invertible, so there exists a (2-sided) inverse $\sigma^{-1}:S\to S$ satisfying $\sigma\circ\sigma^{-1}=\sigma^{-1}\circ\sigma=1$.
\end{enumerate}

In the special case where $S=\{1,2,\dots,n\}$, the symmetric group on $S$ is called the \emph{symmetric group of degree $n$}, denoted by $S_n$.

\begin{proposition*}
If $|S|\ge3$ then $\Sym(S)$ is non-abelian.
\end{proposition*}

\begin{proof}
Let $S=\{x_1,x_2,x_3\}$ where three elements are distinct.
\end{proof}

\begin{proposition*}
$|S_n|=n!$
\end{proposition*}

\begin{proof}
Obvious, since there are $n!$ permutations of $\{1,2,\dots,n\}$.
\end{proof}
\end{example}

\begin{example}[Matrix groups]
For $n\in\ZZ^+$, let $GL_n(\FF)$ be the set of all $n\times n$ invertible matrices whose entries are in $\FF$:
\[GL_n(\FF)=\{A\in M_{n\times n}(\FF)\mid\det(A)\neq0\}.\]

We show that $GL_n(\FF)$ is a group under matrix multiplication; $GL_n(\FF)$ is the \vocab{general linear group} of degree $n$.
\begin{enumerate}[label=(\roman*)]
\item Since $\det(AB)=\det(A)\cdot\det(B)$, it follows that if $\det(A)\neq0$ and $\det(B)\neq0$, then $\det(AB)\neq0$, so $GL_n(\FF)$ is closed under matrix multiplication.
\item Matrix multiplication is associative.
\item $\det(A)\neq0$ if and only if $A$ has an inverse matrix, so each $A\in GL_n(\FF)$ has an inverse $A^{-1}\in GL_n(\FF)$ such that
\[AA^{-1}=A^{-1}A=I\]
where $I$ is the $n\times n$ identity matrix.
\end{enumerate}
\end{example}

\begin{example}[Quaternion group]
The \vocab{Quaternion group} $Q_8$ is defined by
\[Q_8=\{1,-1,i,-i,j,-j,k,-k\}\]
with product $\cdot$ computed as follows:
\begin{itemize}
\item $1\cdot a=a\cdot 1=a$ for all $a\in Q_8$
\item $(-1)\cdot(-1)=1$
\item $(-1)\cdot a=a\cdot(-1)=-a$ for all $a\in Q_8$
\item $i\cdot i=j\cdot j=k\cdot k=-1$
\item $i\cdot j=k$, $j\cdot i=-k$, $j\cdot k=i$, $k\cdot j=-i$, $k\cdot i=j$, $i\cdot k=-j$
\end{itemize}
Note that $Q_8$ is a non-abelian group of order $8$.
\end{example}

\subsection{Subgroups}
When given a set with certain properties, it is natural to consider its subsets that inherit the same properties.

\begin{definition}[Subgroup]
Let $G$ be a group. Non-empty $H\subset G$ is a \vocab{subgroup}\index{subgroup} of $G$, denoted by $H\le G$, if $H$ is a group under the product in $G$.
\end{definition}

That is, $H\le G$ if and only if
\begin{enumerate}[label=(\roman*)]
\item $1\in H$;\hfill(identity)
\item $ab\in H$ for all $a,b\in H$;\hfill(closure)
\item $a^{-1}\in H$ for all $a\in H$.\hfill(inverses)
\end{enumerate}

\begin{remark}
There is no need to check that associativity holds in $H$, as it follows from associativity in $G$.
\end{remark}

Every group $G$ has two obvious subgroups: the group $G$ itself, and the \emph{trivial subgroup} $\{1\}$. A subgroup is a \emph{proper subgroup} if it is not one of those two.

It would be useful to have some criterion for deciding whether a given subset of a group is a subgroup.

\begin{lemma}[Subgroup criterion]
Let $G$ be a group. Then $H\le G$ if and only if
\begin{enumerate}[label=(\roman*)]
\item $H\neq\emptyset$;
\item $ab^{-1}\in H$ for all $a,b\in H$. 
\end{enumerate}
\end{lemma}

\begin{proof} \

\fbox{$\implies$} If $H\le G$, then we are done, by definition of subgroup.

\fbox{$\impliedby$} Check group axioms:
\begin{enumerate}[label=(\roman*)]
\item Since $H\neq\emptyset$, there exists $a\in H$. Then $1=aa^{-1}\in H$.
\item Since $1\in H$ and $a\in H$, then $a^{-1}=1a^{-1}\in H$.
\item For any $a,b\in H$, $a,b^{-1}\in H$, so by (ii), $a(b^{-1})^{-1}=ab\in H$.
\end{enumerate}
\end{proof}

\begin{proposition}
Let $G$ be a group, $H,K\le G$. Then $H\cap K\le G$.
\end{proposition}

\begin{proof}
Apply the subgroup criterion:
\begin{enumerate}[label=(\roman*)]
\item Since $1\in H$ and $1\in K$, then $1\in H\cap K$ so $H\cap K\neq\emptyset$.
\item Let $a,b\in H\cap K$. Then $a,b\in H$ and $a,b\in K$. Since $H,K\le G$, by the subgroup criterion, $ab^{-1}\in H$ and $ab^{-1}\in K$, so $ab^{-1}\in H\cap K$.
\end{enumerate}
\end{proof}

\begin{corollary}
Let $G$ be a group, $\{H_i\mid i\in I\}$ is a collection of subgroups of $G$. Then
\[\bigcap_{i\in I}H_i\le G.\]
\end{corollary}

\subsection{Cyclic Groups}
\begin{definition}
The \vocab{cyclic subgroup} $H$ generated by $a\in G$, denoted by $H=\langle a\rangle$, is the set of all powers of $a$:
\[H=\{\cdots,a^{-2},a^{-1},1,a,a^2,\dots\}.\]
$a$ is a \emph{generator} of $H$.
\end{definition}

You should verify that that $\langle a\rangle$ is indeed a subgroup of $G$. Furthermore, $\langle a\rangle$ is the smallest subgroup of $G$ that contains $a$.

\begin{remark}
A cyclic subgroup may have more than one generator. For example, if $H=\langle a\rangle$, then also $H=\langle a^{-1}\rangle$ because $(a^{-1})^n=a^{-n}\in H$ for $n\in\ZZ$ so does $-n$, thus
\[\{a^n\mid n\in\ZZ\}=\{(a^{-1})^n\mid n\in\ZZ\}.\] 
\end{remark}

\begin{lemma}
Cyclic groups are abelian.
\end{lemma}

\begin{proof}
Let $G$ be a cyclic group. For $a^i,a^j\in G$, by the laws of exponents,
\[a^i a^j=a^{i+j}=a^j a^i.\]
\end{proof}

\begin{proposition}
A subgroup of a cyclic group is cyclic.
\end{proposition}

\begin{proof}
Let $a\in G$, $H\le\langle a\rangle$. If $H=\{1\}$ then trivially $H$ is cyclic.

Suppose that $H$ contains some other element $b\neq1$. Then $b=a^n$ for some integer $n$. Since $H$ is a subgroup, $b^{-1}=a^{-n}\in H$. Since either $n$ or $-n$ is positive, we can assume $H$ contains positive powers of $a$ and $n>0$. Let $m$ be the smallest positive integer such that $a^m\in H$ (such an $m$ exist by the well-ordering principle).

\begin{claim}
$h=a^m$ is a generator for $H$.
\end{claim}

We need to show that every $h^\prime\in H$ can be written as a power of $h$. Since $h^\prime\in H$ and $H\le\langle a\rangle$, $h^\prime=a^k$ for some integer $k$. By the division algorithm, there exist integers $q,r$ such that $k=qm+r$ with $0\le r<m$. Hence
\[a^k=a^{qm+r}=(a^m)^q a^r=h^q a^r\]
so $a^r=a^k h^{-q}$. Since $a^k,h^{-q}\in H$, we must have $a^r\in H$. By the minimality of $m$, we must have $m=0$ and so $k=qm$. Hence
\[h^\prime=a^k=a^{qm}=h^q\]
and $H$ is generated by $h$.
\end{proof}

\begin{corollary}\label{cor:subgroup-Z}
The subgroups of $\ZZ$ are exactly $n\ZZ$ for $n=0,1,2,\dots$.
\end{corollary}

There are two possibilities: Either the powers $x^n$ represent distinct elements, or they do not. We analyse the case that the powers of $x$ are not distinct (finite cyclic groups).

\begin{proposition}\label{prop:cyclic-group-order}
Let $a\in G$, let $S=\{k\in\ZZ\mid a^k=1\}$.
\begin{enumerate}[label=(\roman*)]
\item $S\le\ZZ$.
\item $a^r=a^s$ (with $r\ge s$) if and only if $a^{r-s}=1$.
\item Suppose that $S$ is not the trivial subgroup. Then $S=n\ZZ$ for some $n\in\ZZ^+$. The powers $1,a,a^2,\dots,a^{n-1}$ are the distinct elements of $\langle a\rangle$, and the order of $\langle a\rangle$ is $n$.
\end{enumerate}
\end{proposition}

\begin{proof} \
\begin{enumerate}[label=(\roman*)]
\item $a^0=1$ so $0\in S$.

If $k,l\in S$, $a^k=1$ and $a^l=1$ so $a^{k+l}=a^ka^l=1$ then $k+l\in S$. 

If $k\in S$, $a^k=1$, then $a^{-k}=(a^k)^{-1}=1$ so $-k\in S$.
\item This follows from the cancellation law.
\item Suppose that $S\neq\{0\}$. Then \cref{cor:subgroup-Z} shows that $S=n\ZZ$, where $n$ is the smallest positive integer in $S$.

Let $a^k\in\langle a\rangle$. By the division algorithm, write $k=qn+r$ with $0\le r<n$. Then $a^{qn}=1^q=1$ so $a^k=a^{qn}a^r=a^r$. Hence $a^k$ is equal to one of the powers $1,a,a^2,\dots,a^{n-1}$. It follows from (ii) that these powers are distinct, because $a^n$ is the smallest positive power equal to $1$.
\end{enumerate}
\end{proof}

The group $\langle a\rangle=\{1,a,\dots,a^{n-1}\}$ described by (iii) above is called a \emph{cyclic group of order $n$}.

More generally, we make the following definition.

\begin{definition}[Subgroup generated by subset of group]
Let $G$ be a group, $S\subset G$. The \vocab{subgroup generated by $S$}, denoted by $\langle S\rangle$, is the smallest subgroup of $G$ which contains $S$.

If $\langle S\rangle=G$, then the elements of $S$ are said to be \emph{generators} of $G$.
\end{definition}

\begin{notation}
If $a\in G$, then we write $\langle a\rangle$ (rather than the more accurate but cumbersome $\langle\{a\}\rangle$).
\end{notation}

\subsection{Order}
\begin{definition}[Order]
Let $G$ be a group, $a\in G$. If there is a positive integer $k$ such that $a^k=1$, then the \vocab{order} of $g$ is defined as
\[o(a)\coloneqq\min\{m>0\mid a^m=1\}.\]
Otherwise we say that the order of $a$ is infinite.
\end{definition}

\begin{proposition}
If $G$ is finite, then $o(a)$ is finite for each $a\in G$.
\end{proposition}

\begin{proof}
Consider the list
\[a,a^2,a^3,\dots\in G.\]
Since $G$ is finite, this list must have repeats. Hence $a^i=a^j$ for some integers $i>j$, so $a^{i-j}=1$. This shows that $\{m>0\mid a^m=e\}$ is non-empty and thus has a minimal element.
\end{proof}

\begin{proposition}
If $a\in G$ and $o(a)$ is finite, then $a^n=1$ if and only if $o(a)\mid n$.
\end{proposition}

\begin{proof} \

\fbox{$\impliedby$} Suppose $o(a)\mid n$. Then $n=ko(a)$ for some $k\in\ZZ$, so
\[a^n=\brac{a^{o(a)}}^k=1^k=1.\]
\fbox{$\implies$} Suppose $a^n=1$. By the division algorithm, there exists integers $q,r$ such that $n=qo(a)+r$, where $0\le r<o(a)$. Then
\[a^r=a^{n-qo(a)}=a^n\brac{a^{o(a)}}^{-q}=1.\]
By the minimality of $o(a)$, we must have $r=0$, and so $n=qo(a)$ implies $o(a)\mid n$.
\end{proof}

\begin{corollary}
Let $G$ be a cyclic group, $a\in G$. Then $a^k=a^m$ if and only if $m\equiv k\pmod{o(a)}$.
\end{corollary}
\pagebreak

\section{Cosets}
\begin{definition}[Coset]\index{coset}
Let $H\le G$. For $a\in G$, a \vocab{left coset}\index{coset!left coset} and \vocab{right coset}\index{coset!right coset} of $H$ in $G$ are
\begin{align*}
aH&\coloneqq\{ah\mid h\in H\}\\
Ha&\coloneqq\{ha\mid h\in H\}
\end{align*}
Any element of a coset is called a \emph{representative} for the coset.
\end{definition}

The set of left cosets is given by
\[(G/H)_{l}\coloneqq\{aH\mid a\in G\}.\]
Similarly, the set of right cosets is given by
\[(G/H)_{r}\coloneqq\{Ha\mid a\in G\}.\]

\begin{lemma}
Let $H\le G$. Then $aH=H$ if and only if $a\in H$. (Similarly, $Ha=H$ if and only if $a\in H$.)
\end{lemma}

\begin{proof} \

\fbox{$\implies$} Suppose $aH=H$. Then $ah\in H$ for some $h\in H$. Let $k=ah$, then $a=kh^{-1}\in H$.

\fbox{$\impliedby$} Let $a\in H$. Then $aH\subset H$.

Since $a^{-1}\in H$, $a^{-1}H\subset H$. Then $H=eH=(aa^{-1})H=a(a^{-1})H\subset aH$. Hence $aH=H$.
\end{proof}

The next result shows when two cosets are equal.
\begin{lemma}
Let $H\le G$, $a,b\in G$. Then $aH=bH$ if and only if $a^{-1}b\in H$.
\end{lemma}

\begin{proof}
\begin{align*}
aH=bH&\iff a^{-1}(aH)=a^{-1}bH\\
&\iff (a^{-1}a)H=(a^{-1}b)H\\
&\iff H=(a^{-1}b)H
\end{align*}
Note that from the previous result, $H=(a^{-1}b)H$ if and only if $a^{-1}b\in H$.
\end{proof}

\begin{proposition}
Let $H\le G$. Then $(G/H)_l$ forms a partition of $G$. (Similar remarks hold for right cosets.)
\end{proposition}

We need to prove the following.
\begin{enumerate}[label=(\roman*)]
\item For all $a\in G$, $aH\neq\emptyset$.
\item $\bigcup_{a\in G}aH=G$.
\item For every $a,b\in G$, $aH\cap bH=\emptyset$ or $aH=bH$.
\end{enumerate}

\begin{proof} \
\begin{enumerate}[label=(\roman*)]
\item Since $H\le G$, $e\in H$. Thus for all $a\in G$, $a=ae\in aH$ so $aH\neq\emptyset$.
\item For all $a\in G$, $aH\subset G$, then $\bigcup_{a\in G}aH\subset G$. Note that $a\in G$ implies $a=ae\in aH$, and so $G=\bigcup_{a\in G}g\subset\bigcup_{a\in G}aH$. By double inclusion we are done.
\item If $aH\cap bH=\emptyset$, then we are done. If $aH\cap bH\neq\emptyset$ we need to show $aH=bH$. Let $x\in G$ such that $x\in aH\cap bH$. Then $x=ah_1=bh_2$ for $h_1,h_2\in H$ so $h_1=a^{-1}bh_2$. Notice that $a^{-1}b=h_1h_2^{-1}\in H$ and thus $aH=bH$.
\end{enumerate}
\end{proof}

\subsection{Lagrange's Theorem}
\begin{definition}[Index]
Let $H\le G$. The \vocab{index}\index{index} of $H$ in $G$ is the number of left cosets of $H$ in $G$, denoted by $|G:H|$.
\end{definition}

The following result shows that $H$ partitions $G$ into equal-sized parts.

\begin{lemma}
The cosets of $H$ in $G$ are the same size as $H$; that is, for all $a\in G$, $|aH|=|H|$.
\end{lemma}

\begin{proof}
Let $f:H\to aH$ which sends $h\mapsto ah$. For $h_1,h_2\in H$,
\begin{align*}
f(h_1)=f(h_2)
&\implies ah_1=ah_2\\
&\implies a^{-1}ah_1=a^{-1}ah_2\\
&\implies h_1=h_2
\end{align*}
thus $f$ is an injective mapping. Note that $f$ is surjective by the definition of $aH$. Since $f$ is bijective, $|H|=|aH|$.
\end{proof}

\begin{theorem}[Lagrange's theorem]
Let $G$ be a finite group, $H\le G$. Then $|G|=|H|\:|G:H|$.
\end{theorem}

\begin{proof}
Let $|H|=n$, and let $|G:H|=k$. Since $G$ is partitioned into $k$ disjoint subsets, each of which has cardinality $n$, we have $|G|=kn$, or
\begin{equation}\label{eqn:counting-formula}
|G|=|H|\:|G:H|
\end{equation}
as desired.
\end{proof}

\cref{eqn:counting-formula} is known as the \emph{Counting Formula}.

\begin{corollary}
The order of an element of a finite group divides the order of the group.
\end{corollary}

\begin{proof}
Let $a\in G$. Then by \cref{prop:cyclic-group-order}, $o(a)=|\langle a\rangle|$.

Since $\angbrac{a}$ is a subgroup of $G$, by Lagrange's Theorem, $|\angbrac{a}|$ divides $|G|$; that is, $o(a)$ divides $|G|$.
\end{proof}

\begin{corollary}
A group of prime order is cyclic.
\end{corollary}

\begin{proof}
Let $|G|=p$ be prime. Let $a\in G$, $a\neq1$. We will show that $G=\langle a\rangle$.

Since $o(a)\mid |G|=p$ and $o(a)>1$, we must have $o(a)=p$. Notice that this is also the order of $\langle a\rangle$. Since $G$ has order $p$, thus $\langle a\rangle=G$.
\end{proof}

This corollary classifies groups of prime order $p$. They form one isomorphism class: the class of the cyclic groups of order $p$.

The next result is of great interest in number theory. The \emph{Euler $\phi$-function} $\phi(n)$ is defined for all positive integers as follows:
\[\phi(n)=\begin{cases}
1&(n=1)\\
\text{number of positive integers less than $n$, relatively prime to $n$}&(n>1)
\end{cases}\]

\begin{theorem}[Euler]
If $n$ is a positive integer, and $a$ is coprime to $n$, then
\[a^{\phi(n)}\equiv1\pmod n.\]
\end{theorem}

\begin{theorem}[Fermat]
If $p$ is prime, and $a$ is any integer, then
\[a^p\equiv a\pmod p.\]
\end{theorem}

\begin{proof}
If $n$ is a prime number $p$, then $\phi(p)=p-1$. We consider two cases.
\begin{itemize}
\item If $a$ is coprime to $p$, then by Euler's totient theorem, $a^{p-1}\equiv1\pmod p$, and the desired result follows.
\item If $a$ is not coprime to $p$, since $p$ is prime, we must have that $p\mid a$, so that $a\equiv0\pmod p$. Hence $0\equiv a^p\equiv a\pmod p$ here also.
\end{itemize}
\end{proof}

\subsection{Counting Principle}
We generalise the notion of cosets, as defined earlier.

\begin{definition}
Let $H,K\le G$, define
\[HK=\{hk\mid h\in H,k\in K\}.\]
\end{definition}

\begin{lemma}
$HK\le G$ if and only if $HK=KH$.
\end{lemma}

\begin{proof} \

\fbox{$\impliedby$} Suppose $HK=KH$; that is, if $h\in H$ and $k\in K$, then $hk=k_1h_1$ for some $k_1\in K,h_1\in H$.

We now show that $HK$ is a subgroup of $G$:
\begin{enumerate}[label=(\roman*)]
\item $1\in H$ and $1\in K$, so $1\in HK$.
\item Let $x=hk\in HK$, $y=h^\prime k^\prime\in HK$. then
\[xy=hkh^\prime k^\prime.\]
Note that $kh^\prime\in KH=HK$, so $kh^\prime=h_2k_2$ for some $h_2\in H,k_2\in K$. Then
\[xy=h(h_2k_2)k^\prime=(hh_2) (k_2k^\prime)\in HK.\]
Thus $HK$ is closed.
\item Let $x\in HK$, then $x=hk$ for some $h\in H,k\in K$. Thus
\[x^{-1}=(hk)^{-1}=k^{-1}h^{-1}\in KH=HK,\]
so $x^{-1}\in HK$.
\end{enumerate}

\fbox{$\implies$} Suppose $HK\le G$. 
\begin{itemize}
\item Let $x\in KH$, so $x=kh$ for some $k\in K,h\in H$. Then
\[x=kh=(h^{-1}k^{-1})^{-1}\in HK.\]
Thus $KH\subset HK$.
\item Let $x\in HK$. Since $HK\le G$, $HK$ is closed under inverses, so $x^{-1}=hk\in HK$. Then
\[x=(x^{-1})^{-1}=(hk)^{-1}=k^{-1}h^{-1}\in KH.\]
Thus $HK\subset KH$.
\end{itemize}
Hence $HK=KH$.
\end{proof}

An interesting special case is the situation when $G$ is an abelian group, for in that case trivially $HK=KH$. Thus as a consequence we have the following result.

\begin{corollary}
Let $H,K\le G$, where $G$ is abelian. Then $HK\le G$.
\end{corollary}

\begin{proposition}
If $H,K\le G$ are finite groups, then
\[|HK|=\frac{|H||K|}{|H\cap K|}.\]
\end{proposition}

\begin{proof}
Notice that $HK$ is a union of left cosets of $K$, namely
\[HK=\bigcup_{h\in H}hK.\]

\end{proof}

\subsection{Normal Subgroups, Quotient Groups}
\begin{definition}[Normal subgroup]
Let $G$ be a group. $H\le G$ is a \vocab{normal subgroup} of $G$, denoted by $H\triangleleft G$, if
\[aH=Ha\quad(\forall a\in G)\]
\end{definition}

If $G$ has no non-trivial normal subgroup, then $G$ is a \emph{simple group}.

\begin{remark}
This does \emph{not} mean that $ah=ha$ for all $a\in G$, $h\in H$ or that $G$ is abelian. Although we can easily see that all subgroups of abelian groups are normal. In general, a left coset does not equal the right coset.
\end{remark}

\begin{lemma}
The following are equivalent.
\begin{enumerate}[label=(\roman*)]
\item $H\triangleleft G$.
\item $ghg^{-1}\in H$ for all $g\in G$, $h\in H$.
\item $gHg^{-1}=H$ for all $g\in G$.
\end{enumerate}
\end{lemma}

\begin{proof} \

\fbox{(i)$\iff$(ii)} In the forward direction, $aH=Ha$ for all $a\in G$. Let $g\in G$, $x\in H$. Then $gH=Hg$ so $gx=h^\prime g$ for some $h^\prime\in H$. Then $gxg^{-1}=h^\prime gg^{-1}=h^\prime\in H$.

In the reverse direction, $ghg^{-1}\in H$ for all $g\in G$, $h\in H$. Fix $g$. Then $ghg^{-1}\in H$ implies $gh\in Hg$ for all $h\in H$. So $gH\subset Hg$. Similarly $gH\supset Hg$, so $gH=Hg$.

\fbox{(i)$\iff$(iii)} $H\triangleleft G$ if and only if for all $g\in G$,
\begin{align*}
gH=Hg&\iff(gH)g^{-1}=(Hg)g^{-1}\\
&\iff gHg^{-1}=H
\end{align*}
\end{proof}

\begin{remark}
We frequently use (ii) to check if a subgroup is a normal subgroup.
\end{remark}

\begin{definition}[Quotient group]
Let $G$ be a group, $H\triangleleft G$. Then the \vocab{quotient group} of $G$ by $H$ is
\[G/H\coloneqq\{aH\mid a\in G\}.\]
\end{definition}

\begin{lemma}
$G/H$ is a group under the following operation: for all $aH,bH\in G/H$,
\[(aH)(bH)=a(Hb)H=a(bH)H=abH\]
\end{lemma}

\begin{proof}
Check group axioms.
\begin{enumerate}[label=(\roman*)]
\item For $a,b,c\in G$,
\[(aH)(bHcH)=(aH)(bcH)=a(bc)H=(ab)cH=(aHbH)cH\]
so the operation is associative.
\item The identity of $G/H$ is the coset $eH=H$.
\item For $aH\in G/H$, the inverse of $aH$ is $a^{-1}H$ as is immediate from the definition of the product:
\[(aH)(a^{-1}H)=aa^{-1}H=H\implies(aH)^{-1}=a^{-1}H.\]
\end{enumerate}
\end{proof}

\begin{lemma}
Let $G$ be a finite group, $H\triangleleft G$. Then
\[|G/H|=|G:H|=\frac{|G|}{|H|}.\]
\end{lemma}

\begin{proof}
Since $G/H$ has as its elements the left cosets of $H$ in $G$, and since there are precisely $|G:H|$ such cosets, by Lagrange's theorem, we obtain the desired result.
\end{proof}

\begin{definition}[Quotient map]
Let $H\triangleleft G$. The \vocab{quotient map} is the map $\pi:G\to G/H$ which sends $a\mapsto aH$.
\end{definition}
\pagebreak

\section{Homomorphisms and Isomorphisms}
In this section, we make precise the notion of when two groups ``look the same''; that is, they have the same group-theoretic structure. This is the notion of an \emph{isomorpism} between two groups.

When we talk about functions between groups it makes sense to limit our scope to functions that preserve the group operation (morphisms in the category of groups). More precisely:

\begin{definition}[Homomorphism]
Let $(G,\ast)$ and $(H,\diamond)$ be groups. A map $\phi:G\to H$ is called a \vocab{homomorphism}\index{homomorphism} if, for all $x,y\in G$,
\[\phi(x\ast y)=\phi(x)\diamond\phi(y).\]
\end{definition}

When the group operations for $G$ and $H$ are not explicitly written, we have
\[\phi(xy)=\phi(x)\phi(y).\]

\begin{definition}[Isomorphism]
An \vocab{isomorphism}\index{isomorphism} $\phi:G\to H$ is a bijective homomorphism.

If $\phi:G\to H$ is an isomorphism, then $G$ and $H$ are \emph{isomorphic}, denoted by $G\cong H$.
\end{definition}

An \emph{automorphism} of a group $G$ is an isomorphism from $G$ to $G$; the automorphisms of $G$ form a group $\Aut(G)$ under composition. An \emph{endomorphism} of $G$ is a homomorphism from $G$ to $G$.

\begin{example}
$(\RR,+)\cong(\RR^+,\times)$, as the exponential map $\exp:\RR\to\RR^+$ defined by $\exp(x)=e^x$ is an isomorphism from $(\RR,+)$ to $(\RR^+,\times)$.
\begin{enumerate}[label=(\roman*)]
\item $\exp$ is a bijection since it has an inverse function (namely $\ln$).
\item $\exp$ preserves the group operations since $e^{x+y}=e^xe^y$.
\end{enumerate}
\end{example}

\begin{proposition}
Let $\phi:G\to H$ be a homomorphism. Let $g\in G$, $n\in\ZZ$. Then
\begin{enumerate}[label=(\roman*)]
\item $\phi(1_G)=1_H$;
\item $\phi(g^{-1})=\brac{\phi(g)}^{-1}$;
\item $\phi(g^n)=\brac{\phi(g)}^n$.
\end{enumerate}
\end{proposition}

\begin{proof} \
\begin{enumerate}[label=(\roman*)]
\item $\phi(1_G)=\phi(1_G 1_G)=\phi(1_G)\phi(1_G)$, then apply $\phi(1_G)^{-1}$ to both sides to get $\phi(1_G)=1_H$.

\item $\phi(g)\phi(g^{-1})=\phi(gg^{-1})=\phi(1_G)=1_H$.

\item Note more generally that we can show $\phi(g^n)=(\phi(g))^n$ for $n>0$ by induction. For $n=-k<0$ we have
\[\phi(g^n)=\phi((g^{-1})^k)=(\phi(g^{-1}))^k=(\phi(g)^{-1})^k=\phi(g)^n.\]
\end{enumerate}
\end{proof}

\begin{proposition}
Quotient maps are homomorphisms.
\end{proposition}

\begin{proof}
Let $\pi:G\to G/H$ which sends $g\mapsto gH$ be a quotient map. Then for all $x,y\in G$,
\[\pi(xy)=(xy)H=(xH)(yH)=\pi(x)\pi(y).\]
\end{proof}

\subsection{Kernel and Image}
\begin{definition}[Kernel and image]
Let $\phi:G\to H$ be a homomorphism. Then the \vocab{kernel}\index{kernel} of $\phi$ is
\[\ker\phi\coloneqq\{g\in G\mid \phi(g)=1_H\}\subset G.\]
The \vocab{image}\index{image} of $G$ under $\phi$ is
\[\im\phi\coloneqq\phi(G)=\{\phi(g)\mid g\in G\}\subset H.\]
\end{definition}

\begin{remark}
$\im\phi$ is the usual set theoretic image of $\phi$.
\end{remark}

\begin{proposition}
Let $\phi:G\to H$ be a homomorphism. Then
\begin{enumerate}[label=(\roman*)]
\item $\ker\phi\triangleleft G$;
\item $\im\phi\le H$.
\end{enumerate}
\end{proposition}

\begin{proof} \
\begin{enumerate}[label=(\roman*)]
\item Apply the subgroup criterion. Since $1_G\in\ker\phi$, $\ker\phi\neq\emptyset$. Let $x,y\in\ker\phi$; that is, $\phi(x)=\phi(y)=1_H$. Then
\[\phi(xy^{-1})=\phi(x)\phi(y)^{-1}=1_H\]
so $xy^{-1}\in\ker\phi$. By the subgroup criterion, $\ker\phi\le G$.

Let $x\in\ker\phi$, $g\in G$. Then
\[\phi(gxg^{-1})=\phi(g)\phi(x)\phi(g^{-1})=1,\]
so $gxg^{-1}\in\ker\phi$. Hence $\ker\phi\triangleleft G$.

\item Since $\phi(1_G)=1_H$, $1_H\in\im\phi$ so $\im\phi\neq\emptyset$. Let $x,y\in\im\phi$. Then there exists $a,b\in G$ such that $\phi(a)=x$, $\phi(b)=y$. Then
\[xy^{-1}=\phi(a)\phi(b)^{-1}=\phi(ab^{-1})\]
so $xy^{-1}\in\im\phi$. By the subgroup criterion, $\im\phi\le G$.
\end{enumerate}
\end{proof}

The following result is a useful characterisation for injective homomorphisms.

\begin{lemma}
Let $\phi:G\to H$ be a homomorphism. Then $\phi$ is injective if and only if $\ker\phi=\{1_G\}$.
\end{lemma}

\begin{proof} \

\fbox{$\implies$} Suppose $\phi$ is injective. Since $\phi(1_G)=1_H$, $1_G\in\ker\phi$ so $\{1_G\}\subset\ker\phi$. 

Conversely, let $x\in\ker\phi$, so $\phi(x)=1_H$. Then $\phi(x)=1_H=\phi(1_G)$, so by injectivity $x=1_G$. Hence $\ker\phi\subset\{1_G\}$, so $\ker\phi=\{1_G\}$.

\fbox{$\impliedby$} Suppose $\ker\phi=\{1_G\}$. Suppose $\phi(a)=\phi(b)$, then $\phi(ab^{-1})=\phi(a)\phi(b^{-1})=\phi(a)\phi(a)^{-1}=1_H$. Hence $ab^{-1}\in\ker\phi=\{1_G\}$, so $ab^{-1}=1_G$ and thus $a=b$. Therefore $\phi$ is injective.
\end{proof}

\begin{lemma}
Let $\phi:G\to H$ be an isomorphism. Then the inverse map $\phi^{-1}:H\to G$ is also an isomorphism.
\end{lemma}

\begin{proof}
The inverse of a bijective map is bijective. Hence it suffices to show that $\phi^{-1}(x)\phi^{-1}(y)=\phi^{-1}(xy)$ for all $x,y\in H$.

Let $a=\phi^{-1}(x)$, $b=\phi^{-1}(y)$, $c=\phi^{-1}(xy)$; we will show that $ab=c$. Since $\phi$ is bijective, it suffices to show that $\phi(ab)=\phi(c)$.

Since $\phi$ is a homomorphism,
\[\phi(ab)=\phi(a)\phi(b)=xy=\phi(c).\]
\end{proof}

\subsection{Isomorphism Theorems}
\begin{theorem}[First isomorphism theorem]
Let $\phi:G\to H$ be a homomorphism. Then
\[G/\ker\phi\cong\im\phi(G).\]
\end{theorem}

\begin{proof}
Let $K=\ker\phi$. 
Let
\begin{align*}
\theta:G/K&\to\im\phi\\
\forall x\in G,\quad xK&\mapsto\phi(x)
\end{align*}
\begin{claim}
$\theta$ is an isomorphism.
\end{claim}
We first need to check if $\theta$ is well-defined: let $x,y\in G$. Suppose $xK=yK$. Then
\begin{align*}
&xK=yK\\
\iff& x^{-1}y\in K\\
\iff&\phi(x^{-1}y)=1_H\\
\iff&\phi(x)^{-1}\phi(y)=1_H\\
\iff&\phi(x)=\phi(y)
\end{align*}

Next we show that $\theta$ is a homomorphism: for all $x,y\in G$,
\[\theta\brac{xKyK}=\theta(xyK)=\phi(xy)=\phi(x)\phi(y)=\theta(xK)\theta(yK).\]

Finally we show that $\theta$ is bijective:
\begin{itemize}
\item $\theta$ is injective since
\[\theta(xK)=\theta(yK)\implies\phi(x)=\phi(y)\implies xK=yK.\]
\item $\theta$ is surjective, since
\begin{align*}
\im\theta&=\{\theta(xK)\mid x\in G\}\\
&=\{\phi(x)\mid x\in G\}\\
&=\im\phi.
\end{align*}
\end{itemize}
\end{proof}

\begin{theorem}[Second isomorphism theorem]
Let $A\le G$, $B\triangleleft G$. Then
\begin{enumerate}[label=(\roman*)]
\item $AB\le G$;
\item $B\triangleleft AB$;
\item $A\cap B\triangleleft A$;
\item $AB/B\cong A/(A\cap B)$.
\end{enumerate}
\end{theorem}

\begin{theorem}[Third isomorphism theorem]
Let $H,K\triangleleft G$, $H\le K$. Then $K/H\triangleleft G/H$, and
\[(G/H)/(K/H)\cong G/K.\]
If we denote the quotient by $H$ with a bar, this can be written
\[\overline{G}/\overline{K}\cong G/K.\]
\end{theorem}

\begin{theorem}[Fourth isomorphism theorem]

\end{theorem}

\begin{theorem}[Cayley's theorem]

\end{theorem}
\pagebreak

\section{Group Actions}
We move now, from thinking of groups in their own right, to thinking of how groups can move sets around---for example, how $S_n$ permutes $\{1,2,\dots,n\}$ and matrix groups move vectors.

\begin{definition}[Group action]
A \vocab{group action} of a group $G$ on a set $A$ is a map from $G\times A\to A$ (written as $g\cdot a$, for all $g\in G$, $a\in A$) satisfying the following properties:
\begin{enumerate}[label=(\roman*)]
\item $g_1\cdot(g_2\cdot a)=(g_1g_2)\cdot a$, for all $g_1,g_2\in G$, $a\in A$;
\item $1_G\cdot a=a$ for all $a\in A$.
\end{enumerate}
We say that $G$ is a group acting on a set $A$.
\end{definition}

Intuitively, a group action of $G$ on a set $A$ means that every element $g$ in $G$ acts as a permutation on $A$ in a manner consistent with the group operations in $G$. There is also a notion of left \emph{action} and \emph{right action}.

For the following defintions, let $G$ be a group, and $A\subset G$ be non-empty.

\begin{definition}[Centraliser]
The \vocab{centraliser} of $A$ in $G$ is defined by
\[C_G(A)\coloneqq\{g\in G\mid\forall a\in A,gag^{-1}=a\}.\]
\end{definition}

Since $gag^{-1}=a$ if and only if $ga=ag$, $C_G(A)$ is the set of elements of $G$ which commute with every element of $A$.

We check that $C_G(A)\le G$:
\begin{enumerate}[label=(\roman*)]
\item $e\in C_G(A)$, so $C_G(A)\neq\emptyset$.
\item Let $x,y\in C_G(A)$; that is, for all $a\in A$, $xax^{-1}=a$ and $yay^{-1}=a$. Then
\begin{align*}
(xy)a(xy)^{-1}&=(xy)a(y^{-1}x^{-1})\\
&=x(yay^{-1})x^{-1}\\
&=xax^{-1}=a
\end{align*}
so $xy\in C_G(A)$. Hence $C_G(A)$ is closed under products.
\item Let $x\in C_G(A)$; that is, for all $a\in A$, $xax^{-1}=a$. Applying $x^{-1}$ to both sides gives $ax^{-1}=x^{-1}a$. Applying $x$ to both sides gives $a=x^{-1}ax$, so $x^{-1}\in C_G(A)$. Hence $C_G(A)$ is closed under taking inverses.
\end{enumerate}

\begin{notation}
In the special case when $A=\{a\}$ we simply write $C_G(a)$ instead of $C_G(\{a\})$. In this case $a^n\in C_G(a)$ for all $n\in\ZZ$.
\end{notation}

\begin{definition}[Centre]
The \vocab{centre} of $G$ is the set of elements which commute with all the elements of $G$:
\[Z(G)\coloneqq\{g\in G\mid\forall x\in G,gx=xg\}.\]
\end{definition}

Note that $Z(G)=C_G(G)$, so the argument above proves $Z(G)\le G$ as a special case.

\begin{definition}[Normaliser]
Define $gAg^{-1}=\{gag^{-1}\mid a\in A\}$. The \vocab{normaliser} of $A$ in $G$ is
\[N_G(A)\coloneqq\{g\in G\mid gAg^{-1}=A\}.\]
\end{definition}

Notice that if $g\in C_G(A)$, then $gag^{-1}=a\in A$ for all $a\in A$ so $C_G(A)\le N_G(A)$. The proof that $N_G(A)\le G$ is similar to the one that $C_G(A)\le G$.

\begin{definition}[Stabiliser]
If $G$ is a group acting on a set $S$, $s\in S$, then the \vocab{stabiliser} of $s$ in $G$ is
\[G_s\coloneqq\{g\in G\mid g\cdot s=s\}.\]
\end{definition}

\begin{notation}
Denote the set of all fixed points to be $S^G=\{s\in S\mid\forall g\in G, gs=g\}$.
\end{notation}

We check that $G_s\le G$:
\begin{enumerate}[label=(\roman*)]
\item By definition of group action, $1_G\cdot a=a$, so $1_G\in G_s$.
\item Let $x,y\in G_s$, then
\begin{align*}
(xy)\cdot s&=x\cdot(y\cdot s)\\
&=x\cdot s=s
\end{align*}
so $xy\in G_s$. Hence $G_s$ is closed under products.
\item Let $x\in G_s$; that is, $x\cdot s=s$. Then
\begin{align*}
x^{-1}\cdot s&=x^{-1}\cdot(x\cdot s)\\
&=(x^{-1}x)\cdot s\\
&=e\cdot s=s
\end{align*}
so $x^{-1}\in G_s$. Hence $G_s$ is closed under taking inverses.
\end{enumerate}

\begin{definition}
The \emph{kernel} of the action of $G$ on $S$ is
\[\{g\in G\mid\forall s\in S, g\cdot s=s\}.\]
\end{definition}

\begin{definition}[Orbit]
Let $G$ be a group that acts on a set $S$. Define the \vocab{orbit} of a group element $s\in S$ as
\[G(s)\coloneqq\{g\cdot s\in S\mid g\in G\}.\]
\end{definition}

\subsection{Conjugation}

\subsection{Sylow's Theorem}
\begin{definition}[Sylow $p$-subgroup]
Let $G$ be a group, and let $p$ be a prime.
\begin{enumerate}[label=(\roman*)]
\item A group of order $p^\alpha$ ($\alpha\ge1$) is called a \emph{$p$-group}. Subgroups of $G$ which are $p$-groups are called \emph{$p$-subgroups}.
\item If $|G|=p^\alpha m$ ($p\nmid m$), then a subgroup of order $p^\alpha$ is called a \vocab{Sylow $p$-subgroup} of $G$.
\end{enumerate}
\end{definition}

\begin{notation}
The set of Sylow $p$-subgroups of $G$ is denoted by $Syl_p(G)$, and the number of Sylow $p$-subgroups of $G$ is denoted by $n_p(G)$ (or just $n_p$ when $G$ is clear from the context).
\end{notation}

\begin{theorem}[Sylow's theorem]
Let $|G|=p^\alpha m$, where $p$ is a prime and $p\nmid m$.
\begin{enumerate}[label=(\roman*)]
\item Sylow $p$-subgroups of $G$ exist, i.e. $Syl_p(G)\neq\emptyset$.
\item If $P$ is a Sylow $p$-subgroup of $G$, and $Q$ is any $p$-subgroup of $G$, then there exists $g\in G$ such that $Q\le gPg^{-1}$, i.e. $Q$ is contained in some conjugate of $P$. In particular, any two Sylow $p$-subgroups of $G$ are conjugate in $G$.
\item $n_p\equiv1\pmod p$. Furthermore, $n_p$ is the index in $G$ of the normaliser $N_G(P)$ for any Sylow $p$-subgroup $P$, hence $n_p\mid m$.
\end{enumerate}
\end{theorem}
\pagebreak

\section{Group Product, Finite Abelian Groups}
\begin{definition}[Direct product]
The \vocab{direct product} $G_1\times\cdots\times G_n$ of the groups $(G_1,\ast_1),\dots,(G_n,\ast_n)$ is the Cartesian product
\[G_1\times\cdots\times G_n\coloneqq\{(g_1,\dots,g_n)\mid g_i\in G_i\}\]
with operation defined componentwise:
\[(g_1,\dots,g_n)\ast(h_1,\dots,h_n)=(g_1\ast_1 h_1,\dots,g_n\ast_n h_n).\]
\end{definition}

\begin{proposition}
If $G_1,\dots,G_n$ are groups, then
\[|G_1\times\cdots\times G_n|=|G_1|\:|G_2|\cdots|G_n|.\]
\end{proposition}

\begin{proof}
Let $G=G_1\times\cdots\times G_n$. The proof that the group axioms hold for $G$ is straightforward since each axiom is a consequence of the fact that the same axiom holds for each $G_i$, and the operation on $G$ defined componentwise.

The number of $n$-tuples in $G$ follows from simple combinatorics.
\end{proof}
\pagebreak

\section*{Exercises}
\addcontentsline{toc}{section}{Exercises}
\begin{exercise}
Show that any two cyclic groups of the same order are isomorphic.
\end{exercise}

\begin{solution}
Suppose $\langle x\rangle$ and $\langle y\rangle$ are both cyclic groups of order $n$. We first prove the case where $n<\infty$. We claim that the map $\phi:\langle x\rangle\to\langle y\rangle$ which sends $x^k\mapsto y^k$ is an isomorphism.
\begin{lemma*}
Let $G$ be a group, $g\in G$, let $m,n\in\ZZ$. Denote $d=\gcd(m,n)$. If $g^n=1$ and $g^m=1$, then $g^d=1$.
\end{lemma*}
\begin{proof}
By Bezout's lemma, since $d=\gcd(m,n)$, then there exists $q,r\in\ZZ$ such that $qm+rn=d$. Thus
\[g^d=g^{qm+rn}=\brac{g^m}^q\brac{g^n}^r=1.\]
\end{proof}
We first show that $\phi$ is well-defined; that is, $x^r=x^s\implies \phi(x^r)=\phi(x^s)$. Note that $x^{r-s}=e$, so by the above lemma, $n\mid r-s$. Write $r=tn+s$ for some $t\in\ZZ$, so
\[\phi(x^r)=\phi(x^{tn+s})=y^{tn+s}=(y^n)^ty^s=y^s=\phi(x^s).\]

We then show that $\phi$ is a homomorphism:
\[\phi(x^ax^b)=\phi(x^{a+b})=y^{a+b}=y^ay^b=\phi(x^a)\phi(x^b).\]

Finally we show that $\phi$ is bijective. Since the element $y^k$ of $\langle y\rangle$ is in the image of $x^k$ under $\phi$, $\phi$ is surjective. Since both groups have the same finite order, any surjection from one to the other is a bijection. Therefore $\phi$ is an isomorphism.

We now prove the case where $n=\infty$. If $\langle x\rangle$ is an infinite cyclic group, let $\phi:\ZZ\to\langle x\rangle$ be defined by $\phi(k)=x^k$. (This map is well-defined since there is no ambiguity in the representation of elements in the domain.)

Since $x^a\neq x^b$ for all distinct $a,b\in\ZZ$, $\phi$ is injective. By definition of a cyclic group, $\phi$ is surjective. As above, the laws of exponents ensure $\phi$ is a homomorphism. Hence $\phi$ is an isomorphism.
\end{solution}
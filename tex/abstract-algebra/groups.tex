\chapter{Groups}\label{chap:groups}
\section{Introduction to Groups}
\subsection{Definitions and Properties}
\begin{definition}[Binary operation]
A \vocab{binary operation} $\ast$ on a set $G$ is a function $\ast:G\times G\to G$. For any $a,b\in G$, we write $a \ast b$ for the image of $(a,b)$ under $\ast$.

$\ast$ is \emph{associative} on $G$ if $(a\ast b)\ast c=a\ast(b\ast c)$ for all $a,b,c\in G$.

$\ast$ is \emph{commutative} on $G$ if $a\ast b=b\ast a$ for all $a,b\in G$.
\end{definition}

\begin{definition}[Group]
A \vocab{group}\index{group} $(G,\ast)$ consists of a set $G$ and a binary operation $\ast$ on $G$ satisfying the following group axioms:
\begin{enumerate}[label=(\roman*)]
\item Associativity: $a\ast(b\ast c)=(a\ast b)\ast c$ for all $a,b,c\in G$.
\item Identity: there exists identity element $e\in G$ such that $a\ast e=e\ast a=a$ for all $a\in G$.
\item Invertibility: for all $a\in G$, there exists inverse $c\in G$ such that $a\ast c=c\ast a=e$.
\end{enumerate}

$G$ is \vocab{abelian}\footnote{after the Norwegian mathematician Niels 
Abel (1802--1829)} if the operation is commutative; it is \emph{non-abelian} if otherwise.
\end{definition}

\begin{remark}
When verifying that $(G,\ast)$ is a group we have to check (i), (ii), (iii) above and also that $\ast$ is a binary operation closed in $G$---that is, $a\ast b\in G$ for all $a,b\in G$.
\end{remark}

\begin{notation}
We simply denote a group $(G,\ast)$ by $G$ if the operation is clear.
\end{notation}

\begin{notation}
We abbreviate $a\ast b$ to just $ab$ if the operation is clear.
\end{notation}

\begin{notation}
Since $\ast$ is associative, we omit unnecessary parentheses and write $(ab)c=a(bc)=abc$.
\end{notation}

\begin{notation}
For any $a\in G$, $n\in\ZZ^+$ we denote $a^n=\underbrace{a\cdot a\cdots a}_\text{$n$ times}$.
\end{notation}

\begin{notation}
We write $(\ZZ,+)$, $(\QQ,+)$, $(\RR,+)$, $(\CC,+)$ as simply $\ZZ$, $\QQ$, $\RR$, $\CC$.
\end{notation}

\begin{example}
\begin{itemize}
\item $\ZZ$, $\QQ$, $\RR$, $\CC$ are groups, with identity $0$ and (additive) inverse $-a$ for all $a$.
\item $\QQ\setminus\{0\}$, $\RR\setminus\{0\}$, $\CC\setminus\{0\}$, $\QQ^+$, $\RR^+$ are groups under $\times$, with identity $1$ and (multiplicative) inverse $\frac{1}{a}$ for all $a$; $\ZZ\setminus\{0\}$ is not a group under $\times$, because all elements except for $\pm1$ do not have an inverse in $\ZZ\setminus\{0\}$.
\item For $n\in\ZZ^+$, $\ZZ_n$ is an abelian group under $+$.
\item For $n\in\ZZ^+$, $(\ZZ_n)^\times$ is an abelian group under multiplication.
\end{itemize}
\end{example}

\begin{proposition}
Let $G$ be a group. Then
\begin{enumerate}[label=(\roman*)]
\item the identity of $G$ is unique,
\item for each $a\in G$, $a^{-1}$ is unique,
\item $(a^{-1})^{-1}=a$ for all $a\in G$,
\item $(ab)^{-1}=b^{-1}a^{-1}$,
\item for any $a_1,\dots,a_n\in G$, $a_1\cdots a_n$ is independent of how we arrange the parantheses (generalised associative law).
\end{enumerate}
\end{proposition}

\begin{proof} \
\begin{enumerate}[label=(\roman*)]
\item Suppose otherwise, then $e$ and $e^\prime$ are identites of $G$. We have
\[e=ee^\prime=e^\prime\]
where the first equality holds as $e^\prime$ is an identity, and the second equality holds as $e$ is an identity. Since $e=e^\prime$, the identity is unique.
\item Suppose otherwise, then $b$ and $c$ are both inverses of $a$. Let $e$ be the identity of $G$. Then $ab=e$, $ca=e$. Thus
\[c=ce=c(ab)=(ca)b=eb=b.\]
Hence the inverse is unique.
\item To show $(a^{-1})^{-1}=a$ is exactly the problem of showing that $a$ is the inverse of $a^{-1}$, which is by definition of the inverse (with the roles of $a$ and $a^{-1}$ interchanged).
\item Let $c=(ab)^{-1}$. Then $(ab)c=e$, or $a(bc)=e$ by associativity, which gives $bc=a^{-1}$ and thus $c=b^{-1}a^{-1}$ by multiplying $b^{-1}$ on both sides.
\item The result is trivial for $n=1,2,3$. For all $k<n$ assume that any $a_1\cdots a_k$ is independent of parantheses. Then
\[(a_1\cdots a_n)=(a_1\cdots a_k)(a_{k+1}\cdots a_n).\]
Then by assumption both are independent of parentheses since $k,n-k<n$ so by induction we are done.
\end{enumerate}
\end{proof}

\begin{notation}
Since the inverse is unique, we denote the inverse of $a\in G$ by $a^{-1}$.
\end{notation}

\begin{proposition}[Cancellation law]
Let $a,b\in G$. Then the equations $ax=b$ and $ya=b$ have unique solutions for $x,y\in G$. In particular, we can cancel on the left and right.
\end{proposition}

\begin{proof}
We can solve $ax=b$ by applying $a^{-1}$ to both sides of the equation to get $x=a^{-1}b$. The uniqueness of $x$ follows because $a^{-1}$ is unique. A similar case holds for $ya=b$.
\end{proof}

\begin{definition}[Order of a group]
Let $G$ be a group. Its cardinality $|G|$ is called the \vocab{order} of $G$. We say that a group $G$ is a \emph{finite group} if $|G|<\infty$.
\end{definition}

One way to represent a finite group is by means of the group table or Cayley table\footnote{after
the English mathematician Arthur Cayley (1821 -- 1895)}. Let $G=\{e,g_2,g_3,\dots,g_n\}$ be a finite group. The Cayley table (or group table) of $G$ is a square grid which contains all the possible products of two elements from $G$. The product $g_ig_j$ appears in the $i$-th row and $j$-th column of the Cayley table.

\begin{remark}
Note that a group is abelian if and only if its Cayley table is symmetric about the main (top-left to bottom-right) diagonal.
\end{remark}

\subsection{Examples}
\begin{example}[Product group]
Let $(G,\ast_G)$ and $(H,\ast_H)$ be groups. Then the operation $\ast$ is defined on $G\times H$ by
\[(g_1,h_1)\ast(g_2,h_2)=(g_1\ast_G g_2,h_1\ast_H h_2)\]
for all $g_1,g_2\in G$, $h_1,h_2\in H$. $(G\times H, \ast)$ is called the \emph{product group} of $G$ and $H$.

We check that the product group is a group:
\begin{enumerate}[label=(\roman*)]
\item Since $\ast_G$ and $\ast_H$ are both associative binary operations, it follows that $\ast$ is also an associative binary operation on $G \times H$.
\item We also note
\[e_{G\times H}=(e_G,e_H),\quad(g,h)^{-1}=(g^{-1},h^{-1})\]
as for any $g \in G$, $h \in H$,
\[(e_G,e_H)\ast(g,h)=(g,h)=(g,h)\ast(e_G,e_H).\]
\item As for identity,
\[(g^{-1},h^{-1})\ast(g,h)=(e_G,e_H)=(g,h)\ast(g^{-1},h^{-1}).\]
\end{enumerate}
\end{example}

\begin{example}[Dihedral groups]
An important family of groups is the \vocab{dihedral groups}. For $n\in\ZZ^+$, $n\ge3$, let $D_{2n}$ be the set of symmetries\footnote{a symmetry is any rigid motion of the $n$-gon which can be effected by taking a copy of the $n$-gon, moving this copy in any fashion in $3$-space and then placing the copy back on the original $n$-gon so it exactly covers it. A symmetry can be a reflection or a rotation.} of a regular $n$-gon.

\begin{remark}
Here ``D'' stands for ``dihedral'', meaning two-sided.
\end{remark}

To visualise this, we first choose a labelling of the $n$ vertices. Then each symmetry $S$ can be described uniquely by the corresponding permutation $\sigma$ of $\{1,2,\dots,n\}$ where if the symmetry $s$ puts vertex $i$ in the place where vertex $j$ was originally, then $\sigma$ is the permutation sending $i$ to $j$.

We now make $D_{2n}$ into a group. For $S,T\in D_{2n}$, define the binary operation $ST$ to be the symmetry obtained by first applying $T$ then $S$ to the $n$-gon (this is analagous to function composition). If $S$ and $T$ effect the permutations $\sigma$ and $\tau$ respectively on the vertices, then $ST$ effects $\sigma\circ\tau$.

\begin{enumerate}[label=(\roman*)]
\item The binary operation on $D_{2n}$ is associative since the composition of functions is associative.
\item The identity of $D_{2n}$ is the identity symmetry, which leaves all vertices fixed, denoted by $1$.
\item The inverse of $S\in D_{2n}$ is the symmetry which reverses all rigid motions of $S$ (so if $S$ effects permutation $\sigma$ on the vertices, $S^{-1}$ effects $\sigma^{-1}$).
\end{enumerate}

Let $r$ be the rotation clockwise about the origin by $\frac{2\pi}{n}$ radians, let $s$ be the reflection about the line of symmetry through the first labelled vertex and the origin.

\begin{proposition*} \
\begin{enumerate}[label=(\roman*)]
\item $|r|=n$
\item $|s|=2$
\item $s\neq r^i$ for all $i$
\item $sr^i\neq sr^j$ for all $i\neq j$ ($0\le i,j\le n-1$), so
\[D_{2n}=\{1,r,\dots,r^{n-1},s,sr,\dots,sr^{n-1}\}\]
and thus $|D_{2n}|=2n$.
\item $rs=sr^{-1}$
\item $r^is=sr^{-i}$
\end{enumerate}
\end{proposition*}

\begin{proof} \
\begin{enumerate}[label=(\roman*)]
\item It is obvious that $1,r,r^2,\dots,r^{n-1}$ are all distinct and $r^n=1$, so $|r|=n$.
\item This is fairly obvious: either reflect or do not reflect.
\item This is also obvious: the effect of any reflection cannot be obtained from any form of rotation.
\item Just cancel on the left and use the fact that $|r|=n$. We assume that $i\not\equiv j\pmod n$.
\item Omitted.
\item By (5), this is true for $i=1$. Assume it holds for $k<n$. Then $r^{k+1}s=r(r^ks)=rsr^{-k}$. Then $rs=sr^{-1}$ so $rsr^{-k}=sr^{-1}r^{-k}=sr^{-k-1}$ so we are done.
\end{enumerate}
\end{proof}

A presentation for the dihedral group $D_{2n}$ using generators and relations is
\[D_{2n}=\langle r,s\mid r^n=s^2=1,rs=sr^{-1}\rangle.\]
\end{example}

\begin{example}[Permutation groups]
Let $S$ be a non-empty set. A bijection $S\to S$ is called a \emph{permutation} of $S$; the set of permutations of $S$ is denoted by $\Sym(S)$.

We now show that $\Sym(S)$ is a group under function composition $\circ$; $(\Sym(S),\circ)$ is the \vocab{symmetric group} on $S$. Note that $\circ$ is a binary operation on $\Sym(S)$ since if $\sigma:S\to S$ and $\tau:S\to S$ are both bijections, then $\sigma\circ\tau$ is also a bijection from $S$ to $S$.
\begin{enumerate}[label=(\roman*)]
\item Function composition is associative so $\circ$ is associative.
\item The identity of $\Sym(S)$ is the identity map $1$, defined by $1(a)=a$ for all $a\in S$.
\item For every permutation $\sigma$, $\sigma$ is bijective and thus invertible, so there exists a (2-sided) inverse $\sigma^{-1}:S\to S$ satisfying $\sigma\circ\sigma^{-1}=\sigma^{-1}\circ\sigma=1$.
\end{enumerate}

In the special case where $S=\{1,2,\dots,n\}$, the symmetric group on $S$ is called the \emph{symmetric group of degree $n$}, denoted by $S_n$.

\begin{proposition*}
If $|S|\ge3$ then $\Sym(S)$ is non-abelian.
\end{proposition*}

\begin{proof}
Let $S=\{x_1,x_2,x_3\}$ where three elements are distinct.
\end{proof}

\begin{proposition*}
$|S_n|=n!$
\end{proposition*}

\begin{proof}
Obvious, since there are $n!$ permutations of $\{1,2,\dots,n\}$.
\end{proof}
\end{example}

\begin{example}[Matrix groups]
For $n\in\ZZ^+$, let $GL_n(\FF)$ be the set of all $n\times n$ invertible matrices whose entries are in $\FF$:
\[GL_n(\FF)=\{A\in M_{n\times n}(\FF)\mid\det(A)\neq0\}.\]

We show that $GL_n(\FF)$ is a group under matrix multiplication; $GL_n(\FF)$ is the \vocab{general linear group} of degree $n$.
\begin{enumerate}[label=(\roman*)]
\item Since $\det(AB)=\det(A)\cdot\det(B)$, it follows that if $\det(A)\neq0$ and $\det(B)\neq0$, then $\det(AB)\neq0$, so $GL_n(\FF)$ is closed under matrix multiplication.
\item Matrix multiplication is associative.
\item $\det(A)\neq0$ if and only if $A$ has an inverse matrix, so each $A\in GL_n(\FF)$ has an inverse $A^{-1}\in GL_n(\FF)$ such that
\[AA^{-1}=A^{-1}A=I\]
where $I$ is the $n\times n$ identity matrix.
\end{enumerate}
\end{example}

\begin{example}[Quaternion group]
The \vocab{Quaternion group} $Q_8$ is defined by
\[Q_8=\{1,-1,i,-i,j,-j,k,-k\}\]
with product $\cdot$ computed as follows:
\begin{itemize}
\item $1\cdot a=a\cdot 1=a$ for all $a\in Q_8$
\item $(-1)\cdot(-1)=1$
\item $(-1)\cdot a=a\cdot(-1)=-a$ for all $a\in Q_8$
\item $i\cdot i=j\cdot j=k\cdot k=-1$
\item $i\cdot j=k$, $j\cdot i=-k$, $j\cdot k=i$, $k\cdot j=-i$, $k\cdot i=j$, $i\cdot k=-j$
\end{itemize}
Note that $Q_8$ is a non-abelian group of order $8$.
\end{example}

\subsection{Cyclic Groups and Order}
\begin{definition}[Cyclic group]
Let $G$ be a group, $g\in G$. If for every $x\in G$, there exists $n\in\ZZ$ such that $g^n=x$ then $g$ is the \vocab{generator} of $G$; denote $G=\langle g\rangle$.

$G$ is \vocab{cyclic} if there is a generator for $G$ in $G$.
\end{definition}

\begin{remark}
A cyclic group may have more than one generator. For example, if $G=\langle g\rangle$, then also $G=\langle g^{-1}\rangle$ because $(g^{-1})^n=g^{-n}\in G$ for $n\in\ZZ$ so does $-n$, thus
\[\{g^n\mid n\in\ZZ\}=\{(g^{-1})^n\mid n\in\ZZ\}.\] 
\end{remark}

\begin{example}
$\ZZ$ is a cyclic group with generators $1$ and $-1$.
\end{example}

\begin{notation}
For each $n\in\ZZ^+$, $C_n$ denotes the cyclic group of order $n$:
\[C_n=\{e,g,g^2,\dots,g^{n-1}\}\]
which satisfy $g^n=e$. Thus given two elements in $C_n$, we define
\[g^i\ast g^j=\begin{cases}
g^{i+j}&(0\le i+j<n)\\
g^{i+j-n}&(n\le i+j\le 2n-2)
\end{cases}\]
\end{notation}

\begin{proposition}
Cyclic groups are abelian.
\end{proposition}

\begin{proof}
Let $G$ be a cyclic group. For $g^i,g^j\in G$, by the laws of exponents,
\[g^i g^j=g^{i+j}=g^j g^i.\]
\end{proof}

\begin{definition}[Order]
Let $G$ be a group, $g\in G$. If there is a positive integer $k$ such that $g^k=e$, then the \vocab{order} of $g$ is defined as
\[o(g)\coloneqq\min\{m>0\mid g^m=e\}.\]
Otherwise we say that the order of $g$ is infinite.
\end{definition}

\begin{proposition}
If $G$ is finite, then $o(g)$ is finite for each $g\in G$.
\end{proposition}

\begin{proof}
Consider the list
\[g,g^2,g^3,g^4,\dots\in G.\]
As $G$ is finite, then this list must have repeats. Hence there are integers $i>j$ such that $g^i=g^j$. So $g^{i-j}=e$ showing that $\{m>0\mid g^m=e\}$ is non-empty and so has a minimal element.
\end{proof}

\begin{proposition}
If $g\in G$ and $o(g)$ is finite, then $g^n=e$ if and only if $o(g)\mid n$.
\end{proposition}

\begin{proof} \

\fbox{$\impliedby$} Suppose $o(g)\mid n$. Then $n=ko(g)$ for some $k\in\ZZ$, so
\[g^n=\brac{g^{o(g)}}^k=e^k=e.\]
\fbox{$\implies$} Suppose $g^n=e$. By the division algorithm, there exists integers $q,r$ such that $n=qo(g)+r$, where $0\le r<o(g)$. Then
\[g^r=g^{n-qo(g)}=g^n\brac{g^{o(g)}}^{-q}=e.\]
By the minimality of $o(g)$, we must have $r=0$, and so $n=qo(g)$ implies $o(g)\mid n$.
\end{proof}

\begin{corollary}
Let $G$ be a cyclic group, $g\in G$. Then $g^k=g^m$ if and only if $m\equiv k\pmod{o(g)}$.
\end{corollary}

\begin{proposition}
If $G=\langle g\rangle$, then $|G|=o(g)$ (where if one side of this equality is infinite, so is the other). More specifically,
\begin{enumerate}[label=(\roman*)]
\item if $|G|=n<\infty$, then $g^n=e$ and $e,g,g^2,\dots,g^{n-1}$ are all the distinct elements of $G$;
\item if $|G|=\infty$, then $g^n\neq e$ for all $n\neq0$, and $g^a\neq g^b$ for all $a,b\in\ZZ$, $a\neq b$.
\end{enumerate}
\end{proposition}

\begin{proof} \
\begin{enumerate}[label=(\roman*)]
\item We first show that all the elements are distinct. If $g^a=g^b$ for $0\le a<b<n$, then $g^{b-a}=e$, which contradicts the minimality of $o(g)$. Thus $G$ has at least $n$ elements and it remains to show that these are all of them.

For the element $g^t$, by the division algorithm, we can write $t=qn+r$ where $0\le r<n$. Then
\[g^t=g^{qn+r}=\brac{g^n}^q g^r=g^r\in\{e,g,g^2,\dots,g^{n-1}\}\]
since $0\le r<n$.

\item Suppose $o(g)=\infty$, so no positive power of $g$ is the identity. If $g^a=g^b$ for some $a,b\in\ZZ$, $a<b$, then $g^{b-a}=e$, contradicting the previous statement. Thus distinct powers of $g$ are distinct elements of $G$, so $|G|=\infty$.
\end{enumerate}
\end{proof}

Note that a given cyclic group may have more than one generator. The next results determine precisely which powers of $g$ generate the group $\langle g\rangle$.

\begin{proposition}
Let $G$ be a group, $g\in G$. Let $a\in\ZZ\setminus\{0\}$.
\begin{enumerate}[label=(\roman*)]
\item If $o(g)=\infty$, then $o(g^a)=\infty$.
\item If $o(g)=n<\infty$, then $o(g)=\dfrac{n}{\gcd(n,a)}$. In particular, if $a\mid n$, then $o(g^a)=\dfrac{n}{a}$.
\end{enumerate}
\end{proposition}

\begin{proof} \
\begin{enumerate}[label=(\roman*)]
\item Suppose, for a contradiction, that $o(g)=\infty$ but $o(g^a)=m<\infty$. Then by definition of order,
\[e=(g^a)^m=g^{am}.\]
Also,
\[g^{-am}=\brac{g^{am}}^{-1}=e^{-1}=e.\]
Now one of $am$ or $-am$ is positive (since $a\neq0$ and $m\neq0$), so some positive power of $g$ is the identity. This contradicts the hypothesis $o(g)=\infty$.

\item 
\end{enumerate}
\end{proof}

\begin{proposition}
Let $G=\langle g\rangle$.
\begin{enumerate}[label=(\roman*)]
\item If $o(g)=\infty$, then $G=\langle g^a\rangle$ if and only if $a=\pm1$.
\item If $o(g)=n<\infty$, then $G=\langle x^a\rangle$ if and only if $\gcd(a,n)=1$. In particular, the number of generators of $G$ is $\phi(n)$ (where $\phi$ is Euler's totient function).
\end{enumerate}
\end{proposition}

\subsection{Subgroups}
\begin{definition}[Subgroup]
Let $G$ be a group. A non-empty $H\subset G$ is a \vocab{subgroup}\index{subgroup} of $G$, denoted $H\le G$, if $H$ is closed under products and inverses; that is,
\begin{enumerate}[label=(\roman*)]
\item $e\in H$;
\item $xy\in H$ for all $x,y\in H$;
\item $x^{-1}\in H$ for all $x\in H$.
\end{enumerate}
\end{definition}

\begin{remark}
If $\ast$ is an associative (respectively, commutative) binary operation on $G$ and $\ast$ is restricted to some $H\subset G$ is a binary operation on $H$, then $\ast$ is automatically associative (respectively, commutative) on $H$ as well.
\end{remark}

The following result provides a convenient method to determine if a given subset of a group is a subgroup.

\begin{lemma}[Subgroup criterion]
Let $G$ be a group, $H\subset G$ is non-empty. Then $H\le G$ if and only if $xy^{-1}\in H$ for all $x,y\in H$. 
\end{lemma}

\begin{proof} \

\fbox{$\implies$} If $H\le G$, then we are done, by definition of subgroup.

\fbox{$\impliedby$} We want to prove that for non-empty $H\subset G$, if $xy^{-1}\in H$ for all $x,y\in H$, then $H\le G$, by checking the group axioms:
\begin{enumerate}[label=(\roman*)]
\item Since $H\neq\emptyset$, take $x\in H$, let $y=x$, then $e=xx^{-1}\in H$, so $H$ contains the identity of $G$.
\item Since $e\in H$, $x\in H$, then $x^{-1}\in H$ so $H$ is closed under taking inverses.
\item For any $x,y\in H$, $x,y^{-1}\in H$, so by (ii), $x(y^{-1})^{-1}=xy\in H$, so $H$ is closed under multiplication.
\end{enumerate}
\end{proof}

\begin{proposition}
Let $G=\langle g\rangle$ be a cyclic group. Then every subgroup of $G$ is cyclic.
\end{proposition}

\begin{proposition}
Let $G$ be a group, $H,K\le G$. Then $H\cap K\le G$.
\end{proposition}

\begin{proof}
Apply the subgroup criterion:
\begin{enumerate}[label=(\roman*)]
\item Since $e_G\in H$ and $e_G\in K$, we have $e_G\in H\cap K$, so $H\cap K\neq\emptyset$.
\item Let $a,b\in H\cap K$. Then $a,b\in H$ and $a,b\in K$. Since $H,K\le G$, by the subgroup criterion, $ab^{-1}\in H$ and $ab^{-1}\in K$, so $ab^{-1}\in H\cap K$.
\end{enumerate}
\end{proof}

\begin{corollary}
Let $G$ be a group, $\{H_i\mid i\in I\}$ is a collection of subgroups of $G$. Then
\[\bigcap_{i\in I}H_i\le G.\]
\end{corollary}

Thus we may make the following definition.
\begin{definition}[Subgroup generated by subset of group]
Let $G$ be a group, $S\subset G$. The \vocab{subgroup generated by $S$}, denoted by $\langle S\rangle$, is the smallest subgroup of $G$ which contains $S$.

If $g\in G$, then we write $\langle g\rangle$ (rather than the more accurate but cumbersome $\langle\{g\}\rangle$).

If $\langle S\rangle=G$, then the elements of $S$ are said to be \emph{generators} of $G$.
\end{definition}

\begin{example}
If $G$ is abelian and $g,h\in G$ then
\[\langle g,h\rangle=\{g^rh^s\mid r,s\in\ZZ\}.\]
\begin{proof}
Certainly $\{g^rh^s\mid r,s\in\ZZ\}\subset\langle g,h\rangle$. However, when $G$ is abelian (or indeed if just $gh=hg$), then $\{g^rh^s\mid r,s\in\ZZ\}$ is a subgroup as follows:
\begin{enumerate}[label=(\roman*)]
\item $e=g^0h^0\in\{g^rh^s\mid r,s\in\ZZ\}$
\item $(g^kh^l)(g^Kh^L)=g^{k+K}h^{l+L}\in\{g^rh^s\mid r,s\in\ZZ\}$
\item $(g^kh^l)^{-1}=h^{-l}g^{-k}=g^{-k}h^{-l}\in\{g^rh^s\mid r,s\in\ZZ\}$
\end{enumerate}
\end{proof}
\end{example}
\pagebreak

\section{Cosets and Lagrange's Theorem}
\begin{definition}[Coset]\index{coset}
Let $H\le G$. For $a\in G$, a \vocab{left coset}\index{coset!left coset} and \vocab{right coset}\index{coset!right coset} of $H$ in $G$ are
\begin{align*}
aH&\coloneqq\{ah\mid h\in H\}\\
Ha&\coloneqq\{ha\mid h\in H\}
\end{align*}
Any element of a coset is called a \emph{representative} for the coset.
\end{definition}

The set of left cosets is given by
\[(G/H)_{l}\coloneqq\{aH\mid a\in G\}.\]
Similarly, the set of right cosets is given by
\[(G/H)_{r}\coloneqq\{Ha\mid a\in G\}.\]

\begin{lemma}
Let $H\le G$. Then $aH=H$ if and only if $a\in H$. (Similarly, $Ha=H$ if and only if $a\in H$.)
\end{lemma}

\begin{proof} \

\fbox{$\implies$} Suppose $aH=H$. Then $ah\in H$ for some $h\in H$. Let $k=ah$, then $a=kh^{-1}\in H$.

\fbox{$\impliedby$} Let $a\in H$. Then $aH\subset H$.

Since $a^{-1}\in H$, $a^{-1}H\subset H$. Then $H=eH=(aa^{-1})H=a(a^{-1})H\subset aH$. Hence $aH=H$.
\end{proof}

The next result shows when two cosets are equal.
\begin{lemma}
Let $H\le G$, $a,b\in G$. Then $aH=bH$ if and only if $a^{-1}b\in H$.
\end{lemma}

\begin{proof}
\begin{align*}
aH=bH&\iff a^{-1}(aH)=a^{-1}bH\\
&\iff (a^{-1}a)H=(a^{-1}b)H\\
&\iff H=(a^{-1}b)H
\end{align*}
Note that from the previous result, $H=(a^{-1}b)H$ if and only if $a^{-1}b\in H$.
\end{proof}

\begin{proposition}
Let $H\le G$. Then $(G/H)_l$ forms a partition of $G$. (Similar remarks hold for right cosets.)
\end{proposition}

We need to prove the following.
\begin{enumerate}[label=(\roman*)]
\item For all $a\in G$, $aH\neq\emptyset$.
\item $\bigcup_{a\in G}aH=G$.
\item For every $a,b\in G$, $aH\cap bH=\emptyset$ or $aH=bH$.
\end{enumerate}

\begin{proof} \
\begin{enumerate}[label=(\roman*)]
\item Since $H\le G$, $e\in H$. Thus for all $a\in G$, $a=ae\in aH$ so $aH\neq\emptyset$.
\item For all $a\in G$, $aH\subset G$, then $\bigcup_{a\in G}aH\subset G$. Note that $a\in G$ implies $a=ae\in aH$, and so $G=\bigcup_{a\in G}g\subset\bigcup_{a\in G}aH$. By double inclusion we are done.
\item If $aH\cap bH=\emptyset$, then we are done. If $aH\cap bH\neq\emptyset$ we need to show $aH=bH$. Let $x\in G$ such that $x\in aH\cap bH$. Then $x=ah_1=bh_2$ for $h_1,h_2\in H$ so $h_1=a^{-1}bh_2$. Notice that $a^{-1}b=h_1h_2^{-1}\in H$ and thus $aH=bH$.
\end{enumerate}
\end{proof}

\begin{definition}[Index]
The number of left cosets of $H$ in $G$ is called the \vocab{index} of $H$ in $G$, denoted by $|G:H|$.
\end{definition}

The following result shows that $H$ partitions $G$ into equal-sized parts.

\begin{lemma}
The cosets of $H$ in $G$ are the same size as $H$; that is, for all $a\in G$, $|aH|=|H|$.
\end{lemma}

\begin{proof}
Let $f:H\to aH$ which sends $h\mapsto ah$. For $h_1,h_2\in H$,
\begin{align*}
f(h_1)=f(h_2)
&\implies ah_1=ah_2\\
&\implies a^{-1}ah_1=a^{-1}ah_2\\
&\implies h_1=h_2
\end{align*}
thus $f$ is an injective mapping. Note that $f$ is surjective by the definition of $aH$. Since $f$ is bijective, $|H|=|aH|$.
\end{proof}

\begin{theorem}[Lagrange's theorem]
Let $G$ be a finite group, $H\le G$. Then $|G|=|H|\:|G:H|$.
\end{theorem}

\begin{proof}
Let $|H|=n$, and let $|G:H|=k$. Since $G$ is partitioned into $k$ disjoint subsets, each of which has cardinality $n$, we have $|G|=kn$, or
\[|G|=|H|\:|G:H|\]
as desired.
\end{proof}

\begin{theorem}[Fermat's little theorem]
For every finite group $G$, for all $a \in G$, $a^{|G|}=e$.
\end{theorem}

\begin{proof}
Consider the subgroup $H$ generated by $a$; that is,
\[H=\{a^i\mid i\in\ZZ\}.\]
Since $G$ is finite and $|H|<|G|$, $H$ must be finite, so the infinite sequence $a^0=e,a^1,a^2,a^3,\dots$ must repeat, say $a^i=a^j$ ($i<j$). Let $k=j-i$. Multiplying both sides by $a^{-i}=\brac{a^{-1}}^i$, we get $a^{j-i} = a^k = e$. Suppose $k$ is the least positive integer for which this holds. Then
\[H=\{a^0,a^1,a^2,\dots,a^{k-1}\},\]
and thus $|H|=k$. By Lagrange's theorem, $k$ divides $|G|$, so
\[a^{|G|}=(a^k)^\frac{|G|}{k}=e.\]
\end{proof}

\begin{theorem}[Fermat--Euler Theorem (or Euler's totient theorem)]
If $a$ and $N$ are coprime, then $a^{\phi(N)}\equiv1\pmod N$, where $\phi$ is Euler's totient function.
\end{theorem}

\begin{proposition}
A group of prime order is cyclic.
\end{proposition}

\begin{definition}
Let $H,K\le G$, define
\[HK=\{hk\mid h\in H,k\in K\}.\]
\end{definition}

\begin{proposition}
If $H,K\le G$ are finite groups, then
\[|HK|=\frac{|H||K|}{|H\cap K|}.\]
\end{proposition}

\begin{proof}
Notice that $HK$ is a union of left cosets of $K$, namely
\[HK=\bigcup_{h\in H}hK.\]

\end{proof}
\pagebreak

\section{Normal Subgroups, Quotient Groups}
\begin{definition}[Normal subgroup]
Let $G$ be a group. $H\le G$ is a \vocab{normal subgroup} of $G$, denoted by $H\triangleleft G$, if
\[aH=Ha\quad(\forall a\in G)\]
\end{definition}

If $G$ has no non-trivial normal subgroup, then $G$ is a \emph{simple group}.

\begin{remark}
This does \emph{not} mean that $ah=ha$ for all $a\in G$, $h\in H$ or that $G$ is abelian. Although we can easily see that all subgroups of abelian groups are normal. In general, a left coset does not equal the right coset.
\end{remark}

\begin{lemma}
The following are equivalent.
\begin{enumerate}[label=(\roman*)]
\item $H\triangleleft G$.
\item $ghg^{-1}\in H$ for all $g\in G$, $h\in H$.
\item $gHg^{-1}=H$ for all $g\in G$.
\end{enumerate}
\end{lemma}

\begin{proof} \

\fbox{(i)$\iff$(ii)} In the forward direction, $aH=Ha$ for all $a\in G$. Let $g\in G$, $x\in H$. Then $gH=Hg$ so $gx=h^\prime g$ for some $h^\prime\in H$. Then $gxg^{-1}=h^\prime gg^{-1}=h^\prime\in H$.

In the reverse direction, $ghg^{-1}\in H$ for all $g\in G$, $h\in H$. Fix $g$. Then $ghg^{-1}\in H$ implies $gh\in Hg$ for all $h\in H$. So $gH\subset Hg$. Similarly $gH\supset Hg$, so $gH=Hg$.

\fbox{(i)$\iff$(iii)} $H\triangleleft G$ if and only if for all $g\in G$,
\begin{align*}
gH=Hg&\iff(gH)g^{-1}=(Hg)g^{-1}\\
&\iff gHg^{-1}=H
\end{align*}
\end{proof}

\begin{definition}[Quotient group]
Let $G$ be a group, $H\triangleleft G$. Then the \vocab{quotient group} of $G$ by $H$ is
\[G/H\coloneqq\{aH\mid a\in G\}.\]
\end{definition}

\begin{proposition}
$G/H$ is a group under the following operation. Let $aH,bH\in G/H$. Then the product of $aH$ and $bH$ is $(aH)(bH)$.
\[(aH)(bH)=a(Hb)H=a(bH)H=abH\]
\end{proposition}

\begin{proof}
Check group axioms.
\begin{enumerate}[label=(\roman*)]
\item For $a,b,c\in G$,
\begin{align*}
(aH)(bHcH)&=(aH)(bcH)\\
&=a(bc)H\\
&=(ab)cH\\
&=(aHbH)cH
\end{align*}
so the operation is associative.
\item The identity of $G/H$ is the coset $eH$.
\item For $aH\in G/H$, the inverse of $aH$ is $a^{-1}H$ as is immediate from the definition of the product.
\end{enumerate}
\end{proof}

\begin{lemma}
Let $G$ be a finite group, $H\triangleleft G$. Then
\[|G/H|=|G:H|=\frac{|G|}{|H|}.\]
\end{lemma}

\begin{definition}[Quotient map]
Let $H\triangleleft G$. The map $\pi:G\to G/H$ which sends $g\mapsto gH$ is called the \vocab{quotient map}.
\end{definition}
\pagebreak

\section{Homomorphisms and Isomorphisms}
In this section, we make precise the notion of when two groups ``look the same''; that is, they have the same group-theoretic structure. This is the notion of an \emph{isomorpism} between two groups.

\subsection{Definitions and Examples}
\begin{definition}[Homomorphism]
Let $(G,\ast)$ and $(H,\diamond)$ be groups. A map $\phi:G\to H$ is called a \vocab{homomorphism}\index{homomorphism} if, for all $x,y\in G$,
\[\phi(x\ast y)=\phi(x)\diamond\phi(y).\]
\end{definition}

When the group operations for $G$ and $H$ are not explicitly written, we have
\[\phi(xy)=\phi(x)\phi(y).\]

\begin{definition}[Isomorphism]
$\phi:G\to H$ is called an \vocab{isomorphism}\index{isomorphism} if
\begin{enumerate}[label=(\roman*)]
\item $\phi$ is a homomorphism;
\item $\phi$ is a bijection.
\end{enumerate}
Then $G$ and $H$ are said to be \vocab{isomorphic}, denoted by $G\cong H$.
\end{definition}

Intuitively, $G$ and $H$ are the same group except that the elements and the operations may be written differently in $G$ and $H$.

We also have the following terminology: An \emph{automorphism} of a group $G$ is an isomorphism from $G$ to $G$. The automorphisms of $G$ form a group $\Aut(G)$ under composition. An endomorphism of $G$ is a homomorphism from $G$ to $G$. (Rarely used) A \emph{monomorphism} is an injective homomorphism and an \emph{epimorphism} is a surjective homomorphism.

\begin{example}
For any group $G$, $G\cong G$ as the identity map provides an isomorphism from $G$ to itself. (Exercise: prove that the identity map is the \emph{only} isomorphism from $G$ to itself.)

$\ZZ\cong10\ZZ$ as the map $\phi:\ZZ\to10\ZZ$ by $x\mapsto 10x$ is a homomorphism and a bijection.
\end{example}

\begin{example}
$(\RR,+)\cong(\RR^+,\times)$.

\begin{proof}
The exponential map $\exp:\RR\to\RR^+$ defined by $\exp(x)=e^x$ is an isomorphism from $(\RR,+)$ to $(\RR^+,\times)$.
\begin{enumerate}[label=(\roman*)]
\item $\exp$ is a bijection since it has an inverse function (namely $\ln$).
\item $\exp$ preserves the group operations since $e^{x+y}=e^xe^y$.
\end{enumerate}

We see that both the elements and the operations are different yet the two groups are isomorphic, that is, as groups they have identical structures.
\end{proof}
\end{example}

\begin{proposition}
Let $\phi:G\to H$ be a homomorphism. Let $g\in G$, $n\in\ZZ$. Then
\begin{enumerate}[label=(\roman*)]
\item $\phi(e_G)=e_H$;
\item $\phi(g^{-1})=\brac{\phi(g)}^{-1}$;
\item $\phi(g^n)=\brac{\phi(g)}^n$.
\end{enumerate}
\end{proposition}

\begin{proof} \
\begin{enumerate}[label=(\roman*)]
\item $\phi(e_G)=\phi(e_G e_G)=\phi(e_G)\phi(e_G)$, then apply $\phi(e_G)^{-1}$ to both sides to get $\phi(e_G)=e_H$.

\item $\phi(g)\phi(g^{-1})=\phi(gg^{-1})=\phi(e_G)=e_H$.

\item Note more generally that we can show $\phi(g^n)=(\phi(g))^n$ for $n>0$ by induction. For $n=-k<0$ we have
\[\phi(g^n)=\phi((g^{-1})^k)=(\phi(g^{-1}))^k=(\phi(g)^{-1})^k=\phi(g)^n.\]
\end{enumerate}
\end{proof}

\begin{theorem}
Any two cyclic groups of the same order are isomorphic.
\end{theorem}

\begin{proof}
Suppose $\langle x\rangle$ and $\langle y\rangle$ are both cyclic groups of order $n$. We first prove the case where $n<\infty$. We claim that the map $\phi:\langle x\rangle\to\langle y\rangle$ which sends $x^k\mapsto y^k$ is an isomorphism.
\begin{lemma*}
Let $G$ be a group, $g\in G$, let $m,n\in\ZZ$. Denote $d=\gcd(m,n)$. If $g^n=e$ and $g^m=e$, then $g^d=e$.
\end{lemma*}
\begin{proof}
By Bezout's lemma, since $d=\gcd(m,n)$, then there exists $q,r\in\ZZ$ such that $qm+rn=d$. Thus
\[g^d=g^{qm+rn}=\brac{g^m}^q\brac{g^n}^r=e.\]
\end{proof}
We first show that $\phi$ is well-defined; that is, $x^r=x^s\implies \phi(x^r)=\phi(x^s)$. Note that $x^{r-s}=e$, so by the above lemma, $n\mid r-s$. Write $r=tn+s$ for some $t\in\ZZ$, so
\[\phi(x^r)=\phi(x^{tn+s})=y^{tn+s}=(y^n)^ty^s=y^s=\phi(x^s).\]

We then show that $\phi$ is a homomorphism:
\[\phi(x^ax^b)=\phi(x^{a+b})=y^{a+b}=y^ay^b=\phi(x^a)\phi(x^b).\]

Finally we show that $\phi$ is bijective. Since the element $y^k$ of $\langle y\rangle$ is in the image of $x^k$ under $\phi$, $\phi$ is surjective. Since both groups have the same finite order, any surjection from one to the other is a bijection. Therefore $\phi$ is an isomorphism.

We now prove the case where $n=\infty$. If $\langle x\rangle$ is an infinite cyclic group, let $\phi:\ZZ\to\langle x\rangle$ be defined by $\phi(k)=x^k$. (This map is well-defined since there is no ambiguity in the representation of elements in the domain.)

Since $x^a\neq x^b$ for all distinct $a,b\in\ZZ$, $\phi$ is injective. By definition of a cyclic group, $\phi$ is surjective. As above, the laws of exponents ensure $\phi$ is a homomorphism. Hence $\phi$ is an isomorphism.
\end{proof}

\subsection{Kernel and Image}
\begin{definition}[Kernel and image]
Let $\phi:G\to H$ be a homomorphism. Then the \vocab{kernel}\index{kernel} of $\phi$ is
\[\ker\phi\coloneqq\{g\in G\mid \phi(g)=e_H\}\subset G.\]
The \vocab{image}\index{image} of $G$ under $\phi$ is
\[\im\phi\coloneqq\phi(G)=\{\phi(g)\mid g\in G\}\subset H.\]
\end{definition}

\begin{remark}
$\im\phi$ is the usual set theoretic image of $\phi$.
\end{remark}

\begin{proposition}
Let $\phi:G\to H$ be a homomorphism. Then
\begin{enumerate}[label=(\roman*)]
\item $\ker\phi\triangleleft G$;
\item $\im\phi\le H$.
\end{enumerate}
\end{proposition}

\begin{proof} \
\begin{enumerate}[label=(\roman*)]
\item Apply the subgroup criterion. Since $e_G\in\ker\phi$, $\ker\phi\neq\emptyset$. Let $x,y\in\ker\phi$; that is, $\phi(x)=\phi(y)=e_H$. Then
\[\phi(xy^{-1})=\phi(x)\phi(y)^{-1}=e_H\]
so $xy^{-1}\in\ker\phi$. By the subgroup criterion, $\ker\phi\le G$.



\item Since $\phi(e_G)=e_H$, $e_H\in\im\phi$ so $\im\phi\neq\emptyset$. Let $x,y\in\im\phi$. Then there exists $a,b\in G$ such that $\phi(a)=x$, $\phi(b)=y$. Then
\[xy^{-1}=\phi(a)\phi(b)^{-1}=\phi(ab^{-1})\]
so $xy^{-1}\in\im\phi$. By the subgroup criterion, $\im\phi\le G$.
\end{enumerate}
\end{proof}

\begin{proposition}
Let $\phi:G\to H$ be a homomorphism. Then $\phi$ is injective if and only if $\ker\phi=\{e_G\}$.
\end{proposition}

\begin{proof} \

\fbox{$\implies$} Suppose $\phi$ is injective. Since $\phi(e_G)=e_H$, $e_G\in\ker\phi$ so $\{e_G\}\subset\ker\phi$. 

Conversely, let $g\in\ker\phi$, so $\phi(g)=e_H$. Then $\phi(g)=e_H=\phi(e_H)$, so by injectivity, $g=e_G$ and thus $g=e_G$. Hence $\ker\phi\subset\{e_G\}$, so $\ker\phi=\{e_G\}$.

\fbox{$\impliedby$} Suppose $\ker\phi=\{e_G\}$. Suppose $\phi(a)=\phi(b)$, then
\begin{align*}
\phi(a)&=\phi(b)\\
\phi(a)\phi(b)^{-1}&=\phi(b)\phi(b)^{-1}\\
\phi(a)\phi(b)^{-1}&=e_H\\
\phi(ab^{-1})&=e_H
\end{align*}
Hence $ab^{-1}\in\ker\phi=\{e_G\}$, so $ab^{-1}=e_G$ and thus $a=b$. Therefore $\phi$ is injective.
\end{proof}

\subsection{Isomorphism Theorems}
\begin{theorem}[First isomorphism theorem]
Let $\phi:G\to H$ be a homomorphism. Then
\[G/\ker\phi\cong\im\phi(G).\]
\end{theorem}

\begin{theorem}[Second isomorphism theorem]
Let $A,B\le G$, $A\le N_G(B)$. Then
\begin{enumerate}[label=(\roman*)]
\item $AB\le G$;
\item $B\triangleleft AB$;
\item $A\cap B\triangleleft A$;
\item $AB/B\cong A/A\cap B$.
\end{enumerate}
\end{theorem}

\begin{theorem}[Third isomorphism theorem]
Let $H,K\triangleleft G$, $H\le K$. Then $K/H\triangleleft G/H$, and
\[(G/H)/(K/H)\cong G/K.\]
If we denote the quotient by $H$ with a bar, this can be written
\[\overline{G}/\overline{K}\cong G/K.\]
\end{theorem}

\begin{theorem}[Fourth isomorphism theorem]

\end{theorem}

\begin{theorem}[Cayley's theorem]

\end{theorem}
\pagebreak

\section{Group Actions}
We move now, from thinking of groups in their own right, to thinking of how groups can move sets around---for example, how $S_n$ permutes $\{1,2,\dots,n\}$ and matrix groups move vectors.

\begin{definition}[Group action]
A \vocab{group action} of a group $G$ on a set $A$ is a map from $G\times A\to A$ (written as $g\cdot a$, for all $g\in G$, $a\in A$) satisfying the following properties:
\begin{enumerate}[label=(\roman*)]
\item $g_1\cdot(g_2\cdot a)=(g_1g_2)\cdot a$, for all $g_1,g_2\in G$, $a\in A$;
\item $e_G\cdot a=a$ for all $a\in A$.
\end{enumerate}
We say that $G$ is a group acting on a set $A$.
\end{definition}

Intuitively, a group action of $G$ on a set $A$ means that every element $g$ in $G$ acts as a permutation on $A$ in a manner consistent with the group operations in $G$. There is also a notion of left \emph{action} and \emph{right action}.

For the following defintions, let $G$ be a group, and $A\subset G$ be non-empty.

\begin{definition}[Centraliser]
The \vocab{centraliser} of $A$ in $G$ is defined by
\[C_G(A)\coloneqq\{g\in G\mid\forall a\in A,gag^{-1}=a\}.\]
\end{definition}

Since $gag^{-1}=a$ if and only if $ga=ag$, $C_G(A)$ is the set of elements of $G$ which commute with every element of $A$.

We check that $C_G(A)\le G$:
\begin{enumerate}[label=(\roman*)]
\item $e\in C_G(A)$, so $C_G(A)\neq\emptyset$.
\item Let $x,y\in C_G(A)$; that is, for all $a\in A$, $xax^{-1}=a$ and $yay^{-1}=a$. Then
\begin{align*}
(xy)a(xy)^{-1}&=(xy)a(y^{-1}x^{-1})\\
&=x(yay^{-1})x^{-1}\\
&=xax^{-1}=a
\end{align*}
so $xy\in C_G(A)$. Hence $C_G(A)$ is closed under products.
\item Let $x\in C_G(A)$; that is, for all $a\in A$, $xax^{-1}=a$. Applying $x^{-1}$ to both sides gives $ax^{-1}=x^{-1}a$. Applying $x$ to both sides gives $a=x^{-1}ax$, so $x^{-1}\in C_G(A)$. Hence $C_G(A)$ is closed under taking inverses.
\end{enumerate}

\begin{notation}
In the special case when $A=\{a\}$ we simply write $C_G(a)$ instead of $C_G(\{a\})$. In this case $a^n\in C_G(a)$ for all $n\in\ZZ$.
\end{notation}

\begin{definition}[Centre]
The \vocab{centre} of $G$ is the set of elements which commute with all the elements of $G$:
\[Z(G)\coloneqq\{g\in G\mid\forall x\in G,gx=xg\}.\]
\end{definition}

Note that $Z(G)=C_G(G)$, so the argument above proves $Z(G)\le G$ as a special case.

\begin{definition}[Normaliser]
Define $gAg^{-1}=\{gag^{-1}\mid a\in A\}$. The \vocab{normaliser} of $A$ in $G$ is
\[N_G(A)\coloneqq\{g\in G\mid gAg^{-1}=A\}.\]
\end{definition}

Notice that if $g\in C_G(A)$, then $gag^{-1}=a\in A$ for all $a\in A$ so $C_G(A)\le N_G(A)$. The proof that $N_G(A)\le G$ is similar to the one that $C_G(A)\le G$.

\begin{definition}[Stabiliser]
If $G$ is a group acting on a set $S$, $s\in S$, then the \vocab{stabiliser} of $s$ in $G$ is
\[G_s\coloneqq\{g\in G\mid g\cdot s=s\}.\]
\end{definition}

\begin{notation}
Denote the set of all fixed points to be $S^G=\{s\in S\mid\forall g\in G, gs=g\}$.
\end{notation}

We check that $G_s\le G$:
\begin{enumerate}[label=(\roman*)]
\item By definition of group action, $e_G\cdot a=a$, so $e_G\in G_s$.
\item Let $x,y\in G_s$, then
\begin{align*}
(xy)\cdot s&=x\cdot(y\cdot s)\\
&=x\cdot s=s
\end{align*}
so $xy\in G_s$. Hence $G_s$ is closed under products.
\item Let $x\in G_s$; that is, $x\cdot s=s$. Then
\begin{align*}
x^{-1}\cdot s&=x^{-1}\cdot(x\cdot s)\\
&=(x^{-1}x)\cdot s\\
&=e\cdot s=s
\end{align*}
so $x^{-1}\in G_s$. Hence $G_s$ is closed under taking inverses.
\end{enumerate}

\begin{definition}
The \emph{kernel} of the action of $G$ on $S$ is
\[\{g\in G\mid\forall s\in S, g\cdot s=s\}.\]
\end{definition}

\begin{definition}[Orbit]
Let $G$ be a group that acts on a set $S$. Define the \vocab{orbit} of a group element $s\in S$ as
\[G(s)\coloneqq\{g\cdot s\in S\mid g\in G\}.\]
\end{definition}

\subsection{Conjugation}

\subsection{Sylow's Theorem}
\begin{definition}[Sylow $p$-subgroup]
Let $G$ be a group, and let $p$ be a prime.
\begin{enumerate}[label=(\roman*)]
\item A group of order $p^\alpha$ ($\alpha\ge1$) is called a \emph{$p$-group}. Subgroups of $G$ which are $p$-groups are called \emph{$p$-subgroups}.
\item If $|G|=p^\alpha m$ ($p\nmid m$), then a subgroup of order $p^\alpha$ is called a \vocab{Sylow $p$-subgroup} of $G$.
\end{enumerate}
\end{definition}

\begin{notation}
The set of Sylow $p$-subgroups of $G$ is denoted by $Syl_p(G)$, and the number of Sylow $p$-subgroups of $G$ is denoted by $n_p(G)$ (or just $n_p$ when $G$ is clear from the context).
\end{notation}

\begin{theorem}[Sylow's theorem]
Let $|G|=p^\alpha m$, where $p$ is a prime and $p\nmid m$.
\begin{enumerate}[label=(\roman*)]
\item Sylow $p$-subgroups of $G$ exist, i.e. $Syl_p(G)\neq\emptyset$.
\item If $P$ is a Sylow $p$-subgroup of $G$, and $Q$ is any $p$-subgroup of $G$, then there exists $g\in G$ such that $Q\le gPg^{-1}$, i.e. $Q$ is contained in some conjugate of $P$. In particular, any two Sylow $p$-subgroups of $G$ are conjugate in $G$.
\item $n_p\equiv1\pmod p$. Furthermore, $n_p$ is the index in $G$ of the normaliser $N_G(P)$ for any Sylow $p$-subgroup $P$, hence $n_p\mid m$.
\end{enumerate}
\end{theorem}
\pagebreak

\section{Group Product, Finite Abelian Groups}
\begin{definition}[Direct product]
The \vocab{direct product} $G_1\times\cdots\times G_n$ of the groups $(G_1,\ast_1),\dots,(G_n,\ast_n)$ is the Cartesian product
\[G_1\times\cdots\times G_n\coloneqq\{(g_1,\dots,g_n)\mid g_i\in G_i\}\]
with operation defined componentwise:
\[(g_1,\dots,g_n)\ast(h_1,\dots,h_n)=(g_1\ast_1 h_1,\dots,g_n\ast_n h_n).\]
\end{definition}

\begin{proposition}
If $G_1,\dots,G_n$ are groups, then
\[|G_1\times\cdots\times G_n|=|G_1|\:|G_2|\cdots|G_n|.\]
\end{proposition}

\begin{proof}
Let $G=G_1\times\cdots\times G_n$. The proof that the group axioms hold for $G$ is straightforward since each axiom is a consequence of the fact that the same axiom holds for each $G_i$, and the operation on $G$ defined componentwise.

The number of $n$-tuples in $G$ follows from simple combinatorics.
\end{proof}


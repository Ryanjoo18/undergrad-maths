\chapter{Fields}\label{chap:fields}
\section{Finite Extensions and Algebraic Extensions}
One of the first invariants associated with any field $F$ is its characteristic:

\begin{definition}[Characteristic]
The \vocab{characteristic} of a field $F$, denoted $\ch(F)$, is the smallest positive integer $p$ such that $p\cdot 1_F=0$ if such a $p$ exists and is defined to be $0$ otherwise.
\end{definition}

\begin{proposition}
$\ch(F)$ is either $0$ or a prime $p$. If $\ch(F)=p$ then for any $\alpha\in F$,
\[p\cdot\alpha=\underbrace{\alpha+\cdots+\alpha}_\text{$p$ times}=0.\]
\end{proposition}

\begin{definition}[Prime subfield]
The \vocab{prime subfield} of a field $F$ is the subfield of $F$ generated by the multiplicative identity $1_F$ of $F$. It is (isomorphic to) either $\QQ$ (if $\ch(F)=0$) or $\FF_p$ (if $\ch(F)=p$).
\end{definition}

\begin{notation}
We shall usually denote the identity $1_F$ of a field $F$ simply by $1$.
\end{notation}

\begin{definition}[Extension field]
If $K$ is a field containing the subfield $F$, then $K$ is said to be an \vocab{extension field} (or simply an \emph{extension}) of $F$, denoted by $K/F$.
\end{definition}

In particular, every field $F$ is an extension of its prime subfield. The field $F$ is sometimes called the \vocab{base field} of the extension.

\section{Splitting Fields}
\section{Separable Extensions}

\chapter{Galois Theory}
\section{Basic Definitions}
Let $K$ be a field.
\begin{definition}
An isomorphism $\sigma:K\to K$ is called an \vocab{automorphism} of $K$. The collection of automorphisms of $K$ is denoted $\Aut(K)$. If $\alpha\in K$ we write $\sigma\alpha$ for $\sigma(\alpha)$.

An automorphism $\sigma\in\Aut(K)$ is said to \vocab{fix} $\alpha\in K$ if $\sigma\alpha=\alpha$. If $F\subset K$ then an automorphism $\sigma$ is said to fix $F$ if it fixes all the elements of $F$, i.e. $\sigma a=a$ for all $a\in F$.
\end{definition}
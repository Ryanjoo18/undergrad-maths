\chapter{Measures}\label{chap:measures}
\section{Introduction}
One of the most venerable problems in geometry is to determine the area of volume of a region in the plane or in 3-space.

Ideally, for $n\in\NN$ we would like to have a function $\mu$ that assigns to each $E\subset\RR^n$ a number $\mu(E)\in[0,\infty]$, the $n$-dimensional measure of $E$.
We desire such a function $\mu$ to possess the following properties:
\begin{enumerate}
\item If $E_1,E_2,\dots$ is a finite or infinite sequence of disjoint sets, then
\[\mu(E_1\cup E_2\cup\cdots)=\mu(E_1)+\mu(E_2)+\cdots.\]
\item If $E$ is congruent to $F$ (that is, if $E$ can be transformed into $F$ by translations, rotations, and reflections), then $\mu(E)=\mu(F)$.
\item $\mu(Q)=1$, where $Q$ is the unit cube
\[Q=\{x\in\RR^n\mid 0\le x_i<1,\:i=1,\dots,n\}.\]
\end{enumerate}

Unfortunately, these conditions are mutually inconsistent.

\begin{proposition}
There does not exist such a volume measure defined on $\mathcal{P}(\RR^n)$.
\end{proposition}

\begin{proof}
It suffices to prove the case when $n=1$. Define an equivalence relation $\sim$ on $\RR$ by
\[x\sim y\iff x-y\in\QQ.\]
We check that this is an equivalence relation:
\begin{enumerate}[label=(\roman*)]
\item $x-x=0$ implies $x\sim x$.
\item $x-y\in\QQ$ implies $y-x\in\QQ$.
\item $x-y,y-z\in\QQ$ implies $x-z=(x-y)+(y-z)\in\QQ$.
\end{enumerate}

Let $N\subset[0,1)$ be a subset containing exactly one representative from each equivalence class of $\sim$. (To find such an $N$, one must invoke the axiom of choice.)

Next, let $R=\QQ\cap[0,1)$, and for each $r\in R$ let
\[N_r=\{x+r\mid x\in N\cap[0,1-r]\}\cup\{x+r-1\mid x\in N\cap[1-r,1]\}.\]
That is, to obtain $N_r$, shift $N$ to the right by $r$ units and then shift the part that sticks out beyond $[0,1)$ one unit to the left.
Then $N_r\subset[0,1)$, and every $x\in[0,1)$ belongs to precisely one $N_r$.

Indeed, if $y$ is the element of $N$ that belongs to the equivalence class of $x$, then $x\in N_r$, where $r=x-y$ if $x\ge y$ or $r=x+y-1$ if $x<y$; on the other hand, if $x\in N_r\cap N_s$, then $x-r$ (or $x-r+1$) and $x-s$ (or $x-s+1$) would be distinct elements of $N$ belonging to the same equivalence class, which is impossible.

Suppose now that $\mu\colon\mathcal{P}(\RR)\to[0,\infty]$ satisfies (1), (2) and (3). By (1) and (2),
\[\mu(N)=\mu(N\cap[0,1-r))+\mu(N\cap[1-r,1])=\mu(N_r)\]
for any $r\in R$. Also, since $R$ is countable and $[0,1)$ is the disjoint union of the $N_r$'s,
\[\mu([0,1])=\sum_{r\in R}\mu(N_r)\]
by (1) again. But $\mu([0,1])=1$ by (3), and since $\mu(N_r)=\mu(N)$, the sum on the right is either $0$ (if $\mu(N)=0$) or $\infty$ (if $\mu(N)>0$). Hence no such $\mu$ can exist.
\end{proof}

\begin{theorem}[Banach--Tarski paradox]
Let $U$ and $V$ be arbitrary bounded open sets in $\RR^n$, $n\ge 3$. There exist $k\in\NN$ and subsets $E_1,\dots,E_k,F_1,\dots,F_k$ of $\RR^n$ such that
\begin{enumerate}
\item the $E_j$'s are disjoint and their union is $U$;
\item the $F_j$'s are disjoint and their union is $V$;
\item $E_j$ is congruent to $F_j$ for $j=1,\dots,k$.
\end{enumerate}
\end{theorem}
\pagebreak

\section{$\sigma$-algebras}
\begin{definition}[$\sigma$-algebra]\label{defn:sigma-algebra}
Let $X$ be a non-empty set. We say a non-empty collection of subsets $\mathcal{A}\subset\mathcal{P}(X)$ is an \vocab{algebra} on $X$ if
\begin{enumerate}[label=(\roman*)]
\item if $E_1,\dots,E_n\in\mathcal{A}$, then $\bigcup_{i=1}^{n}E_i\in\mathcal{A}$;\hfill(closed under finite unions)
\item if $E\in\mathcal{A}$, then $E^c\in\mathcal{A}$.\hfill(closed under complements)
\end{enumerate}
A \vocab{$\sigma$-algebra} is an algebra that is closed under \emph{countable unions}.

If $\mathcal{A}$ is a $\sigma$-algebra on $X$, then we say $(X,\mathcal{A})$ is a \vocab{measurable space}, and the members of $\mathcal{A}$ are called the \emph{measurable sets} in $X$.
\end{definition}

\begin{notation}
If the $\sigma$-algebra $\mathcal{A}$ is clear, we simply omit it and denote a measurable space as $X$.
\end{notation}

Suppose $\mathcal{A}$ is an algebra of subsets of $X$. Then we immediately deduce the following properties.
\begin{itemize}
\item $X=E\cup E^c\in\mathcal{A}$ and $\emptyset=X^c\in\mathcal{A}$.
\item If $E_1,\dots,E_n\in\mathcal{A}$, then by de Morgan's laws,
\[\bigcap_{i=1}^{n}E_i=\brac{\bigcup_{i=1}^{n}{E_i}^c}^c\in\mathcal{A}\]
so $\mathcal{A}$ is closed under finite intersections.
If $\mathcal{A}$ is a $\sigma$-algebra, then $\mathcal{A}$ is closed under countable intersections.
\item If $A,B\in\mathcal{A}$, since $A\setminus B=B^c\cap A$, then $A\setminus B\in\mathcal{A}$.
\end{itemize}

\begin{example} \
\begin{itemize}
\item If $X$ is any set, $\mathcal{P}(X)$ is a $\sigma$-algebra.
\item If $X$ is any set, $\{\emptyset,X\}$ is a $\sigma$-algebra.
\item If $X$ is any set, and $E\subset X$, $\{\emptyset,E,E^c,X\}$ is a $\sigma$-algebra.
\item If $X$ is uncountable, then
\[\mathcal{A}=\{E\subset X\mid\text{$E$ or $E^c$ is countable}\}\]
is a $\sigma$-algebra, called the \emph{$\sigma$-algebra of countable or co-countable sets}.
\end{itemize}
\end{example}

\begin{definition}
Given $\mathcal{E}\subset\mathcal{P}(X)$, the $\sigma$-algebra \emph{generated} by $\mathcal{E}$, denoted by $\mathcal{M}(\mathcal{E})$, is the smallest $\sigma$-algebra containing $\mathcal{E}$.
\end{definition}

That is, $\mathcal{M}(\mathcal{E})$ is the intersection of all $\sigma$-algebras containing $\mathcal{E}$:
\[\mathcal{M}(\mathcal{E})=\bigcap_{\substack{\mathcal{A}\supset\mathcal{E}\\ \text{$\mathcal{A}$ is $\sigma$-algebra}}}\mathcal{A}.\]

$\mathcal{M}(\mathcal{E})$ exists since $\mathcal{P}(X)$ is a $\sigma$-algebra (so the intersection is non-empty), and the intersection of any family of $\sigma$-algebras on $X$ is itself a $\sigma$-algebra.

The following observation is often useful:

\begin{lemma}
If $\mathcal{E}\subset\mathcal{M}(\mathcal{F})$, then $\mathcal{M}(\mathcal{E})\subset\mathcal{M}(\mathcal{F})$.
\end{lemma}

\begin{proof}
$\mathcal{M}(\mathcal{F})$ is a $\sigma$-algebra containing $\mathcal{E}$; it therefore contains $\mathcal{M}(\mathcal{E})$.
\end{proof}

We have come to an important example of a $\sigma$-algebra.

\begin{definition}[Borel $\sigma$-algebra]
If $X$ is any metric space (or topological space), the \vocab{Borel $\sigma$-algebra} on $X$, denoted by $\mathcal{B}(X)$, is the $\sigma$-algebra generated by the family of open sets in $X$.

Its members are called \vocab{Borel sets}.
\end{definition}

$B_X$ thus includes open sets, closed sets, countable intersections of open sets, countable unions of closed sets, and so forth.

We introduce some standard terminology for the levels in this hierarchy.
\begin{itemize}
\item A countable intersection of open sets is called a $G_\delta$ set.
\item A countable union of closed sets is called a $F_\sigma$ set.
\item A countable union of $G_\delta$ sets is called a $G_{\delta\sigma}$ set; a countable intersection of $F_\sigma$ sets is called a $F_{\sigma\delta}$ set; and so forth.
\end{itemize}
($\delta$ and $\sigma$ stand for the German ``Durchschnitt'' and ``Summe'', that is, intersection and union.)

The Borel $\sigma$-algebra on $\RR$ will play a fundamental role in what follows. It can be generated in a number of different ways:

\begin{proposition}
$\mathcal{B}(\RR)$ is generated by each of the following:
\begin{enumerate}[label=(\roman*)]
\item the open intervals: $\mathcal{E}_1=\{(a,b)\mid a<b\}$;
\item the closed intervals: $\mathcal{E}_2=\{[a,b]\mid a<b\}$;
\item the half-open intervals: $\mathcal{E}_3=\{(a,b]\mid a<b\}$ or $\mathcal{E}_4=\{[a,b)\mid a<b\}$;
\item the open rays: $\mathcal{E}_5=\{(a,\infty)\mid a\in\RR\}$ or $\mathcal{E}_6=\{(-\infty,a)\mid a\in\RR\}$;
\item the closed rays: $\mathcal{E}_7=\{[a,\infty)\mid a\in\RR\}$ or $\mathcal{E}_8=\{(-\infty,a]\mid a\in\RR\}$.
\end{enumerate}
\end{proposition}

\begin{proof}

\end{proof}

Suppose $(X,\mathcal{M},\mu)$ is a measure space. We say $E\in\mathcal{M}$ is a \emph{null set} if $\mu(E)=0$. 
%By subadditivity, any countable union of null sets is a null set, a fact which we shall use frequently.

If $\mu(E)=0$ and $F\subset E$, then $\mu(F)=0$ by monotonicity provided that $F\in\mathcal{M}$, but in general it need not be true that $F\in\mathcal{M}$.
A measure whose domain includes all subsets of null sets is called \vocab{complete}.
The next result states that completeness can be achieved by enlarging the domain of $\mu$.

\begin{proposition}
Suppose $(X,\mathcal{M},\mu)$ is a measure space. Let $\mathcal{N}=\{N\in\mathcal{M}\mid\mu(N)=0\}$ and
\[\overline{\mathcal{M}}=\{E\cup F\mid E\in\mathcal{M},F\subset N\text{ for some }N\in\mathcal{N}\}.\]
Then $\overline{\mathcal{M}}$ is a $\sigma$-algebra, and there exist a unique extension $\overline{\mu}$ of $\mu$ to a complete measure on $\overline{\mathcal{M}}$.
\end{proposition}

$\overline{\mu}$ is called the \emph{completion} of $\mu$; $\overline{\mathcal{M}}$ is called the \emph{completion} of $\mathcal{M}$ with respect to $\mu$.

\begin{proof}
We check that $\overline{\mathcal{M}}$ is a $\sigma$-algebra.
\begin{enumerate}[label=(\roman*)]
\item Since $\mathcal{M}$ and $\mathcal{N}$ are closed under countable unions, so is $\overline{\mathcal{M}}$.
\item Let $E\cup F\in\overline{\mathcal{M}}$, where $E\in\mathcal{M}$ and $F\subset N\in\mathcal{N}$.
\end{enumerate}
\end{proof}
\pagebreak

\section{Measures}
\begin{definition}[Measure]
Let $X$ be a set equipped with a $\sigma$-algebra $\mathcal{M}$.
A \vocab{measure}\index{measure} on $\mathcal{M}$ is a function $\mu\colon\mathcal{M}\to[0,\infty]$ such that
\begin{enumerate}[label=(\roman*)]
\item $\mu(\emptyset)=0$;
\item if $E_1,E_2,\dots$ is a sequence of disjoint sets in $\mathcal{M}$, then $\mu\brac{\bigcup_{i=1}^{\infty}E_i}=\sum_{i=1}^{\infty}\mu(E_i)$.
\end{enumerate}
If $\mu$ is a measure on $(X,\mathcal{M})$, we call $(X,\mathcal{M},\mu)$ a \vocab{measure space}\index{measure space}.
\end{definition}

Property (ii) is called \emph{countable additivity}. It implies \emph{finite additivity}:
\begin{enumerate}
\item[(ii')] if $E_1,\dots,E_n$ are disjoint sets in $\mathcal{M}$, then $\mu(\bigcup_{i=1}^{n}E_i)=\sum_{i=1}^{n}\mu(E_i)$,
\end{enumerate}
because one can take $E_i=\emptyset$ for $i>n$. A function $\mu$ that satisfies (i) and (ii') but not necessarily (ii) is called a \emph{finitely additive measure}.

We introduce some standard terminology concerning the ``size'' of $\mu$.
\begin{itemize}
\item If $\mu(X)<\infty$ (which implies that $\mu(E)<\infty$ for all $E\in\mathcal{M}$), $\mu$ is called \emph{finite}.
\item If $X=\bigcup_{i=1}^{\infty}E_i$ where $E_i\in\mathcal{M}$ and $\mu(E_i)<\infty$ for all $i$, $\mu$ is called \emph{$\sigma$-finite}.
\item More generally, if $E=\bigcup_{i=1}^{\infty}E_i$ where $E_i\in\mathcal{M}$ and $\mu(E_i)<\infty$ for all $i$, the set $E$ is said to be \emph{$\sigma$-finite} for $\mu$.
\item If for each $E\in\mathcal{M}$ with $\mu(E)=\infty$ there exists $F\in\mathcal{M}$ with $F\subset E$ and $0<\mu(F)<\infty$, $\mu$ is called \emph{semifinite}.
\end{itemize}

\begin{example} \
\begin{itemize}
\item For any $E\subset X$, where $X$ is any set, define $\mu(E)=\infty$ if $E$ is an infinite set, and let $\mu(E)$ be the number of points in $E$ if $E$ is finite. This $\mu$ is called the \emph{counting measure} on $X$. 
\item Fix $x_0\in X$. For any $E\subset X$, let
\[\mu(E)=\begin{cases}
1&(x_0\in E)\\
0&(x_0\notin E)
\end{cases}\]
This $\mu$ is called the \emph{unit mass} concentrated at $x_0$.

\item A \emph{probability measure} on $\Omega$ is a measure $\PP$ such that $\PP(\Omega)=1$.

\item On $\mathcal{B}(\RR)$, define a measure $\mu\brac{(a,b)}=b-a$ for any $a,b\in\RR$, $a<b$. This is called the \emph{Lebesgue measure} on $\RR$.
\end{itemize}
\end{example}

\begin{lemma}[Basic properties of measures]
Suppose $(X,\mathcal{M},\mu)$ is a measure space.
\begin{enumerate}[label=(\roman*)]
\item If $E,F\in\mathcal{M}$ and $E\subset F$, then $\mu(E)\le\mu(F)$.\hfill(monotonicity)
\item If $E_1,E_2,\dots\in\mathcal{M}$, then $\mu\brac{\bigcup_{n=1}^{\infty}E_i}\le\sum_{n=1}^{\infty}\mu(E_n)$.\hfill(subadditivity)
\item If $E_1,E_2,\dots\in\mathcal{M}$ and $E_1\subset E_2\subset\cdots$, then\hfill(continuity from below)
\[\mu\brac{\bigcup_{n=1}^{\infty}E_n}=\lim_{n\to\infty}\mu(E_n).\]
\item If $E_1,E_2,\dots\in\mathcal{M}$, $E_1\supset E_2\supset\cdots$, and $\mu(E_1)<\infty$, then\hfill(continuity from above)
\[\mu\brac{\bigcap_{n=1}^{\infty}E_n}=\lim_{n\to\infty}\mu(E_n).\]
\end{enumerate}
\end{lemma}

\begin{proof} \
\begin{enumerate}[label=(\roman*)]
\item If $E\subset F$, note that $F=E\cup(F\setminus E)$. Then
\[\mu(F)=\mu(E)+\mu(F\setminus E)\ge\mu(E).\]

\item Let $F_1=E_1$ and $F_n=E_n\setminus\brac{\bigcup_{i=1}^{n-1}E_i}$ for $n>1$. Then the $F_n$'s are disjoint and $\bigcup_{i=1}^{n}F_i=\bigcup_{i=1}^{n}E_i$ for all $n$. Hence by (i),
\begin{align*}
\mu\brac{\bigcup_{n=1}^{\infty}E_n}
&=\mu\brac{\bigcup_{n=1}^{\infty}F_n}\\
&=\sum_{n=1}^{\infty}\mu(F_n)&&[\text{by countable additivity}]\\
&\le\sum_{n=1}^{\infty}\mu(E_n)&&[\text{by monotonicity, since $F_n\subset E_n$}]
\end{align*}

\item Suppose $E_1\subset E_2\subset\cdots$. Then we have
\begin{align*}
\mu\brac{\bigcup_{n=1}^{\infty}E_n}
&=\mu\brac{\bigcup_{n=1}^{\infty}E_n\setminus E_{n-1}}&&[\text{setting $E_0=\emptyset$}]\\
&=\sum_{n=1}^{\infty}\mu(E_n\setminus E_{n-1})&&[\text{by countable additivity}]\\
&=\lim_{n\to\infty}\sum_{i=1}^{n}\mu(E_i\setminus E_{i-1})\\
&=\lim_{n\to\infty}\mu(E_n)&&[\text{by finite additivity}]
\end{align*}

\item Let $F_n=E_1\setminus E_n$; then $F_1\subset F_2\subset\cdots$, $\mu(E_1)=\mu(F_n)+\mu(E_n)$, and $\bigcup_{n=1}^{\infty}F_n=E_1\setminus\brac{\bigcap_{n=1}^{\infty}E_n}$. By (iii),
\begin{align*}
\mu(E_1)&=\mu\brac{\bigcap_{n=1}^{\infty}E_n}+\lim_{n\to\infty}\mu(F_n)\\
&=\mu\brac{\bigcap_{n=1}^{\infty}E_n}+\lim_{n\to\infty}\brac{\mu(E_1)-\mu(E_n)}\\
&=\mu\brac{\bigcap_{n=1}^{\infty}E_n}+\mu(E_1)-\lim_{n\to\infty}\mu(E_n)
\end{align*}
Since $\mu(E_1)<\infty$, we may subtract it from both sides to yield the desired result.
\end{enumerate}
\end{proof}

\begin{lemma}[Inclusion--exclusion formula]
Suppose $(X,\mathcal{M},\mu)$ is a measure space. If $E,F\in\mathcal{M}$, then
\[\mu(E)+\mu(F)=\mu(E\cup F)+\mu(E\cap F).\]
\end{lemma}

\begin{proof}
We have
\begin{align*}
\mu(E)+\mu(F)
&=\mu\brac{(E\setminus F)\cup(E\cap F)}+\mu\brac{(F\setminus E)\cup(E\cap F)}\\
&=\mu(E\setminus F)+\mu(E\cap F)+\mu(F\setminus E)+\mu(E\cap F)\\
&=\mu(E\cup F)+\mu(E\cap F).
\end{align*}
\end{proof}
\pagebreak

\section{Outer Measures}
Recall the procedure used in calculus to define the area of a bounded region $E\subset\RR^2$: one draws a grid of rectangles in the plane and approximates the area of $E$ from above and below by the sum of areas of rectangles. 
The limits of these approximations as the grid is taken finer and finer give the ``inner area'' and ``outer area'' of $E$; if they are equal, their common value is the ``area'' of $E$.



\begin{definition}[Outer measure]
An \vocab{outer measure}\index{outer measure} on a non-empty set $X$ is a function $\mu^*\colon\mathcal{P}(X)\to[0,\infty]$ such that
\begin{enumerate}[label=(\roman*)]
\item $\mu^*(\emptyset)=0$;
\item if $A\subset B$, then $\mu^*(A)\le\mu^*(B)$;\hfill(monotonicity)
\item $\mu^*\brac{\bigcup_{i=1}^{\infty}A_i}\le\sum_{i=1}^{\infty}\mu^*(A_i)$.\hfill(subadditivity)
\end{enumerate}
\end{definition}

The most common way to obtain outer measures is to start with a family $\mathcal{E}$ of ``elementary sets'' on which a notion of measure is defined and then to approximate arbitrary sets ``from the outside'' by countable unions of members of $\mathcal{E}$.

\begin{proposition}
Let $\mathcal{E}\subset\mathcal{P}(X)$ and $\rho\colon\mathcal{E}\to[0,\infty]$ be such that $\emptyset\in\mathcal{E}$, $X\in\mathcal{E}$, and $\rho(\emptyset)=0$. For any $A\subset X$, define
\[\mu^*(A)=\inf\crbrac{\sum_{n=1}^{\infty}\mu(E_n)\:\bigg|\:E_n\in\mathcal{E},\:A\subset\bigcup_{n=1}^{\infty}E_n}.\]
Then $\mu^*$ is an outer measure.
\end{proposition}

\begin{theorem}[Carath\'{e}odory's theorem]

\end{theorem}

Our first applications of Carath\'{e}odory's theorem will be in the context of extending measures from algebras to $\sigma$-algebras.

\begin{definition}[Premeasure]
Suppose $\mathcal{A}\subset\mathcal{P}(X)$ is an algebra. We say $\mu_0\colon\mathcal{A}\to[0,\infty]$ is a \vocab{premeasure} if
\begin{enumerate}[label=(\roman*)]
\item $\mu_0(\emptyset)=0$;
\item 
\end{enumerate}
\end{definition}

\begin{proposition}
If $\mu_0$ is a premeasure on $\mathcal{A}$, and $\mu^*$ is defined by (1.12), then
\begin{enumerate}[label=(\roman*)]
\item $\mu^*|_{\mathcal{A}}=\mu_0$;
\item every set in $\mathcal{A}$ is $\mu^*$ measurable.
\end{enumerate}
\end{proposition}

\begin{theorem}

\end{theorem}
\pagebreak

\section{Borel Measures on the Real Line}
We are now in a position to construct a definitive theory for measuring subsets of $\RR$ based on the idea that the measure of an interval is its length. 
We begin with a more general construction that yields a large family of measures on $\RR$ whose domain is the Borel $\sigma$-algebra $\mathcal{B}(\RR)$; such measures are called \emph{Borel measures} on $\RR$.


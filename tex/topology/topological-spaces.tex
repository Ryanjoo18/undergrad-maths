\chapter{Topological Spaces and Continuous Functions}\label{chap:topological-spaces}
\section{Topological Spaces}
\begin{definition}[Topological space]
A \vocab{topology} on a set $X$ is a collection $\mathcal{T}$ of subsets of $X$ satisfying:
\begin{enumerate}[label=(\roman*)]
\item $X,\emptyset\in\mathcal{T}$;
\item if $U_i\in\mathcal{T}$ for all $i\in I$, then $\bigcup_{i\in I}U_i\in\mathcal{T}$;\hfill(closed under arbitrary unions)
\item if $U_1,\dots,U_n\in\mathcal{T}$, then $\bigcap_{i=1}^{n}U_i\in\mathcal{T}$.\hfill(closed under finite intersections)
\end{enumerate}
A set $X$ for which a topology $\mathcal{T}$ has been specified is called a \vocab{topological space}, denoted by $(X,\mathcal{T})$. $U\subset X$ is called an \emph{open set} of $X$ if $U\in\mathcal{T}$.
\end{definition}

\begin{notation}
When $\mathcal{T}$ is understood, we talk about the topological space $X$.
\end{notation}

\begin{example}
Let $X$ be any non-empty set.
\begin{itemize}
\item The \emph{discrete topology} on $X$ is the set of all subsets of $X$; that is, $\mathcal{T}=\mathcal{P}(X)$.
\item The \emph{indiscrete topology} (or \emph{trivial topology}) on $X$ is $\mathcal{T}=\{X,\emptyset\}$.
\item The \emph{co-finite topology} on $X$ consists of the empty set together with every subset $U$ of $X$ such that $X\setminus U$ is finite.
\item Let $\mathcal{T}_c$ be the collection of all subsets $U\subset X$ such that $U^c$ either is countable or is all of $X$. Then $\mathcal{T}_c$ is a topology on $X$.
\end{itemize}
\end{example}

\begin{definition}
Suppose $\mathcal{T}$ and $\mathcal{T}^\prime$ are two topologies on a given set $X$. We say that
\begin{enumerate}[label=(\roman*)]
\item $\mathcal{T}$ is \vocab{finer} than $\mathcal{T}^\prime$ if $\mathcal{T}\supset\mathcal{T}^\prime$;
\item $\mathcal{T}$ is \emph{coarser} than $\mathcal{T}^\prime$ if $\mathcal{T}\subset\mathcal{T}^\prime$;
\item $\mathcal{T}$ is \emph{comparable} with $\mathcal{T}^\prime$ if either $\mathcal{T}\supset\mathcal{T}^\prime$ or $\mathcal{T}\subset\mathcal{T}^\prime$.
\end{enumerate}
\end{definition}

\begin{remark}
The indiscrete topology is the coarsest topology possible, while the discrete topology is the finest topology possible.
\end{remark}

\section{Basis for a Topology}
In linear algebra, every vector space is generated by a basis. In topology, we have a similar notion, as it is usually hard to define a topology by specifying all the open sets.

\begin{definition}[Basis]\label{defn:topology-basis}
A \vocab{basis} for a topology on $X$ is a collection $\mathcal{B}$ of subsets of $X$ (called \emph{basis elements}) if
\begin{enumerate}[label=(\roman*)]
\item for all $x\in X$, there exists $B\in\mathcal{B}$ such that $x\in B$;
\item for all $B_1,B_2\in\mathcal{B}$ and $x\in B_1\cap B_2$, there exists $B_3\in\mathcal{B}$ such that $x\in B_3\subset B_1\cap B_2$.
\end{enumerate}
We define the \emph{topology $\mathcal{T}$ generated by basis $\mathcal{B}$} as
\begin{equation}
U\in\mathcal{T}\iff\forall x\in U,\:\exists B\in\mathcal{B},\:x\in B\subset U.
\end{equation}
\end{definition}

We now check that the collection $\mathcal{T}$ generated by the basis $\mathcal{B}$ is, in fact, a topology on $X$.
\begin{enumerate}[label=(\roman*)]
\item $\emptyset$ satisfies the defining condition of openness vacuously, so $\emptyset\in\mathcal{T}$. $X\in\mathcal{T}$ follows from (i) of \cref{defn:topology-basis}.
\item Consider a collection $\{U_i\mid i\in I\}$ of elements of $\mathcal{T}$. We want to show that $U=\bigcup_{i\in I}U_i\in\mathcal{T}$. 

Given $x\in U$, there exists $i\in I$ such that $x\in U_i$. Since $U_i\in\mathcal{T}$, there exists $B\in\mathcal{B}$ such that $x\in B\subset U_i$. Thus $x\in B\subset U$, so $U\in\mathcal{T}$.

\item Take two elements $U_1,U_2\in\mathcal{T}$, we want to show that $U_1\cap U_2\in\mathcal{T}$.

Given $x\in U_1\cap U_2$, choose $B_1\in\mathcal{B}$ such that $x\in B_1\subset U_1$; choose $B_2\in\mathcal{B}$ such that $x\in B_2\subset U_2$. Then $x\in B_1\cap B_2$.

Since $\mathcal{B}$ is a basis, by (ii) of \cref{defn:topology-basis}, there exists $B_3\in\mathcal{B}$ such that $x\in B_3\subset B_1\cap B_2$. Thus $U_1\cap U_2\in\mathcal{T}$.

Finally, we show by induction that any finite intersection $U_1\cap\cdots\cap U_n\in\mathcal{T}$. This is trivial for $n=1$; suppose it true for $n-1$ and prove it
for $n$. Now
\[(U_1\cap\cdots\cap U_n)=(U_1\cap\cdots\cap U_{n-1})\cap U_n.\]
By hypothesis, $U_1\cap\cdots\cap U_{n-1}\in\mathcal{T}$. Thus by the result just proved, the intersection of $U_1\cap\cdots\cap U_{n-1}$ and $U_n$ also belongs to $\mathcal{T}$.
\end{enumerate}

Another way of describing the topology generated by a basis is given in the following result:

\begin{lemma}\label{lemma:topology-generated-basis-unions}
Let $\mathcal{T}$ be the topology on $X$ generated by basis $\mathcal{B}$. Then $\mathcal{T}$ equals the collection of all unions of elements of $\mathcal{B}$.
\end{lemma}

\begin{proof}
Let $\mathcal{B}=\{B_i\mid i\in I\}$.
\begin{itemize}
\item If $B_i\in\mathcal{B}$, see that
\[\forall x\in B,\:x\in B\subset B\implies B\in\mathcal{T}.\]
Since $\mathcal{T}$ is a topology, the arbitrary unions of $B_i$'s must be in $\mathcal{T}$.
\item Conversely, given $U\in\mathcal{T}$, for each $x\in U$, there exists $B_x\in\mathcal{B}$ such that $x\in B_x\subset U$. Then $U=\bigcup_{x\in U}B_x$, so $U$ is a union of elements of $\mathcal{B}$.
\end{itemize}
\end{proof}

\begin{remark}
The above result states that every $U\in\mathcal{T}$ can be expressed as a union of basis elements.
\end{remark}

We have described in two different ways how to go from a basis to the topology it generates. Sometimes we need to go in the reverse direction, from a topology to a basis generating it. Here is one useful way of obtaining a basis for a given topology.

\begin{lemma}
Let $(X,\mathcal{T})$ be a topological space. Suppose that $\mathcal{C}$ is a collection of open sets of $X$, such that
\[\forall U\in\mathcal{T},\quad\forall x\in U,\quad\exists C\in\mathcal{C},\quad x\in C\subset U.\]
Then $\mathcal{C}$ is a basis for $\mathcal{T}$.
\end{lemma}

\begin{proof}
We first show that $\mathcal{C}$ is a basis.
\begin{enumerate}[label=(\roman*)]
\item For all $x\in X$, since $X\in\mathcal{T}$, by hypothesis, there exists $C\in\mathcal{C}$ such that $x\in C\subset X$.
\item Let $x\in C_1\cap C_2$, where $C_1,C_2\in\mathcal{C}\subset\mathcal{T}$. Thus $C_1,C_2\in\mathcal{T}$, so $C_1\cap C_2\in\mathcal{T}$. Hence by hypothesis, there exists $C_3\in\mathcal{C}$ such that $x\in C_3\subset C_1\cap C_2$.
\end{enumerate}

Let $\mathcal{T}^\prime$ be the topology generated by $\mathcal{C}$. We will show that $\mathcal{T}=\mathcal{T}^\prime$.
\begin{itemize}
\item Let $U\in\mathcal{T}$, $x\in U$. By hypothesis, there exists $C\in\mathcal{C}$ such that $x\in C\subset U$. By definition, $U\in\mathcal{T}^\prime$. Hence $\mathcal{T}\subset\mathcal{T}^\prime$.
\item Conversely, let $W\in\mathcal{T}^\prime$. By \cref{lemma:topology-generated-basis-unions}, $W$ is a union of elements of $\mathcal{C}$. Since each element of $\mathcal{C}$ is an element of $\mathcal{T}$ (and thus open), and a union of open sets is open, so $W\in\mathcal{T}$. Hence $\mathcal{T}^\prime\subset\mathcal{T}$.
\end{itemize}
\end{proof}

When topologies are given by bases, the next result is a criterion to determine whether one topology is finer than another.

\begin{lemma}
Let $\mathcal{B}$ and $\mathcal{B}^\prime$ be bases for the topologies $\mathcal{T}$ and $\mathcal{T}^\prime$ respectively on $X$. Then the following are equivalent:
\begin{enumerate}[label=(\roman*)]
\item $\mathcal{T}^\prime$ is finer than $\mathcal{T}$.
\item For all $x\in X$, and for all $B\in\mathcal{B}$ such that $x\in B$, there exists $B^\prime\in\mathcal{B}^\prime$ such that $x\in B^\prime\subset B$.
\end{enumerate}
\end{lemma}

\begin{proof} \

\fbox{(ii)$\implies$(i)} Let $U\in\mathcal{T}$. To show that $\mathcal{T}\subset\mathcal{T}^\prime$, we want to show that $U\in\mathcal{T}^\prime$.

Let $x\in U$. Since $\mathcal{B}$ generates $\mathcal{T}$, there exists $B\in\mathcal{B}$ such that $x\in B\subset U$. By (ii), there exists $B^\prime\in\mathcal{B}^\prime$ such that $x\in B^\prime\subset B$. Then $x\in B^\prime\subset U$, so $U\in\mathcal{T}^\prime$, by definition.

\fbox{(i)$\implies$(ii)} We are given $x\in X$ and $B\in\mathcal{B}$, with $x\in B$.

Now $B\in\mathcal{T}$ by definition, and $\mathcal{T}\subset\mathcal{T}^\prime$ by (i); therefore, $B\in\mathcal{T}^\prime$. Since $\mathcal{T}^\prime$ is generated by $\mathcal{B}^\prime$, there exists $B^\prime\in\mathcal{B}^\prime$ such that $x\in B^\prime\subset B$.
\end{proof}

We now define three topologies on the real line $\RR$.

\begin{definition} \
\begin{enumerate}[label=(\roman*)]
\item Let $\mathcal{B}$ be the collection of all open intervals in $\RR$. The topology generated by $\mathcal{B}$ is called the \vocab{standard topology} on $\RR$.

Whenever we consider $\RR$, we shall suppose it is given this topology unless stated otherwise. 

\item Let $\mathcal{B}^\prime$ be the collection of all half-open intervals of the form $[a,b)$. The topology generated by $\mathcal{B}^\prime$ is called the \vocab{lower limit topology} on $\RR$. 

When $\RR$ is given the lower limit topology, we denote it by $\RR_\ell$.

\item Let $K=\{\frac{1}{n}\mid n\in\ZZ^+\}$, and let $\mathcal{B}^{\prime\prime}$ be the collection of all open intervals $(a,b)$, along with all sets of the form $K\setminus(a,b)$. The topology generated by $B^{\prime\prime}$ is called the \vocab{$K$-topology} on $\RR$.

When $\RR$ is given this topology, we denote it by $\RR_K$.
\end{enumerate}
\end{definition}

It is easy to see that all three of these collections are bases; in each case, the intersection of two basis elements is either another basis element or is empty. The relation between these topologies is the following:

\begin{lemma}
The topologies of $\RR_\ell$ and $\RR_K$ are strictly finer than the standard topology on $\RR$, but are not comparable with one another.
\end{lemma}

\begin{definition}[Subbasis]
A \vocab{subbasis} $\mathcal{S}$ for a topology on $X$ is a collection of subsets of $X$ whose union equals $X$.

The \emph{topology $\mathcal{T}$ generated by the subbasis} $\mathcal{S}$ is defined as the collection of all unions of finite intersections of elements of $\mathcal{S}$:
\[U\in\mathcal{T}\iff U=\text{union of finite intersections in }\mathcal{S}.\]
\end{definition}

We now check that $\mathcal{T}$ is a topology. Consider the collection
\[\mathcal{B}=\{\text{all finite intersections of elements of }\mathcal{S}\}.\]
It suffices to show that $\mathcal{B}$ is a basis, for then by \cref{lemma:topology-generated-basis-unions}, the collection $\mathcal{T}$ of all unions of elements of $\mathcal{B}$ is a topology.
\begin{enumerate}[label=(\roman*)]
\item Given $x\in X$, it belongs to an element of $\mathcal{S}$ and hence to an element of $\mathcal{B}$.
\item Let
\[B_1=S_1\cap\cdots\cap S_m,\quad B_2=S_1^\prime\cap\cdots\cap S_n^\prime\]
be two elements of $\mathcal{B}$. Their intersection
\[B_1\cap B_2=(S_1\cap\cdots\cap S_m)\cap(S_1^\prime\cap\cdots\cap S_n^\prime)\]
is also a finite intersection of elements of $\mathcal{S}$, so it belongs to $\mathcal{B}$.
\end{enumerate}
\pagebreak

\section{Examples of Topologies}
\subsection{Order Topology}
\begin{definition}[Order topology]
Let $(X,<)$, $|X|>1$. Let $\mathcal{B}$ be the collection of all sets of the following types:
\begin{enumerate}[label=(\roman*)]
\item All open intervals $(a,b)$ in $X$.
\item All intervals of the form $[a_0,b)$, where $a_0$ is the smallest element (if any) of $X$.
\item All intervals of the form $(a,b_0]$, where $b_0$ is the largest element (if any) of $X$.
\end{enumerate}
The topology generated by $\mathcal{B}$ is called the \vocab{order topology}.
\end{definition}

We need to check that $\mathcal{B}$ is a basis of $X$.
\begin{enumerate}[label=(\roman*)]
\item Every $x\in X$ lies in some element of $\mathcal{B}$: the smallest element (if any) lies in all sets of type (ii), the largest element (if any) lies in all sets of type (iii), and every other element lies in a set of type (i).
\item The intersection of any two sets of the preceding types is a set of one of these types, or is empty. Several cases need to be checked; we leave it to you.

For instance, let $x\in(a,b)\cap(c,d)$. Let $p=\max\{a,c\}$, $q=\min\{b,d\}$. Then $x\in(p,q)\subset(a,b)\cap(c,d)$, where $(p,q)\in\mathcal{B}$.
\end{enumerate}

\begin{example}
\begin{itemize}
\item The standard topology on $\RR$ is just the order topology derived from the usual order on $\RR$.
\end{itemize}
\end{example}

\begin{definition}
Let $(X,<)$, $a\in X$. Then the following subsets of $X$ are \vocab{rays} determined by $a$:
\begin{align*}
(a,+\infty)&=\{x\in X\mid x>a\},\\
[a,+\infty)&=\{x\in X\mid x\ge a\},\\
(-\infty,a)&=\{x\in X\mid x<a\},\\
(-\infty,a]&=\{x\in X\mid x\le a\}.
\end{align*}
\end{definition}

$(a,+\infty)$ and $(-\infty,a)$ are called \emph{open rays}, since they are open; for instance, $(a,+\infty)=\bigcup_{x>a}(a,x)$. Similarly, $[a,+\infty)$ and $(-\infty,a]$ are \emph{closed rays}.

\begin{lemma}
The collection of open rays form a subbasis for the order topology.
\end{lemma}

\begin{proof}
Let $\mathcal{T}$ be the order topology on $X$, let $\mathcal{T}^\prime$ be the topology generated by the subbasis of open rays. We will show that $\mathcal{T}=\mathcal{T}^\prime$.
\begin{itemize}
\item Because the open rays are open in the order topology, the topology they generate is contained in the order topology. Hence $\mathcal{T}^\prime\subset\mathcal{T}$.
\item On the other hand, every basis element for the order topology equals a finite intersection of open rays; the interval $(a,b)$ equals the intersection of $(-\infty,b)$ and $(a,+\infty)$, while $[a_0,b)$ and $(a,b_0]$, if they exist, are themselves open rays. Hence the topology generated by the open rays contains the order topology, so $\mathcal{T}\subset\mathcal{T}^\prime$.
\end{itemize}
\end{proof}

\subsection{Product Topology}
\begin{definition}
Let $(X,\mathcal{T}_X)$ and $(Y,\mathcal{T}_Y)$ be topological spaces. The \vocab{product topology} on $X\times Y$ is the topology $\mathcal{T}_{X\times Y}$ with basis
\[\mathcal{B}=\{U\times V\mid U\in\mathcal{T}_X,V\in\mathcal{T}_Y\}.\]
\end{definition}

We first check that $\mathcal{B}$ is a basis.
\begin{enumerate}[label=(\roman*)]
\item $X\times Y$ is a basis element, so every element of $X\times Y$ is contained in $X\times Y$.
\item Let $U_1\times V_1,U_2\times V_2\in\mathcal{B}$. Then their intersection is
\[(U_1\times V_1)\cap(U_2\times V_2)=(U_1\cap U_2)\times(V_1\cap V_2).\]
Since $U_1\cap U_2\in\mathcal{T}_X$, $V_1\cap V_2\in\mathcal{T}_Y$, we have that $(U_1\cap U_2)\times(V_1\cap V_2)\in\mathcal{B}$.
\end{enumerate}

\subsection{Subspace Topology}
\begin{definition}[Subspace]
Let $(X,\mathcal{T})$ be a topological space. If $Y\subset X$, the collection
\[\mathcal{T}_Y\coloneqq\{Y\cap U\mid U\in\mathcal{T}\}\]
is a topology on $Y$, called the \vocab{subspace topology}. With this topology, $Y$ is called a \vocab{subspace} of $X$; its open sets consist of all intersections of open sets of $X$ with $Y$.
\end{definition}

We check that $\mathcal{T}_Y$ is a topology.

\begin{lemma}
If $\mathcal{B}$ is a basis for the topology of $X$, then
\[\mathcal{B}_Y=\{B\cap Y\mid B\in\mathcal{B}\}\]
is a basis for the subspace topology on $Y$.
\end{lemma}

\begin{lemma}
Let $Y$ be a subspace of $X$. If $U$ is open in $Y$, and $Y$ is open in $X$, then $U$ is open in $X$.
\end{lemma}

\begin{proposition}
If $A$ is a subspace of $X$, and $B$ is a subspace of $Y$, then the product topology on $A\times B$ is the same as the topology $A\times B$ inherits as a subspace of $X\times Y$.
\end{proposition}
\pagebreak

\section{Closed Sets and Limit Points}
Let $X$ be a topological space.

Note that if $U$ is an open set containing $x$, we often say that $U$ is a \vocab{neighbourhood} of $x$.

\subsection{Closed Sets}
\begin{definition}[Closed set]
$A\subset X$ is \vocab{closed} if its complement $A^c$ is open.
\end{definition}

The collection of closed subsets of a space $X$ has properties similar to those satisfied by the collection of open subsets of $X$:
\begin{lemma}
Let $X$ be a topological space.
\begin{enumerate}[label=(\roman*)]
\item $\emptyset$ and $X$ are closed.
\item Arbitrary intersections of closed sets are closed.
\item Finite unions of closed sets are closed.
\end{enumerate}
\end{lemma}

\begin{proof} \
\begin{enumerate}[label=(\roman*)]
\item $\emptyset$ and $X$ are closed because they are the complements of the open sets $X$ and $\emptyset$, respectively.
\item Suppose $\{A_i\mid i\in I\}$ is a collection of closed sets. By de Morgan's laws,
\[\brac{\bigcap_{i\in I}A_i}^c=\bigcup_{i\in I}{A_i}^c.\]
Since ${A_i}^c$'s are open, the RHS is open since it is an arbitrary union of open sets. Hence $\bigcap A_i$ is closed.
\item Suppose $A_i$ is closed for $i=1,\dots,n$. Then
\[\brac{\bigcup_{i=1}^{n}A_i}^c=\bigcap_{i=1}^{n}{A_i}^c.\]
The RHS is a finite intersection of open sets and is thus open. Hence $\bigcup A_i$ is closed.
\end{enumerate}
\end{proof}

\begin{remark}
Note that $\emptyset$ and $X$ are both open and closed. This explains the statement ``a door is not a set'': a door must be either open or closed, and cannot be both, while a set can be open, or closed, or both, or neither!
\end{remark}

If $Y$ is a subspace of $X$, we say $A$ is closed in $Y$ if $A\subset Y$ and $A$ is closed in the subspace topology of $Y$ (that is, if $Y\setminus A$ is open in $Y$). We have the following result:

\begin{proposition}
Let $Y$ be a subspace of $X$. Then $A$ is closed in $Y$ if and only if it equals the intersection of a closed set of $X$ with $Y$.
\end{proposition}

\begin{proof} \

\fbox{$\impliedby$} Assume that $A=C\cap Y$, where $C$ is closed in $X$. Then $X\setminus C$ is open in $X$, so that $(X\setminus C)\cap Y$ is open in $Y$, by definition of the subspace topology. But $(X\setminus C)\cap Y=Y\setminus A$. Hence $Y\setminus A$ is open in $Y$, so that $A$ is closed in $Y$.

\fbox{$\implies$} Suppose $A$ is closed in $Y$. Then $Y\setminus A$ is open in $Y$, so that by definition it equals the intersection of an open set $U$ of $X$ with $Y$. The set $X\setminus U$ is closed in $X$, and $A=Y\cap(X\setminus U)$, so that $A$ equals the intersection of a closed set of $X$ with $Y$, as desired.
\end{proof}

\begin{proposition}
Let $Y$ be a subspace of $X$. If $A$ is closed in $Y$, and $Y$ is closed in $X$, then $A$ is closed in $X$.
\end{proposition}

\subsection{Closure and Interior}
\begin{definition}
The \vocab{interior} of $A\subset X$ is the union of all open sets contained in $A$, denoted by $\Int A$.

The \vocab{closure} of $A$ is the intersection of all closed sets contained in $A$, denoted by $\overline{A}$.
\end{definition}

\begin{proposition}
Let $Y$ be a subspace of $X$; let $A\subset Y$, let $\overline{A}$ denote the closure of $A$ in $X$. Then the closure of $A$ in $Y$ equals $\overline{A}\cap Y$.
\end{proposition}



\subsection{Limit Points}
\begin{definition}
Suppose $A\subset X$. $x\in X$ is a limit point of $A$ if every neighbourhood of $x$ intersects $A$ in some point other than $x$ itself.
\end{definition}

$A^\prime$ denotes the set of all limit points of $A$.

\begin{proposition}
Let $A\subset X$. Then $\overline{A}=A\cup A^\prime$.
\end{proposition}

\begin{corollary}
$A\subset X$ is closed if and only if it contains all its limit points.
\end{corollary}

\subsection{Hausdorff Spaces}
\begin{definition}[Hausdorff space]
A \vocab{Hausdorff space} is a topological space $X$ such that for all distinct $x_1,x_2\in X$, there exist neighbourhoods $U_1$ and $U_2$ of $x_1$ and $x_2$ respectively that are disjoint.
\end{definition}

\begin{proposition}
Every finite point set in a Hausdorff space $X$ is closed.
\end{proposition}

The condition that finite point sets be closed is in fact weaker than the Hausdorff condition. For example, $\RR$ in the finite complement topology is not a Hausdorff space, but it is a space in which finite point sets are closed. The condition that finite point sets be closed has been given a name of its own: it is called the \emph{T1 axiom}.

\begin{proposition}
Let $X$ be a space satisfying the T1 axiom; let $A\subset X$. Then $x$ is a limit point of $A$ if and only if every neighborhood of $x$ contains infinitely many points of $A$.
\end{proposition}

\begin{proposition}
If $X$ is a Hausdorff space, then a sequence of points of $X$ converges to at most one point of $X$.
\end{proposition}

\begin{proposition}
Every simply ordered set is a Hausdorff space in the order topology. The product of two Hausdorff spaces is a Hausdorff space. A subspace of a Hausdorff space is a Hausdorff space.
\end{proposition}

\begin{comment}
\begin{definition}
A topological space $(X,\mathcal{T})$ is \vocab{metrisable} if it arises from (at least oe) metric space $(X,d)$, i.e. there is at least one metric $d$ on $X$ such that $\mathcal{T}=\mathcal{T}_d$.
\end{definition}

\begin{definition}
Two metrics on a set are \vocab{topologically equivalent} if they give rise to the same topology.
\end{definition}

\begin{example}
\begin{itemize}
\item The metrics $d_1$, $d_2$, $d_\infty$ on $\RR^n$ are all topologically equivalent. (Recall that $d_1$, $d_2$, $d_\infty$ are the metrics arising from the norms $\norm{\cdot}_1$, $\norm{\cdot}_2$, $\norm{\cdot}_\infty$, respectively.)
We shall call the topology defined by the above metrics the \emph{standard} (or canonical) topology on $\RR^n$.
\item The discrete topology on a non-empty set $X$ is metrisable, using the metric
\[d(x,y)=\begin{cases}
0&\text{if }x=y,\\
1&\text{if }x\neq y.
\end{cases}\]
It is easy to check that this is a metric. To see that is gives the discrete topology, consider any subset $U\subset X$. Then for every $x\in U$, $B_\frac{1}{2}(x)\subset U$.
\end{itemize}
\end{example}

\begin{definition}
Given two topologies $\mathcal{T}_1$ and $\mathcal{T}_2$ on the same set, we say $\mathcal{T}_1$ is \vocab{coarser} than $\mathcal{T}_2$ if $\mathcal{T}_1\subset\mathcal{T}2$.
\end{definition}

\begin{remark}
For any space $(X,\mathcal{T})$, the indiscrete topology on $X$ is coarser than $\mathcal{T}$ which in turn is coarser than the discrete topology on $X$.
\end{remark}

\begin{definition}
Let $(X,\mathcal{T})$ be a topological space. A subset $V$ of $X$ is \vocab{closed} in $X$ if $X\setminus V$ is open in X (i.e. $X\setminus V\in\mathcal{T}$).
\end{definition}

\begin{example} \
\begin{itemize}
\item In the space $[0,1)$ with the usual topology coming from the Euclidean metric, $[1/2,1)$ is closed.
\item In a discrete space, all subsets are closed since their complements are open.
\item In the co-finite topology on a set $X$, a subset is closed if and only if it is finite or all of $X$.
\end{itemize}
\end{example}

\begin{proposition}
Let $X$ be a topological space. Then
\begin{enumerate}[label=(\roman*)]
\item $X$, $\emptyset$ are closed in $X$;
\item if $V_1$, $V_2$ are closed in $X$ then $V_1\cup V_2$ is closed in $X$;
\item if $V_i$ is closed in $X$ for all $i\in I$ then $\bigcap_{i\in I}Vi$ is closed in $X$.
\end{enumerate}
\end{proposition}

\begin{proof}
These properties follow from (i), (ii), (iii) of definition of topological space, and from the De Morgan laws.
\end{proof}

\begin{definition}[Convergent sequence]
A sequence $\{x_n\}_{n\in\NN}$ in a topological space $X$ converges to a point $x\in X$ if given any open set $U$ containing $x$ there exists $N\in\NN$ such that $x_n\in U$ for all $n>N$.
\end{definition}

\begin{example} \
\begin{itemize}
\item In a metric space this is equivalent to the metric definition of convergence.
\item In an indiscrete topological space $X$ any sequence converges to any point $x\in X$.
\item In an infinite space $X$ with the co-finite topology any sequence $\{x_n\}$ of pairwise distinct elements (i.e. such that $x_n\neq x_m$ when $n\neq m$) converges to any point $x\in X$.
\end{itemize}
\end{example}
\end{comment}
\part{Topology}\label{part:topology}
You have already studied metric spaces in some detail. These are objects where one has a notion of distance between points, satisfying some simple axioms. They have a rich and interesting theory, which leads to such concepts as connectedness, completeness and
compactness.

Two metric spaces are viewed as ``the same'' if there is an isometry between them, which is a bijection that preserves distances. But there is a much more flexible notion of equivalence: two spaces are homeomorphic if there is a continuous bijection between them with continuous inverse. Many properties of metric spaces are preserved by a homeomorphism (for example, connectedness and compactness). Thus homeomorphic metric spaces may have very different metrics, but nevertheless have many properties in common. The conclusion to draw from this is that a metric is, frequently, a somewhat artificial and rigid piece of structure. So, one is led naturally to the study of Topology. The fundamental objects in Topology are topological spaces. Here, there is no metric in general. But one still has a notion of open sets, and so concepts such as connectedness and compactness continue to make sense.

Why study Topology? The reason is that it simultaneously simplifies and generalises the theory of metric spaces. By discarding the metric, and focusing solely on the more basic and fundamental notion of an open set, many arguments and proofs are simplified. And many constructions (such as the important concept of a quotient space) cannot be carried out in the setting of metric spaces: they need the more general framework of topological spaces. But perhaps the most important reason is that the spaces that arise naturally in Topology have a particularly beautiful theory.
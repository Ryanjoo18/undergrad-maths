\chapter{Calculus}
\section{Multivariable Calculus}
\subsection{Partial Differentiation}
\begin{definition}[Partial derivative]
Let $f:\RR^n\to\RR$ be a function of $n$ variables. The \vocab{partial derivative} of $f$ with respect to the $i$-th variable is the function
\[\pdv{f}{x_i}=\lim_{h\to0}\frac{f(x_1,\dots,x_{i-1},x_i+h,x_{i+1},\dots,x_n)-f(x_1,\dots,x_n)}{h}.\]
\end{definition}

\begin{notation}
We define second and higher order partial derivatives in a similar manner to how we define them for full derivatives. So in the case of second order partial derivatives of a function $f(x,y)$, we have
\begin{align*}
\pdv[2]{f}{x}&=\pdv{}{x}\brac{\pdv{f}{x}}=f_{xx},\\
\pdv[2]{f}{y}&=\pdv{}{y}\brac{\pdv{f}{y}}=f_{yy},\\
\pdv{f}{y,x}&=\pdv{}{y}\brac{\pdv{f}{x}}=f_{xy},\\
\pdv{f}{x,y}&=\pdv{}{x}\brac{\pdv{f}{y}}=f_{yx}.
\end{align*}
\end{notation}

If $f_{xy}$ and $f_{yx}$ are both defined and continuous in a region containing the point $(a,b)$, then 
\[f_{xy}(a,b)=f_{yx}(a,b);\]
this is known as \emph{Clairaut's theorem}. A consequence of this theorem is that we don't need to keep track of the order in which we take derivatives.

\begin{theorem}[Chain rule]
Let $F(t)=f\brac{u(t),v(t)}$ with $u$ and $v$ differentiable and $f$ being continuously differentiable in each variable. Then
\begin{equation}
\dv{F}{t}=\pdv{f}{u}\dv{u}{t}+\pdv{f}{v}\dv{v}{t}.
\end{equation}
\end{theorem}



\subsection{Coordinate systems and Jacobians}
\subsection{Double Integrals}
\subsection{Parametric representation of curves and surfaces}
\subsection{The gradient vector}
\subsection{Taylor's theorem}
\subsection{Critical points}
\subsection{Lagrange multipliers}
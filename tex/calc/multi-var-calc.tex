\chapter{Multivariable Calculus}
\section{Partial Differentiation}
\subsection{Computation of Partial Derivatives}
\begin{definition}[Partial derivative]
Let $f:\RR^n\to\RR$ be a function of $n$ variables. The \vocab{partial derivative} of $f$ with respect to the $i$-th variable at the point $(p_1,\dots,p_n)$ is the function
\[\pdv{f}{x_i}(p_1,\dots,p_n)=\lim_{h\to0}\frac{f(p_1,\dots,p_i+h,\dots,p_n)-f(p_1,\dots,p_n)}{h}.\]
\end{definition}

\begin{notation}
We shall occasionally write $f_x$ for $\pdv{f}{x}$, etc.
\end{notation}

\begin{remark}
If $f(x)$ is a function of a single variable, then $\dv{f}{x}=\pdv{f}{x}$.
\end{remark}

\begin{notation}
We define second and higher order partial derivatives in a similar manner to how we define them for full derivatives. So in the case of second order partial derivatives of a function $f(x,y)$, we have
\begin{align*}
\pdv[2]{f}{x}&=\pdv{}{x}\brac{\pdv{f}{x}}=f_{xx},\\
\pdv[2]{f}{y}&=\pdv{}{y}\brac{\pdv{f}{y}}=f_{yy},\\
\pdv{f}{y,x}&=\pdv{}{y}\brac{\pdv{f}{x}}=f_{xy},\\
\pdv{f}{x,y}&=\pdv{}{x}\brac{\pdv{f}{y}}=f_{yx}.
\end{align*}
\end{notation}

\begin{proposition}[Clairaut's theorem]
If $f_{xy}$ and $f_{yx}$ are both defined and continuous in a region containing the point $(a,b)$, then 
\[f_{xy}(a,b)=f_{yx}(a,b).\]
\end{proposition}

A consequence of this theorem is that we don't need to keep track of the order in which we take derivatives.

Recall the chain rule for one variable, which arises when we want to find the derivative of the composition of two functions $f\brac{u(x)}$ with respect to $x$. 
Likewise we might have a function $f(u,v)$ of two variables $u$ and $v$, each of which is itself a function of $x$ and $y$; thus make the composition
\[F(x,y)=f\brac{u(x,y),v(x,y)},\]
which is a function of $x$ and $y$, and we might then want to calculate the partial derivatives $\pdv{F}{x}$ and $\pdv{F}{y}$.

\begin{lemma}[Chain rule]
Let $F(t)=f\brac{u(t),v(t)}$ with $u$ and $v$ differentiable and $f$ being continuously differentiable in each variable. Then
\begin{equation}
\dv{F}{t}=\pdv{f}{u}\dv{u}{t}+\pdv{f}{v}\dv{v}{t}.
\end{equation}
\end{lemma}

\begin{corollary}
Let $F(x,y)=f\brac{u(x,y),v(x,y)}$, with $u$ and $v$ differentiable in each variable, and $f$ being continuously differentiable in each variable. Then
\begin{equation}
\begin{split}
\pdv{F}{x}&=\pdv{f}{u}\pdv{u}{x}+\pdv{f}{v}\pdv{v}{x},\\
\pdv{F}{y}&=\pdv{f}{u}\pdv{u}{y}+\pdv{f}{v}\pdv{v}{y}.
\end{split}
\end{equation}
\end{corollary}

\begin{proof}
This follows from the previous theorem by treating first $x$ and then $y$ as constants when differentiating.
\end{proof}

\subsection{Partial Differential Equations}

\pagebreak

\section{Multiple Integrals}
\begin{definition}
By a \vocab{scalar field} $\phi$ on $\RR^3$ we shall mean a map $\phi:\RR^3\to\RR$; by a \vocab{vector field} $\vb{F}$ on $\RR^3$ we shall mean a map $\vb{F}:\RR^3\to\RR^3$.
\end{definition}

\begin{remark}
We will typically We will typically assume that scalar and vector fields are smooth -- their partial derivatives exist with respect to $x$, $y$ and $z$ to all orders -- for brevity, this will not always be stated.

Occasionally we may consider more general scalar fields $\phi:\RR^n\to\RR$ and vector fields $\vb{F}:\RR^n\to\RR^m$.
\end{remark}

\subsection{Double Integrals}
An informal definition of a double integral is as follows:

Consider a region $R\subset\RR^2$, together with a scalar field $\phi(x,y)$. Partition the region into $n$ area elements of equal area $\delta A=\delta x\delta y$. Suppose the scalar field $\phi(\vb{r})$ takes $\phi_i$ at the centre of the $i$-th element.

On partitioning with smaller and smaller area elements, by taking the limit $\delta A\to0$,
\[\lim\sum_{i=1}^{n}\phi_i\delta A=\lim\sum_{i=1}^{n}\phi_i\delta x\delta y\coloneqq\iint_R\phi\dd{A}=\iint_R\phi\dd{x}\dd{y}.\]

\begin{proposition}[Properties of double integrals]
The following properties are inherited from integration with respect to one variable.
\begin{enumerate}[label=(\roman*)]
\item Linearity: for $a,b\in\RR$,
\[\iint_R\brac{af(x,y)+bg(x,y)}\dd{A}=a\iint_Rf(x,y)\dd{A}+b\iint_Rg(x,y)\dd{A}.\]
\item Order: if $f(x,y)\ge g(x,y)$ for all $(x,y)\in R$ then
\[\iint_Rf(x,y)\dd{A}\ge\iint_Rg(x,y)\dd{A}.\]
\item Domain splitting: if $R=R_1\cup R_2$ and $R_1\cap R_2=\emptyset$ then
\[\iint_Rf(x,y)\dd{A}=\iint_{R_1}f(x,y)\dd{A}+\iint_{R_2}f(x,y)\dd{A}.\]
\end{enumerate}
\end{proposition}

\begin{proposition}[Fubini's theorem]
Given a continuous, bounded function $f(x,y)$ on a rectangular domain $R=X\times Y$, then
\begin{equation}
\iint_{R}f(x,y)\dd{A}=\int_{Y}\int_{X}f(x,y)\dd{x}\dd{y}=\int_{X}\int_{Y}f(x,y)\dd{y}\dd{x}.
\end{equation}
\end{proposition}

Sometimes it may be more convenient to work with coordinates other than Cartesian, so we would need a change of variables of integration. 
The \emph{Jacobian}\footnote{after the German mathematician Carl Jacobi (1804--1851)}, or rather its modulus, is a measure of how a general mapping stretches space locally, near a particular point, even when this stretching effect varies from point to point.

\begin{definition}[Jacobian]
Given two coordinates $u(x,y)$ and $v(x,y)$ which depend on variables $x$ and $y$, the \vocab{Jacobian} is defined to be the determinant
\[\frac{\partial(u,v)}{\partial(x,y)}=\begin{vmatrix}
\pdv{u}{x}&\pdv{u}{y}\\
\pdv{v}{x}&\pdv{v}{y}
\end{vmatrix}.\]
In 3D,
\[\frac{\partial(u,v,w)}{\partial(x,y,z)}=\begin{vmatrix}
\pdv{u}{x}&\pdv{u}{y}&\pdv{u}{z}\\
\pdv{v}{x}&\pdv{v}{y}&\pdv{v}{z}\\
\pdv{w}{x}&\pdv{w}{y}&\pdv{w}{z}
\end{vmatrix}.\]
\end{definition}

That is to say, under the transformation $(x,y)\to(u,v)$, the area element $\dd{x}\dd{y}$ in the $xy$-plane is equivalent to $\displaystyle\absolute{\frac{\partial(u,v)}{\partial(x,y)}}\dd{u}\dd{v}$, where $\dd{u}\dd{v}$ is the area element in $uv$-plane. If $A$ is a domain in the $xy$-plane mapped one to one and onto a domain $B$ in the $uv$-plane then
\[\int_Af(x,y)\dd{x}\dd{y}=\int_Bf\brac{x(u,v),y(u,v)}\absolute{\frac{\partial(u,v)}{\partial(x,y)}}\dd{u}\dd{v}.\]

\begin{example}[Plane polar coordinates]
For $P=(x,y)\in\RR^2\setminus\{(0,0)\}$ we can determine the position of $P$ by its distance $r$ from the origin $(0,0)$ and the anti-clockwise angle $\theta$ that $\overrightarrow{OP}$ makes with the $x$-axis.

Let
\[x=r\cos\theta,\quad y=r\sin\theta.\]
Then the Jacobian is
\[\frac{\partial(x,y)}{\partial(r,\theta)}=\begin{vmatrix}
\cos\theta&-r\sin\theta\\
\sin\theta&r\cos\theta
\end{vmatrix}=r.\]
\end{example}

Notice that
\[\frac{\partial(x,y)}{\partial(r,\theta)}\frac{\partial(r,\theta)}{\partial(x,y)}=1.\]
This result holds more generally:

\begin{proposition}
Let $r$ and $s$ be functions of variables $u$ and $v$, which are in turn functions of $x$ and $y$. Then
\[\frac{\partial(r,s)}{\partial(x,y)}=\frac{\partial(r,s)}{\partial(u,v)}\frac{\partial(u,v)}{\partial(x,y)}.\]
\end{proposition}

\begin{example}[Parabolic coordinates]
Let
\[x=\frac{1}{2}(u^2-v^2),\quad y=uv.\]

Then the Jacobian is
\[\frac{\partial(x,y)}{\partial(u,v)}=\begin{vmatrix}
u&-v\\
v&u
\end{vmatrix}=u^2+v^2.\]
\end{example}

We can naturally extend plane polar coordinates into three dimensions by adding a $z$ coordinate.

\begin{example}[Cylindrical polar coordinates]
Let
\[x=r\cos\theta,\quad y=r\sin\theta,\quad z=z.\]
Then the Jacobian is
\[\frac{\partial(x,y,z)}{\partial(r,\theta,z)}=\begin{vmatrix}
\cos\theta&-r\sin\theta&0\\
\sin\theta&r\cos\theta&0\\
0&0&1
\end{vmatrix}=r.\]
\end{example}

\begin{example}[Spherical polar coordinates]
Let $(x,y,z)$ be the Cartesian coordinates for a general point $P\in\RR^3\setminus\{(0,0,0)\}$. Let $r$ be the distance between $P$ and $O$, let $\theta$ be the angle from the $z$-axis to $\overrightarrow{OP}$. Change $(x,y)$ to its polar coordinates $(\rho\cos\phi,\rho\sin\phi)$, so that $\rho=r\sin\theta$.

Let
\[x=r\sin\theta\cos\phi,\quad y=r\sin\theta\sin\phi,\quad z=r\cos\theta.\]

Then the Jacobian is
\[\frac{\partial(x,y,z)}{\partial(r,\theta,\phi)}=\begin{vmatrix}
\sin\theta\cos\phi&r\cos\theta\cos\phi&-r\sin\theta\sin\phi\\
\sin\theta\sin\phi&r\cos\theta\sin\phi&r\sin\theta\cos\phi\\
\cos\theta&-r\sin\theta&0
\end{vmatrix}=r^2\sin\theta.\]
\end{example}

\subsection{Volume Integrals}
Consider a scalar field $\phi(x,y,z)$, and a three-dimensional region $D\subset\RR^3$. Partition $V$ into $n$ cubic volume elements of equal volume $\delta V=\delta x\delta y\delta z$. Let $\phi_i$ denote the value of $\phi$ at the centre of the $i$-th volume element. On partitioning with smaller and smaller volume elements, and taking the limit $\delta V\to0$,
\[\lim\sum_{i=1}^{n}\phi_i\delta V\coloneqq\iiint_{D}\phi\dd{V}.\]

\section{Surfaces}
\subsection{Parametric Representation}
\begin{definition}[Tangent plane]
Let $\vb{r}:U\to\RR^3$ be a smooth parametrised surface\footnote{A \emph{smooth parametrised surface} is a map $\vb{r}$, given by the parameterisation \[r:U\to\RR^3:(u,v)\mapsto\brac{x(u,v),y(u,v),z(u,v)},\] where $U$ is an open subset of $\RR^2$, such that
\begin{enumerate}[label=(\roman*)]
\item $x$, $y$, $z$ have continuous partial derivatives with respect to $u$ and $v$ of all orders;
\item $\vb{r}$ is a bijection, with both $\vb{r}$ and $\vb{r}^{-1}$ being continuous;
\item at each point the vectors $\pdv{\vb{r}}{u}$ and $\pdv{\vb{r}}{v}$ are linearly independent.
\end{enumerate}
}, and let $\vb{p}$ be a point on the surface. The plane containing $\vb{p}$ and which is parallel to the vectors
\[\pdv{\vb{r}}{u}(\vb{p})\quad\text{and}\quad\pdv{\vb{r}}{v}(\vb{p})\]
is called the \vocab{tangent plane}\index{tangent plane} to $\vb{r}(U)$ at $\vb{p}$.
\end{definition}

\begin{remark}
For a smooth parametrised surface, these vectors are linearly independent, so the tangent plane is well-defined.
\end{remark}

\begin{definition}[Normal vector]
Any vector in the direction
\[\pdv{\vb{r}}{u}(\vb{p})\times\pdv{\vb{r}}{v}(\vb{p})\]
is said to be \vocab{normal} to the surface at $\vb{p}$.
\end{definition}

\subsection{Scalar Line Integrals}
In this section we will introduce the idea of the scalar line integral of a vector field. The physical interpretation of such integrals depends on the nature of the particular vector field under consideration. In the case of force fields, the scalar line integral represents the work done by the force.

\begin{definition}[Scalar line integral]
The \vocab{scalar line integral} of a vector field $\vb{F}(\vb{r})$ along a path $C$ given by $\vb{r}=\vb{r}(t)$, from $\vb{r}(t_0)$ to $\vb{r}(t_1)$, is
\begin{equation}
\int_C\vb{F}(\vb{r})\cdot\dd{\vb{r}}=\int_{t_0}^{t_1}\vb{F}(t)\cdot\dv{\vb{r}}{t}\dd{t}.
\end{equation}
\end{definition}

The path $C$ in the integral is a directed curve, with start point $A$ and end point $B$; thus we often write the integral as
\[\int_{A}^{B}\vb{F}(\vb{r})\cdot\dd{\vb{r}}.\]
Traversing the same path in the opposite direction changes the sign of the scalar line integral, so
\[\int_{B}^{A}\vb{F}(\vb{r})\cdot\dd{\vb{r}}=-\int_{A}^{B}\vb{F}(\vb{r})\cdot\dd{\vb{r}}.\]

\subsection{Length of Curve}
Consider the scalar line integral of the vector field $\vb{F}(\vb{r})$ along the curve $C$ given by $\vb{r}(t)$, from $\vb{r}(t_0)$ to $\vb{r}(t_1)$, namely
\[\int_C\vb{F}(\vb{r})\cdot\dd{\vb{r}}=\int_{t_0}^{t_1}\vb{F}(t)\cdot\dv{\vb{r}}{t}\dd{t},\]
and let $t$ represent time. Then $\vb{r}(t)$ represents the point on the curve $C$ corresponding to time $t$, and $\dot{\vb{r}}(t)$ represents the velocity of the point as it moves along the curve. Now let
\[\vb{F}(t)=\frac{\dot{\vb{r}}(t)}{|\dot{\vb{r}}(t)|};\]
then $\vb{F}(t)$ represents a unit vector in the direction of $\dot{\vb{r}}(t)$. We have
\[\vb{F}(t)\cdot\dv{\vb{r}}{t}=\frac{\dot{\vb{r}}(t)}{|\dot{\vb{r}}(t)|}\cdot\dot{\vb{r}}(t)=|\dot{\vb{r}}(t)|,\]
so that the scalar line integral becomes
\begin{equation}
\int_C\vb{F}(\vb{r})\cdot\dd{\vb{r}}=\int_{t_0}^{t_1}|\dot{\vb{r}}(t)|\dd{t}=\int_{t_0}^{t_1}\sqrt{\brac{\dv{x}{t}}^2+\brac{\dv{y}{t}}^2}\dd{t}.
\end{equation}

\subsection{Surface Integrals}
Let $\vb{r}:U\to\RR^3$ be a smooth parametrised surface with
\[\vb{r}(u,v)=\brac{x(u,v),y(u,v),z(u,v)}\]

\section{Line Integrals}
\begin{definition}
By a \vocab{curve} we shall mean a piecewise smooth function $\gamma:I\to\RR^3$ defined on an interval $I$ of $\RR$. Notice that order on $I$ also gives the curve $\gamma$ an \emph{orientation}.
\end{definition}

We shall also use the term curve to describe the images of such maps $\gamma$. Given such an image then it will be the image of more than one such map $\gamma$ and we will talk about parameterisations $\gamma_1$ and $\gamma_2$ of the curve.



\section{Grad, Div, Curl}
\subsection{Definitions and Identities}
The differential operator
\[\nabla=\brac{\pdv{}{x},\pdv{}{y},\pdv{}{z}}\]
is called \emph{del} or \emph{nabla}.

\begin{definition}[Gradient]
Let $\phi:\RR^3\to\RR$ be a scalar field. Then the \vocab{gradient} of $\phi$ is
\[\nabla\phi\coloneqq\brac{\pdv{\phi}{x},\pdv{\phi}{y},\pdv{\phi}{z}}.\]
\end{definition}

Note $\nabla$ takes scalar fields to vector fields.

\begin{definition}[Divergence]
Let $\vb{F}:\RR^3\to\RR^3$ be a vector field, with $\vb{F}=(F_1,F_2,F_3)$. Then the \vocab{divergence} of $\FF$ is
\[\nabla\cdot\vb{F}\coloneqq\pdv{F_1}{x}+\pdv{F_2}{y}+\pdv{F_3}{z}.\]
\end{definition}

Note $\nabla\cdot$ takes vector fields to scalar fields.

\begin{definition}[Curl]
Let $\vb{F}:\RR^3\to\RR^3$ be a vector field, with $\vb{F}=(F_1,F_2,F_3)$. Then the \vocab{curl} of $\vb{F}$ is
\[\nabla\times\vb{F}=\begin{vmatrix}
\vb{i}&\vb{j}&\vb{k}\\
\pdv{}{x}&\pdv{}{y}&\pdv{}{z}\\
F_1&F_2&F_3
\end{vmatrix}=
\brac{\pdv{F_3}{y}-\pdv{F_2}{z},\pdv{F_1}{z}-\pdv{F_3}{x},\pdv{F_2}{x}-\pdv{F_1}{y}}.\]
\end{definition}

Note $\nabla\times$ takes vector fields to vector fields.

We also have the differential operator
\[\vb{F}\cdot\nabla=(F_1,F_2,F_3)\cdot\brac{\pdv{}{x},\pdv{}{y},\pdv{}{z}}=F_1\pdv{}{x}+F_2\pdv{}{y}+F_3\pdv{}{z},\]
which gives the directional derivative in the direction of $\vb{F}$ if $\vb{F}$ is a unit vector.

\begin{definition}[Laplacian]
For any scalar field, the \vocab{Laplacian} is
\[\nabla^2\phi\coloneqq\nabla\cdot\nabla\phi=\pdv[2]{\phi}{x}+\pdv[2]{\phi}{y}+\pdv[2]{\phi}{z}.\]
\end{definition}

\begin{notation}
There are neater ways to write out the above formulae:
\[\nabla\phi=\sum_i\pdv{\phi}{x_i}\vb{e}_i,\quad\nabla\cdot\vb{F}=\sum_i\vb{e}_i\cdot\pdv{\vb{F}}{x_i},\quad\nabla\times\vb{F}=\sum_i\vb{e}_i\times\pdv{\vb{F}}{x_i},\quad\vb{F}\times\nabla=\sum_i(\vb{F}\cdot\vb{e}_i)\pdv{}{x_i},\]
where the dummy variable $i$ ranges over $1,2,3$, and $\vb{e}_1,\vb{e}_2,\vb{e}_3$ is any right-handed orthonormal basis.
\end{notation}

\begin{proposition}
Let $\phi$ and $\psi$ be differentiable functions of $x,y,z$. Then
\begin{enumerate}[label=(\roman*)]
\item $\nabla(\phi\psi)=\phi\nabla\psi+\psi\nabla\phi$;
\item $\nabla(\phi^n)=n\phi^{n-1}\nabla\phi$;
\item $\nabla\brac{\frac{\phi}{\psi}}=\frac{\psi\nabla\phi-\phi\nabla\psi}{\psi^2}$;
\item $\nabla\brac{\phi(\psi(\vb{x}))}=\phi^\prime(g(\vb{x}))\nabla g(\vb{x})$.
\end{enumerate}
\end{proposition}

\begin{proposition}
Let $\FF$ be a vector field on $\RR^3$, $\phi$ be a scalar field on $\RR^3$. Then
\begin{enumerate}[label=(\roman*)]
\item $\nabla\times\nabla\phi=\vb{0}$;
\item $\nabla\cdot(\nabla\times\vb{F})=0$.
\end{enumerate}
\end{proposition}

\begin{proof} \
\begin{enumerate}[label=(\roman*)]
\item \[\nabla\times\nabla\phi
=\begin{vmatrix}
\vb{i}&\vb{j}&\vb{k}\\
\pdv{}{x}&\pdv{}{y}&\pdv{}{z}\\
\pdv{\phi}{x}&\pdv{\phi}{y}&\pdv{\phi}{z}
\end{vmatrix}=\brac{\pdv{\phi}{y,z}-\pdv{\phi}{z,y},\pdv{\phi}{z,x}-\pdv{\phi}{x,z},\pdv{\phi}{x,y}-\pdv{\phi}{y,x}}=\vb{0}.\]

\item \begin{align*}
\nabla\cdot(\nabla\times\vb{F})
&=\nabla\cdot\brac{\pdv{F_3}{y}-\pdv{F_2}{z},\pdv{F_1}{z}-\pdv{F_3}{x},\pdv{F_2}{x}-\pdv{F_1}{y}}\\
&=\brac{\pdv{F_3}{x,y}-\pdv{F_2}{x,z}}+\brac{\pdv{F_1}{y,z}-\pdv{F_3}{y,x}}+\brac{\pdv{F_2}{z,x}-\pdv{F_1}{z,y}}\\
&=\brac{\pdv{F_1}{y,z}-\pdv{F_1}{z,y}}+\brac{\pdv{F_2}{z,x}-\pdv{F_2}{x,z}}+\brac{\pdv{F_3}{x,y}-\pdv{F_3}{y,x}}=0.
\end{align*}
\end{enumerate}
\end{proof}

\begin{proposition}[Product rules]
Let $\vb{F}$ be a vector field on $\RR^3$, $\phi$ be a scalar field on $\RR^3$. Then
\begin{enumerate}[label=(\roman*)]
\item $\nabla\cdot(\phi\vb{F})=\nabla\phi\cdot\vb{F}+\phi\nabla\cdot\vb{F}$;
\item $\nabla\times(\phi\vb{F})=\phi\nabla\times\vb{F}+\nabla\phi\times\vb{F}$.
\end{enumerate}
\end{proposition}

\begin{proposition}[Further identities]
Let $\vb{F}$ and $\vb{G}$ be vector fields on $\RR^3$. Then
\begin{enumerate}[label=(\roman*)]
\item $\nabla(\vb{F}\cdot\vb{G})=(\vb{F}\cdot\nabla)\vb{G}+(\vb{G}\cdot\nabla)\vb{F}+\vb{F}\times(\nabla\times\vb{G})+\vb{G}\times(\nabla\times\vb{F})$;
\item $\nabla\cdot(\vb{F}\times\vb{G})=\vb{G}\cdot(\nabla\times\vb{F})-\vb{F}\cdot(\nabla\times\vb{G})$;
\item $\nabla\times(\vb{F}\times\vb{G})=\vb{F}(\nabla\cdot\vb{G})-\vb{G}(\nabla\cdot\vb{F})+(\vb{G}\cdot\nabla)\vb{F}-(\vb{F}\cdot\nabla)\vb{G}$;
\item $\nabla\times(\nabla\times\vb{F})=\nabla(\nabla\cdot\vb{F})-\nabla^2\vb{F}$.
\end{enumerate}
\end{proposition}

\begin{proposition}
Consider a given vector field $\vb{F}$ for which there exists a scalar function $\phi$ such that $\vb{F}=\nabla\phi$. Then
\begin{equation}
\int_{A}^{B}\nabla\phi\cdot\dd{\vb{r}}=\phi(B)-\phi(A),
\end{equation}
where $A$ and $B$ are the start and end points, respectively, of the curve along which we are integrating.
\end{proposition}

\begin{proof}
By the chain rule, we have
\begin{align*}
\int_{A}^{B}\nabla\phi\cdot\dd{\vb{r}}
&=\int_{A}^{B}\brac{\pdv{\phi}{x},\pdv{\phi}{y},\pdv{\phi}{z}}\cdot\brac{\dv{x}{t},\dv{y}{t},\dv{z}{t}}\dd{t}\\
&=\int_{A}^{B}\brac{\pdv{\phi}{x}\dv{x}{t}+\pdv{\phi}{y}\dv{y}{t}+\pdv{\phi}{z}\dv{z}{t}}\dd{t}\\
&=\int_{A}^{B}\dv{\phi}{t}\dd{t}\\
&=\phi(B)-\phi(A).
\end{align*}
\end{proof}

\begin{remark}
Note that this result is independent of the curve itself -- the scalar line integrals for which this result holds are \emph{path-independent}.
\end{remark}

\begin{definition}[Directional derivative]
Let $\phi:\RR^3\to\RR$ be a differentiable scalar function, let $\vb{u}$ be a unit vector. Then the \vocab{directional derivative} of $\phi$ at $\vb{a}$ in the direction $\vb{u}$ is
\[\lim_{t\to0}\frac{\phi(\vb{a}+t\vb{u})-\phi(\vb{a})}{t}.\]
This is the rate of change of the function $\phi$ at $\vb{a}$ in the direction $\vb{u}$.
\end{definition}

\begin{proposition}
The directional derivative of $\phi$ at $\vb{a}$ in direction $\vb{u}$ equals $\nabla\phi(\vb{a})\cdot\vb{u}$.
\end{proposition}

\begin{proof}
Let
\begin{align*}
\Phi(t)&=\phi(\vb{a}+t\vb{u})\\
&=\phi\brac{a_1+tu_1,a_2+tu_2,a_3+tu_3}.
\end{align*}
where $\vb{a}=(a_1,a_2,a_3)$, $\vb{u}=(u_1,u_2,u_3)$. Then by definition of single variable derivative,
\[\lim_{t\to0}\frac{\phi(\vb{a}+t\vb{u})-\phi(\vb{a})}{t}=\lim_{t\to0}\frac{\Phi(t)-\Phi(0)}{t}=\Phi^\prime(0).\]
Now, by the chain rule,
\begin{align*}
\Phi^\prime(0)&=\dv{\Phi}{t}\bigg|_{t=0}\\
&=\pdv{\phi}{x}\dv{x}{t}+\pdv{\phi}{y}\dv{y}{t}+\pdv{\phi}{z}\dv{z}{t}\bigg|_{t=0}\\
&=\pdv{\phi}{x}(\vb{a})\dv{x}{t}+\pdv{\phi}{y}(\vb{a})\dv{y}{t}+\pdv{\phi}{z}(\vb{a})\dv{z}{t}\\
&=\pdv{\phi}{x}(\vb{a})u_1+\pdv{\phi}{y}(\vb{a})u_2+\pdv{\phi}{z}(\vb{a})u_3\\
&=\brac{\pdv{\phi}{x}(\vb{a}),\pdv{\phi}{y}(\vb{a}),\pdv{\phi}{z}(\vb{a})}\cdot\brac{u_1,u_2,u_3}\\
&=\nabla\phi(\vb{a})\cdot\vb{u}.
\end{align*}
\end{proof}

\begin{corollary}
The rate of change of $\phi$ is the greatest in the direction $\nabla\phi$; that is when $\vb{u}=\frac{\nabla\phi}{|\nabla\phi|}$, and the maximum rate of change is given by $|\nabla\phi|$.
\end{corollary}

\begin{definition}[Level set]
A \vocab{level set} of $\phi:\RR^3\to\RR$ is a set of points
\[\{(x,y,z)\in\RR^3\mid \phi(x,y,z)=c\}\]
for some $c\in\RR$. For suitably well behaved functions $\phi$ and constants $c$, the level set is a
surface in $\RR^3$.
\end{definition}

\begin{proposition}
Given a surface $S\subset\RR^3$ with equation $\phi(x,y,z)=c$ and a point $\vb{p}\in S$, then $\nabla\phi(\vb{p})$ is normal to $S$ at $\vb{p}$.
\end{proposition}

\begin{proof}
Let $u$ and $v$ be coordinates near $\vb{p}$, and let $\vb{r}:(u,v)\to\vb{r}(u,v)$ be a parameterisation of part of $S$. Recall that the normal to $S$ at $\vb{p}$ is in the direction
\[\pdv{\vb{r}}{u}\times\pdv{\vb{r}}{v}.\]
Note also that $\phi\brac{\phi(u,v)}=c$, and so $\pdv{\phi}{u}=\pdv{\phi}{v}=0$. If we write $\vb{r}(u,v)=\brac{x(u,v),y(u,v),z(u,v)}$ then we see that
\begin{align*}
\nabla\phi\cdot\pdv{\vb{r}}{u}
&=\brac{\pdv{\phi}{x},\pdv{\phi}{y},\pdv{\phi}{z}}\cdot\brac{\pdv{x}{u},\pdv{y}{u},\pdv{z}{u}}\\
&=\pdv{\phi}{x}\pdv{x}{u}+\pdv{\phi}{y}\pdv{y}{u}+\pdv{\phi}{z}\pdv{z}{u}\\
&=\pdv{\phi}{u}=0,
\end{align*}
where the penultimate line follows from the chain rule. Similarly,
\[\nabla\phi\cdot\pdv{\vb{r}}{v}=0,\]
and hence $\nabla\phi$ is in the direction of $\pdv{\vb{r}}{u}\times\pdv{\vb{r}}{v}$, and so is normal to the surface $S$.
\end{proof}

\subsection{Divergence and Stokes' Theorems}
\begin{theorem}[Divergence Theorem]
Let $V\subset\RR^3$ with piecewise smooth boundary $\partial V$. Let $\vb{F}$ be a differentiable vector field on $V$. Then
\begin{equation}
\iiint_V\nabla\cdot\vb{F}\dd{V}=\iint_{\partial V}\vb{F}\cdot\dd{\vb{S}},
\end{equation}
where $\dd{\vb{S}}$ is oriented in the direction of the outward pointing normal from $V$.
\end{theorem}
\pagebreak

\section*{Exercises}
\begin{prbm}
Calculate the double integral
\[\int_{-\infty}^{\infty}\int_{-\infty}^{\infty}e^{-x^2-y^2}\dd{x}\dd{y}\]
via a change to polar coordinates. Hence deduce that
\[\int_{-\infty}^{\infty}e^{-s^2}\dd{s}=\sqrt{\pi}.\]
\end{prbm}

\begin{solution}
On changing to polar coordinates,
\begin{align*}
\int_{-\infty}^{\infty}\int_{-\infty}^{\infty}e^{-x^2-y^2}\dd{x}\dd{y}
&=\int_{0}^{\infty}\int_{0}^{2\pi}e^{-r^2}\dd{\theta}\dd{r}\\
&=2\pi\int_{0}^{\infty}e^{-r^2}r\dd{r}\\
&=-\pi\sqbrac{e^{-r^2}}_{0}^{\pi}=\pi.
\end{align*}
Hence we can write
\begin{align*}
\pi&=\int_{-\infty}^{\infty}\int_{-\infty}^{\infty}e^{-x^2-y^2}\dd{x}\dd{y}\\
&=\int_{-\infty}^{\infty}\int_{-\infty}^{\infty}e^{-x^2}e^{-y^2}\dd{x}\dd{y}\\
&=\brac{\int_{-\infty}^{\infty}e^{-x^2}\dd{x}}\brac{\int_{-\infty}^{\infty}e^{-y^2}\dd{y}}\\
&=\brac{\int_{-\infty}^{\infty}e^{-x^2}\dd{x}}^2,
\end{align*}
from which the final result follows. Note that we have used the fact that the double integral is \emph{separable}.
\end{solution}
\chapter{Logic}
Mathematical logic is about providing a uniform, unambiguous language for mathematics, making precise what a proof is, explaining and guaranteeing exactness, rigour and certainty in mathematics, as well as establishing the foundations of mathematics.

\section{Propositional Calculus}
This section is basically a recap of \cref{chap:logic-proofs}.

The alphabet of propositional calculus consists of the following (abstract) symbols:
\begin{itemize}
\item the \vocab{propositional variables} $p_0,p_1,\dots,p_n,\dots$
\item \vocab{negation} $\lnot$ - the unary connective not
\item four binary connectives $\rightarrow,\land,\lor,\leftrightarrow$ which mean implies, and, or, if and only if respectively
\item two punctuation marks $($ and $)$ which are left parenthesis and right parenthesis respectively.
\end{itemize}

This alphabet is denoted by $\mathcal{L}$.

\begin{notation}
Note also that we use $\rightarrow$, and not $\implies$.
\end{notation}

\begin{definition}[String]
A \vocab{string} (from $\mathcal{L}$) is any finite sequence of symbols from $\mathcal{L}$.
\end{definition}

\begin{example} \
\begin{itemize}
\item $\rightarrow p_{17}()$
\item $((p_0\land p_1)\rightarrow\lnot p_2)$
\item $))\lnot)p_{32}$
\end{itemize}
\end{example}

The \vocab{length} of a string is the number of symbols in it. So the strings in the examples have length 4, 10, 5 respectively. (A propositional variable has length 1.)

We now single out from all strings those which make grammatical sense (formulas).

\begin{definition}[Formula]
The notion of a \vocab{formula} of $\mathcal{L}$ is defined (recursively) by the following rules:
\begin{enumerate}[label=(\roman*)]
\item every propositional variable is a formula
\item if the string $A$ is a formula then so is $\lnot A$
\item if the strings $A$ and $B$ are both formulas then so are the strings
\begin{align*}
(A\rightarrow B)&\text{ read $A$ implies $B$}\\
(A\land B)&\text{ read $A$ and $B$}\\
(A\lor B)&\text{ read $A$ or $B$}\\
(A\leftrightarrow B)&\text{ read $A$ if and only if $B$}
\end{align*}
\item Nothing else is a formula, i.e. a string $\phi$ is a formula if and only if $\phi$ can be obtained from propositional variables by finitely many applications of the formation rules (ii) and (iii)
\end{enumerate}
\end{definition}
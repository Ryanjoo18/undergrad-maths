\chapter{Set Theory}\label{chap:set-theory}

\begin{learn}
\item recap basic definitions relating to sets (excluding detailed axiomatic discussions);
\item define relations and related concepts including binary relation, partial order, total order, well order, equivalence relations, equivalence relations, equivalence class, quotient set, partition of a set;
\item define functions, injectivity, surjectivity, bijectivity, composition, invertibility, monotonicity;
\end{learn}

\section{Basics}
\subsection{Definitions and Notations}
A \vocab{set} $S$ can be loosely defined as a collection of objects\footnote{\emph{Russell's paradox}, after the mathematician and philosopher Bertrand Russell (1872--1970), provides a warning as to the looseness of our definition of a set. Suppose $H$ is the collection of sets that are not elements of themselves; that is,
\[H=\{S\mid S\notin S\}.\]

The problem arises when we ask the question of whether or not $H$ is itself in $H$? On one hand, if $H\notin H$ then $H$ meets the precise criterion for being in $H$ and so $H\in H$, a contradiction. On the other hand, if $H\in H$ then by the property required for this to be the case, $H\notin H$, another contradiction. Thus we have a paradox: $H$ is neither in $H$, nor not in $H$.

The modern resolution of Russell's paradox is that we have taken too naive an understanding of ``collection'', and that Russell's ``set'' $H$ is in fact not a set. It does not fit within axiomatic set theory (which relies on the so-called ZF axioms), and so the question of whether or not $H$ is in $H$ simply doesn't make sense.}. For a set $S$, we write $x \in S$ to mean that $x$ is an \vocab{element} of $S$, and $x \notin S$ if otherwise. 

To describe a set, one can list its elements explicitly. A set can also be defined in terms of some property $P(x)$ that the elements $x \in S$ satisfy, denoted by the following set builder notation:
\[\{x\in S \mid P(x)\}\]

Some basic sets (of numbers) you should be familiar with:
\begin{itemize}
\item $\NN=\{1,2,3,\dots\}$ denotes the natural numbers (non-negative integers).
\item $\ZZ=\{\dots,-2,-1,0,1,2,\dots\}$ denotes the integers.
\item $\QQ=\{\frac{p}{q} \mid p,q\in\ZZ, q\neq0\}$ denotes the rational numbers.
\item $\RR$ denotes the real numbers (the construction of which using Dedekind cuts will be discussed in \cref{chap:number-systems}).
\item $\CC=\{x+yi \mid x,y\in\RR\}$ denotes the complex numbers.
\end{itemize}

We have that
\[\NN\subset\ZZ\subset\QQ\subset\RR\subset\CC.\]

The \vocab{empty set} is the set with no elements, denoted by $\emptyset$.

$A$ is a \vocab{subset} of $B$ if every element of $A$ is in $B$, denoted by $A\subset B$:
\[A\subset B\iff(\forall x)(x\in A \implies x\in B)\]
We denote $A\subsetneq B$ to explicitly mean that $A\subset B$ and $A\neq B$; we call $A$ a \vocab{proper subset} of $B$.

\begin{proposition}[$\subset$ is transitive]
If $A \subset B$ and $B \subset C$, then $A \subset C$.
\end{proposition}

\begin{proof}
For all $x\in A$, $x\in B$
Let $x\in A$. 
Since $A \subset B$ and $x\in A$, $x\in B$. 
Since $B \subset C$ and $x\in B$, $x\in C$. 
Hence $A \subset C$.
\end{proof}

$A$ and $B$ are \vocab{equal} if and only if they contain the same elements, denoted by $A=B$. 

\begin{proposition}[Double inclusion]
Let $A\subset S$ and $B\subset S$. Then
\[A=B\iff (A\subset B)\land(B\subset A)\]
\end{proposition}

\begin{proof}
We have 
\begin{align*}
A = B &\iff (\forall x)[x \in A \iff x \in B] \\
&\iff (\forall x)[(x \in A \implies x \in B) \land (x \in B \implies x \in A)] \\
&\iff \{(\forall x)[x \in A \implies x \in B]\} \land {(\forall x)[x \in B \implies x \in A)]} \\
&\iff (A \subset B) \land (B \subset A)
\end{align*}
\end{proof}

\begin{remark}
Double inclusion is a useful tool to prove that two sets are equal.
\end{remark}

Some frequently occurring subsets of $\RR$ are known as \vocab{intervals}, which can be visualised as sections of the real line. We define \emph{bounded intervals}
\begin{align*}
(a,b)&=\{x\in\RR\mid a<x<b\},\\
[a,b]&=\{x\in\RR\mid a\le x\le b\},\\
[a,b)&=\{x\in\RR\mid a\le x<b\},\\
(a,b]&=\{x\in\RR\mid a<x\le b\},
\end{align*}
and \emph{unbounded intervals}
\begin{align*}
(a,\infty)&=\{x\in\RR\mid a<x\},\\
[a,\infty)&=\{x\in\RR\mid a\le x\},\\
(-\infty,a)&=\{x\in\RR\mid x<a\},\\
(\infty,a]&=\{x\in\RR\mid x\le a\}.
\end{align*}
An interval of the first type $(a,b)$ is called an \emph{open interval}; an interval of the second type $[a,b]$ is called a \emph{closed interval}. Note that if $a=b$, then $[a,b]=\{a\}$, while $(a,b)=[a,b)=(a,b]=\emptyset$.

The \vocab{power set} $\mathcal{P}(A)$ of $A$ is the set of all subsets of $A$ (including the set itself and the empty set):
\[\mathcal{P}(A)=\{S\mid S\subset A\}.\]

An \vocab{ordered pair} is denoted by $(a,b)$, where the order of the elements matters. Two pairs $(a_1,b_1)$ and $(a_2,b_2)$ are equal if and only if $a_1=a_2$ and $b_1=b_2$.  Similarly, we have ordered triples $(a,b,c)$, quadruples $(a,b,c,d)$ and so on. If there are $n$ elements it is called an \vocab{$n$-tuple}.

The \vocab{Cartesian product} of sets $A$ and $B$, denoted by $A \times B$, is the set of all ordered pairs with the first element of the pair coming from $A$ and the second from $B$:
\[A\times B\coloneqq\{(a,b)\mid a\in A,b\in B\}.\]
More generally, we define $A_1 \times A_2 \times \cdots \times A_n$ to be the set of all ordered $n$-tuples $(a_1, a_2, \dots, a_n)$, where $a_i \in A_i$ for $1 \le i \le n$. If all the $A_i$ are the same, we write the product as $A^n$.

\begin{example}
$\RR^2$ is the Euclidean plane, $\RR^3$ is the Euclidean space, and $\RR^n$ is the $n$-dimensional Euclidean space.
\begin{align*}
\RR \times \RR = \RR^2 &= \{(x,y) \mid x,y \in \RR\} \\
\RR \times \RR \times \RR = \RR^3 &= \{(x,y,z) \mid x,y,z \in \RR\} \\
\RR^n &= \{(x_1,x_2,\dots,x_n) \mid x_1,x_2,\dots,x_n \in \RR\}
\end{align*}
\end{example}

To generalise the notion of intervals, we define a \emph{$k$-cell} as
\[\{(x_1,\dots,x_n)\in\RR^k\mid a_i\le x_i\le b_i (1\le i\le k)\}.\]
For example, a $1$-cell is an interval, a $2$-cell is a rectangle, and a $3$-cell is a rectangular solid. In this regard, we can think of a $k$-cell as a higher-dimensional version of a rectangle or rectangular solid; it is the Cartesian product of $k$ closed intervals in $\RR$.

\subsection{Algebra of Sets}
We now disuss the algebra of sets. Given $A \subset S$ and $B \subset S$,
\begin{enumerate}[label=(\roman*)]
\item The \vocab{union} $A \cup B$ is the set consisting of elements that are in $A$ or $B$ (or both):
\[ A\cup B=\{x \in S \mid x\in A \lor x\in B\} \]

\item The \vocab{intersection} $A \cap B$ is the set consisting of elements that are in both $A$ and $B$:
\[ A\cap B=\{x \in S \mid x\in A \land x\in B\} \]

$A$ and $B$ are \vocab{disjoint} if both sets have no element in common:
\[ A\cap B = \emptyset \]
\end{enumerate}

More generally, we can take unions and intersections of arbitrary numbers of sets (could be finitely or infinitely many). Given a family of subsets $\{A_i\mid i\in I\}$ where $I$ is an \emph{indexing set}, we write
\[\bigcup_{i\in I}A_i=\{x \mid \exists i\in I, x\in A_i\},\]
and
\[\bigcap_{i\in I}A_i=\{x \mid \forall i\in I, x\in A_i\}.\]

\begin{enumerate}[resume*]
\item The \vocab{complement} of $A$, denoted by $A^c$, is the set containing elements that are not in A:
\[ A^c = \{x \in S \mid x \notin A\} \]

\item The \vocab{set difference}, or complement of $B$ in $A$, denoted by $A\setminus B$, is the subset consisting of those elements that are in $A$ and not in $B$:
\[ A\setminus B = \{x \in A \mid x \notin B\} \]
Note that $A\setminus B = A \cap B^c$.
\end{enumerate}

\begin{proposition}[Distributive laws]
Let $A,B,C\subset S$. Then
\begin{enumerate}[label=(\roman*)]
\item $A\cup(B\cap C)=(A\cup B)\cap(A\cup C)$;
\item $A\cap(B\cup C)=(A\cap B)\cup(A\cap C)$.
\end{enumerate}
\end{proposition}

\begin{proof} \
\begin{enumerate}[label=(\roman*)]
\item Suppose $x\in A\cup(B \cap C)$. Then
\begin{align*}
x\in A\cup(B \cap C)
&\iff x\in A\quad\lor\quad x\in B\cap C\\
&\iff x\in A\quad\lor\quad (x\in B)\land (x\in C)\\
&\iff (x\in A)\lor (x\in B)\quad\land\quad (x\in A)\lor (x\in C)\\
&\iff x\in A\cup B\quad\land\quad x\in A\cup C\\
&\iff x\in (A\cup B)\cap(A\cup C).
\end{align*}
Thus $A \cup (B \cap C) \subset (A \cup B) \cap (A \cup C)$.

Conversely suppose that $x \in (A \cup B) \cap (A \cup C)$. Then go in the reverse direction of the above steps to show that $(A\cup B)\cap (A\cup C)\subset A\cup(B\cap C)$.

By double inclusion, $(A\cup B)\cap(A\cup C)=A\cup(B\cap C)$.

\item Similar.
\end{enumerate}
\end{proof}

\begin{proposition}[de Morgan's laws]
Let $A,B\subset S$. Then
\begin{enumerate}[label=(\roman*)]
\item $(A \cup B)^c = A^c \cap B^c$;
\item $(A \cap B)^c = A^c \cup B^c$.
\end{enumerate}
\end{proposition}

\begin{proof} \
\begin{enumerate}[label=(\roman*)]
\item \begin{align*}
x\in(A\cup B)^c&\iff x\notin A\cup B\\
&\iff x\notin A\quad\land\quad x\notin B\\
&\iff x\in A^c\quad\land\quad x\in B^c\\
&\iff x\in A^c\cap B^c
\end{align*}

\item Similar.
\end{enumerate}
\end{proof}

De Morgan's laws extend naturally to any number of sets, so if $\{A_i \mid i \in I\}$ is a family of subsets of $S$, then
\begin{align*}
\brac{\bigcap_{i\in I}A_i}^c&=\bigcup_{i\in I}{A_i}^c,\\
\brac{\bigcup_{i\in I}A_i}^c&=\bigcap_{i\in I}{A_i}^c
\end{align*}

\begin{exercise}
Prove the following:
\begin{enumerate}[label=(\roman*)]
\item $\brac{\bigcup_{i\in I}A_i}\cup B=\bigcup_{i\in I}(A_i\cup B)$
\item $\brac{\bigcap_{i\in I}A_i}\cup B=\bigcap_{i\in I}(A_i\cup B)$
\item $\brac{\bigcup_{i\in I}A_i}\cup\brac{\bigcup_{j\in J}B_j}=\bigcup_{(i,j)\in I\times J}(A_i\cup B_j)$
\item $\brac{\bigcap_{i\in I}A_i}\cup\brac{\bigcap_{j\in J}B_j}=\bigcap_{(i,j)\in I\times J}(A_i\cup B_j)$
\end{enumerate}
\end{exercise}

\begin{exercise}
Let $S\subset A\times B$. Express the set $A_S$ of all elements of $A$ which appear as the first entry in at least one of the elements in $S$.

($A_S$ here may be called the projection of $S$ onto $A$.)
\end{exercise}
\pagebreak

\section{Relations}
\subsection{Definition and Examples}
\begin{definition}[Relation]
$R$ is a \vocab{relation}\index{relation} between $A$ and $B$ if $R\subset A\times B$; $a\in A$ and $b\in B$ are said to be \emph{related} if $(a,b)\in R$, denoted $a R b$.
\end{definition}

\begin{remark}
A relation is a set of ordered pairs.
\end{remark}

Visually speaking, a relation is uniquely determined by a simple bipartite graph over $A$ and $B$. On the bipartite graph, this is usually represented by an edge between $a$ and $b$.

\begin{example}
In many cases we do not actually use $R$ to write the relation because there is some other conventional notation:
\begin{itemize}
\item The ``less than or equal to'' relation $\le$ on the set of real numbers is
\[\{(x,y) \in \RR^2 \mid x \le y\}\subset\RR^2;\]
we write $x\le y$ if $(x,y)$ is in this set.
\item The ``divides'' relation $\mid$ on $\NN$ is
\[\{(m,n)\in\NN^2\mid m\text{ divides }n\}\subset\NN^2;\]
we write $m\mid n$ if $(m,n)$ is in this set.
\item For a set $S$, the ``subset'' relation $\subset$ on $\mathcal{P}(S)$ is
\[\{(A,B)\in\mathcal{P}(S)^2\mid A\subset B\}\subset\mathcal{P}(S)^2;\]
we write $A\subset B$ if $(A,B)$ is in this set.
\end{itemize}
\end{example}

If $A \times B$ is the smallest Cartesian product of which $R$ is a subset, we call $A$ and $B$ the \emph{domain} and \emph{range} of $R$ respectively, denoted by $\dom R$ and $\ran R$ respectively.

\begin{example}
Given $R=\{(1,a),(1,b),(2,b),(3,b)\}$, then $\dom R=\{1,2,3\}$ and $\ran R=\{a,b\}$.
\end{example}

\begin{definition}[Binary relation]
A \vocab{binary relation}\index{relation!binary relation} in $A$ is a relation between $A$ and itself; that is, $R \subset A \times A$.
\end{definition}

\subsection{Properties of Relations}
Let $A$ be a set, $R$ a relation on $A$, $x,y,z \in A$. We say that
\begin{enumerate}[label=(\roman*)]
\item $R$ is \vocab{reflexive} if $xRx$ for all $x\in A$;
\item $R$ is \vocab{symmetric} if $xRy \implies yRx$;
\item $R$ is \vocab{anti-symmetric} if $xRy \text{ and } yRx \implies x=y$;
\item $R$ is \vocab{transitive} if $xRy \text{ and } yRz \implies xRz$.
\end{enumerate}

\begin{example}[Less than or equal to]
The relation $\le$ on $R$ is reflexive, anti-symmetric, and transitive, but not symmetric. 
\end{example}

\begin{definition}
A \vocab{partial order}\index{relation!partial order} on a non-empty set $A$ is a relation on $A$ satisfying reflexivity, anti-symmetry and transitivity.

A \vocab{total order}\index{relation!total order} on $A$ is a partial order on $A$ such that if for every $x,y\in A$, either $xRy$ or $yRx$.

A \vocab{well order}\index{relation!well order} on $A$ is a total order on $A$ such that every non-empty subset of $A$ has a minimal element; that is, for each non-empty $B\subset A$ there exists $s\in B$ such that $s\le b$ for all $b\in B$.
\end{definition}

\begin{example}
You should verify the following:
\begin{itemize}
\item Less than: the relation $<$ on $R$ is not reflexive, symmetric, or anti-symmetric, but it is transitive.
\item Not equal to: the relation $\neq$ on $R$ is not reflexive, anti-symmetric or transitive, but it is symmetric.
\end{itemize}
\end{example}

\begin{definition}
Let the non-empty set $A$ be partially ordered by $\le$.
\begin{itemize}
\item A subset $B\subset A$ is called a \vocab{chain} if for all $x,y\in B$, either $x\le y$ or $y\le x$.
\item An \vocab{upper bound} for a subset $B\subset A$ is an element $u\in A$ such that $b\le u$ for all $b\in B$.
\item A \vocab{maximal element} of $A$ is an element $m\in A$ such that $m\le x$ for any $x\in A$, then $m=x$.
\end{itemize}
\end{definition}

\begin{lemma}[Zorn's lemma]
If $A$ is a non-empty partially ordered set in which every chain has an upper bound, then $A$ has a maximal element.
\end{lemma}

It is a non-trivial result that Zorn's lemma is independent of the usual (Zermelo--Fraenkel) axioms of set theory in the sense that if the axioms of set theory are consistent, then so are these axioms together with Zorn's lemma; and if the axioms of set theory are consistent, then so are these axioms together with the negation of Zorn's lemma.

\begin{lemma}[Axiom of choice]
The Cartesian product of any non-empty collection of non-empty sets is non-empty. In other words, if $I$ is any non-empty (indexing) set and $A_i$ is a non-empty set for all $i\in I$, then there exists a choice function from $I$ to $\bigcup_{i\in I}A_i$.
\end{lemma}

\begin{lemma}[Well-ordering principle]
Every non-empty set $A$ has a well-ordering.
\end{lemma}

\begin{theorem}
Assuming the usual (Zermelo--Fraenkel) axioms of set theory, the following are equivalent:
\begin{enumerate}[label=(\roman*)]
\item Zorn's lemma
\item Axiom of choice
\item Well-ordering principle
\end{enumerate}
\end{theorem}

\begin{proof}
This follows from elementary set theory. We refer the reader to \textit{Real and Abstract Analysis} by Hewitt and Stromberg, Section 3.
\end{proof}

\subsection{Equivalence Relations}
One important type of relation is an equivalence relation. An equivalence relation is a way of saying two objects are, in some particular sense, ``the same''.

\begin{definition}[Equivalence relation]
A relation $R$ on a set $A$ is an \vocab{equivalence relation}\index{relation!equivalence relation} if it is reflexive, symmetric and transitive.
\end{definition}

\begin{notation}
We use $\sim$ to denote the equivalence relation $R$ in $A\times A$: we denote $a\sim b$ for $(a,b)\in R$.
\end{notation}

An equivalence relation provides a way of grouping together elements that can be viewed as being the same:

\begin{definition}[Equivalence class]
Given an equivalence relation $\sim$ on a set $A$, and given $x \in A$, the \vocab{equivalence class}\index{equivalence class} of $x$ is
\[[x]\coloneqq\{y\in A\mid y\sim x\}.\]
\end{definition}

\begin{proposition}
Every two equivalence classes are disjoint; thus the union of equivalence classes form the entire set.
\end{proposition}

\begin{remark}
From the point of view of the elements, every element belongs to one and only one equivalence class.
\end{remark}

\begin{definition}[Quotient set]
The \vocab{quotient set}\index{quotient set} is the set of all equivalence classes, denoted by $A/\sim$.
\end{definition}

Grouping the elements of a set into equivalence classes provides a partition of the set, which we define as follows:

\begin{definition}[Partition]
A \vocab{partition}\index{partition} of a set $A$ is a collection of subsets $\{A_i\subset A\mid i\in I\}$, where $I$ is an indexing set, with the property that
\begin{enumerate}[label=(\roman*)]
\item $A_i\neq\emptyset$ for all $i\in I$ (all the subsets are non-empty)
\item $\bigcup_{i\in I}A_i=A$ (every member of $A$ lies in one of the subsets)
\item $A_i\cap A_j=\emptyset$ for every $i\neq j$ (the subsets are disjoint)
\end{enumerate}
The subsets are called the \emph{parts} of the partition.
\end{definition}

\begin{example}[Modular arithmetic]
Let $n$ be a fixed positive integer. Define a relation on $\ZZ$ by
\[a\sim b\iff n\mid(b-a).\]
\begin{proposition}
$a\sim b$ is a equivalence relation.
\end{proposition}
\begin{proof} \
\begin{enumerate}[label=(\roman*)]
\item $a\sim a$ so $\sim$ is reflexive.
\item $a\sim b\implies b\sim a$ for any integers $a$ and $b$, so $\sim$ is symmetric.
\item If $a\sim b$ and $b\sim c$ then $n\mid(a-b)$ and $n\mid(b-c)$, so $n\mid(a-b)+(b-c)=(a-c)$, so $a\sim c$ and $\sim$ is transitive.
\end{enumerate}
\end{proof}

\begin{notation}
We write $a\equiv b\pmod n$ if $a\sim b$.
\end{notation}

\begin{notation}
For any $k\in\ZZ$ we denote the equivalence class of $a$ by $[a]$, called the \emph{congruence class} (or \emph{residue class}) of $a$ mod $n$, which consists of the integers which differ from $a$ by an integral multiple of $n$; that is,
\[[a]=\{a+kn\mid k\in\ZZ\}.\]
\end{notation}

There are precisely $n$ distinct congruence classes mod $n$, namely
\[[0],[1],\dots,[n-1],\]
determined by the possible remainders after division by $n$; and these residue classes partition the integers $\ZZ$. The set of equivalence classes under this equivalence relation is denoted by $\ZZ/n\ZZ$, and called the \emph{integers modulo $n$}.

Define addition and multiplication on $\ZZ/n\ZZ$ as follows: for $[a],[b]\in\ZZ/n\ZZ$,
\begin{align*}
[a]+[b]&=[a+b]\\
[a][b]&=[ab].
\end{align*}

This means that to compute the sum / product of two elements $[a],[b]\in\ZZ/n\ZZ$, take any \emph{representative} $a\in[a]$, $b\in[b]$, and add / multiply integers $a$ and $b$ as usual in $\ZZ$, then take the congruence class containing the result.

\begin{proposition}
Addition and mulltiplication on $\ZZ/n\ZZ$ are well-defined; that is, they do not depend on the choices of representatives for the classes involved. More precisely, if $a_1,a_2\in\ZZ$ and $b_1,b_2\in\ZZ$ with $\overline{a_1}=\overline{b_1}$ and $\overline{a_2}=\overline{b_2}$, then $\overline{a_1+a_2}=\overline{b_1+b_2}$ and $\overline{a_1a_2}=\overline{b_1b_2}$, i.e., If
\[a_1\equiv b_1\pmod n,\quad a_2\equiv b_2\pmod n\]
then
\[a_1+a_2\equiv b_1+b_2\pmod n,\quad a_1a_2\equiv b_1b_2\pmod n.\]
\end{proposition}

\begin{proof}
Suppose $a_1\equiv b_1\pmod n$, i.e., $n\mid(a_1-b_1)$. Then $a_1=b_1+sn$ for some integer $s$. Similarly, $a_2\equiv b_2\pmod n$ means $a_2=b_2+tn$ for some integer $t$.

Then $a_1+a_2=(b_1+b_2)+(s+t)n$ so that $a_1+a_2\equiv b_1+b_2\pmod n$, which shows that the sum of the residue classes is independent of the representatives chosen.

Similarly, $a_1a_2=(b_1+sn)(b_2+tn)=b_1b_2+(b_1t+b_2s+stn)n$ shows that $a_1a_2\equiv b_1b_2\pmod n$ and so the product of the residue classes is also independent of the representatives chosen.
\end{proof}

An important subset of $\ZZ/n\ZZ$ consists of the collection of congruence classes which have a multiplicative inverse in $\ZZ/n\ZZ$:
\[(\ZZ/n\ZZ)^\times\coloneqq\{[a]\in\ZZ/n\ZZ\mid\exists[c]\in\ZZ/n\ZZ,[a][c]=[1]\}.\]

\begin{proposition}
$(\ZZ/n\ZZ)^\times$ is also the collection of congruence classes whose representatives are relatively prime to $n$:
\[(\ZZ/n\ZZ)^\times=\{[a]\in\ZZ/n\ZZ\mid(a,n)=1\}.\]
\end{proposition}
\end{example}
\pagebreak

\section{Functions}
\subsection{Definitions and Examples}
\begin{definition}[Function]
A \vocab{function}\index{function} $f:X\to Y$ is a mapping of every element of $X$ to some element of $Y$; $X$ and $Y$ are known as the \emph{domain} and \emph{codomain} of $f$ respectively.
\end{definition}

\begin{remark}
The definition requires that a unique element of the codomain is assigned for every element of the domain. For example, for a function $f:\RR \to \RR$, the assignment $f(x)=\frac{1}{x}$ is not sufficient as it fails at $x=0$. Similarly, $f(x)=y$ where $y^2=x$ fails because $f(x)$ is undefined for $x<0$, and for $x>0$ it does not return a unique value; in such cases, we say the the function is \emph{ill-defined}. We are interested in the opposite; functions that are \emph{well-defined}.
\end{remark}

\begin{definition}
Given a function $f:X \to Y$, the \vocab{image}\index{function!image} (or \emph{range}) of $f$ is
\[f(X)\coloneqq\{f(x)\mid x\in X\}\subset Y.\]
More generally, given $A \subset X$, the image of $A$ under $f$ is
\[f(A)\coloneqq\{f(x)\mid x\in A\}\subset Y.\]
Given $B \subset Y$, the \vocab{pre-image}\index{function!pre-image} of $B$ under $f$ is
\[f^{-1}(B)\coloneqq\{x\mid f(x)\in B\}\subset X.\]
\end{definition}

\begin{remark}
Beware the potentially confusing notation: for $x \in X$, $f(x)$ is a single element of $Y$, but for $A \subset X$, $f(A)$ is a set (a subset of $Y$). Note also that $f^{-1}(B)$ should be read as ``the pre-image of $B$'' and not as ``$f$-inverse of $B$''; the pre-image is defined even if no inverse function exists (in which case $f^{-1}$ on its own has no meaning; we discuss invertibility of a function below).
\end{remark}

\begin{exercise}
Prove the following statements:
\begin{enumerate}[label=(\roman*)]
\item $f(A\cup B)=f(A)\cup f(B)$
\item $f(A_1\cup\cdots\cup A_n)=f(A_1)\cup\cdots\cup f(A_n)$
\item $f(\bigcup_{\lambda\in A}A_\lambda)=\bigcup_{\lambda\in A}f(A_\lambda)$
\item $f(A\cap B)\subset f(A)\cap f(B)$
\item $f^{-1}(f(A))\supset A$
\item $f(f^{-1}(A))\subset A$
\item $f^{-1}(A\cup B)=f^{-1}(A)\cup f^{-1}(B)$
\item $f^{-1}(A\cap B)=f^{-1}(A)\cap f^{-1}(B)$
\item $f^{-1}(A_1\cup\cdots\cup A_n)=f^{-1}(A_1)\cup\cdots\cup f^{-1}(A_n)$
\item $f^{-1}(\bigcup_{\lambda\in A}A_\lambda)=\bigcup_{\lambda\in A}f^{-1}(A_\lambda)$
\end{enumerate}
\end{exercise}

If a function is defined on some larger domain than we care about, it may be helpful to restrict the domain:

\begin{definition}[Restriction]
Given a function $f:X \to Y$ and a subset $A \subset X$, the \vocab{restriction}\index{function!restriction} of $f$ to $A$ is the map $f|_A:A \to Y$.
\end{definition}

\begin{remark}
The restriction is almost the same function as the original $f$ -- just the domain has changed.
\end{remark}

Another rather trivial but nevertheless important function is the identity map:

\begin{definition}[Identity map]
Given a set $X$, the \vocab{identity} $\id_X:X \to X$ is defined by
\[\id_X(x)=x\quad(\forall x\in X)\]
\end{definition}

\begin{notation}
If the domain is unambiguous, the subscript may be omitted.
\end{notation}

\subsection{Injectivity, Surjectivity, Bijectivity}
\begin{definition}
Let $f:X\to Y$ be a function.
\begin{enumerate}[label=(\roman*)]
\item $f$ is \vocab{injective}\index{function!injectivity} if each element of $Y$ has at most one element of $X$ that maps to it:
\[\forall x_1,x_2\in X,\quad f(x_1)=f(x_2) \implies x_1=x_2\]

\item $f$ is \vocab{surjective}\index{function!surjectivity} if every element of $Y$ is mapped to at least one element of $X$:
\[ \forall y\in Y,\quad\exists x\in X,\quad f(x)=y \]

\item $f$ is \vocab{bijective}\index{function!bijectivity} if it is both injective and surjective; a bijective function is termed a \emph{bijection}.
\end{enumerate}
\end{definition}

\begin{notation}
Given two sets $X$ and $Y$, write $X\sim Y$ to denote the existence of a bijection from $X$ to $Y$.
\end{notation}

\begin{proposition}[$\sim$ is transitive]
If $X\sim Y$ and $Y\sim Z$, then $X\sim Z$.
\end{proposition}

\begin{theorem}[Cantor--Schroder--Bernstein]
If $f:A\to B$ and $g:B\to A$ are both injections, then $A\sim B$.
\end{theorem}

\begin{proof}
%https://web.williams.edu/Mathematics/lg5/CanBer.pdf
\end{proof}

\subsection{Composition}
\begin{definition}[Composition]
Given $f:X\to Y$ and $g:Y\to Z$, the \vocab{composition} $g\circ f:X\to Z$ is defined by
\[ (g \circ f)(x)=g(f(x))\quad(\forall x \in X)\]
\end{definition}

The composition of functions is not commutative. However, composition is associative, as the following results shows:

\begin{proposition}[Associativity of composition]
Let $f:X\to Y$, $g:Y\to Z$, $h:Z\to W$. Then
\[f\circ (g\circ h)=(f\circ g)\circ h.\]
\end{proposition}

\begin{proof}
Let $x \in X$. By the definition of composition, we have
\[ (f \circ (g \circ h))(x) = f((g \circ h)(x)) = f(g(h(x))) = (f \circ g)(h(x)) = ((f \circ g) \circ h)(x). \]
\end{proof}

\begin{proposition}[Composition preserves injectivity and surjectivity] \
\begin{enumerate}[label=(\roman*)]
\item If $f:X \to Y$ is injective and $g:Y \to Z$ is injective, then $g \circ f:X \to Z$ is injective.
\item If $f:X\to Y$ is surjective and $g:Y\to Z$ is surjective, then $g \circ f:X\to Z$ is surjective.
\end{enumerate}
\end{proposition}

\begin{proof} \
\begin{enumerate}[label=(\roman*)]
\item Let $f:X \to Y$ and $g:Y \to Z$ be injective. To prove that $g \circ f:X\to Z$ is injective, we need to prove: for all $x,x^\prime\in X$, 
\[(g \circ f)(x)=(g\circ f)(x^\prime) \implies x=x^\prime.\]

Suppose that $(g \circ f)(x) = (g \circ f)(x^\prime)$. Then by definition
\[g\brac{f(x)}=g\brac{f(x^\prime)}.\]
Injectivity of $g$ implies
\[f(x)=f(x^\prime),\]
and injectivity of $f$ implies
\[x=x^\prime.\]

\item Let $f:X\to Y$ and $g:Y\to Z$ be surjective. To prove that $g\circ f:X\to Z$ is surjective, we need to prove that for any $z\in Z$, there exists $x\in X$ such that $(g\circ f)(x)=z$.

Let $z\in Z$. By surjectivity of $g:Y\to Z$, there exists $y\in Y$ such that $g(y)=z$. By surjectivity of $f:X\to Y$, there exists $x\in X$ such that $f(x)=y$. This means that there exists $x\in X$ such that $g\brac{f(x)}=g(y)=z$, as desired.
\end{enumerate}
\end{proof}

\begin{proposition}
$f:X\to Y$ is injective if and only if for any set $Z$ and any functions $g_1,g_2:Z\to X$,
\[f\circ g_1=f\circ g_2 \implies g_1=g_2.\]
\end{proposition}

\begin{proof} \

\fbox{$\implies$} Suppose $f$ is injective, and suppose $f\circ g_1=f\circ g_2$. Let $z\in Z$. Then we have
\[f\brac{g_1(z)}=f\brac{g_2(z)}.\]
Injectivity of $f$ implies
\[g_1(z)=g_2(z),\]
so $g_1=g_2$ (since the choice of $z\in Z$ is arbitrary).

\fbox{$\impliedby$} Pick $Z=\{1\}$, basically some random one-element set. Then for $x,y\in X$, define
\begin{align*}
g_1:Z\to X,&\quad g_1(1)=x,\\
g_2:Z\to Y,&\quad g_2(1)=y.
\end{align*}
Then for $x,y\in X$,
\[ f(x)=f(y) \implies f(g_1(1))=f(g_2(1)) \implies g_1(1)=g_2(1) \implies x=y \]
which shows that $f$ is injective.
\end{proof}

\begin{proposition}
$f:X\to Y$ is surjective if and only if for any set $Z$ and any functions $g_1,g_2:Y\to Z$,
\[g_1 \circ f=g_2 \circ f \implies g_1=g_2.\]
\end{proposition}

\begin{proof} \

\fbox{$\implies$} Suppose that $f$ is surjective. Let $y\in Y$. Surjectivity of $f$ means there exists $x\in X$ such that $f(x)=y$. Then
\[g_1\circ f=g_2 \circ f\implies g_1\brac{f(x)}=g_2\brac{f(x)}\implies g_1(y)=g_2(y) \]
so $g_1=g_2$.

\fbox{$\impliedby$} We prove the contrapositive. Suppose $f$ is not surjective, then there exists $y \in Y$ such that for all $x \in X$ we have $f(x)\neq y$. We then aim to construct set $Z$ and $g_1,g_2:Y\to Z$ such that
\begin{enumerate}[label=(\roman*)]
\item $g_1(y) \neq g_2(y)$
\item $\forall y^\prime \neq y, g_1(y^\prime)=g_2(y^\prime)$
\end{enumerate}

Because if this is satisfied, then $\forall x \in X$, since $f(x)\neq y$ we have from (ii) that $g_1(f(x))=g_2(f(x))$; thus $g_1 \circ f=g_2 \circ f$, and yet from (i) we have $g_1 \neq g_2$.

We construct $Z=Y\cup\{1,2\}$ for some random $1,2 \notin Y$.

Then we define
\begin{align*}
&g_1:Y\to Z,g_1(y)=1,g_1(y^\prime)=y^\prime
&g_2:Y\to Z,g_2(y)=2,g_2(y^\prime)=y^\prime
\end{align*}

Then when $y$ is not in the image of $f$, these two functions will satisfy $g_1 \circ f=g_2 \circ f$ but not $g_1=g_2$.

So conversely, if for any set $Z$ and any functions $g_i:Y \to Z$ we have $g_1 \circ f=g_2 \circ f \implies g_1=g_2$, such a value $y$ that is in the codomain but not in the range of $f$ cannot appear, and hence $f$ must be surjective.
\end{proof}

\subsection{Invertibility}
Recalling that $\id_X$ is the identity map on $X$, we can define invertibility:

\begin{definition}[Invertibility]
A function $f:X\to Y$ is \vocab{invertible}\index{invertibility (function)} if there exists $g:Y\to X$ such that $g\circ f=\id_X$ and $f\circ g=\id_Y$; $g$ is known as the \emph{inverse} of $f$.
\end{definition}

\begin{proposition}[Uniqueness of inverse]
If $f:X \to Y$ is invertible then its inverse is unique.
\end{proposition}

\begin{proof}
Let $g_1$ and $g_2$ be two functions for which $g_i \circ f = \id_X$ and $f \circ g_i = \id_Y$. Using the fact that composition is associative, and the definition of the identity maps, we can write
\[ g_1 = g_1 \circ \id_Y = g_1 \circ (f \circ g_2) = (g_1 \circ f) \circ g_2 = \id_X \circ g_2 = g_2.\]
\end{proof}

Since the inverse is unique, we can give it a notation.

\begin{notation}
The inverse of $f$ is denoted by $f^{-1}$
\end{notation}

\begin{remark}
Note that directly from the definition, if $f$ is invertible then $f^{-1}$ is also invertible, and $(f^{-1})^{-1}=f$.
\end{remark}

The following result shows how to invert the composition of invertible functions:

\begin{proposition}[Inverse of composition]
Let $f:X \to Y$, $g:Y \to Z$. If $f$ and $g$ are invertible, then $g \circ f$ is invertible, and
\[(g \circ f)^{-1}=f^{-1}\circ g^{-1}.\]
\end{proposition}

\begin{proof}
Making repeated use of the fact that function composition is associative, and the definition of the inverses $f^{-1}$ and $g^{-1}$, we note that
\begin{align*}
(f^{-1}\circ g^{-1}) \circ (g \circ f) 
&= ((f^{-1} \circ g^{-1}) \circ g) \circ f \\
&= (f^{-1} \circ (g^{-1} \circ g)) \circ f \\
&= (f^{-1} \circ \id_Y) \circ f \\
&= f^{-1} \circ f \\
&= \id_X
\end{align*}
and similarly,
\begin{align*}
(g \circ f) \circ (f^{-1} \circ g^{-1}) 
&= g \circ (f \circ (f^{-1} \circ g^{-1})) \\
&= g \circ ((f \circ f^{-1}) \circ g^{-1}) \\
&= g \circ (\id_Y \circ g^{-1}) \\
&= g \circ g^{-1} \\
&= \id_Z
\end{align*}
which shows that $f^{-1} \circ g^{-1}$ satisfies the properties required to be the inverse of $g \circ f$.
\end{proof}

The following result provides an important and useful criterion for invertibility:

\begin{lemma}[Invertibility criterion]
$f:X \to Y$ is invertible if and only if it is bijective.
\end{lemma}

\begin{proof} \

\fbox{$\implies$} Suppose $f$ is invertible, so it has an inverse $f^{-1}: Y \to X$. We first show $f$ is injective. Suppose that $f(x_1) = f(x_2)$ for some $x_1, x_2 \in X$. Then applying $f^{-1}$ to both sides and noting that $f^{-1}\circ f=\id_X$ by definition,
\[x_1=f^{-1}\brac{f(x_1)}=f^{-1}\brac{f(x_2)}=x_2.\]
Thus $f$ is injective.

We now show that $f$ is surjective. Let $y \in Y$, and note that $f^{-1}(y) \in X$ has the property that $f(f^{-1}(y)) = y$. So $f$ is surjective. Therefore $f$ is bijective.

\fbox{$\impliedby$} Suppose $f$ is bijective. We want to show that there exists $g:Y \to X$ such that $g \circ f = \id_X$ and $f \circ g = \id_Y$.

Since $f$ is surjective, we know that for any $y \in Y$, there is an $x \in X$ such that $f(x) = y$. Furthermore, since $f$ is injective, we know that this $x$ is unique. So for each $y \in Y$ there is a unique $x \in X$ such that $f(x) = y$. This recipe provides a well-defined function $g(y) = x$, for which we have $g(f(x)) = x$ for any $x \in X$ and $f(g(y)) = y$ for any $y \in Y$. So $g$ satisfies the property required to be an inverse of $f$ and therefore $f$ is invertible.
\end{proof}

It is also possible to define left-inverse and right-inverse functions as functions that partially satisfy the definition of the inverse:

\begin{definition}
A function $f:X \to Y$ is \vocab{left invertible} if there exists a function $g:Y \to X$ such that $g \circ f = \id_X$, and is \vocab{right invertible} if there exists a function $h: Y \to X$ such that $f \circ h = \id_Y$.
\end{definition}

As may be somewhat apparent from the previous proof, being left- and right-invertible is equivalent to being injective and surjective, respectively. We leave this as an exercise to show.

\subsection{Monotonicity}
\begin{definition}[Monotonicity]
$f:[a,b]\to\RR$ is called
\begin{enumerate}[label=(\roman*)]
\item \vocab{increasing}, if any $a<x_1\le x_2<b$, there is $f(x_1)\le f(x_2)$;
\item \vocab{decreasing}, if any $a<x_1\le x_2<b$, there is $f(x_1)\ge f(x_2)$;
\end{enumerate}
$f$ is \vocab{monotonic}\index{function!monotonicity} if it is increasing or decreasing.
\end{definition}

Suppose $f(x)$ is continuous in $[a,b]$. To locate the roots of $f(x)=0$:
\begin{itemize}
\item If $f(a)$ and $f(b)$ have \emph{opposite} signs, i.e. $f(a)f(b)<0$, then there is an odd number of real roots (counting repeated) in $[a,b]$.

Furthermore, if $f$ is either strictly increasing or decreasing in $[a,b]$, then $f(x)=0$ has exactly one real root in $[a,b]$.

\item If $f(a)$ and $f(b)$ have \emph{same} signs, i.e. $f(a)f(b)>0$, then there is an even number of roots (counting repeated) in $[a,b]$.
\end{itemize}

\begin{definition}[Convexity]
A function $f$ is \vocab{convex} if for all $x_1,x_2\in D_f$ and $0\le t\le 1$, we have
\[ f(tx_1+(1-t)x_2)\le tf(x_1)+(1-t)f(x_2).\]
$f$ is \emph{strictly convex} if the $\le$ sign above is replaced with a strict inequality $<$.

Similarly, $f$ is \vocab{concave} if for all $x_1,x_2\in D_f$ and $0\le t\le 1$, we have
\[ f(tx_1+(1-t)x_2)\ge tf(x_1)+(1-t)f(x_2). \]
$f$ is \emph{strictly concave} if the $\ge$ sign above is replaced with a strict inequality $>$.
\end{definition}
\pagebreak

\section{Ordered Sets and Boundedness}
Let $S$ be a set.
\begin{definition}[Order]
An \vocab{order}\index{order} on $S$ is a binary relation, denoted by $<$, with the following properties:
\begin{enumerate}[label=(\roman*)]
\item Trichotomy: $\forall x,y \in S$, one and only one of the following statements is true:
\[x<y,\quad x=y,\quad y<x.\]
\item Transitivity: $\forall x,y,z \in S$, if $x<y$ and $y<z$, then $x<z$.
\end{enumerate}
We call $(S,<)$ an \vocab{ordered set}.
\end{definition}

\begin{notation}
$x \le y$ indicates that $x<y$ or $x=y$, without specifying which of these two is to hold. In other words, $x\le y$ is the negation of $x>y$.
\end{notation}

\begin{definition}[Boundedness]
Suppose $S$ is an ordered set, and $E\subset S$.
\begin{itemize}
\item If there exists $\beta\in S$ such that $x\le\beta$ for all $x\in E$, we say that $E$ is \vocab{bounded above}, and call $\beta$ an \vocab{upper bound} of $E$.
\item If there exists $\beta\in S$ such that $x\ge\beta$ for all $x\in E$, we say that $E$ is \vocab{bounded below}, and call $\beta$ a \vocab{lower bound} of $E$.
\end{itemize}
$E$ is \vocab{bounded} in $S$ if it is bounded above and below.
\end{definition}

\begin{definition}[Supremum]
Suppose $S$ is an ordered set, $E\subset S$, and $E$ is bounded above. We call $\alpha\in S$ the \vocab{supremum}\index{supremum} of $E$, denoted by $\alpha=\sup E$, if it satisfies the following properties:
\begin{enumerate}[label=(\roman*)]
\item $\alpha$ is an upper bound for $E$;
\item if $\beta<\alpha$ then $\beta$ is not an upper bound of $E$, i.e. $\exists x\in S\suchthat x>\beta$ (least upper bound).
\end{enumerate}
\end{definition}

\begin{definition}[Infimum]
We cal $\alpha\in S$ the \vocab{infimum}\index{infimum} of $E$, denoted by $\alpha=\inf E$, if it satisfies the following properties:
\begin{enumerate}[label=(\roman*)]
\item $\alpha$ is a lower bound for $E$;
\item if $\beta>\alpha$ then $\beta$ is not a lower bound of $E$, i.e. $\exists x\in S\suchthat x<\beta$ (greatest lower bound).
\end{enumerate}
\end{definition}

\begin{proposition}[Uniqueness of suprenum]
If $E$ has a supremum, then it is unique.
\end{proposition}

\begin{proof}
Assume that $M$ and $N$ are suprema of $E$.

Since $N$ is a supremum, it is an upper bound for $E$. Since $M$ is a supremum, then it is the least upper bound and thus $M \le N$. 

Similarly, since $M$ is a supremum, it is an upper bound for $E$; since $N$ is a supremum, it is a least upper bound and thus $N \le M$.

Since $N \le M$ and $M \le N$, thus $M=N$. Therefore, a supremum for a set is unique if it exists.
\end{proof}

\begin{definition}
An ordered set $S$ is said to have the \vocab{least-upper-bound property} (l.u.b.) if the following is true: if non-empty $E\subset S$ is bounded above, then $\sup E\in S$.

Similarly, $S$ has the greatest-lower-bound property if the following is true: if non-empty $E\subset S$ is bounded below, then $\inf E\in S$.
\end{definition}

We shall now show that there is a close relation between greatest lower bounds and least upper bounds, and that every ordered set with the least-upper-bound property also has the greatest-lower-bound property.

\begin{theorem}
Suppose $S$ is an ordered set. If $S$ has the least-upper-bound property, then $S$ has the greatest-lower-bound property.
\end{theorem}

\begin{proof}
Suppose $B\subset S$, $B\neq\emptyset$ is bounded below. We want to show that $\inf B\in S$. To do so, let $L\subset S$ be the set of all lower bounds of $B$; that is, $L=\{y\in S\mid y\le x\forall x\in B\}$. If we can show that $\inf B=\sup L$, then we are done.

Since $B$ is bounded below, $L\neq\emptyset$. Since every $x\in B$ is an upper bound of $L$, $L$ is bounded above. Then since $S$ has the least-upper-bound property, we have that $\sup L\in S$.

To show that $\sup L=\inf B$, we need to show that $\sup L$ is a lower bound of $B$, and $\sup L$ is the greatest of the lower bounds.

Suppose $\gamma<\sup L$, then $\gamma$ is not an upper bound of $L$. Since $B$ is the set of upper bounds of $L$, $\gamma\notin B$. Considering the contrapositive, if $\gamma\in B$, then $\gamma\ge\sup L$. Hence $\sup L$ is a lower bound of $B$, and thus $\sup L\in L$.

If $\sup L<\beta$ then $\beta\notin L$, since $\sup L$ is an upper bound of $L$. In other words, $\sup L$ is a lower bound of $B$, but $\beta$ is not if $\beta>\sup L$. This means that $\sup L=\inf B$.
\end{proof}

\begin{corollary}
If $S$ has the greatest-lower-bound property, then it has the least-upper-bound property.

Hence $S$ has the least-upper-bound property if and only if $S$ has the greatest-lower-bound property.
\end{corollary}

Let's explore some useful properties of sup and inf.

\begin{proposition}[Comparison theorem]
Let $S, T \subset \RR$ be non-empty sets such that $s \le t$ for every $s \in S$ and $t \in T$. If $T$ has a supremum, then so does $S$, and $\sup S \le \sup T$.
\end{proposition}

\begin{proof}
Let $\tau = \sup T$. Since $\tau$ is a supremum for $T$, then $t \le \tau$ for all $t \in T$. Let $s \in S$ and choose any $t \in T$. Then, since $s \le t$ and $t \le \tau$ , then $s \le t$. Thus, $\tau$ is an upper bound for $S$. 

By the Completeness Axiom, $S$ has a supremum, say $\sigma = \sup S$. We will show that $\sigma \le \tau$. Notice that, by the above, $\tau$ is an upper bound for $S$. Since $\sigma$ is the least upper bound for $S$, then $\sigma \le \tau$. Therefore,
\[\sup S \le \sup T.\]
\end{proof}

Let's explore some useful properties of sup and inf.

\begin{proposition}
Let $S, T$ be non-empty subsets of $\RR$, with $S \subset T$ and with $T$ bounded above. Then $S$ is bounded above, and $\sup S \le \sup T$.
\end{proposition}
\begin{proof}
Since $T$ is bounded above, it has an upper bound, say $b$. Then $t \le b$ for all $t \in T$, so certainly $t \le b$ for all $t \in S$, so $b$ is an upper bound for $S$.

Now $S, T$ are non-empty and bounded above, so by completeness each has a supremum. Note that $\sup T$ is an upper bound for $T$ and hence also for $S$, so $\sup T \ge \sup S$ (since $\sup S$ is the least upper bound for $S$).
\end{proof}

\begin{proposition}
Let $T \subset \RR$ be non-empty and bounded below. Let $S = \{-t \mid t \in T\}$. Then $S$ is non-empty and bounded above. Furthermore, $\inf T$ exists, and $\inf T = -\sup S$.
\end{proposition}
\begin{proof}
Since $T$ is non-empty, so is $S$. Let $b$ be a lower bound for $T$, so $t \ge b$ for all $t \in T$. Then $-t \le -b$ for all $t \in T$, so $s \le -b$ for all $s \in S$, so $-b$ is an upper
bound for $S$.

Now $S$ is non-empty and bounded above, so by completeness it has a
supremum. Then $s \le \sup S$ for all $s \in S$, so $t \ge -\sup S$ for all $t \in T$, so $-\sup S$ is a lower bound for $T$.

Also, we saw before that if $b$ is a lower bound for $T$ then $-b$ is an upper bound for $S$. Then $-b \ge \sup S$ (since $\sup S$ is the least upper bound), so $b \le -\sup S$. So $-\sup S$ is the greatest lower bound.

So $\inf T$ exists and $\inf T = -\sup S$.
\end{proof}

\begin{proposition}[Approximation property]
Let $S\subset\RR$ be non-empty and bounded above. For any $\epsilon > 0$, there is $s_\epsilon \in S$ such that $\sup S-\epsilon < s_\epsilon \le \sup S$.
\end{proposition}

\begin{proof}
Take $\epsilon > 0$.

Note that by definition of the supremum we have $s \le \sup S$ for all $s \in S$. Suppose, for a contradiction, that $\sup S-\epsilon \ge s$ for all $s \in S$.

Then $\sup S-\epsilon$ is an upper bound for $S$, but $\sup S-\epsilon < \sup S$, which is a contradiction.

Hence there is $s_\epsilon \in S$ with $\sup S-\epsilon<s_\epsilon$.
\end{proof}

If we are dealing with rational numbers, the sup/inf of a set may not exist. For example, a set of numbers in $\QQ$, defined by $\{[\pi\cdot10^n]/10^n\}$.
3,3.1,3.14,3.141,3.1415,3.14159,...
But this set does not have an infimum in $\QQ$.

By ZFC, we have the Completeness Axiom, which states that any non-empty set $A \subset \RR$ that is bounded above has a supremum; in other words, if $A$ is a non-empty set of real numbers that is bounded above, there exists a $M \in \RR$ such that $M = \sup A$.

\begin{exercise}
Consider the set
\[\crbrac{\frac{1}{n}\mid n\in\ZZ^{+}}.\]
\begin{enumerate}[label=(\alph*)]
\item Show that $\max S=1$.
\item Show that if $d$ is a lower bound for $S$, then $d \le 0$.
\item Use (b) to show that $0 = \inf S$.
\end{enumerate}
\end{exercise}

\begin{exercise}
Find, with proof, the supremum and/or infimum of $\{\frac{1}{n}\}$.
\end{exercise}

\begin{solution}
For the suprenum,
\[ \sup\crbrac{\frac{1}{n}} = \max\crbrac{\frac{1}{n}} = 1. \]
For the infinum, for all positive $a$ we can pick $n=[\frac{1}{a}]+1$, then $a>\frac{1}{n}$. Hence 
\[ \inf\crbrac{\frac{1}{n}}=0. \]
\end{solution}

\begin{exercise}
Find, with proof, the supremum and/or infimum of $\{\sin n\}$.
\end{exercise}

\begin{proof}
The answer is easy to guess: $\pm1$

For the supremum, we need to show that $1$ is the smallest we can pick, so for any $a=1-\epsilon<1$ we want to find an integer $n$ close enough to $2k\pi+\dfrac{\pi}{2}$ so that $\sin n > a$.

Whenever we want to show the approximations between rational and irrational numbers we should think of the \textbf{pigeonhole principle}.
\[ 2k\pi+\frac{\pi}{2}=6k+(2\pi-6)k+\frac{\pi}{2} \]
Consider the set of fractional parts $\{(2\pi-6)k\}$. Since this an infinite set, for any small number $\delta$ there is always two elements $\{(2\pi-6)a\}<\{(2\pi-6)b\}$ such that
\[ |\{(2\pi-6)b\}-\{(2\pi-6)a\}|<\epsilon \]

Then $\{(2\pi-6)(b-a)\}<\delta$

We then multiply by some number $m$ (basically adding one by one) so that
\[ 0 \le \{(2\pi-6)\cdot m(b-a)\}-\brac{2-\frac{\pi}{2}}<\delta \]

Picking $k=m(b-a)$ thus gives
\begin{align*}
2k\pi+\frac{\pi}{2} &= 6k+(2\pi-6)k+\frac{\pi}{2} \\
&= 6k+[(2\pi-6)k]+2+{(2\pi-6)k}-\brac{2-\frac{\pi}{2}}
\end{align*}

Thus $n=6k+[(2\pi-6)k]+2$ satisfies $\absolute{2k\pi+\dfrac{\pi}{2}-n}<\delta$

Now we're not exactly done here because we still need to talk about how well $\sin n$ approximates to 1.

We need one trigonometric fact: $\sin x<x$ for $x>0$. (This simply states that the area of a sector in the unit circle is larger than the triangle determined by its endpoints.)

\begin{align*}
\sin n&=\sin\brac{n-\brac{2k\pi+\frac{\pi}{2}}+\brac{2k\pi+\frac{\pi}{2}}} \\
&=\cos\brac{n-\brac{2k\pi+\frac{\pi}{2}}} \\
&=\cos\theta
\end{align*}

\[ 1 - \sin n = 2 \sin^2 \frac{\theta}{2} = 2 \sin^2 \absolute{\frac{\theta}{2}} \le \frac{\theta^2}{2}<\delta \]

Hence we simply pick $\delta=\epsilon$ to ensure that $1 - \sin n<\epsilon$, and we're done.
\end{proof}
\pagebreak

\section{Cardinality}
This section is about formalising the notion of the ``size'' of a set.

\begin{definition}
Two sets $A$ and $B$ said to be \vocab{equivalent} (or have the same \emph{cardinal number}), denoted by $A\sim B$, if there exists a bijection $f:A\to B$. 
\end{definition}

\begin{notation}
For $n\in\ZZ^+$, let
\begin{align*}
\ZZ^+&=\{i\in\ZZ\mid i\ge1\},\\
\ZZ_n^+&=\{i\in\ZZ^+\mid 1\le i\le n\},\\
n\ZZ^+&=\{ni\mid i\in\ZZ^+\}.
\end{align*}
\end{notation}

\begin{definition}
For any set $A$, we say
\begin{itemize}
\item $A$ is \vocab{finite} if $A\sim\ZZ_n^+$ for some integer $n$, the \textbf{cardinality} of $A$ is $|A|=n$; $A$ is \textbf{infinite} if $A$ is not finite;
\item $A$ is \vocab{countable} if $A\sim \ZZ^+$; $A$ is \textbf{uncountable} if $A$ is neither finite nor countable; $A$ is \textbf{at most countable} if $A$ is finite or countable.
\end{itemize}
\end{definition}

\subsection{Finite Sets}
For finite sets, we can do some arithmetic with their cardinalities.

\begin{proposition}[Subsets of a finite set]
If a set $A$ is finite with $|A| = n$, then its power set has $|\mathcal{P}(A)| = 2^n$.
\end{proposition}

\begin{proof}
We use induction. For the initial step, note that if $|A| = 0$ then $A = \emptyset$ has no elements, so there is a single subset $\emptyset$, and therefore $|\mathcal{P}(A)| = 1 = 2^0$.

Now suppose that $n \ge 0$ and that $|P(S)| = 2^n$ for any set S with $|S| = n$. Let $A$ be any set with $|A| = n+1$. By definition, this means that there is an element $a$ and a set $A_0 = A\setminus\{a\}$ with $|A_0| = n$. Any subset of A must either contain the element a or not, so we can partition $\mathcal{P}(A) = P(A_0) \cup \{S \cup \{a\} \mid S \in P(A_0)\}$. These two sets are disjoint, and each of them has cardinality $|P(A_0)| = 2^n$ by the inductive hypothesis. Hence $|\mathcal{P}(A)| = 2^n + 2^n = 2^{n+1}$.

Thus, by induction, the result holds for all $n$.
\end{proof}

Another way to see this is through combinatorics: Consider the process of creating a subset. We can do this systematically by going through each of the $|A|$ elements in $A$ and making the yes/no decision whether to put it in the subset. Since there are $|A|$ such choices, that yields $2^{|A|}$ different combinations of elements and therefore $2^{|A|}$ different subsets.

\begin{theorem}[Cantor's Theorem]\label{thrm:cantor}
For a set $A$, finite or infinite,
\[|A|<|\mathcal{P}(A)|.\]
\end{theorem}

\begin{proof}
Suppose, for a contradiction, that $|A|=|\mathcal{P}(A)|$. Then there exists a bijection $f:A\to\mathcal{P}(A)$. Put
\[B=\{x\in A\mid x\notin f(A)\}.\]

Now consider any $x\in A$. In the first case, $x\in f(A)$, then
\[x\in f(A)\iff x\notin B,\]
thus $f(A)\neq B$. In the second case, $x\notin f(A)$, then 
\[x\notin f(A)\iff x\in B,\]
thus $f(x)\neq B$. Hence $f$ is not surjective, which is a contradiction.
\end{proof}

\begin{corollary}
For all $n\in\ZZ_0^+$,
\[n<2^n.\]
\end{corollary}

\begin{proof}
This can be easily proven through induction.
\end{proof}

\begin{proposition}
Let $A$ and $B$ be finite sets. Then $|A \cup B| = |A| + |B| - |A \cap B|$.
\end{proposition}

\begin{proof}
The proof is left as an exercise.
\end{proof}

\begin{theorem}[Principle of Inclusion and Exclusion]
Let $S_i$ be finite sets. Then
\begin{equation}
\absolute{\bigcup_{i=1}^nS_i}=\sum_{i=1}|S_i|-\sum_{1\le i<j\le n}|S_i\cap S_j|+\sum_{1\le i<j<k\le n}|S_i\cap S_j\cap S_k|+\cdots+(-1)^{n+1}\absolute{\bigcap_{i=1}^nS_i}.
\end{equation}
\end{theorem}

\begin{proof}
By induction.
\end{proof}

\begin{proof}[Alternative proof]
Let $U$ be a finite set (interpreted as the universal set), and $S\subset U$. Define the characteristic/indicator function of $S$ by
\[ \chi_S(x)=\begin{cases}
1&(x\in S)\\
0&(x\notin S)
\end{cases} \]
In other words,
\[ x\in S\iff\chi_S(x)=1 \]
and equivalently,
\[ x\notin S\iff\chi_S(x)=0. \]
Let $S_1,S_2\subset U$ be given. Then for any $x\in U$ it holds that
\[ \chi_{S_1\cap S_2}(x)=\chi_{S_1}(x)\cdot\chi_{S_2}(x) \]
where $\cdot$ denotes ordinary multiplication.

Similarly,
\[ \chi_{S_1\cup S_2}(x)=1-\brac{1-\chi_{S_1}(x)}\cdot\brac{1-\chi_{S_2}(x)}. \]
In general, for any $x\in U$ it holds that
\[ \chi_{S_1\cup\cdots\cup S_n}(x)=1-\brac{1-\chi_{S_1}(x)}\cdots\brac{1-\chi_{S_n}(x)} \]
for any $S_1,\dots,S_n\subset U$.

Since $x\in S$ if and only if $\chi_S(x)=1$, it follows that
\[ |S|=\sum_{x\in U}\chi_S(x). \]
To prove the PIE, we calculate
\begin{align*}
&|S_1\cup\cdots\cup S_n|\\
&=\sum_{x\in U}\chi_{S_1\cup\cdots\cup S_n}(x)\\
&=\sum_{x\in U}1-\brac{1-\chi_{S_1}(x)}\cdots\brac{1-\chi_{S_n}(x)}\\
&=\brac{\chi_{S_1}(x)+\cdots+\chi_{S_n}(x)}-\brac{\chi_{S_1}(x)\chi_{S_2}(x)+\cdots+\chi_{S_{n-1}}(x)\chi_{S_n}(x)}+\cdots+(-1)^{n+1}\chi_{S_1}(x)\cdots\chi_{S_n}(x)\\
&=\brac{\chi_{S_1}(x)+\cdots+\chi_{S_n}(x)}-\brac{\chi_{S_1\cap S_2}(x)+\cdots+\chi_{S_{n-1}\cap S_n}(x)}+\cdots+(-1)^{n+1}\chi_{S_1\cap\cdots\cap S_n}(x)\\
&=\sum_{i=1}^n|S_i|-\sum_{J\subset\{1,\dots,n\},|J|=2}\absolute{\bigcap_{j\in J}S_j}+\cdots+(-1)^{k+1}\sum_{J\subset\{1,\dots,n\},|J|=k}\absolute{\bigcap_{j\in J}S_j}+\cdots+(-1)^{n+1}\absolute{\bigcap_{i=1}^nS_i}.
\end{align*}
\end{proof}

\subsection{Countability}
For two finite sets $A$ and $B$, we evidently have $A\sim B$ if and only if $A$ and $B$ contain the same number of elements. For infinite sets, however, the idea of ``having the same number of elements'' becomes quite vague, whereas the notion of bijectivity retains its clarity.

\begin{proposition}
$n\ZZ^+$ is countable.
\end{proposition}

\begin{proof}
Let $f:\ZZ^+\to n\ZZ^+$ be given by 
\[f(k)=nk.\]
For any $k_1,k_2\in\ZZ^+$, $nk_1=nk_2$ implies $k_1=k_2$ so $f$ is injective. For any $x\in n\ZZ^+$, $x=ni$ for some $i\in\ZZ^+$, thus $\frac{x}{n}=i\in\ZZ^+$ so $f$ is surjective. Hence $f$ is bijective, $n\ZZ^+\sim\ZZ^+$ and we are done.
\end{proof}

\begin{proposition}
$\ZZ$ is countable.
\end{proposition}

\begin{proof}
Consider the following arrangement of the elements of $\ZZ$ and $\ZZ^+$:
\begin{align*}
\ZZ&:\quad0,1,-1,2,-2,3,-3,\dots\\
\ZZ^+&:\quad1,2,3,4,5,6,7,\dots
\end{align*}
The function $f:\ZZ^+\to\ZZ$ defined as
\[f(n)=\begin{cases}
\dfrac{n}{2}&\text{($n$ even)}\\[1ex]
-\dfrac{n-1}{2}&\text{($n$ odd)}
\end{cases}\]
is bijective.
\end{proof}

\begin{proposition}
Every infinite subset of a countable set $A$ is countable.
\end{proposition}

\begin{proof}
Suppose $E\subset A$, and $E$ is infinite. Arrange the elements $x\in A$ in a sequence $\{x_n\}$ of distinct elements. Construct a sequence $\{n_k\}$ as follows: Let $n_1$ be the smallest positive integer such that $x_{n_1}\in E$. Having chosen $n_1,\dots,n_{k-1}$ ($k=2,3,4,\dots$), let $n_k$ be the smallest integer greater than $n_{k-1}$ such that $x_{n_k}\in E$.

Let $f:\ZZ^+\to E$ be defined as
\[f(k)=x_{n_k},\]
which is bijective. Hence $E\sim\ZZ^+$, $E$ is countable.
\end{proof}

This shows that countable sets represent the ``smallest'' infinity: No uncountable set can be a subset of a countable set.

\begin{proposition}
Let $(E_n)$ be a sequence of countable sets, put
\[S=\bigcup_{n=1}^\infty E_n.\]
Then $S$ is countable.
\end{proposition}

\begin{proof}
Let every set $E_n$ be arranged in a sequence $\{x_{n_k}\}$ ($k=1,2,3,\dots$), and consider the infinite array
\begin{align*}
&x_{11}\quad x_{12}\quad x_{13}\quad x_{14}\quad\cdots\\
&x_{21}\quad x_{22}\quad x_{23}\quad x_{24}\quad\cdots\\
&x_{31}\quad x_{32}\quad x_{33}\quad x_{34}\quad\cdots\\
&x_{41}\quad x_{42}\quad x_{43}\quad x_{44}\quad\cdots\\
&\vdots
\end{align*}
in which the elements of $E_n$ form the $n$-th row. The array contains all elements of $S$. These elements can be arranged in a sequence
\[x_{11},x_{21},x_{12},x_{31},x_{22},x_{13},x_{41},x_{32},x_{23},x_{14},\dots\]
If any two of the sets En have elements in common, these will appear more than once in (17). Hence there is a subset $T$ of the set of all positive integers such that S ~ T, which shows that $S$ is at most countable (Theorem 2.8). Since $E_1\subset S$, and $E_1$ is infinite, $S$ is infinite, and thus countable. 
\end{proof}

\begin{corollary}
Suppose $A$ is at most countable, and, for every $\alpha\in A$, $B_\alpha$ is at most countable. Put
\[T=\bigcup_{\alpha\in A}B_\alpha.\]
Then $T$ is at most coutable.
\end{corollary}

\begin{proposition}
Let $A$ be a countable set, and let $B_n$ be the set of all $n$-tuples $(a_1,\dots,a_n)$, where $a_i\in A$. Then $B_n$ is countable.
\end{proposition}

\begin{corollary}
$\QQ$ is countable.
\end{corollary}

\begin{proposition}
The set of all algebraic numbers is countable. (Exercise 2)
\end{proposition}

\begin{proposition}
Let A be the set of all sequences whose elements are the digits 0
and 1. This set A is uncountable. 
\end{proposition}

The idea of the above proof was first used by Cantor, and is called Cantor's diagonal process; for, if the sequences $s_1,s_2,s_3,\dots$ are placed in an array like (16), it is the elements on the diagonal which are involved in the construction of the new sequence.

\begin{corollary}
$\RR$ is uncountable.
\end{corollary}

\begin{proof}
This follows from the binary representation of the real numbers.
\end{proof}

\subsection{Infinite Sets}
A consequence of Cantor's Theorem (\cref{thrm:cantor}) is that there is no largest infinity. Since there are an infinite number of different sizes of infinity, it makes sense for us to order them from smallest onwards.

The \vocab{aleph numbers} are a sequence of numbers used to represent the cardinality of infinite sets, given by
\[\aleph_0,\aleph_1,\aleph_2,\aleph_3,\dots,\]
where $\aleph_0=|\NN|$.

Another set of infinite cardinals is the set of \vocab{beth numbers},
\[\beth_0,\beth_1,\beth_2,\beth_3,\dots,\]
where
\begin{align*}
\beth_0&=\aleph_0,\\
\beth_1&=2^{\aleph_0}=|\mathcal{P}(\NN)|=|\RR|,
\end{align*}
and in general, for all $n$, we can recursively define
\[\beth_{n+1}=2^{\beth_n}.\]

A natural question to ask is if the aleph numbers and beth numbers line up. We have defined $\beth_0=\aleph_0$, but is $\beth_1=\aleph_1$? Another way to ask this question is whether
\[|\mathcal{P}(\NN)|=|\RR|.\]
This is called the \textbf{continuum hypothesis}. In fact it has been show that the continuum hypothesis can neither be proved nor disproved using the standard ZFC set theory axioms. The generalised continuum hypothesis is as follows:
\begin{align*}
2^{\aleph_n}&=\aleph_{n+1}\quad(\forall n)\\
\aleph_n&=\beth_n\quad(\forall n)
\end{align*}
A restatement of the above is that there is no set $S$ such that
\[\aleph_n<|S|<2^{\aleph_n}.\]
\pagebreak

\section*{Exercises}
\begin{prbm}
Let $A$ be the set of all complex polynomials in $n$ variables. Given a subset $T \subset A$, define the \textit{zeros} of $T$ as the set
\[ Z(T) = \{P=(a_1,\dots,a_n) \mid f(P)=0 \text{ for all } f \in T\} \]
A subset $Y \in \CC^n$ is called an algebraic set if there exists a subset $T \subset A$ such that $Y=Z(T)$.

Prove that the union of two algebraic sets is an algebraic set.
\end{prbm}
\begin{proof}
We would like to consider $T=\{f_1, f_2, \dots\}$ expressed as indexed sets $T=\{f_i\}$. Then $Z(T)$ can also be expressed as $\{P \mid \forall i, f_i(P)=0\}$.

Suppose that we have two algebraic sets $X$ and $Y$. Let $X=Z(S)$, $Y=Z(T)$ where $S,T$ are subsets of $A$ (basically, they are certain sets of polynomials). Then
\[ X=\{P \mid \forall f \in S, f(P)=0\} \]
\[ Y=\{P \mid \forall g \in T, g(P)=0\} \]

We imagine that for $P\in X\cap Y$, we have $f(P)=0$ or $g(P)=0$. Hence we consider the set of polynomials
\[ U=\{f\cdot g \mid f\in S, g\in T\} \]

For any $P\in X\cup Y$ and for any $fg\in U$ where $f\in S$ and $f\in g$, either $f(P)=0$ or $g(P)=0$, hence $fg(P)=0$ and thus $P\in Z(U)$.

On the other hand if $P\in Z(U)$, suppose otherwise that $P$ is not in $X\cup Y$, then $P$ is neither in $X$ nor in $Y$. This means that there exists $f\in S,g\in T$ such that $f(P)\neq0$ and $g(P)\neq0$, hence $fg(P)\neq0$. This is a contradiction as $P\in Z(U)$ implies $fg(P)=0$. Hence we have $X\cup Y=Z(U)$ and thus $X\cup Y$ is an algebraic set.

Now the other direction is simpler and can actually be generalised: The intersection of arbitrarily many algebraic sets is algebraic. 

The basic result is that if $X=Z(S)$ and $Y=Z(T)$ then $X\cap Y=Z(S\cup T)$. 
\end{proof}
\pagebreak

\begin{prbm}
Let $A = \RR$ and for any $x, y \in A$, $x \sim y$ if and only if $x-y \in \ZZ$. For any two equivalence classes $[x], [y] \in A/\sim$, define
\[ [x] + [y] = [x + y] \text{ and } -[x] = [-x] \]
\begin{enumerate}[label=(\alph*)]
\item Show that the above definitions are well-defined.
\item Find a one-to-one correspondence $\phi:X \to Y$ between $X = A/\sim$ and $Y:|z| = 1$, i.e. the unit circle in $\CC$, such that for any $[x_1], [x_2] \in X$ we have
\[ \phi([x_1])\phi([x_2]) = \phi([x_1 + x_2]) \]
\item Show that for any $[x] \in X$,
\[ \phi(-[x]) = \phi([x])^{-1} \]
\end{enumerate}
\end{prbm}

\begin{solution} \ 
\begin{enumerate}[label=(\alph*)]
\item 
\[ (x^\prime+y^\prime)-(x+y)=(x^\prime-x)+(y^\prime-y)\in \ZZ \]
Thus $[x^\prime+y^\prime]=[x+y]$

\[ (-x^\prime)-(-x)=-(x^\prime-x)\in \ZZ \]
Thus $[-x^\prime]=[-x]$.

\item Complex numbers in the polar form: $z=re^{i\theta}$

Then the correspondence is given by $\phi([x])=e^{2\pi ix}$
\[ [x]=[y] \iff x-y\in \ZZ \iff e^{2\pi i(x-y)}=1 \iff e^{2\pi ix}=e^{2\pi iy} \]
Hence this is a bijection.

Before that, we also need to show that $\phi$ is well-defined, which is almost the same as the above.

If we choose another representative $x^\prime$ then
\[ \phi([x])=e^{2\pi ix^\prime} = e^{2\pi ix}\cdot e^{2\pi i(x^\prime-x)} = e^{2\pi ix} \]

\item You can either refer to the specific correspondence $\phi([x])=e^{2\pi ix}$ or use its properties.
\[ \phi(-[x])\phi([x]) = \phi([-x])\phi([x]) = \phi([-x+x]) = \phi([0]) = 1 \]
\end{enumerate}
\end{solution}
\pagebreak

\begin{prbm}[Complex Numbers]
Let $\RR[x]$ denote the set of real polynomials. Define
\[ \CC=\RR[x]/(x^2+1)\RR[x] \]
where
\[ f(x)\sim g(x) \iff x^2+1 \text{ divides } f(x)-g(x). \]
The complex number $a+bi$ is defined to be the equivalence class of $a+bx$.
\begin{enumerate}[label=(\alph*)]
\item Define the sum and product of two complex numbers and show that such definitions are well-defined.
\item Define the reciprocal of a complex number.
\end{enumerate}
\end{prbm}
\pagebreak
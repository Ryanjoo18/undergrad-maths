\chapter{Calculus}
\section{Single Variable Calculus}
\subsection{Integration Techniques}
We review the following basic techniques for evaluating integrals:
\begin{itemize}
\item Integration by substitution
\item Integration by parts, reduction formula
\end{itemize}

\begin{exercise}
Evaluate
\[I=\int_{0}^{1}\frac{1}{\sqrt{4-2x-x^2}}\dd{x}.\]
\end{exercise}

\begin{solution}
The integral is close to the known integral $\int\frac{1}{\sqrt{1-x^2}}\dd{x}$. By completing the square we may write
\begin{align*}
4-2x-x^2&=5-(x+1)^2\\
&=5\brac{1-\brac{\frac{x+1}{\sqrt{5}}}^2}.
\end{align*}
Making the substitution $u=\frac{x+1}{\sqrt{5}}$ we have $\dd{u}=\frac{1}{\sqrt{5}}\dd{x}$, so that
\begin{align*}
I&=\int_{0}^{1}\frac{1}{\sqrt{4-2x-x^2}}\dd{x}\\
&=\frac{1}{\sqrt{5}}\int_{0}^{1}\frac{1}{\sqrt{1-\brac{\frac{x+1}{\sqrt{5}}}^2}}\dd{x}\\
&=\frac{1}{\sqrt{5}}\int_{\frac{1}{\sqrt{5}}}^{\frac{2}{\sqrt{5}}}\frac{\sqrt{5}}{\sqrt{1-u^2}}\dd{u}\\
&=\int_{\frac{1}{\sqrt{5}}}^{\frac{2}{\sqrt{5}}}\frac{1}{\sqrt{1-u^2}}\dd{u}\\
&=\sin^{-1}\frac{2}{\sqrt{5}}-\sin^{-1}\frac{1}{\sqrt{5}}.
\end{align*}
\end{solution}

Now let us recall the technique of integration by parts. This is the integral form of the product rule for derivatives, that is $(fg)^\prime=f^\prime g+fg^\prime$, where $f$ and $g$ are functions of $x$, and the prime denotes the derivative with respect to $x$. Thus we have
\[f(x)g(x)=\int g(x)f^\prime(x)\dd{x}+\int f(x)g^\prime(x)\dd{x}\]
and we arrange the terms to obtain
\[\int f(x)g^\prime(x)\dd{x}=f(x)g(x)-\int g(x)f^\prime(x)\dd{x}.\]
Similarly, for definite integrals we have
\[\int_{a}^{b}f(x)g^\prime(x)\dd{x}=\sqbrac{f(x)g(x)}_a^b-\int_{a}^{b}g(x)f^\prime(x)\dd{x}.\]

\begin{exercise}
Evaluate
\[I=\int xe^x\dd{x}.\]
\end{exercise}

\begin{solution}
We have $f(x)=x$ and $g^\prime(x)=e^x$, so that $f^\prime(x)=1$ and $g(x)=e^x$. Thus
\begin{align*}
I&=\int xe^x\dd{x}\\
&=xe^x-\int e^x\dd{x}\\
&=xe^x-e^x+c.
\end{align*}
\end{solution}

Sometimes, after two applications of the ``by parts'' formula, we almost get back to where we started:

\begin{exercise}
Evaluate
\[I=\int e^x\sin x\dd{x}.\]
\end{exercise}

\begin{solution}
\begin{align*}
\int e^x\sin x\dd{x}
&=e^x\sin x-\int e^x\cos x\dd{x}\\
&=e^x\sin x-e^x\cos x-\int e^x\sin x\dd{x}.
\end{align*}
Now we see that we have returned to our original integral, so that we can rearrange this equation to obtain
\[\int e^x\sin x\dd{x}=\frac{1}{2}e^x\brac{\sin x-\cos x}+c.\]
\end{solution}

Finally in this section we look at an example of a \emph{reduction formula}.

\begin{exercise}
Consider $I_n=\int\cos^n x\dd{x}$ where $n$ is a non-negative integer. Find a reduction formula for $I_n$, and use this formula to evaluate $\int\cos^7x\dd{x}$.
\end{exercise}

\begin{solution}
The aim here is to write $I_n$ in terms of other $I_k$ where $k<n$, so that eventually we are reduced to calculating $I_0$ or $I_1$, say, both of which are easily found (analagous to a recurrence relation). Using integration by parts we have:
\begin{align*}
I_n&=\int\cos^n x\dd{x}\\
&=\int\cos^{n-1}\times\cos x\dd{x}\\
&=\cos^{n-1}x\sin x+(n-1)\int\cos^{n-2}x\sin^2 x\dd{x}\\
&=\cos^{n-1}x\sin x+(n-1)\int\cos^{n-2}x(1-\cos^2 x)\dd{x}\\
&=\cos^{n-1}x\sin x+(n-1)\brac{I_{n-2}-I_n}.
\end{align*}

Rearranging this to make $I_n$ the subject we obtain
\[I_n=\frac{1}{n}\cos^{n-1}x\sin x+\frac{n-1}{n}I_{n-2}.\]

With this reduction formula, $I_n$ can be rewritten in terms of simpler and simpler integrals until we are left only needing to calculate $I_0$ if $n$ is even, or $I_1$ if $n$ is odd. Therefore, $I_7$ can be found as follows:
\begin{align*}
I_7&=\frac{1}{7}\cos^6x\sin x+\frac{6}{7}I_5\\
&=\frac{1}{7}\cos^6x\sin x+\frac{6}{7}\brac{\frac{1}{5}\cos^4x\sin x+\frac{4}{5}I_3}\\
&=\frac{1}{7}\cos^6x\sin x+\frac{6}{7}\brac{\frac{1}{5}\cos^4x\sin x+\frac{4}{5}\brac{\frac{1}{3}\cos^2x\sin x+\frac{2}{3}I_1}}\\
&=\frac{1}{7}\cos^6x\sin x+\frac{6}{35}\cos^4x\sin x+\frac{24}{105}\cos^2x\sin x+\frac{48}{105}\sin x+c.
\end{align*}
\end{solution}

\subsection{First Order Differential Equations}
An \vocab{ordinary differential equation} (ODE) is an equation relating a variable, say $x$, a function, say $y$, of the variable $x$, and finitely many of the derivatives of $y$ with respect to $x$. That is, an ODE can be written in the form
\[f\brac{x,y,\dv{y}{x},\dv[2]{y}{x},\dots,\dv[k]{y}{x}}=0\]
for some function $f$, $k\in\NN$. Here $x$ is the independent variable and the ODE governs how the dependent variable $y$ varies with $x$. The equation may have no, one or many functions $y(x)$ which satisfy it; the problem is usually to find the most general solution $y(x)$, a function which satisfies the differential equation.

We say that an ODE has \emph{order} $k$ if it involves derivatives of order $k$ and less. Thus first order differential equations take the form
\[\dv{y}{x}=f(x,y).\]
In general, a $k$-th order ODE takes the form
\[a_k(x)\dv[k]{y}{x}+a_{k-1}(x)\dv[k-1]{y}{x}+\cdots+a_1(x)\dv{y}{x}+a_0(x)y=f(x),\]
where $a_k(x)\neq0$. The ODE is \emph{homogeneous} if $f(x)=0$ for all $x$, and \emph{inhomogeneous} otherwise. 

The following are some standard methods for solving first order ODEs:
\begin{itemize}
\item Direct integration
\item Separation of variables
\item Reduction to separable form by substitution
\item Exact differential equations
\item Integrating factors
\end{itemize}

If the ODE takes the form
\[\dv{y}{x}=f(x),\]
then we can solve this by direct integration:

\begin{exercise}
Find the general solution to
\[\dv{y}{x}=x^2\sin x.\]
\end{exercise}

\begin{solution}
Integrating both sides with respect to $x$ and then integrating the RHS by parts, we have
\[y=-x^2\cos x+2x\sin x+2\cos x+c.\]
\end{solution}

When the ODE is \emph{separable}, that is, it takes the form
\[\dv{y}{x}=a(x)b(y),\]
where $a(x)$ and $b(y)$ are functions of $x$ and $y$ respectively, we can solve this by separating the variables:
\[\frac{1}{b(y)}\dv{y}{x}=a(x),\]
then integrating both sides with respect to $x$ we find
\[\int\frac{1}{b(y)}\dd{y}=\int a(x)\dd{x}.\]
Here we have assumed that $b(y)\neq0$; if $b(y)=0$ then the solution is $y=c$ for some constant $c$.

Some first order differential equations can be transformed by a suitable substitution into separable form.

\begin{exercise}
Find the general solution to
\[\dv{y}{x}=\sin(x+y+1).\]
\end{exercise}

\begin{solution}
Let $u=x+y+1$, so that $\dv{u}{x}=1+\dv{y}{x}$. Then the original equation can be written as
\[\dv{u}{x}=1+\sin u,\]
which is separable. We have 
\[\frac{1}{1+\sin u}\dv{u}{x}=1,\]
which integrates to
\[\int\frac{1}{1+\sin u}\dd{u}=x+c.\]
Let us evaluate the integral on the LHS:
\begin{align*}
\int\frac{1}{1+\sin u}\dd{u}&=\int\frac{1-\sin u}{(1+\sin u)(1-\sin u)}\dd{u}\\
&=\int\frac{1-\sin u}{1-\sin^2 u}\dd{u}\\
&=\int\frac{1-\sin u}{\cos^2 u}\dd{u}\\
&=\int\sec^2 u\dd{u}-\int\sec u\tan u\dd{u}\\
&=\tan u-\sec u+c.
\end{align*}
Therefore the general solution is
\[\tan(x+y+1)-\sec(x+y+1)=x+c.\]
This solution, where we have not found $y$ in terms of $x$, is called an \emph{implicit solution}.
\end{solution}

A special group of first order differential equations are homogeneous ones, of the form
\[\dv{y}{x}=f\brac{\frac{y}{x}}.\]
These can be solved by the substitution of the form $y(x)=xv(x)$, so that the ODE becomes
\[x\dv{v}{x}=f(v)-v,\]
which is separable.

Now we look specifically at first order linear ODEs, which take the general form
\[\dv{y}{x}+p(x)y=q(x).\]
We see that the homogeneous form, that is when $q(x)=0$, is separable. The inhomogeneous form can be solved by multiplying an \emph{integrating factor} $I(x)$ given by
\[I(x)=e^{\int p(x)\dd{x}}\]
on both sides of the equation, so that
\[e^{\int p(x)\dd{x}}\dv{y}{x}+p(x)e^{\int p(x)\dd{x}}y=e^{\int p(x)\dd{x}}q(x).\]
Using the product rule for derivatives, this gives
\[\dv{}{x}\brac{e^{\int p(x)\dd{x}}y}=e^{\int p(x)\dd{x}}q(x),\]
which we can integrate directly to find $y(x)$:
\[y(x)=e^{-\int p(x)\dd{x}}\brac{\int e^{\int p(x)\dd{x}}q(x)\dd{x}+c}.\]

\begin{exercise}
Solve the linear differential equation
\[\dv{y}{x}+2xy=2xe^{-x^2}.\]
\end{exercise}

\begin{solution}
The integrating factor is $I(x)=e^{\int 2x\dd{x}}=e^{x^2}$. Multiplying through by this factor gives
\[e^{x^2}\dv{y}{x}+2xe^{x^2}y=2x,\]
that is
\[\dv{}{x}\brac{e^{x^2}y}=2x.\]
Integrating both sides with respect to $x$ we find
\[e^{x^2}y=x^2+c,\]
so that the general solution is
\[y=\brac{x^2+c}e^{-x^2}.\]
\end{solution}

\subsection{Second Order Linear Differential Equations}


\section{Multivariable Calculus}
\subsection{Partial Differentiation}
\begin{definition}[Partial derivative]
Let $f:\RR^n\to\RR$ be a function of $n$ variables. The \vocab{partial derivative} of $f$ with respect to the $i$-th variable is the function
\[\pdv{f}{x_i}=\lim_{h\to0}\frac{f(x_1,\dots,x_{i-1},x_i+h,x_{i+1},\dots,x_n)-f(x_1,\dots,x_n)}{h}.\]
\end{definition}

\begin{notation}
We define second and higher order partial derivatives in a similar manner to how we define them for full derivatives. So in the case of second order partial derivatives of a function $f(x,y)$, we have
\begin{align*}
\pdv[2]{f}{x}&=\pdv{}{x}\brac{\pdv{f}{x}}=f_{xx},\\
\pdv[2]{f}{y}&=\pdv{}{y}\brac{\pdv{f}{y}}=f_{yy},\\
\pdv{f}{y,x}&=\pdv{}{y}\brac{\pdv{f}{x}}=f_{xy},\\
\pdv{f}{x,y}&=\pdv{}{x}\brac{\pdv{f}{y}}=f_{yx}.
\end{align*}
\end{notation}

If $f_{xy}$ and $f_{yx}$ are both defined and continuous in a region containing the point $(a,b)$, then 
\[f_{xy}(a,b)=f_{yx}(a,b);\]
this is known as \emph{Clairaut's theorem}. A consequence of this theorem is that we don't need to keep track of the order in which we take derivatives.

\begin{theorem}[Chain rule]
Let $F(t)=f\brac{u(t),v(t)}$ with $u$ and $v$ differentiable and $f$ being continuously differentiable in each variable. Then
\begin{equation}
\dv{F}{t}=\pdv{f}{u}\dv{u}{t}+\pdv{f}{v}\dv{v}{t}.
\end{equation}
\end{theorem}



\subsection{Coordinate systems and Jacobians}
\subsection{Double Integrals}
\subsection{Parametric representation of curves and surfaces}
\subsection{The gradient vector}
\subsection{Taylor's theorem}
\subsection{Critical points}
\subsection{Lagrange multipliers}
\documentclass[11pt,a4paper,oneside]{book}
\usepackage[top=1.0in, bottom=1.0in, left=0.8in, right=0.8in, heightrounded]{geometry}
\raggedbottom

\usepackage{ryan}
\usetikzlibrary{hobby}    

% Header and footer
\usepackage{fancyhdr} 
\pagestyle{fancy}
\fancyhf{}
\fancyhead[L]{\uppercase{\footnotesize\sffamily\leftmark}}
\fancyhead[R]{{\footnotesize\sffamily\thepage}}

\renewcommand{\headrulewidth}{0pt}
\setlength{\headsep}{0.4in}
\renewcommand{\arraystretch}{1.5}
\setlength{\parindent}{0pt}
\setlength{\parskip}{0.7em}
\setstretch{1.2}

\newcommand\nbvspace[1][3]{\vspace*{\stretch{#1}}} % command to provide stretchy vertical space in proportion

% Conditionals
\newif\ifprelim
\newif\ifcalc
\newif\iflinalg
\newif\ifabsalg
\newif\ifranalysis
\newif\ifcanalysis
\newif\iftop
\newif\ifgraph

% Bibliography
\usepackage[
backend=biber,
style=alphabetic,
sorting=nyt
]{biblatex}
\addbibresource{biblio.bib}

\makeindex % Print index

% Versions
%\excludeversion{proof}

\newenvironment{learn}{\begin{mdframed}
{\itshape\bfseries\Large Learning Outcomes}

In this chapter, we will
\begin{itemize}[noitemsep,topsep=0pt]
}{\end{itemize}\end{mdframed}}

% Upper and lower integral
\makeatletter
\NewDocumentCommand{\lowerint}{}{\mathop{}\mathpalette\lowerint@\relax\!\int}
\NewDocumentCommand{\upperint}{}{\mathop{}\mathpalette\upperint@\relax\!\int}
\newcommand{\lowerint@}[2]{%
  \begingroup
  \sbox\z@{$\m@th#1\int$}%
  \lowup@{l}{\underline}{#1}%
  \endgroup
}
\newcommand{\upperint@}[2]{%
  \begingroup
  \sbox\z@{$\m@th#1\int$}%
  \lowup@{r}{\overline}{#1}%
  \endgroup
}
\newcommand{\lowup@}[3]{%
  % #1 = l (lower) or r (upper)
  % #2 = \underline (lower) or \overline (upper)
  % #3 = math style
  \rlap{%
    \hspace{0.05\wd\z@}%
    \makebox[0.9\wd\z@][#1]{%
      $\m@th#2{%
        \hspace{0.4\wd\z@}%
        \ifx#3\displaystyle\else\hspace{0.1\wd\z@}\fi
        \vphantom{\copy\z@}%
      }$%
    }%
    \hspace{0.05\wd\z@}%
  }%
}
\makeatother

\begin{document}
%%%%%%%%%%%%%%% [TOPICS] %%%%%%%%%%%%%%%
\prelimtrue %Prelim
%\calctrue %Calc
\absalgtrue %Abstract Algebra
\linalgtrue %Linear Algebra
\ranalysistrue %Real Analysis
%\canalysistrue %Complex Analysis
%\toptrue %Topology
%\graphtrue %Graph Theory

\begin{center}
\

\vspace{3cm}

{\color{schoolbusyellow}\Huge
\uppercase{Topics in}\\[1em]
\uppercase{Pure Mathematics}}

\vspace{9cm}

{\color{schoolbusyellow}\huge Ryan Joo}
\end{center}
\thispagestyle{empty}
\pagecolor{smalt(darkpowderblue)}
\pagebreak

\pagecolor{white}
\thispagestyle{empty}
\

\vfill

\begin{quote}
\textit{The mathematician does not study mathematics because it is useful; he studies it because he delights in it and he delights in it because it is beautiful.}

\begin{flushright}--- Henri Poincar\'{e} (1854--1912)\\
French mathematician and theoretical physicist\end{flushright}
\end{quote}

\vfill

Copyright \copyright \ 2025 by Ryan Joo.

This book is licensed under the terms of the Creative Commons Attribution-NonCommercial 4.0 International License (\url{https://creativecommons.org/licenses/by-nc/4.0}), which permits any noncommercial use, sharing, adaptation, distribution and reproduction in any medium or format, as long as you give appropriate credit to original author and source, provide a link to the Creative Commons license, and indicate if changes were made. The images or other third party material in this book are included in the book's Creative Commons license, unless indicated otherwise in a credit line to the material. If material is not included in the book's Creative Commons license and your intended use is not permitted by statutory regulation or exceeds the permitted use, you will need to obtain permission directly from the copyright holder.

This is (still!) an incomplete draft. Please send corrections and comments to \url{ryanjooruian18@gmail.com}, or pull-request at \url{https://github.com/Ryanjoo18/undergrad-maths}.

Typeset using \LaTeX.

Last updated \today.
\pagebreak

\frontmatter
\section*{Preface}
\ifprelim
The reader is not assumed to have any mathematical prerequisites, although some experience with proofs may be helpful. \textbf{Preliminary topics} such as logic and methods of proofs (\cref{chap:logic-proofs}), and basic set theory (\cref{chap:set-theory}) are covered in \cref{part:prelim}.
\fi

\ifabsalg
\cref{part:abstract-algebra} covers \textbf{abstract algebra}, which follows \cite{dummit-foote,artin}.
\begin{itemize}
\item \cref{chap:groups} introduces groups.
\end{itemize}
\fi

\iflinalg
\cref{part:linear-algebra} covers \textbf{linear algebra}, which follows \cite{axler}.
\begin{itemize}
\item \cref{chap:vector-spaces} introduces vector spaces, subspaces, span, linear independence, bases and dimension.
\item \cref{chap:linear-maps} concerns linear maps and related concepts.
\end{itemize}
\fi

\ifanalysis
\cref{part:real-analysis} covers \textbf{real analysis}, which follows \cite{rudin,apostol}.
\begin{itemize}
\item \cref{chap:number-systems} introduces the real and complex number systems.
\item \cref{chap:basic-topology} covers basic point-set topology, in the context of metric spaces.
\item \cref{chap:num-seq-series} concerns numerical sequences and series, in particular their convergence.
\item \cref{chap:real-analysis_continuity} covers continuity of functions.
\item \cref{chap:differentiation} covers differentiation.
\item \cref{chap:rs-integration} covers Riemann--Stieljes integration.
\item \cref{chap:func-seq-series} covers sequences and series of functions.
\item \cref{chap:special-functions} covers some special functions, most notably power series and the fourier series.
\end{itemize}
\fi

\iftop
\cref{part:topology} covers \textbf{general topology}, which follows \cite{munkres}.
\fi

For ease of reference, important terms are \vocab{coloured} when first defined, and are included in the glossary; less important terms are \emph{italicised} when first defined, and are not included in the glossary.
\pagebreak

\section*{Note on Problem Solving}
Mathematics is about problem solving. In \cite{polya}, George P\'{o}lya outlined the following problem solving cycle.
\begin{enumerate}
\item \textbf{Understand the problem}

Ask yourself the following questions:
\begin{itemize}
\item Do you understand all the words used in stating the problem?
\item Is it possible to satisfy the condition? Is the condition sufficient to determine the unknown? Or is it insufficient? Or redundant? Or contradictory?
\item What are you asked to find or show? Can you restate the problem in your own words?
\item Draw a figure. Introduce suitable notation.
\item Is there enough information to enable you to find a solution?
\end{itemize}

\item \textbf{Devise a plan}

A partial list of heuristics -- good rules of thumb to solve problems -- is included:
\begin{multicols}{2}
\begin{itemize}
\item Guess and check
\item Look for a pattern
\item Make an orderly list
\item Draw a picture
\item Eliminate possibilities
\item Solve a simpler problem
\item Use symmetry
\item Use a model
\item Consider special cases
\item Work backwards
\item Use direct reasoning
\item Use a formula
\item Solve an equation
\item Be ingenious
\end{itemize}
\end{multicols}

\item \textbf{Execute the plan}

This step is usually easier than devising the plan. In general, all you need is care and patience, given that you have the necessary skills. Persist with the plan that you have chosen. If it continues not to work discard it and choose another. Don't be misled, this is how mathematics is done, even by professionals.

\begin{itemize}
\item Carrying out your plan of the solution, check each step. Can you see clearly that the step is correct? Can you prove that it is correct?
\end{itemize}

\item \textbf{Check and expand}

P\'{o}lya mentions that much can be gained by taking the time to reflect and look back at what you have done, what worked, and what didn't. Doing this will enable you to predict what strategy to use to solve future problems.

Look back reviewing and checking your results. Ask yourself the following questions:
\begin{itemize}
\item Can you check the result? Can you check the argument?
\item Can you derive the solution differently? Can you see it at a glance?
\item Can you use the result, or the method, for some other problem?
\end{itemize}
\end{enumerate}

Building on P\'{o}lya's problem solving strategy, Schoenfeld \cite{schoenfeld} came up with the following framework for problem solving, consisting of four components:
\begin{enumerate}
\item \textbf{Cognitive resources}: the body of facts and procedures at one's disposal.
\item \textbf{Heuristics}: `rules of thumb' for making progress in difficult situations.
\item \textbf{Control}: having to do with the efficiency with which individuals utilise the knowledge at their disposal. Sometimes, this is referred to as metacognition, which can be roughly translated as `thinking about one's own thinking'.
\begin{enumerate}
\item These are questions to ask oneself to monitor one's thinking.
\begin{itemize}
    \item What (exactly) am I doing? [Describe it precisely.] Be clear what I am doing NOW. Why am I doing it? [Tell how it fits into the solution.]
    \item Be clear what I am doing in the context of the BIG picture -- the solution. Be clear what I am going to do NEXT.
\end{itemize}

\item Stop and reassess your options when you
\begin{itemize}
    \item cannot answer the questions satisfactorily [probably you are on the wrong track]; OR
    \item are stuck in what you are doing [the track may not be right or it is right but it is at that moment too difficult for you].
\end{itemize}

\item Decide if you want to
\begin{itemize}
    \item carry on with the plan,
    \item abandon the plan, OR
    \item put on hold and try another plan.
\end{itemize}
\end{enumerate}

\item \textbf{Belief system}: one's perspectives regarding the nature of a discipline and how one goes about working on it.
\end{enumerate}
\pagebreak

\tableofcontents
\pagebreak
%\printglossary[type=\acronymtype]

\mainmatter
%%%%%%%%%%%%%%% PRELIMINARY TOPICS
\ifprelim
    \part{Preliminary Topics}\label{part:prelim}
    \chapter{Mathematical Reasoning and Logic}\label{chap:logic-proofs}

\begin{learn}
\item introduce basic logic;
\item introduce common methods of proof.
\end{learn}

\section{Zeroth-order Logic}
A \vocab{proposition} is a sentence which has exactly one truth value, i.e. it is either true or false, but not both and not neither. A proposition is denoted by uppercase letters such as $P$ and $Q$. If the proposition $P$ depends on a variable $x$, it is sometimes helpful to denote it by $P(x)$. 

We can do some algebra on propositions, which include
\begin{enumerate}[label=(\roman*)]
\item \vocab{equivalence}, denoted by $P\iff Q$, which means $P$ and $Q$ are logically equivalent statements;

\item \vocab{conjunction}, denoted by $P\land Q$, which means ``$P$ and $Q$'';

\item \vocab{disjunction}, denoted by $P\lor Q$, which means ``$P$ or $Q$'';

\item \vocab{negation}, denoted by $\lnot P$, which means ``not $P$''.
\end{enumerate}

Here are some useful properties when handling logical statements. You can easily prove all of them using truth tables.
\begin{proposition} \
\begin{enumerate}[label=(\roman*)]
\item Double negation law: $P\iff\lnot(\lnot P)$.
\item Commutative: $P \land Q \iff Q \land P$, $P \lor Q \iff Q \lor P$.
\item Conjunction is associative: $(P\land Q)\land R \iff P\land (Q\land R)$.
\item Disjunction is associative: $(P\lor Q)\lor R \iff P\lor (Q\lor R)$.
\item Conjunction distributes over disjunction: $P\land(Q\lor R) \iff (P\land Q)\lor(P\land Q)$.
\item Disjunction distributes over conjunction: $P\lor(Q\land R) \iff (P\lor Q)\land(P\lor R)$.
\end{enumerate}
\end{proposition}

\begin{proposition}[de Morgan's laws]
\[ \lnot(P \lor Q) \iff (\lnot P \land \lnot Q) \]
\[ \lnot (P\land Q) \iff (\lnot P\lor \lnot Q) \]
\end{proposition}

\subsection{If, only if}
\vocab{Implication} is denoted by $P \implies Q$, which means ``$P$ implies $Q$'', i.e. if $P$ holds then $Q$ also holds. It is equivalent to saying ``If $P$ then $Q$''. $P \implies Q$ is known as a \emph{conditional statement}, where $P$ is known as the \emph{hypothesis} and $Q$ is known as the \emph{conclusion}. The only case when $P \implies Q$ is false is when the hypothesis $P$ is true and the conclusion $Q$ is false.

Statements of this form are probably the most common, although they may sometimes appear quite differently. The following all mean the same thing:
\begin{enumerate}[label=(\roman*)]
\item if $P$ then $Q$;
\item $P$ implies $Q$;
\item $P$ only if $Q$;
\item $P$ is a sufficient condition for $Q$;
\item $Q$ is a necessary condition for $P$.
\end{enumerate}

Given $P\implies Q$,
\begin{itemize}
\item its \vocab{converse} is $Q \implies P$; both are not logically equivalent;
\item its \vocab{inverse} is $\lnot P \implies \lnot Q$, i.e. the hypothesis and conclusion of the statement are both negated; both are not logically equivalent;
\item the \vocab{contrapositive} is $\lnot Q\implies\lnot P$; both are logically equivalent.
\end{itemize}

To prove $P \implies Q$, start by assuming that $P$ holds and try to deduce through some logical steps that $Q$ holds too. Alternatively, start by assuming that $Q$ does not hold and show that $P$ does not hold (that is, we prove the contrapositive).

\subsection{If and only if, iff}
\vocab{Bidirectional implication} is denoted by $P \iff Q$, which means both $P \implies Q$ and $Q \implies P$; $P \iff Q$ is known as a \emph{biconditional statement}. We can read this as ``$P$ if and only if $Q$''. The letters ``iff'' are also commonly used to stand for ``if and only if''.

$P \iff Q$ is true exactly when $P$ and $Q$ have the same truth value.

These statements are usually best thought of separately as ``if'' and ``only if'' statements. To prove $P \iff Q$, prove the statement in both directions, i.e. prove both $P \implies Q$ and $Q \implies P$. Remember to make very clear, both to yourself and in your written proof, which direction you are doing.

\section{First-order Logic}
The \vocab{universal quantifier} is denoted by $\forall$, which means ``for all'' or ``for every''. A \emph{universal statement} takes the form $\forall x\in X, P(x)$.

The \vocab{existential quantifier} is denoted by $\exists$, which means ``there exists''. An \emph{existential statement} takes the form $\exists x\in X, P(x)$, where $X$ is known as the \emph{domain}.

\begin{proposition}[de Morgan's laws]
\[ \lnot \forall x\in X,P(x) \iff \exists x\in X,\lnot P(x) \]
\[ \lnot \exists x\in X,P(x) \iff \forall x\in X,\lnot P(x) \]
\end{proposition}

\begin{exercise}
Negate the statement
\[ \text{for all real numbers } x, \text{ if } x>2, \text{ then } x^2>4 \]
\end{exercise}
\begin{solution}
In logical notation, this statement is $(\forall x \in \RR)[x>2 \implies x^2>4]$.
\begin{align*}
\lnot\{(\forall x \in \RR)[x>2 \implies x^2>4]\} 
&\iff (\exists x \in \RR) \lnot[x>2 \implies x^2>4] \\
&\iff (\exists x \in \RR) \lnot [(x\le2) \lor (x^2>4)] \\
&\iff (\exists x \in \RR) [(x>2) \land (x^2\le4)]
\end{align*}
\end{solution}

\begin{exercise}
Negate surjectivity.
\end{exercise}
\begin{solution}
If $f:X\to Y$ is not surjective, then it means that there exists $y \in Y$ not in the image of $X$, i.e. for all $x$ in $X$ we have $f(x)\neq y$.
\begin{align*}
\lnot \forall y \in Y, \exists x \in X, f(x)=y 
&\iff \exists y \in Y, \lnot (\exists x \in X, f(x)=y) \\
&\iff \exists y \in Y, \forall x \in X, \lnot (f(x)=y) \\
&\iff \exists y \in Y, \forall x \in X, f(x) \neq y
\end{align*}
\end{solution}

To prove a statement of the form $\forall x \in X \suchthat P(x)$, start the proof with ``Let $x \in X$.'' or ``Suppose $x \in X$ is given.'' to address the quantifier with an arbitrary $x$; provided no other assumptions about $x$ are made during the course of proving $P(x)$, this will prove the statement for all $x \in X$. 

To prove a statement of the form $\exists x \in X \suchthat P(x)$, there is not such a clear steer about how to continue: you may need to show the existence of an $x$ with the right properties; you may need to demonstrate logically that such an $x$ must exist because of some earlier assumption, or it may be that you can show constructively how to find one; or you may be able to prove by contradiction, supposing that there is no such $x$ and consequently arriving at some inconsistency.

\begin{remark}
Read from left to right, and as new elements or statements are introduced they are allowed to depend on previously introduced elements but cannot depend on things that are yet to be mentioned.
\end{remark}

\begin{remark}
To avoid confusion, it is a good idea to keep to the convention that the quantifiers come first, before any statement to which they relate.
\end{remark}
\pagebreak

\section{Methods of Proof}
A \vocab{direct proof} of $P \implies Q$ is a series of valid arguments that start with the hypothesis $P$ and end with the conclusion $Q$. It may be that we can start from $P$ and work directly to $Q$, or it may be that we make use of $P$ along the way.

A \vocab{proof by contrapositive} of $P \implies Q$ is to prove instead $\lnot Q \implies \lnot P$.

A \vocab{disproof by counterexample} is to providing a counterexample in order to refute or disprove a conjecture. The counterexample must make the hypothesis a true statement, and the conclusion a false statement. In seeking counterexamples, it is a good idea to keep the cases you consider simple, rather than searching randomly. It is often helpful to consider ``extreme'' cases; for example, something is zero, a set is empty, or a function is constant.

A \vocab{proof by cases} is to first dividing the situation into cases which exhaust all the possibilities, and then show that the statement follows in all cases.

\subsection{Proof by Contradiction}
A \vocab{proof by contradiction} of $P$ involves first supposing $P$ is false, i.e. $\lnot P$; to prove $P \implies Q$ by contradiction, suppose $P\land\lnot Q$. Then show through some logical reasoning that this leads to a contradiction or inconsistency. We may arrive at something that contradicts the hypothesis $P$, or something that contradicts the initial supposition that $Q$ is not true, or we may arrive at something that we know to be universally false.

\begin{exercise}[Irrationality of $\sqrt{2}$]
Prove that $\sqrt{2}$ is irrational.
\end{exercise}
\begin{solution}
We prove by contradiction. Suppose otherwise, that $\sqrt{2}$ is rational. Then $\sqrt{2}=\dfrac{a}{b}$ for some $a,b\in\ZZ,b\neq 0$, $a,b$ coprime.

Squaring both sides gives
\[a^2=2b^2.\]
Since RHS is even, LHS must also be even. Hence it follows that $a$ is even. Let $a=2k$ where $k\in\ZZ$. Substituting $a = 2k$ into the above equation and simplifying it gives us
\[b^2=2k^2.\]
This means that $b^2$ is even, from which follows again that $b$ is even. This contradicts the assumption that $a$ and $b$ coprime, so we are done.
\end{solution}

\begin{exercise}[Euclid]
Prove that there are infinitely many prime numbers.
\end{exercise}

\begin{solution}
Suppose otherwise, that only finitely many prime numbers exist. List them as $p_1,\dots,p_n$. The number $N=p_1p_2\cdots p_n+1$ is divisible by a prime $p$, yet is coprime to $p_1,\dots,p_n$. Therefore, $p$ does not belong to our list of all prime numbers, a contradiction.
\end{solution}

To \vocab{prove uniqueness}, we can either assume $\exists x,y \in S$ such that $P(x) \land P(y)$ is true and show $x=y$, or argue by assuming that $\exists x,y \in S$ are distinct such that $P(x) \land P(y)$, then derive a contradiction. $\exists!$ denotes ``there exists a unique''. To prove uniqueness and existence, we also need to show that $\exists x \in S \suchthat P(x)$ is true.

\subsection{Proof of Existence}
To prove existential statements, we can adopt two approaches:
\begin{enumerate}
\item Constructive proof (direct proof)

To prove statements of the form $\exists x\in X \suchthat P(x)$, find or construct \vocab{a specific example} for $x$. To prove statements of the form $\forall y\in Y,\:\exists x\in X\suchthat P(x,y)$, construct example for $x$ in terms of $y$ (since $x$ is dependent on $y$).

In both cases, you have to justify that your example $x$
\begin{enumerate}
\item belongs to the domain $X$, and
\item satisfies the condition $P$.
\end{enumerate}

\item Non-constructive proof (indirect proof)

Use when specific examples are not easy or not possible to find or construct.
Make arguments why such objects have to exist.
May need to use proof by contradiction.
Use definition, axioms or results that involve existential statements.
\end{enumerate}

\begin{exercise}
Prove that we can find $100$ consecutive positive integers which are all composite numbers.
\end{exercise}

\begin{proof}
We can prove this existential statement via constructive proof.

Our goal is to find integers $n,n+1,n+2,\dots,n+99$, all of which are composite.

Take $n=101!+2$. Then $n$ has a factor of $2$ and hence is composite. Similarly, $n+k=101!+(k+2)$ has a factor $k+2$ and hence is composite for $k=1,2,\dots,99$.

Hence the existential statement is proven.
\end{proof}

\begin{exercise}
Prove that for all rational numbers $p$ and $q$ with $p<q$, there is a rational number $x$ such that $p<x<q$.
\end{exercise}
\begin{proof}
We prove this by construction. Our goal is to find such a rational $x$ in terms of $p$ and $q$.

We take the average. Let $x=\dfrac{p+q}{2}$ which is a rational number.

Since $p<q$, 
\[ x=\frac{p+q}{2}<\frac{q+q}{2}=q \implies x<q \]
Similarly,
\[ x=\frac{p+q}{2}>\frac{p+p}{2}=p \implies p<x \]
Hence we have shown the existence of rational number $x$ such that $p<x<q$.

\begin{remark}
For this type of question, there are two parts to prove: firstly, $x$ satisfies the given statement; secondly, $x$ is within the domain (for this question we do not have to prove $x$ is rational since $\QQ$ is closed under addition).
\end{remark}
\end{proof}

\begin{exercise}
Prove that for all rational numbers $p$ and $q$ with $p<q$, there is an irrational number $r$ such that $p<r<q$.
\end{exercise}
\begin{proof}
We prove this by construction. Similarly, our goal is to find an irrational $r$ in terms of $p$ and $q$.

Note that we cannot simply take $r=\dfrac{p+q}{2}$; a simple counterexample is the case $p=-1,q=1$ where $r=0$ is clearly not irrational.

Since $p$ lies in between $p$ and $q$, let $r=p+c$ where $0<c<q-p$. Since $c<q-p$, we have $c=\dfrac{q-p}{k}$ for some $k>1$; to make $c$ irrational, we take $k$ to be irrational.

Take $r=p+\dfrac{q-p}{\sqrt{2}}$. We need to show $r$ is irrational and $p<r<q$.

\textbf{Part 1:} $p<r<q$

Since $q<p$, $r=p+\text{(positive number)}>p$. On the other hand, $\dfrac{q-p}{\sqrt{2}}<q-p$ so $r<p+(q-p)=q$.

\textbf{Part 2:} $r$ is irrational

We prove by contradiction. Suppose $r$ is rational. We have $\sqrt{2}=\dfrac{q-p}{r-p}$. Since $p,q,r$ are all rational (and $r-p\neq0$), RHS is rational. This implies that LHS is rational, i.e. $\sqrt{2}$ is rational, a contradiction.
\end{proof}

Non-constructive proof:
\begin{exercise}
Prove that every integer greater than $1$ is divisible by a prime.
\end{exercise}

\begin{proof}
If $n$ is prime, then we are done as $n\mid n$.

If $n$ is not prime, then $n$ is composite. So $n$ has a divisor $d_1$ such that $1<d_1<n$. If $d_1$ is prime then we are done as $d_1\mid n$. If $d_1$ is not prime then $d_1$ is composite, has divisor $d_2$ such that $1<d_2<n$.

If $d_2$ is prime, then we are done as $d_2\mid d_1$ and $d_1\mid n$ imply $d_2\mid n$. If $d_2$ is not prime then $d_2$ is composite, has divisor $d_3$ such that $1<d_3<d_2$.

Continuing in this manner after $k$ times, we will get
\[ 1<d_k<d_{k-1}<\cdots<d_2<d_1<n \]
where $d_i\mid n$ for all $i$.

Since there can only be a finite number of $d_i$'s between $1$ and $n$, this process must stop after finite steps. On the other hand, the process will stop only if there is a $d_i$ which is a prime. Hence we conclude that there must be a divisor $d_i$ of $n$ that is prime.
\end{proof}

\begin{remark}
This proof is also known as \emph{proof by infinite descent}, a method which relies on the well-ordering principle on $\NN$.
\end{remark}

\begin{exercise}
Prove that the equation $x^2+y^2=3z^2$ has no solutions $(x,y,z)$ in integers where $z\neq0$.
\end{exercise}

\begin{proof}
Suppose we have a solution $(x,y,z)$. Without loss of generality, we may assume that $z>0$. By the least integer principle, we may also assume that our solution has $z$ minimal. Taking remainders modulo $3$, we see that
\[ x^2+y^2\equiv0\pmod3 \]
Recalling that squares may only be congruent to $0$ or $1$ modulo $3$, we conclude that
\[ x^2\equiv y^2\equiv 0 \implies x \equiv y \equiv 0 \pmod 3 \]
Writing $x=3a$ and $y=3b$ we obtain
\[ 9a^2+9b^2=3z^2 \implies 3(a^2+b^2)=z^2 \implies 3\mid z^2 \implies 3\mid z \]
Now let $z=3c$ and cancel $3$'s to obtain
\[ a^2+b^2=3c^2. \]
We have therefore constructed another solution $(a,b,c)=\brac{\frac{x}{3},\frac{y}{3},\frac{z}{3}}$ to the original equation. However $0<c<z$ contradicts the minimality of $z$.
\end{proof}

\subsection{Proof by Mathematical Induction}
Induction is an extremely powerful method of proof used throughout mathematics. It deals with infinite families of statements which come in the form of lists. The idea behind induction is in showing how each statement follows from the previous one on the list -- all that remains is to kick off this logical chain reaction from some starting point.

We shall assume that $\NN$ satisfies the \emph{well-ordering principle}: every nonempty $S\subset\NN$ has a least element; that is, there exists $m\in S$ such that $m\le k$ for all $k\in S$.

\begin{remark}
The well-ordering principle does not hold for $\ZZ$, $\QQ$, and $\RR$.
\end{remark}

\begin{lemma}
Let $S\subset\NN$. If
\begin{enumerate}[label=(\roman*)]
\item $1\in S$
\item $k\in S\implies k+1\in S$
\end{enumerate}
then $S=\NN$.
\end{lemma}

\begin{proof}
If $S=\NN$ then we are done. Now suppose $S\neq\NN$. Then $\NN\setminus S$ is not empty. By the well-ordering principle, $\NN\setminus S$ has a least element $p$. Since $1\in S$, we must have $p>1$. By (ii), $p=(p-1)+1\in S$. But this contradicts $p\in\NN\setminus S$.
\end{proof}

\begin{theorem}[Principle of mathematical induction]\label{thrm:pmi}
Let $P(n)$ be a family of statements indexed by $\NN$. Suppose that 
\begin{enumerate}[label=(\roman*)]
\item $P(1)$ is true;
\item for all $k\in\NN$, $P(k)\implies P(k+1)$.
\end{enumerate}
Then $P(n)$ is true for all $n\in\NN$.
\end{theorem}

(i) is known as the \textbf{base case}, (ii) is known as the \textbf{inductive step}. Using logic notation, the above can be written as
\[ \{P(1) \land (\forall n \in \NN) [P(k) \implies P(k+1)]\} \implies (\forall n \in \NN)P(n) \]

\begin{proof}
Apply the above lemma to the set $S=\{n\in\NN\mid P(n)\text{ is true}\}$.
\end{proof}

\begin{exercise}
Prove that for any $n\in\NN$,
\[\sum_{i=1}^n i=\frac{n(n+1)}{2}.\]
\end{exercise}

\begin{proof}
Let $\displaystyle P(n):\sum_{i=1}^n i=\frac{n(n+1)}{2}$.

Clearly $P(1)$ holds. Now suppose $P(k)$ holds for some $k\in\NN$, $k\ge1$; that is,
\[\sum_{i=1}^k i=\frac{k(k+1)}{2}.\]
Adding $k+1$ to both sides,
\begin{align*}
\sum_{i=1}^{k+1} i&=\frac{k(k+1)}{2}+(k+1)\\
&=\frac{(k+1)(k+2)}{2}\\
&=\frac{(k+1)[(k+1)+1]}{2}
\end{align*}
thus $P(k+1)$ is true.

Since $P(1)$ true and $P(k)\implies P(k+1)$ for all $k\in\NN$, $k\ge1$,\\
$P(n)$ is true for all $n\in\NN$.
\end{proof}

\begin{exercise}[Bernoulli's inequality]
Let $x\in\RR$, $x>-1$. Let $n\in\ZZ^+$. Then
\[(1+x)^n\ge1+nx.\]
\end{exercise}

\begin{proof}
We prove by induction on $n$. Fix $x>-1$. Let $P(n):(1+x)^n\ge1+nx$.

The base case $P(1)$ is clear.

Suppose that $P(k)$ is true for some $k\in\ZZ^+$, $k\ge1$. That is, $(1+x)^k\ge1+kx$. Note that $1+x>0$, and $kx^2\ge0$ (since $k>0$ and $x^2\ge0$). Then
\begin{align*}
(1+x)^{k+1}&=(1+x)(1+x)^k\\
&\ge(1+x)(1+kx)\quad\text{[induction hypothesis]}\\
&=1+(k+1)x+kx^2\\
&\ge1+(k+1)x\quad[\because kx^2\ge0]
\end{align*}
so $P(k+1)$ is true. Hence by induction, the result holds.
\end{proof}

A corollary of induction is if the family of statements holds for $n\ge N$, rather than necessarily $n\ge0$:

\begin{corollary}
Let $P(n)$ be a family of statements indexed by integers $n \ge N$ for $N\in\ZZ$. Suppose that 
\begin{enumerate}[label=(\roman*)]
\item $P(N)$ is true;
\item for all $k \ge N$, $P(k) \implies P(k+1)$. 
\end{enumerate}
Then $P(n)$ is true for all $n \ge N$.
\end{corollary}

\begin{proof}
Apply \cref{thrm:pmi} to the statement $Q(n)=P(n+N)$ for $n\in\NN$.
\end{proof}

Another variant on induction is when the inductive step relies on some earlier case(s) but not necessarily the immediately previous case.

\begin{theorem}[Strong induction]
Let $P(n)$ be a family of statements indexed by $\NN$. Suppose that
\begin{enumerate}[label=(\roman*)]
\item $P(1)$ is true;
\item for all $k\in\NN$, $P(1)\land\cdots\land P(k)\implies P(k+1)$.
\end{enumerate}
Then $P(n)$ is true for all $n\in\NN$.
\end{theorem}

\begin{proof}
Let $Q(n)$ be the statement ``$P(k)$ holds for $k=0,1,\dots,n$''. Then the conditions for the strong form are equivalent to (i) $Q(0)$ holds and (ii) for $n\in\NN$, $Q(n)\implies Q(n+1)$. It follows by induction that $Q(n)$ holds for all $n\in\NN$, and hence $P(n)$ holds for all $n$.
\end{proof}

\begin{exercise}[Fundamental theorem of arithmetic]
Prove that every natural number greater than $1$ may be expressed as a product of one or more prime numbers.
\end{exercise}

\begin{proof}
Let $P(n)$: $n$ may be expressed as a product of prime numbers. 

Clearly $P(2)$ holds, since $2$ is itself prime. 

Let $n\ge 2$ be a natural number and suppose that $P(m)$ holds for all $m<n$.

\begin{itemize}
\item If $n$ is prime then it is trivially the product of the single prime number $n$. 

\item If $n$ is not prime, then there must exist some $r, s > 1$ such that $n = rs$. By the inductive hypothesis, each of $r$ and $s$ can be written as a product of primes, and therefore $n = rs$ is also a product of primes.
\end{itemize}

In both cases, $P(n)$ holds. Hence by strong induction, $P(n)$ is true for all $n\in\NN$.
\end{proof}

The following is also another variant on induction.

\begin{theorem}[Cauchy induction]
Let $P(n)$ be a family of statements indexed by $\NN_{\ge2}$. Suppose that
\begin{enumerate}[label=(\roman*)]
\item $P(2)$ is true;
\item for all $k\in\NN$, $P(k)\implies P(2k)$ and $P(k)\implies (k-1)$.
\end{enumerate}
Then $P(n)$ is true for all $n\in\NN_{\ge2}$.
\end{theorem}

\begin{exercise}[AM--GM inequality]
Given $n\in\NN$, prove that for positive reals $a_1,a_2,dots,a_n$,
\[\frac{a_1+a_2+\cdots+a_n}{n}\ge\sqrt[n]{a_1a_2\cdots a_n}.\]
\end{exercise}

\begin{proof}
Let $\displaystyle P(n):\frac{a_1+a_2+\cdots+a_n}{n}\ge\sqrt[n]{a_1a_2\cdots a_n}$.

Base case $P(2)$ is true because\[\frac{a_1+a_2}{2}\ge\sqrt{a_1a_2} \iff (a_1+a_2)^2\ge 4a_1a_2 \iff (a_1-a_2)^2\ge0\]

Next we show that $P(n)\implies P(2n)$
\[\frac{a_1+a_2+\cdots+a_{2n}}{2n}=\frac{\frac{a_1+a_2+\cdots+a_n}{n}+\frac{a_{n+1}+a_{n+2}+\cdots+a_{2n}}{n}}{2}\]\[\frac{\frac{a_1+a_2+\cdots+a_n}{n}+\frac{a_{n+1}+a_{n+2}+\cdots+a_{2n}}{n}}{2}\ge\frac{\sqrt[n]{a_1a_2\cdots a_n}+\sqrt[n]{a_{n+1}a_{n+2}\cdots a_{2n}}}{2}\]\[\frac{\sqrt[n]{a_1a_2\cdots a_n}+\sqrt[n]{a_{n+1}a_{n+2}\cdots a_{2n}}}{2}\ge\sqrt{\sqrt[n]{a_1a_2\cdots a_n}\sqrt[n]{a_{n+1}a_{n+2}\cdots a_{2n}}}\]\[\sqrt{\sqrt[n]{a_1a_2\cdots a_n}\sqrt[n]{a_{n+1}a_{n+2}\cdots a_{2n}}}=\sqrt[2n]{a_1a_2\cdots a_{2n}}\]
The first inequality follows from $n$-variable AM--GM, which is true by assumption, and the second inequality follows from 2-variable AM--GM, which is proven above.

Finally we show that $P(n)\implies P(n-1)$. By $n$-variable AM--GM, $\frac{a_1+a_2+\cdots+a_n}{n}\ge\sqrt[n]{a_1a_2\cdots a_n}$ Let $a_n=\frac{a_1+a_2+\cdots+a_{n-1}}{n-1}$ Then we have\[\frac{a_1+a_2+\cdots+a_{n-1}+\frac{a_1+a_2+\cdots+a_{n-1}}{n-1}}{n}=\frac{a_1+a_2+\cdots+a_{n-1}}{n-1}\]So,\[\frac{a_1+a_2+\cdots+a_{n-1}}{n-1}\ge\sqrt[n]{a_1a_2\cdots a_{n-1}\cdot \frac{a_1+a_2+\cdots+a_{n-1}}{n-1}}\]\[\Rightarrow\left(\frac{a_1+a_2+\cdots+a_{n-1}}{n-1}\right)^n\ge a_1a_2\cdots a_{n-1}\cdot \frac{a_1+a_2+\cdots+a_{n-1}}{n-1}\]\[\Rightarrow\left(\frac{a_1+a_2+\cdots+a_{n-1}}{n-1}\right)^{n-1}\ge a_1a_2\cdots a_{n-1}\]\[\Rightarrow \frac{a_1+a_2+\cdots+a_{n-1}}{n-1}\ge\sqrt[n-1]{a_1a_2\cdots a_{n-1}}\]
By Cauchy induction, this proves the AM--GM inequality for $n$ variables.
\end{proof}

\subsection{Pigeonhole Principle}
\begin{theorem}[Pigeonhole principle]
If $kn+1$ objects are distributed among $n$ boxes, one of the boxes will contain at least $k+1$ objects.
\end{theorem}

\begin{exercise}[IMO 1972]
Prove that every set of 10 two-digit integer numbers has two disjoint subsets with the same sum of elements.
\end{exercise}

\begin{solution}
Let $S$ be the set of $10$ numbers. It has $2^{10}-2=1022$ subsets that differ from both $S$ and the empty set. They are the ``pigeons''.

If $A\subset S$, the sum of elements of $A$ cannot exceed $91+92+\cdots+99=855$. The numbers between 1 and 855, which are all possible sums, are the ``holes''.

Because the number of ``pigeons'' exceeds the number of ``holes'', there will be two ``pigeons'' in the same ``hole''. Specifically, there will be two subsets with the same sum of elements. Deleting the common elements, we obtain two disjoint sets with the same sum of elements.
\end{solution}

\begin{exercise}[Putnam 2006]
Prove that for every set $X=\{x_1,x_2,\dots,x_n\}$ of $n$ real numbers, there exists a nonempty subset $S$ of $X$ and an integer $m$ such that
\[\absolute{m+\sum_{x\in S}s}\le\frac{1}{n+1}.\]
\end{exercise}

\begin{solution}
Recall that the fractional part of a real number $x$ is $x-\floor{x}$. Let us look at the fractional parts of the numbers $x_1,x_1+x_2,\dots,x_1+x_2+\cdots+x_n$. If any of them is either in the interval $\sqbrac{0,\frac{1}{n+1}}$ or $\sqbrac{\frac{n}{n+1},1}$, then we are done. If not, we consider these $n$ numbers as the ``pigeons'' and the $n-1$ intervals $\sqbrac{\frac{1}{n+1},\frac{2}{n+1}},\sqbrac{\frac{2}{n+1},\frac{3}{n+1}},\dots,\sqbrac{\frac{n-1}{n+1},\frac{n}{n+1}}$ as the ``holes''. By the pigeonhole principle, two of these sums, say $x_1+x_2+\cdots+x_k$ and $x_1+x_2+\cdots+x_{k+m}$, belong to the same interval. But then their difference $x_{k+1}+\cdots+x_{k+m}$ lies within a distance of $\frac{1}{n+1}$ of an integer, and we are done.
\end{solution}
\pagebreak

\section*{Exercises}
\begin{prbm}
Use the Unique Factorisation Theorem to prove that, if a positive integer $n$ is not a perfect square, then $\sqrt{n}$ is irrational.

[The Unique Factorisation Theorem states that every integer $n>1$ has a unique standard factored form, i.e. there is exactly one way to express $n=p_1^{k_1}p_2^{k_2}\cdots p_t^{k_t}$ where $p_1<p_2<\cdots<p_t$ are distinct primes and $k_1,k_2,\dots,k_t$ are some positive integers.]
\end{prbm}

\begin{proof}
Prove by contradiction. Suppose $n$ is not a perfect square and $\sqrt{n}$ is rational. Then $\sqrt{n}=\frac{a}{b}$ for some $a,b\in\ZZ$. Squaring both sides and clearing denominator gives 
\begin{equation*}\tag{$\ast$}
nb^2=a^2.
\end{equation*}

Consider the standard factored forms of $n$, $a$ and $b$:
\[ n=p_1^{k_1}p_2^{k_2}\cdots p_t^{k_t} \]
\[ a=q_1^{e_1}q_2^{e_2}\cdots q_u^{e_u} \implies a^2=q_1^{2e_1}q_2^{2e_2}\cdots q_u^{2e_u} \]
\[ b=r_1^{f_1}r_2^{f_2}\cdots r_v^{f_v} \implies b^2=r_1^{2f_1}r_2^{2f_2}\cdots r_v^{2f_v} \]
i.e. the powers of primes in the standard factored form of $a^2$ and $b^2$ are all even integers. 

This means the powers $k_i$ of primes $p_i$ in the standard factored form of $n$ are also even by Unique Factorisation Theorem. Note that all $p_i$ appear in the standard factored form of $a^2$ with even power $2c_i$, because of $(\ast)$. By UFT, $p_i$ must also appear in the standard factored form of $nb^2$ with the same even power $2c_i$.

If $p_i\nmid b$, then $k_i=2c_i$ which is even. If $p_i\mid b$, then $p_i$ will appear in $b^2$ with even power $2d_i$. So $k_i+2d_i=2c_i$, and hence $k_i=2(c_i-d_i)$, which is again even.

Hence $n=p_1^{k_1}p_2^{k_2}\cdots p_t^{k_t}=\brac{p_1^{\frac{k_1}{2}}p_2^{\frac{k_2}{2}}\cdots p_t^{\frac{k_t}{2}}}^2$.

Since $\frac{k_i}{2}$ are all integers, $p_1^{\frac{k_1}{2}}p_2^{\frac{k_2}{2}}\cdots p_t^{\frac{k_t}{2}}$ is an integer and $n$ is a perfect square. This contradicts the given hypothesis that $n$ is not a perfect square.
\end{proof}

\begin{prbm}
Prove that for every pair of irrational numbers $p$ and $q$ such that $p<q$, there is an irrational $x$ such that $p<x<q$.
\end{prbm}

\begin{proof}
Consider the average of $p$ and $q$: $p<\dfrac{p+q}{2}<q$.

If $\dfrac{p+q}{2}$ is irrational, take $x=\dfrac{p+q}{2}$ and we are done.

If $\dfrac{p+q}{2}$ is rational, call it $r$, take the average of $p$ and $r$: $p<\dfrac{p+r}{2}<r<q$. Since $p$ is irrational and $r$ is rational, $\dfrac{p+r}{2}$ is irrational. In this case, we take $x=\dfrac{3p+q}{4}$.
\end{proof}

\begin{prbm}
Given $n$ real numbers $a_1,a_2,\dots,a_n$. Show that there exists an $a_i$ ($1\le i\le n$) such that $a_i$ is greater than or equal to the mean (average) value of the $n$ numbers.
\end{prbm}

\begin{proof}
Prove by contradiction.

Let $\bar{a}$ denote the mean value of the $n$ given numbers. Suppose $a_i<\bar{a}$ for all $a_i$. Then
\[ \bar{a}=\frac{a_1+a_2+\cdots+a_n}{n}<\frac{\bar{a}+\bar{a}+\cdots+\bar{a}}{n}=\frac{n\bar{a}}{n}=\bar{a}. \]
We derive $\bar{a}<\bar{a}$, which is a contradiction.

Hence there must be some $a_i$ such that $a_i>\bar{a}$.
\end{proof}

\begin{prbm}
Prove that the following statement is false: there is an irrational number $a$ such that for all irrational number $b$, $ab$ is rational.
\end{prbm}

\textbf{Thought process:} prove the negation of the statement: for every irrational number $a$, there is an irrational number $b$ such that $ab$ is irrational.

\textbf{Proving technique:} constructive proof (note that we can consider multiple cases and construct more than one $b$)

\begin{proof}
Given an irrational number $a$, let us consider $\dfrac{\sqrt{2}}{a}$.

\textbf{Case 1:} $\dfrac{\sqrt{2}}{a}$ is irrational.

Take $b=\dfrac{\sqrt{2}}{a}$. Then $ab=\sqrt{2}$ which is irrational.

\textbf{Case 2:} $\dfrac{\sqrt{2}}{a}$ is rational.

Then the reciprocal $\dfrac{a}{\sqrt{2}}$. Since $\sqrt{6}$ is irrational, the product $\brac{\dfrac{a}{\sqrt{2}}}\sqrt{6}=a\sqrt{3}$ is irrational. Take $b=\sqrt{3}$, which is irrational. Then $ab=a\sqrt{3}$ which is irrational.
\end{proof}

\begin{prbm}
Prove that there are infinitely many prime numbers that are congruent to $3$ modulo $4$.
\end{prbm}

\begin{proof}
Prove by contradiction.

Suppose there are only finitely many primes that are congruent to $3$ modulo $4$. Let $p_1,p_2,\dots,p_m$ be the list of all the primes that are congruent to $3$ modulo $4$.

We construct an integer $M$ by $M=(p_1p_2\cdots p_m)^2+2$.

We have the following observation:
\begin{enumerate}[label=(\roman*)]
\item  $M\equiv 3 \mod 4$.
\item Every $p_i$ divides $M-2$.
\item None of the $p_i$ divides $M$. [Otherwise, together with (ii), this will imply $p_i$ divides $2$, which is impossible.]
\item $M$ is not a prime number. [Otherwise, by (i), $M$ is a prime number congruent to $3$ modulo $4$. But $M\neq p_i$ for all $1\le i\le m$. This contradicts the assumption that $p_1,p_2,\dots,p_m$ are all the prime numbers congruent to $3$ modulo $4$.]
\end{enumerate}

From the above discussion, we know that $M$ is a composite number by (iv). So it has a prime factorization $M=q_1q_2\cdots q_k$.

Since $M$ is odd, all these prime factors $q_j$ must be odd, and hence $q_j$ must be congruent to either $1$ or $3$ modulo $4$.

By (iii), $q_j$ cannot be any of the $p_i$. So all $q_j$ must be congruent to $1$ modulo $4$. Then $M$, which is the product of $q_j$, must also be congruent to $1$ modulo $4$.

This contradicts (i) that $M$ is congruent to $3$ modulo $4$.

Hence we conclude that there must be infinitely many primes that are congruent to $3$ modulo $4$.
\end{proof}

\begin{prbm}
Prove that, for any positive integer $n$, there is a perfect square $m^2$ ($m$ is an integer) such that $n\le m^2\le 2n$.
\end{prbm}

\begin{proof}
Prove by contradiction.

Suppose otherwise, that $n>m^2$ and $(m+1)^2>2n$ so that there is no square between $n$ and $2n$, then
\[ (m+1)^2>2n>2m^2. \]
Since we are dealing with integers and the inequalities are strict, we get
\[ (m+1)^2\ge2m^2+2 \]
which simplifies to
\[ 0\ge m^2-2m+1=(m-1)^2 \]
The only value for which this is possible is $m=1$, but you can eliminate that easily enough.
\end{proof}

\begin{prbm}
Prove that for every positive integer $n\ge4$,
\[ n!>2^n. \]
\end{prbm}

\begin{proof}
Let $P(n):n!>2^n$

\textbf{Base case:} $P(4)$

LHS: $4!=4\times3\times2\times1=24$, RHS: $2^4=16<24$

So $P(4)$ is true.

\textbf{Inductive step:} $P(k) \implies P(k+1)$ for all $k\in\NN_{\ge4}$
\begin{align*}
k! &> 2^k \\
(k+1)k! &> 2^k(k+1) \\
&> 2^k2 \quad \text{since from $k\ge4$, $k+1\ge5>2$} \\
&= 2^{k+1}
\end{align*}
hence proven $P(k) \implies P(k+1)$ for integers $k\ge4$.

By PMI, we have proven $P(n)$ for all integers $n\ge4$.
\end{proof}

\begin{prbm}
Prove by mathematical induction, for $n\ge2$,
\[ \sqrt[n]{n}<2-\frac{1}{n}. \]
\end{prbm}

\begin{proof}
Let $P(n):\sqrt[n]{n}<2 - \dfrac{1}{n}$ for $n \ge 2$.

\textbf{Base case:} $P(2)$

When $n=2$, $\sqrt{2}=1.41\dots<2-\dfrac{1}{2}=1.5$ which is true. Hence $P(2)$ is true.

\textbf{Inductive step:} $P(k)\implies P(k+1)$ for all $k\in\NN_{\ge2}$

Assume $P(k)$ is true for $k \ge 2, k \in \NN$, i.e.
\[ \sqrt[k]{k}<2 - \dfrac{1}{k} \implies k<\brac{2-\frac{1}{k}}^k \]

We want to prove that $P(k+1)$ is true, i.e.
\[ k+1<\brac{2-\frac{1}{k+1}}^{k+1} \]

Since $k>2$, we have 
\begin{align*}
\brac{2-\frac{1}{k+1}}^{k+1}
&> \brac{2-\frac{1}{k}}^{k+1} \quad \because k>2 \\
&= \brac{2-\frac{1}{k}}^k\brac{2-\frac{1}{k}} \\
&> k\brac{2-\frac{1}{k}} \quad \text{[by inductive hypothesis]} \\
&= 2k-1=k+k-1 > k-1 \because k>2
\end{align*}
Hence $P(k+1)$ is true.

Since $P(2)$ is true and $P(k)\implies P(k+1)$, by mathematical induction $P(n)$ is true.
\end{proof}

\begin{prbm}
Prove that for all integers $n \ge 3$, 
\[ \brac{1+\frac{1}{n}}^n<n \]
\end{prbm}

\begin{proof}
\textbf{Base case:} $P(3)$

On the LHS, $\brac{1+\dfrac{1}{3}}^3=\dfrac{64}{27}=2\dfrac{10}{27}<3$. Hence $P(3)$ is true.

\textbf{Inductive step:} $P(k)\implies P(k+1)$ for all $k\in\NN_{\ge3}$

Our inductive hypothesis is
\[ \brac{1+\frac{1}{k}}^k<k \]
Multiplying both sides by $\brac{1+\dfrac{1}{k}}$ (to get a $k+1$ in the power),
\[ \brac{1+\frac{1}{k}}^k\brac{1+\frac{1}{k}}=\brac{1+\frac{1}{k}}^{k+1}<k\brac{1+\frac{1}{k}}=k+1  \]
Since $k<k+1 \iff \dfrac{1}{k}>\dfrac{1}{k+1}$, 
\[ \brac{1+\frac{1}{k}}^{k+1} > \brac{1+\frac{1}{k+1}}^{k+1} \]
The rest of the proof follows easily.
\end{proof}

A sequence of integers $F_i$, where integer $1\le i\le n$, is called the \emph{Fibonacci sequence} if and only if it is defined recursively by $F_1=1$, $F_2=1$, $F_n=F_{n-1}+F_{n-2}$ for $n>2$.

\begin{prbm}
Let $a_i$ where integer $1\le i\le n$ be a sequence of integers defined recursively by initial conditions $a_1=1$, $a_2=1$, $a_3=3$ and the recurrence relation $a_n=a_{n-1}+a_{n-2}+a_{n-3}$ for $n>3$.

For all $n\in\NN$, prove that
\[ a_n\le2^{n-1}. \]
\end{prbm}

\begin{proof}
Let $P(n):a_n\le2^{n-1}$.

Given the recurrence relation, it could be possible to use $P(k),P(k+1),P(k+2)$ to prove $P(k+3)$ for all $k\in\NN$.

\textbf{Base case:} $P(1),P(2),P(3)$

$P(1):a_1=1\le2^{1-1}=1$ is true.

$P(2):a_2=1\le2^{2-1}=2$ is true.

$P(3):a_3=3\le2^{3-1}=4$ is true.

\textbf{Inductive step:} $P(k)\land P(k+1)\land P(k+2)\implies P(k+3)$ for all $k\in\NN$

By inductive hypothesis, for $k\in\NN$ we have $a_k\le2^k, a_{k+1}\le2^{k+1}, a_{k+2}\le2^{k+2}$.
\begin{align*}
a_{k+3} &= a_k+a_{k+1}+a_{k+2} \quad \text{[start from recurrence relation]} \\
&\le 2^k+2^{k+1}+2^{k+2} \quad \text{[use inductive hypothesis]} \\
&= 2^k(1+2+2^2) \\
&< 2^k(2^3) \quad \text{[approximation, since $1+2+2^2<2^3$]} \\
&= 2^{k+3}
\end{align*}
which is precisely $P(k+3):a_{k+3}\le2^{k+3}$.
\end{proof}

\begin{prbm}
For $m,n\in\NN$, prove that
\[ F_{n+m+1}=F_nF_m+F_{n+1}F_{m+1}. \]
\end{prbm}

\begin{proof}
For $n\in\NN$, take $P(n):F_{n+m+1}=F_nF_m+F_{n+1}F_{m+1}$ for all $m\in\NN$ in the cases $k=n$ and $k=n+1$.

So we are using induction to progress through $n$ and dealing with $m$ simultaneously at each stage. 

To verify $P(0)$, we note that
\[ F_{m+1}=F_0F_m+F_1F_{m+1} \]
and
\[ F_{m+2}=F_1F_m+F_2F_{m+1} \]
for all $m$, as $F_0=0$ and $F_1=F_2=1$.

For the inductive step we assume $P(n)$, i.e. that for all $m\in\NN$,
\begin{align*}
F_{n+m+1}&=F_nF_m+F_{n+1}F_{m+1},\\
F_{n+m+2}&=F_{n+1}F_m+F_{n+2}F_{m+1}.
\end{align*}

Then
\begin{align*}
F_{n+m+3}
&=F_{n+m+2}+F_{n+m+1}\\
&=F_n F_m+F_{n+1}F_{m+1}+F_{n+1}F_m+F_{n+2}F_{m+1}\\
&=(F_n+F_{n+1})Fm+(F_{n+1}+F_{n+2})F_{m+1}\\
&=F_{n+2}F_m+F_{n+3}F_{m+1}
\end{align*}
thus $P(n+1)$ is true, for all $m\in\NN$.
\end{proof}
    \chapter{Set Theory}\label{chap:set-theory}

\begin{summary}
\item Basic definitions relating to sets (excluding detailed axiomatic discussions).
\item Relations and related concepts including binary relation, partial order, total order, well order, equivalence relations, equivalence relations, equivalence class, quotient set, partition.
\item Functions, injectivity, surjectivity, bijectivity, composition, invertibility.
\end{summary}

\section{Basics}
\subsection{Definitions and Notations}
A \vocab{set}\index{set} $S$ can be loosely defined as a collection of objects\footnote{\emph{Russell's paradox}, after the mathematician and philosopher Bertrand Russell (1872--1970), provides a warning as to the looseness of our definition of a set. Suppose $H$ is the collection of sets that are not elements of themselves; that is,
\[H=\{S\mid S\notin S\}.\]

The problem arises when we ask the question of whether or not $H$ is itself in $H$? On one hand, if $H\notin H$ then $H$ meets the precise criterion for being in $H$ and so $H\in H$, a contradiction. On the other hand, if $H\in H$ then by the property required for this to be the case, $H\notin H$, another contradiction. Thus we have a paradox: $H$ is neither in $H$, nor not in $H$.

The modern resolution of Russell's paradox is that we have taken too naive an understanding of ``collection'', and that Russell's ``set'' $H$ is in fact not a set. It does not fit within axiomatic set theory (which relies on the so-called ZF axioms), and so the question of whether or not $H$ is in $H$ simply doesn't make sense.}. For a set $S$, we write $x \in S$ to mean that $x$ is an \vocab{element}\index{set!element} of $S$, and $x \notin S$ if otherwise. 

To describe a set, one can list its elements explicitly. A set can also be defined in terms of some property $P(x)$ that the elements $x \in S$ satisfy, denoted by the following set builder notation:
\[\{x\in S \mid P(x)\}\]

Some basic sets (of numbers) you should be familiar with:
\begin{itemize}
\item $\NN=\{1,2,3,\dots\}$ denotes the natural numbers (non-negative integers).
\item $\ZZ=\{\dots,-2,-1,0,1,2,\dots\}$ denotes the integers.
\item $\QQ=\crbrac{\frac{p}{q}\:\big|\:p,q\in\ZZ, q\neq0}$ denotes the rational numbers.
\item $\RR$ denotes the real numbers (the construction of which using Dedekind cuts will be discussed in \cref{chap:number-systems}).
\item $\CC=\{x+yi \mid x,y\in\RR\}$ denotes the complex numbers.
\end{itemize}

We have that $\NN\subset\ZZ\subset\QQ\subset\RR\subset\CC$.

The \vocab{empty set}\index{set!empty set} is the set with no elements, denoted by $\emptyset$.

$A$ is a \vocab{subset}\index{set!subset} of $B$ if every element of $A$ is in $B$, denoted by $A\subset B$:
\[A\subset B\iff(\forall x)(x\in A \implies x\in B)\]
We denote $A\subsetneq B$ to explicitly mean that $A\subset B$ and $A\neq B$; we call $A$ a \emph{proper subset} of $B$.

\begin{lemma}[$\subset$ is transitive]
If $A \subset B$ and $B \subset C$, then $A \subset C$.
\end{lemma}

\begin{proof}
Let $x\in A$. 
Since $A \subset B$ and $x\in A$, $x\in B$. 
Since $B \subset C$ and $x\in B$, $x\in C$. 
Hence $A \subset C$.
\end{proof}

$A$ and $B$ are \vocab{equal} if and only if they contain the same elements, denoted by $A=B$. 

\begin{lemma}[Double inclusion]
Let $A\subset S$ and $B\subset S$. Then
\[A=B\iff (A\subset B)\land(B\subset A)\]
\end{lemma}

\begin{proof}
We have 
\begin{align*}
A = B &\iff (\forall x)[x \in A \iff x \in B] \\
&\iff (\forall x)[(x \in A \implies x \in B) \land (x \in B \implies x \in A)] \\
&\iff \{(\forall x)[x \in A \implies x \in B]\} \land {(\forall x)[x \in B \implies x \in A)]} \\
&\iff (A \subset B) \land (B \subset A)
\end{align*}
\end{proof}

\begin{remark}
Double inclusion is a useful tool to prove that two sets are equal.
\end{remark}

Some frequently occurring subsets of $\RR$ are known as \vocab{intervals}\index{set!interval}, which can be visualised as sections of the real line. We define \emph{bounded intervals}
\begin{align*}
(a,b)&=\{x\in\RR\mid a<x<b\},\\
[a,b]&=\{x\in\RR\mid a\le x\le b\},\\
[a,b)&=\{x\in\RR\mid a\le x<b\},\\
(a,b]&=\{x\in\RR\mid a<x\le b\},
\end{align*}
and \emph{unbounded intervals}
\begin{align*}
(a,\infty)&=\{x\in\RR\mid a<x\},\\
[a,\infty)&=\{x\in\RR\mid a\le x\},\\
(-\infty,a)&=\{x\in\RR\mid x<a\},\\
(\infty,a]&=\{x\in\RR\mid x\le a\}.
\end{align*}
An interval of the first type $(a,b)$ is called an \emph{open interval}; an interval of the second type $[a,b]$ is called a \emph{closed interval}. Note that if $a=b$, then $[a,b]=\{a\}$, while $(a,b)=[a,b)=(a,b]=\emptyset$.

The \vocab{power set}\index{set!power set} $\mathcal{P}(A)$ of $A$ is the set of all subsets of $A$ (including the set itself and the empty set):
\[\mathcal{P}(A)=\{S\mid S\subset A\}.\]

An \vocab{ordered pair}\index{set!ordered pair} is denoted by $(a,b)$, where the order of the elements matters. Two pairs $(a_1,b_1)$ and $(a_2,b_2)$ are equal if and only if $a_1=a_2$ and $b_1=b_2$.  Similarly, we have ordered triples $(a,b,c)$, quadruples $(a,b,c,d)$ and so on. If there are $n$ elements it is called an \emph{$n$-tuple}.

The \vocab{Cartesian product}\index{set!Cartesian product} of sets $A$ and $B$, denoted by $A \times B$, is the set of all ordered pairs with the first element of the pair coming from $A$ and the second from $B$:
\[A\times B\coloneqq\{(a,b)\mid a\in A,b\in B\}.\]
More generally, we define $A_1 \times A_2 \times \cdots \times A_n$ to be the set of all ordered $n$-tuples $(a_1, a_2, \dots, a_n)$, where $a_i \in A_i$ for $1 \le i \le n$. If all the $A_i$ are the same, we write the product as $A^n$.

\begin{example}
$\RR^2$ is the Euclidean plane, $\RR^3$ is the Euclidean space, and $\RR^n$ is the $n$-dimensional Euclidean space.
\begin{align*}
\RR \times \RR = \RR^2 &= \{(x,y) \mid x,y \in \RR\} \\
\RR \times \RR \times \RR = \RR^3 &= \{(x,y,z) \mid x,y,z \in \RR\} \\
\RR^n &= \{(x_1,x_2,\dots,x_n) \mid x_1,x_2,\dots,x_n \in \RR\}
\end{align*}
\end{example}

\subsection{Algebra of Sets}
We now disuss the algebra of sets. Given $A \subset S$ and $B \subset S$,
\begin{enumerate}[label=(\roman*)]
\item The \vocab{union}\index{set!union} $A \cup B$ is the set consisting of elements that are in $A$ or $B$ (or both):
\[ A\cup B=\{x \in S \mid x\in A \lor x\in B\} \]

\item The \vocab{intersection}\index{set!intersection} $A \cap B$ is the set consisting of elements that are in both $A$ and $B$:
\[ A\cap B=\{x \in S \mid x\in A \land x\in B\} \]

$A$ and $B$ are \vocab{disjoint}\index{set!disjoint} if both sets have no element in common: $A\cap B=\emptyset$.
\end{enumerate}

More generally, we can take unions and intersections of arbitrary numbers of sets (could be finitely or infinitely many). Given a family of sets $\{A_i\mid i\in I\}$ where $I$ is an \emph{indexing set}, we write
\[\bigcup_{i\in I}A_i=\{x \mid \exists i\in I, x\in A_i\},\]
and
\[\bigcap_{i\in I}A_i=\{x \mid \forall i\in I, x\in A_i\}.\]

\begin{enumerate}[resume*]
\item The \vocab{complement}\index{set!complement} of $A$, denoted by $A^c$, is the set containing elements that are not in A:
\[ A^c = \{x \in S \mid x \notin A\} \]

\item The \vocab{set difference}\index{set!set difference}, or complement of $B$ in $A$, denoted by $A\setminus B$, is the subset consisting of those elements that are in $A$ and not in $B$:
\[ A\setminus B = \{x \in A \mid x \notin B\} \]
Note that $A\setminus B = A \cap B^c$.
\end{enumerate}

\begin{lemma}[Distributive laws]
Let $A,B,C\subset S$. Then
\begin{enumerate}[label=(\roman*)]
\item $A\cup(B\cap C)=(A\cup B)\cap(A\cup C)$;
\item $A\cap(B\cup C)=(A\cap B)\cup(A\cap C)$.
\end{enumerate}
\end{lemma}

\begin{proof} \
\begin{enumerate}[label=(\roman*)]
\item Suppose $x\in A\cup(B \cap C)$. Then
\begin{align*}
x\in A\cup(B \cap C)
&\iff x\in A\quad\lor\quad x\in B\cap C\\
&\iff x\in A\quad\lor\quad (x\in B)\land (x\in C)\\
&\iff (x\in A)\lor (x\in B)\quad\land\quad (x\in A)\lor (x\in C)\\
&\iff x\in A\cup B\quad\land\quad x\in A\cup C\\
&\iff x\in (A\cup B)\cap(A\cup C).
\end{align*}
Thus $A \cup (B \cap C) \subset (A \cup B) \cap (A \cup C)$.

Conversely suppose that $x \in (A \cup B) \cap (A \cup C)$. Then go in the reverse direction of the above steps to show that $(A\cup B)\cap (A\cup C)\subset A\cup(B\cap C)$.

By double inclusion, $(A\cup B)\cap(A\cup C)=A\cup(B\cap C)$.

\item Similar.
\end{enumerate}
\end{proof}

\begin{lemma}[de Morgan's laws]
Let $A,B\subset S$. Then
\begin{enumerate}[label=(\roman*)]
\item $(A \cup B)^c = A^c \cap B^c$;
\item $(A \cap B)^c = A^c \cup B^c$.
\end{enumerate}
\end{lemma}

\begin{proof} \
\begin{enumerate}[label=(\roman*)]
\item \begin{align*}
x\in(A\cup B)^c&\iff x\notin A\cup B\\
&\iff x\notin A\quad\land\quad x\notin B\\
&\iff x\in A^c\quad\land\quad x\in B^c\\
&\iff x\in A^c\cap B^c
\end{align*}

\item Similar.
\end{enumerate}
\end{proof}

De Morgan's laws extend naturally to any number of sets. Suppose $\{A_i\mid i\in I\}$ is a family of subsets of $S$, then
\begin{align*}
\brac{\bigcap_{i\in I}A_i}^c&=\bigcup_{i\in I}{A_i}^c,\\
\brac{\bigcup_{i\in I}A_i}^c&=\bigcap_{i\in I}{A_i}^c.
\end{align*}

\begin{exercise}
Prove the following:
\begin{enumerate}[label=(\roman*)]
\item $\brac{\bigcup_{i\in I}A_i}\cup B=\bigcup_{i\in I}(A_i\cup B)$
\item $\brac{\bigcap_{i\in I}A_i}\cup B=\bigcap_{i\in I}(A_i\cup B)$
\item $\brac{\bigcup_{i\in I}A_i}\cup\brac{\bigcup_{j\in J}B_j}=\bigcup_{(i,j)\in I\times J}(A_i\cup B_j)$
\item $\brac{\bigcap_{i\in I}A_i}\cup\brac{\bigcap_{j\in J}B_j}=\bigcap_{(i,j)\in I\times J}(A_i\cup B_j)$
\end{enumerate}
\end{exercise}

\begin{exercise}
Let $S\subset A\times B$. Express the set $A_S$ of all elements of $A$ which appear as the first entry in at least one of the elements in $S$.

($A_S$ here may be called the projection of $S$ onto $A$.)
\end{exercise}
\pagebreak

\section{Relations}
\subsection{Definition and Examples}
\begin{definition}[Relation]
$R$ is a \vocab{relation}\index{relation} between $A$ and $B$ if $R\subset A\times B$; $a\in A$ and $b\in B$ are said to be \emph{related} if $(a,b)\in R$, denoted $a R b$.
\end{definition}

\begin{remark}
A relation is a set of ordered pairs.
\end{remark}

Visually speaking, a relation is uniquely determined by a simple bipartite graph over $A$ and $B$. On the bipartite graph, this is usually represented by an edge between $a$ and $b$.

\begin{example}
In many cases we do not actually use $R$ to write the relation because there is some other conventional notation:
\begin{itemize}
\item The ``less than or equal to'' relation $\le$ on the set of real numbers is
\[\{(x,y) \in \RR^2 \mid x \le y\}\subset\RR^2;\]
we write $x\le y$ if $(x,y)$ is in this set.
\item The ``divides'' relation $\mid$ on $\NN$ is
\[\{(m,n)\in\NN^2\mid m\text{ divides }n\}\subset\NN^2;\]
we write $m\mid n$ if $(m,n)$ is in this set.
\item For a set $S$, the ``subset'' relation $\subset$ on $\mathcal{P}(S)$ is
\[\{(A,B)\in\mathcal{P}(S)^2\mid A\subset B\}\subset\mathcal{P}(S)^2;\]
we write $A\subset B$ if $(A,B)$ is in this set.
\end{itemize}
\end{example}

If $A \times B$ is the smallest Cartesian product of which $R$ is a subset, we call $A$ and $B$ the \emph{domain} and \emph{range} of $R$ respectively, denoted by $\dom R$ and $\ran R$ respectively.

\begin{example}
Given $R=\{(1,a),(1,b),(2,b),(3,b)\}$, then $\dom R=\{1,2,3\}$ and $\ran R=\{a,b\}$.
\end{example}

\begin{definition}[Binary relation]
A \vocab{binary relation}\index{relation!binary relation} in $A$ is a relation between $A$ and itself; that is, $R \subset A \times A$.
\end{definition}

\subsection{Properties of Relations}
Let $A$ be a set, $R$ a relation on $A$, $x,y,z \in A$. We say that
\begin{enumerate}[label=(\roman*)]
\item $R$ is \vocab{reflexive} if $xRx$ for all $x\in A$;
\item $R$ is \vocab{symmetric} if $xRy \implies yRx$;
\item $R$ is \vocab{anti-symmetric} if $xRy \text{ and } yRx \implies x=y$;
\item $R$ is \vocab{transitive} if $xRy \text{ and } yRz \implies xRz$.
\end{enumerate}

\begin{example}[Less than or equal to]
The relation $\le$ on $R$ is reflexive, anti-symmetric, and transitive, but not symmetric. 
\end{example}

\begin{definition}
A \vocab{partial order}\index{relation!partial order} on a non-empty set $A$ is a relation on $A$ satisfying reflexivity, anti-symmetry and transitivity.

A \vocab{total order}\index{relation!total order} on $A$ is a partial order on $A$ such that if for every $x,y\in A$, either $xRy$ or $yRx$.

A \vocab{well order}\index{relation!well order} on $A$ is a total order on $A$ such that every non-empty subset of $A$ has a minimal element; that is, for each non-empty $B\subset A$ there exists $s\in B$ such that $s\le b$ for all $b\in B$.
\end{definition}

\begin{example} \
\begin{itemize}
\item Less than: the relation $<$ on $R$ is not reflexive, symmetric, or anti-symmetric, but it is transitive.
\item Not equal to: the relation $\neq$ on $R$ is not reflexive, anti-symmetric or transitive, but it is symmetric.
\end{itemize}
\end{example}

\subsection{Equivalence Relations}
One important type of relation is an equivalence relation. An equivalence relation is a way of saying two objects are, in some particular sense, ``the same''.

\begin{definition}[Equivalence relation]
A relation $\sim$ on a set $A$ is an \vocab{equivalence relation}\index{equivalence relation} if it is reflexive, symmetric and transitive.
\end{definition}

\begin{notation}
We denote $a\sim b$ for $(a,b)\in R$.
\end{notation}

An equivalence relation provides a way of grouping together elements that can be viewed as being the same:

\begin{definition}[Equivalence class]
Given an equivalence relation $\sim$ on a set $A$, and given $x \in A$, the \vocab{equivalence class}\index{equivalence relation!equivalence class} of $x$ is
\[[x]\coloneqq\{y\in A\mid y\sim x\}.\]
\end{definition}

Grouping the elements of a set into equivalence classes provides a partition of the set, which we define as follows:

\begin{definition}[Partition]
A \vocab{partition}\index{equivalence relation!partition} of a set $A$ is a collection of subsets $\{A_i\subset A\mid i\in I\}$, where $I$ is an indexing set, with the property that
\begin{enumerate}[label=(\roman*)]
\item $A_i\neq\emptyset$ for all $i\in I$ (all the subsets are non-empty)
\item $\bigcup_{i\in I}A_i=A$ (every member of $A$ lies in one of the subsets)
\item $A_i\cap A_j=\emptyset$ for every $i\neq j$ (the subsets are disjoint)
\end{enumerate}
The subsets are called the \emph{parts} of the partition.
\end{definition}

\begin{proposition}
Let $\sim$ be an equivalence relation on a non-empty set $X$. Then the equivalence classes under $\sim$ are a partition of $X$.
\end{proposition}

To prove this, we need to show that
\begin{enumerate}[label=(\roman*)]
\item every equivalence class is non-empty;
\item every element of $X$ is an element of an equivalence class;
\item every element of $X$ lies in exactly one equivalence class.
\end{enumerate}

\begin{proof} \
\begin{enumerate}[label=(\roman*)]
\item An equivalence class $[x]$ contains $x$ as $x\sim x$, by reflexivity of the relation. Thus $[x]\neq\emptyset$.
\item From (i), note that every $x\in X$ is in the equivalence class $[x]$, so every element of $X$ is an element of at least one equivalence class.
\item Suppose otherwise, for a contradiction, that some element of $X$ lies in more than one equivalence class. Let $x\in X$ such that $x\in[y]$ and $x\in[z]$; we want to show that $[y]=[z]$ (using double inclusion).

Let $a\in[y]$, so $a\sim y$. ALso $x\in[y]$ so $x\sim y$. By symmetry, $y\sim x$. By transitivity, $a\sim x$. Now $x\in[z]$ so $x\sim z$ and similarly $a\sim z$ thus $a\in[z]$. Hence $[y]\subset[z]$.

By the same argument, $[z]\subset[y]$. Hence $[y]=[z]$.
\end{enumerate}
\end{proof}

\begin{definition}[Quotient set]
The \vocab{quotient set}\index{equivalence relation!quotient set} is the set of all equivalence classes, denoted by $A/\sim$.
\end{definition}

\begin{example}[Modular arithmetic]
Let $n$ be a fixed positive integer. Define a relation on $\ZZ$ by
\[a\sim b\iff n\mid(b-a).\]
\begin{proposition*}
$a\sim b$ is a equivalence relation.
\end{proposition*}
\begin{proof} \
\begin{enumerate}[label=(\roman*)]
\item $a\sim a$ so $\sim$ is reflexive.
\item $a\sim b\implies b\sim a$ for any integers $a$ and $b$, so $\sim$ is symmetric.
\item If $a\sim b$ and $b\sim c$ then $n\mid(a-b)$ and $n\mid(b-c)$, so $n\mid(a-b)+(b-c)=(a-c)$, so $a\sim c$ and $\sim$ is transitive.
\end{enumerate}
\end{proof}

\begin{notation}
We write $a\equiv b\pmod n$ if $a\sim b$.
\end{notation}

\begin{notation}
For any $k\in\ZZ$ we denote the equivalence class of $a$ by $[a]$, called the \emph{congruence class} (or \emph{residue class}) of $a$ mod $n$, which consists of the integers which differ from $a$ by an integral multiple of $n$; that is,
\[[a]=\{a+kn\mid k\in\ZZ\}.\]
\end{notation}

There are precisely $n$ distinct congruence classes mod $n$, namely
\[[0],[1],\dots,[n-1],\]
determined by the possible remainders after division by $n$; and these residue classes partition the integers $\ZZ$. The set of equivalence classes under this equivalence relation is denoted by $\ZZ/n\ZZ$, and called the \emph{integers modulo $n$}.

Define addition and multiplication on $\ZZ/n\ZZ$ as follows: for $[a],[b]\in\ZZ/n\ZZ$,
\begin{align*}
[a]+[b]&=[a+b]\\
[a][b]&=[ab].
\end{align*}

This means that to compute the sum / product of two elements $[a],[b]\in\ZZ/n\ZZ$, take any \emph{representative} $a\in[a]$, $b\in[b]$, and add / multiply integers $a$ and $b$ as usual in $\ZZ$, then take the congruence class containing the result.

\begin{proposition*}
Addition and mulltiplication on $\ZZ/n\ZZ$ are well-defined; that is, they do not depend on the choices of representatives for the classes involved. More precisely, if $a_1,a_2\in\ZZ$ and $b_1,b_2\in\ZZ$ with $\overline{a_1}=\overline{b_1}$ and $\overline{a_2}=\overline{b_2}$, then $\overline{a_1+a_2}=\overline{b_1+b_2}$ and $\overline{a_1a_2}=\overline{b_1b_2}$, i.e., If
\[a_1\equiv b_1\pmod n,\quad a_2\equiv b_2\pmod n\]
then
\[a_1+a_2\equiv b_1+b_2\pmod n,\quad a_1a_2\equiv b_1b_2\pmod n.\]
\end{proposition*}

\begin{proof}
Suppose $a_1\equiv b_1\pmod n$, i.e., $n\mid(a_1-b_1)$. Then $a_1=b_1+sn$ for some integer $s$. Similarly, $a_2\equiv b_2\pmod n$ means $a_2=b_2+tn$ for some integer $t$.

Then $a_1+a_2=(b_1+b_2)+(s+t)n$ so that $a_1+a_2\equiv b_1+b_2\pmod n$, which shows that the sum of the residue classes is independent of the representatives chosen.

Similarly, $a_1a_2=(b_1+sn)(b_2+tn)=b_1b_2+(b_1t+b_2s+stn)n$ shows that $a_1a_2\equiv b_1b_2\pmod n$ and so the product of the residue classes is also independent of the representatives chosen.
\end{proof}

An important subset of $\ZZ/n\ZZ$ consists of the collection of congruence classes which have a multiplicative inverse in $\ZZ/n\ZZ$:
\[(\ZZ/n\ZZ)^\times\coloneqq\{[a]\in\ZZ/n\ZZ\mid\exists[c]\in\ZZ/n\ZZ,[a][c]=[1]\}.\]

\begin{proposition*}
$(\ZZ/n\ZZ)^\times$ is also the collection of congruence classes whose representatives are relatively prime to $n$:
\[(\ZZ/n\ZZ)^\times=\{[a]\in\ZZ/n\ZZ\mid(a,n)=1\}.\]
\end{proposition*}
\end{example}

\subsection{Axiom of Choice and Its Equivalences}
\begin{definition}
Let $(P,\le)$ be a partially ordered set. Suppose $A\subset P$.
\begin{enumerate}[label=(\roman*)]
\item $u\in P$ is an \vocab{upper bound} for $A$ if $x\le u$ for all $x\in A$.
\item $m\in P$ is a \vocab{maximal element} of $P$ if $x\in P$ and $m\le x$ implies $m=x$.
\item Similarly we define \vocab{lower bound} and \vocab{minimal element}.
\item $C\subset P$ is called a \vocab{chain} if either $x\le y$ or $y\le x$ for all $x,y\in C$.
\end{enumerate}
\end{definition}

This terminology of partially ordered sets will often be applied to an arbitrary family of sets. When this is done, it should be understood that the family is being regarded as a partially ordered set under the relation $\subsetneq$. Thus a maximal member of $\mathscr{A}$ is a set $M\in\mathscr{A}$ such that $M$ is a proper subset of no other member of $\mathscr{A}$; a chain of sets is a family $\mathscr{C}$ of sets such that $A\subsetneq B$ or $B\subsetneq A$ for all $A,B\in\mathscr{C}$.

\begin{definition}
Let $\mathscr{F}$ be a family of sets. Then $\mathscr{F}$ is said to be a \emph{family of finite character} if for each set $A$, we have $A\in\mathscr{F}$ if and only if each finite subset of $A$ is in $\mathscr{F}$.
\end{definition}

We shall need the following technical fact.

\begin{lemma}
Let $\mathscr{F}$ be a family of finite character, and let $\mathscr{C}$ be a chain in $\mathscr{F}$. Then $\bigcup\mathscr{C}\in\mathscr{F}$.
\end{lemma}

\begin{proof}
It suffices to show that each finite subset of $\bigcup\mathscr{C}$ is in $\mathscr{F}$. Let $F=\{x_1,\dots,x_n\}\subset\bigcup\mathscr{C}$. Then there exist sets $C_1,\dots,C_n\in\mathscr{C}$ such that $x_i\in C_i$ ($i=1,\dots,n$). Since $\mathscr{C}$ is a chain, there exists $i_0\in\{1,\dots,n\}$ such that $C_i\subsetneq C_{i_0}$ for $i=1,\dots,n$. Then $F\subset C_{i_0}\in\mathscr{F}$. But $\mathscr{F}$ is of finite character, and so $F\in\mathscr{F}$.
\end{proof}

\begin{theorem}
The following are equivalent:
\begin{enumerate}[label=(\roman*)]
\item \emph{Axiom of choice}: The Cartesian product of any non-empty collection of non-empty sets is non-empty.
\item \emph{Tukey's lemma}: Every non-empty family of finite character has a maximal member.
\item \emph{Hausdorff maximality principle}: Every non-empty partially ordered set contains a maximal chain.
\item \emph{Zorn's lemma}: Every non-empty partially ordered set in which every chain has an upper bound has a maximal element.
\item \emph{Well-ordering principle}: Every non-empty set has a well-ordering.
\end{enumerate}
\end{theorem}

\begin{proof}
We direct the reader to Section 3 of \cite{hewitt-stromberg} for the complete proof.
\end{proof}

\begin{remark}
It is a non-trivial result that Zorn's lemma is independent of the usual (Zermelo--Fraenkel) axioms of set theory in the sense that if the axioms of set theory are consistent, then so are these axioms together with Zorn's lemma; and if the axioms of set theory are consistent, then so are these axioms together with the negation of Zorn's lemma.
\end{remark}
\pagebreak

\section{Functions}
\begin{definition}[Function]
A \vocab{function}\index{function} $f:X\to Y$ is a mapping of every element of $X$ to some element of $Y$; $X$ and $Y$ are known as the \emph{domain} and \emph{codomain} of $f$ respectively.
\end{definition}

\begin{remark}
The definition requires that a unique element of the codomain is assigned for every element of the domain. For example, for a function $f:\RR \to \RR$, the assignment $f(x)=\frac{1}{x}$ is not sufficient as it fails at $x=0$. Similarly, $f(x)=y$ where $y^2=x$ fails because $f(x)$ is undefined for $x<0$, and for $x>0$ it does not return a unique value; in such cases, we say the the function is \emph{ill-defined}. We are interested in the opposite; functions that are \emph{well-defined}.
\end{remark}

If a function is defined on some larger domain than we care about, it may be helpful to restrict the domain:

\begin{definition}[Restriction]
Given a function $f:X \to Y$ and a subset $A \subset X$, the \vocab{restriction}\index{function!restriction} of $f$ to $A$ is the map $f|_A:A \to Y$.
\end{definition}

\begin{remark}
The restriction is almost the same function as the original function---just the domain has changed.
\end{remark}

Another rather trivial but nevertheless important function is the identity map:

\begin{definition}[Identity map]
Given a set $X$, the \vocab{identity} $\id_X:X \to X$ is defined by
\[\id_X(x)=x\quad(\forall x\in X)\]
\end{definition}

\begin{notation}
If the domain is unambiguous, the subscript may be omitted.
\end{notation}

\subsection{Images and Pre-images}
\begin{definition}
Suppose $f:X\to Y$. The \vocab{image}\index{function!image} of $f$ is
\[f(X)\coloneqq\{f(x)\mid x\in X\}\subset Y.\]
More generally, the image of $A\subset X$ under $f$ is
\[f(A)\coloneqq\{f(x)\mid x\in A\}\subset Y.\]
The \vocab{pre-image}\index{function!pre-image} of $B\subset Y$ under $f$ is
\[f^{-1}(B)\coloneqq\{x\in X\mid f(x)\in B\}.\]
\end{definition}

\begin{remark}
Note the distinction between ``codomain'' and ``range''.
\end{remark}

\begin{lemma}
Let $f:X\to Y$. Suppose $A\subset X$ and $B\subset Y$.
\begin{enumerate}[label=(\roman*)]
\item If $A=f^{-1}(B)$, then $f(A)\subset B$.
\item If $B=f(A)$, then $A\subset f^{-1}(B)$. 
\end{enumerate}
\end{lemma}

\begin{proof} \
\begin{enumerate}[label=(\roman*)]
\item Let $y=f(A)$. Then $y=f(x)$ for some $x\in A$. 
Since $A=f^{-1}(B)$, then $x\in f^{-1}(B)$. Then $f(x)=w$ for some $w\in B$. 
Thus $y=f(x)=w\in B$. Hence $f(A)\subset B$.

\item Let $x\in A$. Then $f(x)\in f(A)=B$; let $f(x)=y$ for some $y\in B$. Consider $y\in B$; it could have one or more elements of $A$ mapped to it. Hence $A\subset f^{-1}(B)$. 
\end{enumerate}
\end{proof}

\begin{remark}
In general, we cannot conclude that $B=f(A)$ implies $A=f^{-1}(B)$.
\end{remark}

We can express the previous result as follows:
\[f\brac{f^{-1}(B)}\subset B,\quad A\subset f^{-1}\brac{f(A)}.\]

\begin{lemma}[Algebra of pre-images]
Suppose $f:X\to Y$. Then
\begin{enumerate}[label=(\roman*)]
\item $f^{-1}(A^c)=f^{-1}(A)^c$ for every $A\subset Y$;
\item $f^{-1}\brac{\bigcup_{i\in I}A_i}=\bigcup_{i\in I}f^{-1}(A_i)$;
\item $f^{-1}\brac{\bigcap_{i\in I}A_i}=\bigcap_{i\in I}f^{-1}(A_i)$.
\end{enumerate}
\end{lemma}

\begin{proof} \
\begin{enumerate}[label=(\roman*)]
\item Suppose $A\subset Y$. Let $x\in X$, then
\begin{align*}
x\in f^{-1}(A^c)&\iff f(x)\in A^c\\
&\iff f(x)\notin A\\
&\iff x\notin f^{-1}(A)\\
&\iff x\in f^{-1}(A)^c
\end{align*}
Hence $f^{-1}(A^c)=f^{-1}(A)^c$.

\item Suppose $\{A_i\mid i\in I\}$ is a collection of subsets of $Y$. Then
\begin{align*}
x\in f^{-1}\brac{\bigcup_{i\in I}A_i}
&\iff f(x)\in\bigcup_{i\in I}A_i\\
&\iff f(x)\in A_i\text{ for some }i\in I\\
&\iff x\in f^{-1}(A_i)\text{ for some }i\in I\\
&\iff x\in\bigcup_{i\in I}f^{-1}(A_i)
\end{align*}
Hence $f^{-1}\brac{\bigcup_{i\in I}A_i}=\bigcup_{i\in I}f^{-1}(A_i)$.

\item Suppose $\{A_i\mid i\in I\}$ is a collection of subsets of $Y$. Then
\begin{align*}
x\in f^{-1}\brac{\bigcap_{i\in I}A_i}
&\iff f(x)\in\bigcap_{i\in I}A_i\\
&\iff f(x)\in A_i\text{ for every }i\in I\\
&\iff x\in f^{-1}(A_i)\text{ for every }i\in I\\
&\iff x\in\bigcap_{i\in I}f^{-1}(A_i)
\end{align*}
Hence $f^{-1}\brac{\bigcap_{i\in I}A_i}=\bigcap_{i\in I}f^{-1}(A_i)$.
\end{enumerate}
\end{proof}
%https://math.stackexchange.com/questions/359693/overview-of-basic-results-about-images-and-preimages


\subsection{Injectivity, Surjectivity, Bijectivity}
\begin{definition}
Suppose $f:X\to Y$.
\begin{enumerate}[label=(\roman*)]
\item $f$ is \vocab{injective}\index{function!injectivity} (or \emph{one-to-one}) if each element of $Y$ has at most one element of $X$ that maps to it:
\[\forall x_1,x_2\in X,\quad f(x_1)=f(x_2) \implies x_1=x_2\]

\item $f$ is \vocab{surjective}\index{function!surjectivity} (or \emph{onto}) if every element of $Y$ is mapped to at least one element of $X$:
\[ \forall y\in Y,\quad\exists x\in X,\quad f(x)=y \]

\item $f$ is \vocab{bijective}\index{function!bijectivity} if it is both injective and surjective; a bijective function is termed a \emph{bijection}.
\end{enumerate}
\end{definition}

\begin{notation}
We write $X\sim Y$ if there exists a bijection $f:X\to Y$.
\end{notation}

\subsection{Composition}
\begin{definition}[Composition]
Given $f:X\to Y$ and $g:Y\to Z$, the \vocab{composition} $g\circ f:X\to Z$ is defined by
\[ (g \circ f)(x)=g(f(x))\quad(\forall x \in X)\]
\end{definition}

The composition of functions is not commutative. However, composition is associative, as the following results shows:

\begin{proposition}[Associativity of composition]
Suppose $f:X\to Y$, $g:Y\to Z$, $h:Z\to W$. Then
\[f\circ (g\circ h)=(f\circ g)\circ h.\]
\end{proposition}

\begin{proof}
Let $x\in X$. By the definition of composition, we have
\[(f\circ(g\circ h))(x)=f((g\circ h)(x))=f(g(h(x)))=(f\circ g)(h(x))=((f\circ g)\circ h)(x).\]
\end{proof}

\begin{proposition}[Composition preserves injectivity and surjectivity] \
\begin{enumerate}[label=(\roman*)]
\item If $f:X \to Y$ is injective and $g:Y \to Z$ is injective, then $g \circ f:X \to Z$ is injective.
\item If $f:X\to Y$ is surjective and $g:Y\to Z$ is surjective, then $g \circ f:X\to Z$ is surjective.
\end{enumerate}
\end{proposition}

\begin{proof} \
\begin{enumerate}[label=(\roman*)]
\item Let $f:X \to Y$ and $g:Y \to Z$ be injective. To prove that $g \circ f:X\to Z$ is injective, we need to prove: for all $x,x^\prime\in X$, 
\[(g \circ f)(x)=(g\circ f)(x^\prime) \implies x=x^\prime.\]

Suppose that $(g \circ f)(x) = (g \circ f)(x^\prime)$. Then by definition
\[g\brac{f(x)}=g\brac{f(x^\prime)}.\]
Injectivity of $g$ implies
\[f(x)=f(x^\prime),\]
and injectivity of $f$ implies
\[x=x^\prime.\]

\item Let $f:X\to Y$ and $g:Y\to Z$ be surjective. To prove that $g\circ f:X\to Z$ is surjective, we need to prove that for any $z\in Z$, there exists $x\in X$ such that $(g\circ f)(x)=z$.

Let $z\in Z$. By surjectivity of $g:Y\to Z$, there exists $y\in Y$ such that $g(y)=z$. By surjectivity of $f:X\to Y$, there exists $x\in X$ such that $f(x)=y$. This means that there exists $x\in X$ such that $g\brac{f(x)}=g(y)=z$, as desired.
\end{enumerate}
\end{proof}

\begin{proposition}
$f:X\to Y$ is injective if and only if for any set $Z$ and any functions $g_1,g_2:Z\to X$,
\[f\circ g_1=f\circ g_2 \implies g_1=g_2.\]
\end{proposition}

\begin{proof} \

\fbox{$\implies$} Suppose $f$ is injective, and suppose $f\circ g_1=f\circ g_2$. Let $z\in Z$. Then we have
\[f\brac{g_1(z)}=f\brac{g_2(z)}.\]
Injectivity of $f$ implies
\[g_1(z)=g_2(z),\]
so $g_1=g_2$ (since the choice of $z\in Z$ is arbitrary).

\fbox{$\impliedby$} Pick $Z=\{1\}$, basically some random one-element set. Then for $x,y\in X$, define
\begin{align*}
g_1:Z\to X,&\quad g_1(1)=x,\\
g_2:Z\to Y,&\quad g_2(1)=y.
\end{align*}
Then for $x,y\in X$,
\[ f(x)=f(y) \implies f(g_1(1))=f(g_2(1)) \implies g_1(1)=g_2(1) \implies x=y \]
which shows that $f$ is injective.
\end{proof}

\begin{proposition}
$f:X\to Y$ is surjective if and only if for any set $Z$ and any functions $g_1,g_2:Y\to Z$,
\[g_1 \circ f=g_2 \circ f \implies g_1=g_2.\]
\end{proposition}

\begin{proof} \

\fbox{$\implies$} Suppose that $f$ is surjective. Let $y\in Y$. Surjectivity of $f$ means there exists $x\in X$ such that $f(x)=y$. Then
\[g_1\circ f=g_2 \circ f\implies g_1\brac{f(x)}=g_2\brac{f(x)}\implies g_1(y)=g_2(y) \]
so $g_1=g_2$.

\fbox{$\impliedby$} We prove the contrapositive. Suppose $f$ is not surjective, then there exists $y \in Y$ such that for all $x \in X$ we have $f(x)\neq y$. We then aim to construct set $Z$ and $g_1,g_2:Y\to Z$ such that
\begin{enumerate}[label=(\roman*)]
\item $g_1(y) \neq g_2(y)$
\item $\forall y^\prime \neq y, g_1(y^\prime)=g_2(y^\prime)$
\end{enumerate}

Because if this is satisfied, then $\forall x \in X$, since $f(x)\neq y$ we have from (ii) that $g_1(f(x))=g_2(f(x))$; thus $g_1 \circ f=g_2 \circ f$, and yet from (i) we have $g_1 \neq g_2$.

We construct $Z=Y\cup\{1,2\}$ for some random $1,2 \notin Y$.

Then we define
\begin{align*}
&g_1:Y\to Z,g_1(y)=1,g_1(y^\prime)=y^\prime\\
&g_2:Y\to Z,g_2(y)=2,g_2(y^\prime)=y^\prime
\end{align*}

Then when $y$ is not in the image of $f$, these two functions will satisfy $g_1 \circ f=g_2 \circ f$ but not $g_1=g_2$.

So conversely, if for any set $Z$ and any functions $g_i:Y \to Z$ we have $g_1 \circ f=g_2 \circ f \implies g_1=g_2$, such a value $y$ that is in the codomain but not in the range of $f$ cannot appear, and hence $f$ must be surjective.
\end{proof}

\begin{lemma}[Inverse image of composition]
Suppose $f:X\to Y$, $g:Y\to Z$. Then
\[(g\circ f)^{-1}(A)=f^{-1}\brac{g^{-1}(A)}\]
for every $A\subset Z$.
\end{lemma}

\begin{proof}
Suppose $A\subset Z$. Let $x\in X$, then we have
\begin{align*}
x\in(g\circ f)^{-1}(A)
&\iff (g\circ f)(x)\in A\\
&\iff g\brac{f(x)}\in A\\
&\iff f(x)\in g^{-1}(A)\\
&\iff x\in f^{-1}\brac{g^{-1}(A)}
\end{align*}
Hence $(g\circ f)^{-1}(A)=f^{-1}\brac{g^{-1}(A)}$.
\end{proof}

\subsection{Invertibility}
Recalling that $\id_X$ is the identity map on $X$, we can define invertibility.

\begin{definition}[Invertibility]
Suppose $f:X\to Y$. We say that
\begin{enumerate}[label=(\roman*)]
\item $f$ is \vocab{left-invertible} if there exists $g:Y\to X$ such that $g\circ f=\id_X$; $g$ is a \emph{left-inverse} of $f$;
\item $f$ is \vocab{right-invertible} if there exists $h:Y\to X$ such that $f\circ h=\id_Y$; $h$ is a \emph{right-inverse} of $h$;
\item $f$ is \vocab{invertible}\index{function!invertibility} if there exists $k:Y\to X$ which is a left and right inverse of $f$; $k$ is an \emph{inverse} of $f$.
\end{enumerate}
\end{definition}

\begin{remark}
Notice that if $g$ is left-inverse to $f$ then $f$ is right-inverse to $g$. A function can have more than one left-inverse, or more than one right-inverse.
\end{remark}

\begin{example}
Let
\begin{align*}
f:\RR\to[0,\infty),&\quad f(x)=x^2\\
g:[0,\infty)\to\RR,&\quad g(x)=\sqrt{x}
\end{align*}
\begin{itemize}
\item $f$ is not left-invertible. Suppose otherwise, for a contradiction, that $h$ is a left inverse of $f$, so that $hf=\id_\RR$. Then 
\end{itemize}
\end{example}

\begin{proposition}[Uniqueness of inverse]
If $f:X\to Y$ is invertible then its inverse is unique.
\end{proposition}

\begin{proof}
Let $g_1$ and $g_2$ be two functions for which $g_i \circ f = \id_X$ and $f \circ g_i = \id_Y$. Using the fact that composition is associative, and the definition of the identity maps, we can write
\[ g_1 = g_1 \circ \id_Y = g_1 \circ (f \circ g_2) = (g_1 \circ f) \circ g_2 = \id_X \circ g_2 = g_2.\]
\end{proof}

Since the inverse is unique, we can give it a notation.

\begin{notation}
The inverse of $f$ is denoted by $f^{-1}$
\end{notation}

\begin{remark}
Note that directly from the definition, if $f$ is invertible then $f^{-1}$ is also invertible, and $(f^{-1})^{-1}=f$.
\end{remark}

The following result provides an important and useful criterion for invertibility.

\begin{lemma}[Invertibility criterion]
Suppose $f:X\to Y$. Then
\begin{enumerate}[label=(\roman*)]
\item $f$ is left-invertible if and only if $f$ is injective;
\item $f$ is right-invertible if and only if $f$ is surjective;
\item $f$ is invertible if and only if $f$ is bijective.
\end{enumerate}
\end{lemma}

\begin{proof} \
\begin{enumerate}[label=(\roman*)]
\item \fbox{$\implies$} Suppose $f$ is left-invertible; let $g$ be a left-inverse of $f$, so $g\circ f=\id_X$.

Now suppose $f(a)=f(b)$. Then applying $g$ to both sides gives $g\brac{f(a)}=g\brac{f(b)}$, so $a=b$.

\fbox{$\impliedby$} Let $f$ be injective. Choose any $x_0$ in the domain of $f$. Define $g:Y\to X$ as follows; note that each $y\in Y$ is either in the image of $f$ or not.
\begin{itemize}
\item If $y$ is in the image of $f$, it equals $f(x)$ for a \emph{unique} $x\in X$ (uniqueness is because of the injectivity of $f$), so define $g(y)=x$.
\item If $y$ is not in the image of $f$, define $g(y)=x_0$. 
\end{itemize}
Clearly $g\circ f=\id_X$.

\item \fbox{$\implies$} Suppose $f$ is right-invertible; let $g$ be a right-inverse of $f$, so $f\circ g=\id_Y$.

Let $y\in Y$. Then $f\brac{g(y)}=\id_Y(y)=y$ so $y\in f(X)$. Thus $f(X)=Y$ so $f$ is surjective.

\fbox{$\impliedby$} Suppose $f$ is surjective. Let $y\in Y$, then $y$ is in the image of $f$, so we can choose an element $g(y)\in X$ such that $f\brac{g(y)}=y$. This defines a function $g:Y\to X$ which is evidently a right-inverse of $f$.

\item \fbox{$\implies$} Suppose $f$ is invertible. Then $f$ is left-invertible and right-invertible. By (i) and (ii), $f$ is injective and surjective, so $f$ is bijective.

\fbox{$\impliedby$} Suppose $f$ is bijective. Then by (i) and (ii), $f$ has a left-inverse $g:Y\to X$ and a right-inverse $h:Y\to X$. But ``invertible'' requires a single function to be \emph{both} a left and right inverse, so we need to show that $g=h$:
\[g=g\circ\id_Y=g\circ(f\circ h)=(g\circ f)\circ h=\id_X\circ h=h\]
so $g=h$ is an inverse of $f$.
\end{enumerate}
\end{proof}

The following result shows how to invert the composition of invertible functions.

\begin{proposition}[Inverse of composition]
Suppose $f:X \to Y$, $g:Y \to Z$. If $f$ and $g$ are invertible, then $g \circ f$ is invertible, and
\[(g \circ f)^{-1}=f^{-1}\circ g^{-1}.\]
\end{proposition}

\begin{proof}
Making repeated use of the fact that function composition is associative, and the definition of the inverses $f^{-1}$ and $g^{-1}$, we note that
\begin{align*}
(f^{-1}\circ g^{-1}) \circ (g \circ f) 
&= ((f^{-1} \circ g^{-1}) \circ g) \circ f \\
&= (f^{-1} \circ (g^{-1} \circ g)) \circ f \\
&= (f^{-1} \circ \id_Y) \circ f \\
&= f^{-1} \circ f \\
&= \id_X
\end{align*}
and similarly,
\begin{align*}
(g \circ f) \circ (f^{-1} \circ g^{-1}) 
&= g \circ (f \circ (f^{-1} \circ g^{-1})) \\
&= g \circ ((f \circ f^{-1}) \circ g^{-1}) \\
&= g \circ (\id_Y \circ g^{-1}) \\
&= g \circ g^{-1} \\
&= \id_Z
\end{align*}
which shows that $f^{-1} \circ g^{-1}$ satisfies the properties required to be the inverse of $g \circ f$.
\end{proof}

\begin{corollary}
If $f_1,\dots,f_n$ are invertible and the composition $f_1\circ\cdots\circ f_n$ makes sense, then it is also invertible and its inverse is
\[f_n^{-1}\circ\cdots\circ f_1^{-1}.\]
\end{corollary}

\begin{proposition}
$\sim$ is an equivalence relation between sets.
\end{proposition}

\begin{proof}
We need to prove (i) reflexivity, (ii) symmetry, and (iii) transitivity.
\begin{enumerate}[label=(\roman*)]
\item The identity map gives a bijection from a set to itself.
\item Suppose $f:X\to Y$ is a bijection. Then $f$ is invertible, with inverse $f^{-1}:Y\to X$. Since $f^{-1}$ is invertible (with inverse $f$), it is bijective.
\item Suppose $f:X\to Y$ and $g:Y\to Z$ are bijections, and thus they are invertible. Then by the previous result, $g\circ f$ is invertible and thus bijective.
\end{enumerate}
\end{proof}

\begin{theorem}[Cantor--Schr\"{o}der--Bernstein]
If $f:X\to Y$ and $g:Y\to X$ are injective, then $A\sim B$.
\end{theorem}
\pagebreak

\section{Cardinality}
This section is about formalising the notion of the ``size'' of a set.

\begin{definition}
$A$ and $B$ said to be \vocab{equivalent} (or have the same \emph{cardinality}), denoted by $A\sim B$, if there exists a bijection $f:A\to B$. 
\end{definition}

\begin{notation}
For $n\in\NN$, denote
\begin{align*}
\NN_n&=\{k\in\NN\mid 1\le k\le n\},\\
n\NN&=\{nk\mid k\in\NN\}.
\end{align*}
\end{notation}

\begin{definition}
For any set $A$, we say
\begin{enumerate}[label=(\roman*)]
\item $A$ is \vocab{finite} if $A\sim\NN_n$ for some integer $n\in\NN$, then the \emph{cardinality} of $A$ is $|A|=n$; $A$ is \emph{infinite} if $A$ is not finite;
\item $A$ is \vocab{countable} if $A\sim \NN$; $A$ is \emph{uncountable} if $A$ is neither finite nor countable; $A$ is \emph{at most countable} if $A$ is finite or countable.
\end{enumerate}
\end{definition}

\begin{remark}
Any countable set can be ``listed'' in a sequence $a_1,a_2,\dots$ of distinct terms. This technique is particularly useful when there is not possible to deduce an explicit formula for a bijection.
\end{remark}

\begin{proposition}
$\NN$ is infinite.
\end{proposition}

\begin{proof}
We want to show that there does not exist a bijection from $\NN_n$ to $\NN$, for all $n\in\NN$. We prove by induction on $n$.

For the base case $n=1$, if there exists a function $f_1:\{1\}\to\NN$, consider the set $\NN\setminus f_1(\{1\})$. It is not empty, so $f_1$ is not surjective, thus it is not bijective.

For the inductive step, we want to show if there does not exist a bijection from $\NN_k$ to $\NN$, then there does not exist a bijection from $\NN_{k+1}$ to $\NN$. We prove the contrapositive: if there exists a bijection from $\NN_{k+1}\to\NN$, then there exists a bijection from $\NN_k$ to $\NN$.

Suppose $h:\NN_{k+1}\to\NN$ is a bijection. If remove the element $k+1$, then there exists a bijection from $\NN_k$ to $\NN\setminus\{h(k+1)\}$. But $\NN\setminus\{h(k+1)\}\sim\NN$ so $\NN_k\sim\NN$.
\end{proof}

\begin{corollary}
Any countable set is infinite.
\end{corollary}

\begin{comment}
\subsection{Finite Sets}
For finite sets, we can do some arithmetic with their cardinalities.

\begin{proposition}[Subsets of a finite set]
If a set $A$ is finite with $|A| = n$, then its power set has $|\mathcal{P}(A)| = 2^n$.
\end{proposition}

\begin{proof}
We use induction. For the initial step, note that if $|A| = 0$ then $A = \emptyset$ has no elements, so there is a single subset $\emptyset$, and therefore $|\mathcal{P}(A)| = 1 = 2^0$.

Now suppose that $n \ge 0$ and that $|P(S)| = 2^n$ for any set S with $|S| = n$. Let $A$ be any set with $|A| = n+1$. By definition, this means that there is an element $a$ and a set $A_0 = A\setminus\{a\}$ with $|A_0| = n$. Any subset of A must either contain the element a or not, so we can partition $\mathcal{P}(A) = P(A_0) \cup \{S \cup \{a\} \mid S \in P(A_0)\}$. These two sets are disjoint, and each of them has cardinality $|P(A_0)| = 2^n$ by the inductive hypothesis. Hence $|\mathcal{P}(A)| = 2^n + 2^n = 2^{n+1}$.

Thus, by induction, the result holds for all $n$.
\end{proof}

Another way to see this is through combinatorics: Consider the process of creating a subset. We can do this systematically by going through each of the $|A|$ elements in $A$ and making the yes/no decision whether to put it in the subset. Since there are $|A|$ such choices, that yields $2^{|A|}$ different combinations of elements and therefore $2^{|A|}$ different subsets.

\begin{theorem}[Cantor's Theorem]\label{thrm:cantor}
For a set $A$, finite or infinite,
\[|A|<|\mathcal{P}(A)|.\]
\end{theorem}

\begin{proof}
Suppose, for a contradiction, that $|A|=|\mathcal{P}(A)|$. Then there exists a bijection $f:A\to\mathcal{P}(A)$. Put
\[B=\{x\in A\mid x\notin f(A)\}.\]

Now consider any $x\in A$. In the first case, $x\in f(A)$, then
\[x\in f(A)\iff x\notin B,\]
thus $f(A)\neq B$. In the second case, $x\notin f(A)$, then 
\[x\notin f(A)\iff x\in B,\]
thus $f(x)\neq B$. Hence $f$ is not surjective, which is a contradiction.
\end{proof}

\begin{corollary}
For all $n\in\ZZ_0^+$,
\[n<2^n.\]
\end{corollary}

\begin{proof}
This can be easily proven through induction.
\end{proof}

\begin{proposition}
Let $A$ and $B$ be finite sets. Then $|A \cup B| = |A| + |B| - |A \cap B|$.
\end{proposition}

\begin{proof}
The proof is left as an exercise.
\end{proof}

\begin{theorem}[Principle of Inclusion and Exclusion]
Let $S_i$ be finite sets. Then
\begin{equation}
\absolute{\bigcup_{i=1}^nS_i}=\sum_{i=1}|S_i|-\sum_{1\le i<j\le n}|S_i\cap S_j|+\sum_{1\le i<j<k\le n}|S_i\cap S_j\cap S_k|+\cdots+(-1)^{n+1}\absolute{\bigcap_{i=1}^nS_i}.
\end{equation}
\end{theorem}

\begin{proof}
By induction.
\end{proof}

\begin{proof}[Alternative proof]
Let $U$ be a finite set (interpreted as the universal set), and $S\subset U$. Define the characteristic/indicator function of $S$ by
\[ \chi_S(x)=\begin{cases}
1&(x\in S)\\
0&(x\notin S)
\end{cases} \]
In other words,
\[ x\in S\iff\chi_S(x)=1 \]
and equivalently,
\[ x\notin S\iff\chi_S(x)=0. \]
Let $S_1,S_2\subset U$ be given. Then for any $x\in U$ it holds that
\[ \chi_{S_1\cap S_2}(x)=\chi_{S_1}(x)\cdot\chi_{S_2}(x) \]
where $\cdot$ denotes ordinary multiplication.

Similarly,
\[ \chi_{S_1\cup S_2}(x)=1-\brac{1-\chi_{S_1}(x)}\cdot\brac{1-\chi_{S_2}(x)}. \]
In general, for any $x\in U$ it holds that
\[ \chi_{S_1\cup\cdots\cup S_n}(x)=1-\brac{1-\chi_{S_1}(x)}\cdots\brac{1-\chi_{S_n}(x)} \]
for any $S_1,\dots,S_n\subset U$.

Since $x\in S$ if and only if $\chi_S(x)=1$, it follows that
\[ |S|=\sum_{x\in U}\chi_S(x). \]
To prove the PIE, we calculate
\begin{align*}
&|S_1\cup\cdots\cup S_n|\\
&=\sum_{x\in U}\chi_{S_1\cup\cdots\cup S_n}(x)\\
&=\sum_{x\in U}1-\brac{1-\chi_{S_1}(x)}\cdots\brac{1-\chi_{S_n}(x)}\\
&=\brac{\chi_{S_1}(x)+\cdots+\chi_{S_n}(x)}-\brac{\chi_{S_1}(x)\chi_{S_2}(x)+\cdots+\chi_{S_{n-1}}(x)\chi_{S_n}(x)}+\cdots+(-1)^{n+1}\chi_{S_1}(x)\cdots\chi_{S_n}(x)\\
&=\brac{\chi_{S_1}(x)+\cdots+\chi_{S_n}(x)}-\brac{\chi_{S_1\cap S_2}(x)+\cdots+\chi_{S_{n-1}\cap S_n}(x)}+\cdots+(-1)^{n+1}\chi_{S_1\cap\cdots\cap S_n}(x)\\
&=\sum_{i=1}^n|S_i|-\sum_{J\subset\{1,\dots,n\},|J|=2}\absolute{\bigcap_{j\in J}S_j}+\cdots+(-1)^{k+1}\sum_{J\subset\{1,\dots,n\},|J|=k}\absolute{\bigcap_{j\in J}S_j}+\cdots+(-1)^{n+1}\absolute{\bigcap_{i=1}^nS_i}.
\end{align*}
\end{proof}

\subsection{Countability}
For two finite sets $A$ and $B$, we evidently have $A\sim B$ if and only if $A$ and $B$ contain the same number of elements. For infinite sets, however, the idea of ``having the same number of elements'' becomes quite vague, whereas the notion of bijectivity retains its clarity.
\end{comment}

\begin{example}
$\NN$ is countable since the identity map from $\NN$ to $\NN$ is a bijection.
\end{example}

\begin{example}
$n\NN$ is countable.
\begin{proof}
Let $f:\NN\to n\NN$ which sends $k\mapsto nk$. We now need to show that $f$ is (i) injective, and (ii) surjective.
\begin{enumerate}[label=(\roman*)]
\item For any $k_1,k_2\in\NN$, $nk_1=nk_2$ implies $k_1=k_2$ so $f$ is injective.
\item For any $x\in n\NN$, $x=nk$ for some $k\in\NN$, thus $\frac{x}{n}=k\in\NN$ so $f$ is surjective.
\end{enumerate}
Hence $f$ is bijective, so $n\NN\sim\NN$ and we are done.
\end{proof}
\end{example}

\begin{example}
$\ZZ$ is countable.
\begin{proof}
Consider the following arrangement of the elements of $\ZZ$ and $\NN$:
\begin{align*}
\ZZ&:\quad0,1,-1,2,-2,3,-3,\dots\\
\NN&:\quad1,2,3,4,5,6,7,\dots
\end{align*}
In fact we can write an explicit formula for a bijection $f:\NN\to\ZZ$ where
\[f(n)=\begin{cases}
\dfrac{n}{2}&\text{($n$ even)}\\[1ex]
-\dfrac{n-1}{2}&\text{($n$ odd)}
\end{cases}\]
\end{proof}
\end{example}

\begin{proposition}\label{prop:infinite-subset-countable}
Every infinite subset of a countable set is countable.
\end{proposition}

\begin{proof}
Let $S$ be the countable set. Then we can arrange the elements of $S$ in a sequence $(s_n)$ of distinct elements:
\[s_1,s_2,\dots\]
Suppose $E\subset S$ is infinite. The main idea is to show that we can list out the elements of $E$ in a sequence. We now construct a sequence $(n_k)$ as follows: Let
\begin{align*}
n_1&=\min\{i\mid s_i\in E\}\\
n_2&=\min\{i\mid s_i\in E,i>n_1\}\\
&\vdots\\
n_k&=\min\{i\mid s_i\in E,i>n_{k-1}\}.
\end{align*}
Then
\[E=\{s_{n_1},s_{n_2},\dots\},\]
where we note that the function $f(k)=s_{n_k}$ ($k=1,2,\dots$) is bijective. Hence $E\sim\NN$, as desired.
\end{proof}

\begin{remark}
This shows that countable sets represent the ``smallest'' infinity: No uncountable set can be a subset of a countable set.
\end{remark}

\begin{proposition}\label{prop:union-countable}
The countable union of countable sets is countable.
\end{proposition}

\begin{proof}
Let $\{A_n\mid n\in\NN\}$ be a fanily of countable sets; clearly this is a countable collection of sets (indexed by $\NN$). Then we want to show that the union
\[S=\bigcup_{n=1}^\infty A_n\]
is countable.

Since every set $A_n$ is countable, we can list its elements in a sequence $(a_{nk})$ ($k=1,2,3,\dots$). Arrange the elements of all the sets in $\{A_n\}$ in the form of an infinite array, containing all elements of $S$, where the elements of $A_n$ form the $n$-th row.
\begin{table}[H]
\centering
\begin{tabular}{cccccc}
$A_1$:&$\cancelto{}{a_{11}}$ & $\cancelto{}{a_{12}}$ & $\cancelto{}{a_{13}}$ & $\cancelto{}{a_{14}}$ & $\cdots$\\
$A_2$:&$\cancelto{}{a_{21}}$ & $\cancelto{}{a_{22}}$ & $\cancelto{}{a_{23}}$ & $\cancelto{}{a_{24}}$ & $\cdots$\\
$A_3$:&$\cancelto{}{a_{31}}$ & $\cancelto{}{a_{32}}$ & $\cancelto{}{a_{33}}$ & $\cancelto{}{a_{34}}$ & $\cdots$\\
$A_4$:&$\cancelto{}{a_{41}}$ & $\cancelto{}{a_{42}}$ & $\cancelto{}{a_{43}}$ & $\cancelto{}{a_{44}}$ & $\cdots$\\
$\vdots$ & & & & &
\end{tabular}
\end{table}
We then zigzag our way through the array, and arrange these elements in a sequence
\[a_{11},a_{21},a_{12},a_{31},a_{22},a_{13},a_{41},a_{32},a_{23},a_{14},\dots\]
thus $S$ is countable, and we are almost done!

A small problem is that if any two of the sets $A_n$ have elements in common, these will appear more than once in the above sequence. Then we take a subset $T\subset S$, where every element only appears once. Note that $T$ is an infinite subset, since $A_1\subset T$ is infinite. Then since $T$ is an infinite subset of a countable set $S$, by \cref{prop:infinite-subset-countable}, $T$ is countable.
\end{proof}

\begin{remark}
If we were to instead start by going down by the first row of the above array, then we would not get to the second row (and beyond); all that would show is the first row is countable. Instead, we wind our way through diagonally, ensuring that we hit every number of the array.
\end{remark}

\begin{corollary}
Suppose $A$ is an indexing set that is at most countable. Let $\{B_\alpha\mid\alpha\in A\}$ be a family of sets that are at most countable. Then the union
\[\bigcup_{\alpha\in A}B_\alpha\]
is at most coutable.
\end{corollary}

\begin{proposition}
Let $A$ be a countable set. For $n\in\NN$, let
\[B_n=\{(a_1,\dots,a_n)\mid a_i\in A\}.\]
Then $B_n$ is countable.
\end{proposition}

\begin{proof}
We prove by induction on $n$. That $B_1$ is countable is evident, since $B_1=A$.

Now suppose $B_{n-1}$ is countable. The elements of $B_n$ are of the form 
\[(b,a)\quad(b\in B_{n-1},a\in A)\]
For every fixed $b$, the set of ordered pairs $(b,a)$ is equivalent to $A$, and hence countable. Thus $B_n$ is a union of countable sets. By \cref{prop:union-countable}, $B_n$ is countable.
\end{proof}

\begin{corollary}
$\QQ$ is countable.
\end{corollary}

\begin{proof}
Note that every $x\in\QQ$ is of the form $\frac{b}{a}$, where $a,b\in\ZZ$. By the previous result, taking $n=2$, the set of pairs $(a,b)$ and therefore the set of fractions $\frac{b}{a}$ is countable.
\end{proof}

That not all infinite sets are, however, countable, is shown by the next result.

\begin{proposition}
Let $A$ be the set of all sequences whose elements are the digits $0$ and $1$. Then $A$ is uncountable. 
\end{proposition}

\begin{proof}
Let $E\subset A$ be countable, consisting of the sequences $s_1,s_2,s_3,\dots$.

We construct a new sequence $s$ as follows:
\[\text{$n$-th digit of $s$}=\begin{cases}
0&\text{if $n$-th digit in $s_n$ is $1$,}\\
1&\text{if $n$-th digit in $s_n$ is $0$.}
\end{cases}\]
Then the sequence $s$ differs from every member of $E$ in at least one place, so $s\notin E$. But clearly $s\in A$; hence $E\subsetneq A$.

We have shown that every countable subset of $A$ is a proper subset of $A$. It follows that $A$ is uncountable (for otherwise $A$ would be a proper subset of $A$, which is absurd).
\end{proof}

\begin{remark}
The idea of the above proof is called \emph{Cantor's diagonal process}, first used by Cantor. This is because if elements of the sequences $s_1,s_2,s_3,\dots$ are listed out in an array, it is the elements on the diagonal which are involved in the construction of the new sequence.
\end{remark}

\begin{corollary}
$\RR$ is uncountable.
\end{corollary}

\begin{proof}
This follows from the binary representation of the real numbers.
\end{proof}

\begin{theorem}[Cantor's theorem]
For any set $A$, we have $A\not\sim\mathcal{P}(A)$.
\end{theorem}

\begin{proof}
Suppose otherwise, for a contradiction, that $A\sim\mathcal{P}(A)$. Then there exists a bijection $f:A\to\mathcal{P}(A)$. Then for each $x\in A$, $f(x)$ is a subset of $A$. Now consider the ``anti-diagonal'' set
\[B=\{x\in A\mid x\notin f(x)\}.\]
That is, $B$ is the subset of $A$ containing all $x\in A$ such that $x$ is not in the set $f(x)$. Since $B\subset A$, we have $B\in\mathcal{P}(A)$. Since $f$ is bijective (in particular surjective), there exists $x\in A$ such that $f(x)=B$. Now there are two cases: (i) $x\in B$, or (ii) $x\notin B$.
\begin{enumerate}[label=(\roman*)]
\item If $x\in B$, then by definition of the set $B$ it must be the case that $x\notin f(x)$. But since $f(x)=B$, we then have $x\notin D$. This is absurd since we cannot have $x\in B$ and $x\notin B$ simultaneously.
\item If $x\notin B$, by definition of the set $B$, this implies that $x\in f(x)$. But $f(x)=B$. So we have $x\in B$ and $x\notin B$, which is again absurd.
\end{enumerate}
In either case, we have reached a contradiction. Hence there cannot exist a surjective (and thus bijective) function $A\to\mathcal{P}(A)$.
\end{proof}
\pagebreak

\section*{Exercises}
\begin{exercise}
Prove the following statements:
\begin{enumerate}[label=(\roman*)]
\item $f(A\cup B)=f(A)\cup f(B)$
\item $f(A_1\cup\cdots\cup A_n)=f(A_1)\cup\cdots\cup f(A_n)$
\item $f(\bigcup_{\lambda\in A}A_\lambda)=\bigcup_{\lambda\in A}f(A_\lambda)$
\item $f(A\cap B)\subset f(A)\cap f(B)$
\item $f^{-1}(f(A))\supset A$
\item $f(f^{-1}(A))\subset A$
\item $f^{-1}(A\cup B)=f^{-1}(A)\cup f^{-1}(B)$
\item $f^{-1}(A\cap B)=f^{-1}(A)\cap f^{-1}(B)$
\item $f^{-1}(A_1\cup\cdots\cup A_n)=f^{-1}(A_1)\cup\cdots\cup f^{-1}(A_n)$
\item $f^{-1}(\bigcup_{\lambda\in A}A_\lambda)=\bigcup_{\lambda\in A}f^{-1}(A_\lambda)$
\end{enumerate}
\end{exercise}

\begin{exercise}
Let $A$ be the set of all complex polynomials in $n$ variables. Given a subset $T \subset A$, define the \textit{zeros} of $T$ as the set
\[ Z(T) = \{P=(a_1,\dots,a_n) \mid f(P)=0 \text{ for all } f \in T\} \]
A subset $Y \in \CC^n$ is called an algebraic set if there exists a subset $T \subset A$ such that $Y=Z(T)$.

Prove that the union of two algebraic sets is an algebraic set.
\end{exercise}
\begin{proof}
We would like to consider $T=\{f_1, f_2, \dots\}$ expressed as indexed sets $T=\{f_i\}$. Then $Z(T)$ can also be expressed as $\{P \mid \forall i, f_i(P)=0\}$.

Suppose that we have two algebraic sets $X$ and $Y$. Let $X=Z(S)$, $Y=Z(T)$ where $S,T$ are subsets of $A$ (basically, they are certain sets of polynomials). Then
\[ X=\{P \mid \forall f \in S, f(P)=0\} \]
\[ Y=\{P \mid \forall g \in T, g(P)=0\} \]

We imagine that for $P\in X\cap Y$, we have $f(P)=0$ or $g(P)=0$. Hence we consider the set of polynomials
\[ U=\{f\cdot g \mid f\in S, g\in T\} \]

For any $P\in X\cup Y$ and for any $fg\in U$ where $f\in S$ and $f\in g$, either $f(P)=0$ or $g(P)=0$, hence $fg(P)=0$ and thus $P\in Z(U)$.

On the other hand if $P\in Z(U)$, suppose otherwise that $P$ is not in $X\cup Y$, then $P$ is neither in $X$ nor in $Y$. This means that there exists $f\in S,g\in T$ such that $f(P)\neq0$ and $g(P)\neq0$, hence $fg(P)\neq0$. This is a contradiction as $P\in Z(U)$ implies $fg(P)=0$. Hence we have $X\cup Y=Z(U)$ and thus $X\cup Y$ is an algebraic set.

Now the other direction is simpler and can actually be generalised: The intersection of arbitrarily many algebraic sets is algebraic. 

The basic result is that if $X=Z(S)$ and $Y=Z(T)$ then $X\cap Y=Z(S\cup T)$. 
\end{proof}

\begin{exercise}
Let $A=\RR$ and for any $x, y \in A$, $x \sim y$ if and only if $x-y \in \ZZ$. For any two equivalence classes $[x], [y] \in A/\sim$, define
\[ [x] + [y] = [x + y] \text{ and } -[x] = [-x] \]
\begin{enumerate}[label=(\alph*)]
\item Show that the above definitions are well-defined.
\item Find a one-to-one correspondence $\phi:X \to Y$ between $X = A/\sim$ and $Y:|z| = 1$, i.e. the unit circle in $\CC$, such that for any $[x_1], [x_2] \in X$ we have
\[ \phi([x_1])\phi([x_2]) = \phi([x_1 + x_2]) \]
\item Show that for any $[x] \in X$,
\[ \phi(-[x]) = \phi([x])^{-1} \]
\end{enumerate}
\end{exercise}

\begin{solution} \ 
\begin{enumerate}[label=(\alph*)]
\item 
\[ (x^\prime+y^\prime)-(x+y)=(x^\prime-x)+(y^\prime-y)\in \ZZ \]
Thus $[x^\prime+y^\prime]=[x+y]$

\[ (-x^\prime)-(-x)=-(x^\prime-x)\in \ZZ \]
Thus $[-x^\prime]=[-x]$.

\item Complex numbers in the polar form: $z=re^{i\theta}$

Then the correspondence is given by $\phi([x])=e^{2\pi ix}$
\[ [x]=[y] \iff x-y\in \ZZ \iff e^{2\pi i(x-y)}=1 \iff e^{2\pi ix}=e^{2\pi iy} \]
Hence this is a bijection.

Before that, we also need to show that $\phi$ is well-defined, which is almost the same as the above.

If we choose another representative $x^\prime$ then
\[ \phi([x])=e^{2\pi ix^\prime} = e^{2\pi ix}\cdot e^{2\pi i(x^\prime-x)} = e^{2\pi ix} \]

\item You can either refer to the specific correspondence $\phi([x])=e^{2\pi ix}$ or use its properties.
\[ \phi(-[x])\phi([x]) = \phi([-x])\phi([x]) = \phi([-x+x]) = \phi([0]) = 1 \]
\end{enumerate}
\end{solution}

\begin{exercise}[Complex Numbers]
Let $\RR[x]$ denote the set of real polynomials. Define
\[ \CC=\RR[x]/(x^2+1)\RR[x] \]
where
\[ f(x)\sim g(x) \iff x^2+1 \text{ divides } f(x)-g(x). \]
The complex number $a+bi$ is defined to be the equivalence class of $a+bx$.
\begin{enumerate}[label=(\alph*)]
\item Define the sum and product of two complex numbers and show that such definitions are well-defined.
\item Define the reciprocal of a complex number.
\end{enumerate}
\end{exercise}

\begin{exercise}[\cite{rudin} 2.2]
$z\in\CC$ is said to be \emph{algebraic} if there exist integers $a_0,\dots,a_n$, not all zero, such that
\[a_0z^n+a_1z^{n-1}+\cdots+a_{n-1}z+a_n=0.\]
Prove that the set of all algebraic numbers is countable. \emph{Hint}: For every positive integer $N$ there are only finitely many equations with
\[n+|a_0|+|a_1|+\cdots+|a_n|=N.\]
\end{exercise}

\begin{solution}
Following the hint, let $A_N$ be the set of numbers $z$ that satisfy $a_0z^n+a_1z^{n-1}+\cdots+a_{n-1}z+a_n=0$, for some coefficients $a_0,\dots,a_n$ which satisfy
\[n+|a_0|+|a_1|+\cdots+|a_n|=N.\]

By the fundamental theorem of algebra, $a_0z^n+a_1z^{n-1}+\cdots+a_{n-1}z+a_n=0$ has at most $n$ solutions, so each $A_N$ is finite. Hence the set of algebraic numbers, which is the union
\[\bigcup_{N=2}^{\infty}A_N\]
is at most countable. Since all rational numbers are algebraic, it follows that the set of algebraic numbers is exactly countable. 
\end{solution}

\begin{exercise}[\cite{rudin} 2.3]
Prove that there exist real numbers which are not algebraic.
\end{exercise}

\begin{solution}
By the previous exercise, the set of real algebraic numbers is countable. If every real number were algebraic, the entire set of real numbers would be countable, a contradiction.
\end{solution}

\begin{exercise}[\cite{rudin} 2.4]
Is the set of irrational real numbers countable?
\end{exercise}

\begin{solution}
No. If $\RR\setminus\QQ$ were countable, $\RR=\QQ\cup(\RR\setminus\QQ)$ would be countable, which is clearly false.
\end{solution}
\fi

%%%%%%%%%%%%%%% CALC
\ifcalc
    \part{Calculus}
    \chapter{Single Variable Calculus}
\section{Integration Techniques}
We review the following basic techniques for evaluating integrals:
\begin{itemize}
\item Integration by substitution
\item Integration by parts, reduction formula
\end{itemize}

\begin{exercise}
Evaluate
\[I=\int_{0}^{1}\frac{1}{\sqrt{4-2x-x^2}}\dd{x}.\]
\end{exercise}

\begin{solution}
The integral is close to the known integral $\int\frac{1}{\sqrt{1-x^2}}\dd{x}$. By completing the square we may write
\begin{align*}
4-2x-x^2&=5-(x+1)^2\\
&=5\brac{1-\brac{\frac{x+1}{\sqrt{5}}}^2}.
\end{align*}
Making the substitution $u=\frac{x+1}{\sqrt{5}}$ we have $\dd{u}=\frac{1}{\sqrt{5}}\dd{x}$, so that
\begin{align*}
I&=\int_{0}^{1}\frac{1}{\sqrt{4-2x-x^2}}\dd{x}\\
&=\frac{1}{\sqrt{5}}\int_{0}^{1}\frac{1}{\sqrt{1-\brac{\frac{x+1}{\sqrt{5}}}^2}}\dd{x}\\
&=\frac{1}{\sqrt{5}}\int_{\frac{1}{\sqrt{5}}}^{\frac{2}{\sqrt{5}}}\frac{\sqrt{5}}{\sqrt{1-u^2}}\dd{u}\\
&=\int_{\frac{1}{\sqrt{5}}}^{\frac{2}{\sqrt{5}}}\frac{1}{\sqrt{1-u^2}}\dd{u}\\
&=\sin^{-1}\frac{2}{\sqrt{5}}-\sin^{-1}\frac{1}{\sqrt{5}}.
\end{align*}
\end{solution}

Now let us recall the technique of integration by parts. This is the integral form of the product rule for derivatives, that is $(fg)^\prime=f^\prime g+fg^\prime$, where $f$ and $g$ are functions of $x$, and the prime denotes the derivative with respect to $x$. Thus we have
\[f(x)g(x)=\int g(x)f^\prime(x)\dd{x}+\int f(x)g^\prime(x)\dd{x}\]
and we arrange the terms to obtain
\[\int f(x)g^\prime(x)\dd{x}=f(x)g(x)-\int g(x)f^\prime(x)\dd{x}.\]
Similarly, for definite integrals we have
\[\int_{a}^{b}f(x)g^\prime(x)\dd{x}=\sqbrac{f(x)g(x)}_a^b-\int_{a}^{b}g(x)f^\prime(x)\dd{x}.\]

\begin{exercise}
Evaluate
\[I=\int xe^x\dd{x}.\]
\end{exercise}

\begin{solution}
We have $f(x)=x$ and $g^\prime(x)=e^x$, so that $f^\prime(x)=1$ and $g(x)=e^x$. Thus
\begin{align*}
I&=\int xe^x\dd{x}\\
&=xe^x-\int e^x\dd{x}\\
&=xe^x-e^x+c.
\end{align*}
\end{solution}

Sometimes, after two applications of the ``by parts'' formula, we almost get back to where we started:

\begin{exercise}
Evaluate
\[I=\int e^x\sin x\dd{x}.\]
\end{exercise}

\begin{solution}
\begin{align*}
\int e^x\sin x\dd{x}
&=e^x\sin x-\int e^x\cos x\dd{x}\\
&=e^x\sin x-e^x\cos x-\int e^x\sin x\dd{x}.
\end{align*}
Now we see that we have returned to our original integral, so that we can rearrange this equation to obtain
\[\int e^x\sin x\dd{x}=\frac{1}{2}e^x\brac{\sin x-\cos x}+c.\]
\end{solution}

Finally in this section we look at an example of a \emph{reduction formula}.

\begin{exercise}
Consider $I_n=\int\cos^n x\dd{x}$ where $n$ is a non-negative integer. Find a reduction formula for $I_n$, and use this formula to evaluate $\int\cos^7x\dd{x}$.
\end{exercise}

\begin{solution}
The aim here is to write $I_n$ in terms of other $I_k$ where $k<n$, so that eventually we are reduced to calculating $I_0$ or $I_1$, say, both of which are easily found (analagous to a recurrence relation). Using integration by parts we have:
\begin{align*}
I_n&=\int\cos^n x\dd{x}\\
&=\int\cos^{n-1}\times\cos x\dd{x}\\
&=\cos^{n-1}x\sin x+(n-1)\int\cos^{n-2}x\sin^2 x\dd{x}\\
&=\cos^{n-1}x\sin x+(n-1)\int\cos^{n-2}x(1-\cos^2 x)\dd{x}\\
&=\cos^{n-1}x\sin x+(n-1)\brac{I_{n-2}-I_n}.
\end{align*}

Rearranging this to make $I_n$ the subject we obtain
\[I_n=\frac{1}{n}\cos^{n-1}x\sin x+\frac{n-1}{n}I_{n-2}.\]

With this reduction formula, $I_n$ can be rewritten in terms of simpler and simpler integrals until we are left only needing to calculate $I_0$ if $n$ is even, or $I_1$ if $n$ is odd. Therefore, $I_7$ can be found as follows:
\begin{align*}
I_7&=\frac{1}{7}\cos^6x\sin x+\frac{6}{7}I_5\\
&=\frac{1}{7}\cos^6x\sin x+\frac{6}{7}\brac{\frac{1}{5}\cos^4x\sin x+\frac{4}{5}I_3}\\
&=\frac{1}{7}\cos^6x\sin x+\frac{6}{7}\brac{\frac{1}{5}\cos^4x\sin x+\frac{4}{5}\brac{\frac{1}{3}\cos^2x\sin x+\frac{2}{3}I_1}}\\
&=\frac{1}{7}\cos^6x\sin x+\frac{6}{35}\cos^4x\sin x+\frac{24}{105}\cos^2x\sin x+\frac{48}{105}\sin x+c.
\end{align*}
\end{solution}

\section{First Order Differential Equations}
An \vocab{ordinary differential equation} (ODE) is an equation relating a variable, say $x$, a function, say $y$, of the variable $x$, and finitely many of the derivatives of $y$ with respect to $x$. That is, an ODE can be written in the form
\[f\brac{x,y,\dv{y}{x},\dv[2]{y}{x},\dots,\dv[k]{y}{x}}=0\]
for some function $f$, $k\in\NN$. Here $x$ is the independent variable and the ODE governs how the dependent variable $y$ varies with $x$. The equation may have no, one or many functions $y(x)$ which satisfy it; the problem is usually to find the most general solution $y(x)$, a function which satisfies the differential equation.

We say that an ODE has \emph{order} $k$ if it involves derivatives of order $k$ and less. Thus first order differential equations take the form
\[\dv{y}{x}=f(x,y).\]
In general, a $k$-th order ODE takes the form
\[a_k(x)\dv[k]{y}{x}+a_{k-1}(x)\dv[k-1]{y}{x}+\cdots+a_1(x)\dv{y}{x}+a_0(x)y=f(x),\]
where $a_k(x)\neq0$. The ODE is \emph{homogeneous} if $f(x)=0$ for all $x$, and \emph{inhomogeneous} otherwise. 

The following are some standard methods for solving first order ODEs:
\begin{itemize}
\item Direct integration
\item Separation of variables
\item Reduction to separable form by substitution
\item Exact differential equations
\item Integrating factors
\end{itemize}

If the ODE takes the form
\[\dv{y}{x}=f(x),\]
then we can solve this by direct integration:

\begin{exercise}
Find the general solution to
\[\dv{y}{x}=x^2\sin x.\]
\end{exercise}

\begin{solution}
Integrating both sides with respect to $x$ and then integrating the RHS by parts, we have
\[y=-x^2\cos x+2x\sin x+2\cos x+c.\]
\end{solution}

When the ODE is \emph{separable}, that is, it takes the form
\[\dv{y}{x}=a(x)b(y),\]
where $a(x)$ and $b(y)$ are functions of $x$ and $y$ respectively, we can solve this by separating the variables:
\[\frac{1}{b(y)}\dv{y}{x}=a(x),\]
then integrating both sides with respect to $x$ we find
\[\int\frac{1}{b(y)}\dd{y}=\int a(x)\dd{x}.\]
Here we have assumed that $b(y)\neq0$; if $b(y)=0$ then the solution is $y=c$ for some constant $c$.

Some first order differential equations can be transformed by a suitable substitution into separable form.

\begin{exercise}
Find the general solution to
\[\dv{y}{x}=\sin(x+y+1).\]
\end{exercise}

\begin{solution}
Let $u=x+y+1$, so that $\dv{u}{x}=1+\dv{y}{x}$. Then the original equation can be written as
\[\dv{u}{x}=1+\sin u,\]
which is separable. We have 
\[\frac{1}{1+\sin u}\dv{u}{x}=1,\]
which integrates to
\[\int\frac{1}{1+\sin u}\dd{u}=x+c.\]
Let us evaluate the integral on the LHS:
\begin{align*}
\int\frac{1}{1+\sin u}\dd{u}&=\int\frac{1-\sin u}{(1+\sin u)(1-\sin u)}\dd{u}\\
&=\int\frac{1-\sin u}{1-\sin^2 u}\dd{u}\\
&=\int\frac{1-\sin u}{\cos^2 u}\dd{u}\\
&=\int\sec^2 u\dd{u}-\int\sec u\tan u\dd{u}\\
&=\tan u-\sec u+c.
\end{align*}
Therefore the general solution is
\[\tan(x+y+1)-\sec(x+y+1)=x+c.\]
This solution, where we have not found $y$ in terms of $x$, is called an \emph{implicit solution}.
\end{solution}

A special group of first order differential equations are homogeneous ones, of the form
\[\dv{y}{x}=f\brac{\frac{y}{x}}.\]
These can be solved by the substitution of the form $y(x)=xv(x)$, so that the ODE becomes
\[x\dv{v}{x}=f(v)-v,\]
which is separable.

Now we look specifically at first order linear ODEs, which take the general form
\[\dv{y}{x}+p(x)y=q(x).\]
We see that the homogeneous form, that is when $q(x)=0$, is separable. The inhomogeneous form can be solved by multiplying an \emph{integrating factor} $I(x)$ given by
\[I(x)=e^{\int p(x)\dd{x}}\]
on both sides of the equation, so that
\[e^{\int p(x)\dd{x}}\dv{y}{x}+p(x)e^{\int p(x)\dd{x}}y=e^{\int p(x)\dd{x}}q(x).\]
Using the product rule for derivatives, this gives
\[\dv{}{x}\brac{e^{\int p(x)\dd{x}}y}=e^{\int p(x)\dd{x}}q(x),\]
which we can integrate directly to find $y(x)$:
\[y(x)=e^{-\int p(x)\dd{x}}\brac{\int e^{\int p(x)\dd{x}}q(x)\dd{x}+c}.\]

\begin{exercise}
Solve the linear differential equation
\[\dv{y}{x}+2xy=2xe^{-x^2}.\]
\end{exercise}

\begin{solution}
The integrating factor is $I(x)=e^{\int 2x\dd{x}}=e^{x^2}$. Multiplying through by this factor gives
\[e^{x^2}\dv{y}{x}+2xe^{x^2}y=2x,\]
that is
\[\dv{}{x}\brac{e^{x^2}y}=2x.\]
Integrating both sides with respect to $x$ we find
\[e^{x^2}y=x^2+c,\]
so that the general solution is
\[y=\brac{x^2+c}e^{-x^2}.\]
\end{solution}

\subsection{Second Order Linear Differential Equations}
The main subject of this section is linear ODEs with constant coefficients, but before we look at these we give two theorems that are valid in the more general case.

Recall that a homogeneous linear ODE of order $k$ takes the form
\[a_k(x)\dv[k]{y}{x}+a_{k-1}(x)\dv[k-1]{y}{x}+\cdots+a_1(x)\dv{y}{x}+a_0y=0.\]
It turns out that the space of solutions to this ODE has some nice algebraic properties.

\begin{theorem}
The space of solutions of a homogeneous linear ODE is a real vector space; that is, if $y_1$ and $y_2$ are solutions, then $\alpha_1y_1+\alpha_2y_2$ is also a solution, for $\alpha_1,\alpha_2\in\RR$.
\end{theorem}

\begin{proof}
Suppose
\begin{equation*}\tag{1}
a_k(x)\dv[k]{y_1}{x}+a_{k-1}(x)\dv[k-1]{y_1}{x}+\cdots+a_1(x)\dv{y_1}{x}+a_0(x)y_1=0,
\end{equation*}
\begin{equation*}\tag{2}
a_k(x)\dv[k]{y_2}{x}+a_{k-1}(x)\dv[k-1]{y_2}{x}+\cdots+a_1(x)\dv{y_2}{x}+a_0(x)y_2=0.
\end{equation*}
$\alpha_1\times(1)+\alpha_2\times(2)$ gives
\[a_k(x)\dv[k]{(\alpha_1y_1+\alpha_2y_2)}{x}+a_{k-1}(x)\dv[k-1]{(\alpha_1y_1+\alpha_2y_2)}{x}+\cdots+a_1(x)\dv{(\alpha_1y_1+\alpha_2y_2)}{x}+a_0(x)(\alpha_1y_1+\alpha_2y_2)=0,\]
which shows that $\alpha_1y_1+\alpha_2y_2$ is also a solution to the ODE.
\end{proof}

\begin{remark}
The above holds simply due to differentiation being a linear map.
\end{remark}

In the case where the ODE is linear but inhomogeneous, solving the inhomogeneous equation still strongly relates to the solution of the associated homogeneous equation.

\begin{theorem}
Let $y_p(x)$ be a solution (known as a \emph{particular integral}) of the homogeneous ODE
\[a_k(x)\dv[k]{y}{x}+a_{k-1}(x)\dv[k-1]{y}{x}+\cdots+a_1(x)\dv{y}{x}+a_0(x)y=f(x).\]
Then $y(x)$ is a solution if and only if $y(x)$ can be written as
\[y(x)=y_c(x)+y_p(x),\]
where $y_c(x)$ is a solution of the corresponding homogeneous linear ODE (known as the \emph{complementary function}).
\end{theorem}

\begin{proof}
If $y(x)=y_c(x)+y_p(x)$ is a solution, then
\[a_k(x)\dv[k]{(y_c+y_p)}{x}+a_{k-1}(x)\dv[k-1]{(y_c+y_p)}{x}+\cdots+a_1(x)\dv{(y_c+y_p)}{x}+a_0(x)(y_c+y_p)=f(x).\]
Rearranging brackets,
\[\brac{a_k(x)\dv[k]{y_c}{x}+\cdots+a_0(x)y_c}+\brac{a_k(x)\dv[k]{y_p}{x}+\cdots+a_0(x)y_p}=f(x),\]
where the second bracket equals $f(x)$, since $y_p(x)$ is a solution. Hence the first bracket must equal zero, that is $y_c(x)$ is a solution of the corresponding homogeneous ODE.
\end{proof}

We now introduce a method for finding a second solution to a second order homogeneous linear ODE, when one solution has already been found.

Suppose $z(x)\neq0$ is a non-trivial solution to
    \chapter{Multivariable Calculus}
\section{Partial Differentiation}
\subsection{Computation of Partial Derivatives}
\begin{definition}[Partial derivative]
Let $f:\RR^n\to\RR$ be a function of $n$ variables. The \vocab{partial derivative} of $f$ with respect to the $i$-th variable at the point $(p_1,\dots,p_n)$ is the function
\[\pdv{f}{x_i}(p_1,\dots,p_n)=\lim_{h\to0}\frac{f(p_1,\dots,p_i+h,\dots,p_n)-f(p_1,\dots,p_n)}{h}.\]
\end{definition}

\begin{notation}
We shall occasionally write $f_x$ for $\pdv{f}{x}$, etc.
\end{notation}

\begin{remark}
If $f(x)$ is a function of a single variable, then $\dv{f}{x}=\pdv{f}{x}$.
\end{remark}

\begin{notation}
We define second and higher order partial derivatives in a similar manner to how we define them for full derivatives. So in the case of second order partial derivatives of a function $f(x,y)$, we have
\begin{align*}
\pdv[2]{f}{x}&=\pdv{}{x}\brac{\pdv{f}{x}}=f_{xx},\\
\pdv[2]{f}{y}&=\pdv{}{y}\brac{\pdv{f}{y}}=f_{yy},\\
\pdv{f}{y,x}&=\pdv{}{y}\brac{\pdv{f}{x}}=f_{xy},\\
\pdv{f}{x,y}&=\pdv{}{x}\brac{\pdv{f}{y}}=f_{yx}.
\end{align*}
\end{notation}

\begin{proposition}[Clairaut's theorem]
If $f_{xy}$ and $f_{yx}$ are both defined and continuous in a region containing the point $(a,b)$, then 
\[f_{xy}(a,b)=f_{yx}(a,b).\]
\end{proposition}

A consequence of this theorem is that we don't need to keep track of the order in which we take derivatives.

Recall the chain rule for one variable, which arises when we want to find the derivative of the composition of two functions $f\brac{u(x)}$ with respect to $x$. 
Likewise we might have a function $f(u,v)$ of two variables $u$ and $v$, each of which is itself a function of $x$ and $y$; thus make the composition
\[F(x,y)=f\brac{u(x,y),v(x,y)},\]
which is a function of $x$ and $y$, and we might then want to calculate the partial derivatives $\pdv{F}{x}$ and $\pdv{F}{y}$.

\begin{lemma}[Chain rule]
Let $F(t)=f\brac{u(t),v(t)}$ with $u$ and $v$ differentiable and $f$ being continuously differentiable in each variable. Then
\begin{equation}
\dv{F}{t}=\pdv{f}{u}\dv{u}{t}+\pdv{f}{v}\dv{v}{t}.
\end{equation}
\end{lemma}

\begin{corollary}
Let $F(x,y)=f\brac{u(x,y),v(x,y)}$, with $u$ and $v$ differentiable in each variable, and $f$ being continuously differentiable in each variable. Then
\begin{equation}
\begin{split}
\pdv{F}{x}&=\pdv{f}{u}\pdv{u}{x}+\pdv{f}{v}\pdv{v}{x},\\
\pdv{F}{y}&=\pdv{f}{u}\pdv{u}{y}+\pdv{f}{v}\pdv{v}{y}.
\end{split}
\end{equation}
\end{corollary}

\begin{proof}
This follows from the previous theorem by treating first $x$ and then $y$ as constants when differentiating.
\end{proof}

\subsection{Partial Differential Equations}

\pagebreak

\section{Multiple Integrals}
\begin{definition}
By a \vocab{scalar field} $\phi$ on $\RR^3$ we shall mean a map $\phi:\RR^3\to\RR$; by a \vocab{vector field} $\vb{F}$ on $\RR^3$ we shall mean a map $\vb{F}:\RR^3\to\RR^3$.
\end{definition}

\begin{remark}
We will typically We will typically assume that scalar and vector fields are smooth -- their partial derivatives exist with respect to $x$, $y$ and $z$ to all orders -- for brevity, this will not always be stated.

Occasionally we may consider more general scalar fields $\phi:\RR^n\to\RR$ and vector fields $\vb{F}:\RR^n\to\RR^m$.
\end{remark}

\subsection{Double Integrals}
An informal definition of a double integral is as follows:

Consider a region $R\subset\RR^2$, together with a scalar field $\phi(x,y)$. Partition the region into $n$ area elements of equal area $\delta A=\delta x\delta y$. Suppose the scalar field $\phi(\vb{r})$ takes $\phi_i$ at the centre of the $i$-th element.

On partitioning with smaller and smaller area elements, by taking the limit $\delta A\to0$,
\[\lim\sum_{i=1}^{n}\phi_i\delta A=\lim\sum_{i=1}^{n}\phi_i\delta x\delta y\coloneqq\iint_R\phi\dd{A}=\iint_R\phi\dd{x}\dd{y}.\]

\begin{proposition}[Properties of double integrals]
The following properties are inherited from integration with respect to one variable.
\begin{enumerate}[label=(\roman*)]
\item Linearity: for $a,b\in\RR$,
\[\iint_R\brac{af(x,y)+bg(x,y)}\dd{A}=a\iint_Rf(x,y)\dd{A}+b\iint_Rg(x,y)\dd{A}.\]
\item Order: if $f(x,y)\ge g(x,y)$ for all $(x,y)\in R$ then
\[\iint_Rf(x,y)\dd{A}\ge\iint_Rg(x,y)\dd{A}.\]
\item Domain splitting: if $R=R_1\cup R_2$ and $R_1\cap R_2=\emptyset$ then
\[\iint_Rf(x,y)\dd{A}=\iint_{R_1}f(x,y)\dd{A}+\iint_{R_2}f(x,y)\dd{A}.\]
\end{enumerate}
\end{proposition}

\begin{proposition}[Fubini's theorem]
Given a continuous, bounded function $f(x,y)$ on a rectangular domain $R=X\times Y$, then
\begin{equation}
\iint_{R}f(x,y)\dd{A}=\int_{Y}\int_{X}f(x,y)\dd{x}\dd{y}=\int_{X}\int_{Y}f(x,y)\dd{y}\dd{x}.
\end{equation}
\end{proposition}

Sometimes it may be more convenient to work with coordinates other than Cartesian, so we would need a change of variables of integration. 
The \emph{Jacobian}\footnote{after the German mathematician Carl Jacobi (1804--1851)}, or rather its modulus, is a measure of how a general mapping stretches space locally, near a particular point, even when this stretching effect varies from point to point.

\begin{definition}[Jacobian]
Given two coordinates $u(x,y)$ and $v(x,y)$ which depend on variables $x$ and $y$, the \vocab{Jacobian} is defined to be the determinant
\[\frac{\partial(u,v)}{\partial(x,y)}=\begin{vmatrix}
\pdv{u}{x}&\pdv{u}{y}\\
\pdv{v}{x}&\pdv{v}{y}
\end{vmatrix}.\]
In 3D,
\[\frac{\partial(u,v,w)}{\partial(x,y,z)}=\begin{vmatrix}
\pdv{u}{x}&\pdv{u}{y}&\pdv{u}{z}\\
\pdv{v}{x}&\pdv{v}{y}&\pdv{v}{z}\\
\pdv{w}{x}&\pdv{w}{y}&\pdv{w}{z}
\end{vmatrix}.\]
\end{definition}

That is to say, under the transformation $(x,y)\to(u,v)$, the area element $\dd{x}\dd{y}$ in the $xy$-plane is equivalent to $\displaystyle\absolute{\frac{\partial(u,v)}{\partial(x,y)}}\dd{u}\dd{v}$, where $\dd{u}\dd{v}$ is the area element in $uv$-plane. If $A$ is a domain in the $xy$-plane mapped one to one and onto a domain $B$ in the $uv$-plane then
\[\int_Af(x,y)\dd{x}\dd{y}=\int_Bf\brac{x(u,v),y(u,v)}\absolute{\frac{\partial(u,v)}{\partial(x,y)}}\dd{u}\dd{v}.\]

\begin{example}[Plane polar coordinates]
For $P=(x,y)\in\RR^2\setminus\{(0,0)\}$ we can determine the position of $P$ by its distance $r$ from the origin $(0,0)$ and the anti-clockwise angle $\theta$ that $\overrightarrow{OP}$ makes with the $x$-axis.

Let
\[x=r\cos\theta,\quad y=r\sin\theta.\]
Then the Jacobian is
\[\frac{\partial(x,y)}{\partial(r,\theta)}=\begin{vmatrix}
\cos\theta&-r\sin\theta\\
\sin\theta&r\cos\theta
\end{vmatrix}=r.\]
\end{example}

Notice that
\[\frac{\partial(x,y)}{\partial(r,\theta)}\frac{\partial(r,\theta)}{\partial(x,y)}=1.\]
This result holds more generally:

\begin{proposition}
Let $r$ and $s$ be functions of variables $u$ and $v$, which are in turn functions of $x$ and $y$. Then
\[\frac{\partial(r,s)}{\partial(x,y)}=\frac{\partial(r,s)}{\partial(u,v)}\frac{\partial(u,v)}{\partial(x,y)}.\]
\end{proposition}

\begin{example}[Parabolic coordinates]
Let
\[x=\frac{1}{2}(u^2-v^2),\quad y=uv.\]

Then the Jacobian is
\[\frac{\partial(x,y)}{\partial(u,v)}=\begin{vmatrix}
u&-v\\
v&u
\end{vmatrix}=u^2+v^2.\]
\end{example}

We can naturally extend plane polar coordinates into three dimensions by adding a $z$ coordinate.

\begin{example}[Cylindrical polar coordinates]
Let
\[x=r\cos\theta,\quad y=r\sin\theta,\quad z=z.\]
Then the Jacobian is
\[\frac{\partial(x,y,z)}{\partial(r,\theta,z)}=\begin{vmatrix}
\cos\theta&-r\sin\theta&0\\
\sin\theta&r\cos\theta&0\\
0&0&1
\end{vmatrix}=r.\]
\end{example}

\begin{example}[Spherical polar coordinates]
Let $(x,y,z)$ be the Cartesian coordinates for a general point $P\in\RR^3\setminus\{(0,0,0)\}$. Let $r$ be the distance between $P$ and $O$, let $\theta$ be the angle from the $z$-axis to $\overrightarrow{OP}$. Change $(x,y)$ to its polar coordinates $(\rho\cos\phi,\rho\sin\phi)$, so that $\rho=r\sin\theta$.

Let
\[x=r\sin\theta\cos\phi,\quad y=r\sin\theta\sin\phi,\quad z=r\cos\theta.\]

Then the Jacobian is
\[\frac{\partial(x,y,z)}{\partial(r,\theta,\phi)}=\begin{vmatrix}
\sin\theta\cos\phi&r\cos\theta\cos\phi&-r\sin\theta\sin\phi\\
\sin\theta\sin\phi&r\cos\theta\sin\phi&r\sin\theta\cos\phi\\
\cos\theta&-r\sin\theta&0
\end{vmatrix}=r^2\sin\theta.\]
\end{example}

\subsection{Volume Integrals}
Consider a scalar field $\phi(x,y,z)$, and a three-dimensional region $D\subset\RR^3$. Partition $V$ into $n$ cubic volume elements of equal volume $\delta V=\delta x\delta y\delta z$. Let $\phi_i$ denote the value of $\phi$ at the centre of the $i$-th volume element. On partitioning with smaller and smaller volume elements, and taking the limit $\delta V\to0$,
\[\lim\sum_{i=1}^{n}\phi_i\delta V\coloneqq\iiint_{D}\phi\dd{V}.\]

\section{Surfaces}
\subsection{Parametric Representation}
\begin{definition}[Tangent plane]
Let $\vb{r}:U\to\RR^3$ be a smooth parametrised surface\footnote{A \emph{smooth parametrised surface} is a map $\vb{r}$, given by the parameterisation \[r:U\to\RR^3:(u,v)\mapsto\brac{x(u,v),y(u,v),z(u,v)},\] where $U$ is an open subset of $\RR^2$, such that
\begin{enumerate}[label=(\roman*)]
\item $x$, $y$, $z$ have continuous partial derivatives with respect to $u$ and $v$ of all orders;
\item $\vb{r}$ is a bijection, with both $\vb{r}$ and $\vb{r}^{-1}$ being continuous;
\item at each point the vectors $\pdv{\vb{r}}{u}$ and $\pdv{\vb{r}}{v}$ are linearly independent.
\end{enumerate}
}, and let $\vb{p}$ be a point on the surface. The plane containing $\vb{p}$ and which is parallel to the vectors
\[\pdv{\vb{r}}{u}(\vb{p})\quad\text{and}\quad\pdv{\vb{r}}{v}(\vb{p})\]
is called the \vocab{tangent plane}\index{tangent plane} to $\vb{r}(U)$ at $\vb{p}$.
\end{definition}

\begin{remark}
For a smooth parametrised surface, these vectors are linearly independent, so the tangent plane is well-defined.
\end{remark}

\begin{definition}[Normal vector]
Any vector in the direction
\[\pdv{\vb{r}}{u}(\vb{p})\times\pdv{\vb{r}}{v}(\vb{p})\]
is said to be \vocab{normal} to the surface at $\vb{p}$.
\end{definition}

\subsection{Scalar Line Integrals}
In this section we will introduce the idea of the scalar line integral of a vector field. The physical interpretation of such integrals depends on the nature of the particular vector field under consideration. In the case of force fields, the scalar line integral represents the work done by the force.

\begin{definition}[Scalar line integral]
The \vocab{scalar line integral} of a vector field $\vb{F}(\vb{r})$ along a path $C$ given by $\vb{r}=\vb{r}(t)$, from $\vb{r}(t_0)$ to $\vb{r}(t_1)$, is
\begin{equation}
\int_C\vb{F}(\vb{r})\cdot\dd{\vb{r}}=\int_{t_0}^{t_1}\vb{F}(t)\cdot\dv{\vb{r}}{t}\dd{t}.
\end{equation}
\end{definition}

The path $C$ in the integral is a directed curve, with start point $A$ and end point $B$; thus we often write the integral as
\[\int_{A}^{B}\vb{F}(\vb{r})\cdot\dd{\vb{r}}.\]
Traversing the same path in the opposite direction changes the sign of the scalar line integral, so
\[\int_{B}^{A}\vb{F}(\vb{r})\cdot\dd{\vb{r}}=-\int_{A}^{B}\vb{F}(\vb{r})\cdot\dd{\vb{r}}.\]

\subsection{Length of Curve}
Consider the scalar line integral of the vector field $\vb{F}(\vb{r})$ along the curve $C$ given by $\vb{r}(t)$, from $\vb{r}(t_0)$ to $\vb{r}(t_1)$, namely
\[\int_C\vb{F}(\vb{r})\cdot\dd{\vb{r}}=\int_{t_0}^{t_1}\vb{F}(t)\cdot\dv{\vb{r}}{t}\dd{t},\]
and let $t$ represent time. Then $\vb{r}(t)$ represents the point on the curve $C$ corresponding to time $t$, and $\dot{\vb{r}}(t)$ represents the velocity of the point as it moves along the curve. Now let
\[\vb{F}(t)=\frac{\dot{\vb{r}}(t)}{|\dot{\vb{r}}(t)|};\]
then $\vb{F}(t)$ represents a unit vector in the direction of $\dot{\vb{r}}(t)$. We have
\[\vb{F}(t)\cdot\dv{\vb{r}}{t}=\frac{\dot{\vb{r}}(t)}{|\dot{\vb{r}}(t)|}\cdot\dot{\vb{r}}(t)=|\dot{\vb{r}}(t)|,\]
so that the scalar line integral becomes
\begin{equation}
\int_C\vb{F}(\vb{r})\cdot\dd{\vb{r}}=\int_{t_0}^{t_1}|\dot{\vb{r}}(t)|\dd{t}=\int_{t_0}^{t_1}\sqrt{\brac{\dv{x}{t}}^2+\brac{\dv{y}{t}}^2}\dd{t}.
\end{equation}

\subsection{Surface Integrals}
Let $\vb{r}:U\to\RR^3$ be a smooth parametrised surface with
\[\vb{r}(u,v)=\brac{x(u,v),y(u,v),z(u,v)}\]

\section{Line Integrals}
\begin{definition}
By a \vocab{curve} we shall mean a piecewise smooth function $\gamma:I\to\RR^3$ defined on an interval $I$ of $\RR$. Notice that order on $I$ also gives the curve $\gamma$ an \emph{orientation}.
\end{definition}

We shall also use the term curve to describe the images of such maps $\gamma$. Given such an image then it will be the image of more than one such map $\gamma$ and we will talk about parameterisations $\gamma_1$ and $\gamma_2$ of the curve.



\section{Grad, Div, Curl}
\subsection{Definitions and Identities}
The differential operator
\[\nabla=\brac{\pdv{}{x},\pdv{}{y},\pdv{}{z}}\]
is called \emph{del} or \emph{nabla}.

\begin{definition}[Gradient]
Let $\phi:\RR^3\to\RR$ be a scalar field. Then the \vocab{gradient} of $\phi$ is
\[\nabla\phi\coloneqq\brac{\pdv{\phi}{x},\pdv{\phi}{y},\pdv{\phi}{z}}.\]
\end{definition}

Note $\nabla$ takes scalar fields to vector fields.

\begin{definition}[Divergence]
Let $\vb{F}:\RR^3\to\RR^3$ be a vector field, with $\vb{F}=(F_1,F_2,F_3)$. Then the \vocab{divergence} of $\FF$ is
\[\nabla\cdot\vb{F}\coloneqq\pdv{F_1}{x}+\pdv{F_2}{y}+\pdv{F_3}{z}.\]
\end{definition}

Note $\nabla\cdot$ takes vector fields to scalar fields.

\begin{definition}[Curl]
Let $\vb{F}:\RR^3\to\RR^3$ be a vector field, with $\vb{F}=(F_1,F_2,F_3)$. Then the \vocab{curl} of $\vb{F}$ is
\[\nabla\times\vb{F}=\begin{vmatrix}
\vb{i}&\vb{j}&\vb{k}\\
\pdv{}{x}&\pdv{}{y}&\pdv{}{z}\\
F_1&F_2&F_3
\end{vmatrix}=
\brac{\pdv{F_3}{y}-\pdv{F_2}{z},\pdv{F_1}{z}-\pdv{F_3}{x},\pdv{F_2}{x}-\pdv{F_1}{y}}.\]
\end{definition}

Note $\nabla\times$ takes vector fields to vector fields.

We also have the differential operator
\[\vb{F}\cdot\nabla=(F_1,F_2,F_3)\cdot\brac{\pdv{}{x},\pdv{}{y},\pdv{}{z}}=F_1\pdv{}{x}+F_2\pdv{}{y}+F_3\pdv{}{z},\]
which gives the directional derivative in the direction of $\vb{F}$ if $\vb{F}$ is a unit vector.

\begin{definition}[Laplacian]
For any scalar field, the \vocab{Laplacian} is
\[\nabla^2\phi\coloneqq\nabla\cdot\nabla\phi=\pdv[2]{\phi}{x}+\pdv[2]{\phi}{y}+\pdv[2]{\phi}{z}.\]
\end{definition}

\begin{notation}
There are neater ways to write out the above formulae:
\[\nabla\phi=\sum_i\pdv{\phi}{x_i}\vb{e}_i,\quad\nabla\cdot\vb{F}=\sum_i\vb{e}_i\cdot\pdv{\vb{F}}{x_i},\quad\nabla\times\vb{F}=\sum_i\vb{e}_i\times\pdv{\vb{F}}{x_i},\quad\vb{F}\times\nabla=\sum_i(\vb{F}\cdot\vb{e}_i)\pdv{}{x_i},\]
where the dummy variable $i$ ranges over $1,2,3$, and $\vb{e}_1,\vb{e}_2,\vb{e}_3$ is any right-handed orthonormal basis.
\end{notation}

\begin{proposition}
Let $\phi$ and $\psi$ be differentiable functions of $x,y,z$. Then
\begin{enumerate}[label=(\roman*)]
\item $\nabla(\phi\psi)=\phi\nabla\psi+\psi\nabla\phi$;
\item $\nabla(\phi^n)=n\phi^{n-1}\nabla\phi$;
\item $\nabla\brac{\frac{\phi}{\psi}}=\frac{\psi\nabla\phi-\phi\nabla\psi}{\psi^2}$;
\item $\nabla\brac{\phi(\psi(\vb{x}))}=\phi^\prime(g(\vb{x}))\nabla g(\vb{x})$.
\end{enumerate}
\end{proposition}

\begin{proposition}
Let $\FF$ be a vector field on $\RR^3$, $\phi$ be a scalar field on $\RR^3$. Then
\begin{enumerate}[label=(\roman*)]
\item $\nabla\times\nabla\phi=\vb{0}$;
\item $\nabla\cdot(\nabla\times\vb{F})=0$.
\end{enumerate}
\end{proposition}

\begin{proof} \
\begin{enumerate}[label=(\roman*)]
\item \[\nabla\times\nabla\phi
=\begin{vmatrix}
\vb{i}&\vb{j}&\vb{k}\\
\pdv{}{x}&\pdv{}{y}&\pdv{}{z}\\
\pdv{\phi}{x}&\pdv{\phi}{y}&\pdv{\phi}{z}
\end{vmatrix}=\brac{\pdv{\phi}{y,z}-\pdv{\phi}{z,y},\pdv{\phi}{z,x}-\pdv{\phi}{x,z},\pdv{\phi}{x,y}-\pdv{\phi}{y,x}}=\vb{0}.\]

\item \begin{align*}
\nabla\cdot(\nabla\times\vb{F})
&=\nabla\cdot\brac{\pdv{F_3}{y}-\pdv{F_2}{z},\pdv{F_1}{z}-\pdv{F_3}{x},\pdv{F_2}{x}-\pdv{F_1}{y}}\\
&=\brac{\pdv{F_3}{x,y}-\pdv{F_2}{x,z}}+\brac{\pdv{F_1}{y,z}-\pdv{F_3}{y,x}}+\brac{\pdv{F_2}{z,x}-\pdv{F_1}{z,y}}\\
&=\brac{\pdv{F_1}{y,z}-\pdv{F_1}{z,y}}+\brac{\pdv{F_2}{z,x}-\pdv{F_2}{x,z}}+\brac{\pdv{F_3}{x,y}-\pdv{F_3}{y,x}}=0.
\end{align*}
\end{enumerate}
\end{proof}

\begin{proposition}[Product rules]
Let $\vb{F}$ be a vector field on $\RR^3$, $\phi$ be a scalar field on $\RR^3$. Then
\begin{enumerate}[label=(\roman*)]
\item $\nabla\cdot(\phi\vb{F})=\nabla\phi\cdot\vb{F}+\phi\nabla\cdot\vb{F}$;
\item $\nabla\times(\phi\vb{F})=\phi\nabla\times\vb{F}+\nabla\phi\times\vb{F}$.
\end{enumerate}
\end{proposition}

\begin{proposition}[Further identities]
Let $\vb{F}$ and $\vb{G}$ be vector fields on $\RR^3$. Then
\begin{enumerate}[label=(\roman*)]
\item $\nabla(\vb{F}\cdot\vb{G})=(\vb{F}\cdot\nabla)\vb{G}+(\vb{G}\cdot\nabla)\vb{F}+\vb{F}\times(\nabla\times\vb{G})+\vb{G}\times(\nabla\times\vb{F})$;
\item $\nabla\cdot(\vb{F}\times\vb{G})=\vb{G}\cdot(\nabla\times\vb{F})-\vb{F}\cdot(\nabla\times\vb{G})$;
\item $\nabla\times(\vb{F}\times\vb{G})=\vb{F}(\nabla\cdot\vb{G})-\vb{G}(\nabla\cdot\vb{F})+(\vb{G}\cdot\nabla)\vb{F}-(\vb{F}\cdot\nabla)\vb{G}$;
\item $\nabla\times(\nabla\times\vb{F})=\nabla(\nabla\cdot\vb{F})-\nabla^2\vb{F}$.
\end{enumerate}
\end{proposition}

\begin{proposition}
Consider a given vector field $\vb{F}$ for which there exists a scalar function $\phi$ such that $\vb{F}=\nabla\phi$. Then
\begin{equation}
\int_{A}^{B}\nabla\phi\cdot\dd{\vb{r}}=\phi(B)-\phi(A),
\end{equation}
where $A$ and $B$ are the start and end points, respectively, of the curve along which we are integrating.
\end{proposition}

\begin{proof}
By the chain rule, we have
\begin{align*}
\int_{A}^{B}\nabla\phi\cdot\dd{\vb{r}}
&=\int_{A}^{B}\brac{\pdv{\phi}{x},\pdv{\phi}{y},\pdv{\phi}{z}}\cdot\brac{\dv{x}{t},\dv{y}{t},\dv{z}{t}}\dd{t}\\
&=\int_{A}^{B}\brac{\pdv{\phi}{x}\dv{x}{t}+\pdv{\phi}{y}\dv{y}{t}+\pdv{\phi}{z}\dv{z}{t}}\dd{t}\\
&=\int_{A}^{B}\dv{\phi}{t}\dd{t}\\
&=\phi(B)-\phi(A).
\end{align*}
\end{proof}

\begin{remark}
Note that this result is independent of the curve itself -- the scalar line integrals for which this result holds are \emph{path-independent}.
\end{remark}

\begin{definition}[Directional derivative]
Let $\phi:\RR^3\to\RR$ be a differentiable scalar function, let $\vb{u}$ be a unit vector. Then the \vocab{directional derivative} of $\phi$ at $\vb{a}$ in the direction $\vb{u}$ is
\[\lim_{t\to0}\frac{\phi(\vb{a}+t\vb{u})-\phi(\vb{a})}{t}.\]
This is the rate of change of the function $\phi$ at $\vb{a}$ in the direction $\vb{u}$.
\end{definition}

\begin{proposition}
The directional derivative of $\phi$ at $\vb{a}$ in direction $\vb{u}$ equals $\nabla\phi(\vb{a})\cdot\vb{u}$.
\end{proposition}

\begin{proof}
Let
\begin{align*}
\Phi(t)&=\phi(\vb{a}+t\vb{u})\\
&=\phi\brac{a_1+tu_1,a_2+tu_2,a_3+tu_3}.
\end{align*}
where $\vb{a}=(a_1,a_2,a_3)$, $\vb{u}=(u_1,u_2,u_3)$. Then by definition of single variable derivative,
\[\lim_{t\to0}\frac{\phi(\vb{a}+t\vb{u})-\phi(\vb{a})}{t}=\lim_{t\to0}\frac{\Phi(t)-\Phi(0)}{t}=\Phi^\prime(0).\]
Now, by the chain rule,
\begin{align*}
\Phi^\prime(0)&=\dv{\Phi}{t}\bigg|_{t=0}\\
&=\pdv{\phi}{x}\dv{x}{t}+\pdv{\phi}{y}\dv{y}{t}+\pdv{\phi}{z}\dv{z}{t}\bigg|_{t=0}\\
&=\pdv{\phi}{x}(\vb{a})\dv{x}{t}+\pdv{\phi}{y}(\vb{a})\dv{y}{t}+\pdv{\phi}{z}(\vb{a})\dv{z}{t}\\
&=\pdv{\phi}{x}(\vb{a})u_1+\pdv{\phi}{y}(\vb{a})u_2+\pdv{\phi}{z}(\vb{a})u_3\\
&=\brac{\pdv{\phi}{x}(\vb{a}),\pdv{\phi}{y}(\vb{a}),\pdv{\phi}{z}(\vb{a})}\cdot\brac{u_1,u_2,u_3}\\
&=\nabla\phi(\vb{a})\cdot\vb{u}.
\end{align*}
\end{proof}

\begin{corollary}
The rate of change of $\phi$ is the greatest in the direction $\nabla\phi$; that is when $\vb{u}=\frac{\nabla\phi}{|\nabla\phi|}$, and the maximum rate of change is given by $|\nabla\phi|$.
\end{corollary}

\begin{definition}[Level set]
A \vocab{level set} of $\phi:\RR^3\to\RR$ is a set of points
\[\{(x,y,z)\in\RR^3\mid \phi(x,y,z)=c\}\]
for some $c\in\RR$. For suitably well behaved functions $\phi$ and constants $c$, the level set is a
surface in $\RR^3$.
\end{definition}

\begin{proposition}
Given a surface $S\subset\RR^3$ with equation $\phi(x,y,z)=c$ and a point $\vb{p}\in S$, then $\nabla\phi(\vb{p})$ is normal to $S$ at $\vb{p}$.
\end{proposition}

\begin{proof}
Let $u$ and $v$ be coordinates near $\vb{p}$, and let $\vb{r}:(u,v)\to\vb{r}(u,v)$ be a parameterisation of part of $S$. Recall that the normal to $S$ at $\vb{p}$ is in the direction
\[\pdv{\vb{r}}{u}\times\pdv{\vb{r}}{v}.\]
Note also that $\phi\brac{\phi(u,v)}=c$, and so $\pdv{\phi}{u}=\pdv{\phi}{v}=0$. If we write $\vb{r}(u,v)=\brac{x(u,v),y(u,v),z(u,v)}$ then we see that
\begin{align*}
\nabla\phi\cdot\pdv{\vb{r}}{u}
&=\brac{\pdv{\phi}{x},\pdv{\phi}{y},\pdv{\phi}{z}}\cdot\brac{\pdv{x}{u},\pdv{y}{u},\pdv{z}{u}}\\
&=\pdv{\phi}{x}\pdv{x}{u}+\pdv{\phi}{y}\pdv{y}{u}+\pdv{\phi}{z}\pdv{z}{u}\\
&=\pdv{\phi}{u}=0,
\end{align*}
where the penultimate line follows from the chain rule. Similarly,
\[\nabla\phi\cdot\pdv{\vb{r}}{v}=0,\]
and hence $\nabla\phi$ is in the direction of $\pdv{\vb{r}}{u}\times\pdv{\vb{r}}{v}$, and so is normal to the surface $S$.
\end{proof}

\subsection{Divergence and Stokes' Theorems}
\begin{theorem}[Divergence Theorem]
Let $V\subset\RR^3$ with piecewise smooth boundary $\partial V$. Let $\vb{F}$ be a differentiable vector field on $V$. Then
\begin{equation}
\iiint_V\nabla\cdot\vb{F}\dd{V}=\iint_{\partial V}\vb{F}\cdot\dd{\vb{S}},
\end{equation}
where $\dd{\vb{S}}$ is oriented in the direction of the outward pointing normal from $V$.
\end{theorem}
\pagebreak

\section*{Exercises}
\begin{prbm}
Calculate the double integral
\[\int_{-\infty}^{\infty}\int_{-\infty}^{\infty}e^{-x^2-y^2}\dd{x}\dd{y}\]
via a change to polar coordinates. Hence deduce that
\[\int_{-\infty}^{\infty}e^{-s^2}\dd{s}=\sqrt{\pi}.\]
\end{prbm}

\begin{solution}
On changing to polar coordinates,
\begin{align*}
\int_{-\infty}^{\infty}\int_{-\infty}^{\infty}e^{-x^2-y^2}\dd{x}\dd{y}
&=\int_{0}^{\infty}\int_{0}^{2\pi}e^{-r^2}\dd{\theta}\dd{r}\\
&=2\pi\int_{0}^{\infty}e^{-r^2}r\dd{r}\\
&=-\pi\sqbrac{e^{-r^2}}_{0}^{\pi}=\pi.
\end{align*}
Hence we can write
\begin{align*}
\pi&=\int_{-\infty}^{\infty}\int_{-\infty}^{\infty}e^{-x^2-y^2}\dd{x}\dd{y}\\
&=\int_{-\infty}^{\infty}\int_{-\infty}^{\infty}e^{-x^2}e^{-y^2}\dd{x}\dd{y}\\
&=\brac{\int_{-\infty}^{\infty}e^{-x^2}\dd{x}}\brac{\int_{-\infty}^{\infty}e^{-y^2}\dd{y}}\\
&=\brac{\int_{-\infty}^{\infty}e^{-x^2}\dd{x}}^2,
\end{align*}
from which the final result follows. Note that we have used the fact that the double integral is \emph{separable}.
\end{solution}
\fi

%%%%%%%%%%%%%%% ABSTRACT ALGEBRA
\ifabsalg
    \part{Abstract Algebra}\label{part:abstract-algebra}
    \chapter{Groups}\label{chap:groups}
\section{Introduction to Groups}
\subsection{Definitions and Properties}
\begin{definition}[Binary operation]
A \vocab{binary operation} $\ast$ on a set $G$ is a function $\ast:G\times G\to G$. For any $a,b\in G$, we write $a \ast b$ for the image of $(a,b)$ under $\ast$.

$\ast$ is \emph{associative} on $G$ if $(a\ast b)\ast c=a\ast(b\ast c)$ for all $a,b,c\in G$.

$\ast$ is \emph{commutative} on $G$ if $a\ast b=b\ast a$ for all $a,b\in G$.
\end{definition}

\begin{definition}[Group]
A \vocab{group}\index{group} $(G,\ast)$ consists of a set $G$ and a binary operation $\ast$ on $G$ satisfying the following group axioms:
\begin{enumerate}[label=(\roman*)]
\item Associativity: $a\ast(b\ast c)=(a\ast b)\ast c$ for all $a,b,c\in G$.
\item Identity: there exists identity element $e\in G$ such that $a\ast e=e\ast a=a$ for all $a\in G$.
\item Invertibility: for all $a\in G$, there exists inverse $c\in G$ such that $a\ast c=c\ast a=e$.
\end{enumerate}

$G$ is \vocab{abelian}\footnote{after the Norwegian mathematician Niels 
Abel (1802--1829)} if the operation is commutative; it is \emph{non-abelian} if otherwise.
\end{definition}

\begin{remark}
When verifying that $(G,\ast)$ is a group we have to check (i), (ii), (iii) above and also that $\ast$ is a binary operation closed in $G$---that is, $a\ast b\in G$ for all $a,b\in G$.
\end{remark}

\begin{notation}
We simply denote a group $(G,\ast)$ by $G$ if the operation is clear.
\end{notation}

\begin{notation}
We abbreviate $a\ast b$ to just $ab$ if the operation is clear.
\end{notation}

\begin{notation}
Since $\ast$ is associative, we omit unnecessary parentheses and write $(ab)c=a(bc)=abc$.
\end{notation}

\begin{notation}
For any $a\in G$, $n\in\ZZ^+$ we denote $a^n=\underbrace{a\cdot a\cdots a}_\text{$n$ times}$.
\end{notation}

\begin{notation}
We write $(\ZZ,+)$, $(\QQ,+)$, $(\RR,+)$, $(\CC,+)$ as simply $\ZZ$, $\QQ$, $\RR$, $\CC$.
\end{notation}

\begin{example}
\begin{itemize}
\item $\ZZ$, $\QQ$, $\RR$, $\CC$ are groups, with identity $0$ and (additive) inverse $-a$ for all $a$.
\item $\QQ\setminus\{0\}$, $\RR\setminus\{0\}$, $\CC\setminus\{0\}$, $\QQ^+$, $\RR^+$ are groups under $\times$, with identity $1$ and (multiplicative) inverse $\frac{1}{a}$ for all $a$; $\ZZ\setminus\{0\}$ is not a group under $\times$, because all elements except for $\pm1$ do not have an inverse in $\ZZ\setminus\{0\}$.
\item For $n\in\ZZ^+$, $\ZZ_n$ is an abelian group under $+$.
\item For $n\in\ZZ^+$, $(\ZZ_n)^\times$ is an abelian group under multiplication.
\end{itemize}
\end{example}

\begin{proposition}
Let $G$ be a group. Then
\begin{enumerate}[label=(\roman*)]
\item the identity of $G$ is unique,
\item for each $a\in G$, $a^{-1}$ is unique,
\item $(a^{-1})^{-1}=a$ for all $a\in G$,
\item $(ab)^{-1}=b^{-1}a^{-1}$,
\item for any $a_1,\dots,a_n\in G$, $a_1\cdots a_n$ is independent of how we arrange the parantheses (generalised associative law).
\end{enumerate}
\end{proposition}

\begin{proof} \
\begin{enumerate}[label=(\roman*)]
\item Suppose otherwise, then $e$ and $e^\prime$ are identites of $G$. We have
\[e=ee^\prime=e^\prime\]
where the first equality holds as $e^\prime$ is an identity, and the second equality holds as $e$ is an identity. Since $e=e^\prime$, the identity is unique.
\item Suppose otherwise, then $b$ and $c$ are both inverses of $a$. Let $e$ be the identity of $G$. Then $ab=e$, $ca=e$. Thus
\[c=ce=c(ab)=(ca)b=eb=b.\]
Hence the inverse is unique.
\item To show $(a^{-1})^{-1}=a$ is exactly the problem of showing that $a$ is the inverse of $a^{-1}$, which is by definition of the inverse (with the roles of $a$ and $a^{-1}$ interchanged).
\item Let $c=(ab)^{-1}$. Then $(ab)c=e$, or $a(bc)=e$ by associativity, which gives $bc=a^{-1}$ and thus $c=b^{-1}a^{-1}$ by multiplying $b^{-1}$ on both sides.
\item The result is trivial for $n=1,2,3$. For all $k<n$ assume that any $a_1\cdots a_k$ is independent of parantheses. Then
\[(a_1\cdots a_n)=(a_1\cdots a_k)(a_{k+1}\cdots a_n).\]
Then by assumption both are independent of parentheses since $k,n-k<n$ so by induction we are done.
\end{enumerate}
\end{proof}

\begin{notation}
Since the inverse is unique, we denote the inverse of $a\in G$ by $a^{-1}$.
\end{notation}

\begin{proposition}[Cancellation law]
Let $a,b\in G$. Then the equations $ax=b$ and $ya=b$ have unique solutions for $x,y\in G$. In particular, we can cancel on the left and right.
\end{proposition}

\begin{proof}
We can solve $ax=b$ by applying $a^{-1}$ to both sides of the equation to get $x=a^{-1}b$. The uniqueness of $x$ follows because $a^{-1}$ is unique. A similar case holds for $ya=b$.
\end{proof}

\begin{definition}[Order of a group]
Let $G$ be a group. Its cardinality $|G|$ is called the \vocab{order} of $G$. We say that a group $G$ is a \emph{finite group} if $|G|<\infty$.
\end{definition}

One way to represent a finite group is by means of the group table or Cayley table\footnote{after
the English mathematician Arthur Cayley (1821 -- 1895)}. Let $G=\{e,g_2,g_3,\dots,g_n\}$ be a finite group. The Cayley table (or group table) of $G$ is a square grid which contains all the possible products of two elements from $G$. The product $g_ig_j$ appears in the $i$-th row and $j$-th column of the Cayley table.

\begin{remark}
Note that a group is abelian if and only if its Cayley table is symmetric about the main (top-left to bottom-right) diagonal.
\end{remark}

\subsection{Examples}
\begin{example}[Product group]
Let $(G,\ast_G)$ and $(H,\ast_H)$ be groups. Then the operation $\ast$ is defined on $G\times H$ by
\[(g_1,h_1)\ast(g_2,h_2)=(g_1\ast_G g_2,h_1\ast_H h_2)\]
for all $g_1,g_2\in G$, $h_1,h_2\in H$. $(G\times H, \ast)$ is called the \emph{product group} of $G$ and $H$.

We check that the product group is a group:
\begin{enumerate}[label=(\roman*)]
\item Since $\ast_G$ and $\ast_H$ are both associative binary operations, it follows that $\ast$ is also an associative binary operation on $G \times H$.
\item We also note
\[e_{G\times H}=(e_G,e_H),\quad(g,h)^{-1}=(g^{-1},h^{-1})\]
as for any $g \in G$, $h \in H$,
\[(e_G,e_H)\ast(g,h)=(g,h)=(g,h)\ast(e_G,e_H).\]
\item As for identity,
\[(g^{-1},h^{-1})\ast(g,h)=(e_G,e_H)=(g,h)\ast(g^{-1},h^{-1}).\]
\end{enumerate}
\end{example}

\begin{example}[Dihedral groups]
An important family of groups is the \vocab{dihedral groups}. For $n\in\ZZ^+$, $n\ge3$, let $D_{2n}$ be the set of symmetries\footnote{a symmetry is any rigid motion of the $n$-gon which can be effected by taking a copy of the $n$-gon, moving this copy in any fashion in $3$-space and then placing the copy back on the original $n$-gon so it exactly covers it. A symmetry can be a reflection or a rotation.} of a regular $n$-gon.

\begin{remark}
Here ``D'' stands for ``dihedral'', meaning two-sided.
\end{remark}

To visualise this, we first choose a labelling of the $n$ vertices. Then each symmetry $S$ can be described uniquely by the corresponding permutation $\sigma$ of $\{1,2,\dots,n\}$ where if the symmetry $s$ puts vertex $i$ in the place where vertex $j$ was originally, then $\sigma$ is the permutation sending $i$ to $j$.

We now make $D_{2n}$ into a group. For $S,T\in D_{2n}$, define the binary operation $ST$ to be the symmetry obtained by first applying $T$ then $S$ to the $n$-gon (this is analagous to function composition). If $S$ and $T$ effect the permutations $\sigma$ and $\tau$ respectively on the vertices, then $ST$ effects $\sigma\circ\tau$.

\begin{enumerate}[label=(\roman*)]
\item The binary operation on $D_{2n}$ is associative since the composition of functions is associative.
\item The identity of $D_{2n}$ is the identity symmetry, which leaves all vertices fixed, denoted by $1$.
\item The inverse of $S\in D_{2n}$ is the symmetry which reverses all rigid motions of $S$ (so if $S$ effects permutation $\sigma$ on the vertices, $S^{-1}$ effects $\sigma^{-1}$).
\end{enumerate}

Let $r$ be the rotation clockwise about the origin by $\frac{2\pi}{n}$ radians, let $s$ be the reflection about the line of symmetry through the first labelled vertex and the origin.

\begin{proposition*} \
\begin{enumerate}[label=(\roman*)]
\item $|r|=n$
\item $|s|=2$
\item $s\neq r^i$ for all $i$
\item $sr^i\neq sr^j$ for all $i\neq j$ ($0\le i,j\le n-1$), so
\[D_{2n}=\{1,r,\dots,r^{n-1},s,sr,\dots,sr^{n-1}\}\]
and thus $|D_{2n}|=2n$.
\item $rs=sr^{-1}$
\item $r^is=sr^{-i}$
\end{enumerate}
\end{proposition*}

\begin{proof} \
\begin{enumerate}[label=(\roman*)]
\item It is obvious that $1,r,r^2,\dots,r^{n-1}$ are all distinct and $r^n=1$, so $|r|=n$.
\item This is fairly obvious: either reflect or do not reflect.
\item This is also obvious: the effect of any reflection cannot be obtained from any form of rotation.
\item Just cancel on the left and use the fact that $|r|=n$. We assume that $i\not\equiv j\pmod n$.
\item Omitted.
\item By (5), this is true for $i=1$. Assume it holds for $k<n$. Then $r^{k+1}s=r(r^ks)=rsr^{-k}$. Then $rs=sr^{-1}$ so $rsr^{-k}=sr^{-1}r^{-k}=sr^{-k-1}$ so we are done.
\end{enumerate}
\end{proof}

A presentation for the dihedral group $D_{2n}$ using generators and relations is
\[D_{2n}=\langle r,s\mid r^n=s^2=1,rs=sr^{-1}\rangle.\]
\end{example}

\begin{example}[Permutation groups]
Let $S$ be a non-empty set. A bijection $S\to S$ is called a \emph{permutation} of $S$; the set of permutations of $S$ is denoted by $\Sym(S)$.

We now show that $\Sym(S)$ is a group under function composition $\circ$; $(\Sym(S),\circ)$ is the \vocab{symmetric group} on $S$. Note that $\circ$ is a binary operation on $\Sym(S)$ since if $\sigma:S\to S$ and $\tau:S\to S$ are both bijections, then $\sigma\circ\tau$ is also a bijection from $S$ to $S$.
\begin{enumerate}[label=(\roman*)]
\item Function composition is associative so $\circ$ is associative.
\item The identity of $\Sym(S)$ is the identity map $1$, defined by $1(a)=a$ for all $a\in S$.
\item For every permutation $\sigma$, $\sigma$ is bijective and thus invertible, so there exists a (2-sided) inverse $\sigma^{-1}:S\to S$ satisfying $\sigma\circ\sigma^{-1}=\sigma^{-1}\circ\sigma=1$.
\end{enumerate}

In the special case where $S=\{1,2,\dots,n\}$, the symmetric group on $S$ is called the \emph{symmetric group of degree $n$}, denoted by $S_n$.

\begin{proposition*}
If $|S|\ge3$ then $\Sym(S)$ is non-abelian.
\end{proposition*}

\begin{proof}
Let $S=\{x_1,x_2,x_3\}$ where three elements are distinct.
\end{proof}

\begin{proposition*}
$|S_n|=n!$
\end{proposition*}

\begin{proof}
Obvious, since there are $n!$ permutations of $\{1,2,\dots,n\}$.
\end{proof}
\end{example}

\begin{example}[Matrix groups]
For $n\in\ZZ^+$, let $GL_n(\FF)$ be the set of all $n\times n$ invertible matrices whose entries are in $\FF$:
\[GL_n(\FF)=\{A\in M_{n\times n}(\FF)\mid\det(A)\neq0\}.\]

We show that $GL_n(\FF)$ is a group under matrix multiplication; $GL_n(\FF)$ is the \vocab{general linear group} of degree $n$.
\begin{enumerate}[label=(\roman*)]
\item Since $\det(AB)=\det(A)\cdot\det(B)$, it follows that if $\det(A)\neq0$ and $\det(B)\neq0$, then $\det(AB)\neq0$, so $GL_n(\FF)$ is closed under matrix multiplication.
\item Matrix multiplication is associative.
\item $\det(A)\neq0$ if and only if $A$ has an inverse matrix, so each $A\in GL_n(\FF)$ has an inverse $A^{-1}\in GL_n(\FF)$ such that
\[AA^{-1}=A^{-1}A=I\]
where $I$ is the $n\times n$ identity matrix.
\end{enumerate}
\end{example}

\begin{example}[Quaternion group]
The \vocab{Quaternion group} $Q_8$ is defined by
\[Q_8=\{1,-1,i,-i,j,-j,k,-k\}\]
with product $\cdot$ computed as follows:
\begin{itemize}
\item $1\cdot a=a\cdot 1=a$ for all $a\in Q_8$
\item $(-1)\cdot(-1)=1$
\item $(-1)\cdot a=a\cdot(-1)=-a$ for all $a\in Q_8$
\item $i\cdot i=j\cdot j=k\cdot k=-1$
\item $i\cdot j=k$, $j\cdot i=-k$, $j\cdot k=i$, $k\cdot j=-i$, $k\cdot i=j$, $i\cdot k=-j$
\end{itemize}
Note that $Q_8$ is a non-abelian group of order $8$.
\end{example}

\subsection{Cyclic Groups and Order}
\begin{definition}[Cyclic group]
Let $G$ be a group, $g\in G$. If for every $x\in G$, there exists $n\in\ZZ$ such that $g^n=x$ then $g$ is the \vocab{generator} of $G$; denote $G=\langle g\rangle$.

$G$ is \vocab{cyclic} if there is a generator for $G$ in $G$.
\end{definition}

\begin{remark}
A cyclic group may have more than one generator. For example, if $G=\langle g\rangle$, then also $G=\langle g^{-1}\rangle$ because $(g^{-1})^n=g^{-n}\in G$ for $n\in\ZZ$ so does $-n$, thus
\[\{g^n\mid n\in\ZZ\}=\{(g^{-1})^n\mid n\in\ZZ\}.\] 
\end{remark}

\begin{example}
$\ZZ$ is a cyclic group with generators $1$ and $-1$.
\end{example}

\begin{notation}
For each $n\in\ZZ^+$, $C_n$ denotes the cyclic group of order $n$:
\[C_n=\{e,g,g^2,\dots,g^{n-1}\}\]
which satisfy $g^n=e$. Thus given two elements in $C_n$, we define
\[g^i\ast g^j=\begin{cases}
g^{i+j}&(0\le i+j<n)\\
g^{i+j-n}&(n\le i+j\le 2n-2)
\end{cases}\]
\end{notation}

\begin{proposition}
Cyclic groups are abelian.
\end{proposition}

\begin{proof}
Let $G$ be a cyclic group. For $g^i,g^j\in G$, by the laws of exponents,
\[g^i g^j=g^{i+j}=g^j g^i.\]
\end{proof}

\begin{definition}[Order]
Let $G$ be a group, $g\in G$. If there is a positive integer $k$ such that $g^k=e$, then the \vocab{order} of $g$ is defined as
\[o(g)\coloneqq\min\{m>0\mid g^m=e\}.\]
Otherwise we say that the order of $g$ is infinite.
\end{definition}

\begin{proposition}
If $G$ is finite, then $o(g)$ is finite for each $g\in G$.
\end{proposition}

\begin{proof}
Consider the list
\[g,g^2,g^3,g^4,\dots\in G.\]
As $G$ is finite, then this list must have repeats. Hence there are integers $i>j$ such that $g^i=g^j$. So $g^{i-j}=e$ showing that $\{m>0\mid g^m=e\}$ is non-empty and so has a minimal element.
\end{proof}

\begin{proposition}
If $g\in G$ and $o(g)$ is finite, then $g^n=e$ if and only if $o(g)\mid n$.
\end{proposition}

\begin{proof} \

\fbox{$\impliedby$} Suppose $o(g)\mid n$. Then $n=ko(g)$ for some $k\in\ZZ$, so
\[g^n=\brac{g^{o(g)}}^k=e^k=e.\]
\fbox{$\implies$} Suppose $g^n=e$. By the division algorithm, there exists integers $q,r$ such that $n=qo(g)+r$, where $0\le r<o(g)$. Then
\[g^r=g^{n-qo(g)}=g^n\brac{g^{o(g)}}^{-q}=e.\]
By the minimality of $o(g)$, we must have $r=0$, and so $n=qo(g)$ implies $o(g)\mid n$.
\end{proof}

\begin{corollary}
Let $G$ be a cyclic group, $g\in G$. Then $g^k=g^m$ if and only if $m\equiv k\pmod{o(g)}$.
\end{corollary}

\begin{proposition}
If $G=\langle g\rangle$, then $|G|=o(g)$ (where if one side of this equality is infinite, so is the other). More specifically,
\begin{enumerate}[label=(\roman*)]
\item if $|G|=n<\infty$, then $g^n=e$ and $e,g,g^2,\dots,g^{n-1}$ are all the distinct elements of $G$;
\item if $|G|=\infty$, then $g^n\neq e$ for all $n\neq0$, and $g^a\neq g^b$ for all $a,b\in\ZZ$, $a\neq b$.
\end{enumerate}
\end{proposition}

\begin{proof} \
\begin{enumerate}[label=(\roman*)]
\item We first show that all the elements are distinct. If $g^a=g^b$ for $0\le a<b<n$, then $g^{b-a}=e$, which contradicts the minimality of $o(g)$. Thus $G$ has at least $n$ elements and it remains to show that these are all of them.

For the element $g^t$, by the division algorithm, we can write $t=qn+r$ where $0\le r<n$. Then
\[g^t=g^{qn+r}=\brac{g^n}^q g^r=g^r\in\{e,g,g^2,\dots,g^{n-1}\}\]
since $0\le r<n$.

\item Suppose $o(g)=\infty$, so no positive power of $g$ is the identity. If $g^a=g^b$ for some $a,b\in\ZZ$, $a<b$, then $g^{b-a}=e$, contradicting the previous statement. Thus distinct powers of $g$ are distinct elements of $G$, so $|G|=\infty$.
\end{enumerate}
\end{proof}

Note that a given cyclic group may have more than one generator. The next results determine precisely which powers of $g$ generate the group $\langle g\rangle$.

\begin{proposition}
Let $G$ be a group, $g\in G$. Let $a\in\ZZ\setminus\{0\}$.
\begin{enumerate}[label=(\roman*)]
\item If $o(g)=\infty$, then $o(g^a)=\infty$.
\item If $o(g)=n<\infty$, then $o(g)=\dfrac{n}{\gcd(n,a)}$. In particular, if $a\mid n$, then $o(g^a)=\dfrac{n}{a}$.
\end{enumerate}
\end{proposition}

\begin{proof} \
\begin{enumerate}[label=(\roman*)]
\item Suppose, for a contradiction, that $o(g)=\infty$ but $o(g^a)=m<\infty$. Then by definition of order,
\[e=(g^a)^m=g^{am}.\]
Also,
\[g^{-am}=\brac{g^{am}}^{-1}=e^{-1}=e.\]
Now one of $am$ or $-am$ is positive (since $a\neq0$ and $m\neq0$), so some positive power of $g$ is the identity. This contradicts the hypothesis $o(g)=\infty$.

\item 
\end{enumerate}
\end{proof}

\begin{proposition}
Let $G=\langle g\rangle$.
\begin{enumerate}[label=(\roman*)]
\item If $o(g)=\infty$, then $G=\langle g^a\rangle$ if and only if $a=\pm1$.
\item If $o(g)=n<\infty$, then $G=\langle x^a\rangle$ if and only if $\gcd(a,n)=1$. In particular, the number of generators of $G$ is $\phi(n)$ (where $\phi$ is Euler's totient function).
\end{enumerate}
\end{proposition}

\subsection{Subgroups}
\begin{definition}[Subgroup]
Let $G$ be a group. A non-empty $H\subset G$ is a \vocab{subgroup}\index{subgroup} of $G$, denoted $H\le G$, if $H$ is closed under products and inverses; that is,
\begin{enumerate}[label=(\roman*)]
\item $e\in H$;
\item $xy\in H$ for all $x,y\in H$;
\item $x^{-1}\in H$ for all $x\in H$.
\end{enumerate}
\end{definition}

\begin{remark}
If $\ast$ is an associative (respectively, commutative) binary operation on $G$ and $\ast$ is restricted to some $H\subset G$ is a binary operation on $H$, then $\ast$ is automatically associative (respectively, commutative) on $H$ as well.
\end{remark}

The following result provides a convenient method to determine if a given subset of a group is a subgroup.

\begin{lemma}[Subgroup criterion]
Let $G$ be a group, $H\subset G$ is non-empty. Then $H\le G$ if and only if $xy^{-1}\in H$ for all $x,y\in H$. 
\end{lemma}

\begin{proof} \

\fbox{$\implies$} If $H\le G$, then we are done, by definition of subgroup.

\fbox{$\impliedby$} We want to prove that for non-empty $H\subset G$, if $xy^{-1}\in H$ for all $x,y\in H$, then $H\le G$, by checking the group axioms:
\begin{enumerate}[label=(\roman*)]
\item Since $H\neq\emptyset$, take $x\in H$, let $y=x$, then $e=xx^{-1}\in H$, so $H$ contains the identity of $G$.
\item Since $e\in H$, $x\in H$, then $x^{-1}\in H$ so $H$ is closed under taking inverses.
\item For any $x,y\in H$, $x,y^{-1}\in H$, so by (ii), $x(y^{-1})^{-1}=xy\in H$, so $H$ is closed under multiplication.
\end{enumerate}
\end{proof}

\begin{proposition}
Let $G=\langle g\rangle$ be a cyclic group. Then every subgroup of $G$ is cyclic.
\end{proposition}

\begin{proposition}
Let $G$ be a group, $H,K\le G$. Then $H\cap K\le G$.
\end{proposition}

\begin{proof}
Apply the subgroup criterion:
\begin{enumerate}[label=(\roman*)]
\item Since $e_G\in H$ and $e_G\in K$, we have $e_G\in H\cap K$, so $H\cap K\neq\emptyset$.
\item Let $a,b\in H\cap K$. Then $a,b\in H$ and $a,b\in K$. Since $H,K\le G$, by the subgroup criterion, $ab^{-1}\in H$ and $ab^{-1}\in K$, so $ab^{-1}\in H\cap K$.
\end{enumerate}
\end{proof}

\begin{corollary}
Let $G$ be a group, $\{H_i\mid i\in I\}$ is a collection of subgroups of $G$. Then
\[\bigcap_{i\in I}H_i\le G.\]
\end{corollary}

Thus we may make the following definition.
\begin{definition}[Subgroup generated by subset of group]
Let $G$ be a group, $S\subset G$. The \vocab{subgroup generated by $S$}, denoted by $\langle S\rangle$, is the smallest subgroup of $G$ which contains $S$.

If $g\in G$, then we write $\langle g\rangle$ (rather than the more accurate but cumbersome $\langle\{g\}\rangle$).

If $\langle S\rangle=G$, then the elements of $S$ are said to be \emph{generators} of $G$.
\end{definition}

\begin{example}
If $G$ is abelian and $g,h\in G$ then
\[\langle g,h\rangle=\{g^rh^s\mid r,s\in\ZZ\}.\]
\begin{proof}
Certainly $\{g^rh^s\mid r,s\in\ZZ\}\subset\langle g,h\rangle$. However, when $G$ is abelian (or indeed if just $gh=hg$), then $\{g^rh^s\mid r,s\in\ZZ\}$ is a subgroup as follows:
\begin{enumerate}[label=(\roman*)]
\item $e=g^0h^0\in\{g^rh^s\mid r,s\in\ZZ\}$
\item $(g^kh^l)(g^Kh^L)=g^{k+K}h^{l+L}\in\{g^rh^s\mid r,s\in\ZZ\}$
\item $(g^kh^l)^{-1}=h^{-l}g^{-k}=g^{-k}h^{-l}\in\{g^rh^s\mid r,s\in\ZZ\}$
\end{enumerate}
\end{proof}
\end{example}
\pagebreak

\section{Cosets and Lagrange's Theorem}
\begin{definition}[Coset]\index{coset}
Let $H\le G$. For $a\in G$, a \vocab{left coset}\index{coset!left coset} and \vocab{right coset}\index{coset!right coset} of $H$ in $G$ are
\begin{align*}
aH&\coloneqq\{ah\mid h\in H\}\\
Ha&\coloneqq\{ha\mid h\in H\}
\end{align*}
Any element of a coset is called a \emph{representative} for the coset.
\end{definition}

The set of left cosets is given by
\[(G/H)_{l}\coloneqq\{aH\mid a\in G\}.\]
Similarly, the set of right cosets is given by
\[(G/H)_{r}\coloneqq\{Ha\mid a\in G\}.\]

\begin{lemma}
Let $H\le G$. Then $aH=H$ if and only if $a\in H$. (Similarly, $Ha=H$ if and only if $a\in H$.)
\end{lemma}

\begin{proof} \

\fbox{$\implies$} Suppose $aH=H$. Then $ah\in H$ for some $h\in H$. Let $k=ah$, then $a=kh^{-1}\in H$.

\fbox{$\impliedby$} Let $a\in H$. Then $aH\subset H$.

Since $a^{-1}\in H$, $a^{-1}H\subset H$. Then $H=eH=(aa^{-1})H=a(a^{-1})H\subset aH$. Hence $aH=H$.
\end{proof}

The next result shows when two cosets are equal.
\begin{lemma}
Let $H\le G$, $a,b\in G$. Then $aH=bH$ if and only if $a^{-1}b\in H$.
\end{lemma}

\begin{proof}
\begin{align*}
aH=bH&\iff a^{-1}(aH)=a^{-1}bH\\
&\iff (a^{-1}a)H=(a^{-1}b)H\\
&\iff H=(a^{-1}b)H
\end{align*}
Note that from the previous result, $H=(a^{-1}b)H$ if and only if $a^{-1}b\in H$.
\end{proof}

\begin{proposition}
Let $H\le G$. Then $(G/H)_l$ forms a partition of $G$. (Similar remarks hold for right cosets.)
\end{proposition}

We need to prove the following.
\begin{enumerate}[label=(\roman*)]
\item For all $a\in G$, $aH\neq\emptyset$.
\item $\bigcup_{a\in G}aH=G$.
\item For every $a,b\in G$, $aH\cap bH=\emptyset$ or $aH=bH$.
\end{enumerate}

\begin{proof} \
\begin{enumerate}[label=(\roman*)]
\item Since $H\le G$, $e\in H$. Thus for all $a\in G$, $a=ae\in aH$ so $aH\neq\emptyset$.
\item For all $a\in G$, $aH\subset G$, then $\bigcup_{a\in G}aH\subset G$. Note that $a\in G$ implies $a=ae\in aH$, and so $G=\bigcup_{a\in G}g\subset\bigcup_{a\in G}aH$. By double inclusion we are done.
\item If $aH\cap bH=\emptyset$, then we are done. If $aH\cap bH\neq\emptyset$ we need to show $aH=bH$. Let $x\in G$ such that $x\in aH\cap bH$. Then $x=ah_1=bh_2$ for $h_1,h_2\in H$ so $h_1=a^{-1}bh_2$. Notice that $a^{-1}b=h_1h_2^{-1}\in H$ and thus $aH=bH$.
\end{enumerate}
\end{proof}

\begin{definition}[Index]
The number of left cosets of $H$ in $G$ is called the \vocab{index} of $H$ in $G$, denoted by $|G:H|$.
\end{definition}

The following result shows that $H$ partitions $G$ into equal-sized parts.

\begin{lemma}
The cosets of $H$ in $G$ are the same size as $H$; that is, for all $a\in G$, $|aH|=|H|$.
\end{lemma}

\begin{proof}
Let $f:H\to aH$ which sends $h\mapsto ah$. For $h_1,h_2\in H$,
\begin{align*}
f(h_1)=f(h_2)
&\implies ah_1=ah_2\\
&\implies a^{-1}ah_1=a^{-1}ah_2\\
&\implies h_1=h_2
\end{align*}
thus $f$ is an injective mapping. Note that $f$ is surjective by the definition of $aH$. Since $f$ is bijective, $|H|=|aH|$.
\end{proof}

\begin{theorem}[Lagrange's theorem]
Let $G$ be a finite group, $H\le G$. Then $|G|=|H|\:|G:H|$.
\end{theorem}

\begin{proof}
Let $|H|=n$, and let $|G:H|=k$. Since $G$ is partitioned into $k$ disjoint subsets, each of which has cardinality $n$, we have $|G|=kn$, or
\[|G|=|H|\:|G:H|\]
as desired.
\end{proof}

\begin{theorem}[Fermat's little theorem]
For every finite group $G$, for all $a \in G$, $a^{|G|}=e$.
\end{theorem}

\begin{proof}
Consider the subgroup $H$ generated by $a$; that is,
\[H=\{a^i\mid i\in\ZZ\}.\]
Since $G$ is finite and $|H|<|G|$, $H$ must be finite, so the infinite sequence $a^0=e,a^1,a^2,a^3,\dots$ must repeat, say $a^i=a^j$ ($i<j$). Let $k=j-i$. Multiplying both sides by $a^{-i}=\brac{a^{-1}}^i$, we get $a^{j-i} = a^k = e$. Suppose $k$ is the least positive integer for which this holds. Then
\[H=\{a^0,a^1,a^2,\dots,a^{k-1}\},\]
and thus $|H|=k$. By Lagrange's theorem, $k$ divides $|G|$, so
\[a^{|G|}=(a^k)^\frac{|G|}{k}=e.\]
\end{proof}

\begin{theorem}[Fermat--Euler Theorem (or Euler's totient theorem)]
If $a$ and $N$ are coprime, then $a^{\phi(N)}\equiv1\pmod N$, where $\phi$ is Euler's totient function.
\end{theorem}

\begin{proposition}
A group of prime order is cyclic.
\end{proposition}

\begin{definition}
Let $H,K\le G$, define
\[HK=\{hk\mid h\in H,k\in K\}.\]
\end{definition}

\begin{proposition}
If $H,K\le G$ are finite groups, then
\[|HK|=\frac{|H||K|}{|H\cap K|}.\]
\end{proposition}

\begin{proof}
Notice that $HK$ is a union of left cosets of $K$, namely
\[HK=\bigcup_{h\in H}hK.\]

\end{proof}
\pagebreak

\section{Normal Subgroups, Quotient Groups}
\begin{definition}[Normal subgroup]
Let $G$ be a group. $H\le G$ is a \vocab{normal subgroup} of $G$, denoted by $H\triangleleft G$, if
\[aH=Ha\quad(\forall a\in G)\]
\end{definition}

If $G$ has no non-trivial normal subgroup, then $G$ is a \emph{simple group}.

\begin{remark}
This does \emph{not} mean that $ah=ha$ for all $a\in G$, $h\in H$ or that $G$ is abelian. Although we can easily see that all subgroups of abelian groups are normal. In general, a left coset does not equal the right coset.
\end{remark}

\begin{lemma}
The following are equivalent.
\begin{enumerate}[label=(\roman*)]
\item $H\triangleleft G$.
\item $ghg^{-1}\in H$ for all $g\in G$, $h\in H$.
\item $gHg^{-1}=H$ for all $g\in G$.
\end{enumerate}
\end{lemma}

\begin{proof} \

\fbox{(i)$\iff$(ii)} In the forward direction, $aH=Ha$ for all $a\in G$. Let $g\in G$, $x\in H$. Then $gH=Hg$ so $gx=h^\prime g$ for some $h^\prime\in H$. Then $gxg^{-1}=h^\prime gg^{-1}=h^\prime\in H$.

In the reverse direction, $ghg^{-1}\in H$ for all $g\in G$, $h\in H$. Fix $g$. Then $ghg^{-1}\in H$ implies $gh\in Hg$ for all $h\in H$. So $gH\subset Hg$. Similarly $gH\supset Hg$, so $gH=Hg$.

\fbox{(i)$\iff$(iii)} $H\triangleleft G$ if and only if for all $g\in G$,
\begin{align*}
gH=Hg&\iff(gH)g^{-1}=(Hg)g^{-1}\\
&\iff gHg^{-1}=H
\end{align*}
\end{proof}

\begin{definition}[Quotient group]
Let $G$ be a group, $H\triangleleft G$. Then the \vocab{quotient group} of $G$ by $H$ is
\[G/H\coloneqq\{aH\mid a\in G\}.\]
\end{definition}

\begin{proposition}
$G/H$ is a group under the following operation. Let $aH,bH\in G/H$. Then the product of $aH$ and $bH$ is $(aH)(bH)$.
\[(aH)(bH)=a(Hb)H=a(bH)H=abH\]
\end{proposition}

\begin{proof}
Check group axioms.
\begin{enumerate}[label=(\roman*)]
\item For $a,b,c\in G$,
\begin{align*}
(aH)(bHcH)&=(aH)(bcH)\\
&=a(bc)H\\
&=(ab)cH\\
&=(aHbH)cH
\end{align*}
so the operation is associative.
\item The identity of $G/H$ is the coset $eH$.
\item For $aH\in G/H$, the inverse of $aH$ is $a^{-1}H$ as is immediate from the definition of the product.
\end{enumerate}
\end{proof}

\begin{lemma}
Let $G$ be a finite group, $H\triangleleft G$. Then
\[|G/H|=|G:H|=\frac{|G|}{|H|}.\]
\end{lemma}

\begin{definition}[Quotient map]
Let $H\triangleleft G$. The map $\pi:G\to G/H$ which sends $g\mapsto gH$ is called the \vocab{quotient map}.
\end{definition}
\pagebreak

\section{Homomorphisms and Isomorphisms}
In this section, we make precise the notion of when two groups ``look the same''; that is, they have the same group-theoretic structure. This is the notion of an \emph{isomorpism} between two groups.

\subsection{Definitions and Examples}
\begin{definition}[Homomorphism]
Let $(G,\ast)$ and $(H,\diamond)$ be groups. A map $\phi:G\to H$ is called a \vocab{homomorphism}\index{homomorphism} if, for all $x,y\in G$,
\[\phi(x\ast y)=\phi(x)\diamond\phi(y).\]
\end{definition}

When the group operations for $G$ and $H$ are not explicitly written, we have
\[\phi(xy)=\phi(x)\phi(y).\]

\begin{definition}[Isomorphism]
$\phi:G\to H$ is called an \vocab{isomorphism}\index{isomorphism} if
\begin{enumerate}[label=(\roman*)]
\item $\phi$ is a homomorphism;
\item $\phi$ is a bijection.
\end{enumerate}
Then $G$ and $H$ are said to be \vocab{isomorphic}, denoted by $G\cong H$.
\end{definition}

Intuitively, $G$ and $H$ are the same group except that the elements and the operations may be written differently in $G$ and $H$.

We also have the following terminology: An \emph{automorphism} of a group $G$ is an isomorphism from $G$ to $G$. The automorphisms of $G$ form a group $\Aut(G)$ under composition. An endomorphism of $G$ is a homomorphism from $G$ to $G$. (Rarely used) A \emph{monomorphism} is an injective homomorphism and an \emph{epimorphism} is a surjective homomorphism.

\begin{example}
For any group $G$, $G\cong G$ as the identity map provides an isomorphism from $G$ to itself. (Exercise: prove that the identity map is the \emph{only} isomorphism from $G$ to itself.)

$\ZZ\cong10\ZZ$ as the map $\phi:\ZZ\to10\ZZ$ by $x\mapsto 10x$ is a homomorphism and a bijection.
\end{example}

\begin{example}
$(\RR,+)\cong(\RR^+,\times)$.

\begin{proof}
The exponential map $\exp:\RR\to\RR^+$ defined by $\exp(x)=e^x$ is an isomorphism from $(\RR,+)$ to $(\RR^+,\times)$.
\begin{enumerate}[label=(\roman*)]
\item $\exp$ is a bijection since it has an inverse function (namely $\ln$).
\item $\exp$ preserves the group operations since $e^{x+y}=e^xe^y$.
\end{enumerate}

We see that both the elements and the operations are different yet the two groups are isomorphic, that is, as groups they have identical structures.
\end{proof}
\end{example}

\begin{proposition}
Let $\phi:G\to H$ be a homomorphism. Let $g\in G$, $n\in\ZZ$. Then
\begin{enumerate}[label=(\roman*)]
\item $\phi(e_G)=e_H$;
\item $\phi(g^{-1})=\brac{\phi(g)}^{-1}$;
\item $\phi(g^n)=\brac{\phi(g)}^n$.
\end{enumerate}
\end{proposition}

\begin{proof} \
\begin{enumerate}[label=(\roman*)]
\item $\phi(e_G)=\phi(e_G e_G)=\phi(e_G)\phi(e_G)$, then apply $\phi(e_G)^{-1}$ to both sides to get $\phi(e_G)=e_H$.

\item $\phi(g)\phi(g^{-1})=\phi(gg^{-1})=\phi(e_G)=e_H$.

\item Note more generally that we can show $\phi(g^n)=(\phi(g))^n$ for $n>0$ by induction. For $n=-k<0$ we have
\[\phi(g^n)=\phi((g^{-1})^k)=(\phi(g^{-1}))^k=(\phi(g)^{-1})^k=\phi(g)^n.\]
\end{enumerate}
\end{proof}

\begin{theorem}
Any two cyclic groups of the same order are isomorphic.
\end{theorem}

\begin{proof}
Suppose $\langle x\rangle$ and $\langle y\rangle$ are both cyclic groups of order $n$. We first prove the case where $n<\infty$. We claim that the map $\phi:\langle x\rangle\to\langle y\rangle$ which sends $x^k\mapsto y^k$ is an isomorphism.
\begin{lemma*}
Let $G$ be a group, $g\in G$, let $m,n\in\ZZ$. Denote $d=\gcd(m,n)$. If $g^n=e$ and $g^m=e$, then $g^d=e$.
\end{lemma*}
\begin{proof}
By Bezout's lemma, since $d=\gcd(m,n)$, then there exists $q,r\in\ZZ$ such that $qm+rn=d$. Thus
\[g^d=g^{qm+rn}=\brac{g^m}^q\brac{g^n}^r=e.\]
\end{proof}
We first show that $\phi$ is well-defined; that is, $x^r=x^s\implies \phi(x^r)=\phi(x^s)$. Note that $x^{r-s}=e$, so by the above lemma, $n\mid r-s$. Write $r=tn+s$ for some $t\in\ZZ$, so
\[\phi(x^r)=\phi(x^{tn+s})=y^{tn+s}=(y^n)^ty^s=y^s=\phi(x^s).\]

We then show that $\phi$ is a homomorphism:
\[\phi(x^ax^b)=\phi(x^{a+b})=y^{a+b}=y^ay^b=\phi(x^a)\phi(x^b).\]

Finally we show that $\phi$ is bijective. Since the element $y^k$ of $\langle y\rangle$ is in the image of $x^k$ under $\phi$, $\phi$ is surjective. Since both groups have the same finite order, any surjection from one to the other is a bijection. Therefore $\phi$ is an isomorphism.

We now prove the case where $n=\infty$. If $\langle x\rangle$ is an infinite cyclic group, let $\phi:\ZZ\to\langle x\rangle$ be defined by $\phi(k)=x^k$. (This map is well-defined since there is no ambiguity in the representation of elements in the domain.)

Since $x^a\neq x^b$ for all distinct $a,b\in\ZZ$, $\phi$ is injective. By definition of a cyclic group, $\phi$ is surjective. As above, the laws of exponents ensure $\phi$ is a homomorphism. Hence $\phi$ is an isomorphism.
\end{proof}

\subsection{Kernel and Image}
\begin{definition}[Kernel and image]
Let $\phi:G\to H$ be a homomorphism. Then the \vocab{kernel}\index{kernel} of $\phi$ is
\[\ker\phi\coloneqq\{g\in G\mid \phi(g)=e_H\}\subset G.\]
The \vocab{image}\index{image} of $G$ under $\phi$ is
\[\im\phi\coloneqq\phi(G)=\{\phi(g)\mid g\in G\}\subset H.\]
\end{definition}

\begin{remark}
$\im\phi$ is the usual set theoretic image of $\phi$.
\end{remark}

\begin{proposition}
Let $\phi:G\to H$ be a homomorphism. Then
\begin{enumerate}[label=(\roman*)]
\item $\ker\phi\triangleleft G$;
\item $\im\phi\le H$.
\end{enumerate}
\end{proposition}

\begin{proof} \
\begin{enumerate}[label=(\roman*)]
\item Apply the subgroup criterion. Since $e_G\in\ker\phi$, $\ker\phi\neq\emptyset$. Let $x,y\in\ker\phi$; that is, $\phi(x)=\phi(y)=e_H$. Then
\[\phi(xy^{-1})=\phi(x)\phi(y)^{-1}=e_H\]
so $xy^{-1}\in\ker\phi$. By the subgroup criterion, $\ker\phi\le G$.



\item Since $\phi(e_G)=e_H$, $e_H\in\im\phi$ so $\im\phi\neq\emptyset$. Let $x,y\in\im\phi$. Then there exists $a,b\in G$ such that $\phi(a)=x$, $\phi(b)=y$. Then
\[xy^{-1}=\phi(a)\phi(b)^{-1}=\phi(ab^{-1})\]
so $xy^{-1}\in\im\phi$. By the subgroup criterion, $\im\phi\le G$.
\end{enumerate}
\end{proof}

\begin{proposition}
Let $\phi:G\to H$ be a homomorphism. Then $\phi$ is injective if and only if $\ker\phi=\{e_G\}$.
\end{proposition}

\begin{proof} \

\fbox{$\implies$} Suppose $\phi$ is injective. Since $\phi(e_G)=e_H$, $e_G\in\ker\phi$ so $\{e_G\}\subset\ker\phi$. 

Conversely, let $g\in\ker\phi$, so $\phi(g)=e_H$. Then $\phi(g)=e_H=\phi(e_H)$, so by injectivity, $g=e_G$ and thus $g=e_G$. Hence $\ker\phi\subset\{e_G\}$, so $\ker\phi=\{e_G\}$.

\fbox{$\impliedby$} Suppose $\ker\phi=\{e_G\}$. Suppose $\phi(a)=\phi(b)$, then
\begin{align*}
\phi(a)&=\phi(b)\\
\phi(a)\phi(b)^{-1}&=\phi(b)\phi(b)^{-1}\\
\phi(a)\phi(b)^{-1}&=e_H\\
\phi(ab^{-1})&=e_H
\end{align*}
Hence $ab^{-1}\in\ker\phi=\{e_G\}$, so $ab^{-1}=e_G$ and thus $a=b$. Therefore $\phi$ is injective.
\end{proof}

\subsection{Isomorphism Theorems}
\begin{theorem}[First isomorphism theorem]
Let $\phi:G\to H$ be a homomorphism. Then
\[G/\ker\phi\cong\im\phi(G).\]
\end{theorem}

\begin{theorem}[Second isomorphism theorem]
Let $A,B\le G$, $A\le N_G(B)$. Then
\begin{enumerate}[label=(\roman*)]
\item $AB\le G$;
\item $B\triangleleft AB$;
\item $A\cap B\triangleleft A$;
\item $AB/B\cong A/A\cap B$.
\end{enumerate}
\end{theorem}

\begin{theorem}[Third isomorphism theorem]
Let $H,K\triangleleft G$, $H\le K$. Then $K/H\triangleleft G/H$, and
\[(G/H)/(K/H)\cong G/K.\]
If we denote the quotient by $H$ with a bar, this can be written
\[\overline{G}/\overline{K}\cong G/K.\]
\end{theorem}

\begin{theorem}[Fourth isomorphism theorem]

\end{theorem}

\begin{theorem}[Cayley's theorem]

\end{theorem}
\pagebreak

\section{Group Actions}
We move now, from thinking of groups in their own right, to thinking of how groups can move sets around---for example, how $S_n$ permutes $\{1,2,\dots,n\}$ and matrix groups move vectors.

\begin{definition}[Group action]
A \vocab{group action} of a group $G$ on a set $A$ is a map from $G\times A\to A$ (written as $g\cdot a$, for all $g\in G$, $a\in A$) satisfying the following properties:
\begin{enumerate}[label=(\roman*)]
\item $g_1\cdot(g_2\cdot a)=(g_1g_2)\cdot a$, for all $g_1,g_2\in G$, $a\in A$;
\item $e_G\cdot a=a$ for all $a\in A$.
\end{enumerate}
We say that $G$ is a group acting on a set $A$.
\end{definition}

Intuitively, a group action of $G$ on a set $A$ means that every element $g$ in $G$ acts as a permutation on $A$ in a manner consistent with the group operations in $G$. There is also a notion of left \emph{action} and \emph{right action}.

For the following defintions, let $G$ be a group, and $A\subset G$ be non-empty.

\begin{definition}[Centraliser]
The \vocab{centraliser} of $A$ in $G$ is defined by
\[C_G(A)\coloneqq\{g\in G\mid\forall a\in A,gag^{-1}=a\}.\]
\end{definition}

Since $gag^{-1}=a$ if and only if $ga=ag$, $C_G(A)$ is the set of elements of $G$ which commute with every element of $A$.

We check that $C_G(A)\le G$:
\begin{enumerate}[label=(\roman*)]
\item $e\in C_G(A)$, so $C_G(A)\neq\emptyset$.
\item Let $x,y\in C_G(A)$; that is, for all $a\in A$, $xax^{-1}=a$ and $yay^{-1}=a$. Then
\begin{align*}
(xy)a(xy)^{-1}&=(xy)a(y^{-1}x^{-1})\\
&=x(yay^{-1})x^{-1}\\
&=xax^{-1}=a
\end{align*}
so $xy\in C_G(A)$. Hence $C_G(A)$ is closed under products.
\item Let $x\in C_G(A)$; that is, for all $a\in A$, $xax^{-1}=a$. Applying $x^{-1}$ to both sides gives $ax^{-1}=x^{-1}a$. Applying $x$ to both sides gives $a=x^{-1}ax$, so $x^{-1}\in C_G(A)$. Hence $C_G(A)$ is closed under taking inverses.
\end{enumerate}

\begin{notation}
In the special case when $A=\{a\}$ we simply write $C_G(a)$ instead of $C_G(\{a\})$. In this case $a^n\in C_G(a)$ for all $n\in\ZZ$.
\end{notation}

\begin{definition}[Centre]
The \vocab{centre} of $G$ is the set of elements which commute with all the elements of $G$:
\[Z(G)\coloneqq\{g\in G\mid\forall x\in G,gx=xg\}.\]
\end{definition}

Note that $Z(G)=C_G(G)$, so the argument above proves $Z(G)\le G$ as a special case.

\begin{definition}[Normaliser]
Define $gAg^{-1}=\{gag^{-1}\mid a\in A\}$. The \vocab{normaliser} of $A$ in $G$ is
\[N_G(A)\coloneqq\{g\in G\mid gAg^{-1}=A\}.\]
\end{definition}

Notice that if $g\in C_G(A)$, then $gag^{-1}=a\in A$ for all $a\in A$ so $C_G(A)\le N_G(A)$. The proof that $N_G(A)\le G$ is similar to the one that $C_G(A)\le G$.

\begin{definition}[Stabiliser]
If $G$ is a group acting on a set $S$, $s\in S$, then the \vocab{stabiliser} of $s$ in $G$ is
\[G_s\coloneqq\{g\in G\mid g\cdot s=s\}.\]
\end{definition}

\begin{notation}
Denote the set of all fixed points to be $S^G=\{s\in S\mid\forall g\in G, gs=g\}$.
\end{notation}

We check that $G_s\le G$:
\begin{enumerate}[label=(\roman*)]
\item By definition of group action, $e_G\cdot a=a$, so $e_G\in G_s$.
\item Let $x,y\in G_s$, then
\begin{align*}
(xy)\cdot s&=x\cdot(y\cdot s)\\
&=x\cdot s=s
\end{align*}
so $xy\in G_s$. Hence $G_s$ is closed under products.
\item Let $x\in G_s$; that is, $x\cdot s=s$. Then
\begin{align*}
x^{-1}\cdot s&=x^{-1}\cdot(x\cdot s)\\
&=(x^{-1}x)\cdot s\\
&=e\cdot s=s
\end{align*}
so $x^{-1}\in G_s$. Hence $G_s$ is closed under taking inverses.
\end{enumerate}

\begin{definition}
The \emph{kernel} of the action of $G$ on $S$ is
\[\{g\in G\mid\forall s\in S, g\cdot s=s\}.\]
\end{definition}

\begin{definition}[Orbit]
Let $G$ be a group that acts on a set $S$. Define the \vocab{orbit} of a group element $s\in S$ as
\[G(s)\coloneqq\{g\cdot s\in S\mid g\in G\}.\]
\end{definition}

\subsection{Conjugation}

\subsection{Sylow's Theorem}
\begin{definition}[Sylow $p$-subgroup]
Let $G$ be a group, and let $p$ be a prime.
\begin{enumerate}[label=(\roman*)]
\item A group of order $p^\alpha$ ($\alpha\ge1$) is called a \emph{$p$-group}. Subgroups of $G$ which are $p$-groups are called \emph{$p$-subgroups}.
\item If $|G|=p^\alpha m$ ($p\nmid m$), then a subgroup of order $p^\alpha$ is called a \vocab{Sylow $p$-subgroup} of $G$.
\end{enumerate}
\end{definition}

\begin{notation}
The set of Sylow $p$-subgroups of $G$ is denoted by $Syl_p(G)$, and the number of Sylow $p$-subgroups of $G$ is denoted by $n_p(G)$ (or just $n_p$ when $G$ is clear from the context).
\end{notation}

\begin{theorem}[Sylow's theorem]
Let $|G|=p^\alpha m$, where $p$ is a prime and $p\nmid m$.
\begin{enumerate}[label=(\roman*)]
\item Sylow $p$-subgroups of $G$ exist, i.e. $Syl_p(G)\neq\emptyset$.
\item If $P$ is a Sylow $p$-subgroup of $G$, and $Q$ is any $p$-subgroup of $G$, then there exists $g\in G$ such that $Q\le gPg^{-1}$, i.e. $Q$ is contained in some conjugate of $P$. In particular, any two Sylow $p$-subgroups of $G$ are conjugate in $G$.
\item $n_p\equiv1\pmod p$. Furthermore, $n_p$ is the index in $G$ of the normaliser $N_G(P)$ for any Sylow $p$-subgroup $P$, hence $n_p\mid m$.
\end{enumerate}
\end{theorem}
\pagebreak

\section{Group Product, Finite Abelian Groups}
\begin{definition}[Direct product]
The \vocab{direct product} $G_1\times\cdots\times G_n$ of the groups $(G_1,\ast_1),\dots,(G_n,\ast_n)$ is the Cartesian product
\[G_1\times\cdots\times G_n\coloneqq\{(g_1,\dots,g_n)\mid g_i\in G_i\}\]
with operation defined componentwise:
\[(g_1,\dots,g_n)\ast(h_1,\dots,h_n)=(g_1\ast_1 h_1,\dots,g_n\ast_n h_n).\]
\end{definition}

\begin{proposition}
If $G_1,\dots,G_n$ are groups, then
\[|G_1\times\cdots\times G_n|=|G_1|\:|G_2|\cdots|G_n|.\]
\end{proposition}

\begin{proof}
Let $G=G_1\times\cdots\times G_n$. The proof that the group axioms hold for $G$ is straightforward since each axiom is a consequence of the fact that the same axiom holds for each $G_i$, and the operation on $G$ defined componentwise.

The number of $n$-tuples in $G$ follows from simple combinatorics.
\end{proof}


%    \chapter{Rings}\label{chap:rings}
\begin{summary}
\item Prime ideals and maximal ideals, relation to fields and integral domains. Application of quotients to constructing fields by adjunction of elements. Degree of a field extension, the tower law.
\item Euclidean domains. Principal ideal domains. EDs are PIDs. Unique factorisation for PIDs. Gauss's lemma and Eisenstein's criterion for irreducibility.
\end{summary}

\section{Rings}
\subsection{Definitions and Properties}
\begin{definition}[Ring]
A \vocab{ring}\index{ring} $R$ is a set together with two binary operations $+$ and $\times$ (called addition and multiplication), satisfying the following axioms:
\begin{enumerate}[label=(\roman*)]
\item $(R,+)$ is an abelian group, with identity $0$;
\item $\times$ is associative: $(a\times b)\times c=a\times(b\times c)$ for all $a,b,c\in R$;
\item $\times$ distributes over $+$: for all $a,b,c\in R$,
\begin{align*}
a\times(b+c)&=(a\times b)+(a\times c),\\
(a+b)\times c&=(a\times c)+(b\times c).
\end{align*}
\end{enumerate}
\end{definition}

\begin{notation}
We simply write $ab$ rather than $a\times b$, for $a,b\in R$.
\end{notation}

\begin{notation}
Denote the additive identity of $a\in R$ by $-a$.
\end{notation}

We say $R$ is \emph{commutative} if multiplication is commutative.

We say $R$ has an \emph{identity} if there exists $1\in R$ such that
\[1\times a=a\times 1=a\quad(a\in R).\]

In general, a ring may not necessarily be commutative or have multiplicative inverses; when they do, we give such rings special names.

\begin{definition}
A ring $R$ with identity $1$, where $1\neq0$, is called a \vocab{division ring}\index{division ring} if every $a\in R\setminus\{0\}$ has a multiplicative inverse, i.e., exists $b\in R$ such that $ab=ba=1$.

A commutative division ring is called a \vocab{field}\index{field}.
\end{definition}

\begin{figure}[H]
\centering
\begin{tikzpicture}
\node at (-0.5,0) {ring};
\draw[->] (0.5,0.1) -- (2.5,1);
\node at (1.1,0.9) {$ab=ba$};
\node at (4.5,1) {commutative ring};
\draw[->] (0.5,-0.1) -- (2.5,-1);
\node at (1.5,-1) {$\exists a^{-1}$};
\node at (4.5,-1) {division ring};
\draw[->] (6.5,1) -- (8.5,0.1);
\node at (7.5,1) {$\exists a^{-1}$};
\draw[->] (6.5,-1) -- (8.5,-0.1);
\node at (7.9,-0.9) {$ab=ba$};
\node at (9.5,0) {field};
\end{tikzpicture}
\end{figure}

\begin{example} \
\begin{itemize}
\item $\ZZ$ is the prototypical ring; it is not a field.
\item $\QQ$, $\RR$, $\CC$ are fields.
\item $\ZZ/n\ZZ$ is a commutative ring with identity $\bar{1}$ under addition and multiplication of residue classes.
\item $2\ZZ$ is a commutative ring without identity.
\item The trivial ring $R=\{0\}$ is a commutative ring with identity $1=0$.
\item The ring of (polynomial/continuous/differentiable) functions on $\RR$.
\item The endomorphism ring $\End_\RR(V)$ of a vector space $V$ over $\RR$ is a non-commutative ring.
\item The Hamilton Quaternions $\HH$. Historically the first example of a non-commutative ring.
\end{itemize}
\end{example}

\begin{lemma}[Basic properties]
Let $R$ be a ring.
\begin{enumerate}[label=(\roman*)]
\item $0a=a0=0$ for all $a\in R$.
\item $(-a)b=a(-b)=-(ab)$ for all $a,b\in R$.
\item $(-a)(-b)=ab$ for all $a,b\in R$.
\item If $R$ has identity $1$, then the identity is unique and $-a=(-1)a$.
\end{enumerate}
\end{lemma}

\begin{proof}
These all follow from the distributive laws and cancellation in the additive group $(R,+)$.
\begin{enumerate}[label=(\roman*)]
\item We have $0a=(0+0)a=0a+0a$. Then the cancellation law implies $0a=0$. 

Similarly, $a0=a(0+0)=a0+a0$. Thus $a0=0$.

\item We have $ab+(-a)b=\brac{a+(-a)}b=0b=0$. Thus $(-a)b=-(ab)$. 

Similarly, $ab+a(-b)=a\brac{b+(-b)}=a0=0$. Thus $a(-b)=-(ab)$.

\item Using (ii), $(-a)(-b)=-a(-b)=-\brac{-(ab)}=ab$.

\item Suppose $1$ and $1^\prime$ are identities of $R$. Then $1=1\times1^\prime=1^\prime$.

Since $(-1)a+a=(-1)a+1a=\brac{(-1)+1}a=0a=0$, it follows that $-a=(-1)a$.
\end{enumerate}
\end{proof}

\subsection{Subrings}
Having defined the notion of a ring, there is a natural notion of a subring.

\begin{definition}[Subring]
Let $R$ be a ring. We say $S\subset R$ is a \vocab{subring}\index{subring} of $R$, if $S$ is a subgroup of $R$ that is closed under multiplication.
\end{definition}

\begin{example} \
\begin{itemize}
\item $\ZZ$ is a subring of $\QQ$, and $\QQ$ is a subring of $\RR$.
\item $n\ZZ=\{nk\in\ZZ\mid k\in\ZZ\}$ is a subring of $\ZZ$.
\item The real-valued differentiable functions on $\RR$ form a subring of the ring of continuous functions.
\item $\ZZ[i]=\{x+yi\mid x,y\in\ZZ\}$ is a subring of $\CC$, called the ring of Guassian integers.
\item $\QQ[\sqrt{2}]=\{x+y\sqrt{2}\mid x,y\in\QQ\}$ is a subring of $\RR$.
\item $S=\ZZ+\ZZ i+\ZZ j+\ZZ k$ form a subring of $\HH$.
\end{itemize}
We will use the square brackets notation quite frequently. It should be clear what it should mean, and we will define it properly later.
\end{example}

If $R$ contains $1$, then $S$ is a (unital) subring if $1_R\in S$. We assume subrings are unital unless otherwise specified.

\begin{lemma}[Subring criterion]
Let $R$ be a ring, $S\subset R$. Then $S$ is a subring of $R$ if and only if
\begin{enumerate}[label=(\roman*)]
\item $1\in S$;\hfill(non-empty)
\item $ab,a-b\in S$ for all $a,b\in S$.\hfill(closed under multiplication and subtraction)
\end{enumerate}
\end{lemma}

\begin{proof} \

\fbox{$\implies$} 

\fbox{$\impliedby$} The condition that $a-b\in S$ for all $a,b\in S$ implies that $S$ is an additive subgroup by the subgroup test (note that as $1\in S$ we know that $S$ is nonempty). The other conditions for a subring hold directly.
\end{proof}

\subsection{Units, Zero Divisors}
Recall that in a ring we do not require that non-zero elements have a multiplicative inverse. 
Nevertheless, since the multiplication operation is associative and there is a multiplicative identity, the elements which happen to have multiplicative inverses form a group:

\begin{definition}
Let $R$ be a ring, with identity $1\neq0$. We say $u\in R$ is a \vocab{unit}\index{unit} in $R$ if there exists $v\in R$ such that $uv=vu=1$.

We say $a\in R\setminus\{0\}$ is a \vocab{zero divisor}\index{zero divisor} if there exists $b\in R\setminus\{0\}$ such that either $ab=0$ or $ba=0$.
\end{definition}

\begin{remark}
A zero divisor can not be a unit.
\end{remark}

Let $R^\times$ denote the set of units in $R$.

\begin{lemma*}
$R^\times$ forms a group under multiplication.
\end{lemma*}

We call $R^\times$ the \vocab{group of units}\index{group of units}.

\begin{proof} \
\begin{enumerate}[label=(\roman*)]
\item Evidently $1\in R^\times$.
\item Let $u_1,u_2\in R^\times$. Then $u_1v_1=1$, $u_2v_2=1$ for some $v_1,v_2\in R$. Thus $(u_1u_2)(v_2v_1)=1$. Similarly $(v_2v_1)(u_1u_2)=1$. Hence $u_1u_2\in R^\times$.
\item Let $u\in R^\times$. Then $uv=vu=1$ for some $v\in R$. Taking inverse gives $u^{-1}v^{-1}=v^{-1}u^{-1}=1$. Hence $u^{-1}\in R^\times$.
\end{enumerate}
\end{proof}

\begin{example} \
\begin{itemize}
\item The ring $\ZZ$ has no zero divisors and its only units are $\pm1$.
\item The group of units of $\ZZ/n\ZZ$ is $(\ZZ/n\ZZ)^\times$. Recall that $(\ZZ/n\ZZ)^\times=\{a\in\ZZ/n\ZZ\mid(a,n)=1\}$. All elements not in $(\ZZ/n\ZZ)^\times$ are zero divisors. In sum, every non-zero element of $\ZZ/n\ZZ$ is either a unit or a zero divisor.
\end{itemize}
\end{example}

Rings having some of the same characteristics as $\ZZ$ are given a name:

\begin{definition}[Integral domain]
If a commutative ring with identity $1\neq0$ has no zero divisors, it is called an \vocab{integral domain}\index{integral domain}.
\end{definition}

\begin{example} \
\begin{itemize}
\item $\ZZ$ is an integral domain.
\item All fields are integral domains.
\end{itemize}
\end{example}

The absence of zero divisors in integral domains give these rings a cancellation property:

\begin{lemma} \
\begin{enumerate}[label=(\roman*)]
\item Let $a,b,c\in R$, $a$ is not a zero divisor. If $ab=ac$, then either $a=0$ or $b=c$.
\item In particular, for any $a,b,c$ in an integral domain and $ab=ac$, then either $a=0$ or $b=c$.
\end{enumerate}
\end{lemma}

\begin{proof} \
\begin{enumerate}[label=(\roman*)]
\item If $ab=ac$, then $a(b-c)=0$. Since $a$ is not a zero divisor, we have either $a=0$ or $b-c=0$.
\item This follows from (i) and the definition of an integral domain.
\end{enumerate}
\end{proof}

\begin{corollary}
Any finite integral domain is a field.
\end{corollary}

In this terminology, a field is a commutative ring $F$ with identity $1\neq0$ in which every non-zero element is a unit, i.e., $F^\times=F\setminus\{0\}$.

\begin{proof}
Let $R$ be a finite integral domain, let $a\in R\setminus\{0\}$.

By the cancellation law, the map $x\mapsto ax$ is an injective function. Since $R$ is finite, this map is also surjective. In particular, there exists $b\in R$ such that $ab=1$, i.e., $a$ is a unit in $R$. Since $a$ was an arbitrary non-zero element, $R$ is a field.
\end{proof}

\begin{corollary}
If $p$ is a prime, $\ZZ/p\ZZ$ is a field, usually denoted by $\FF_p$.
\end{corollary}

\subsection{Examples}
\begin{example}[Matrix rings]
Let $R$ be a (often commutative) ring with $1$. We define the matrix ring $M_{n\times n}(R)$ as the set consisting of
\[(a_{ij})_{n\times n},\quad a_{ij}\in R.\]
Addition and multiplication on $M_{n\times n}(R)$ is defined following the matrix multiplication in linear algebra.

If we take $R=\RR$, then $M_{n\times n}(\RR)$ the usual matrix algebra. We have the subring of diagonal matrices, and the subring of upper triangular matrices.
\end{example}

\begin{example}[Group rings]
Let $R$ be a commutative ring with $1$. Let $G$ be a finite group. We define the group ring $R[G]$ as the set consisting of
\[\sum_{g\in G}a_g g\quad(a_g\in R).\]
Addition on $R[G]$ is defined in the obvious/naive way. The multiplication is via the following example
\[(a_g g+a_h h)(a_{g^\prime}g^\prime+a_{h^\prime}h^\prime)=a_g a_{g^\prime}gg^\prime+a_h a_{g^\prime}hg^\prime+a_g a_{h^\prime}gh^\prime+a_h a_{h^\prime}hh^\prime,\]

where $gg^\prime, hg^\prime, gh^\prime, hh^\prime$ are the group multiplication in $G$.

\begin{lemma*}
Let $R$ be a commutative ring with $1$. Let $G$ be a finite group.
\begin{enumerate}[label=(\roman*)]
\item Let $e\in G$ be the identity element. Then $1_e$ is the identity of the ring $R[G]$.
\item Let $e\neq g\in G$. Then $1-g$ is a zero divisor.
\item Let $H$ be a subgroup of $G$. Then $R[H]$ is a subring of $R[G]$.
\item The ring $R[G]$ is commutative if and only if $G$ is commutative.
\end{enumerate}
\end{lemma*}
\end{example}

\begin{example}[Product of rings]
Let $R$ and $S$ be two rings. We define the ring $R\times S$ as follows: as a set $R\times S$ is the same as the Cartesian product of sets; we define the addition and multiplication component wise:
\begin{align*}
(a,b)+(c,d)&=(a+c,b+d),\\
(a,b)\times(c,d)&=(ac,bd).
\end{align*}
\end{example}
\pagebreak

\section{Homomorphisms and Isomorphisms}
\subsection{Ideals}
\begin{definition}[Ideal]
Let $R$ be a ring. We say $I\subset R$ is a \vocab{left ideal}\index{ideal!left ideal} if
\begin{enumerate}[label=(\roman*)]
\item $(I,+)$ is a subgroup of $(R,+)$;\hfill(additive subgroup)
\item $ax\in I$ for all $a\in R$, $x\in I$.\hfill(closed under left multiplication)
\end{enumerate}

We define a \vocab{right ideal}\index{ideal!right ideal} similarly.

We say $I$ is a (two-sided) \vocab{ideal}\index{ideal} of $R$, if $I$ is both a left ideal and a right ideal of $R$.
\end{definition}

We say $I$ is a \emph{proper ideal} if $I\neq R$.

\begin{remark}
For commutative rings, left ideals, right ideals, and (two-sided) ideals coincide
\end{remark}

\begin{example} \
\begin{itemize}
\item Trivial ideals: the zero ideal $\{0\}$ and the whole ring $R$ are two-sided ideals.

$R$ is also called the \emph{unit ideal}: if $x\in R^\times\cap I$, then $x^{-1}x=1\in I$, so $a\times 1=a\in I$ for all $a\in R$. Thus $I=R$. (We have shown $I=R$ if and only if $1\in I$.) 

This implies that in a field $F$, the only ideals are $\{0\}$ and $F$, since if $I\neq\{0\}$, let $x\in F\setminus\{0\}$, then $x$ is a unit, so $I=F$.

\item The even numbers $2\ZZ=(2)$ is an ideal of $\ZZ$.
\end{itemize}
\end{example}

The next definition provides a way to generate an ideal from an element of a ring.

\begin{definition}[Principal ideal]
Let $R$ be a ring, and let $a\in R$. The \vocab{principal left ideal} generated by $a$ is
\[(a)\colonequals\{xa\mid x\in R\}.\] 
\end{definition}

More generally, let $a_1,\dots,a_n\in R$. Define 
\[(a_1,\dots,a_n)\colonequals\{x_1a_1+\cdots+x_na_n\mid x_i\in R\}.\]
We call $a_1,\dots,a_n$ \emph{generators} for this ideal.

\begin{lemma*}
$(a_1,\dots,a_n)$ is a left ideal.
\end{lemma*}

\begin{proof}
If $y_1,\dots,y_n,x_1,\dots,x_n\in R$ then
\begin{align*}
(x_1a_1+\cdots+x_na_n)+(y_1a_1+\cdots+y_na_n)
&=x_1a_1+y_1a_1+\cdots+x_na_n+y_na_n\\
&=(x_1+y_1)a_1+\cdots+(x_n+y_n)a_n.
\end{align*}
If $z\in R$, then
\[z(x_1a_1+\cdots+x_na_n)=zx_1a_1+zx_na_n.\]
Finally,
\[0=0a_1+\cdots+0a_n.\]
\end{proof}

\begin{example}
Let $R$ be a ring. Let $L, M$ be left ideals. Define the product
\[LM=\{x_1y_1+\cdots+x_ny_n\mid x_i\in L,y_i\in M\}.\]
Then $LM$ is also a left ideal.

If $L, M, N$ are left ideals, then $(LM)N=L(MN)$.
\end{example}

\begin{example}
Let $L, M$ be left ideals. Define the sum
\[L+M=\{x+y\mid x\in L,y\in M\}.\]
Then $L+M$ is a left ideal. 

If $L,M,N$ are left ideals, then $L(M+N)=LM+LN$. 
\end{example}

\begin{example}
The ideals of $\ZZ$ are $(n)$ for $n\in\NN$, and $\{0\}$.
\begin{proof}
Let $I\neq\{0\}$ be an ideal of $\ZZ$. Let $a\in I$ be non-zero. Since $a,-a\in I$, $I$ contains a natural number. By well-ordering, there is a minimal $n\in\NN\cap I$.

Clearly $(n)\subset I$, since all multiples of $n$ are contained in $I$. If $x\in I$ and $x\neq(n)$, by the division algorithm, we can write $x=qn+r$ for some $q,r\in\ZZ$, $0<r<n$. But $qn\in I$, so $x-qn=r\in I$. Then $r$ is a smaller natural number in $I$, which contradicts the minimality of $n$. Thus $I=(n)$.
\end{proof}
This shows that $\ZZ$ is a principal ideal domain (all ideals of $\ZZ$ are principal ideals).
\end{example}

\subsection{Homomorphisms}
\begin{definition}
We say $\phi\colon R\to S$ is a \vocab{homomorphism}\index{homomorphism} if it satisfies
\begin{enumerate}[label=(\roman*)]
\item $\phi(a+b)=\phi(a)+\phi(b)$ for all $a,b\in R$;
\item $\phi(ab)=\phi(a)\phi(b)$ for all $a,b\in R$;
\item $\phi(1_R)=1_S$.
\end{enumerate}
An \vocab{isomorphism}\index{isomorphism} is a bijective homomorphism. Two rings $R$ and $S$ are \vocab{isomorphic}, denoted by $R\cong S$, if there exists an isomorphism between $R$ and $S$.
\end{definition}

\begin{remark}
For groups, condition (iii) is not required in the definition: $\phi(1)\phi(x)=\phi(1x)=\phi(x)$ then we can cancel on both sides due to the existence of (multiplicative) inverse.
\end{remark}

An isomorphism between a ring with itself is called an \emph{automorphism}. 

An injective homomorphism $\phi\colon R\to S$ is called an \vocab{embedding}\footnote{If $\phi$ is injective, then $R\cong\im\phi$, where $\im\phi$ is a subring of $S$, so we can think of $R$ as a ``subring'' of $S$; hence the term \emph{embedding} to mean that $R$ is ``contained in'' $S$.}; we say $R$ is \emph{embedded} in $S$.

\begin{definition}
Let $\phi\colon R\to S$ be a homomorphism. The \vocab{kernel}\index{kernel} of $\phi$ is its kernel viewed as a homomorphism of additive groups:
\[\ker\phi\colonequals\{r\in R\mid\phi(r)=0\}.\]
The \vocab{image} of $\phi$ is
\[\im\phi\colonequals\{s\in S\mid \exists r\in R, \phi(r)=s\}.\]
\end{definition}

\begin{example} \
\begin{itemize}
\item Consider the quotient map $\pi\colon\ZZ\to\ZZ/n\ZZ$; $\ker\pi=n\ZZ$.
\item The embedding of the subring $n\ZZ\to\ZZ$. The kernel is trivial.
\item The map 
\begin{align*}
\phi\colon\CC[x]&\to\CC\\
f(x)&\mapsto f(a) 
\end{align*}
The kernel is
\[\ker\phi=\{f(x)\in\CC[x]\mid f(a)=0\}=\{(x-a)f(x)\mid f(x)\in\CC[x]\}.\]
\end{itemize}
\end{example}

\begin{lemma}
Let $\phi\colon R\to S$ be a homomorphism. Then
\begin{enumerate}[label=(\roman*)]
\item $\ker\phi$ is a ideal of $R$;
\item $\im\phi$ is a subsring of $S$.
\end{enumerate}
\end{lemma}

\begin{proof} \
\begin{enumerate}[label=(\roman*)]
\item Let $x,y\in\ker\phi$. Then
\[\phi(x-y)=\phi(x)-\phi(y)=0-0=0\]
so $x-y\in\ker\phi$. Thus $\ker\phi$ is an additive subgroup of $R$.

Let $r,r^\prime\in R$, $x\in\ker\phi$. Then
\[\phi(rxr^\prime)=\phi(r)\phi(x)\phi(r^\prime)=\phi(r)0\phi(r^\prime)=0.\]
Thus $rxr^\prime\in\ker\phi$, and so $\ker\phi$ is an ideal of $R$.

\item 
\end{enumerate}
\end{proof}

\begin{lemma}
Let $\phi\colon R\to S$ be a homomorphism. Then $\phi$ is injective if and only if $\ker\phi=\{0\}$.
\end{lemma}

\begin{proof}
This follows from considering $(R,+)$ as an additive group. Then the result follows from group theory.
\end{proof}

\subsection{Quotient Rings}
Let $I\subset R$ be an ideal, and let $a\in R$. Define
\[a+I=\{a+x\mid x\in I\}.\]
This is usually not an ideal, but rather an \emph{additive coset} of $I$ (considering $I$ as an additive subgroup of $R$).
Any element of the coset is called a \emph{representative} of the coset. 

\begin{definition}[Quotient ring]
Let $I\subset R$ be an ideal. Then the \vocab{quotient ring}\index{quotient ring} is
\[R/I\colonequals\{a+I\mid a\in R\}.\]
\end{definition}

\begin{lemma*}
$R/I$ is a ring, with addition and multiplication defined as
\begin{align*}
(a+I)+(b+I)&=(a+b)+I,\\
(a+I)\cdot(b+I)&=ab+I.
\end{align*}
\end{lemma*}

\begin{proof}
Recall that by \ref{lemma:subgroup-of-abelian-group-is-normal}, a subgroup of an abelian group is normal. Since $I$ is an additive subgroup of $R$, and $(R,+)$ is abelian, we have $I\triangleleft R$ under addition. Hence the quotient group $(R/I,+)$ is defined.

We now check that multiplication is well-defined. Suppose $a+I=a^\prime+I$, $b+I=b^\prime+I$. Then $a-a^\prime=r\in I$, $b-b^\prime=s\in I$. Thus 
\[ab=(a^\prime+r)(b^\prime+s)=a^\prime b^\prime+a^\prime s+b^\prime r+rs.\]
Note that $a^\prime s,b^\prime r,rs\in I$. Hence $ab+I=a^\prime b^\prime+I$.

Check that $R/I$ is a ring, with additive identity $0_R+I$ and multiplicative identity $1_R+I$.
\end{proof}

\begin{example}
Take $R=\ZZ$, $I=(n)$ for some $n\in\NN$. We can write $(n)=n\ZZ$, so the quotient ring is $\ZZ/n\ZZ$.
\end{example}

As before, we give a name to the canonical homomorphism from $R$ to $R/I$.

\begin{definition}[Quotient map]
Let $I\subset R$ be an ideal. The \vocab{quotient map}\index{quotient map} is
\begin{align*}
\pi\colon R&\to R/I\\
a&\mapsto a+I
\end{align*}
\end{definition}

\begin{lemma}
Quotient maps are surjective homomorphisms.
\end{lemma}

\begin{proof}
Let $\pi\colon R\to R/I$ be a quotient map.
\begin{itemize}
\item Let $a,b\in R$. Then $\pi(a+b)=(a+b)+I=(a+I)+(b+I)=\pi(a)+\pi(b)$.
\item Let $a,b\in R$. Then $\pi(ab)=ab+I=(a+I)(b+I)=\pi(a)\pi(b)$.
\item $\pi(1_R)=1_R+I$, which is the identity of $R/I$.
\end{itemize}
\end{proof}

In addition,
\[\ker\pi=\{a\in R\mid a+I=0_R+I\}=\{a\in R\mid a\in I\}=I.\]

\subsection{Isomorphism Theorems}
\begin{theorem}[First isomorphism theorem]
Let $\phi\colon R\to S$ be a homomorphism. Then
\begin{equation}
R/\ker\phi\cong\im\phi.
\end{equation}
\end{theorem}

\begin{proof}
Denote $K=\ker\phi$. Consider the map
\begin{align*}
\theta\colon R/K&\to\im\phi\\
a+K&\mapsto\phi(a)
\end{align*}
We claim that $\theta$ is an isomorphism.
\begin{enumerate}
\item We first check that $\theta$ is well-defined. If $a+K=a^\prime+K$, then $a-a^\prime\in K$, so $\phi(a-a^\prime)=0$. Thus $\phi(a)=\phi(a^\prime)$.
\item $\theta$ is a homomorphism: 
\begin{align*}
\theta((a+K)+(b+K))&=\theta((a+b)+K)=\phi(a+b)=\phi(a)+\phi(b)=\theta(a+K)+\theta(b+K)\\
\theta((a+K)(b+K))&=\theta(ab+K)=\phi(ab)=\phi(a)\phi(b)=\theta(a+K)\theta(b+K)\\
\theta(0_R+I)&=\phi(0_R)=0_S
\end{align*}
\item $\theta$ is injective: $\theta(a+K)=\theta(b+K)\implies\phi(a)=\phi(b)\implies a+K=b+K$.
\item $\theta$ is surjective: Let $x\in\im\phi$. Then $x=\phi(a)$ for some $a\in R$. Thus $\theta(a+K)=\phi(a)=x$.
\end{enumerate}
\end{proof}

\begin{theorem}[Second isomorphism theorem]
Let $A$ be a subring, and $B$ be an ideal of $R$. Then
\begin{equation}
(A+B)/B\cong A/(A\cap B).
\end{equation}
\end{theorem}

\begin{lemma*} \
\begin{enumerate}[label=(\roman*)]
\item $A+B=\{a+b\mid a\in A,b\in B\}$ is a subring of $R$;
\item $A\cap B$ is an ideal of $A$.
\end{enumerate}
\end{lemma*}

\begin{theorem}[Third isomorphism theorem]
Let $I$ and $J$ be ideals of $R$, with $I\subset J$. Then
\begin{equation}
(R/I)(J/I)\cong R/J.
\end{equation}
\end{theorem}

\begin{lemma*}
$J/I$ is an ideal of $R/I$.
\end{lemma*}

\begin{theorem}[Fourth isomorphism theorem]
Let $I$ be an ideal of $R$. The correspondance $A\leftrightarrow A/I$ is an inclusion preserving bijection between the set of subrings of $A$ of $R$ that contain $I$ and the set of subrings of $R/I$. Furthermore, $A$ (a subring containing $I$) is an ideal of $R$ if and only if $A/I$ is an ideal of $R/I$.
\end{theorem}

\subsection{Chinese Remainder Theorem}
\begin{definition}
Let $R$ be a commutative ring. We say two ideals $I,J\subset R$ are \emph{coprime} if
\[I+J=R.\]
\end{definition}

In particular, there $i\in I$, $j\in J$ such that $i+j=1$.

\begin{theorem}[Chinese remainder theorem]
Let $R$ be a commutative ring.
Suppose $I$ and $J$ are coprime ideals of $R$. 
Then for any $a,b\in R$, there exists $x\in R$ such that
\[x\in(a+I)\cap(b+J).\]
\end{theorem}

\begin{proof}
Let $i\in I$ and $j\in J$ be such that $i+j=1$. 
\begin{claim}
$x=aj+bi$.
\end{claim}
We can write
\[x=a(1-i)+bi=a+(b-a)i\in a+I.\]
Similarly,
\[x=aj+b(1-j)=b+(a-b)j\in b+J.\]
\end{proof}

modular arithmetic

\subsection{Prime and Maximal Ideals}
Let $R$ be a commutative ring.

\begin{definition}
An ideal $P\subsetneq R$ is \vocab{prime} if $ab\in P$ implies either $a\in P$ or $b\in P$.

An ideal $M\subsetneq R$ is \vocab{maximal} if there is no ideal between $M$ and $R$, i.e., $M\subset I\subset R$ implies $I=M$ or $I=R$.
\end{definition}

\begin{example}
In $\ZZ$, $(p)$ is a prime ideal for prime $p$.

Further $p\ZZ\subset U=n\ZZ\subset\ZZ$, and $p\in U$, then $p=nq$ for some $q\in\ZZ$. But $p$ is prime and $n\neq1$ so $n=p$. Thus $U=p\ZZ$. Thus $p\ZZ$, for $p$ prime, is a maximal ideal in $\ZZ$. Note that $0\subset p\ZZ\subset\ZZ$, so $0$ is not a maximal ideal in $\ZZ$.
\end{example}

\begin{lemma}
Let $R$ be a commutative ring.
\begin{enumerate}[label=(\roman*)]
\item A maximal ideal is prime.
\item An ideal $P$ is prime if and only if $R/P$ is integral.
\item An ideal $M$ is maximal if and only if $R/M$ is a field.
\end{enumerate}
\end{lemma}

\begin{proof} \
\begin{enumerate}[label=(\roman*)]
\item Suppose $M$ is a maximal ideal. Let $ab\in M$, WLOG assume $a\notin M$. Then $M\subsetneq(a)+M=R$, since $M$ is a maximal ideal. 

Thus $xa+m=1$ for some $x\in R$, $m\in M$. Then $b=xab+mb\in M$, since $ab,m\in M$. Hence $M$ is prime.
\end{enumerate}
\end{proof}

\subsection{Characteristic of Ring}
In the following, let $R=\{0\}$ be a ring; let $e$ denote the identity of $R$ (to distinguish it from the identity of $\ZZ$). 
For any $a\in R$, $n\in\ZZ$, we can define an integer multiple of a ring element:
\[na=\begin{cases}
\underbrace{a+\cdots+a}_\text{$n$ times}&(n>0)\\
-(ka)&(n<0,n=-k)\\
0&(n=0)
\end{cases}\]

Consider the map
\begin{align*}
f\colon\ZZ&\to R\\
n&\mapsto ne
\end{align*}
Then this is a homomorphism (this is a bit tedious, since we have to consider $n>0$, $n<0$ or $n=0$).
Now let $f\colon\ZZ\to R$ be any homomorphism. By definition, $f(1)=e$. Then if $n>0$, $f(n)=f(1+\cdots+1)=f(1)+\cdots+f(1)=nf(1)=ne$. Hence there is one and only one homomorphism $\ZZ\to R$.

Assume $R\neq\{0\}$. Let $f\colon\ZZ\to R$ be \emph{the} homomorphism. Since $\ker f$ is an ideal of $\ZZ$, $\ker f=n\ZZ$ for some integer $n\ge 0$. (Note that $n\neq 1$, otherwise $\ker f=\ZZ$ so $\im f=\{0\}$, but $f(1)=e\neq 0$.) 

By the first isomorphism theorem, $\ZZ/n\ZZ\cong\im f$. In practice, we do not make any distinction between $\ZZ/n\ZZ$ and its image in $R$, and we agree to say that ``$R$ contains $\ZZ/n\ZZ$ as a subring''. 

Suppose $n\neq0$. Then for all $a\in R$,
\[\underbrace{a+\cdots+a}_\text{$n$ times}=na=(ne)a=f(n)a=0a=a.\]
We call $n$ the \vocab{characteristic} of $R$, or say $R$ has characteristic $n$, and denote $n=\mathrm{char}(R)$.

\begin{remark}
If $n=0$, then $\ZZ/0\ZZ=\ZZ$, so rings of characteristic $0$ are infinite (since it contains a subring isomorphic to $\ZZ$, which is infinite).
\end{remark}

Note that $n\neq0$ is the smallest positive integer $m$ such that $me=0$. This is because $m\in\ker f$, so $n\mid m$, which implies $n\le m$.

\begin{lemma}
Suppose $R$ is an integral ring. Then $\mathrm{char}(R)$ is either $0$ or prime.
\end{lemma}

\begin{proof}
Suppose $n=\mathrm{char}(R)\neq0$. Suppose, for a contradiction, that $n$ is composite. Then $n=mk$, where $m,k>1$. Then $m,k<n$. 

By minimality of $n$, we have $me,ke\neq 0$. But $(me)(ke)=mke=ne=0$. This implies that $R$ has zero divisors, which contradicts the assumption that $R$ is an integral ring.
\end{proof}

\begin{lemma}[Freshman's dream]
Let $R$ be commutative with prime characteristic $p$. Then $(x+y)^p=x^p+y^p$ for all $x,y\in R$.
\end{lemma}

\begin{proof}
Since $R$ is commutative, we have the binomial expansion:
\[(x+y)^p=\sum_{i=1}^{p}\binom{p}{i}x^i y^{p-i}.\]
(We require $R$ to be commutative, so that we can freely move variables around in order to raise them by powers.) 
For $i\in\{1,\dots,p-1\}$, $\binom{p}{i}$ is divisible by $p$. Since $\mathrm{char}(R)=p$, multiples of $p$ equal $0$. Hence $(x+y)^p=x^p+0+\cdots+0+y^p=x^p+y^p$.
\end{proof}

Let $K$ be a field, and let $f\colon\ZZ\to K$ be the homomorphism from the integers to $K$. If $\ker f=\{0\}$, then $K$ contains $\ZZ$ as a subring, and we say that $K$ has \emph{characteristic} $0$. If $\ker f=p\ZZ$ for some prime $p$, then we say $K$ has \emph{characteristic} $p$. 

The field $\ZZ/p\ZZ$ is sometimes denoted by $\FF_p$, and is called the \emph{prime field}, of characteristic $p$. This prime field $\FF_p$ is contained in every field of characteristic $p$. 

\subsection{Quotient Fields}
Recall that we can construct $\QQ$ from $\ZZ$, using equivalence classes of ordered pairs whose elements are in $\ZZ$. 
Instead of $\ZZ$, our discussion will apply to an arbitrary integral ring $R$. 

Let $(a,b),(c,d)\in R\times R^*$, where $R^*=R\setminus\{0\}$; we call these ordered pairs \emph{quotients}. Define a relation $R\times R^*$:
\[(a,b)\sim(c,d)\iff ad=bc.\]

\begin{lemma*}
$\sim$ is an equivalence relation on $R\times R^*$.
\end{lemma*}

\begin{proof} \
\begin{enumerate}[label=(\roman*)]
\item Since $ab=ba$, we have $(a,b)\sim(a,b)$.
\item Suppose $(a,b)\sim(c,d)$. Then $ad=bc$, or $cb=da$. This implies $(c,d)\sim(a,b)$.
\item Suppose $(a,b)\sim(c,d)$ and $(c,d)\sim(e,f)$. Then
\[ad=bc,\quad cf=de.\]
Thus 
\[adf=bcf=bde,\]
so $daf-dbe=0$. Then $d(af-be)=0$. Since $R$ has no divisors of $0$, and $d\neq0$, it follows that $af-de=0$, i.e., $af=be$. This means that $(a,b)\sim(e,f)$.
\end{enumerate}
\end{proof}

We denote the equivalence class of $(a,b)$ by $a/b$; that is,
\[\frac{a}{b}=\{(c,d)\in R\times R^*\mid(a,b)\sim(c,d)\}.\]
Then the \vocab{quotient field} (or \emph{field of fractions}) of $R$ is the set of equivalence classes:
\[\mathrm{Frac}(R)\colonequals(R\times R^*)/\sim\]
with addition and multiplication defined by
\begin{align*}
\frac{a}{b}+\frac{c}{d}&=\frac{ad+bc}{bd},\\
\frac{a}{b}\frac{c}{d}&=\frac{ac}{bd}.
\end{align*}

\begin{lemma}
$\mathrm{Frac}(R)$ is a field, with addition and multiplication being defined as above.
\end{lemma}

\begin{proof}
We first check that addition and multiplication, as defined above, are well-defined.
\begin{description}
\item[Addition] Suppose $a/b=a^\prime/b^\prime$ and $c/d=c^\prime/d^\prime$. We must show that
\[\frac{ad+bc}{bd}=\frac{a^\prime d^\prime+b^\prime c^\prime}{b^\prime d^\prime}.\]
This is true if and only if
\[b^\prime d^\prime(ad+bc)=bd(a^\prime d^\prime+b^\prime c^\prime),\]
or in other words,
\[b^\prime d^\prime ad+b^\prime d^\prime bc=bda^\prime d^\prime+bdb^\prime c^\prime.\]
But $ab^\prime=a^\prime b$ and $cd^\prime=c^\prime d$ by assumption. Hence the above equation holds.

\item[Multiplication] Suppose $a/b=a^\prime/b^\prime$ and $c/d=c^\prime/d^\prime$.
\end{description}

The verification that $\mathrm{Frac}(R)$ is a commutative ring with identity is left as an exercise; note that the additive identity is $0/1$, and the multiplicative identity is $1/1$ (where $1$ is the identity of $R$).

We now show that $\mathrm{Frac}(R)$ is a field.
Note that if $a/b=0/1$, then $(a,b)\sim(0,1)$, so $a=0$.
Thus if $a/b\neq0/1$ is a non-zero element, then $a\neq0$. 
Then $(b,a)$ and subsequently $b/a$ is well-defined (since $a\neq 0$).
The multiplicative of $a/b$ is then $b/a$:
\[\frac{a}{b}\frac{b}{a}=\frac{b}{a}\frac{a}{b}=\frac{ab}{ab}=\frac{1}{1}.\]
Hence every non-zero element in $\mathrm{Frac}(R)$ has a multiplicative inverse, so $\mathrm{Frac}(R)$ is a field.
\end{proof}

\begin{example} \
\begin{itemize}
\item $\QQ=\mathrm{Frac}(\ZZ)$.
\item $\QQ[i]=\mathrm{Frac}(\ZZ[i])$, the field of Gaussian rationals.
\item The quotient field of a field is canonically isomorphic to the field itself.
\end{itemize}
\end{example}

\begin{lemma}
$R$ is embedded in $\mathrm{Frac}(R)$.
\end{lemma}

\begin{proof}
Consider the map
\begin{align*}
\phi\colon R&\to\mathrm{Frac}(R)\\
a&\mapsto a/1
\end{align*}
We claim that $\phi$ is an embedding (injective homomorphism).
\begin{enumerate}
\item $\phi$ is a homomorphism:
\begin{align*}
\phi(a+b)&=\frac{a+b}{1}=\frac{a}{1}+\frac{b}{1}=\phi(a)+\phi(b)\\
\phi(ab)&=\frac{ab}{1}=\frac{a}{1}\frac{b}{1}=\phi(a)\phi(b)\\
\phi(1)&=\frac{1}{1}
\end{align*}

\item $\phi$ is injective: $\phi(a)=\phi(b)\implies a/1=b/1\implies a=b$.
\end{enumerate}
\end{proof}

We often think of rationals as an integer dividing another non-zero integer, instead of considering them as equivalence classes.
We now show this more generally.

\begin{lemma}
Suppose $R$ is a subring of a field $F$. (Thus $R$ is an integral domain.) Then
\[\mathrm{Frac}(R)\cong\{ab^{-1}\mid a,b\in R, b\neq0\}.\]
\end{lemma}

\begin{proof}
We see that $\{ab^{-1}\mid a,b\in R, b\neq0\}$ is a field, which is a subfield of $F$.
Consider the map
\[a/b\mapsto ab^{-1}.\]
We claim this is an isomorphism.
\end{proof}

Hence we often call the field $\{ab^{-1}\mid a,b\in R, b\neq0\}$ the \emph{quotient field} of $R$ in $F$; there can be no confusion with this terminology due to the above isomorphism. In view of this, the element $ab^{-1}$ of $F$ is also denoted by $a/b$. 

\begin{proposition}
Let $\phi\colon R\to F$ be an embedding of an integral domain $R$ into a field $F$. 
Then there exists a unique extension $\phi^*\colon\mathrm{Frac}(R)\to F$ which is also an embedding.

($\phi^*$ being an extension of $\phi$ means $\phi^*|_{R}=\phi$.)
\end{proposition}

\begin{proof} \

\fbox{Existence} Define the map
\begin{align*}
\phi^*\colon\mathrm{Frac}(R)&\to F\\
\frac{a}{b}&\mapsto\frac{\phi(a)}{\phi(b)}
\end{align*}
\begin{enumerate}
\item We first check that $\phi^*$ is well-defined. 
Suppose $a/b=c/d$. Then $ad=bc$, so $\phi(ad)=\phi(bc)$, or $\phi(a)\phi(d)=\phi(b)\phi(c)$, which implies $\frac{\phi(a)}{\phi(b)}=\frac{\phi(c)}{\phi(d)}$. (Note that $b,d\neq0$, so $\phi(b),\phi(d)\neq0$, since $\ker\phi=\{0\}$ due to injectivity.)

\item $\phi^*$ is a homomorphism:
\begin{align*}
\phi^*\brac{\frac{a}{b}+\frac{c}{d}}&=\phi^*\brac{\frac{ad+bc}{bd}}=\frac{\phi(ad+bc)}{\phi(bd)}=\frac{\phi(a)\phi(d)+\phi(b)\phi(c)}{\phi(b)\phi(d)}\\
&=\frac{\phi(a)}{\phi(b)}+\frac{\phi(c)}{\phi(d)}=\phi^*\brac{\frac{a}{b}}+\phi^*\brac{\frac{c}{d}}\\
\phi^*\brac{\frac{a}{b}\frac{c}{d}}&=\phi^*\brac{\frac{ac}{bd}}=\frac{\phi(ac)}{\phi(bd)}=\frac{\phi(a)\phi(c)}{\phi(b)\phi(d)}=\frac{\phi(a)}{\phi(b)}\frac{\phi(c)}{\phi(d)}=\phi^*\brac{\frac{a}{b}}\phi^*\brac{\frac{c}{d}}\\
\phi^*\brac{\frac{1}{1}}&=\frac{\phi(1)}{\phi(1)}=\frac{1}{1}=1
\end{align*}

\item $\phi^*$ is injective: Let $a/b\in\ker\phi^*$. Then $\phi^*(a/b)=\phi(a)/\phi(b)=0$, so $\phi(a)=0$. By injectivity, $a=0$, since $\ker\phi=\{0\}$. This implies $a/b=0/1$, so $\ker\phi^*=\{0\}$.

\item $\phi^*$ is an extension of $\phi$: Since $\phi(1)=1$, we have $\phi^*(a/1)=\phi(a)/1=\phi(a)$ for all $a\in R$.
\end{enumerate}

\fbox{Uniqueness} 
Suppose we have yet to define $\phi^*$ as above. Then
\[\phi^*\brac{\frac{a}{b}}=\phi^*\brac{\frac{a}{1}\frac{1}{b}}=\phi^*\brac{\frac{a}{1}}\phi^*\brac{\frac{1}{b}}=\phi^*\brac{\frac{a}{1}}\phi^*\brac{\frac{b}{1}}^{-1}=\phi(a)\phi(b)^{-1}.\]
Hence there is only one map $\phi^*$, defined as above, which satisfies the above conditions.
\end{proof}
\pagebreak

\section{Euclidean Domains, Principal Ideal Domains, and Unique Factorisation Domains}
\subsection{Euclidean Domains}
\begin{definition}[Euclidean domain]
An integral domain $R$ is called a \vocab{Euclidean domain}\index{Euclidean domain} if there exists $d:R\setminus\{0\}\to\ZZ_{\ge0}$ that satisfies: for all $a,b\in R$, $b\neq0$, there exists $q,r\in R$ such that $a=bq+r$ and $r=0$ or $d(r)<d(b)$.
\end{definition}

\begin{example} \
\begin{itemize}
\item Consider $\ZZ$. Let $a,b\in\ZZ$, $b\neq0$. Then $a=bq+r$, $d:\ZZ\setminus\{0\}\to\ZZ_{\ge0}$ and $d(x)=|x|$.
\item Let $F$ be a field, $F[x]$ be the ring of polynomials with elements of $F$ as coefficients. Consider long division. $d:F[x]\setminus\{0\}\to\ZZ_{\ge0}$ and $d(f(x))\colonequals\deg f$.
\item Consider $\ZZ[i]=\{a+bi\mid a,b\in\ZZ\}$ where $i^2=-1$. Then $\ZZ[i]$ is an integral domain with unit $1=1+0i$ Then $d:\ZZ[i]\setminus\{0\}\to\ZZ_{\ge0}$ under $d(a+bi)=a^2+b^2$.
\end{itemize}
\end{example}

\begin{theorem}
Let $R$ be a Euclidean domain, $I$ be an ideal in $R$. Then there exists $a_0\in R$ such that $I=Ra_0$, i.e., $I$ is a principal ideal.
\end{theorem}

\subsection{Principal Ideal Domains}
\begin{definition}[Principal ideal domain]
A commutative ring $R$ such that all ideals are principal is called a \vocab{principal ideal domain} (PID).
\end{definition}

\begin{proposition}
Every Euclidean domain is a PID.
\end{proposition}

\begin{proposition}
Every field is a Euclidean domain.
\end{proposition}

\subsection{Unique Factorisation Domains}

\pagebreak

\section{Polynomial Rings}
\subsection{Polynomials and Polynomial Functions}
Let $R$ be a commutative ring.
Define the \vocab{polynomial ring}
\[R[t]\colonequals\{a_0+a_1t+\cdots+a_nt^n\mid a_i\in R\}.\]
That is, $R[t]$ is the set of polynomials in $t$ with coefficients in $R$.

\begin{remark}
A rigorous definition of the polynomial ring can be found in \cite{lang-undergrad-algebra}.
\end{remark}

Let $R$ be a subring of a commutative ring $S$. If $f\in R[t]$ is a polynomial, then we may define the associated \emph{polynomial function}
\[f_S\colon S\to S\]
by letting for $x\in S$
\[f_S(x)=f(x)=a_0+a_1x+\cdots+a_nx^n.\]
Hence $f_S$ is a function (mapping) from $S$ to itself, determined by the polynomial $f$. 

Given $x\in S$, there is a homomorphism $R[t]\to S$ which maps $f\mapsto f_S(x)$.
(show why)



\subsection{Greatest Common Divisor}

\subsection{Unique Factorisation}

\subsection{Partial Fractions}

\subsection{Polynomials Over Rings and Over the Integers}

\subsection{Principal Rings and Factorial Rings}

\subsection{Polynomials in Several Variables}

\subsection{Symmetric Polynomials}

\subsection{The Mason--Stothers Theorem}

\subsection{The $abc$ Conjecture}

\begin{comment}

Let $R$ be a commutative ring with $1$. Let $x$ be a formal variable. We define the \vocab{polynomial ring} $R[x]$ as
\[R[x]=\{a_nx^n+\cdots+a_1x+a_0\mid a_i\in R\}.\]
Addition works as follows:
\[f(x)=\sum_{k=0}^{n}a_kx^k, g(x)=\sum_{j=0}^{m}b_jx^j,\quad f(x)+g(x)=\sum_{i=0}^{\max\{n,m\}}(a_i+b_i)x^i.\]
Multiplication works as follows:
\[f(x)g(x)=\sum_{k=0}^{m+n}\brac{\sum_{i=0}^{k}a_ib_{k-i}}x^k.\]

\begin{lemma}
Let $R$ be an integral domain. Then $R[x]$ is also an integral domain.
\end{lemma}

\begin{definition}
Let $R\subset R[x]$ be a subring, $f(x)\in R[x]$, $f(x)=a_nx^n+\cdots+a_1x+a_0$.
\begin{itemize}
\item If $a_n\neq0$, then $\deg f\colonequals n$.
\item If $a_n=1$, then $f$ is a \emph{monic polynomial}.
\item If $R$ is an integral domain, then $f(x)g(x)\in R[x]\setminus\{0\}$, $\deg fg=\deg f+\deg g$.
\end{itemize}
\end{definition}

\begin{remark}
For general commutative rings, $\deg fg\le \deg f+\deg g$.
\end{remark}
\end{comment}

%    \chapter{Vector Spaces and Modules}
\section{Vector Spaces and Bases}
\begin{definition}[Vector space]
A vector space $V$ over a field $K$ is an additive (abelian) group, together with a multiplication of elements of $V$ by elements of $V$ by elements of $K$, i.e. an association
\[(x,v)\mapsto xv\]
from $K\times V$ to $V$, satisfying the following conditions:

\end{definition}

\subsection{Dimension of a Vector Space}
\subsection{Matrices and Linear Maps}

\pagebreak

\section{Modules}
We may consider a generalisation of the notion of vector space over a field, namely \emph{module} over a ring. 

\begin{definition}[Module]

\end{definition}

\section{Factor Modules}
\section{Free Abelian Groups}
\section{Modules over Principal Rings}
\section{Eigenvectors and Eigenvalues}
\section{Polynomials of Matrices and Linear Maps} 
%    \chapter{Fields}\label{chap:fields}
\section{Introduction to Fields}
Recall that:
\begin{definition}[Field]
A \vocab{field} $F$ is a commutative ring with identity $1_F$ in which every non-zero element has an inverse.
\end{definition}

%subfield

One of the first invariants associated with any field $F$ is its characteristic:

\begin{definition}[Characteristic]
The \vocab{characteristic} of a field $F$, denoted $\ch(F)$, is the smallest positive integer $p$ such that $p\cdot 1_F=0$ if such a $p$ exists and is defined to be $0$ otherwise.
\end{definition}

\begin{proposition}
$\ch(F)$ is either $0$ or a prime $p$. If $\ch(F)=p$ then for any $\alpha\in F$,
\[p\cdot\alpha=\underbrace{\alpha+\cdots+\alpha}_\text{$p$ times}=0.\]
\end{proposition}

\begin{definition}[Prime subfield]
The \vocab{prime subfield} of a field $F$ is the subfield of $F$ generated by the multiplicative identity $1_F$ of $F$. It is (isomorphic to) either $\QQ$ (if $\ch(F)=0$) or $\FF_p$ (if $\ch(F)=p$).
\end{definition}

\begin{notation}
We shall usually denote the identity $1_F$ of a field $F$ simply by $1$.
\end{notation}

\begin{definition}[Extension field]
If $K$ is a field containing the subfield $F$, then $K$ is said to be an \vocab{extension field} (or simply an \emph{extension}) of $F$, denoted by $K/F$.
\end{definition}

In particular, every field $F$ is an extension of its prime subfield. The field $F$ is sometimes called the \vocab{base field} of the extension.

\chapter{Galois Theory}
\section{Basic Definitions}
Let $K$ be a field.
\begin{definition}
An isomorphism $\sigma:K\to K$ is called an \vocab{automorphism} of $K$. The collection of automorphisms of $K$ is denoted $\Aut(K)$. If $\alpha\in K$ we write $\sigma\alpha$ for $\sigma(\alpha)$.

An automorphism $\sigma\in\Aut(K)$ is said to \vocab{fix} $\alpha\in K$ if $\sigma\alpha=\alpha$. If $F\subset K$ then an automorphism $\sigma$ is said to fix $F$ if it fixes all the elements of $F$, i.e. $\sigma a=a$ for all $a\in F$.
\end{definition}
\fi

%%%%%%%%%%%%%%% LINEAR ALGEBRA
\iflinalg
    \part{Linear Algebra}\label{part:linear-algebra}
    \chapter{Vector Spaces}\label{chap:vector-spaces}
This chapter introduces vector spaces and subspaces.

\section{Definition of Vector Space}
\begin{notation}
A field is denoted by $\FF$, which can mean either $\RR$ or $\CC$. $\FF^n$ is the set of $n$-tuples whose elements belong to $\FF$:
\[\FF^n\coloneqq\{(x_1,\dots,x_n)\mid x_i\in\FF\}\]
For $(x_1,\dots,x_n)\in\FF^n$ and $i=1,\dots,n$, we say that $x_i$ is the $i$-th coordinate of $(x_1,\dots,x_n)$.
\end{notation}

\begin{definition}[Vector space]
$V$ is a \vocab{vector space}\index{vector space} over $\FF$ if the following properties hold:
\begin{enumerate}[label=(\roman*)]
\item Addition is commutative: $u+v=v+u$ for all $u,v\in V$
\item Addition is associative: $(u+v)+w=u+(v+w)$ for all $u,v,w\in V$

Multiplication is associative: $(ab)v=a(bv)$ for all $v\in V$, $a,b\in\FF$
\item Additive identity: there exists $\vb{0}\in V$ such that $v+\vb{0}=v$ for all $v\in V$
\item Additive inverse: for every $v\in V$, there exists $w\in V$ such that $v+w=\vb{0}$
\item Multiplicative identity: $1v=v$ for all $v\in V$
\item Distributive properties: $a(u+v)=au+av$ and $(a+b)v=av+bv$ for all $a,b,\in\FF$ and $u,v\in V$
\end{enumerate}
\end{definition}

\begin{notation}
For the rest of this text, $V$ denotes a vector space over $\FF$.
\end{notation}

\begin{example}
$\RR^n$ is a vector space over $\RR$, $\CC^n$ is a vector space over $\CC$.
\end{example}

Elements of a vector space are called \textbf{vectors} or \textbf{points}.

The scalar multiplication in a vector space depends on $\FF$. Thus when we
need to be precise, we will say that $V$ is a vector space over $\FF$ instead of saying simply that $V$ is a vector space. For example, $\RR^n$ is a vector space over $\RR$, and $\CC^n$ is a vector space over $\CC$. A vector space over $\RR$ is called a \textbf{real vector space}\index{vector space!real vector space}; a vector space over $\CC$ is called a \textbf{complex vector space}\index{vector space!complex vector space}.

\begin{proposition}[Uniqueness of additive identity]
A vector space has a unique additive identity.
\end{proposition}

\begin{proof}
Suppose otherwise, then $\vb{0}$ and $\vb{0}^\prime$ are additive identities of $V$. Then
\[\vb{0}^\prime=\vb{0}^\prime+\vb{0}=\vb{0}+\vb{0}^\prime=\vb{0}\]
where the first equality holds because $\vb{0}$ is an additive identity, the second equality comes from commutativity, and the third equality holds because $\vb{0}^\prime$ is an additive identity. Thus $\vb{0}^\prime=\vb{0}$.
\end{proof}

\begin{proposition}[Uniqueness of additive inverse]
Every element in a vector space has a unique additive inverse.
\end{proposition}

\begin{proof}
Suppose otherwise, then for $v\in V$, $w$ and $w^\prime$ are additive inverses of $v$. Then
\[w=w+\vb{0}=w+(v+w^\prime)=(w+v)+w^\prime=\vb{0}+w^\prime=w^\prime.\]
Thus $w=w^\prime$.
\end{proof}

Because additive inverses are unique, the following notation now makes sense.

\begin{notation}
Let $v,w\in V$. Then $-v$ denotes the additive inverse of $v$; $w-v$ is defined to be $w+(-v)$.
\end{notation}

We now prove some seemingly trivial facts.

\begin{proposition}[The number 0 times a vector]
For every $v\in V$, $0v=\vb{0}$.
\end{proposition}

\begin{proof}
For $v\in V$, we have
\[0v=(0+0)v=0v+0v.\]
Adding the additive inverse of $0v$ to both sides of the equation gives $\vb{0}=0v$.
\end{proof}

\begin{proposition}[A number times the vector 0]
For every $a\in\FF$, $a\vb{0}=\vb{0}$.
\end{proposition}

\begin{proof}
For $a\in\FF$, we have
\[a\vb{0}=a(\vb{0}+\vb{0})=a\vb{0}+a\vb{0}.\]
Adding the additive inverse of $a\vb{0}$ to both sides of the equation gives $\vb{0}=a\vb{0}$.
\end{proof}

Now we show that if an element of $V$ is multiplied by the scalar $1$, then the result is the additive inverse of the element of $V$.

\begin{proposition}[The number $-1$ times a vector]
For every $v\in V$, $(-1)v=-v$.
\end{proposition}

\begin{proof}
For $v\in V$, we have
\[v+(-1)v=1v+(-1)v=(1+(-1))v=0v=\vb{0}.\]
Since $v+(-1)v=\vb{0}$, $(-1)v$ is the additive inverse of $v$.
\end{proof}

\begin{example}
$\FF^\infty$ is defined to be the set of all sequences of elements of $\FF$:
\[\FF^\infty\coloneqq\{(x_1,x_2,\dots)\mid x_i\in\FF\}\]
\begin{itemize}
\item Addition on $\FF^\infty$ is defined by
\[(x_1,x_2,\dots)+(y_1,y_2,\dots)=(x_1+y_1,x_2+y_2,\dots)\]
\item Scalar multiplication on $\FF^\infty$ is defined by
\[\lambda(x_1,x_2,\dots)=(\lambda x_1,\lambda x_2,\dots)\]
\end{itemize}

Verify that $\FF^\infty$ becomes a vector space over $\FF$. Also verify that the additive identity in $\FF^\infty$ is $\vb{0}=(0,0,\dots)$.
\end{example}

Our next example of a vector space involves a set of functions.

\begin{example}
If $S$ is a set, $\FF^S\coloneqq\{f\mid f:S\to\FF\}$.
\begin{itemize}
\item Addition on $\FF^S$ is defined by 
\[(f+g)(x)=f(x)+g(x)\quad(\forall x\in S)\]
for all $f,g\in\FF^S$.
\item Multiplication on $\FF^S$ is defined by
\[(\lambda f)(x)=\lambda f(x)\quad(\forall x\in S)\]
for all $\lambda\in\FF$, $f\in\FF^S$.
\end{itemize}

Verify that if $S$ is a non-empty set, then $\FF^S$ is a vector space over $\FF$.

Also verify that the additive identity of $\FF^S$ is the function $0:S\to\FF$ defined by
\[0(x)=0\quad(\forall x\in S)\]
and for $f\in\FF^S$, additive inverse of $f$ is the function $-f:S\to\FF$ defined by
\[(-f)(x)=-f(x)\quad(\forall x\in S)\]
\end{example}

\begin{remark}
$\FF^n$ and $\FF^\infty$ are special cases of the vector space $\FF^S$; think of $\FF^n$ as $\FF^{\{1,2,\dots,n\}}$, and $\FF^\infty$ as $\FF^{\{1,2,\dots\}}$.
\end{remark}

\begin{example}[Complexification]
Suppose $V$ is a real vector space. The \emph{complexifcation} of $V$, denoted by $V_\CC$, equals $V\times V$. An element of $V_\CC$ is an ordered pair $(u,v)$, where $u,v\in V$, which we write as $u+iv$.
\begin{itemize}
\item Addition on $V_\CC$ is defined by
\[(u_1+iv_1)+(u_2+iv_2)=(u_1+u_2)+i(v_1+v_2)\]
for all $u_1,v_2,u_2,v_2\in V$.
\item Complex scalar multiplication on $V_\CC$ is defined by
\[(a+bi)(u+iv)=(au-bv)+i(av+bu)\]
for all $a,b\in\RR$ and all $u,v\in V$.
\end{itemize}
You should verify that with the defnitions of addition and scalar multiplication as above, $V_\CC$ is a (complex) vector space.
\end{example}

\section{Subspaces}
\begin{definition}[Subspace]
$U\subseteq V$ is a \vocab{subspace}\index{vector space!subspace} of $V$ if $U$ is also a vector space (with the same addition and scalar multiplication as on $V$). We denote this as $U\le V$.
\end{definition}

The following result is useful in determining whether a given subset of $V$ is a subspace of $V$.

\begin{lemma}[Subspace test]\label{lemma:subspace-conditions}
Suppose $U\subseteq V$. $U\le V$ if and only if $U$ satisfies the following conditions:
\begin{enumerate}[label=(\roman*)]
\item Additive identity: $\vb{0}\in U$
\item Closed under addition: $u+w\in U$ for all $u,w\in U$
\item Closed under scalar multiplication: $\lambda u\in U$ for all $\lambda\in\FF$, $u\in U$
\end{enumerate}
\end{lemma}

\begin{proof} \

\fbox{$\implies$} If $U\le V$, then $U$ satisfies the three conditions above by the definition of vector space.

\fbox{$\impliedby$} Conversely, suppose $U$ satisfies the three conditions above. (i) ensures that the additive identity of $V$ is in $U$. (ii) ensures that addition makes sense on $U$. (iii) ensures that scalar multiplication makes sense on $U$.

If $u\in U$, then $-u=(-1)u\in U$ by (iii). Hence every element of $U$ has an additive inverse in $U$.

The other parts of the definition of a vector space, such as associativity and commutativity, are automatically satisfied for $U$ because they hold on the larger space $V$. Thus $U$ is a vector space and hence is a subspace of $V$.
\end{proof}

\begin{definition}[Sum of subsets]
Suppose $U_1,\dots,U_n\subset V$. The \vocab{sum}\index{sum of subsets} of $U_1,\dots,U_n$ is the set of all possible sums of elements of $U_1,\dots,U_n$:
\[U_1+\cdots+U_n\coloneqq\{u_1+\cdots+u_n\mid u_i\in U_i\}.\]
\end{definition}

\begin{example}
Suppose that $U=\{(x,0,0)\in\FF^3\mid x\in F\}$ and $W=\{(0,y,0)\in\FF^3\mid y\in\FF\}$. Then
\[U+W=\{(x,y,0)\mid x,y\in\FF\}.\]
\end{example}

\begin{example}
Suppose that $U=\{(x,x,y,y)\in\FF^4\mid x,y\in\FF\}$ and $W=\{(x,x,x,y)\in\FF^4\mid x,y\in\FF\}$. Then
\[U+W=\{(x,x,y,z)\in\FF^4\mid x,y,z\in\FF\}.\]
\end{example}

The next result states that the sum of subspaces is a subspace, and is in fact the smallest subspace containing all the summands.

\begin{proposition}
Suppose $U_1,\dots,U_n\le V$. Then $U_1+\cdots+U_n$ is the smallest subspace of $V$ containing $U_1,\dots,U_n$.
\end{proposition}

\begin{proof}
It is easy to see that $\vb{0}\in U_1+\cdots+U_n$ and that $U_1+\cdots+U_n$ is closed under addition and scalar multiplication, hence $U_1+\cdots+U_n\le V$.

Clearly $U_1,\dots,U_n$ are all contained in $U_1+\cdots+U_n$ (to see this, consider sums $u_1+\cdots+u_n$ where all except one of the $u$'s are $\vb{0}$). Conversely, every subspace of $V$ containing $U_1,\dots,U_n$ contains $U_1+\cdots+U_n$ (because subspaces must contain all finite sums of their elements). Thus $U_1+\cdots+U_n$ is the smallest subspace of $V$ containing $U_1,\dots,U_n$.
\end{proof}

\begin{definition}[Direct sum]
Suppose $U_1,\dots,U_n\le V$. The sum $U_1+\cdots+U_n$ is called a \vocab{direct sum}\index{direct sum} if each element of $U_1+\cdots+U_n$ can be written in only one way a sum of $u_1+\cdots+u_n$, $u_i\in U_i$. In this case, we denote the sum as
\[U_1\oplus\cdots\oplus U_n.\]
\end{definition}

\begin{example}
Suppose that $U=\{(x,y,0)\in\FF^3\mid x,y\in\FF\}$ and $W=\{(0,0,z)\in\FF^3\mid z\in\FF\}$. Then $\FF^3=U\oplus W$.
\end{example}

\begin{example}
Suppose $U_i$ is the subspace of $\FF^n$ of those vectors whose coordinates are all 0 except for the $i$-th coordinate; that is, $U_i=\{(0,\dots,0,x,0,\dots,0)\in\FF^n\mid x\in\FF\}$. Then $\FF^n=U_1\oplus\cdots\oplus U_n$.
\end{example}

\begin{lemma}[Condition for direct sum]\label{lemma:condition-direct-sum}
Suppose $V_1,\dots,V_n\le V$, let $W=V_1+\cdots+V_n$. Then the following are equivalent:
\begin{enumerate}[label=(\roman*)]
\item Any element in $W$ can be uniquely expressed as the sum of vectors in $V_1,\dots,V_n$.
\item If $v_i\in V_i$ satisfies $v_1+\cdots+v_n=\vb{0}$, then $v_1=\cdots=v_n=\vb{0}$.
\item For $k=2,\dots,n$, $(V_1+\cdots+V_{k-1})\cap V_k=\{\vb{0}\}$.
\end{enumerate}
\end{lemma}

\begin{proof} \

(i)$\iff$(ii) First suppose $W$ is a direct sum. Then by the definition of direct sum, the only way to write $\vb{0}$ as a sum $u_1+\cdots+u_n$ is by taking $u_i=\vb{0}$.

Now suppose that the only way to write $\vb{0}$ as a sum $v_1+\cdots+v_n$ by taking $v_1=\cdots=v_n=\vb{0}$. For $v\in V_1+\cdots+V_n$, suppose that there is more than one way to represent $v$:
\begin{align*}
v&=v_1+\cdots+v_n\\
v&=v_1^\prime+\cdots+v_n^\prime
\end{align*}
for some $v_i,v_i^\prime\in V_i$. Substracting the above two equations gives
\[\vb{0}=(v_1-v_1^\prime)+\cdots+(v_n-v_n^\prime).\]
Since $v_i-v_i^\prime\in V_i$, we have $v_i-v_i^\prime=\vb{0}$ so $v_i=v_i^\prime$. Hence there is only one unique way to represent $v_1+\cdots+v_n$, thus $W$ is a direct sum.

(ii)$\iff$(iii) First suppose if $v_i\in V_i$ satisfies $v_1+\cdots+v_n=\vb{0}$, then $v_1=\cdots=v_n=\vb{0}$. Let $v_k\in(V_1+\cdots+V_{k-1})\cap V_k$. Then $v_k=v_1+\cdots+v_{k-1}$ where $v_i\in V_i$ ($1\le i\le k-1$). Thus
\begin{align*}
v_1+\cdots+v_{k-1}-v_k&=\vb{0}\\
v_1+\cdots+v_{k-1}+(-v_k)+\vb{0}+\cdots+\vb{0}&=\vb{0}
\end{align*}
by taking $v_{k+1}=\cdots=v_n=\vb{0}$. Then $v_1=\cdots=v_k=\vb{0}$.

Now suppose that for $k=2,\dots,n$, $(V_1+\cdots+V_{k-1})\cap V_k=\{\vb{0}\}$.
\begin{align*}
v_1+\cdots+v_n&=\vb{0}\\
v_1+\cdots+v_{n-1}&=-v_n
\end{align*}
where $v_1+\cdots+v_{n-1}\in V_1+\cdots+V_{n-1}$, $-v_n\in V_n$. Thus
\[v_1+\cdots+v_{n-1}=-v_n\in(V_1+\cdots+V_{n-1})\cap V_n=\{\vb{0}\}\]
so $v_1+\cdots+v_{n-1}=\vb{0}$, $v_n=\vb{0}$. Induction on $n$ gives $v_1=\cdots=v_{n-1}=v_n=\vb{0}$.
\end{proof}

\begin{proposition}
Suppose $U,W\le V$. Then $U+W$ is a direct sum if and only if $U\cap W=\{\vb{0}\}$.
\end{proposition}

\begin{proof} \

\fbox{$\implies$} Suppose that $U+W$ is a direct sum. If $v\in U\cap W$, then $\vb{0}=v+(-v)$, where $v\in U$, $-v\in W$. By the unique representation of $\vb{0}$ as the sum of a vector in $U$ and a vector in $W$, we have $v=\vb{0}$. Thus $U\cap W=\{\vb{0}\}$.

\fbox{$\impliedby$} Suppose $U\cap W=\{\vb{0}\}$. Suppose $u\in U$, $w\in W$, and $0=u+w$. $u=-w\in W$, thus $u\in U\cap W$, so $u=w=\vb{0}$. By \cref{lemma:condition-direct-sum}, $U+W$ is a direct sum.
\end{proof}
\pagebreak

\subsection*{Exercises}
\begin{prbm}
Suppose $W$ is a vector space over $\FF$, $V_1$ and $V_2$ are subspaces of $W$. Show that $V_1\cap V_2$ is a vector space over $\FF$ if and only if $V_1\subset V_2$ or $V_2\subset V_1$.
\end{prbm}

\begin{solution}
The backward direction is trivial. We focus on proving the forward direction.

Supppse otherwise, then $V_1\setminus V_2\neq\emptyset$ and $V_2\setminus V_1\neq\emptyset$. Pick $v_1\in V_1\setminus V_2$ and $v_2\in V_2\setminus V_1$. Then
\begin{align*}
v_1,v_2\in V_1\cup V_2&\implies v_1+v_2\in V_1\cup V_2\\
&\implies v_2,v_1+v_2\in V_2\\
&\implies v_1=(v_1+v_2)-v_2\in V_2
\end{align*}
which is a contradiction.
\end{solution}

\begin{prbm}
Suppose $W$ is a vector space over $\FF$, $V_1,V_2,V_3$ are subspaces of $W$. Then $V_1\cup V_2\cup V_3$ is a vector space over $\FF$ if and only if one of the $V_i$ contains the other two.
\end{prbm}

\begin{solution}
We prove the forward direction. Suppose otherwise, then $v_1\in V_1\setminus(V_2+V_3)$, $v_2\in V_2\setminus(V_1+V_3)$, $v_3\in V_3\setminus(V_1+V_2)$. Consider
\[\{v_1+v_2+v_3,v_1+v_2+2v_3,v_1+2v_2+v_3,v_1+2v_2+2v_3\}\subset V_1\cup V_2\cup V_3\]
Then
\begin{align*}
&(v_1+v_2+2v_3)-(v_1+v_2+v_3)=v_3\notin V_1+V_2\\
&\implies v_1+v_2+v_3\notin V_1+V_2\quad\text{or}\quad v_1+v_2+2v_3\notin V_1+V_2\\
&\implies v_1+v_2+v_3\in V_3\quad\text{or}\quad v_1+v_2+2v_3\in V_3\\
&\implies v_1+v_2\in V_3
\end{align*}
Similarly,
\begin{align*}
&(v_1+2v_2+2v_3)-(v_1+2v_2+v_3)=v_3\notin V_1+V_2\\
&\implies v_1+2v_2+v_3\notin V_1+V_2\quad\text{or}\quad v_1+2v_2+2v_3\notin V_1+V_2\\
&\implies v_1+2v_2+v_3\in V_3\quad\text{or}\quad v_1+2v_2+2v_3\in V_3\\
&\implies v_1+2v_2\in V_3
\end{align*}
This implies $(v_1+2v_2)-(v_1+v_2)=v_2\in V_3$, a contradiction.
\end{solution}
    \chapter{Linear Maps}\label{chap:linear-maps}
\section{Vector Space of Linear Maps}
\begin{definition}[Linear map]
A \vocab{linear map}\index{linear map} from $V$ to $W$ is a function $T:V\to W$ satisfying the following properties:
\begin{enumerate}[label=(\roman*)]
\item Additivity: $T(v+w)=Tv+Tw$ for all $v,w\in V$
\item Homogeneity: $T(\lambda v)=\lambda T(v)$ for all $\lambda\in\FF$, $v\in V$
\end{enumerate}
\end{definition}

\begin{notation}
The set of linear maps from $V$ to $W$ is denoted by $\mathcal{L}(V,W)$; the set of linear maps on $V$ (from $V$ to $V$) is denoted by $\mathcal{L}(V)$.
\end{notation}

The existence part of the next result means that we can find a linear map that takes on whatever values we wish on the vectors in a basis. The uniqueness part of the next result means that a linear map is completely determined by its values on a basis.

\begin{lemma}[Linear map lemma]
Suppose $\{v_1,\dots,v_n\}$ is a basis of $V$, and $w_1,\dots,w_n\in W$. Then there exists a unique linear map $T:V\to W$ such that
\[Tv_i=w_i\quad(i=1,\dots,n)\]
\end{lemma}

\begin{proof}
First we show the existence of a linear map $T$ with the desired property. Define $T:V\to W$ by
\[T(c_1v_1+\cdots+c_nv_n)=c_1w_1+\cdots+c_nw_n,\]
for some $c_i\in\FF$. Since $\{v_1,\dots,v_n\}$ is a basis of $V$, by \cref{lemma:basis-criterion}, each $v\in V$ can be uniquely expressed as a linear combination of $v_1,\dots,v_n$, thus the equation above does indeed define a function $T:V\to W$. For $i$ ($1\le i\le n$), take $c_i=1$ and the other $c$'s equal to $0$, then
\[T\brac{0v_1+\cdots+1v_i+\cdots+0v_n}=0w_1+\cdots+1w_i+\cdots+0w_n\]
which shows that $Tv_i=w_i$.

We now show that $T:V\to W$ is a linear map:
\begin{enumerate}[label=(\roman*)]
\item For $u,v\in V$ with $u=a_1v_1+\cdots+a_nv_n$ and $c_1v_1+\cdots+c_nv_n$,
\begin{align*}
T(u+v)&=T\brac{(a_1+c_1)v_1+\cdots+(a_n+c_n)v_n}\\
&=(a_1+c_1)w_1+\cdots+(a_n+c_n)w_n\\
&=(a_1w_1+\cdots+a_nw_n)+(c_1w_1+\cdots+c_nw_n)\\
&=Tu+Tv.
\end{align*}

\item For $\lambda\in\FF$ and $v=c_1v_1+\cdots+c_nv_n$,
\begin{align*}
T(\lambda v)&=T(\lambda c_1v_1+\cdots+\lambda c_nv_n)\\
&=\lambda c_1w_1+\cdots+\lambda c_nw_n\\
&=\lambda(c_1w_1+\cdots+c_nw_n)\\
&=\lambda Tv.
\end{align*}
\end{enumerate}

To prove uniqueness, now suppose that $T\in\mathcal{L}(V,W)$ and $Tv_i=w_i$ for $i=1,\dots,n$. Let $c_i\in\FF$. The homogeneity of $T$ implies that $T(c_iv_i)=c_iw_i$. The additivity of $T$ now implies that 
\[T(c_1v_1+\cdots+c_nv_n)=c_1w_1+\cdots+c_nw_n.\]
Thus T is uniquely determined on $\spn\{v_1,\dots,v_n\}$. Since $\{v_1,\dots,v_n\}$ is a basis of $V$, this implies that $T$ is uniquely determined on $V$.
\end{proof}

\begin{proposition}
$\mathcal{L}(V,W)$ is a vector space, with the operations addition and scalar multiplication defined as follows: suppose $S,T\in\mathcal{L}(V,W)$, $\lambda\in\FF$,
\begin{enumerate}[label=(\roman*)]
\item $(S+T)(v)=Sv+Tv$
\item $(\lambda T)(v)=\lambda(Tv)$
\end{enumerate}
for all $v\in V$.
\end{proposition}

\begin{proof}
Exercise.
\end{proof}

\begin{definition}[Product of linear maps]
$T\in\mathcal{L}(U,V)$, $S\in\mathcal{L}(V,W)$, then the \vocab{product} $ST\in\mathcal{L}(U,W)$ is defined by
\[(ST)(u)=S(Tu)\quad(\forall u\in U)\]
\end{definition}

\begin{remark}
In other words, $ST$ is just the usual composition $S\circ T$ of two functions.
\end{remark}

\begin{remark}
$ST$ is defined only when $T$ maps into the domain of $S$.
\end{remark}

\begin{proposition}[Algebraic properties of products of linear maps] \
\begin{enumerate}[label=(\roman*)]
\item Associativity: $(T_1T_2)T_3=T_1(T_2T_3)$ for all linear maps $T_1,T_2,T_3$ such that the products make sense (meaning that $T_3$ maps into the domain of $T_2$, $T_2$ maps into the domain of $T_1$)
\item Iidentity: $TI=IT=T$ for all $T\in\mathcal{L}(V,W)$ (the first $I$ is the identity map on $V$, and the second $I$ is the identity map on $W$)
\item Distributive: $(S_1+S_2)T=S_1T+S_2T$ and $S(T_1+T_2)=ST_1+ST_2$ for all $T,T_1,T_2\in\mathcal{L}(U,V)$ and $S,S_1,S_2\in\mathcal{L}(V,W)$
\end{enumerate}
\end{proposition}

\begin{proof}
Exercise.
\end{proof}

\begin{proposition}\label{prop:linear-map-0-0}
Suppose $T\in\mathcal{L}(V,W)$. Then $T(\vb{0})=\vb{0}$.
\end{proposition}

\begin{proof}
By additivity, we have
\[T(\vb{0})=T(\vb{0}+\vb{0})=T(\vb{0})+T(\vb{0}).\]
Add the additive inverse of $T(\vb{0})$ to each side of the equation to conclude that $T(\vb{0})=\vb{0}$.
\end{proof}
\pagebreak

\section{Kernel and Image}
\begin{definition}[Kernel]
Suppose $T\in\mathcal{L}(V,W)$. The \vocab{kernel}\index{kernel} of $T$ is the subset of $V$ consisting of those vectors that $T$ maps to $\vb{0}$:
\[\ker T\coloneqq\{v\in V\mid Tv=\vb{0}\}\subset V.\]
\end{definition}

\begin{proposition}
Suppose $T\in\mathcal{L}(V,W)$. Then $\ker T\le V$.
\end{proposition}

\begin{proof}
By \cref{lemma:subspace-conditions}, we check the conditions of a subspace:
\begin{enumerate}[label=(\roman*)]
\item By \cref{prop:linear-map-0-0}, $T(\vb{0})=\vb{0}$, so $\vb{0}\in\ker T$.
\item For all $v,w\in\ker T$, 
\[T(v+w)=Tv+Tw=\vb{0}\implies v+w\in\ker T\]
so $\ker T$ is closed under addition.
\item For all $v\in\ker T$, $\lambda\in\FF$,
\[T(\lambda v)=\lambda Tv=\vb{0}\implies\lambda v\in\ker T\]
so $\ker T$ is closed under scalar multiplication.
\end{enumerate}
\end{proof}

\begin{definition}[Injectivity]
Suppose $T\in\mathcal{L}(V,W)$. $T$ is \vocab{injective}\index{injectivity} if
\[Tu=Tv\implies u=v.\]
\end{definition}

\begin{proposition}
Suppose $T\in\mathcal{L}(V,W)$. Then $T$ is injective if and only if $\ker T=\{\vb{0}\}$.
\end{proposition}

\begin{proof} \

\fbox{$\implies$} Suppose $T$ is injective. Let $v\in\ker T$, then
\[Tv=\vb{0}=T(\vb{0})\implies v=\vb{0}\]
by the injectivity of $T$. Hence $\ker T=\{\vb{0}\}$ as desired.

\fbox{$\impliedby$} Suppose $\ker T=\{\vb{0}\}$. Let $u,v\in V$ such that $Tu=Tv$. Then
\[T(u-v)=Tu-Tv=\vb{0}.\]
By definition of kernel, $u-v\in\ker T=\{\vb{0}\}$, so $u-v=\vb{0}$, which implies that $u=v$. Hence $T$ is injective, as desired.
\end{proof}

\begin{definition}[Image]
Suppose $T\in\mathcal{L}(V,W)$. The \vocab{image}\index{image} of $T$ is the subset of $W$ consisting of those vectors that are of the form $Tv$ for some $v\in V$:
\[\im T\coloneqq\{Tv\mid v\in V\}\subset W.\]
\end{definition}

\begin{proposition}
Suppose $T\in\mathcal{L}(V,W)$. Then $\im T\le W$.
\end{proposition}

\begin{proof} \
\begin{enumerate}[label=(\roman*)]
\item $T(\vb{0})=\vb{0}$ implies that $\vb{0}\in\im T$.
\item For $w_1,w_2\in\im T$, there exist $v_1,v_2\in V$ such that $Tv_1=w_1$ and $Tv_2=w_2$. Then
\[w_1+w_2=Tv_1+Tv_2=T(v_1+v_2)\in\im T\implies w_1+w_2\in\im T.\]
\item For $w\in\im T$ and $\lambda\in\FF$, there exists $v\in V$ such that $Tv=w$. Then
\[\lambda w=\lambda Tv=T(\lambda v)\in\im T\implies\lambda w\in\im T.\]
\end{enumerate}
\end{proof}

\begin{definition}[Surjectivity]
Suppose $T\in\mathcal{L}(V,W)$. $T$ is \vocab{surjective}\index{surjectivity} if $\im T=W$.
\end{definition}

\subsection{Fundamental Theorem of Linear Maps}
\begin{theorem}[Fundamental theorem of linear maps]
Suppose $V$ is finite-dimensional, $T\in\mathcal{L}(V,W)$. Then $\im T$ is finite-dimensional, and
\begin{equation}
\dim V=\dim\ker T+\dim\im T.
\end{equation}
\end{theorem}

\begin{proof}
Let $\{u_1,\dots,u_m\}$ be basis of $\ker T$, then $\dim\ker T=m$. The linearly independent list $u_1,\dots,u_m$ can be extended to a basis
\[\{u_1,\dots,u_m,v_1,\dots,v_n\}\]
of $V$, thus $\dim V=m+n$. To simultaneously show that $\im T$ is finite-dimensional and $\dim\im T=n$, we prove that $\{Tv_1,\dots,Tv_n\}$ is a basis of $\im T$. Thus we need to show that the set (i) spans $\im T$, and (ii) is linearly independent.

\begin{enumerate}[label=(\roman*)]
\item Let $v\in V$. Since $\{u_1,\dots,u_m,v_1,\dots,v_n\}$ spans $V$, we can write
\[v=a_1u_1+\cdots+a_mu_m+b_1v_1+\cdots+b_nv_n,\]
for some $a_i,b_i\in\FF$. Applying $T$ to both sides of the equation, and noting that $Tu_i=\vb{0}$ since $u_i\in\ker T$,
\begin{align*}
Tv&=T\brac{a_1u_1+\cdots+a_mu_m+b_1v_1+\cdots+b_nv_n}\\
&=a_1\underbrace{Tu_1}_{\vb{0}}+\cdots+a_m\underbrace{Tu_m}_{\vb{0}}+b_1Tv_1+\cdots+b_nv_n\\
&=b_1Tv_1+\cdots+b_nTv_n\in\im T.
\end{align*}
Since every element of $\im T$ can be expressed as a linear combination of $Tv_1,\dots,Tv_n$, we have that $\{Tv_1,\dots,Tv_n\}$ spans $\im T$.

Moreover, since there exists a set of vectors that spans $\im T$, $\im T$ is finite-dimensional.

\item Suppose there exist $c_1,\dots,c_n\in\FF$ such that
\[c_1Tv_1+\cdots+c_nTv_n=\vb{0}.\]
Then
\[T(c_1v_1+\cdots+c_nv_n)=T(\vb{0})=\vb{0},\]
which implies $c_1v_1+\cdots+c_nv_n\in\ker T$. Since $\{u_1,\dots,u_m\}$ is a spanning set of $\ker T$, we can write
\[c_1v_1+\cdots+c_nv_n=d_1u_1+\cdots+d_mu_m\]
for some $d_i\in\FF$, or
\[c_1v_1+\cdots+c_nv_n-d_1u_1-\cdots-d_mu_m=\vb{0}.\]
Since $u_1,\dots,u_m,v_1,\dots,v_n$ are linearly independent, $c_i=d_i=0$. Since $c_i=0$, $\{Tv_1,\dots,Tv_n\}$ is linearly independent.
\end{enumerate}
\end{proof}

We now show that no linear map from a finite-dimensional vector space to a ``smaller'' vector space can be injective, where ``smaller'' is measured by dimension.

\begin{proposition}
Suppose $V$ and $W$ are finite-dimensional vector spaces, $\dim V>\dim W$. Then there does not exist $T\in\mathcal{L}(V,W)$ such that $T$ is injective.
\end{proposition}

\begin{proof}
Since $W$ is finite-dimensional and $\im T\le W$, by \cref{prop:dim-subspace}, we have that $\dim\im T\le\dim W$.

Let $T\in\mathcal{L}(V,W)$. Then
\begin{align*}
\dim\ker T&=\dim V-\dim\im T\tag{1}\\
&\ge\dim V-\dim W\tag{2}\\
&>0
\end{align*}
where (1) follows from the fundamental theorem of linear maps, (2) follows from the above claim.

Since $\dim\ker T>0$. This means that $\ker T$ contains some $v\in V\setminus\{\vb{0}\}$. Since $\ker T\neq\{\vb{0}\}$, $T$ is not injective.
\end{proof}

The next result shows that no linear map from a finite-dimensional vector space to a ``bigger'' vector space can be surjective, where ``bigger'' is also measured by dimension.

\begin{proposition}
Suppose $V$ and $W$ are finite-dimensional vector spaces, $\dim V<\dim W$. Then there does not exist $T\in\mathcal{L}(V,W)$ such that $T$ is surjective.
\end{proposition}

\begin{proof}
Let $T\in\mathcal{L}(V,W)$. Then
\begin{align*}
\dim\im T&=\dim V-\dim\ker T\tag{1}\\
&\le\dim V\tag{2}\\
&<\dim W,
\end{align*}
where (1) follows from the fundamental theorem of linear maps, (2) follows since the dimension of a vector space is non-negative so $\dim\ker T\ge0$.

Since $\dim\im T<\dim W$, $\im T\neq W$ so $T$ is not surjective.
\end{proof}

\begin{example}[Homogeneous system of linear equations]
Consider the homogeneous system of linear equations
\begin{equation*}\tag{$\ast$}
\begin{split}
a_{11}x_1+a_{12}x_2+\cdots+a_{1n}x_n&=0\\
a_{21}x_1+a_{22}x_2+\cdots+a_{2n}x_n&=0\\
\vdots&\\
a_{m1}x_1+a_{m2}x_2+\cdots+a_{mn}x_n&=0
\end{split}
\end{equation*}
where $a_{ij}\in\FF$.

Define $T:\FF^n\to\FF^m$ by
\[T\brac{x_1,\dots,x_n}=\brac{\sum_{i=1}^{n}a_{1i}x_i,\dots,\sum_{i=1}^{m}a_{mi}x_i}.\]
The solution set of $(\ast)$ is given by
\[\ker T=\crbrac{(x_1,\dots,x_n)\in\FF^n\:\bigg|\:\sum_{i=1}^{n}a_{1i}x_i=0,\dots,\sum_{i=1}^{n}a_{mi}x_i=0}.\]

\begin{proposition*}
A homogeneous system of linear equations with more variables than equations has non-zero solutions.
\end{proposition*}

\begin{proof}
If $n>m$, then
\begin{align*}
\dim\FF^n>\dim\FF^m&\implies T\text{ is not injective}\\
&\implies\ker T\neq\{\vb{0}\}\\
&\implies(\ast)\text{ has non-zero solutions}
\end{align*}
\end{proof}

\begin{proposition*}
A system of linear equations with more equations than variables has no solution for some choice of the constant terms.
\end{proposition*}

\begin{proof}
If $n<m$, then
\begin{align*}
\dim\FF^n<\dim\FF^m&\implies T\text{ is not surjective}\\
&\implies\exists(c_1,\dots,c_m)\in\FF^m, \forall(x_1,\dots,x_n)\in\FF^n, T(x_1,\dots,x_n)\neq(c_1,\dots,c_m)
\end{align*}
Thus the choice of constant terms $(c_1,\dots,c_m)$ is such that the system of linear equations
\begin{align*}
a_{11}x_1+\cdots+a_{1n}x_n&=c_1\\
\vdots&\\
a_{m1}x_1+\cdots+a_{mn}x_n&=c_m
\end{align*}
has no solutions $(x_1,\dots,x_n)$.
\end{proof}
\end{example}
\pagebreak

\section{Matrices}
\subsection{Representing a Linear Map by a Matrix}
\begin{definition}[Matrix]
Suppose $m,n\in\NN$. An $m\times n$ \vocab{matrix}\index{matrix} $A$ is a rectangular array with $m$ rows and $n$ columns:
\[A=\begin{pmatrix}
a_{11} & \cdots & a_{1n}\\
\vdots & & \vdots\\
a_{m1} & \cdots & a_{mn}
\end{pmatrix}\]
where $a_{ij}\in\FF$ denotes the entry in row $i$, column $j$. We also denote $A=(a_{ij})_{m\times n}$, and drop the subscript if there is no ambiguity.
\end{definition}

\begin{notation}
$i$ is used for indexing across the $m$ rows, $j$ is used for indexing across the $n$ columns.
\end{notation}

\begin{notation}
$\mathcal{M}_{m\times n}(\FF)$ denotes the set of $m\times n$ matrices with entries in $\FF$.
\end{notation}

As we will soon see, matrices provide an efficient method of recording the values of $Tv_j$'s in terms of a basis of $W$.

\begin{definition}[Matrix of linear map]
Suppose $T\in\mathcal{L}(V,W)$, $\mathcal{V}=\{v_1,\dots,v_n\}$ is a basis of $V$, $\mathcal{W}=\{w_1,\dots,w_m\}$ is a basis of $W$. The matrix of $T$\index{matrix of linear map} with respect to these bases is the $m\times n$ matrix $\mathcal{M}(T)$, whose entries $a_{ij}$ are defined by
\[Tv_j=\sum_{i=1}^{m}a_{ij}w_i.\]
That is, the $j$-th column of $\mathcal{M}(T)$ consists of the scalars $a_{1j},\dots,a_{mj}$ needed to write $Tv_j$ as a linear combination of the bases of $W$.
\end{definition}

\begin{notation}
If the bases of $V$ and $W$ are not clear from the context, we adopt the notation $\mathcal{M}(T;\mathcal{V},\mathcal{W})$.
\end{notation}

%A useful way to remember how $\mathcal{M}(T)$ is constructed from $T$ is to write the bases of $V$ across the top of the matrix, and the bases in $W$ along the left:

\begin{comment}
That is,
\begin{align*}
Tv_1&=a_{11}w_1+a_{21}w_2+\cdots+a_{m1}w_m\\
Tv_2&=a_{12}w_1+a_{22}w_2+\cdots+a_{m2}w_m\\
&\vdots\\
Tv_n&=a_{1n}w_1+a_{2n}w_2+\cdots+a_{mn}w_m
\end{align*}

Thus we can write
\begin{align*}
\mathcal{M}(T)&=T\begin{pmatrix}
v_1&v_2&\cdots&v_n
\end{pmatrix}\\
&=\begin{pmatrix}
w_1&w_2&\cdots&w_m
\end{pmatrix}
\begin{pmatrix}
a_{11}&a_{12}&\cdots&a_{1n}\\
a_{21}&a_{22}&\cdots&a_{2n}\\
\vdots&\vdots&\ddots&\vdots\\
a_{m1}&a_{m2}&\cdots&a_{mn}
\end{pmatrix}\\
&=\begin{pmatrix}
\sum_{k=1}^{m}a_{k1}w_k & \cdots & \sum_{k=1}^{m}a_{kn}w_k
\end{pmatrix}
\end{align*}
\end{comment}

\subsection{Addition and Scalar Multiplication of Matrices}
\begin{definition}[Matrix operations] \
\begin{enumerate}[label=(\roman*)]
\item Addition: the sum of two matrices of the same size is the matrix obtained by adding corresponding entries in the matrices:
\[\begin{pmatrix}
a_{11}&\cdots&a_{1n}\\
\vdots&&\vdots\\
a_{m1}&\cdots&a_{mn}
\end{pmatrix}+
\begin{pmatrix}
c_{11}&\cdots&c_{1n}\\
\vdots&&\vdots\\
c_{m1}&\cdots&c_{mn}
\end{pmatrix}=
\begin{pmatrix}
a_{11}+c_{11}&\cdots&a_{1n}+c_{1n}\\
\vdots&&\vdots\\
a_{m1}+c_{m1}&\cdots&a_{mn}+c_{mn}
\end{pmatrix}.\]

\item Scalar multiplication: the product of a scalar and a matrix is the matrix obtained by multiplying each entry in the matrix by the scalar:
\[\lambda\begin{pmatrix}
a_{11}&\cdots&a_{1n}\\
\vdots&&\vdots\\
a_{m1}&\cdots&a_{mn}
\end{pmatrix}=
\begin{pmatrix}
\lambda a_{11}&\cdots&\lambda a_{1n}\\
\vdots&&\vdots\\
\lambda a_{m1}&\cdots&\lambda a_{mn}
\end{pmatrix}.\]
\end{enumerate}
\end{definition}

\begin{proposition}
Suppose $S,T\in\mathcal{L}(V,W)$. Then
\begin{enumerate}[label=(\roman*)]
\item $\mathcal{M}(S+T)=\mathcal{M}(S)+\mathcal{M}(T)$;
\item $\mathcal{M}(\lambda T)=\lambda\mathcal{M}(T)$ for $\lambda\in\FF$.
\end{enumerate}
\end{proposition}

\begin{proof}
Suppose $S,T\in\mathcal{L}(V,W)$, $\{v_1,\dots,v_n\}$ is a basis of $V$, $\{w_1,\dots,w_m\}$ is a basis of $W$.
\begin{enumerate}[label=(\roman*)]
\item By definition, $\mathcal{M}(S)$ is the matrix whose entries $a_{ij}$ are defined by
\[Sv_j=\sum_{i=1}^{m}a_{ij}w_i.\]
Similarly, $\mathcal{M}(T)$ is the matrix whose entries $b_{ij}$ are defined by
\[Tv_j=\sum_{i=1}^{m}b_{ij}w_i.\]
$\mathcal{M}(S+T)$ is the matrix whose entries $c_{ij}$ are defined by
\begin{align*}
(S+T)v_j&=\sum_{i=1}^{m}c_{ij}w_i\\
Sv_j+Tv_j&=\sum_{i=1}^{m}c_{ij}w_i\\
\sum_{i=1}^{m}a_{ij}w_i+\sum_{i=1}^{m}b_{ij}w_i&=\sum_{i=1}^{m}c_{ij}w_i\\
\sum_{i=1}^{m}(a_{ij}+b_{ij})w_i&=\sum_{i=1}^{m}c_{ij}w_i\\
a_{ij}+b_{ij}&=c_{ij}.
\end{align*}

\item By definition, $\mathcal{M}(T)$ is the matrix whose entries $a_{ij}$ are defined by
\[Tv_j=\sum_{i=1}^{m}a_{ij}w_i.\]
Then for $\lambda\in\FF$, $\mathcal{M}(\lambda T)$ is the matrix whose entries $b_{ij}$ are defined by
\begin{align*}
\lambda Tv_j&=\sum_{i=1}^{m}b_{ij}w_i\\
\lambda\sum_{i=1}^{m}a_{ij}w_i&=\sum_{i=1}^{m}b_{ij}w_i\\
\lambda a_{ij}&=b_{ij}.
\end{align*}
\end{enumerate}
\end{proof}

\begin{proposition}
With addition and scalar multiplication defined as above, $\mathcal{M}_{m\times n}(\FF)$ is a vector space of dimension $mn$.
\end{proposition}

\begin{proof}
The verification that $\mathcal{M}_{m\times n}(\FF)$ is a vector space is left to the reader. Note that the additive identity of $\mathcal{M}_{m\times n}(\FF)$ is the $m\times n$ matrix all of whose entries equal $0$.

The reader should also verify that the list of distinct $m\times n$ matrices that have $0$ in all entries except for a $1$ in one entry is a basis of $\mathcal{M}_{m\times n}(\FF)$. There are $mn$ such matrices, so the dimension of $\mathcal{M}_{m\times n}(\FF)$ equals $mn$.
\end{proof}

\subsection{Matrix Multiplication}
\begin{definition}[Matrix multiplication]
Suppose $A=(a_{ij})_{m\times n}$, $B=(b_{jk})_{n\times p}$. Then $AB=(c_{ik})_{m\times p}$ has entries defined by
\[c_{ik}=\sum_{j=1}^{n}a_{ij}b_{jk}.\]
\end{definition}

\begin{remark}
Thus the entry in row $j$, column $k$ of $AB$ is computed by taking row $j$ of $A$ and column $k$ of $B$, multiplying together corresponding entries, and then summing.
\end{remark}

\begin{remark}
Note that we define the product of two matrices only when the number of columns of the first matrix equals the number of rows of the second matrix.
\end{remark}

In the next result, we assume that the same basis of $V$ is used in considering $T\in\mathcal{L}(U,V)$ and $S\in\mathcal{L}(V,W)$, the same basis of $W$ is used in considering $S\in\mathcal{L}(V,W)$ and $ST\in\mathcal{L}(U,W)$, and the same basis of $U$ is used in considering $T\in\mathcal{L}(U,V)$ and $ST\in\mathcal{L}(U,W)$.

\begin{proposition}[Matrix of product of linear maps]
If $T\in\mathcal{L}(U,V)$ and $S\in\mathcal{L}(V,W)$, then $\mathcal{M}(ST)=\mathcal{M}(S)\mathcal{M}(T)$.
\end{proposition}

\begin{proof}
Suppose $\{v_1,\dots,v_n\}$ is a basis of $V$, $\{w_1,\dots,w_m\}$ is a basis of $W$, $\{u_1,\dots,u_p\}$ is a basis of $U$.

Let $\mathcal{M}(S)=(a_{ij})_{m\times n}$, $\mathcal{M}(T)=(b_{jk})_{n\times p}$, where
\begin{align*}
Sv_j&=\sum_{i=1}^{m}a_{ij}w_i\\
Tu_k&=\sum_{j=1}^{n}b_{jk}v_j.
\end{align*}

For $k=1,\dots,p$, we have
\begin{align*}
(ST)u_k&=S(Tu_k)\\
&=S\brac{\sum_{j=1}^{n}b_{jk}v_j}\\
&=\sum_{j=1}^{n}b_{jk}Sv_j\\
&=\sum_{j=1}^{n}b_{jk}\brac{\sum_{i=1}^{m}a_{ij}w_i}\\
&=\sum_{i=1}^{m}\brac{\sum_{j=1}^{n}a_{ij}b_{jk}}w_i.
\end{align*}
\end{proof}

\begin{notation}
$A_{i,\cdot}$ denotes the row vector corresponding to the $i$-th row of $A$; $A_{\cdot,j}$ denotes the column vector corresponding to the $j$-th column of $A$.
\end{notation}

\begin{proposition}
Suppose $A=(a_{ij})_{m\times n}$, $B=(b_{jk})_{n\times p}$. Let $AB=(c_{ik})_{m\times p}$. Then
\[c_{ik}=A_{i,\cdot}B_{\cdot,k}\]
That is, the entry in row $i$, column $k$ of $AB$ equals (row $i$ of $A$) times (column $k$ of $B$).
\end{proposition}

\begin{proof}
By definition,
\begin{align*}
A_{i,\cdot}B_{\cdot,k}
&=\begin{pmatrix}
a_{i1}&\cdots&a_{in}
\end{pmatrix}
\begin{pmatrix}
b_{1k}\\\vdots\\b_{nk}
\end{pmatrix}\\
&=a_{i1}b_{1k}+\cdots+a_{in}b_{nk}\\
&=\sum_{j=1}^{n}a_{ij}b_{jk}\\
&=c_{ik}.
\end{align*}
\end{proof}

\begin{proposition}
Suppose $A=(a_{ij})_{m\times n}$, $B=(b_{jk})_{n\times p}$. Then
\[(AB)_{\cdot,k}=AB_{\cdot,k}\]
That is, column $k$ of $AB$ equals $A$ times column $k$ of $B$.
\end{proposition}

\begin{proof}
Using the previous result,
\[AB_{\cdot,k}=\begin{pmatrix}
A_{1,\cdot}B_{\cdot,k}\\
\vdots\\
A_{n,\cdot}B_{\cdot,k}
\end{pmatrix}=\begin{pmatrix}
c_{1k}\\\vdots\\c_{nk}
\end{pmatrix}=(AB)_{\cdot,k}\]
\end{proof}

\begin{proposition}[Linear combination of columns]
Suppose $A=(a_{ij})_{m\times n}$, $b=\begin{pmatrix}
b_1\\\vdots\\b_n
\end{pmatrix}$. Then
\[Ab=b_1A_{\cdot,1}+\cdots+b_nA_{\cdot,n}.\]
That is, $Ab$ is a linear combination of the columns of $A$, with the scalars that multiply the columns coming from $b$.
\end{proposition}

\begin{proof}
\begin{align*}
Ab&=\begin{pmatrix}
a_{11}b_1+\cdots+a_{1n}b_n\\
\vdots\\
a_{m1}b_1+\cdots+a_{mn}b_n
\end{pmatrix}\\
&=\begin{pmatrix}
a_{11}b_1\\\vdots\\a_{m1}b_1
\end{pmatrix}+\cdots+\begin{pmatrix}
a_{1n}b_n\\\vdots\\a_{mn}b_n
\end{pmatrix}\\
&=b_1\begin{pmatrix}
a_{11}\\\vdots\\a_{m1}
\end{pmatrix}+\cdots+b_n\begin{pmatrix}
a_{1n}\\\vdots\\a_{mn}
\end{pmatrix}\\
&=b_1A_{\cdot,1}+\cdots+b_nA_{\cdot,n}.
\end{align*}
\end{proof}

The following result states that matrix multiplication can be expressed as linear combinations of columns or rows.

\begin{proposition}
Suppose $C=(c_{ij})_{m\times c}$, $R=(r_{jk})_{c\times n}$. Then
\begin{enumerate}[label=(\roman*)]
\item Columns: for $k=1,\dots,n$, $(CR)_{\cdot,k}$ is a linear combination of $C_{\cdot,1},\dots,C_{\cdot,c}$, with coefficients coming from $R_{\cdot,k}$.
\item Rows: for $i=1,\dots,m$, $(CR)_{i,\cdot}$ is a linear combination of $R_{1,\cdot},\dots,R_{c,\cdot}$, with coefficients coming from $C_{j,\cdot}$.
\end{enumerate}
\end{proposition}

\begin{proof} \
\begin{enumerate}[label=(\roman*)]
\item 
\item 
\end{enumerate}
\end{proof}

\subsection{Rank of a Matrix}
\begin{definition}
Suppose $A\in\mathcal{M}_{m\times n}(\FF)$. Then the \emph{row space}\index{rank!row space} of $A$ is the span of its rows, and the \emph{column space}\index{rank!column space} of $A$ is the span of its columns:
\begin{align*}
\Row(A)&\coloneqq\spn\brac{A_{i,\cdot}\mid 1\le i\le m},\\
\Col(A)&\coloneqq\spn\brac{A_{\cdot,j}\mid 1\le j\le n}.
\end{align*}
The \vocab{row rank}\index{rank!row rank} and \vocab{column rank}\index{rank!column rank} of $A$ are defined as
\begin{align*}
r(A)&\coloneqq\dim\Row(A),\\
c(A)&\coloneqq\dim\Col(A).
\end{align*}
\end{definition}

\begin{definition}[Transpose]
Suppose $A=(a_{ij})_{m\times n}$. Then the \vocab{transpose}\index{matrix!transpose} of $A$ is the matrix $A^T=(b_{ij})_{n\times m}$, whose entries are defined by
\[b_{ij}=a_{ji}.\]
\end{definition}

\begin{proposition}[Properties of transpose]
Suppose $A,B\in\mathcal{M}_{m\times n}(\FF)$, $C\in\mathcal{M}_{n\times p}(\FF)$. Then
\begin{enumerate}[label=(\roman*)]
\item $(A+B)^T=A^T+B^T$;
\item $(\lambda A)^T=\lambda A^T$ for $\lambda\in\FF$;
\item $(AC)^T=C^TA^T$.
\end{enumerate}
\end{proposition}

\begin{lemma}[Column-row factorisation]\label{lemma:column-row-factorisation}
Suppose $A\in\mathcal{M}_{m\times n}(\FF)$, $c(A)\ge1$. Then there exist  $C\in M_{m\times c(A)}(\FF)$, $R\in M_{c(A)\times n}(\FF)$ such that $A=CR$.
\end{lemma}

\begin{proof}
We prove by construction, i.e. construct the required matrices $C$ and $R$.

Each column of $A$ is a $m\times1$ matrix. The set of columns of $A$
\[\{A_{\cdot,1},\dots,A_{\cdot,n}\}\] 
can be reduced to a basis of $\Col(A)$, which has length $c(A)$, by the definition of column rank. The $c(A)$ columns in this basis can be put together to form a $m\times c(A)$ matrix, which we call $C$.

For $k=1,\dots,n$, the $k$-th column of $A$ is a linear combination of the columns of $C$. Make the coefficients of this linear combination into column $k$ of a $c\times n$ matrix, which we call $R$. By , it follows that $A=CR$.
\end{proof}

\begin{lemma}[Column rank equals row rank]
The column rank of a matrix equals to its row rank.
\end{lemma}

\begin{proof}
Suppose $A\in\mathcal{M}_{m\times n}(\FF)$. Let $A=CR$ be the column-row factorisation of $A$, where $C\in\mathcal{M}_{m\times c(A)}(\FF)$, $R\in\mathcal{M}_{c(A)\times n}(\FF)$.


\end{proof}

Since column rank equals row rank, we can dispense with the terms ``column rank'' and ``row rank'', and just use the simpler
term ``rank''.

\begin{definition}[Rank]
The \vocab{rank}\index{rank} of a matrix $A$ is defined as
\[\rank A\coloneqq r(A)=c(A).\]
\end{definition}
\pagebreak

\section{Invertibility and Isomorphism}
\subsection{Invertibility}
\begin{notation}
$I_V\in\mathcal{L}(V)$ denotes the identity map on $V$:
\[Iv=v\quad(\forall v\in V)\]
The subscript is omitted if there is no ambiguity.
\end{notation}

\begin{definition}[Invertibility]
$T\in\mathcal{L}(V,W)$ is \vocab{invertible}\index{invertibility} if there exists $S\in\mathcal{L}(W,V)$ such that $ST=I_V$, $TS=I_W$; $S$ is known as an \emph{inverse} of $T$.
\end{definition}

\begin{proposition}[Uniqueness of inverse]
The inverse of an invertible linear map is unique.
\end{proposition}

\begin{proof}
Suppose $T\in\mathcal{L}(V,W)$ is invertible, $S_1,S_2\in\mathcal{L}(W,V)$ are inverses of $T$. Then
\[S_1=S_1I_W=S_1(TS_2)=(S_1T)S_2=I_VS_2=S_2.\]
Thus $S_1=S_2$.
\end{proof}

Now that we know that the inverse is unique, we can give it a notation.

\begin{notation}
If $T$ is invertible, then its inverse is denoted by $T^{-1}$.
\end{notation}

The following result is useful in determing if a linear map is invertible.

\begin{lemma}[Invertibility criterion]\label{lemma:invertibility-criterion}
Suppose $T\in\mathcal{L}(V,W)$.
\begin{enumerate}[label=(\roman*)]
\item $T$ is invertible $\iff$ $T$ is injective and surjective.
\item If $\dim V=\dim W$, $T$ is invertible $\iff$ $T$ is injective $\iff$ $T$ is surjective.
\end{enumerate}
\end{lemma}

\begin{proof} \
\begin{enumerate}[label=(\roman*)]
\item \fbox{$\implies$} Suppose $T\in\mathcal{L}(V,W)$ is invertible, which has inverse $T^{-1}$. Suppose $Tu=Tv$. Applying $T^{-1}$ to both sides of the equation gives
\[u=T^{-1}Tu=T^{-1}Tv=v\]
so $T$ is injective.

We now show $T$ is surjective. Let $w\in W$. Then $w=T\brac{T^{-1}w}$, which shows that $w\in\im T$, so $\im T=W$. Hence $T$ is surjective.

\fbox{$\impliedby$} Suppose $T$ is injective and surjective.

Define $S\in\mathcal{L}(W,V)$ such that for each $w\in W$, $S(w)$ is the unique element of $V$ such that $T(S(w))=w$ (we can do this due to injectivity and surjectivity). Then we have that $T(ST)v=(TS)Tv=Tv$ and thus $STv=v$ so $ST=I$. It is easy to show that $S$ is a linear map.

\item It suffices to only prove $T\text{ is injective}\iff T\text{ is surjective}$. Then apply the previous result.

\fbox{$\implies$} Suppose $T$ is injective. Then $\dim\ker T=0$. By the fundamental theorem of linear maps, 
\begin{align*}
\dim\im T&=\dim V-\dim\ker T\\
&=\dim V\\
&=\dim W
\end{align*}
which implies that $T$ is surjective.

\fbox{$\impliedby$} Suppose $T$ is surjective, then $\dim\im T=\dim W$. By the fundamental theorem of linear maps,
\begin{align*}
\dim\ker T&=\dim V-\dim\im T\\
&=\dim V-\dim W\\
&=0
\end{align*}
which implies that $T$ is injective.
\end{enumerate}
\end{proof}

\begin{corollary}
Suppose $V$ and $W$ are finite-dimensional, $\dim V=\dim W$, $S\in\mathcal{L}(W,V)$, $T=\mathcal{L}(V,W)$. Then $ST=I$ if and only if $TS=I$.
\end{corollary}

\begin{proof} \

\fbox{$\implies$} Suppose $ST=I$. Let $v\in\ker T$. Then
\[v=Iv=(ST)v=S(Tv)=S(\vb{0})=\vb{0}\implies \ker T=\{\vb{0}\}\]
so $T$ is injective. Since $\dim V=\dim W$, by the previous result, $T$ is invertible.

Since $ST=I$, then
\[S=STT^{-1}=IT^{-1}=T^{-1}\]
so $TS=TT^{-1}=I$, as desired.

\fbox{$\impliedby$} Similar to above; reverse the roles of $S$ and $T$ (and $V$ and $W$) to show that if $TS=I$ then $ST=I$.
\end{proof}

\subsection{Isomorphism}
\begin{definition}[Isomorphism]
An \vocab{isomorphism}\index{isomorphism} is an invertible linear map. $V$ and $W$ are \vocab{isomorphic}, denoted by $V\cong W$, if there exists an isomorphism $T\in\mathcal{L}(V,W)$.
\end{definition}

The following result shows that we need to look at only at the dimension to determine whether two vector spaces are isomorphic.

\begin{lemma}
Suppose $V$ and $W$ are finite-dimensional. Then
\[V\cong W\iff\dim V=\dim W.\]
\end{lemma}

\begin{proof} \

\fbox{$\implies$} Suppose $V\cong W$, then there exists an isomorphism $T\in\mathcal{L}(V,W)$, which is invertible, so $T$ is both injective and surjective, thus $\ker T=\{\vb{0}\}$ and $\im T=W$, implying $\dim\ker T=0$ and $\dim\im T=\dim W$.

By the fundamental theorem of linear maps,
\begin{align*}
\dim V&=\dim\ker T+\dim\im T\\
&=0+\dim W=\dim W.
\end{align*}

\fbox{$\impliedby$} Suppose $V$ and $W$ are finite-dimensional, $\dim V=\dim W=n$. Let $\{v_1,\dots,v_n\}$ be a basis of $V$, $\{w_1,\dots,w_n\}$ be a basis of $W$. 

It suffices to construct an surjective $T\in\mathcal{L}(V,W)$. By the linear map lemma, there exists a linear map $T\in\mathcal{L}(V,W)$ such that
\[Tv_i=w_i\quad(i=1,\dots,n)\]
Let $w\in W$. Then there exist $a_i\in\FF$ such that $w=a_1w_1+\cdots+a_nw_n$. Then
\begin{align*}
T(a_1v_1+\cdots+a_nv_n)=w&\implies w\in\im T\\
&\implies W=\im T\\
&\implies T\text{ is surjective}\\
&\implies T\text{ is invertible.}
\end{align*}
\end{proof}

\begin{proposition}
Suppose $\{v_1,\dots,v_n\}$ is a basis of $V$, $\{w_1,\dots,w_m\}$ is a basis of $W$. Then
\[\mathcal{L}(V,W)\cong\mathcal{M}_{m\times n}(\FF).\]
\end{proposition}

\begin{proof}
We claim that $\mathcal{M}$ is an isomorphism between $\mathcal{L}(V,W)$ and $\mathcal{M}_{m\times n}(\FF)$.

We already noted that $\mathcal{M}$ is linear. We need to prove that $\mathcal{M}$ is (i) injective and (ii) surjective.
\begin{enumerate}[label=(\roman*)]
\item Given $T\in\mathcal{L}(V,W)$, if $\mathcal{M}(T)=0$, then
\[Tv_j=0\quad(j=1,\dots,n)\]
Since $v_1,\dots,v_n$ is a basis of $V$, this implies $T=\vb{0}$, so $\ker\mathcal{M}=\{\vb{0}\}$. Thus $\mathcal{M}$ is injective.

\item Suppose $A\in\mathcal{M}_{m\times n}(\FF)$. By the linear map lemma, there exists $T\in\mathcal{L}(V,W)$ such that
\[Tv_j=\sum_{i=1}^{m}a_{ij}w_i\quad(j=1,\dots,n)\]
Since $\mathcal{M}(T)=A$, $\im\mathcal{M}=\mathcal{M}_{m\times n}(\FF)$ so $\mathcal{M}$ is surjective.
\end{enumerate}
\end{proof}

\begin{corollary}\label{cor:dimension-product}
Suppose $V$ and $W$ are finite-dimensional. Then $\mathcal{L}(V,W)$ is finite-dimensional and
\[\dim\mathcal{L}(V,W)=(\dim V)(\dim W).\]
\end{corollary}

\begin{proof}
Since $\mathcal{L}(V,W)\cong\mathcal{M}_{m\times n}(\FF)$,
\[\dim\mathcal{L}(V,W)=\dim\mathcal{M}_{m\times n}(\FF)=mn=(\dim V)(\dim W).\]
\end{proof}

\subsection{Linear Maps Thought of as Matrix Multiplication}
Previously we defned the matrix of a linear map. Now we defne the matrix of a vector.

\begin{definition}[Matrix of a vector]
Suppose $v\in V$, $\{v_1,\dots,v_n\}$ is a basis of $V$. The matrix of $v$\index{matrix of vector} with respect to this basis is
\[\mathcal{M}(v)=\begin{pmatrix}
b_1\\\vdots\\b_n
\end{pmatrix}\]
where $b_1,\dots,b_n\in\FF$ are such that
\[v=b_1v_1+\cdots+b_nv_n.\]
\end{definition}

\begin{example}
If $x=(x_1,\dots,x_n)\in\FF^n$, then the matrix of the vector $x$ with respect to the standard basis is
\[\mathcal{M}(x)=\begin{pmatrix}
x_1\\\vdots\\x_n
\end{pmatrix}.\]
\end{example}

\begin{proposition}
Suppose $T\in\mathcal{L}(V,W)$. Let $\{v_1,\dots,v_n\}$ be a basis of $V$, $\{w_1,\dots,w_m\}$ be a basis of $W$. Then
\[\mathcal{M}(T)_{\cdot,j}=\mathcal{M}(Tv_j)\quad(j=1,\dots,n)\]
\end{proposition}

\begin{proof}
By definition, the entries of $\mathcal{M}(T)$ are defined such that
\[Tv_j=\sum_{i=1}^{m}a_{ij}w_i\quad(j=1,\dots,n)\]
Then since $Tv_j\in W$, by definition, the matrix of $Tv_j$ with respect to the basis $\{w_1,\dots,w_m\}$ is
\[\mathcal{M}(Tv_j)=\begin{pmatrix}
a_1j\\\vdots\\a_{mj}
\end{pmatrix}\]
which is precisely the $j$-th column of $\mathcal{M}(T)_{\cdot,j}$, for $j=1,\dots,n$.
\end{proof}

The following result shows that linear maps act like matrix multiplication.

\begin{proposition}
Suppose $T\in\mathcal{L}(V,W)$. Let $\{v_1,\dots,v_n\}$ be a basis of $V$, $\{w_1,\dots,w_m\}$ be a basis of $W$. Let $v\in V$, then
\[\mathcal{M}(Tv)=\mathcal{M}(T)\mathcal{M}(v).\]
\end{proposition}

\begin{proof}
Suppose $v=b_1v_1+\cdots+b_nv_n$ for some $b_1,\dots,b_n\in\FF$. Then
\begin{align*}
\mathcal{M}(Tv)&=\mathcal{M}\brac{T(b_1v_1+\cdots+b_nv_n)}\\
&=b_1\mathcal{M}(Tv_1)+\cdots+b_n\mathcal{M}(Tv_n)\\
&=b_1\mathcal{M}(T)_{\cdot,1}+\cdots+b_n\mathcal{M}(T)_{\cdot,n}\\
&=\begin{pmatrix}
\mathcal{M}(T)_{\cdot,1}&\cdots&\mathcal{M}(T)_{\cdot,n}
\end{pmatrix}\begin{pmatrix}
b_1\\\vdots\\b_n
\end{pmatrix}\\
&=\mathcal{M}(T)\mathcal{M}(v).
\end{align*}
\end{proof}

Notice that no bases are in sight in the statement of the next result. Although $\mathcal{M}(T)$ in the next result depends on a choice of bases of $V$ and $W$, the next result shows that the column rank of $\mathcal{M}(T)$ is the same for all such choices (because $\im T$ does not depend on a choice of basis).

\begin{proposition}
Suppose $V$ and $W$ are finite-dimensional, $T\in\mathcal{L}(V,W)$. Then
\[\dim\ker T=\rank\mathcal{M}(T).\]
\end{proposition}

\begin{proof}
Suppose $\{v_1,\dots,v_n\}$ is a basis of $V$, $\{w_1,\dots,w_m\}$ is a basis of $W$. 

The linear map that takes $w\in W$ to $\mathcal{M}(w)$ is an isomorphism from $W$ to $\mathcal{M}_{m\times1}(\FF)$ (consisting of $m\times1$ column vectors).

The restriction of this isomorphism to $\im T$ [which equals $\spn(Tv_1,\dots,Tv_n)$] is an isomorphism from $\im T$ to $\spn(\mathcal{M}(Tv_1),\dots,\mathcal{M}(Tv_n))$. For $j=1,\dots,n$, the $m\times1$ matrix $\mathcal{M}(Tv_j)$ equals column $k$ of $\mathcal{M}(T)$. Thus
\[\dim\ker T=\rank\mathcal{M}(T),\]
as desired.
\end{proof}

\subsection{Change of Basis}
\begin{definition}[Identity matrix]
For $n\in\NN$, the $n\times n$ \vocab{identity matrix}\index{matrix!identity matrix} is
\[I_n=\begin{pmatrix}
1&&0\\
&\ddots&\\
0&&1
\end{pmatrix}.\]
\end{definition}

\begin{remark}
Note that the symbol $I$ is used to denote both the identity operator and the identity matrix. The context indicates which meaning of $I$ is intended. For example, consider the equation $\mathcal{M}(I)=I$; on LHS $I$ denotes the identity operator, and on RHS $I$ denotes the identity matrix.
\end{remark}

\begin{proposition}
Suppose $A\in\mathcal{M}_{n\times n}(\FF)$. Then $AI_n=I_nA=A$.
\end{proposition}

\begin{proof}
Exercise.
\end{proof}

\begin{definition}[Invertible matrix]
$A\in\mathcal{M}_{n\times n}(\FF)$ is called \emph{invertible} if there exists $B\in\mathcal{M}_{n\times n}(\FF)$ such that $AB=BA=I$; we call $B$ an \emph{inverse} of $A$
\end{definition}

\begin{proposition}[Uniqueness of inverse]
Suppose $A$ is an invertible square matrix. Then there exists a unique matrix $B$ such that $AB=BA=I$.
\end{proposition}

\begin{proof}
Suppose otherwise, for a contradiction, that $A$ does not have a unique inverse. Let $B$ and $C$ be inverses of $A$; that is,
\begin{align*}
AB&=BA=I,\\
AC&=CA=I.
\end{align*}
Then 
\[B=BI=BAC=IC=C.\]
\end{proof}

Since the inverse of a matrix is unique, we can give it a notation.

\begin{notation}
The inverse of a matrix $A$ is denoted by $A^{-1}$.
\end{notation}

\begin{proposition} \
\begin{enumerate}[label=(\roman*)]
\item Suppose $A$ is an invertible square matrix. Then $(A^{-1})^{-1}=A$.
\item Suppose $A$ and $C$ are invertible square matrices of the same size. Then $AC$ is invertible, and $(AC)^{-1}=C^{-1}A^{-1}$.
\end{enumerate}
\end{proposition}

\begin{proof} \
\begin{enumerate}[label=(\roman*)]
\item \[A^{-1}A=AA^{-1}=I,\]
so the inverse of $A^{-1}$ is $A$.
\item \begin{align*}
(AC)(C^{-1}A^{-1})
&=A(CC^{-1})A^{-1}\\
&=AIA^{-1}\\
&=AA^{-1}\\
&=I,
\end{align*}
and similarly $(C^{-1}A^{-1})(AC)=I$.
\end{enumerate}
\end{proof}

\begin{proposition}[Matrix of product of linear maps]
Suppose $T\in\mathcal{L}(U,V)$, $S\in\mathcal{L}(V,W)$. Let $\mathcal{U}=\{u_1,\dots,u_m\}$ be a basis of $U$, $\mathcal{V}=\{v_1,\dots,v_n\}$ be a basis of $V$, $\mathcal{W}=\{w_1,\dots,w_p\}$ be a basis of $W$. Then
\[\mathcal{M}\brac{ST;\mathcal{U},\mathcal{W}}=\mathcal{M}\brac{S;\mathcal{V},\mathcal{W}}\mathcal{M}\brac{T;\mathcal{U},\mathcal{V}}.\]
\end{proposition}

\begin{proof}
Refer to previous section. Now we are just being more explicit about the bases involved.
\end{proof}

\begin{corollary}
Suppose that $\mathcal{U}=\{u_1,\dots,u_n\}$ and $\mathcal{V}=\{v_1,\dots,v_n\}$ are bases of $V$. Then the matrices
\[\mathcal{M}\brac{I;\mathcal{U},\mathcal{V}}\quad\text{and}\quad\mathcal{M}\brac{I;\mathcal{V},\mathcal{U}}\]
are invertible, and each is the inverse of the other.
\end{corollary}

\begin{proof}

\end{proof}

\begin{theorem}[Change-of-basis formula]
Suppose $T\in\mathcal{L}(V)$. Let $\mathcal{U}=\{u_1,\dots,u_n\}$ and $\mathcal{V}=\{v_1,\dots,v_n\}$ be bases of $V$. Let
\[A=\mathcal{M}(T;\mathcal{U}),\quad B=\mathcal{M}(T;\mathcal{V}),\]
and $C=\mathcal{M}(I;\mathcal{U},\mathcal{V})$. Then
\begin{equation}
A=C^{-1}BC.
\end{equation}
\end{theorem}

\begin{proof}
Note that
\begin{align*}
\mathcal{M}(T;\mathcal{U},\mathcal{V})
&=\underbrace{\mathcal{M}(T;\mathcal{V})}_{B}\underbrace{\mathcal{M}(I;\mathcal{U},\mathcal{V})}_{C}\\
&=\underbrace{\mathcal{M}(I;\mathcal{U},\mathcal{V})}_{C}\underbrace{\mathcal{M}(T;\mathcal{U})}_{A}
\end{align*}
Hence $BC=CA$, and the desired result follows.
\end{proof}

\begin{proposition}
Suppose $\{v_1,\dots,v_n\}$ is a basis of $V$, $T\in\mathcal{L}(V)$ is invertible. Then
\[\mathcal{M}\brac{T^{-1}}=\brac{\mathcal{M}(T)}^{-1},\]
where both matrices are with respect to the basis $\{v_1,\dots,v_n\}$.
\end{proposition}

\begin{proof}
We have that
\[\mathcal{M}\brac{T^{-1}}\mathcal{M}(T)=\mathcal{M}\brac{T^{-1}T}=\mathcal{M}(I)=I.\]
\end{proof}
\pagebreak

\section{Products and Quotients of Vector Spaces}
\subsection{Products of Vector Spaces}
\begin{definition}[Product]
Suppose $V_1,\dots,V_n$ are vector spaces over $\FF$. The \vocab{product}\index{product of vector spaces} $V_1\times\cdots\times V_n$ is defined by
\[V_1\times\cdots\times V_n\coloneqq\{(v_1,\dots,v_n)\mid v_i\in V_i\}.\]
\end{definition}

\begin{remark}
This is analagous to the Cartesian product of sets.
\end{remark}

\begin{proposition}
$V_1\times\cdots\times V_n$ is a vector space over $\FF$, with addition and scalar multiplication defined by
\begin{align*}
(u_1,\dots,u_n)+(v_1,\dots,v_n)&=(u_1+v_1,\dots,u_n+v_n)\\
\lambda(v_1,\dots,v_n)&=(\lambda v_1,\dots,\lambda v_n)
\end{align*}
\end{proposition}

The following result shows that the dimension of a product is the sum of dimensions.

\begin{proposition}
Suppose $V_1,\dots,V_n$ are finite-dimensional. Then $V_1\times\cdots\times V_n$ is finite-dimensional, and
\[\dim(V_1\times\cdots\times V_n)=\dim V_1+\cdots+\dim V_n.\]
\end{proposition}

\begin{proof}
For each $V_k$ ($k=1,\dots,n$), choose a basis:
\[\mathcal{B}_k=\crbrac{e_{k1},\dots,e_{k\dim V_k}}.\]
For each basis vector of each $V_k$, consider the set consisting of elements of $V_1\times\cdots\times V_n$ that equal the basis vector in the $k$-th slot and $0$ in the other slots:
\[\mathcal{B}=\{(0,\dots,\underbrace{e_{ki}}_\text{$k$-th slot},\dots,0)\mid 1\le i\le\dim V_k,1\le k\le n\}.\]

We want to show that $\mathcal{B}$ is a basis of $V_1\times\cdots\times V_n$. Thus we need to show that it is (i) a spanning set, and (ii) linearly independent.
\begin{enumerate}[label=(\roman*)]
\item Let $(v_1,\dots,v_n)\in V_1\times\cdots\times V_n$. For $k=1,\dots,n$, since $\mathcal{B}_k$ is a basis for $V_k$, we can write
\[v_k=\sum_{i=1}^{\dim V_k}a_{ki}e_{ki}.\]
for some $a_{k1},\dots,a_{k\dim V_k}\in\FF$. Then
\begin{align*}
(v_1,\dots,v_n)
&=\sum_{k=1}^{n}(0,\dots,v_k,\dots,0)\\
&=\sum_{k=1}^{n}\brac{0,\dots,\sum_{i=1}^{\dim V_k}a_{ki}e_{ki},\dots,0}\\
&=\sum_{k=1}^{n}\sum_{i=1}^{\dim V_k}a_{ki}\brac{0,\dots,e_{ki},\dots,0}
\end{align*}
which is a linear combination of vectors in $\mathcal{B}$. Hence $\mathcal{B}$ spans $V_1\times\cdots\times V_n$.

\item Suppose there exist $a_{ki}\in\FF$ such that
\begin{align*}
\sum_{k=1}^{n}\sum_{i=1}^{\dim V_k}a_{ki}\brac{0,\dots,e_{ki},\dots,0}&=\vb{0}\\
\sum_{k=1}^{n}\brac{0,\dots,\sum_{i=1}^{\dim V_k}a_{ki}e_{ki},\dots,0}&=\vb{0}\\
\brac{\sum_{i=1}^{\dim V_1}a_{1i}e_{1i},\sum_{i=1}^{\dim V_2}a_{2i}e_{2i},\dots,\sum_{i=1}^{\dim V_n}a_{ni}e_{ni}}&=\vb{0}
\end{align*}
so for $k=1,\dots,n$,
\[\sum_{i=1}^{\dim V_k}a_{ki}e_{ki}=\vb{0}.\]
By the linear independence of vectors in $\mathcal{B}_k$, we have that
\[a_{k1}=\cdots=a_{k\dim V_k}=0\]
for $k=1,\dots,n$.
\end{enumerate}

Hence
\begin{align*}
\dim(V_1\times\cdots\times V_n)&=|\mathcal{B}|\\
&=|\mathcal{B}_1|+\cdots+|\mathcal{B}_n|\\
&=\dim V_1+\cdots+\dim V_n.
\end{align*}
\end{proof}

Products are also related to direct sums, by the following result.

\begin{proposition}
Suppose that $V_1,\dots,V_n\le V$. Define a linear map 
\begin{align*}
\Gamma:V_1\times\cdots\times V_n&\to V_1+\cdots+V_n\\
(v_1,\dots,v_n)&\mapsto v_1+\cdots+v_n
\end{align*}
Then $V_1+\cdots+V_n$ is a direct sum if and only if $\Gamma$ is injective.
\end{proposition}

\begin{proof} \

\fbox{(i)$\iff$(ii)} Suppose $V_1+\cdots+V_n$ is a direct sum. Let $(v_1,\dots,v_n)\in\ker\Gamma$. Then
\begin{align*}
\Gamma(v_1,\dots,v_n)&=\vb{0}\\
v_1+\cdots+v_n&=\vb{0}\\
v_1=\cdots=v_n&=\vb{0}
\end{align*}
so $(v_1,\dots,v_n)=0$. Hence $\ker\Gamma=0$, thus $\Gamma$ is injective.

\fbox{(ii)$\iff$(i)} Similar to the above proof.
\end{proof}

The next result says that a sum is a direct sum if and only if dimensions add up.

\begin{proposition}
Suppose $V$ is finite-dimensional, $V_1,\dots,V_n\le V$. Then $V_1+\cdots+V_n$ is a direct sum if and only if
\[\dim(V_1+\cdots+V_n)=\dim V_1+\cdots+\dim V_n.\]
\end{proposition}

\begin{proof}
The map $\Gamma$ defined in the previous result is surjective. Thus by the fundamental theorem of linear maps, $\Gamma$ is injective if and only if
\[\dim(V_1+\cdots+V_n)=\dim(V_1\times V_n).\]
Then use the previous two results above.
\end{proof}

\subsection{Quotient Spaces}
\begin{definition}[Coset]
Suppose $v\in V$, $U\subset V$. Then $v+U$ is called a \vocab{coset}\index{coset} of $U$, defined by
\[v+U\coloneqq\{v+u\mid u\in U\}.\]
\end{definition}

\begin{definition}[Quotient space]
Suppose $U\le V$. Then the \vocab{quotient space}\index{quotient space} $V/U$ is the set of cosets of $U$:
\[V/U\coloneqq\{v+U\mid v\in V\}.\]
\end{definition}

\begin{example}
If $U=\{(x,2x)\in\RR^2\mid x\in\RR\}$, then $\RR^2/U$ is the set of lines in $\RR^2$ that have gradient of $2$.
\end{example}

It is obvious that two cosets of a subspace are equal or disjoint. We shall now prove this.

\begin{proposition}
Suppose $U\le V$, and $v,w\in V$. Then
\[v-w\in U\iff v+U=w+U\iff(v+U)\cap(w+U)=\emptyset.\]
\end{proposition}

\begin{proof}
First suppose $v-w\in U$. If $u\in U$, then
\[v+u=w+\brac{(v-w)+u}\in w+U.\]
Thus $v+U\subset w+U$. Similarly, $w+U \subset v+U$. Thus $v+U=w+U$, completing the proof that $v-w\in U$ implies $v+U=w+U$.

The equation $v+U=w+U$ implies that $(v+U)\cap(w+U)\neq\emptyset$.

Now suppose $(v+U)\cap(w+U)\neq\emptyset$. Thus there exist $u_1,u_2\in U$ such that
\[v+u_1=w+u_2.\]
Thus $v-w=u_2-u_1$. Hence $v-w\in U$, showing that $(v+U)\cap(w+U)\neq\emptyset$ implies $v-w\in U$, which completes the proof.
\end{proof}

\begin{proposition}
Suppose $U\le V$. Then $V/U$ is a vector space, with addition and scalar multiplication defined by
\begin{align*}
(v+U)+(w+U)&=(v+w)+U\\
\lambda(v+U)&=(\lambda v)+U
\end{align*}
for all $v,w\in V$, $\lambda\in\FF$.
\end{proposition}

\begin{proof}

\end{proof}

\begin{definition}[Quotient map]
Suppose $U\le V$. The \vocab{quotient map}\index{quotient map} $\pi:V\to V/U$ is the linear map defined by
\[\pi(v)=v+U\]
for all $v\in V$.
\end{definition}

\begin{proposition}[Dimension of quotient space]
Suppose $V$ is finite-dimensional, $U\le V$. Then
\[\dim V/U=\dim V-\dim U.\]
\end{proposition}

\begin{definition}
Suppose $T\in\mathcal{L}(V,W)$. Define $\tilde{T}:V/\ker T\to W$ by
\[\tilde{T}(v+\ker T)=Tv.\]
\end{definition}

\begin{proposition}
Suppose $T\in\mathcal{L}(V,W)$. Then
\begin{enumerate}[label=(\roman*)]
\item $\tilde{T}\circ\pi=T$, where $\pi$ is the quotient map of $V$ onto $V/\ker T$;
\item $\tilde{T}$ is injective;
\item $\im\tilde{T}=\im T$.
\end{enumerate}
\end{proposition}

\begin{theorem}[First isomorphism theorem]
Suppose $T\in\mathcal{L}(V,W)$ is an isomorphism. Then
\begin{equation}
V/\ker T\cong\im T.
\end{equation}
\end{theorem}


\pagebreak

\section{Duality}
\subsection{Dual Space and Dual Map}
Linear maps into the scalar field $\FF$ play a special role in linear algebra, and thus they get a special name.

\begin{definition}[Linear funtional]
A \vocab{linear functional}\index{linear functional} on $V$ is a linear map from $V$ to $\FF$; that is, a linear functional is an element of $\mathcal{L}(V,\FF)$.
\end{definition}

The vector space $\mathcal{L}(V,\FF)$ also gets a special name and special notation.

\begin{definition}[Dual space]
The \vocab{dual space} of $V$ is the vector space of linear functionals on $V$; that is, $V^*\coloneqq\mathcal{L}(V,\FF)$.
\end{definition}

\begin{lemma}
Suppose $V$ is finite-dimensional. Then $V^*$ is also finite-dimensional, and
\[\dim V^*=\dim V.\]
\end{lemma}

\begin{proof}
By , we have
\[\dim V^*\coloneqq\dim\mathcal{L}(V,\FF)=(\dim V)(\dim\FF)=\dim V\]
as desired.
\end{proof}

\begin{definition}[Dual basis]
If $\{v_1,\dots,v_n\}$ is a basis of $V$, then the \vocab{dual basis}\index{dual basis} of $\{v_1,\dots,v_n\}$ is
\[\{\phi_1,\dots,\phi_n\}\subset V^*,\]
where each $\phi_i$ is the linear functional on $V$ such that
\[\phi_i(v_j)=\begin{cases}
1&(i=j)\\
0&(i\neq j)
\end{cases}\]
\end{definition}

The following result states that dual basis gives coefficients for linear combination.

\begin{proposition}
Suppose $\{v_1,\dots,v_n\}$ is a basis of $V$, and $\{\phi_1,\dots,\phi_n\}$ is the dual basis. Then for each $v\in V$,
\[v=\phi_1(v)v_1+\cdots+\phi_n(v)v_n.\]
\end{proposition}

The following result states that the dual basis is a basis of the dual space.

\begin{proposition}
Suppose $V$ is finite-dimensional. Then the dual basis of a basis of $V$ is a basis of $V^*$.
\end{proposition}

\begin{definition}[Dual map]
Suppose $T\in\mathcal{L}(V,W)$. The \vocab{dual map}\index{dual map} of $T$ is the linear map $T^*\in\mathcal{L}(V,W)$ defined for each $\phi\in W^*$ by
\[T^*(\phi)=\phi\circ T.\]
\end{definition}

\begin{proposition}[Algebraic properties of dual map]
Suppose $T\in\mathcal{L}(V,W)$. Then
\begin{enumerate}[label=(\arabic*)]
\item $(S+T)^*=S^*+T^*$ for all $S\in\mathcal{L}(V,W)$
\item $(\lambda T)^*=\lambda T^*$ for all $\lambda\in\FF$
\item $(ST)^*=T^* S^*$ for all $S\in\mathcal{L}(V,W)$
\end{enumerate}
\end{proposition}

\subsection{Kernel and Image of Dual of Linear Map}
The goal of this section is to describe $\ker T^*$ and $\im T^*$ in terms of $\im T$ and $\ker T$.

\begin{definition}[Annihilator]
For $U\subset V$, the \vocab{annihilator}\index{annihilator} of $U$ is defined by
\[U^\circ\coloneqq\{\phi\in V^*\mid\phi(u)=0,\forall u\in U\}.\]
\end{definition}

\begin{proposition}
$U^\circ\le V$.
\end{proposition}

\begin{proposition}[Dimension of annihilator]
Suppose $V$ is finite-dimensional, $U\le V$. Then
\[\dim U^\circ=\dim V-\dim U.\]
\end{proposition}

The following are conditions for the annihilator to equal $\{\vb{0}\}$ or the whole space.

\begin{proposition}
Suppose $V$ is finite-dimensional, $U\le V$. Then
\begin{enumerate}[label=(\roman*)]
\item $U^\circ=\{\vb{0}\}\iff U=V$;
\item $U^\circ=V^*\iff U=\{\vb{0}\}$.
\end{enumerate}
\end{proposition}

The following result concerns $\ker T^*$.

\begin{proposition}
Suppose $V$ and $W$ are finite-dimensional, $T\in\mathcal{L}(V,W)$. Then
\begin{enumerate}[label=(\roman*)]
\item $\ker T^*=(\im T)^\circ$;
\item $\dim\ker T^*=\dim\ker T+\dim W-\dim V$.
\end{enumerate}
\end{proposition}

The next result can be useful because sometimes it is easier to verify that $T^*$ is injective than to show directly that $T$ is surjective.

\begin{proposition}
Suppose $V$ and $W$ are finite-dimensional, $T\in\mathcal{L}(V,W)$. Then
\[T\text{ is surjective}\iff T^*\text{ is injective.}\]
\end{proposition}

The following result concerns $\im T^*$.

\begin{proposition}
Suppose $V$ and $W$ finite-dimensional, $T\in\mathcal{L}(V,W)$. Then
\begin{enumerate}[label=(\roman*)]
\item $\dim\im T^*=\dim\im T$;
\item $\dim T^*=(\ker T)^\circ$.
\end{enumerate}
\end{proposition}

\begin{proposition}
Suppose $V$ and $W$ are finite-dimensional, $T\in\mathcal{L}(V,W)$. Then
\[T\text{ is injective}\iff T^*\text{ is surjective.}\]
\end{proposition}

\subsection{Matrix of Dual of Linear Map}
\begin{proposition}
Suppose $V$ and $W$ are finite-dimensional, $T\in\mathcal{L}(V,W)$. Then
\[\mathcal{M}(T^*)=\brac{\mathcal{M}(T)}^t.\]
\end{proposition}


\pagebreak

\section*{Exercises}
\begin{prbm}[\cite{axler} 3A]
Suppose $b,c\in\RR$. Define $T:\RR^3\to\RR^2$ by
\[T(x,y,z)=(2x-4y+3z+b,6x+cxyz).\]
Show that $T$ is linear if and only if $b=c=0$.
\end{prbm}

\begin{prbm}[\cite{axler} 3A Q11]
Suppose $V$ is finite-dimensional, $T\in\mathcal{L}(V)$. Prove that $T$ is a scalar multiple of the identity if and only if $ST=TS$ for all $S\in\mathcal{L}(V)$.
\end{prbm}

\begin{prbm}[\cite{axler} 3B Q9]
Suppose $T\in\mathcal{L}(V,W)$ is injective, $\{v_1,\dots,v_n\}$ is linearly independent in $V$. Prove that $\{Tv_1,\dots,Tv_n\}$ is linearly independent in $W$.
\end{prbm}

\begin{solution}
Suppose there exist $a_i\in\FF$ such that
\begin{align*}
&a_1Tv_1+\cdots+a_nTv_n=\vb{0}\\
\implies&T(a_1v_1+\cdots+a_nv_n)=0\\
\implies&a_1v_1+\cdots+a_nv_n\in\ker T
\end{align*}

Since $T$ is injective,
\[\ker T=\{\vb{0}\}\implies a_1v_1+\cdots+a_nv_n=\vb{0}\implies a_1=\cdots=a_n=0\]
since $\{v_1,\dots,v_n\}$ is linearly independent.
\end{solution}

\begin{prbm}[\cite{axler} 3B Q11]
Suppose that $V$ is finite-dimensional, $T\in\mathcal{L}(V,W)$. Prove that there exists $U\le V$ such that
\[U\cap\ker T=\{\vb{0}\}\quad\text{and}\quad\im T=T(U).\]
\end{prbm}

\begin{solution}

\end{solution}

\begin{prbm}[\cite{axler} 3B Q19]
Suppose $W$ is finite-dimensional, $T\in\mathcal{L}(V,W)$. Prove that $T$ is injective if and only if there exists $S\in\mathcal{L}(W,V)$ such that $ST$ is the identity operator on $V$.
\end{prbm}

\begin{solution}

\end{solution}

\begin{prbm}[\cite{axler} 3B Q20]
Suppose $W$ is finite-dimensional, $T\in\mathcal{L}(V,W)$. Prove that $T$ is surjective if and only if there exists $S\in\mathcal{L}(W,V)$ such that $TS$ is the identity operator on $W$.
\end{prbm}

\begin{prbm}[\cite{axler} 3B 22]
Suppose $U,V$ are finite-dimensional, $S\in\mathcal{L}(V,W)$, $T\in\mathcal{L}(U,V)$. Prove that
\[\dim\ker ST\le\dim\ker S+\dim\ker T.\]
\end{prbm}

\begin{solution}

\end{solution}

\begin{prbm}[\cite{axler} 3D]
Suppose $T\in\mathcal{L}(V,W)$ is invertible. Show that $T^{-1}$ is invertible and
\[\brac{T^{-1}}^{-1}=T.\]
\end{prbm}

\begin{solution}
$T^{-1}$ is invertible because there exists $T$ such that $TT^{-1}=T^{-1}T=I$. So
\[T^{-1}T=TT^{-1}=I\]
thus $\brac{T^{-1}}^{-1}=T$.
\end{solution}

3D Q11,12,17,22,23,24

\begin{prbm}[\cite{axler} 3D]
Suppose $T\in\mathcal{L}(U,V)$ and $S\in\mathcal{L}(V,W)$ are both invertible linear maps. Prove that $ST\in\mathcal{L}(U,W)$ is invertible and that $(ST)^{-1}=T^{-1}S^{-1}$.
\end{prbm}

\begin{solution}
\[(ST)(T^{-1}S^{-1})=S(TT^{-1})S^{-1}=I=T^{-1}S^{-1}ST.\]
\end{solution}

\begin{prbm}[\cite{axler} 3D]
Suppose $V$ is finite-dimensional and $T\in\mathcal{L}(V,W)$. Prove that the following are equivalent:
\begin{enumerate}[label=(\roman*)]
\item $T$ is invertible;
\item $\{Tv_1,\dots,Tv_n\}$ is a basis of $V$ for every basis $\{v_1,\dots,v_n\}$ of $V$;
\item $\{Tv_1,\dots,Tv_n\}$ is a basis of $V$ for some basis $\{v_1,\dots,v_n\}$ of $V$.
\end{enumerate}
\end{prbm}

\begin{solution} \

(i)$\implies$(ii) It only suffices to prove linear independence. We can show this
\[a_1Tv_1+\cdots+a_nTv_n=0\iff a_1v_1+\cdots+a_nv_n=0\]
since $T$ is injective and thus the only solution is all $a_i$ are identically zero.

(ii)$\implies$(iii) Trivial.

(iii)$\implies$(i) By the linear map lemma, there exists $S\in\mathcal{L}(V)$ such that $S(Tv_i)=v_i$ for all $i$. Such $S$ is the inverse of $T$ (one can verify) and thus $T$ is invertible.
\end{solution}

\begin{prbm}[\cite{axler} 3E]
Suppose $U\le V$, $V/U$ is finite-dimensional. Prove that $V\cong U\times(V/U)$.
\end{prbm}

\begin{solution}
\[\dim V=\dim U+(\dim V-\dim U)=\dim U+\dim(V/U).\]
\end{solution}
    \chapter{Polynomials}
\section{Definitions}
\begin{definition}[Polynomial]
$p:\FF\to\FF$ is a \vocab{polynomial}\index{polynomial} with coefficients in $\FF$ if there exist $a_i\in\FF$ such that
\[p(z)=a_0+a_1z+\cdots+a_nz^n\quad(z\in\FF)\]
\end{definition}

\begin{notation}
The set of polynomials with coefficients in $\FF$ is denoted by $\FF[z]$.
\end{notation}

\begin{proposition}
With the usual operations of addition and scalar multiplication, $\FF[z]$ is a vector space over $\FF$, as you should verify. Hence $\FF[z]$ is a subspace of $\FF^\FF$ (vector space of functions from $\FF$ to $\FF$).
\end{proposition}

\begin{definition}[Degree]
A polynomial $p\in\FF[z]$ is has \vocab{degree}\index{polynomial!degree} $n$, denoted by $\deg p=n$, if there exist scalars $a_0,a_1,\dots,a_n\in\FF$ with $a_n\neq0$ such that $p(z)=a_0+a_1z+\cdots+a_nz^n$ for all $z\in\FF$.
\end{definition}

\begin{notation}
For non-negative integer $n$, $\FF_n[z]$ denotes the set of polynomials with coefficients in $\FF$ and degree at most $n$.
\end{notation}

\begin{proposition}
For non-negative integer $n$, $\FF_n[z]$ is finite-dimensional.
\end{proposition}

\begin{proof}
$\FF_n[z]=\spn(1,z,z^2,\dots,z^n)$ [here we slightly abuse notation by letting $z^k$ denote a function]. 
\end{proof}

\begin{proposition}
$\FF[z]$ is infinite-dimensional.
\end{proposition}

\begin{proof}
Consider any list of elements of $\FF[z]$. Let $n$ denote the highest degree of the polynomials in this list. Then every polynomial in the span of this list has degree at most $n$. Thus $z^{n+1}$ is not in the span of our list. Hence no list spans $\FF[z]$. Thus $\FF[z]$ is infnite-dimensional.
\end{proof}
\pagebreak

\section{Zeros of Polynomials}
\begin{definition}[Zero of polynomial]
$\lambda\in\FF$ is called a \vocab{zero}\index{polynomial!zero} (or \emph{root}) of a polynomial $p\in\FF[z]$ if
\[p(\lambda)=0.\]
\end{definition}

The next result is the key tool that we will use to show that the degree of a polynomial is unique.

\begin{lemma}[Factor theorem]\label{lemma:factor-thm}
Suppose $n\in\ZZ^+$, $p\in\FF_n[z]$. Suppose $\lambda\in\FF$, then $p(\lambda)=0$ if and only if there exists $q\in\FF_{n-1}[z]$ such that
\[p(z)=(z-\lambda)q(z)\quad(\forall z\in\FF)\]
\end{lemma}

\begin{proof} \

\fbox{$\implies$} Suppose $p(\lambda)=0$. Let $a_0,a_1,\dots,a_n\in\FF$ be such that
\[p(z)=a_nz^n+\cdots+a_1z+a_0\quad(\forall z\in\FF)\]
Then for all $z\in\FF$,
\begin{align*}
p(z)&=p(z)-p(\lambda)\\
&=\brac{a_nz^n+\cdots+a_1z+a_0}-\brac{a_n\lambda^n+\cdots+a_1\lambda+a_0}\\
&=a_n\brac{z^n-\lambda^n}+\cdots+a_1(z-\lambda).
\end{align*}
Note that for each $k=1,\dots,n$, we can factorise
\[z^k-\lambda^k=(z-\lambda)\brac{z^{k-1}+z^{k-2}\lambda+\cdots+\lambda^{k-1}}.\]
Thus $p$ equals $z-\lambda$ times some polynomial of degree $n-1$, as desired.

\fbox{$\impliedby$} Now suppose that there exists a polynomial $q\in\FF[z]$ such that
\[p(z)=(z-\lambda)q(z)\quad(\forall z\in\FF)\]
Then
\[p(\lambda)=(\lambda-\lambda)q(\lambda)=0,\]
as desired.
\end{proof}

Now we can prove that the degree of a polynomials determines how many zeros it has.

\begin{proposition}
Suppose $n\in\ZZ^+$, $p\in\FF_n[z]$. Then $p$ has at most $n$ zeros in $\FF$.
\end{proposition}

\begin{proof}
Prove by induction on $n$.

The desired result holds for $n=1$ because if $a_1\neq0$ then the polynomial $a_0+a_1z$ has only one zero (which equals $-\frac{a_0}{a_1}$).

Now assume the desired result holds for $n-1$. If $p$ has no zeros in $\FF$, then the desired result holds and we are done. Thus suppose $p$ has a zero $\lambda\in\FF$. By \cref{lemma:factor-thm}, there exists $q\in\FF[z]$ of degree $n-1$ such that
\[p(z)=(z-\lambda)q(z)\quad(\forall z\in\FF)\]
By the induction hypothesis, $q$ has at most $n-1$ zeros in $\FF$. The equation above shows that the zeros of $p$ in $\FF$ are exactly the zeros of $q$ in $\FF$ along with $\lambda$. Thus $p$ has at most $n$ zeros in $\FF$.
\end{proof}

The result above implies that the coefficients of a polynomial are uniquely determined (because if a polynomial had two different sets of coefficients, then subtracting the two representations of the polynomial would give a polynomial with some nonzero coefficients but infinitely many zeros). In particular, the degree of a polynomial is uniquely defined.
\pagebreak

\section{Division Algorithm for Polynomials}
\begin{proposition}[Division algorithm]
Suppose $p,s\in\FF[z]$, $s\neq0$. Then there exists unique polynomials $q,r\in\FF[z]$, where $\deg r<\deg s$, such that
\[p=sq+r.\]
\end{proposition}

\begin{proof}
Let $n=\deg p$, $m=\deg s$. If $n<m$, take $q=0$ and $r=p$ to get the desired equation.

Now assume that $n\ge m$.
\end{proof}
\pagebreak

\section{Factorisation of Polynomials over $\CC$}
\begin{theorem}[Fundamental theorem of algebra, first version]
Every non-constant polynomial with complex coefficients has a zero in $\CC$.
\end{theorem}

\begin{theorem}[Fundamental theorem of algebra]
If $p\in\CC[z]$ is a non-constant polynomial, then $p$ has a unique factorisation (except for the order of the factors) of the form
\[p(z)=c(z-\lambda_1)\cdots(z-\lambda_n),\]
where $c,\lambda_1,\dots,\lambda_n\in\CC$.
\end{theorem}
\pagebreak

\section{Factorisation of Polynomials over $\RR$}
A polynomial with real coefficients may have no real zeros. For example, the polynomial $x^2+1$ has no real zeros.

To obtain a factorisation theorem over $\RR$, we will use our factorisation theorem over $\CC$. We begin with the next result.

\begin{proposition}
Suppose $p\in\CC[z]$ is a polynomial with real coefficients. If $\lambda\in\CC$ is a zero of $p$, then so is the conjugate $\overline{\lambda}$.
\end{proposition}

We want a factorisation theorem for polynomials with real coefficients. We begin with the following result.

\begin{lemma}[Factorisation of quadratic polynomial]
Suppose $b,c\in\RR$. Then there is a polynomial factorisation of the form
\[x^2+bx+c=(x-\lambda_1)(x-\lambda_2)\]
with $\lambda_1,\lambda_2\in\RR$ if and only if $b^2\ge 4c$.
\end{lemma}

\begin{theorem}[Factorisation of polynomial over $\RR$]
Suppose $p\in\RR[x]$ is a non-constant polynomial. Then $p$ has a unique factorisation (except for the order of the factors) of the form
\[p(x)=c(x-\lambda_1)\cdots(x-\lambda_n)(x^2+b_1x+c_1)\cdots(x^2+b_Nx+c_N),\]
where $c,\lambda_1,\dots,\lambda_n,b_1,\dots,b_N,c_1,\dots,c_N\in\RR$, with ${b_k}^2<4c_k$ for each $k$.
\end{theorem}
    \chapter{Eigenvalues and Eigenvectors}
\section{Invariant Subspaces}
\subsection{Eigenvalues}
\begin{definition}[Operator]
A linear map from a vector space to itself is called an \vocab{operator}\index{operator}.
\end{definition}

\begin{definition}[Invariant subspace]
Suppose $T\in\mathcal{L}(v)$. $U\le V$ is called \vocab{invariant}\index{invariant subspace} under $T$ if $Tu\in U$ for all $u\in U$.
\end{definition}

\begin{definition}[Eigenvalue]
Suppose $T\in\mathcal{L}(V)$. $\lambda\in\FF$ is called an \vocab{eigenvalue}\index{eigenvalue} of $T$ if there exists $v\in V$, $v\neq\vb{0}$ such that $Tv=\lambda v$.
\end{definition}

\begin{lemma}[Equivalent conditions to be an eigenvalue]
Suppose $V$ is finite-dimensional, $T\in\mathcal{L}(V)$, $\lambda\in\FF$. Then the following are equivalent:
\begin{enumerate}[label=(\arabic*)]
\item $\lambda$ is an eigenvalue of $T$.
\item $T-\lambda I$ is not injective.
\item $T-\lambda I$ is not surjective.
\item $T-\lambda I$ is not invertible.
\end{enumerate}
\end{lemma}

\begin{definition}[Eigenvector]
Suppose $T\in\mathcal{L}(V)$, $\lambda\in\FF$ is an eigenvalue of $T$. A vector $v\in V$, $v\neq\vb{0}$ is called an \vocab{eigenvector}\index{eigenvector} of $T$ corresponding to $\lambda$ if $Tv=\lambda v$.
\end{definition}

\begin{proposition}[Linearly independent eigenvectors]
Suppose $T\in\mathcal{L}(V)$. Then every list of eigenvectors of $T$ corresponding to distinct eigenvalues of $T$ is linearly independent.
\end{proposition}

\begin{proposition}
Suppose $V$ is finite-dimensional. Then each operator on $V$ has at most $\dim V$ distinct eigenvalues.
\end{proposition}

\subsection{Polynomials Applied to Operators}
\begin{notation}
Suppose $T\in\mathcal{L}(V)$, $n\in\ZZ^+$. $T^n\in\mathcal{L}(V)$ is defined by $T^n=\underbrace{T\cdots T}_\text{$m$ times}$. $T^0$ is defined to be the identity operator $I$ on $V$. If $T$ is invertible with inverse $T^{-1}$, then $T^{-n}\in\mathcal{L}(V)$ is defined by $T^{-n}=\brac{T^{-1}}^n$.
\end{notation}



\section{The Minimal Polynomial}
\section{Upper-Triangular Matrices}
\section{Diagonalisable Operators}
\section{Commuting Operators}
\fi

%%%%%%%%%%%%%%% REAL ANALYSIS
\ifranalysis
    \part{Real Analysis}\label{part:real-analysis}
    \chapter{Real and Complex Number Systems}\label{chap:number-systems}

\begin{learn}
\item Discuss the construction and properties of the real field $\RR$.
\item Discuss the construction and properties of the complex field $\CC$.
\item Discuss the construction and properties of the Euclidean space $\RR^n$.
\end{learn}

\begin{comment}
\section{Natural Numbers}
In Peano's development, it is assumed that there is a set $\NN$ (the natural numbers) of undefined objects with a distinguished element $1$ such that
\begin{enumerate}[label=(\roman*)]
\item $1$ is a natural number; that is $1\in\NN$;
\item every $n\in\NN$ has a successor $S(n)\in\NN$;
\item for every $n$, $S(n)\neq1$ (there is no number with 1 as successor)
\item if $S(n)=S(m)$, then $n=m$;
\item if $A$ is a set of natural numbers such that $1\in A$ and $n\in A\implies S(n)\in A$, then $A$ contains all natural numbers.
\end{enumerate}
These are known as \vocab{Peano's axioms}.

\begin{theorem}[Archimedean property of $\NN$]
$\NN$ is not bounded above.
\end{theorem}

\begin{proof}
Suppose, for a contradiction, that $\NN$ is bounded above. Then $\NN$ is non-empty and bounded above, so by completeness (of $\RR$) $\NN$ has a supremum.

By the Approximation property with $\epsilon=\frac{1}{2}$, there is a natural number $n\in\NN$ such that $\sup\NN-\frac{1}{2}<n\le\sup\NN$.

Now $n+1\in\NN$ and $n+1>\sup\NN$. This is a contradiction.
\end{proof}
\pagebreak

\section{Integers}
\begin{definition}
For $(a,b),(c,d)\in\NN\times\NN$, we define a relation
\[(a,b)\sim(c,d)\iff a+d=b+c.\]
\end{definition}

\begin{proposition}
$\sim$ is an equivalence relation on $\NN\times\NN$.
\end{proposition}

\begin{proof}
Suppose $(a,b),(c,d),(e,f)\in\NN\times\NN$.
\begin{enumerate}[label=(\roman*)]
\item $\sim$ is reflexive: $(a,b)\sim(a,b)$ because $a+b=b+a$ in $\NN$, by commutativity in $\NN$.
\item $\sim$ is symmetric: If $(a,b)\sim(c,d)$, then $(c,d)\sim(a,b)$ because if $a+d=b+c$, then $c+b=d+a$ in $\NN$.
\item $\sim$ is transitive: 
\end{enumerate}
\end{proof}

%https://www.math.wustl.edu/~freiwald/310integers.pdf
\pagebreak

\section{Rational Numbers}
\begin{notation}
$\ZZ^\prime=\ZZ\setminus\{0\}$.
\end{notation}

\begin{definition}
Let $\sim$ be the binary relation defined on $\ZZ\times\ZZ^\prime$ by
\[ (a,b)\sim(c,d) \iff ad=bc. \]
\end{definition}

\begin{proposition}
$\sim$ is an equivalence on $\ZZ\times\ZZ^\prime$.
\end{proposition}

\begin{proof}
We just check that $\sim$ is transitive. So suppose that $(a,b)\sim(c,d)$ and $(c,d)\sim(e,f)$. Then
\begin{equation*}\tag{1}
ad=bc
\end{equation*}
\begin{equation*}\tag{2}
cf=de
\end{equation*}
Multiplying (1) by $f$ and (2) by $b$, we obtain
\begin{equation*}\tag{3}
adf=bcf
\end{equation*}
\begin{equation*}\tag{4}
bcf=bde
\end{equation*}
Hence $adf=bde$. Since $d\neq0$, the Cancellation Law implies that $af=bc$. Hence $(a,b)\sim(e,f)$.
\end{proof}

\begin{definition}
The set of \vocab{rational numbers} is defined by
\[\QQ\coloneqq\ZZ\times\ZZ^\prime/\sim\]
i.e. $\QQ$ is the set of $\sim$ equivalence classes.
\end{definition}

\begin{notation}
For each $(a,b)\in\ZZ\times\ZZ^\prime$, the corresponding equivalence class is denoted by $[(a,b)]$.
\end{notation}

We define addition $+_\QQ$ and multiplication $\cdot_\QQ$ on $\QQ$ as follows:
\[[(a,b)]+_\QQ[(c,d)]=[(ad+bc,bd)].\]
\[[(a,b)]\cdot_\QQ[(c,d)]=[(ac,bd)].\]

\begin{proposition}
$+_\QQ$ and $\cdot_\QQ$ are well-defined.
\end{proposition}

\begin{lemma}
$\QQ$ is a field, with addition and multiplication as defined above.
\end{lemma}

\begin{proof}
We check the field axioms.
\begin{enumerate}[label=(\roman*)]
\item commutativity of addition
\item associativity of addition
\item Let $0_\QQ=[(0,1)]$. We now show that $0_\QQ$ is an additive identity.

Let $q=[(a,b)]$. Then
\begin{align*}
q+_\QQ 0_\QQ&=[(a,b)]+_\QQ[(0,1)]\\
&=[(a\cdot1+0\cdot b,b\cdot1)]\\
&=[(a,b)]\\
&=q.
\end{align*}
Since for any $q\in\QQ$, $q+_\QQ0_\QQ=q$, thus $0_\QQ$ is an additive identity. Hence an additive identity exists.

\item Consider $r=[(-a,b)]$. Then
\begin{align*}
q+_\QQ r&=[(a,b)]+_\QQ[(-a,b)]\\
&=[(ab+(-a)b,b^2)]\\
&=[(0,b^2)]
\end{align*}
Since $0\cdot1=0\cdot b^2$, we have $(0,b^2)=(0,1)$. Hence 
\begin{align*}
q+_\QQ r&=[(0,b^2)]\\
&=[(0,1)]\\
&=0_\QQ
\end{align*}
Since for any $q\in\QQ$, there exists a unique $r\in\QQ$ such that $q+_\QQ r=0_\QQ$, hence the additive inverse exists.

\item commutativity of multiplication

We want to show that for all $q,r\in\QQ$, $q\cdot_\QQ r=r\cdot_\QQ q$.

\item associativity of multiplication

We want to show that for all $q,r\in\QQ$, $(q\cdot_\QQ r)\cdot_\QQ s=q\cdot_\QQ(r\cdot_\QQ s)$.

\item distributivity

We want to show that for all $q,r,s\in\QQ$, $q\cdot_\QQ(r+_\QQ s)=(q\cdot_\QQ r)+_\QQ(q\cdot_\QQ s)$.

\item Let $1_\QQ=[(1,1)]$. We now show that $1_\QQ$ is a multiplicative identity.

Let $q=[(a,b)]$. Then
\begin{align*}
q\cdot_\QQ1_\QQ
&=[(a,b)]\cdot_\QQ[(1,1)]\\
&=[(a\cdot1,b\cdot1)]\\
&=[(a,b)]\\\
&=1_\QQ
\end{align*}

Since for all $q\in\QQ$, $q\cdot_\QQ1_\QQ=q$, $1_\QQ$ is a multiplicative identity. Hence a multiplicative identity exists.

\item Suppose that $q=[(a,b)]\neq[(0,1)]$. Then $a\neq0$ and so $(b,a)\in\ZZ\times\ZZ^\prime$. Let $r=[(b,a)]$. Then
\begin{align*}
q\cdot_\QQ r
&=[(a,b)]\cdot_\QQ[(b,a)]\\
&=[(ab,ba)]\\
&=[(1,1)]\\
&=1_\QQ.
\end{align*}
\end{enumerate}
Since for every $0_\QQ\neq q\in\QQ$, there exists a unique $r\in\QQ$ such that $q\cdot_\QQ r=1_\QQ$, thus $r$ is a multiplicative inverse. Hence a multiplicative inverse exists.
\end{proof}

Since $\QQ$ is a field, we have the following results:
\begin{enumerate}[label=(\roman*)]
\item The additive identity in $\QQ$ is unique.
\item The additive inverse of an element of $\QQ$ is unique.
\item The multiplicative identity of $\QQ$ is unique.
\item The multiplicative inverse of a nonzero element of $\QQ$ is unique.
\end{enumerate}

\begin{notation}
Since the additive inverse is unique, we denote the additive inverse of $q\in\QQ$ by $-q$; we define the binary operation $-_\QQ$ on $\QQ$ by
\[q-_\QQ r=q+_\QQ(-r).\]
\end{notation}

\begin{notation}
Since the multiplicative inverse is unique, we denote the additive inverse of $q\in\QQ$ by $q^{-1}$.
\end{notation}

Finally we want to define an order relation on $\QQ$.
\begin{definition}[Order on $\QQ$]
Suppose that $r,s\in\QQ$ and that $r=[(a,b)]$ and $s=[(c,d)]$, where $b,d>0$. Then
\[r\le_\QQ s\iff ad<bc.\]
\end{definition}

\begin{proposition}
$<_\QQ$ is well-defined.
\end{proposition}

\begin{definition}
If $q\in\QQ$, then
\begin{itemize}
\item $q$ is \vocab{positive} if and only if $0_\QQ<_\QQ q$,
\item $q$ is \vocab{negative} if and only if $q<_\QQ0_\QQ$.
\end{itemize}
\end{definition}

\begin{definition}
If $q\in\QQ$, then the \vocab{absolute value} of $q$ is
\[|q|=\begin{cases}
-q&\text{if $q$ is negative,}\\
q&\text{if otherwise.}
\end{cases}\]
\end{definition}
\pagebreak
\end{comment}

\section{Real Numbers}
$\QQ$ has some problems, the first of which being \emph{algebraic incompleteness}: there exists equations with coefficients in $\QQ$ but do not have solutions in $\QQ$ (in fact $\RR$ has this problem too, but $\CC$ is algebraically complete, by the Fundamental Theorem of Algebra).

\begin{lemma}
$x^2-2=0$ has no solution in $\QQ$.
\end{lemma}

\begin{proof}
Suppose, for a contradiction, that $x^2-2=0$ has a solution $x=\frac{p}{q}$, $q\neq0$. We also assume $\frac{p}{q}$ is in lowest terms; that is, $p,q$ are coprime. Squaring both sides gives $\frac{p^2}{q^2}=2$, or $p^2=2q^2$. Observe that $p^2$ is even, so $p$ is even; let $p=2m$ for some integer $m$. Then this implies $4m^2=2q^2$, or $2m^2=q^2$. Similarly, $q^2$ is even so $q$ is even.

Since $p$ and $q$ share a common factor of $2$, we have reached a contradiction.
\end{proof}

The second problem is \emph{analytic incompleteness}: there exists a sequence of rational numbers that approach a point that is not in $\QQ$; for example, the sequence
\[1,1.4,1.41,1.414,1.4142,\dots\]
tends to the the irrational number $\sqrt{2}$.

Continuing from the above lemma,
\begin{lemma}
Let
\begin{align*}
A&=\{p\in\QQ\mid p>0,p^2<2\},\\
B&=\{p\in\QQ\mid p>0,p^2>2\}.
\end{align*}
Then $A$ contains no largest number, and $B$ contains no smallest number.
\end{lemma}

\begin{proof}
Prove by construction. We associate with each rational $p>0$ the number
\[q=p-\frac{p^2-2}{p+2}=\frac{2p+2}{p+2}\]
and so
\[q^2-2=\frac{2(p^2-2)}{(p+2)^2}.\]

For any $p\in A$, $q>p$ and $q\in A$ since $q^2<2$, so $A$ has no largest number.

For any $p\in B$, $q<p$ and $q\in B$ since $q^2>2$, so $B$ has no smallest number.
\end{proof}

A direct consequence of this is that $\QQ$ does not have the least-upper-bound property, for $A\subset\QQ$ is bounded above but $A$ has no least upper bound in $\QQ$ [$B$ is the set of all upper bounds of $A$, and $B$ does not have a smallest element].

\subsection{Real Field}
\begin{definition}[Ordered field]
A field $F$ is an \vocab{ordered field} if there eists an order $<$ on $F$ such that for all $x,y,z\in F$,
\begin{enumerate}[label=(\roman*)]
\item if $y<z$ then $x+y<x+z$;
\item if $x>0$ and $y>0$ then $xy>0$.
\end{enumerate}
\end{definition}

\begin{proposition}[Basic properties]
The following statements are true in every ordered field.
\begin{enumerate}[label=(\roman*)]
\item If $x>0$ then $-x<0$, and vice versa.
\item If $x>0$ and $y<z$ then $xy<xz$.
\item If $x<0$ and $y<z$ then $xy>xz$.
\item If $x\neq0$ then $x^2>0$. In particular, $1>0$.
\item If $0<x<y$ then $0<\frac{1}{y}<\frac{1}{x}$.
\end{enumerate}
\end{proposition}

\begin{theorem}[Existence of real field]
There exists an ordered field $\RR$ that
\begin{enumerate}[label=(\roman*)]
\item contains $\QQ$ as a subfield, and
\item has the least-upper-bound property (also known as the completeness axiom).
\end{enumerate}
\end{theorem}

\begin{proof}
We prove by contruction, as follows. 
\end{proof}

One method to construct $\RR$ from $\QQ$ is Dedekind cuts.

\begin{definition*}[Dedekind cut]
A \vocab{Dedekind cut}\index{Dedekind cut} $\alpha\subset\QQ$ satisfies the following properties:
\begin{enumerate}[label=(\roman*)]
\item $\alpha\neq\emptyset$, $\alpha\neq\QQ$;
\item if $p\in\alpha$, $q\in\QQ$ and $q<p$, then $q\in\alpha$;
\item if $p\in\alpha$, then $p<r$ for some $r\in\alpha$.
\end{enumerate}
\end{definition*}

Note that (iii) simply says that $\alpha$ has no largest member; (ii) implies two facts which will be used freely:
\begin{itemize}
\item If $p\in\alpha$ and $q\notin\alpha$ then $p<q$.
\item If $r\notin\alpha$ and $r<s$ then $s\notin\alpha$.
\end{itemize}

\begin{example}
Let $r\in\QQ$ and define
\[ \alpha_r\coloneqq\{p\in\QQ\mid p<r\}. \]
We now check that this is indeed a Dedekind cut.
\begin{enumerate}[label=(\roman*)]
\item $p=1+r\notin\alpha_r$ thus $\alpha_r\neq\QQ$. $p=r-1\in\alpha_r$ thus $\alpha_r\neq\emptyset$.

\item Suppose that $q\in\alpha_r$ and $q^\prime<q$. Then $q^\prime<q<r$ which implies that $q^\prime<r$ thus $q^\prime\in\alpha_r$.

\item Suppose that $q\in\alpha_r$. Consider $\dfrac{q+r}{2}\in\QQ$ and $q<\dfrac{q+r}{2}<r$. Thus $\dfrac{q+r}{2}\in\alpha_r$.
\end{enumerate}
\end{example}

This example shows that every rational $r$ corresponds to a Dedekind cut $\alpha_r$.

\begin{example}
$\sqrt[3]{2}$ is not rational, but it is real. $\sqrt[3]{2}$ corresponds to the cut
\[ \alpha=\{p\in\QQ\mid p^3<2\}. \]
\begin{enumerate}[label=(\roman*)]
\item Trivial.
\item If $q<p$, by the monotonicity of the cubic function, this implies that $q^3<p^3<2$ thus $q\in\alpha$.
\item If $p\in\alpha$, consider $\brac{p+\frac{1}{n}}^3<2$.
\end{enumerate}
\end{example}

\begin{definition*}
The set of real numbers, denoted by $\RR$, is the set of all Dedekind cuts:
\[\RR\coloneqq\{\alpha\mid\alpha\text{ is a Dedekind cut}\}.\]
\end{definition*}

\begin{proposition*}
$\RR$ has an order, where $\alpha<\beta$ is defined to mean that $\alpha\subset\beta$.
\end{proposition*}

\begin{proof}
Let us check if this is a valid order (check for transitivity and trichotomy).
\begin{enumerate}[label=(\roman*)]
\item For $\alpha,\beta,\gamma\in\RR$, if $\alpha<\beta$ and $\beta<\gamma$ it is clear that $\alpha<\gamma$. (A proper subset of a proper subset is a proper subset.)

\item It is clear that at most one of the three relations
\[ \alpha<\beta, \quad \alpha=\beta, \quad \beta<\alpha \]
can hold for any pair $\alpha,\beta$. 

To show that at least one holds, assume that the first two fail. Then $\alpha$ is not a subset of $\beta$. Hence there exists some $p\in\alpha$ with $p\in\beta$.

If $q\in\beta$, it follows that $q<p$ (since $p\notin\beta$), hence $q\in\alpha$, by (ii). Thus $\beta\subset\alpha$. Since $\beta\neq\alpha$, we conclude that $\beta<\alpha$.
\end{enumerate}
Thus $\RR$ is an ordered set.
\end{proof}

\begin{proposition*}
The ordered set $\RR$ has the least-upper-bound property.
\end{proposition*}

\begin{proof}
Let $A\neq\emptyset$, $A\subset\RR$. Assume that $\beta\in\RR$ is an upper bound of $A$.

Define $\beta$ to be the union of all $\alpha\in A$; in other words, $p\in\gamma$ if and only if $p\in\alpha$ for some $\alpha\in A$. We shall prove that $\gamma\in\RR$ by checking the definition of Dedekind cuts:
\begin{enumerate}[label=(\roman*)]
\item Since $A$ is not empty, there exists an $\alpha_0\in A$. This $\alpha_0$ is not empty. Since $\alpha_0\subset\gamma$, $\gamma$ is not empty.

Next, $\gamma\subset\beta$ (since $\alpha\subset\beta$ for every $\alpha\in A$), and therefore $\gamma\neq\QQ$.

\item Pick $p\in\gamma$. Then $p\in\alpha_1$ for some $\alpha_1\in A$. If $q<p$, then $q\in\alpha_1$, hence $q\in\gamma$.

\item If $r\in\alpha_1$ is so chosen that $r>p$, we see that $r\in\gamma$ (since $\alpha_1\subset\gamma$).
\end{enumerate}

Next we prove that $\gamma=\sup A$.
\begin{enumerate}[label=(\roman*)]
\item It is clear that $\alpha\le\gamma$ for every $\alpha\in A$.
\item Suppose $\delta<\gamma$. Then there is an $s\in\gamma$ and that $s\notin\delta$. Since $s\in\gamma$, $s\in\alpha$ for some $\alpha\in A$. Hence $\delta<\alpha$, and $\delta$ is not an upper bound of $A$.
\end{enumerate}
\end{proof}

\begin{definition*}[Addition]
Given $\alpha,\beta\in\RR$, addition is defined as
\[\alpha+\beta\coloneqq\{r\in\QQ\mid r=a+b,a\in\alpha,b\in\beta\}.\]
\end{definition*}

\begin{proposition*}[Addition on $\RR$ is closed]
For all $\alpha,\beta\in\RR$, $\alpha+\beta\in\RR$.
\end{proposition*}

\begin{proof}
We check that $\alpha+\beta$ is a Dedekind cut:
\begin{enumerate}[label=(\roman*)]
\item $\alpha\neq\emptyset$ and $\beta\neq\emptyset$ implies there exists $a\in\alpha$ and $b\in\beta$. Hence $r=a+b\in\alpha+\beta$ so $\alpha+\beta\neq\emptyset$.

Since $\alpha\neq\QQ$ and $\beta\neq\QQ$, there is $c\neq\alpha$ and $d\neq\beta$. $r^\prime=c+d>a+b$ for any $a\in\alpha,b\in\beta$, so $r^\prime\notin\alpha+\beta$. Hence $\alpha+\beta\neq\QQ$.

\item Suppose that $r\in\alpha+\beta$ and $r^\prime<r$. We want to show that $r^\prime\in\alpha+\beta$.

$r=a+b$ for some $a\in\alpha,b\in\beta$. $r^\prime-a<b$. Since $\beta\in\RR$, $r^\prime-a\in\beta$ so $r^\prime-a=b_1$ for some $b_1\in\beta$. Hence $r^\prime=a+b_1\in\alpha+\beta$.

\item Suppose $r\in\alpha+\beta$, so $r=a+b$ for some $a\in\alpha,b\in\beta$. There exists $a^\prime\in\alpha,b^\prime\in\beta$ with $a<a^\prime$ and $b<b^\prime$. Then $r=a+b<a^\prime+b^\prime\in\alpha+\beta$. We define $r^\prime=a^\prime+b^\prime\in\alpha+\beta$ with $r<r^\prime$.
\end{enumerate}
\end{proof}

\begin{proposition*} \
\begin{enumerate}[label=(\roman*)]
\item Addition on $\RR$ is commutative:
$\forall\alpha,\beta\in\RR$, $\alpha+\beta=\beta+\alpha$.
\item Addition on $\RR$ is associative:
$\forall\alpha,\beta,\gamma\in\RR$, $\alpha+(\beta+\gamma)=(\alpha+\beta)+\gamma$.
\item Define $0^*\coloneqq\{p\in\QQ\mid p<0\}$. Then $\alpha+0^*=\alpha$.
\item Fix $\alpha\in\RR$, define $\beta=\{p\in\QQ\mid\exists r>0\suchthat-p-r\notin\alpha\}$. Then $\alpha+\beta=0^*$
\end{enumerate}
\end{proposition*}

\begin{proof} \
\begin{enumerate}[label=(\roman*)]
\item We need to show that $\alpha+\beta\subset\beta+\alpha$ and $\beta+\alpha\subset\alpha+\beta$.

Let $r\in\alpha+\beta$. Then $r=a+b$ for $a\in\alpha$ and $b\in\beta$. Thus $r=b+a$ since $+$ is commutative on $\QQ$. Hence $r\in\beta+\alpha$. Therefore $\alpha+\beta\subset\beta+\alpha$.

Similarly, $\beta+\alpha\subset\alpha+\beta$.

Therefore $\alpha+\beta=\beta+\alpha$.

\item Let $r\in\alpha+(\beta+\gamma)$. Then $r=a+(b+c)$ where $a\in\alpha,b\in\beta,c\in\gamma$. Thus $r=(a+b)+c$ by associativity of $+$ on $\QQ$. Therefore $r\in(\alpha+\beta)+\gamma$, hence $\alpha+(\beta+\gamma)\subset(\alpha+\beta)+\gamma$.

Similarly, $(\alpha+\beta)+\gamma\subset\alpha+(\beta+\gamma)$.

\item Let $r\in\alpha+0^*$. Then $r=a+p$ for some $a\in\alpha,p\in0^*$. Thus $r=a+p<a+0=a$ by ordering on $\QQ$ and identity on $\QQ$. Hence $\alpha+0^*\subset\alpha$.

Let $r\in\alpha$. Then there exists $r^\prime>p$ where $r^\prime\in\alpha$. Thus $r-r^\prime<0$, so $r-r^\prime\in0^*$. We see that
\[ r=\underbrace{r^\prime}_{\in\alpha}+\underbrace{(r-r^\prime)}_{\in0^*}. \]
Hence $\alpha\subset\alpha+0^*$.

\item We first need to show that $\beta$ is a Dedekind cut.
\begin{enumerate}[label=(\roman*)]
\item If $s\notin\alpha$ and $p=-s-1$, then $-p-1\notin\alpha$, hence $p\in B$, so $\beta\neq\emptyset$. If $q\in\alpha$, then $-q\notin\beta$ so $\beta\neq\QQ$.
\item Pick $p\in\beta$ and pick $r>0$ such that $-p-r\notin\alpha$. If $q<p$, then $-q-r>-p-r$, hence $-q-r\notin\alpha$. Thus $q\in\beta$.
\item Put $t=p+\frac{r}{2}$. Then $t>p$, and $-t-\frac{r}{2}=-p-r\notin\alpha$, so $t\in\beta$.
\end{enumerate}
Hence $\beta\in\RR$.

If $r\in\alpha$ and $s\in\beta$, then $-s\notin\alpha$, hence $r<-s$ so $r+s<0$. Thus $\alpha+\beta\subset0^*$.

To prove the opposite inclusion, pick $v\in0^*$, put $w=-\frac{v}{2}$. Then $w>0$, and there exists $n\in\NN$ such that $nw\in\alpha$ but $(n+1)w\notin\alpha$, by the Archimedean property on $\QQ$. Put $p=-(n+2)w$. Then $p\in\beta$, since $-p-w\notin\alpha$, and
\[v=nw+p\in\alpha+\beta.\]
Thus $0*\subset\alpha+\beta$. We conclude that $\alpha+\beta=0^*$.

This $\beta$ will of course be denoted by $-\alpha$.
\end{enumerate}
\end{proof}

We say that a Dedekind cut $\alpha$ is \emph{positive} if $0\in\alpha$ and negative if $0\notin\alpha$. If $\alpha$ is neither positive nor negative, then $\alpha=0^*$.

Multiplication is a little more bothersome than addition in the present context, since products of negative rationals are positive. For this reason we confine ourselves first to $\RR^+$, the set of all $\alpha\in\RR$ with $\alpha>0*$.

For all $\alpha,\beta\in\RR^+$, we define multiplication as
\[\alpha\cdot\beta\coloneqq\{p\in\QQ\mid p\le ab,a\in\alpha,b\in\beta,a,b>0\}.\]

We define $1^*\coloneqq\{q\in\QQ\mid q<1\}$.



\begin{proposition*}[Multiplication on $\RR$ is closed]
For all $\alpha,\beta\in\RR$, $\alpha\cdot\beta\in\RR$.
\end{proposition*}

\begin{proof} \
\begin{enumerate}[label=(\roman*)]
\item $\alpha\neq\emptyset$ means there exists $a\in\alpha,a>0$. Similarly, $\beta\neq\emptyset$ means there exists $b\in\beta,b>0$. Then $a\cdot b\in\QQ$ and $ab\le ab$, so $ab\in\alpha\cdot\beta\neq\emptyset$.

$\alpha\neq\QQ$ means there exists $a^\prime\notin\alpha,a^\prime>a$ for all $a\in\alpha$. $\beta\neq\QQ$ means there exists $b^\prime\in\beta,b^\prime>b$ for all $b\in\beta$. Then $a^\prime b^\prime>ab$ for all $a\in\alpha,b\in\beta$, so $a^\prime b^\prime\notin\alpha\cdot\beta$, thus $\alpha\cdot\beta\neq\QQ$.

\item $p<\alpha\cdot\beta$ means $p\le a\cdot b$ for some $a\in\alpha,b\in\beta,a,b>0$.

For $q<p$, $q<p\le a\cdot b$ so $q\in\alpha\cdot\beta$.

\item $p\in\alpha\cdot\beta$ means $p\le a\cdot b$ for some $a\in\alpha,b\in\beta,a,b>0$. Pick $a^\prime\in\alpha$ and $b^\prime\in\beta$ with $a^\prime>a$ and $b^\prime>b$. Form $a^\prime b^\prime>ab\ge p$, $a^\prime b^\prime\le a^\prime b^\prime$ means $a^\prime b^\prime\in\alpha\cdot\beta$.
\end{enumerate}
Hence $\alpha\cdot\beta$ is a Dedekind cut.
\end{proof}


We complete the definition of multiplication by setting $\alpha0^*=0^*=0^*\alpha$, and by setting
\[\alpha\cdot\beta=\begin{cases}
(-\alpha)(-\beta)&a<0^*,\beta<0^*,\\
-[(-\alpha)\beta]&a<0^*,\beta>0^*,\\
-[\alpha\cdot(-\beta)]&\alpha>0^*,\beta<0^*.
\end{cases}\]

\subsection{Properties of $\RR$}
We now discuss properties of $\RR$.

\begin{theorem}[$\RR$ is archimedian]\label{thrm:r-archimedian}
For any $x\in\RR^+$ and $y\in\RR$, there exists $n\in\NN$ such that $nx>y$.
\end{theorem}

\begin{proof}
Suppose, for a contradiction, that $nx\le y$ for all $n\in\NN$. Then $y$ is an upper bound of $A=\{nx\mid n\in\NN\}$. Since $\RR$ has the least-upper-bound property and $A\subset R$ is bounded above, $M=\sup A\in\RR$.

Consider $M-x$. Since $M-x<M=\sup A$, $M-x$ is not an upper bound of $A$. Then there exists $n_0\in\NN$ such that $M-x\le n_0x$, or $M\le(n_0+1)x$, which is a contradiction.
\end{proof}

\begin{corollary}
Let $\epsilon>0$. Then there exists $n\in\NN$ such that $0<\frac{1}{n}<\epsilon$.
\end{corollary}

\begin{proof}
Take $x=\epsilon$ and $y=1$.
\end{proof}

\begin{theorem}[$\QQ$ is dense in $\RR$]
For any $x,y\in\RR$ with $x<y$, there exists $p\in\QQ$ such that $x<p<y$.
\end{theorem}

\begin{proof}
Prove by construction.

Since $x<y$, we have $y-x>0$. By the archimedian property, there exists $n\in\NN$ such that
\[n(y-x)>1.\]
Apply the archimedian property again to obtain $m_1,m_2\in\NN$ such that $m_1>nx$ and $m_2>-nx$. Then
\[-m_2<nx<m_1.\]
Hence there exists $m\in\NN$ (with $-m_2\le m\le m_1$) such that
\[m-1\le nx<m.\]
If we combine there inequalities, we obtain
\[nx<m\le1+nx<ny.\]
Since $n>0$, it follows that
\[x<\frac{m}{n}<y.\]
Take $p=\frac{m}{n}$, and we are done.
\end{proof}

\begin{theorem}[$\RR$ is closed under taking roots]
For every $x\in\RR^+$ and every $n\in\NN$, there exists a unique $x\in\RR^+$ so that $y^n=x$.
\end{theorem}

\begin{proof}
That there is at most one such $y$ is clear, since $0<y_1<y_2$ implies $y_1^n<y_2^n$. Let
\[E=\{t\in\RR^+\mid t^n<x\}.\]
We first show that $E$ has a supremum:
\begin{enumerate}[label=(\roman*)]
\item If $t=\frac{x}{1-x}$ then $0\le t<1$. Hence $t^n\le t<x$. Thus $t\in E$, and $E\neq\emptyset$.
\item If $t>1+x$ then $t^n\ge t>x$, so that $t\notin E$. Thus $1+x$ is an upper bound of $E$.
\end{enumerate}
Hence $E$ has a supremum; let $y=\sup E$.

To prove that $y^n=x$ we will show that each of the inequalities $y^n<x$ and $y^n>x$ leads to a contradiction. The identity $b^n-a^n=(b-a)\brac{n^{n-1}+b^{n-2}a+\cdots+a^{n-1}}$ yields the inequality
\[b^n-a^n<(b-a)nb^{n-1}\]
when $0<a<b$.

Assume $y^n<x$. Choose $h$ so that $0<h<1$ and
\[h<\frac{x-y^n}{n(y+1)^{n-1}}.\]
Put $a=y$, $b=y+h$. Then
\[(y+h)^n-y^n<hn(y+h)^{n-1}<hn(y+1)^{n-1}<x-y^n.\]
Thus $(y+h)^n<x$, and $y+h\in E$. Since $y+h>y$, this contradicts the fact that $y$ is an upper bound of $E$.

Now assume $y^n>x$. Put
\[k=\frac{y^n-x}{ny^{n-1}}.\]
Then $0<k<y$. If $t\ge y-k$, we conclude that
\[y^n-t^n\le y^n-(y-k)^n<kny^{n-1}=y^n-x.\]
Thus $t^n>x$, and $t\notin E$. It follows that $y-k$ is an upper bound of $E$. But $y-k<y$, which contradicts the fact that $y$ is the \emph{least} upper bound of $E$.

Hence $y^n=x$, and the proof is complete.
\end{proof}

\begin{notation}
This number $y$ is written $\sqrt[n]{x}$ or $x^\frac{1}{n}$.
\end{notation}

\begin{corollary}
If $a,b\in\RR^+$ and $n\in\NN$, then
\[(ab)^\frac{1}{n}=a^\frac{1}{n}b^\frac{1}{n}.\]
\end{corollary}

\begin{proof}
Put $\alpha=a^\frac{1}{n}$, $\beta=b^\frac{1}{n}$. Then
\[ab=\alpha^n\beta^n=(\alpha\beta)^n\]
since multiplication is commutative. The uniqueness assertion of the above result shows that
\[(ab)^\frac{1}{n}=\alpha\beta=a^\frac{1}{n}b^\frac{1}{n}.\]
\end{proof}

\begin{proposition}
Real numbers can be represented by decimal expansions.
\end{proposition}

\begin{proof}
Let $x\in\RR^+$. Let $n_0$ be the largest integer such that $n_0\le x$. (Note that the existence of $n_0$ depends on the archimedian property of $\RR$.) Having chosen $n_0,n_1,\dots,n_{k-1}$, let $n_k$ be the largest integer such that
\[n_0+\frac{n_1}{10}+\cdots+\frac{n_k}{10^k}\le x.\]
Let
\[E=\crbrac{n_0+\frac{n_1}{10}+\cdots+\frac{n_k}{10^k}\:\bigg|\:k=0,1,2,\dots}.\]
Then $x=\sup E$. The decimal expansion of $x$ is
\[n_0.n_1n_2n_3\cdots.\]
Conversely, for any infinite decimal, $E$ is bounded above, and $n_0.n_1n_2n_3\cdots$ is the decimal expansion of $\sup E$.
\end{proof}

\subsection{Extended Real Number System}
\begin{definition}[Extended real number system]
We add two symbols $+\infty$ and $-\infty$ to $\RR$, and denote the union
\[\overline{\RR}=\RR\cup\{\pm\infty\},\]
known as the \vocab{extended real number system}. We preserve the original order in $\RR$, and define
\[-\infty<x<+\infty\]
for every $x\in\RR$.
\end{definition}

\begin{proposition}
Any non-empty $E\subset\overline{\RR}$ has a supremum and infimum.
\end{proposition}

\begin{proof}
If $E$ is bounded above in $\RR$, then we are done. If $E$ is not bounded above in $\RR$, then $\sup E=+\infty$ in the extended real number system.

Exactly the same remarks apply to lower bounds.
\end{proof}

The extended real number system does not form a field, but it is customary to make the following conventions:
\begin{enumerate}[label=(\roman*)]
\item If $x$ is real then
\[ x+\infty=+\infty, \quad x-\infty=-\infty, \quad \frac{x}{+\infty}=\frac{x}{-\infty}=0. \]
\item If $x>0$ then $x\cdot(+\infty)=+\infty$, $x\cdot(-\infty)=-\infty$.
\item If $x<0$ then $x\cdot(+\infty)=-\infty$, $x\cdot(-\infty)=+\infty$.
\end{enumerate}
When it is desired to make the distinction between real numbers on the one hand and the symbols $+\infty$ and $-\infty$ on the other quite explicit, the former are called \emph{finite}.

\section{Complex Field}
Consider the Cartesian product
\[\RR^2\coloneqq\RR\times\RR=\{(a,b)\mid a,b\in\RR\}.\]
A \emph{complex number} is an ordered pair $(a,b)\in\RR^2$. Let $x=(a,b)$, $y=(c,d)$ be two complex numbers. We write $x=y$ if and only if $a=c$ and $b=d$. Define addition and multiplication on $\RR^2$ as
\begin{align*}
x+y&=(a+c,b+d),\\
xy&=(ac-bd,ad+bc).
\end{align*}

\begin{proposition}
$\RR^2$, with addition and multiplication defined as above, is a field, with additive identity $(0,0)$ and multiplicative identity $(1,0)$. We call this structure $\CC$, the \vocab{complex field}.
\end{proposition}

\begin{proof}
Check the field axioms.
\end{proof}

\begin{proposition}
For any $a,b\in\RR$, we have
\begin{align*}
(a,0)+(b,0)&=(a+b,0),\\
(a,0)(b,0)&=(ab,0).
\end{align*}
\end{proposition}

\begin{proof}
Exercise.
\end{proof}

The above result shows that the complex numbers of the form $(a,0)$ have the same arithmetic properties as the corresponding real numbers $a$. We can therefore identify $(a,0)\in\CC$ with $a\in\RR$. This identification gives us $\RR$ as a subfield of $\CC$.

The reader may have noticed that we have defined the complex numbers without any reference to the mysterious square root of $-1$. We now show that the notation $(a,b)$ is equivalent to the more customary $a+bi$.

\begin{definition}[Imaginary number]
$i=(0,1)$.
\end{definition}

\begin{proposition}
$i^2=-1$.
\end{proposition}

\begin{proof}
\[i^2=(0,1)(0,1)=(-1,0)=-1.\]
\end{proof}

\begin{proposition}
For $a,b\in\RR$, $(a,b)=a+bi$.
\end{proposition}

\begin{proof}
\begin{align*}
a+bi&=(a,0)+(b,0)(0,1)\\
&=(a,0)+(0,b)\\
&=(a,b).
\end{align*}
\end{proof}

\begin{definition}
For $a,b\in\RR$, $z=a+bi$, we call $a$ and $b$ the \emph{real part} and \emph{imaginary part} of $z$ respectively, denoted by $a=\Re(z)$, $b=\Im(z)$; $\overline{z}=a-bi$ is called the \emph{conjugate} of $z$.
\end{definition}

\begin{proposition}
For $z,w\in\CC$,
\begin{enumerate}[label=(\roman*)]
\item $\overline{z+w}=\overline{z}+\overline{w}$
\item $\overline{zw}=\overline{z}\overline{w}$
\item $z+\overline{z}=2\Re(z)$, $z-\overline{z}=2i\Im(z)$
\item $z\overline{z}\in\RR$ and $z\overline{z}>0$ (except when $z=0$)
\end{enumerate}
\end{proposition}

\begin{definition}
For $z\in\CC$, its \emph{absolute value} is
\[|z|\coloneqq\brac{z\overline{z}}^\frac{1}{2}.\]
\end{definition}

\begin{proposition}
For $z,w\in\CC$,
\begin{enumerate}[label=(\roman*)]
\item $|z|>0$ unless $z=0$, $|0|=0$
\item $|\overline{z}|=|z|$
\item $|zw|=|z||w|$
\item $|\Re(z)|\le|z|$
\item $|z+w|\le|z|+|w|$
\end{enumerate}
\end{proposition}

\begin{proof} \
\begin{enumerate}[label=(\roman*)]
\item The square root is non-negative, by definition.
\item The conjugate of $\overline{z}$ is $z$, and the rest follows by the definition of absolute value.
\item Let $z=a+bi$, $w=c+di$ with $a,b,c,d\in\RR$. Then
\[|zw|^2=(ac-bd)^2+(ad-bc)^2=(a^2+b^2)(c^2+d^2)=|z|^2|w|^2,\]
or $|zw|^2=\brac{|z||w|}^2$ so the desired result follows.
\item Let $z=a+bi$. Note that $a^2\le a^2+b^2$, hence
\[|\Re(z)|=|a|=\sqrt{a^2}\le\sqrt{a^2+b^2}=|z|.\]
\item Let $z,w\in\CC$. Note that the conjugate of $z\overline{w}$ is $\overline{z}w$, so $z\overline{w}+\overline{z}w=2\Re(z\overline{w})$. Hence
\begin{align*}
|z+w|^2&=(z+w)(\overline{z+w})\\
&=(z+w)(\overline{z}+\overline{w})\\
&=z\overline{z}+z\overline{w}+\overline{z}w+w\overline{w}\\
&=|z|^2+2\Re(z\overline{w})+|w|^2\\
&\le|z|^2+2|z\overline{w}|+|w|^2\\
&=|z|^2+2|z||w|+|w|^2\\
&=\brac{|z|+|w|}^2
\end{align*}
and taking square roots yields the desired result.
\end{enumerate}
\end{proof}

\begin{theorem}[Schwarz inequality]
If $a_1,\dots,a_n,b_1,\dots,b_n\in\CC$, then
\begin{equation}
\absolute{\sum_{i=1}^{n}a_ib_i}^2\le\sum_{i=1}^{n}|a_i|^2\sum_{i=1}^{n}|b_i|^2.
\end{equation}
\end{theorem}

\begin{proof}
Let $A=\sum|a_i|^2$, $B=\sum|b_i|^2$, $C=\sum a_i\overline{b_i}$. If $B=0$, then $b_1=\cdots=b_n=0$, and the conclusion is trivial. Assume therefore that $B>0$. Then we have
\begin{align*}
\sum|Ba_i-Cb_i|^2
&=\sum(Ba_i-Cb_i)(\overline{Ba_i-Cb_i})\\
&=\sum(Ba_i-Cb_i)(B\overline{a_i}-\overline{Cb_i})\\
&=B^2\sum|a_i|^2-B\overline{C}\sum a_i\overline{b_j}-BC\sum\overline{a_i}b_i+|C|^2\sum|b_i|^2\\
&=B^2A-B|C|^2\\
&=B(AB-|C|^2).
\end{align*}
Since each term in the first sum is non-negative, we see that
\[B(AB-|C|^2)\ge0.\]
Since $B>0$, it follows that $AB-|C|^2\ge0$. This is the desired inequality.

(when does equality hold?)
\end{proof}
\pagebreak

\section{Euclidean Spaces}
For $n\in\ZZ^+$, 
\[\RR^n=\{(x_1,\dots,x_n)\mid x_i\in\RR\}\]
where $\vb{x}=(x_1,\dots,x_n)$, $x_i$'s are called the coordinates of $\vb{x}$. The elements of $\RR^n$ are called \emph{points}, or \emph{vectors}.

Define addition and scalar multiplication on $\RR^n$ as follows: for $\vb{x},\vb{y}\in\RR^n$, $\alpha\in\RR$,
\begin{align*}
\vb{x}+\vb{y}&=(x_1+y_1,\dots,x_n+y_n),\\
\alpha\vb{x}&=(\alpha x_1,\dots,\alpha x_n).
\end{align*}

\begin{proposition}
$\RR^n$, with addition and scalar multiplication defined above, is a vector space over $\RR$, where the zero element of $\RR^n$ (sometimes called the origin or the null vector) is the point $\vb{0}$, all of whose coordinates are $0$.
\end{proposition}

\begin{proof}
These two operations satisfy the commutative, associatives, and distributive laws (the proof is trivial, in view of the analagous laws for the real numbers).
\end{proof}

We define the \emph{inner product} of $\vb{x}$ and $\vb{y}$ by
\[\vb{x}\cdot\vb{y}\coloneqq\sum_{i=1}^nx_iy_i,\]
and the \emph{norm} of $\vb{x}$ by
\[\norm{\vb{x}}\coloneqq\brac{\vb{x}\cdot\vb{x}}^\frac{1}{2}=\brac{\sum_{i=1}^n{x_i}^2}^\frac{1}{2}.\]
The structure now defined (the vector space $\RR^n$ with the above inner product and norm) is called the \vocab{Euclidean $n$-space}.

\begin{proposition}
Suppose $\vb{x},\vb{y},\vb{z}\in\RR^n$, $\alpha\in\RR$. Then
\begin{enumerate}[label=(\roman*)]
\item $\norm{\vb{x}}\ge0$
\item $\norm{\vb{x}}=0$ if and only if $\vb{x}=\vb{0}$
\item $\norm{\alpha\vb{x}}=|\alpha|\norm{\vb{x}}$
\item $\norm{\vb{x}\cdot\vb{y}}\le\norm{\vb{x}}\norm{\vb{y}}$
\item $\norm{\vb{x}+\vb{y}}\le\norm{\vb{x}}+\norm{\vb{y}}$
\item $\norm{\vb{x}-\vb{z}}\le\norm{\vb{x}-\vb{y}}+\norm{\vb{y}-\vb{z}}$
\end{enumerate}
\end{proposition}

\begin{proof} \
\begin{enumerate}[label=(\roman*)]
\item Obvious from definition.
\item \begin{align*}
\norm{\vb{x}}=0&\iff\brac{\sum_{i=1}^n{x_i}^2}^\frac{1}{2}=0\\
&\iff\sum_{i=1}^n{x_i}^2=0\\
&\iff x_1=\cdots=x_n=0\\
&\iff\vb{x}=(0,\dots,0)=\vb{0}
\end{align*}
since ${x_i}^2\ge0$.

\item Obvious from definition.
\item This is an immediate consequence of the Cauchy--Schwarz inequality.
\item By (iv) we have
\begin{align*}
\norm{\vb{x}+\vb{y}}&=(\vb{x}+\vb{y})\cdot(\vb{x}+\vb{y})\\
&=\vb{x}\cdot\vb{x}+2\vb{x}\cdot\vb{y}+\vb{y}\cdot\vb{y}\\
&\le\norm{\vb{x}}^2+2\norm{\vb{x}}\norm{\vb{y}}+\norm{\vb{y}}^2\\
&=\brac{\norm{\vb{x}}+\norm{\vb{y}}}^2.
\end{align*}
\item This follows directly from (v) by replacing $\vb{x}$ by $\vb{x}-\vb{y}$ and $\vb{y}$ by $\vb{y}-\vb{z}$.
\end{enumerate}
\end{proof}

\begin{comment}
\begin{definition}
The \vocab{distance between sets} $E\subset\RR^n$ and $F\subset\RR^n$ is defined as
\[ d(E,F)\coloneqq\inf_{x\in E,y\in F}\norm{x-y}. \]
\end{definition}

Obviously $d(E,F)>0$ implies that $E$ and $F$ are disjoint, but $E$ and $F$ may still be disjoint even if $d(E,F)=0$. For example, the closed intervals $E=(-1,0)$ and $F=(0,1)$.

\begin{exercise}
Suppose that $E$ and $F$ are sets in $\RR^n$ where $E$ and $F$ is finite. Prove that $E$ and $F$ are disjoint if and only if $d(E,F)>0$.
\end{exercise}
\end{comment}
\pagebreak

\section*{Exercises}
\begin{prbm}[\cite{rudin} Ch.1 Q1]
If $r\in\QQ\setminus\{0\}$ and $x\in\RR\setminus\QQ$, prove that $r+x\in\RR\setminus\QQ$ and $rx\in\RR\setminus\QQ$.
\end{prbm}

\begin{solution}
We prove by contradiction. Suppose $r+x$ is rational, then $r+x=\dfrac{m}{n},m,n\in\ZZ$, and $m,n$ have no common factors. Then $m=n(r+x)$. Let $r=\frac{p}{q},p,q\in\ZZ$, the former equation implies that $m=n\brac{\frac{p}{q}+x}$, i.e., $qm=n(p+qx)$, giving
\[x=\frac{mq-np}{nq},\]
which says that $x$ can be written as the quotient of two integers, so $x$ is rational, a contradiction.

The proof for the case $rx$ is similar.
\end{solution}

\begin{prbm}[\cite{rudin} Ch.1 Q4]
Let $E$ be a nonempty subset of an ordered set; suppose $\alpha$ is a lower bound of $E$ and $\beta$ is an upper bound of $E$. Prove that $\alpha\le\beta$.
\end{prbm}

\begin{solution}
Let $x\in E$. By definition of lower and upper bounds, $\alpha\le x\le\beta$.
\end{solution}

\begin{prbm}[\cite{rudin} Ch.1 Q5]
Let $A\subset\RR$, $A\neq\emptyset$ be bounded below. Let $-A$ be the set of all numbers $-x$, where $x\in A$. Prove that
\[\inf A=-\sup(-A).\]
\end{prbm}

\begin{solution}
Let $\alpha=\inf A$. If $x\in(-A)$ then $-x\in A$, so $\alpha\le-x$, and so $-\alpha\le x$. This implies that $-\alpha$ is an upper bound for $-A$.

If $\beta<-\alpha$ then $-\beta>\alpha$, and there exists $x\in A$ such that $x<-\beta$. Then $-x\in(-A)$, and $-x>\beta$. This shows that $-\alpha=\sup(-A)$, and we are done.
\end{solution}

\begin{prbm}[\cite{rudin} Ch.1 Q8]
Proe that no order can be defined in $\CC$ that turns it into an ordered field.
\end{prbm}

\begin{solution}
By Proposition 1.18d, an ordering $<$ that makes $\CC$ an ordered field would have to satisfy $-1=i^2>0$, contradicting $1>0$.
\end{solution}

\begin{prbm}[\cite{rudin} Ch.1 Q9, lexicographic order]
Suppose $z=a+bi$, $w=c+di$. Define an order on $\CC$ as follows:
\[z<w\iff\begin{cases}
a<c,\text{ or}\\
a=c,b<d.
\end{cases}\]
Prove that this turns $\CC$ into an ordered set. Does this ordered set have the least upper bound property?
\end{prbm}

\begin{prbm}[\cite{rudin} Ch.1 Q10]
Suppose $z=a+bi$, $w=u+iv$, and
\[a=\brac{\frac{|w|+u}{2}}^\frac{1}{2},\quad b=\brac{\frac{|w|-u}{2}}^\frac{1}{2}.\]
Prove that $z^2=w$ if $v\ge0$ and that $\overline{z}^2=w$ if $v\le0$. Conclude that every complex number (with one exception!) has two complex square roots.
\end{prbm}

\begin{solution}
We have
\[a^2-b^2=\frac{|w|+u}{2}-\frac{|w|-u}{2}=u,\]
and
\[2ab=\brac{|w|+u}^\frac{1}{2}\brac{|w|-u}^\frac{1}{2}=\brac{|w|^2-u^2}^\frac{1}{2}=\brac{v^2}^\frac{1}{2}=|v|.\]
Hence
\[z^2=\brac{a^2-b^2}+2abi=u+|v|i=w\]
if $v\ge0$, and
\[\overline{z}^2=\brac{a^2-b^2}-2abi=u-|v|i=w\]
if $v\le0$. Hence every non-zero $w$ has two square roots $\pm z$ or $\pm\overline{z}$. Of course, $0$ has only one square root, itself.
\end{solution}

\begin{prbm}[\cite{rudin} Ch.1 Q11]
If $z\in\CC$, prove that there exists $r\ge0$ and $w\in\CC$ with $|w|=1$ such that $z=rw$. Are $w$ and $r$ always uniquely determined by $z$?
\end{prbm}

\begin{prbm}[\cite{rudin} Ch.1 Q12, triangle inequality]
If $z_1,\dots,z_n\in\CC$, prove that
\[|z_1+\cdots+z_n|\le|z_1|+\cdots+|z_n|.\]
\end{prbm}

\begin{solution}
By the triangle inequality, $|z_1+z_2|\le|z_1|+|z_2|$. Assume the statement holds for $n-1$. Then
\[|z_1+\cdots+z_{n-1}+z_n|\le|z_1+\cdots+z_{n-1}|+|z_n|\le|z_1|+\cdots+|z_n|,\]
which establishes the claim by induction.
\end{solution}

\begin{prbm}[\cite{rudin} Ch.1 Q13]
If $x,y\in\CC$, prove that
\[\absolute{|x|-|y|}\le|x-y|.\]
\end{prbm}

\begin{solution}
By the triangle inequality,
\[|x|=|(x-y)+y|\le|x-y|+|y|\]
so that
\[|x|-|y|\le|x-y|.\]
Interchanging the roles of $x$ and $y$ in the above, we also have
\[|y|-|x|\le|x-y|\]
so that
\[\absolute{|x|-|y|}\le|x-y|.\]
\end{solution}
    \chapter{Basic Topology}\label{chap:basic-topology}
This chapter discusses basic notions of point set topology, which focuses on the metric space and its related structures. Then we introduce compactness and prove three major results (\cref{thrm:cantor-intersection}, \cref{thrm:heine-borel}, \cref{thrm:bolzano-weierstrass}). We also briefly talk about perfect sets, and connectedness of sets.

\section{Metric Space}
\subsection{Definitions and Examples}
\begin{definition}[Metric space]
A \vocab{metric space}\index{metric space} is a set $X$ with an associated \emph{metric} $d:X\times X\to\RR$, which satisfies the following properties for all $p,q\in X$:
\begin{enumerate}[label=(\roman*)]
\item Positive definitiveness: $d(p,q)\ge0$, where equality holds if and only if $p=q$;
\item Symmetry: $d(p,q)=d(q,p)$;
\item Triangle inequality: $d(p,q)\le d(p,r)+d(r,q)$ for any $r\in X$.
\end{enumerate}
\end{definition}

For the rest of the chapter, $X$ is taken to be a metric space, unless specified otherwise.

\begin{example}[Metrics on $\RR^n$]
Each of the following functions define metrics on $\RR^n$.
\begin{align*}
d_1(x,y)&=\sum_{i=1}^{n}|x_i-y_i|;\\
d_2(x,y)&=\sqrt{\sum_{i=1}^{n}(x_i-y_i)}\\
d_\infty(x,y)&=\max_{i\in\{1,2,\dots,n\}}|x_i-y_i|.
\end{align*}
These are called the $\ell^1$-, $\ell^2$- (or Euclidean) and $\ell^\infty$-distances respectively.

The proof that each of $d_1$, $d_2$, $d_\infty$ is a metric is mostly very routine, with the exception of proving that $d_2$, the Euclidean distance, satisfies the triangle inequality. To establish this, recall that the Euclidean norm $\norm{x}_2$ of a vector $x=(x_1,\dots,x_n)\in\RR^n$ is
\[\norm{x}_2\coloneqq\brac{\sum_{i=1}^n{x_i}^2}^\frac{1}{2}=\langle x,x\rangle^\frac{1}{2},\]
where the inner product is given by
\[\langle x,y\rangle\coloneqq\sum_{i=1}^{n}x_i y_i.\]
Then $d_2(x,y)=\norm{x-y}_2$, and so the triangle inequality is the statement that
\[\norm{w-y}_2\le\norm{w-x}_2+\norm{x-y}_2.\]
This follows immediately by taking $u=w-x$ and $v=x-y$ in the following lemma.

\begin{lemma}
If $u,v\in\RR^n$ then $\norm{u+v}_2\le\norm{u}_2+\norm{v}_2$.
\end{lemma}

\begin{proof}
Since $\norm{u}_2\ge0$ for all $u\in\RR^n$, squaring both sides of the desired inequality gives
\[{\norm{u+v}_2}^2\le{\norm{u}_2}^2+2\norm{u}_2\norm{v}_2+{\norm{v}_2}^2.\]
But since
\[{\norm{u+v}_2}^2=\langle u+v,u+v\rangle={\norm{u}_2}^2+2\langle u,v\rangle+\norm{v}_2^2,\]
this inequality is immediate from the Cauchy--Schwarz inequality, that is to say the inequality
\[|\langle u,v\rangle|\le\norm{u}_2\norm{v}_2.\]
\end{proof}
\end{example}

The following are a few interesting examples of metrics.

\begin{example}[Discrete metric]
The \textbf{discrete metric} on an arbitrary set $X$ is defined as follows:
\[d(x,y)=\begin{cases}
1&\text{if }x\neq y,\\
0&\text{if }x=y.
\end{cases}\]
\end{example}

\begin{example}[2-adic metric]
On $\ZZ$, define $d(x,y)$ to be $2^{-m}$, where $2^m$ is the largest power of two dividing $x-y$. The triangle inequality holds in the following stronger form, known as the ultrametric property:
\[d(x,z)\le\max\{d(x,y),d(y,z)\}.\]
Indeed, this is just a rephrasing of the statement that if $2^m$ divides both $x-y$ and $y-z$, then $2^m$ divides $x-z$.

This metric is very unlike the usual distance. For example, $d(999,1000) = 1$, whilst $d(0,1000)=\frac{1}{8}$.

The role of $2$ can be replaced by any other prime $p$, and the metric may also be
extended in a natural way to the rationals $\QQ$.
\end{example}

\begin{example}[Path metric]
Let $G$ be a graph, that is to say a finite set of vertices $V$ joined by edges. Suppose that $G$ is connected, that is to say that there is a path joining any pair of distinct vertices. Define a distance $d$ as follows: $d(v,v)=0$, and $d(v,w)$ is the length of the shortest path from $v$ to $w$. Then $d$ is a metric on $V$, as can be easily checked.
\end{example}

\begin{example}[Word metric]
Let $G$ be a group, and suppose that it is generated by elements $a$, $b$ and their inverses. Define a distance on $G$ as follows: $d(v,w)$ is the minimal $k$ such that $v=wg_1\cdots g_k$, where $g_i\in\{a,b,a^{-1},b^{-1}\}$ for all $i$.
\end{example}

\begin{example}[Hamming distance]
Let $X=\{0,1\}^n$ (the boolean cube), the set of all strings of $n$ zeroes and ones. Define $d(x,y)$ to be the number of coordinates
in which $x$ and $y$ differ.
\end{example}

\begin{example}[Projective space]
Consider the set $P(\RR^n)$ of one-dimensional subspaces of $\RR^n$, that is to say lines through the origin. One way to define a distance on this set is to take, for lines $L_1,L_2$, the distance between $L_1$ and $L_2$ to be
\[d(L_1,L_2)=\sqrt{1-\frac{|\langle v,w\rangle|^2}{\norm{v}^2\norm{w}^2}},\]
where $v$ and $w$ are any non-zero vectors in $L_1$ and $L_2$ respectively.

When $n=2$, the distance between two lines is $\sin\theta$ where $\theta$ is the angle between those lines.
\end{example}

\begin{comment}
\subsection{Norms}
\begin{definition}[Norms]
Let $V$ be any vector space (over the reals). A function $\norm{\cdot}:V\to[0,\infty)$ is called a \vocab{norm}\index{norm} if it satisfies the following properties:
\begin{enumerate}[label=(\roman*)]
\item $\norm{x}=0$ if and only if $x=0$;
\item $\norm{\lambda x}=|\lambda|\norm{x}$ for all $\lambda\in\RR$, $x\in V$;
\item $\norm{x+y}\le\norm{x}+\norm{y}$ for all $x,y\in V$.
\end{enumerate}
\end{definition}

Given a norm, it is very easy to check that $d(x,y)\coloneqq\norm{x-y}$ defines a metric on $V$. Indeed, we have already seen that when $V=\RR^n$, $\norm{\cdot}_2$ is a norm (and so the name ``Euclidean norm'' is appropriate) and we defined $d_2(x,y)=\norm{x-y}_2$. The other metrics on $\RR^n$ also come from norms: $d_1$ comes from the $\ell^1$-norm
\[\norm{x}_1\coloneqq\sum_{i=1}^n|x_i|,\]
whilst $d_\infty$ comes from the $\ell^\infty$-norm
\[\norm{x}_\infty\coloneqq\max_{i=1,\dots,n}|x_i|.\]
More generally, the family of $\ell^p$-norms are given by
\[\norm{x}_p\coloneqq\brac{\sum_{i=1}^n|x_i|^p}^\frac{1}{p}.\]

The principle of turning norms into metrics is important enough that we state it as a lemma in its own right.

\begin{lemma}
Let $V$ be a vector space over $\RR$, let $\norm{\cdot}$ be a norm on $V$. Define $d:V\times V\to[0,\infty)$ by $d(x,y)\coloneqq\norm{x-y}$. Then $(V,d)$ is a metric space.
\end{lemma}

\begin{proof}
Simply verify the three axioms for a metric space, which directly correspond to the three axioms for a norm.
\end{proof}

\begin{remark}
The converse is very far from true. For instance, the discrete metric does not arise from a norm. All metrics arising from a norm have the \emph{translation invariance property} $d(x+z,y+z)=d(x,y)$, as well as the \emph{scalar invariance} $d(\lambda x,\lambda y)=|\lambda|d(x,y)$, neither of which are properties of arbitrary metrics.

Conversely one can show that a metric with these two additional properties does come from a norm, an exercise we leave to the reader. [Hint: the norm must arise as $\norm{v}=d(v,0)$.]
\end{remark}

We call a vector space endowed with a norm $\norm{\cdot}$ a \vocab{normed space}. Whenever we talk about normed spaces it is understood that we are also thinking of them as metric spaces, with the metric being defined by $d(v,w)=\norm{v-w}$.

Note that we do not assume that the underlying vector space $V$ is finite dimensional. Here are some examples which are not finite-dimensional (whilst we do not prove that they are not finite-dimensional here, it is not hard to do so and we suggest this as an exercise).

\begin{example}[$\ell^p$ spaces]
Let
\begin{align*}
\ell_1&=\crbrac{(x_n)_{n=1}^\infty\:\bigg|\:\sum_{n\ge1}|x_n|<\infty},\\
\ell_2&=\crbrac{(x_n)_{n=1}^\infty\:\bigg|\:\sum_{n\ge1}x_n^2<\infty},\\
\ell_\infty=&\crbrac{(x_n)_{n=1}^\infty\:\bigg|\:\sup_{n\in\NN}|x_n|<\infty}.
\end{align*}
The sets $\ell_1$, $\ell_2$, $\ell_\infty$ are all real vector spaces, and moreover
\begin{align*}
\norm{(x_n)}_1&=\sum_{n\ge1}|x_n|,\\
\norm{(x_n)}_2&=\brac{\sum_{n\ge1}x_n^2}^\frac{1}{2},\\
\norm{(x_n)}_\infty&=\sup_{n\in\NN}|x_n|
\end{align*}
define norms on $\ell_1$, $\ell_2$ and $\ell_\infty$ respectively.

Note that $\ell_2$ is in fact an inner product space where
\[\langle(x_n),(y_n)\rangle=\sum_{n\ge1}x_ny_n,\]
(the fact that the right-hand side converges if $(x_n)$ and $(y_n)$ are in $\ell_2$ follows from the Cauchy--Schwarz inequality). The space $\ell^2$ is known as \vocab{Hilbert space}.
\end{example}
\end{comment}

A metric space $(X,d)$ naturally induces a metric on any of its subsets.

\begin{definition}[Subspace]
Suppose $(X,d)$ is a metric space, $Y\subset X$. Then the restriction of $d$ to $Y\times Y$ gives $Y$ a metric so that $(Y,d_{Y\times Y})$ is a metric space. We call $Y$ equipped with this metric a \vocab{subspace}.
\end{definition}

\begin{example}
Subspaces of $\RR$ include $[0,1]$, $\QQ$, $\ZZ$.
\end{example}

\begin{proposition}[Product space]
If $(X,d_X)$ and $(Y,d_Y)$ are metric spaces, set
\[d_{X\times Y}\brac{(x_1,y_1),(x_2,y_2)}=\sqrt{d_X(x_1,x_2)^2+d_Y(y_1,y_2)^2.}\]
for $x_1,x_2\in X$, $y_1,y_2\in Y$. Then $d_{X\times Y}$ gives a metric on $X\times Y$; we call $X\times Y$ the \emph{product space}.
\end{proposition}

\begin{proof}
Reflexivity and symmetry are obvious. Less clear is the triangle inequality. We need to prove that
\begin{equation*}\tag{1}
\begin{split}
&\sqrt{d_X(x_1,x_3)^2+d_Y(y_1,y_3)^2}+\sqrt{d_X(x_3,x_2)^2+d_Y(y_3,y_2)^2}\\
&\ge\sqrt{d_X(x_1,x_2)^2+d_Y(y_1,y_2)^2}
\end{split}
\end{equation*}
Write $a_1=d_X(x_2,x_3)$, $a_2=d_X(x_1,x_3)$, $a_3=d_X(x_1,x_2)$ and similarly $b_1=d_Y(y_2,y_3)$, $b_2=d_Y(y_1,y_3)$ and $b_3=d_Y(y_1,y_2)$. Thus we want to show
\begin{equation*}\tag{2}
\sqrt{{a_2}^2+{b_2}^2}+\sqrt{{a_1}^2+{b_1}^2}\ge\sqrt{{a_3}^2+{b_3}^2}.
\end{equation*}
To prove this, note that from the triangle inequality we have $a_1+a_2\ge a_3$, $b_1+b_2\ge b_3$. Squaring and adding gives
\[{a_1}^2+{b_1}^2+{a_2}^2+{b_2}^2+2(a_1a_2+b_1b_2)\ge {a_3}^2+{b_3}^2.\]
By Cauchy--Schwarz,
\[a_1a_2+b_1b_2\le\sqrt{{a_1}^2+{b_1}^2}\sqrt{{a_2}^2+{b_2}^2}.\]
Substituting this into the previous line gives precisely the square of (2), and (1) follows.
\end{proof}

\subsection{Balls and Boundedness}
\begin{definition}[Balls]\index{balls}
The \vocab{open ball}\index{balls!open ball} centred at $x\in X$ with radius $r>0$ is defined to be the set
\[B_r(x)\coloneqq\{y\in X\mid d(x,y)<r\}.\]
Similarly the \vocab{closed ball}\index{balls!closed ball} centred at $x$ with radius $r$ is
\[\overline{B}_r(x)\coloneqq\{y\in X\mid d(x,y)\le r\}.\]
The \vocab{punctured ball}\index{balls!punctured ball} is the open ball excluding its centre:
\[B_r(x)\setminus\{x\}=\{y\in X\mid 0<d(x,y)<r\}.\]
\end{definition}

\begin{example}
Considering $\RR^3$ with the Euclidean metric, $B_1(0)$ really is what we understand geometrically as a ball (minus its boundary, the unit sphere), whilst $\overline{B}_1(0)$ contains the unit sphere and everything inside it.
\end{example}

\begin{remark}
We caution that this intuitive picture of the closed ball being the open ball ``together with its boundary'' is totally misleading in general. For instance, in the discrete metric on a set $X$, the open ball $B_1(a)$ contains only the point $a$, whereas the closed ball $\overline{B}_1(a)$ is the whole of $X$.
\end{remark}

\begin{definition}[Bounded]
$E\subset X$ is said to be \vocab{bounded}\index{boundedness} if $E$ is contained in some open ball; that is, there exists $M\in\RR$ and $q\in X$ such that $d(p,q)<M$ for all $p\in E$.
\end{definition}

\begin{proposition}
Let $E\subset X$. Then the following are equivalent:
\begin{enumerate}[label=(\roman*)]
\item $E$ is bounded;
\item $E$ is contained in some closed ball;
\item The set $\{d(x,y)\mid x,y\in E\}$ is a bounded subset of $\RR$.
\end{enumerate}
\end{proposition}

\begin{proof} \

(1)$\implies$(2) This is obvious.

(2)$\implies$(3) This follows immediately from the triangle inequality.

(3)$\implies$(1) Suppose $E$ satisfies (iii), then there exists $r\in\RR$ such that $d(x,y)\le r$ for all $x,y\in E$. If $E=\emptyset$, then $E$ is certainly bounded. Otherwise, let $p\in E$ be an arbitrary point. Then $E\subset B_{r+1}(p)$.
\end{proof}

\subsection{Open and Closed Sets}
\begin{definition}[Neighbourhood]
$N\subset X$ is called a \vocab{neighbourhood}\index{neighbourhood} of $p\in X$ if $B_\delta(p)\subset N$ for some $\delta>0$.
\end{definition}

\begin{definition}[Open set]
$E\subset X$ is \vocab{open}\index{open set} (in $X$) if it is a neighbourhood of each of its elements; that is, for all $x\in E$, $B_\delta(x)\subset E$ for some $\delta>0$.
\end{definition}

\begin{proposition}
Any open ball is open.
\end{proposition}

\begin{proof}
Let $B_r(x)$ be an open ball. Then for any point $y\in B_r(x)$, there is $d(y,x)<r$. Take $\delta=r-d(y,x)$, which is positive.

Consider the ball $B_\delta(y)$. We shall show it lives in $B_r(x)$. For this, take any point $z\in B_\delta(y)$. By the triangle inequality, we have
\begin{align*}
d(z,x)&\le d(z,y)+d(y,x)\\
&<\delta+d(y,x)\\
&=r.
\end{align*}
and so $z\in B_r(x)$. Since for all $y\in B_r(x)$ there exists $\delta>0$ such that $B_\delta(y)\subset B_r(x)$, we have that $B_r(x)$ is open.
\end{proof}

\begin{proposition}\label{prop:open-set-properties}
\begin{enumerate}[label=(\roman*)]
\item Both $\emptyset$ and $X$ are open.
\item For any indexing set $I$ and collection of open sets $\{E_i\mid i\in I\}$, $\bigcup_{i\in I}E_i$ is open.
\item For any \emph{finite} indexing set $I$ and collection of open sets $\{E_i\mid i\in I\}$, $\bigcap_{i\in I}E_i$ is open.
\end{enumerate}
\end{proposition}

\begin{proof} \
\begin{enumerate}[label=(\roman*)]
\item Obvious by definition.
\item If $ x\in\bigcup_{i\in I}E_i$ then there is some $i\in I$ with $x\in E_i$. Since $E_i$ is open, there exists $\delta>0$ such that $B_\delta(x)\subset E_i$ and hence $ B_\delta(x)\in\bigcup_{i\in I}E_i$.
\item Suppose that $I$ is finite and that $ x\in\bigcap_{i\in I}E_i$. For each $i\in I$, we have $x\in E_i$ and so there exists $\delta_i$ such that $B_{\delta_i}(x)\subset E_i$. Set $\delta=\min_{i\in I}\delta_i$, then $\delta>0$ (here it is, of course, crucial that $I$ be finite), and $B_\delta(x)\subset B_{\delta_i}(x)\subset E_i$ for all $i$. Therefore $ B_\delta(x)\subset\bigcap_{i\in I}E_i$.
\end{enumerate}
\end{proof}

\begin{remark}
(1) is in fact a special case of (2) and (3), taking $I$ to be the empty set.
\end{remark}

\begin{remark}
It is extremely important to note that, whilst the indexing set $I$ in (2) can be arbitrary, the indexing set in (3) must be finite. In general, an arbitrary intersection of open sets is not open; for instance, the intervals $E_i=\brac{-\frac{1}{i},\frac{1}{i}}$ are all open in $\RR$, but their intersection $\bigcap_{i=1}^\infty E_i=\{0\}$, which is not an open set.
\end{remark}

\begin{proposition}\label{prop:open-subspace-cap}
Suppose $Y$ is a subspace of $X$. $E\subset Y$ is open relative to $Y$ if and only if $E=Y\cap G$ for some open subset $G$ of $X$.
\end{proposition}

\begin{proof} \

\fbox{$\implies$} Suppose $E$ is open relative to $Y$. Then for each $p\in E$ there exists $r_p>0$ such that the conditions $d(p,q)<r_p$, $q\in Y$ imply $q\in E$.

For each $p\in E$, let the open ball
\[V_p=\{q\in X\mid d(p,q)<r_p\},\]
and define
\[G=\bigcup_{p\in E}V_p.\]
Since $G$ is an intersection of open balls and open balls are open sets, by \cref{prop:open-set-properties}, $G$ is an open subset of $X$. Since $p\in V_p$ for all $p\in E$, it is clear that $E\subset G\cap Y$.

To show the opposite containment, by our choice of $V_p$, we have $V_p\cap Y\subset E$ for every $p\in E$, so that $G\cap Y\subset E$. Hence $E=G\cap Y$.

\fbox{$\impliedby$} Conversely, if $G$ is open in $X$ and $E=G\cap Y$, every $p\in E$ has a neighbourhood $V_p\cap Y\subset E$. Hence $E$ is open relative to $Y$.
\end{proof}

The complement of an open set is a closed set.

\begin{definition}[Closed set]
$E\subset X$ is \vocab{closed}\index{closed set} if its complement $E^c=X\setminus E$ is open.
\end{definition}

%\begin{example}
%The closed interval $[a,b]$, $a\le b$ is closed in $\RR$.
%\end{example}

\begin{proposition}
Any closed ball is closed.
\end{proposition}

\begin{proof}
To prove that $\overline{B}_r(x)=\{y\in X\mid d(x,y)\le r\}$ is closed, we need to show that its complement $\overline{B}_r(x)^c=\{y\in X\mid d(x,y)>r\}$ is open. To do so, we need to show that for all $z\in\overline{B}_r(x)^c$, there exists $\delta>0$ such that $B_\delta(z)\subset\overline{B}_r(x)^c$.

Take $\delta>0$ such that $r+\delta<d(x,z)$; that is, $\delta<d(x,z)-r$.

Pick $y\in B_\delta(z)$. Then $d(y,z)<\delta$. But $r+d(y,z)<d(x,z)$ so $r<d(x,z)-d(y,z)\le d(x,y)$ by triangle inequality. Hence we have $r<d(x,y)$, thus $y\in\overline{B}_r(x)^c$ and so $B_\delta(z)\subset\overline{B}_r(x)^c$. Therefore $\overline{B}_r(x)^c$ is open, so $\overline{B}_r(x)$ is closed.
\end{proof}

\begin{proposition}\label{prop:closed-set-properties}
\begin{enumerate}[label=(\roman*)]
\item Both $\emptyset$ and $X$ are closed.
\item For any indexing set $I$ and collection of closed sets $\{F_i\mid i\in I\}$, $\bigcap_{i\in I}F_i$ is closed.
\item For any \emph{finite} indexing set $I$ and collection of closed sets $\{F_i\mid i\in I\}$, $\bigcup_{i\in I}F_i$ is closed.
\end{enumerate}
\end{proposition}

\begin{proof}
From \cref{prop:open-set-properties}, simply take complements and apply de Morgan's laws.
\end{proof}

\begin{remark}
As above, the indexing set in (3) must be finite; for instane, the closed intervals $F_i=\sqbrac{-1+\frac{1}{n},1-\frac{1}{n}}$ are all closed in $\RR$, but their union $\bigcup_{i=1}^\infty F_i=(-1,1)$ is open, not closed.
\end{remark}

\subsection{Interiors, Closures, Limit Points}
\begin{definition}
The \vocab{interior}\index{interior} of $E\subset X$, denoted by $E^\circ$, is defined to be the union of all open subsets of $X$ contained in $E$.

The \vocab{closure}\index{closure} of $E$, denoted by $\overline{E}$, is defined to be the intersection of all closed subsets of $X$ containing $E$.

The set $\overline{E}\setminus E^\circ$ is known as the \vocab{boundary}\index{boundary} of $E$, denoted by $\partial E$. $p$ is a \vocab{boundary point}\index{boundary point} of $E$ if $p\in\partial E$.

A set $E\subset X$ is said to be \vocab{dense}\index{dense} if $\overline{E}=X$.
\end{definition}

Since an arbitrary union of open sets is open, $E^\circ$ is itself an open set, and it is clearly the unique largest open subset of $X$ contained in $E$. If $E$ is itself open then evidently $E=E^\circ$.

Since an arbitrary intersection of closed sets is closed, $\overline{E}$ is the unique smallest closed subset of $X$ containing $E$. If $E$ is itself closed then evidently $E=\overline{E}$.

If $x\in E^\circ$ we say that $x$ is an \vocab{interior point} of $E$. One can also phrase this in terms of neighbourhoods: the interior of $E$ is the set of all points in $E$ for which $E$ is a neighbourhood.

\begin{example}
Consider the closed interval $[a,b]$ in $\RR$; its interior is just the open interval $(a,b)$.

The rationals $\QQ$ are a dense subset of $\RR$.
\end{example}

Let us give a couple of simple characterisations of the closure of a set.

\begin{proposition}
Suppose $E\subset X$. $p\in\overline{E}$ if and only if every open ball $B_\delta(p)$ contains a point of $E$.
\end{proposition}

\begin{proof} \

\fbox{$\implies$} Suppose that $p\in\overline{E}$. Suppose, for a contradiction, that there exists some open ball $B_\delta(p)$ that does not meet $E$, then $B_\delta(p)^c$ is a closed set containing $E$. Therefore $B_\delta(p)^c$ contains $\overline{E}$, and hence it contains $p$, which is obviously nonsense.

\fbox{$\impliedby$} Suppose that every ball $B_\delta(p)$ meets $E$. Suppose, for a contradiction, that $p\notin\overline{E}$. Then since $\overline{E}^c$ is open, there is a ball $B_\delta(p)$ contained in $\overline{E}^c$, and hence in $E^c$, contrary to assumption.
\end{proof}

\begin{remark}
A particular consequence of this is that $E\subset X$ is dense if and only if it meets every open set in $X$.
\end{remark}

We now introduce the notion of limit points.

\begin{definition}[Limit point]
$p\in E$ is a \vocab{limit point}\index{limit point} (or \emph{accumulation point}) of $E$ if every neighbourhood of $p$ contains some $q\neq p$ such that $q\in E$.

The \vocab{induced set}\index{induced set} of $E$, denoted by $E^\prime$, is the set of all limit points of $E$ in $X$.
\end{definition}

\begin{example}
Consider the metric space $\RR$, $a$ and $b$ are limit points $(a,b]$. The limit point set of $(a,b]$ is $[a,b]$, which is also the closure $(a,b]$.

Consider the metric space $\RR^2$. The limit point set of any open ball $B_r(x)$ is the closed ball $\bar{B}_r(x)$, which is also the closure of $B_r(x)$.

Consider $\QQ\subset\RR$. $\QQ^\prime=\bar{\QQ}=\RR$.
\end{example}

Note that we do not necessarily have $E\subset E^\prime$, that is to say it is quite possible for a point $p\in E$ not to be a limit point of $E$. This occurs if there exists some ball $B_\delta(p)$ such that $B_\delta(p)\cap E=\{p\}$; in this case we say that $p$ is an \vocab{isolated point} of $E$.

\begin{proposition}
If $p$ is a limit point of $E$, then every ball of $p$ contains infinitely many points of $E$.
\end{proposition}

\begin{proof}
Prove by contradiction. Suppose there exists $B_r(p)$ which contains only a finite number of points of $E$: $q_1,\dots,q_n$, where $q_i\neq p$ for $i=1,\dots,n$. Define
\[r=\min\{d(p,q_1),\dots,d(p,q_n)\}.\]
The minimum of a finite set of positive numbers is clearly positive, so that $r>0$.

$B_r(p)$ contains no point $q\in E,q\neq p$ so that $p$ is not a limit point of $E$, a contradiction.
\end{proof}

\begin{corollary}
A finite point set has no limit points.
\end{corollary}

\begin{proposition}
Suppose $E\subset X$. $E^\prime$ is a closed subset of $X$.
\end{proposition}

\begin{proof}
To prove that $E^\prime$ is closed, we need to show that the complement $(E^\prime)^c$ is open.

Suppose $p\in (E^\prime)^c$. Then exists a ball $B_\epsilon(p)$ whose intersection with $E$ is either empty or $\{p\}$. We claim that $B_\frac{\epsilon}{2}(p)\subset (E^\prime)^c$. Let $q\in B_\frac{\epsilon}{2}(p)$. If $q=p$, then clearly $q\in (E^\prime)^c$. If $q\neq p$, there is some ball about $q$ which is contained in $B_\epsilon(p)$, but does not contain $p$: the ball $B_\delta(q)$ where $\delta=\min\brac{\frac{\epsilon}{2},d(p,q)}$ has this property. This ball meets $E$ in the empty set, and so $q\in (E^\prime)^c$ in this case too.
\end{proof}

\begin{proposition}
Suppose $E\subset X$. Then $\overline{E}=E\cup E^\prime$.
\end{proposition}

\begin{proof}
We first show the containment $E\cup E^\prime\subset\overline{E}$. Obviously $E\subset\overline{E}$, so we need only show that $E^\prime\subset\overline{E}$. Suppose $p\in\overline{E}^c$. Since $\overline{E}^c$ is open, there is some ball $B_\epsilon(p)$ which lies in $\overline{E}^c$, and hence also in $E^c$, and therefore a cannot be a limit point of $E$. This concludes the proof of this direction.

Now we look at the opposite containment $\overline{E}\subset E\cup E^\prime$. If $p\in\overline{E}$, we saw in Lemma 5.1.5 that there is a sequence  $(x_n)$ of elements of $E$ with $x_n\to p$. If $x_n=p$ for some $n$ then we are done, since this implies that $p\in E$. Suppose, then, that $x_n\neq p$ for all $n$. Let $\epsilon>0$ be given, for sufficiently large $n$, all the $x_n$ are elements of $B_\epsilon(p)\setminus\{p\}$, and they all lie in $E$. It follows that $p$ is a limit point of $E$, and so we are done in this case also.
\end{proof}

\begin{proposition}
Suppose $E\subsetneq\RR$, $E\neq\emptyset$ be bounded above. Let $y=\sup E$. Then $y\in\overline{E}$. Hence $y\in E$ if $E$ is closed.
\end{proposition}

\begin{proof}
If $y\in E$ then $y\in\overline{E}$. Assume $y\notin E$. For every $h>0$ there exists then a point $x\in E$ such that $y-h<x<y$, for otherwise $y-h$ would be an upper bound of $E$. Thus $y$ is a limit point of $E$. Hence $y\in\overline{E}$.
\end{proof}
\pagebreak

%%%%%%%%%%%%%%%%%%%%%%%%%%%%%%

\begin{comment}
\item A point $x$ is an \vocab{exterior point} of $A$ if it is an interior point of $A^c$.
\item $E$ is compact if it is a bounded closed set.

\begin{proposition}
The set of exterior points, $(A^c)^\circ$ is the same as $(\bar{A})^c$.
\end{proposition}

\begin{proof}
\begin{align*}
x \in (A^c)^\circ 
&\iff \exists \epsilon>0 \text{ such that } B(x,\epsilon) \subset A^c \\
&\iff B(x,\epsilon) \cap A = \emptyset \\
&\iff x \notin A \text{ and } B_0(x,\epsilon) \cap A=\emptyset \\
&\iff x \notin A \cup A^\prime = \bar A \\
&\iff x \in (\bar A^c)
\end{align*}
\end{proof}

\begin{proposition}
\begin{enumerate}[label=(\roman*)]
\item $A^\prime$ is closed.
\item $\bar{A}$ is closed, i.e. $\bar{\bar{A}}=\bar{A}$
\end{enumerate}
\end{proposition}

\begin{proof} \
\begin{enumerate}[label=(\roman*)]
\item In order to show that $A^\prime$ is closed, we need to show that if $x$ is a limit point of $A^\prime$, then $x\in A^\prime$, i.e. $x$ is a limit point of $A$.

So we need to show that limit points of $A^\prime$ are always limit points of $A$: 
Let $x$ be a limit point of $A^\prime$, then for all $\epsilon>0$, $B_0(x,\epsilon/2)$ intersects with $A^\prime$ and we may pick $y \in B_0(x,\epsilon/2)\cap A^\prime$

Now here's the tricky part
Since $y \in A^\prime$, y is a limit point of $A$, hence $B_0(y,|y-x|)$ intersects with $A$ and thus we may pick $z \in B_0(y,|y-x|)\cap A$.

We show that $z \in B_0(x,\epsilon)$:
\[ |z-x|\le|z-y|+|y-x|<2|y-x|<\epsilon, \]
hence $z \in B(x,\epsilon)$.
\[ |z-y|<|x-y|, \]
hence $z \neq x$

$\therefore\:z \in B_0(x,\epsilon)$

\item 
\end{enumerate}
\end{proof}


%%%%%%%%%%%%%%%%%%%%%%%%

\begin{theorem}[Cantor's Intersection Theorem]
Given a decreasing sequence of compact sets $A_1\supset A_2 \supset \cdots$, there exists a point $x\in\RR^n$ such that $x$ belongs to all $A_i$. In other words, $\bigcap_{i=1}^\infty A_i\neq\emptyset$. Moreover, if for all $i\in\NN$ we have $\diam A_{i+1}\le c\cdot\diam A_k$ for some constant $c<1$, then such a point must be unique, i.e. $\bigcap_{i=1}^\infty A_k=\{x\}$ for some $x\in\RR^n$.
\end{theorem}

\begin{theorem}[Heine--Borel Theorem]
A set $A\subset\RR^n$ is compact if and only if every open covering has a finite subcover, i.e. for any family of open sets $\mathcal{U}=\{U_i\}_{i\in I}$ satisfying $A\subset\bigcup_{i\in I}U_i$, there exists $\{U_1,\dots,U_n\}\subset\mathcal{U}$ such that $A\subset\bigcup_{i=1}^n U_i$.
\end{theorem}

\begin{theorem}[Bolzano--Weierstrass Theorem]
Infinite bounded sets in $\RR^n$ must contain limit points.
\end{theorem}

We will follow a very specific sequence of steps to prove these three theorems:
\begin{enumerate}[label=(\alph*)]
\item Cantor Intersection for $n=1$
\item Bolzano--Weierstrass for $n=1$
\item Bolzano--Weierstrass for general $n$
\item Cantor Intersection for general $n$
\item Heine--Borel for general $n$
\end{enumerate}

\begin{proof} \
\begin{enumerate}[label=(\alph*)]
\item Suppose that there is a decreasing sequence of compact sets $A_1, A_2, \dots$ in the real numbers

Since $A_k$ are bounded, we may let $a_k=\inf A_k$
Also since $A_k$ are closed, $a_k \in A_k$

Note that since $A_k$ is a decreasing sequence of sets we have $a_1\le a_2\le\dots$

Also, whenever we have $n>k$, we have $a_n \in A_n$, but $A_n \subset A_k$ and thus $a_n \in A_k$.

Let $b_1=\sup A_1$, then $a_k \in A_1$ and thus $a_k\le b_1$ for all $k$.

This tells us that the sequence $\{a_k\}$ is bounded above, and thus we may let $a=\sup a_k$.

Our goal is to show that the number $a$ appears in all $A_k$, thus showing that the entire intersection $\bigcap A_k$ contains $a$ and thus must be non-empty.

Now we split this in two cases, which asks whether a is simply made from isolated points, or if it is actually some nontrivial point obtained from the boundaries of $A_k$

\textbf{Case 1:} $a_k=a$ for some $k$
In this case we see that $a_k\le a_n\le a$ for all $n>k$ and thus $a_n=a$ in this case, therefore a is an element in $A_n$ for all $n$

In this case you can imagine that there is a possibility where a is an isolated minimum point of $A_n$ which stays there forever in the decreasing sequence of sets

\textbf{Case 2:} $a_k<a$ for all $k$; in this case we see that $a$ is the limit point of the increasing sequence $\{a_k\}$

Exercise 1: Show that $a$ is a limit point of each $A_k$.

Note that $a_n$ is in $A_k$ for each $n>k$, and since $a=\sup\{a_k\}$ where $a_k$ is increasing, we can actually show that a is a limit point of $\{a_n \mid n \le k\}$:
For every $\epsilon>0$, we pick $n_0$ such that $0 < a-a_{n_0} < \epsilon$
Pick $n\prime > \max\{k,n_0\}$, then $a_{n^\prime} \ge a_{n_0}$ and so
\[ 0<a-a_n\prime \le a_{n_0} < \epsilon \]
This shows that there exists $a_n^\prime$ in $B_0(a,\epsilon) \cap \{a_n \mid n>k\}$ for all $\epsilon$, and so $a$ is a limit point of $\{a_n \mid n>k\}$.

Now since $\{a_n|n \ge k\}$ is a subset of $A_k$ we also see that a is a limit point of $A_k$
Finally, since $A_k$ is closed, we conclude that $a$ is in $A_k$ for all $k$, and we are done

Wait hold on, I forgot about the second part

Now we consider a decreasing sequence of compact sets $A_1, A_2, \dots$ such that $\diam A_{k+1} \le c \diam A_k$ for $c<1$.

Suppose otherwise that there exists $x, y$ in $\bigcap A_k$

You can imagine that this will form a fixed distance between two points, and thus there is a constant positive lower bound for the diameters:
\[ \diam A_k \ge |x-y| > 0 \forall k \]

But this cannot be true because $\diam A_{k+1} \le c \diam A_k$ and so the diameter is controlled by a decreasing geometric sequence:
\[ \diam A_{k+1} \le c^k \diam A_1 \]

So we can simply pick a natural number $k$ such that
\[ k > \log_c \frac{|x-y|}{\diam A_1} \]

\item We consider an infinite bounded set $A$ in the real numbers. Since $A$ is bounded, we can pick a closed interval $[a_1,b_1]$ containing $A$.

We then perform a series of binary cuts: Consider the two halves of $[a_1,b_1]$. We know that at least one of these two must contain infinitely many elements in $A$, otherwise $A$ cannot be infinite. We pick this half of the interval and denote it by $[a_2,b_2]$. We continue this to pick a decreasing sequence of closed intervals $[a_n,b_n]$.

Now $\diam [a_{n+1},b_{n+1}] = \frac{1}{2} \diam [a_n,b_n]$, so by the Cantor Intersection Theorem, there exists a unique real number $c$ in the intersection $\bigcap[a_n,b_n]$.

We show that this $c$ is in fact a limit point of $A$.

For any $\epsilon>0$, we need to show that $B_0(c,\epsilon) \cap A \neq \emptyset$, i.e. we need to find an element $x \neq c$ in $A$ that is less than $\epsilon$ apart from $c$.

We then realize that we can simply exploit the decreasing sequence $[a_n,b_n]$
Since $\diam [a_n,b_n]$ is controlled by a decreasing sequence:
\[ \diam [a_{n+1},b_{n+1}] \le 1/2^n \diam [a_1,b_1] \]
We take a sufficiently large n so that $b_n-a_n<\epsilon$
Since $c$ is in $[a_n,b_n]$, for all $x$ in $[a_n,b_n]$ we have $|x-c|\le b_n-a_n<\epsilon$ and therefore $[a_n,b_n]$ is within $B(c,\epsilon)$.

Here's the funny part: $[a_n,b_n]$ contains infinitely many elements of $A$, so it must contain at least one element in A that is not $c$.

Therefore this element $x \neq c$ is in $B_0(c,\epsilon)$.

\item Now we have an infinte bounded set $A$ in $\RR^n$

The idea here is to consecutively come up with better and better sequences of points in $A$. We denote $x_i$ to be the $i$-th coordinate in $\RR^n$.

Our first wish is to pick some elements in $A$ so that they sort of converge at $x_1$.

Because such considerations of 'restricting to a single coordinate' is important here, we define the projection map to the $i$-th coordinate by
\[ f_i(x_1,\dots,x_n)=x_i \]

So, we look at $f_i(A)$ and try to apply BW for the case where $n=1$.

However, the problem is that $f_i(A)$ need not be infinite. For example, the set $\{(0,0),(0,1),(0,2),\dots\}$ projected onto the first coordinate is simply $\{0\}$.

This forces us to consider two cases

Exercise 2: Show that $f_i(A)$ is bounded
This is simple
1. $f_1(A)$ is infinite, then we can apply BW(n=1) to find a real number $c_1$ which is a limit point in $f_1(A)$

Here we can construct a sequence of points 
\[ \{x^{(1),1},x^{(1),2},...\} \]
so that their first coordinates satisfy
\[ |x^{(1),n}_1-c_1| < 1/n \]
for all natural number n
(I know this notation is cumbersome but the problem is that we need multiple sequences for this proof)

2. $f_1(A)$ is finite, then by the Pigeonhole Principle there exists a real number $c_1$ such that its preimage $f_1^{-1}(c_1)$ in $A$ is infinite

In this case we can randomly pick a sequence $\{x^{(1),1},x^{(1),2},\dots\}$ in $A$ so that their first coordinate is equal to $c_1$

I forgot to mention something that is implied, but we actually do have the need to vocabasize that the sequence $\{x^{(1),1},x^{(1),2},\dots\}$ can be chosen to contain mutually distinct entries

Now that we have a sequence that behaves nice on the first coordinate, we may then move on to the second coordinate

Let $A_1=\{x^{(1),1},x^{(1),2},\dots\}$
We again consider $f_2(A_1)$ in two cases, infinite or finite

In any case, we are able to find a subsequence $\{x^{(2),1},x^{(2),2},\dots\}$, where
$x^{(2),k}=x^{(1),n_k}$ for some strictly increasing sequence of natural numbers $n_k$

So that, for the limit point/point with infinite preimage $c_2$, this sequence satisfies
\[ |f_2(x^{(2),n})-c_2| < \frac{1}{n} \]
Note that the property we have for the second case (we in fact have $f_2(x^{(2),n})=c_2$) is just a better version of this.

Now, take note that picking this subsequence does no harm whatsoever towards the first coordinate (if anything it would turn out to be better) since
\[ |f_1(x^{(2),k})-c_1| = |f_1(x^{(1),n_k}-c_1| < \frac{1}{n_k} \le \frac{1}{k} \]
($n_1<\dots<n_k$ is a strictly increasing sequence of natural numbers so $n_k \ge k$)

This continues on until we obtain a sequence of points $\{x^{(n),1},x^{(n),2},\dots\}$ in $A$ so that
\[ |f_i(x^{(n),k}-c_i|<\frac{1}{k} \quad \forall i,k \]

As we can see, the point $c=(c_1,\dots,c_n)$ is in fact a limit point of $A$ as we can always choose a big enough $k$ so that $x^{(n),k}$ is in $B(c,\epsilon) \cap A$.

Since $\{x^{(n),k}\}$ was always chosen to be a sequence of distinct entries, there is no danger for this sequence to always be c, and so c must be a limit point of $A$.

\item We may now return to the general case of Cantor.

Suppose that there is a sequence of decreasing compact sets $A_1,A_2,\dots$ in $\RR^n$. 
Note that every point is contained in $A_1$, so boundedness will never be an issue here.

Since $A_k$ are all nonempty, we can simply pick any element $a_k$ from $A_k$.

For the uncannily specific case that there are only finitely many $\{a_k\}$ chosen, we simply note that, again by Pigeonhole Principle, one of the $a_k$ appears infinitely often; thus for each $A_n$ we simply pick $n_k>n$ so that $A_{n_k}$ contains $a_k$, then $a_k$ is in $A_{n_k}$ which is a subset of $A_n$.

Otherwise, we can then note that $\{a_k\}$ is an infinite bounded set of points, so there must exist a limit point a of $\{a_k\}$.

We can now see that $a$ is always an element of $A_k$:
Using the same technique as Exercise 1, we see that a is a limit point of $\{a_n \mid n>k\}$ and so is a limit point of $A_k$, therefore a is in $A_k$ as $A_k$ is closed.

This proves the first part of the statement
The second part is completely identical to the second part of the $n=1$ case so we don't need to waste our time there either

\item We now consider a compact set A with some open covering $\mathcal{U}$.

This theorem is proved by contradiction: 
Suppose otherwise that set $A$ cannot be covered by any finite collection of open sets in $\mathcal{U}$

Since $A$ is compact, we may enclose it in a closed cube $Q_1$ (whose edges are parallel to the axes)

Now, for each step, we partition $Q$ into $2^n$ cubes by cutting it in half from each direction.

Then, starting from $Q_1$, there must exist one of these smaller cubes, denoted by $Q_2$, such that $A \cap Q_2$ cannot be covered by a finite collection of open sets in $\mathcal{U}$. 
Otherwise, if each $A \cap Q$ has a finite cover, then we simply collect all of these open sets together to form a finite cover of $A$, which violates our assumption.

We continue on to partition $Q_n$ and pick $Q_{n+1}$ so that $A_{n+1}$ has no finite cover (denote $A_n = A \cap Q_n$).

Note that $A$ and $Q_n$ are both compact, so $A_n$ is compact
Also we see that there is a decreasing sequence $A_1,A_2,\dots$
(we can't exactly obtain a relation between $\diam A_n$ and $\diam A_{n+1}$ here)

By Cantor Intersection Theorem we can always find a point $x$ in $A$ located in the intersection $\bigcap A_k$.

Now, since $\mathcal{U}$ is an open covering of $A$, there exists an open set $U$ in $\mathcal{U}$ such that $x\in U$.

The final key step is to exploit the sequence of decreasing cubes $Q_n$. So even though there isn't a clear cut way to control the sizes of $\diam A_n$, we do in fact have the property that $\diam Q_{n+1} = \frac{1}{2^n} \diam Q_1$.

Therefore, by picking a sufficiently large $n$, we can obtain $Q_n$ that is contained in $U$.

But this is a contradiction. 
This is because we've specifically chosen the sequence $A_n$ to be sets that do not possess any finite cover $\{U_1,...,U_n\}$ in $\mathcal{U}$. But here $A_n$ simply would have a one-element cover $\{U\}$.

This completes our proof.
\end{enumerate}
\end{proof}
%https://www.maths.usyd.edu.au/u/bobh/UoS/MATH3901/00met21.pdf
\end{comment}

\section{Compactness}
The following is a useful analogy to visualise the concept of compactness:
\begin{mdframed}
Compactness is like a well-contained space where nothing ``escapes'' or goes off to infinity.

An open cover is a collection of open sets that completely cover the compact set (think of a bunch of overlapping circles covering a shape).

The key feature of compact sets is that from any open cover, you can always select a finite number of sets from the cover that still manage to cover the entire space.
\end{mdframed}

\begin{definition}
Let $\mathcal{U}=\{U_i\mid i\in I\}$ be a collection of open subsets of $X$. We say that $\mathcal{U}$ is an \vocab{open cover}\index{open cover} of a set $K$ if
\[K\subset\bigcup_{i\in I}U_i.\]
If $I^\prime\subset I$ and $K\subset\bigcup_{i\in I^\prime}U_i$, we say that $\{U_i\mid i\in I^\prime\}$ is a \textbf{subcover} of $\mathcal{U}$. If moreover, $I^\prime$ is finite, then it is called a \textbf{finite subcover}.
\end{definition}

\begin{definition}[Compactness]
$K\subset X$ is said to be \vocab{compact}\index{compactness} if every open cover of $K$ contains a finite subcover.
\end{definition}

That is, if $\mathcal{U}=\{U_i\mid i\in I\}$ is an open cover of $K$, then there are finitely many indices $i_1,\dots,i_n$ such that
\[K\subset \bigcup_{k=1}^{n}U_{i_k}.\]

\begin{exercise}
Every finite set is compact.
\end{exercise}

\begin{solution}
Let $E$ be finite. Let $\mathcal{U}=\{U_i\mid i\in I\}$ be an open cover of $E$, then we have that $E\subset\bigcup_{i\in I}$.

For each point $x\in E$, take $i_x$ such that $x\in U_{i_x}$. Let $\mathcal{V}=\{U_{i_x}\mid x\in E\}$. By construction, since $x\in\mathcal{V}$ for all $x\in E$, $E\subset\mathcal{V}$ so $\mathcal{V}$ is an open cover of $E$. Since there are finitely many $x$, $\mathcal{V}$ is thus a finite subcover of $E$, and hence $E$ is compact.
\end{solution}

\begin{proposition}
Suppose $Y$ is a subspace of $X$, and $K\subset Y$. Then $K$ is compact relative to $X$ if and only if $K$ is compact relative to $Y$.
\end{proposition}

\begin{proof} \

\fbox{$\implies$} Suppose $K$ is compact relative to $X$. Let $\mathcal{U}$ be an open cover of $K$; that is, $\mathcal{U}=\{U_i\mid i\in I\}$ is a collection of sets open relative to $Y$, such that $K\subset\bigcup_{i\in I}U_i$. 
By \cref{prop:open-subspace-cap}, for all $i\in I$, there exist $V_i$ open relative to $X$ such that $U_i=Y\cap V_i$. Since $K$ is compact relative to $X$, we have
\begin{equation*}\tag{1}
K\subset\bigcup_{k=1}^{n}V_{i_k}
\end{equation*}
for some choice of finitely many indices $i_1,\dots,i_n$. Since $K\subset Y$, (1) implies that
\begin{equation*}\tag{2}
K\subset\bigcup_{k=1}^{n}U_{i_k}.
\end{equation*}
This proves that $K$ is compact relative to $Y$.

\fbox{$\impliedby$} Suppose $K$ is compact relative to $Y$, let $\mathcal{V}=\{V_i\mid i\in I\}$ be a collection of open subsets of $X$ which covers $K$, and put $U_i=Y\cap V_i$. Then (2) will hold for some choice of $i_1,\dots,i_n$; and since $U_i\subset V_i$, (2) implies (1).
\end{proof}

\begin{proposition}
Compact subsets of metric spaces are closed.
\end{proposition}

\begin{proof}
Let $K\subset X$ be compact. To prove that $K$ is closed, we need to show that $K^c$ is open.

Suppose $p\in X$, $p\neq K$. If $q\in K$, let $V_q$ and $W_q$ be neighbourhoods of $p$ and $q$ respectively, of radius less than $\frac{1}{2}d(p,q)$. Since $K$ is compact, there exists finite many points $q_1,\dots,q_n\in K$ such that
\[K\subset\bigcup_{k=1}^{n}W_{q_k}=W.\]
If $V=\bigcap_{k=1}^{n}V_{q_k}$, then $V$ is a neighbourhood of $p$ which does not intersect $W$. Hence $V\subset K^c$, so $p$ is an interior point of $K^c$. The theorem follows.
\end{proof}

\begin{proposition}
Closed subsets of compact sets are compact.
\end{proposition}

\begin{proof}
Suppose $F\subset K\subset X$, $F$ is closed (relative to $X$), and $K$ is compact.

Let $\mathcal{V}=\{V_i\mid i\in I\}$ be an open cover of $F$. If $F^c$ is adjoined to $\mathcal{V}$, we obtain an open cover $\Omega$ of $K$. Since $K$ is compact, there is a finite subcollection $\Phi$ of $\Omega$ which covers $K$, and hence $F$. If $F^c$ is a member of $\Phi$, we may remove it from $\Phi$ and still retain an open cover of $F$. We have thus shown that a finite subcollection of $\mathcal{V}$ covers $F$.
\end{proof}

\begin{corollary}
If $F$ is closed and $K$ is compact, then $F\cap K$ is compact.
\end{corollary}

\begin{proposition}
If $E$ is an infinite subset of a compact set $K$, then $E$ has a limit point in $K$.
\end{proposition}

\begin{proposition}
If $(I_n)$ is a sequence of intervals in $\RR$ such that $I_i\supset I_{i+1}$, then $\bigcap_{i=1}^{\infty}I_n\neq\emptyset$.
\end{proposition}

\begin{proposition}
If $(I_n)$ is a sequence of $k$-cells such that $I_n\supset I_{n+1}$, then $\bigcap_{n=1}^{\infty}\neq\emptyset$.
\end{proposition}

\begin{proof}
Let $I_n$ consist of all points $\vb{x}=(x_1,\dots,x_k)$ such that
\end{proof}

\begin{proposition}
Every $k$-cell is compact.
\end{proposition}

\begin{theorem}[Cantor's Intersection Theorem]\label{thrm:cantor-intersection}
Given a decreasing sequence of compact sets $A_1\supset A_2 \supset \cdots$, there exists a point $x\in\RR^n$ such that $x$ belongs to all $A_i$. In other words, $\bigcap_{i=1}^\infty A_i\neq\emptyset$. Moreover, if for all $i\in\NN$ we have $\diam A_{i+1}\le c\cdot\diam A_k$ for some constant $c<1$, then such a point must be unique, i.e. $\bigcap_{i=1}^\infty A_k=\{x\}$ for some $x\in\RR^n$.
\end{theorem}

\begin{proposition}
If $E\subset\RR^n$ has one of the following three properties, then it has the other two:
\begin{enumerate}[label=(\roman*)]
\item $E$ is closed and bounded.
\item $E$ is compact.
\item Every infinite subset of $E$ has a limit point in $E$.
\end{enumerate}
\end{proposition}

\begin{theorem}[Heine--Borel Theorem]\label{thrm:heine-borel}
$E\subset\RR^n$ is compact if and only if $E$ is closed and bounded.
\end{theorem}

\begin{proof}

\end{proof}

\begin{theorem}[Bolzano--Weierstrass Theorem]\label{thrm:bolzano-weierstrass}
Every bounded infinite subset of $\RR^n$ has a limit point in $\RR^n$.
\end{theorem}

\begin{proof}

\end{proof}

\begin{comment}
sequential compactness
A set $K$ is compact if and only if every sequence of points in $K$ has a subsequence that converges to a point in $K$.

Any continuous function defined on a compact set is bounded.

extreme value theorem
\end{comment}
\pagebreak

\section{Perfect Sets}
\begin{definition}[Perfect set]
$E$ is \vocab{perfect}\index{perfect set} if $E$ is closed and if every point of $E$ is a limit point of $E$.
\end{definition}

\begin{proposition}
Let $P$ be a non-empty perfect set in $\RR^n$. Then $P$ is uncountable.
\end{proposition}

\begin{corollary}
Every interval $[a,b]$ is uncountable. In particular, $\RR$ is uncountable.
\end{corollary}

The set which we are now going to construct shows that there exist perfect sets in $\RR$ which contain no segment.

Let
\[E_0=[0,1].\]
Remove the segment $\brac{\frac{1}{3},\frac{2}{3}}$ to give
\[E_1=\sqbrac{0,\frac{1}{3}}\cup\sqbrac{\frac{2}{3},1}.\]
Remove the middle thirds of these intervals to give
\[E_2=\sqbrac{0,\frac{1}{9}}\cup\sqbrac{\frac{2}{9},\frac{3}{9}}\cup\sqbrac{\frac{6}{9},\frac{7}{9}}\cup\sqbrac{\frac{8}{9},1}.\]
Repeating this process, we obtain a monotonically decreasing sequence of compact sets $(E_n)$, where $E_1\supset E_2\supset\cdots$ and $E_n$ is the union of $2^n$ intervals, each of length $3^{-n}$.

The \vocab{Cantor set} is defined as
\[P\coloneqq\bigcap_{n=1}^{\infty}E_n.\]

\begin{proposition}
$P$ is compact.
\end{proposition}

\begin{proposition}
$P$ is not empty.
\end{proposition}

\begin{proof}
This follows from Theorem 2.36.
\end{proof}

\begin{proposition}
$P$ contains no segment.
\end{proposition}

\begin{proof}
No segment of the form
\[\brac{\frac{3k+1}{3^m},\frac{3k+2}{3^m}},\]
where $k,m\in\ZZ^+$, has a point in common with $P$. Since every segment $(\alpha,\beta)$ contains a segment of the above form, if
\[3^{-m}<\frac{\beta-\alpha}{6},\]
$P$ contains no segment.
\end{proof}

\begin{proposition}
$P$ is perfect.
\end{proposition}

\begin{proof}
To show that $P$ is perfect, it is enough to show that $P$ contains no isolated point. Let $x\in P$, and let $S$ be any segment containing $x$. Let $I_n$ be that interval of $E_n$ which contains $x$. Choose $n$ large enough, so that $I_n\subset S$. Let $x_n$ be an endpoint of $I_n$, such that $x_n\neq x$.

It follows from the construction of $P$ that $x_n\in P$. Hence $x$ is a limit point of $P$, and $P$ is perfect.
\end{proof}
\pagebreak

\section{Connectedness}
\begin{definition}[Connectedness]
$A$ and $B$ are said to be \vocab{separated} if $A\cap\overline{B}=\emptyset$ and $\bar{A}\cap B=\emptyset$; that is, no point of $A$ lies in the closure of $B$ and no point of $B$ lies in the closure of $A$.

$E\subset X$ is said to be \vocab{connected}\index{connectedness} if $E$ is not a union of two non-empty separated sets. 
\end{definition}

\begin{remark}
Separated sets are of course disjoint, but disjoint sets need not be separated. For example, the interval $[0,1]$ and the segment $(1,2)$ are not separated, since $1$ is a limit point of $(1,2)$. However, the segments $(0,1)$ and $(1,2)$ are separated.
\end{remark}

The connected subsets of the line have a particularly simple structure: 

\begin{proposition}
$E\subset\RR^1$ is connected if and only if it has the following property: if $x,y\in E$ and $x<z<y$, then $z\in E$.
\end{proposition}

\begin{proof} \

\fbox{$\impliedby$} If there exists $x,y\in E$ and some $z\in(x,y)$ such that $z\notin E$, then $E=A_z\cup B_z$ where
\[ A_z=E\cap(-\infty,z), \quad B_z=E\cap(z,\infty). \]
Since $x\in A_z$ and $y\in B_z$, $A$ and $B$ are non-empty. Since $A_z\subset(-\infty,z)$ and $B_z\subset(z,\infty)$, they are separated. Hence $E$ is not connected.

\fbox{$\implies$} Suppose $E$ is not connecetd. Then there are non-empty separated sets $A$ and $B$ such that $A\cup B=E$. Pick $x\in A$, $y\in B$, and WLOG assume that $x<y$. Define
\[z\coloneqq\sup(A\cap[x,y].)\]
By 
\end{proof}

\begin{definition}
We say that a metric space is \vocab{disconnected} if we can write it as the disjoint union of two nonempty open sets. We say that a space is \vocab{connected} if it is not disconnected.
\end{definition}

If $X$ is written as a disjoint union of two nonempty open sets $U$ and $V$ then we say that these sets \vocab{disconnect} $X$.

\begin{example}
If $X=[0,1]\cup[2,3]\subset\RR$ then we have seen that both $[0,1]$ and $[2,3]$ are open in $X$. Since $X$ is their disjoint union, $X$ is disconnected.
\end{example}

The following lemma gives some equivalent ways to formulate the concept of connected space.

\begin{lemma}
The following are equivalent:
\begin{enumerate}[label=(\roman*)]
\item $X$ is connected.
\item If $f:X\to\{0,1\}$ is a continuous function then $f$ is constant.
\item The only subsets of $X$ which are both open and closed are $X$ and $\emptyset$.
\end{enumerate}
(Here the set $\{0,1\}$ is viewed as a metric space via its embedding in $\RR$, or equivalently with the discrete metric.)
\end{lemma}

\begin{proof}

\end{proof}

Frequently one has a metric space $X$ and a subset $E$ of it whose connectedness or otherwise one wishes to ascertain. To this end, it is useful to record the following lemma.

\begin{lemma}
Let $E\subset X$, considered as a metric space with the metric induced from $X$. Then $E$ is connected if and only if the following is true: if $U,V$ are open subsets of $X$, and $U\cap V\cap E=\emptyset$, then $E\subset U\cup V$ implies either $E\subset U$ or $E\subset V$.
\end{lemma}

\begin{proof}

\end{proof}

We now turn to some basic properties of the notion of connectedness. These broadly conform with one's intuition about how connected sets should behave.

\begin{lemma}[Sunflower lemma]
Let $\{E_i\mid i\in I\}$ be a collection of connected subsets of $X$ such that $\bigcap_{i\in I}E_i\neq\emptyset$. Then $\bigcup_{i\in I}E_i$ is connected.
\end{lemma}

\begin{proof}

\end{proof}


    \chapter{Numerical Sequences and Series}\label{chap:num-seq-series}
Tthis chapter will deal primarily with sequences and series in $\RR$ (and also $\CC$). The basic facts about convergence, however, are just as easily explained in a more general setting (metric spaces).

As usual, let $(X,d)$ be a metric space.

\section{Sequences}
\subsection{Convergence}
\begin{definition}[Sequence]
A \vocab{sequence} $(x_n)$ in $X$ is a function $f:\NN\to X$ which maps $n\mapsto x_n$.
\end{definition}

\begin{definition}
The \vocab{range} of a sequence $(x_n)$ is the set
\[\{a\in X\mid\exists n\in\NN, a=x_n\}.\]
Note that the range of a sequence may be a finite set or it may be infinite. $(x_n)$ is \vocab{bounded} if its range is bounded.
\end{definition}

\begin{definition}[Convergence]
A sequence $(x_n)$ \vocab{converges}\index{convergence of sequence} to $x\in X$, denoted by $x_n\to x$, if
\[\forall\epsilon>0,\quad\exists N\in\NN,\quad\forall n\ge N,\quad d(x_n,x)<\epsilon.\]
We call $x$ a \emph{limit} of $(x_n)$. 
If $(x_n)$ does not converge, it is said to \emph{diverge}.
\end{definition}

\begin{remark}
This limit process conveys the intuitive idea that $x_n$ can be made arbitrarily close to $x$, provided that $n$ is sufficiently large.
\end{remark}

\begin{remark}
If $x_n\not\to x$, simply negate the definition for convergence:
\[\exists\epsilon>0,\quad\forall N\in\NN,\quad\exists n\ge N,\quad d(x_n,x)\ge\epsilon.\]
\end{remark}

\begin{remark}
From the definition, the convergence of a sequence depends not only on the sequence itself, but also on the metric space $X$. For instance, the sequence given by $a_n=\frac{1}{n}$ converges in $\RR$ (to $0$), but fails to converge in $\RR^+$. In cases of possible ambiguity, we shall specify ``convergent in $X$'' rather than ``convergent''. 
\end{remark}

\begin{example}
Show that $\frac{1}{n}\to 0$.
\begin{solution}
Fix $\epsilon>0$. By the Archimedian property, there exists $N\in\NN$ such that $\frac{1}{N}<\epsilon$. Take $N=\floor{\frac{1}{\epsilon}}+1$. Then for all $n\ge N$,
\[\absolute{\frac{1}{n}-0}=\frac{1}{n}\le\frac{1}{N}=\frac{1}{\floor{\frac{1}{\epsilon}}+1}<\frac{1}{\frac{1}{\epsilon}}=\epsilon\]
as desired. Therefore $\frac{1}{n}\to0$.
\end{solution}
\end{example}

A useful tip for finding the required $N$ (in terms of $\epsilon$) is to work backwards from the result we wish to show, as illustrated in the following example.

\begin{example}
Let $a_n=1+(-1)^n\frac{1}{\sqrt{n}}$. Show that $a_n\to 1$.

Before our proof, we aim to find some $N\in\NN$ such that if $n\ge N$ then
\begin{align*}
|a_n-1|&<\epsilon\\
\frac{1}{\sqrt{n}}=\absolute{(-1)^n\frac{1}{\sqrt{n}}}&<\epsilon\\
\frac{1}{n}&<\epsilon^2\\
n&>\frac{1}{\epsilon^2}
\end{align*}
Hence take $N=\floor{\frac{1}{\epsilon^2}}+1$.

\begin{solution}
Let $\epsilon>0$ be given. Take $N=\floor{\frac{1}{\epsilon^2}}+1$. If $n\ge N$, then
\begin{align*}
|a_n-1|&=\absolute{(-1)^n\frac{1}{\sqrt{n}}}=\frac{1}{\sqrt{n}}\\
&\le\frac{1}{\sqrt{N}}=\frac{1}{\sqrt{\floor{\frac{1}{\epsilon^2}}+1}}\\
&<\frac{1}{\sqrt{\frac{1}{\epsilon^2}}}=\epsilon
\end{align*}
as desired. Therefore $a_n\to1$.
\end{solution}
\end{example}

\begin{lemma}[Uniqueness of limit]
If a sequence converges, then its limit is unique.
\end{lemma}

\begin{proof}
Let $(x_n)$ be a sequence in $X$. Suppose that $x_n\to x$ and $x_n\to x^\prime$ for $x,x^\prime\in X$. We will show that $x^\prime=x$.

Let $\epsilon>0$ be given. Then there exists $N,N^\prime\in\NN$ such that
\[n\ge N\implies d(x_n,x)<\frac{\epsilon}{2}\]
and
\[n\ge N^\prime\implies d(x_n,x^\prime)<\frac{\epsilon}{2}.\]
Take $N_1\coloneqq\max\{N,N^\prime\}$. If $n\ge N_1$, then both hold. By the triangle inequality,
\begin{align*}
d(x,x^\prime)&\le d(x,x_n)+d(x_n,x^\prime)\\
&<\frac{\epsilon}{2}+\frac{\epsilon}{2}=\epsilon
\end{align*}
for all $\epsilon>0$. Hence $d(x,x^\prime)=0$ and thus $x=x^\prime$.
\end{proof}

Since the limit is unique, we can give it a notation.

\begin{notation}
If $(x_n)$ converges to $x$, we denote $\displaystyle\lim_{n\to\infty}x_n=x$.
\end{notation}

We now outline some important properties of convergent sequences in metric spaces.

\begin{proposition}
Let $(x_n)$ be a sequence in $X$.
\begin{enumerate}[label=(\roman*)]
\item $x_n\to x$ if and only if every open ball of $x$ contains $x_n$ for all but finitely many $n$.
\item If $(x_n)$ converges, then $(x_n)$ is bounded.
\item Suppose $E\subset X$. Then $x$ is a limit point of $E$ if and only if there exists a sequence $(x_n)$ in $E\setminus\{x\}$ such that $x_n\to x$.
\end{enumerate}
\end{proposition}

\begin{proof} \
\begin{enumerate}[label=(\roman*)]
\item \fbox{$\implies$} Suppose $x_n\to x$. Let $\epsilon>0$ be given, then there exists $N\in\NN$ such that
\[n\ge N\implies d(x_n,x)<\epsilon.\]
Corresponding to this $\epsilon$, consider the open ball $B_\epsilon(x)$. Then by definition, for $y\in X$, 
\[d(y,x)<\epsilon\implies y\in B_\epsilon(x).\]
Hence $n\ge N$ implies $x_n\in B_\epsilon(x)$.

\fbox{$\impliedby$} Suppose every open ball of $x$ contains all but finitely many of the $x_n$.

Let $\epsilon>0$ be given. Consider the open ball $B_\epsilon(x)$. Since $B_\epsilon(x)$ is a open ball of $x$, it will also eventually contain all $x_n$; that is, there exists $N\in\NN$ such that if $n\ge N$, then $x_n\in B_\epsilon(x)$, i.e. $d(x_n,x)<\epsilon$. Hence $x_n\to x$.

\item Suppose $x_n\to x$. Let $\epsilon>0$ be given. Then there exists $N\in\NN$ such that $n\ge N$ implies $d(x_n,x)<1$. Now let
\[r=\max\{1,d(x_1,x),\dots,d(x_N,x)\}.\]
Then $d(x_n,x)\le r$ for $n=1,2,\dots,N$, so the range of $x_n$ is bounded by $B_r(x)$. Hence $(x_n)$ is bounded.

\item \fbox{$\implies$} Suppose $x$ is a limit point of $E$. 

Consider a sequence of open balls $\brac{B_\frac{1}{n}(x)}$, for $n\in\NN$. Since $x$ is a limit point, each open ball intersects with $E$ at some point which is not $x$. We pick one such point $x_n$ from each $B_\frac{1}{n}(x)\cap E$. Then
\[d(x_n,x)<\frac{1}{n}.\]
Let $\epsilon>0$ be given. Then by the Archimedian property, there exists $N\in\NN$ such that $\frac{1}{N}<\epsilon$. If $n\ge N$,
\[d(x_n,x)\le\frac{1}{n}\le\frac{1}{N}<\epsilon,\]
which shows that $x_n\to x$.

\fbox{$\impliedby$} Suppose that there exists a sequence $(x_n)$ in $E\setminus\{x\}$ such that $x_n\to x$. Then for each open ball $B_\epsilon(x)$, we can find some $N\in\NN$ such that if $n\in\NN$ then
\[x_n\in B_\epsilon(x).\]
Since $x_n\in E\setminus\{x\}$, this shows that $x$ is a limit point of $E$.
\end{enumerate}
\end{proof}

\begin{proposition}[Ordering]
Suppose $(a_n)$ and $(b_n)$ are convergent sequences, and $a_n \le b_n$. Then
\[\lim_{n\to\infty}a_n\le\lim_{n\to\infty}b_n.\]
\end{proposition}

\begin{proof}
Let $\displaystyle a=\lim_{n\to\infty}a_n$, $\displaystyle b=\lim_{n\to\infty}b_n$. Suppose, for a contradiction, that $a>b$.

Let $\epsilon=a-b>0$ be given. There exists $N_1,N_2\in\NN$ such that
\begin{align*}
n\ge N_1&\implies|a_n-a|<\frac{\epsilon}{2},\\
n\ge N_2&\implies|b_n-b|<\frac{\epsilon}{2}.
\end{align*}
Let $N=\max\{N_1,N_2\}$, then $n\ge N$ implies
\[a_n>a-\frac{\epsilon}{2},\quad b_n<b+\frac{\epsilon}{2}\]
and thus
\[a_n-b_n>a-b-\epsilon=0\]
so $a_n>b_n$, which is a contradiction.
\end{proof}

\begin{remark}
If $a_n<b_n$, we may not necessarily have $\displaystyle\lim_{n\to\infty}a_n<\lim_{n\to\infty}b_n$. For instance, $-\frac{1}{n}<\frac{1}{n}$ but their limits are both $0$.
\end{remark}

\begin{proposition}[Arithmetic properties]
Suppose $(a_n)$ and $(b_n)$ are convergent seqeunces in $\CC$; let $\displaystyle a=\lim_{n\to\infty}a_n$, $\displaystyle b=\lim_{n\to\infty}b_n$. Then
\begin{enumerate}[label=(\roman*)]
\item $\displaystyle\lim_{n\to\infty}ca_n=ca$, where $c$ is a constant\hfill(scalar multiplication)
\item $\displaystyle\lim_{n\to\infty}(a_n+b_n)=a+b$\hfill(addition)
\item $\displaystyle\lim_{n\to\infty}(a_n b_n)=ab$\hfill(multiplication)
\item $\displaystyle\lim_{n\to\infty}\frac{a_n}{b_n}=\frac{a}{b}$ ($b_n\neq0$, $b\neq0$)\hfill(division)
\end{enumerate}
\end{proposition}

\begin{proof} \
\begin{enumerate}[label=(\roman*)]
\item The case where $c=0$ is trivial. Now suppose $c\neq0$. Let $\epsilon>0$ be given. Then there exists $N\in\NN$ such that
\[n\ge N\implies|a_n-a|<\frac{\epsilon}{|c|}.\]
Then if $n\ge N$,
\[|ca_n-ca|=|c|\:|a_n-a|<\epsilon.\]

\item Let $\epsilon>0$ be given. Since $a_n\to a$ and $b_n\to b$, there exists $N_1,N_2\in\NN$ such that
\begin{align*}
n\ge N_1&\implies|a_n-a|<\frac{\epsilon}{2},\\
n\ge N_2&\implies|b_n-b|<\frac{\epsilon}{2}.
\end{align*}
Let $N=\max\{N_1,N_2\}$, then $n\ge N$ implies
\begin{align*}
\absolute{(a_n+b_n)-(a+b)}
&\le|a_n-a|+|b_n-b|\\
&<\frac{\epsilon}{2}+\frac{\epsilon}{2}=\epsilon.
\end{align*}
Hence $\displaystyle\lim_{n\to\infty}(a_n+b_n)=a+b$, as desired.

\item Write
\[a_nb_n-ab=(a_n-a)(b_n-b)+a(b_n-b)+b(a_n-a).\]
Let $\epsilon>0$ be given. Since $a_n\to a$ and $b_n\to b$, there exist $N_1,N_2\in\NN$ such that
\begin{align*}
n\ge N_1&\implies|a_n-a|<\sqrt{\epsilon},\\
n\ge N_2&\implies|b_n-b|<\sqrt{\epsilon}.
\end{align*}
Let $N=\max\{N_1,N_2\}$. Then $n\ge N$ implies
\[|(a_n-a)(b_n-b)|<\epsilon,\]
and thus $\displaystyle\lim_{n\to\infty}(a_n-a)(b_n-b)=0$.

Note that $\displaystyle\lim_{n\to\infty}a(b_n-b)=\lim_{n\to\infty}b(a_n-a)=0$. Hence
\[\lim_{n\to\infty}(a_nb_n-ab)=0.\]

\item Since we have proven multiplication, it suffices to show that $\displaystyle\lim_{n\to\infty}\frac{1}{b_n}=\frac{1}{b}$.

Since $b_n\to b$, there exists $m\in\NN$ such that
\[n\ge m\implies|b_n-b|<\frac{1}{2}|b|.\]
Let $\epsilon>0$ be given. There exists $N\in\NN$, $N>m$ such that
\[n\ge N\implies|b_n-b|<\frac{1}{2}|b|^2\epsilon.\]
Hence for $n\ge N$,
\[\absolute{\frac{1}{b_n}-\frac{1}{b}}=\absolute{\frac{b-b_n}{b_nb}}<\frac{2}{|b|^2}|b_n-b|<\epsilon.\]
\end{enumerate}
\end{proof}

\begin{proposition}[Squeeze theorem]
Let $a_n\le c_n\le b_n$ where $(a_n)$ and $(b_n)$ are convergent sequences such that $\displaystyle\lim_{n\to\infty}a_n=\lim_{n\to\infty}b_n=L$. Then $(c_n)$ is also a convergent sequence, and
\[\lim_{n\to\infty}c_n=L.\]
\end{proposition}

\begin{proof}
Let $\epsilon>0$ be given. There exist $N_1,N_2\in\NN$ such that
\begin{align*}
n\ge N_1&\implies|a_n-L|<\epsilon,\\
n\ge N_2&\implies|b_n-L|<\epsilon.
\end{align*}
In particular, we have
\[a_n>L-\epsilon,\quad b_n<L+\epsilon.\]
Let $N=\max\{N_1,N_2\}$. Then $n\ge N$ implies
\[L-\epsilon<a_n\le c_n\le b_n<L+\epsilon\]
or
\[|c_n-L|<\epsilon.\]
Hence $(c_n)$ is convergent, and $c_n\to L$.
\end{proof}

\subsection{Subsequences}
\begin{definition}[Subsequence]
Given a sequence $(x_n)$, consider a sequence $(n_k)$ of positive integers such that $n_1<n_2<\cdots$. Then $(x_{n_k})$ is called a \vocab{subsequence}\index{subsequence} of $(x_n)$.

If $(x_{n_k})$ converges, its limit is called a \emph{subsequential limit} of $(x_n)$.
\end{definition}

\begin{proposition}
$(x_n)$ converges to $x$ if and only if every subsequence of $(x_n)$ converges to $x$.
\end{proposition}

\begin{proof} \

\fbox{$\implies$} Suppose $x_n\to x$. Let $\epsilon>0$ be give. Then there exists $N\in\NN$ such that
\[n\ge N\implies d(x_n,x)<\epsilon.\]
Every subsequence of $(x_n)$ can be written in the form $(x_{n_k})$ where $n_1<n_2<\cdots$ is a strictly increasing sequence of positive integers. Pick $M$ such that $n_M\ge N$. Then
\[k>M\implies n_k>n_M\ge N\implies d(x_{n_k},x)<\epsilon.\]
Hence every subsequence of $(x_n)$ converges to $x$.

\fbox{$\impliedby$} Suppose every subsequence of $(x_n)$ converges to $x$. Since $(x_n)$ is a subsequence of itself, we must have $x_n\to x$.
\end{proof}

\begin{proposition}
In a compact metric space, any sequence has a convergent subsequence.
\end{proposition}

\begin{proof}
Suppose $(x_n)$ is a sequence in a compact metric space $X$.

Let $E$ be the range of $(x_n)$. We have to consider two cases: (i) $E$ is finite, (ii) $E$ is infinite.
\begin{enumerate}[label=(\roman*)]
\item We prove by directly constructing the desired convergent subsequence.

Notice that there are infinitely many terms in the sequence $(x_n)$, but only finitely many distinct terms in $E$. Hence by the pigeonhole principle, at least one term of $E$ appears infinitely many times in the sequence. That is, there exists $x\in E$ and a sequence $(n_k)$ with $n_1<n_2<\cdots$ such that
\[x_{n_1}=x_{n_2}=\cdots=x.\]
This subsequence $(x_{n_k})$ that we have constructed evidently converges to $x$.

\item If $E$ is infinite, then $E$ is an infinite subset of a compact set. By \cref{prop:infinite-compact-lp}, $E$ has a limit point $x\in X$.

We now construct a subsequence $(x_{n_k})$ of $(x_n)$ such that $x_{n_k}\to x$. Choose $n_1$ so that $d(x,x_{n_1})<1$. Having chosen $n_1,\dots,n_{k-1}$, choose $n_k$ where $n_k>n_{k-1}$ such that $d(x,x_{n_k})<\frac{1}{k}$ (such $n_k$ exists due to \cref{prop:limit-point-inf-points}). Then $x_{n_k}\to x$.
\end{enumerate}
\end{proof}

\begin{corollary}[Bolzano--Weierstrass]
Every bounded sequence in $\RR^k$ contains a convergent subsequence.
\end{corollary}

\begin{proof}
By \cref{prop:closed-bounded-compact-inf-lp}, every bounded sequence in $\RR^k$ lives in a compact subset of $\RR^k$, and therefore it lives in a compact metric space. Hence by the previous result, it contains a convergent subsequence converging to a point in $\RR^k$.
\end{proof}

\begin{lemma}
Suppose $(x_n)$ is a sequence in $X$. Then the subsequential limits of $(x_n)$ form a closed subset of $X$.
\end{lemma}

\begin{proof}
Let $E$ be the set of all subsequential limits of $(x_n)$, let $q$ be a limit point of $E$. We want to show that $q\in E$.

Choose $n_1$ so that $x_{n_1}\neq q$. (If no such $n_1$ exists, then $E$ has only one point, and there is nothing to prove.) Put $\delta=d(q,x_{n_1})$. Suppose $n_1,\dots,n_{i-1}$ are chosen. Since $q$ is a limit point of $E$, there is an $x\in E$ with $d(x,q)<2^{-1}\delta$. Since $x\in E$, there is an $n_i>n_{i-1}$ such that $d(x,x_{n_k})<2^{-i}\delta$. Thus
\[d(q,x_{n_k})<2^{1-i}\delta\]
for $i=1,2,3,\dots$. This says that $(x_{n_k})$ converges to $q$. Hence $q\in E$.
\end{proof}

\subsection{Cauchy Sequences}
This is a very helpful way to determine whether a sequence is convergent or divergent, as it does not require the limit to be known. In the future you will see many instances where the convergence of all sorts of limits are compared with similar counterparts; generally we describe such properties as \emph{Cauchy criteria}.

\begin{definition}[Cauchy sequence]
A sequence $(x_n)$ in $X$ is a \vocab{Cauchy sequence}\index{Cauchy sequence} if 
\[\forall\epsilon>0,\quad\exists N\in\NN,\quad\forall n,m\ge N,\quad d(x_n,x_m)<\epsilon.\]
\end{definition}

\begin{remark}
Intuitively, we see that the distances between any two terms is sufficiently small after a certain point.
\end{remark}

A natural question is regarding the relationship between convergent sequences and Cauchy sequences. We now address this.

\begin{proposition} \
\begin{enumerate}[label=(\roman*)]
\item In any metric space, every convergent sequence is a Cauchy sequence.
\item If $X$ is a compact metric space and if $(x_n)$ is a Cauchy sequence in $X$, then $(x_n)$ converges to some point of $X$.
\item In $\RR^k$, every Cauchy sequence converges. 
\end{enumerate}
\end{proposition}

\begin{remark}
The converse of (i) is not true. For instance, the sequence $\{3,3.1,3.14,3.141,3.1415,\dots\}$ is a Cauchy sequence but does not converge in $\QQ$.
\end{remark}

\begin{proof} \
\begin{enumerate}[label=(\roman*)]
\item Suppose $x_n\to x$. Let $\epsilon>0$. There exists $N\in\NN$ such that for all $n\ge N$,
\[d(x_n,x)<\frac{\epsilon}{2}.\]
Then for all $n,m\ge N$,
\[d(x_n,x_m)\le d(x_n,x)+d(x_m,x)<\frac{\epsilon}{2}+\frac{\epsilon}{2}=\epsilon,\]
as desired. Hence $(x_n)$ is a Cauchy sequence.
\item Let $(x_n)$ be a Cauchy sequence in $X$. Since $X$ is compact, it is sequentially compact. Then there exists a subsequence $(x_{n_k})$ such that $x_{n_k}\to x$.
\begin{claim}
$x_n\to x$.
\end{claim}
Let $\epsilon>0$. Since $(x_n)$ is a Cauchy sequence, there exists $N_1\in\NN$ such that
\[n,m\ge N_1\implies d(x_n-x_m)<\frac{\epsilon}{2}.\]
$x_{n_k}\to x$ implies there exists $N_2\in\NN$ such that
\[n_k\ge N_2\implies d(x_{n_k},x)<\frac{\epsilon}{2}.\]
Let $N=\max\{N_1,N_2\}$, fix some $n_k\ge N$. Then $n\ge N$ implies
\[d(x_n,x)\le d(x_n,x_{n_k})+d(x_{n_k},x)<\frac{\epsilon}{2}+\frac{\epsilon}{2}=\epsilon.\]

\item Suppose $(x_n)$ is a Cauchy sequence.

We perform three steps:
\begin{itemize}
\item We first show that $(x_n)$ is bounded:

Pick $N\in\NN$ such that $|x_n-x_N|\le 1$ for all $n\ge N$. Then
\[|x_n|\le\max\{1+|x_N|,|x_1|,\dots,|x_{N-1}|\}.\]

\item Since $(x_n)$ is bounded, by Bolzano--Weierstrass, $(x_n)$ contains a subsequence $(x_{n_k})$ which converges to $x$.

\item We now show that $x_n\to x$.

Let $\epsilon>0$ be given. Since $(x_n)$ is a Cauchy sequence, there exists $N_1\in\NN$ such that
\[n,m\ge N_1\implies|x_n-x_m|<\frac{\epsilon}{2}.\]

Since $x_{n_k}\to x$, there exists $M\in\NN$ such that for all $k>M$,
\[n_k>n_M\implies |x_{n_k}-x|<\frac{\epsilon}{2}.\]
Now since $n_1<n_2<\cdots$ is a sequence of strictly increasing positive integers, we can pick $i>M$ such that $n_k>N_1$. Then for all $n\ge N_1$, by setting $m=n_k$ we obtain
\[ |x_n-x_{n_k}|<\frac{\epsilon}{2},\quad |x_{n_k}-x| < \frac{\epsilon}{2}.\]
Hence
\[|x_n-x|\le|x_n-x_{n_k}|+|x_{n_k}-x|<\epsilon.\]
Therefore $(x_n)$ is convergent, and $x_n\to x$.
\end{itemize}
\end{enumerate}
\end{proof}

\begin{definition}
A metric space $X$ is \vocab{complete} if every Cauchy sequence in $X$ converges.
\end{definition}

\begin{remark}
The above result shows that that all compact metric spaces and all Euclidean spaces are complete. It also implies that every closed subset $E$ of a complete metric space $X$ is complete. (Every Cauchy sequence in $E$ is a Cauchy sequence in $X$, hence it converges to some $x\in X$, and actually $x\in E$ since $E$ is closed.)
\end{remark}

\begin{example}
The sequence $(x_n)$ is defined as follows:
\[x_n=1+\frac{1}{2}+\cdots+\frac{1}{n}.\]
$(x_n)$ does not converge in $\RR$.
\begin{proof}
We claim that $(x_n)$ is not a Cauchy sequence. WLOG assume $n>m$. Consider
\[|x_n-x_m|=\frac{1}{m+1}+\frac{1}{m+2}+\cdots+\frac{1}{n}\ge\frac{n-m}{n}=1-\frac{m}{n}.\]
Let $n=2m$, then
\[|x_n-x_m|=|x_{2m}-x_m|>\frac{1}{2}.\]
Hence $(x_n)$ is not a Cauchy sequence, so it does not converge.
\end{proof}
\end{example}

\subsection{Monotonic Sequences}
\begin{definition}[Monotonic sequence]
A sequence $(x_n)$ in $\RR$ is
\begin{enumerate}[label=(\roman*)]
\item \emph{monotonically increasing} if $x_n\le x_{n+1}$ for $n\in\NN$;
\item \emph{monotonically decreasing} if $x_n\ge x_{n+1}$ for $n\in\NN$;
\item \vocab{monotonic} if it is either monotonically increasing or monotonically decreasing.
\end{enumerate}
\end{definition}

\begin{lemma}[Monotone convergence theorem]
A monotonic sequence in $\RR$ converges if and only if it is bounded.
\end{lemma}

\begin{proof}
We show the case for monotically increasing sequences; the case for monotonically decreasing sequences is similar.

\fbox{$\impliedby$} Suppose $(x_n)$ is a monotonically increasing sequence bounded above. Let $E$ be the range of $x_n$. By lub property of $\RR$, $E$ has a supremum in $\RR$; let $x=\sup E$.
\begin{claim}
$x_n\to x$.
\end{claim}
By definition of supremum, $x_n\le x$ for all $n\in\NN$. For every $\epsilon>0$, there exists $N\in\NN$ such that
\[x-\epsilon<x_N\le x,\]
otherwise $x-\epsilon$ would be an upper bound of $E$. Since $(x_n)$ is monotically increasing, $n\ge N$ implies $x_N\le x_n\le x$, so 
\[x-\epsilon<x_n\le x,\]
which implies $|x_n-x|<\epsilon$. Hence $x_n\to x$.
\end{proof}

\subsection{Limit Superior and Inferior}
For divergent sequences, we have the following definition.
\begin{definition}
Suppose $(x_n)$ is a sequence in $\RR$. We write $x_n\to\infty$ if
\[\forall M\in\RR,\quad\exists N\in\NN,\quad\forall n\ge N,\quad x_n\ge M.\]

Similarly, we write $x_n\to-\infty$ if 
\[\forall M\in\RR,\quad\exists N\in\NN,\quad\forall n\ge N,\quad x_n\le M.\]
\end{definition}

\begin{definition}
Suppose $(x_n)$ is a sequence in $\RR$. Let $E\subset\overline{\RR}$ be the set of all subsequential limits of $(x_n)$ (possibly including $+\infty$ and $-\infty$). Define 
\begin{align*}
\limsup_{n\to\infty}x_n&\coloneqq\sup E,\\
\liminf_{n\to\infty}x_n&\coloneqq\inf E,
\end{align*}
known as the \vocab{limit superior} and \vocab{limit infimum} of $(x_n)$ respectively.
\end{definition}

\begin{remark}
That is, limit superior is the ``largest'' subsequential limit; limit infimum is the ``smallest'' subsequential limit. 
\end{remark}

\begin{remark}
The limit superior and limit infimum exist due to the existence of supremum and infimum in $\overline{\RR}$.
\end{remark}

\begin{lemma}
Equivalently, we can define the limit superior (limit inferior) as the limit of supremum (infimum) of tails:
\begin{align*}
\limsup_{n\to\infty}x_n&=\lim_{n\to\infty}\brac{\sup_{k\ge n}x_k},\\
\liminf_{n\to\infty}x_n&=\lim_{n\to\infty}\brac{\inf_{k\ge n}x_k}.
\end{align*}
\end{lemma}

\begin{proposition}
Suppose $(x_n)$ is a sequence in $\RR$. Then
\begin{enumerate}[label=(\roman*)]
\item $\displaystyle\limsup_{n\to\infty}x_n\in E$;
\item if $\displaystyle x>\limsup_{n\to\infty}x_n$, there exists $N\in\NN$ such that $x_n<x$ for all $n\ge N$.
\end{enumerate}
Moreover, $\displaystyle\limsup_{n\to\infty}x_n$ is the only number that satisfies (i) and (ii).
\end{proposition}

\begin{proof} \
\begin{enumerate}[label=(\roman*)]
\item We consider three cases for the value of $\displaystyle\limsup_{n\to\infty}x_n$:
\begin{itemize}
    \item If $\displaystyle\limsup_{n\to\infty}x_n=+\infty$, then $\sup E=+\infty$, so $E$ is not bounded above. Hence $(x_n)$ is not bounded above, so $(x_n)$ has a subsequence $(x_{n_k})$ such that $x_{n_k}\to\infty$
    \item If $\displaystyle\limsup_{n\to\infty}x_n\in\RR$, then $\sup E\in\RR$, so $E$ is bounded above. Hence at least one subsequential limit exists, so that (i) follows from Theorems 3.7 and 2.28.
    \item If $\displaystyle\limsup_{n\to\infty}x_n=-\infty$, then $\sup E=-\infty$, so $E$ contains only one element, namely $-\infty$. Hence $(x_n)$ has no subsequential limit. Thus for any $M\in\RR$, $x_n>M$ for at most a finite number of values of $n$, so that $x_n\to-\infty$.
\end{itemize}
\item We prove by contradiction.

Suppose there is a number $\displaystyle x>\limsup_{n\to\infty}x_n$ such that $x_n\ge x$ for infinitely many values of $n$. In that case, there is a number $y\in E$ such that $\displaystyle y\ge x>\limsup_{n\to\infty}x_n$, contradicting the definition of $\displaystyle\limsup_{n\to\infty}x_n$.
\end{enumerate}

We now show uniqueness. Suppose, for a contradiction, that two numbers $p$ and $q$ satisfy (i) and (ii). WLOG assume $p<q$. Then choose $x$ such that $p<x<q$. Since $p$ satisfies (i), we have $x_n<x$ for all $n\ge N$. But then $q$ cannot satisfy (i).
\end{proof}

Of course, an analogous result is true for $\displaystyle\liminf_{n\to\infty}x_n$. 

\begin{example}
\begin{itemize}
\item Let $(x_n)$ be a sequence containing all rationals. Then every real number is a subsequential limit, and 
\[\limsup_{n\to\infty}x_n=+\infty,\quad\liminf_{n\to\infty}=-\infty.\]

\item Let $x_n=\dfrac{(-1)^n}{1+\frac{1}{n}}$. Then
\[\limsup_{n\to\infty}x_n=1,\quad\liminf_{n\to\infty}x_n=-1.\]

\item For a seqeunce $(x_n)$ in $\RR$, $x_n\to x$ if and only if
\[\limsup_{n\to\infty}x_n=\liminf_{n\to\infty}x_n=x.\]
\end{itemize}
\end{example}

\begin{proposition}\label{prop:limsup-liminf-comp}
If $a_n\le b_n$ for $n\ge N$ where $N$ is fixed, then
\begin{align*}
\liminf_{n\to\infty}a_n&\le\liminf_{n\to\infty}b_n,\\
\limsup_{n\to\infty}a_n&\le\limsup_{n\to\infty}b_n.
\end{align*}
\end{proposition}

\begin{proposition}[Arithmetic properties] \
\begin{enumerate}[label=(\roman*)]
\item If $k>0$, $\displaystyle\limsup_{n\to\infty}ka_n=k\limsup_{n\to\infty}a_n$.

If $k<0$, $\displaystyle\limsup_{n\to\infty}ka_n=k\liminf_{n\to\infty}a_n$.

\item $\displaystyle\limsup(a_n+b_n)\le\limsup a_n+\limsup b_n$

Moreover, $\displaystyle\limsup_{n\to\infty}(a_n+b_n)$ may be bounded from below as follows:
\[ \limsup_{n\to\infty}(a_n+b_n)\ge\limsup_{n\to\infty}a_n+\liminf_{n\to\infty}b_n.\]

write down the analogous properties for liminf, and to prove (i) and (ii)
\end{enumerate}
\end{proposition}

Now you should try to prove (i) for liminf as well; as for (ii), try to explain why properties (i),(ii) for limsup and property (i) for liminf would imply property (ii) for liminf
\pagebreak

\section{Series}
\begin{definition}[Series]
Given a sequence $(a_n)$, we associate a sequence $(s_n)$, where
\[s_n=\sum_{k=1}^n a_k=a_1+a_2+\cdots+a_n,\]
where the term $s_n$ is called the \emph{$n$-th partial sum}. The sequence $(s_n)$ is often written as
\[\sum_{n=1}^{\infty}a_n,\]
which we call a \vocab{series}.
\end{definition}

\begin{definition}[Convergence of series]
We say that the series \emph{converges} if $s_n\to s$ (the sequence of partial sums converges), and write $\displaystyle\sum_{n=1}^\infty a_n=s$; that is,
\[\forall\epsilon>0,\quad\exists N\in\NN,\quad\forall n\ge N,\quad\absolute{\sum_{k=1}^n a_k-s}<\epsilon.\]
The number $s$ is called the \emph{sum} of the series. If $(s_n)$ diverges, the series is said to \emph{diverge}.
\end{definition}

\begin{notation}
When there is no possible ambiguity, we write $\displaystyle\sum_{n=1}^{\infty}a_n$ simply as $\sum a_n$.
\end{notation}

The Cauchy criterion can be restated in the following form:

\begin{proposition}[Cauchy criterion]
$\sum a_n$ converges if and only if
\[\forall\epsilon>0,\quad\exists N\in\NN,\quad\forall n\ge m\ge N,\quad\absolute{\sum_{k=m}^n a_k}\le\epsilon.\]
\end{proposition}

\subsection{Convergence Tests}
To determine the convergence of a series, apart from using the definition and the Cauchy criterion, we also have the following methods:
\begin{itemize}
\item Divergence test (\cref{lemma:divergence-test})
\item Boundedness of partial sums (\cref{lemma:bounded-partial-sums}, for series of non-negative terms)
\item Comparison test (\cref{lemma:comparison-test})
\item Root test (\cref{lemma:root-test})
\item Ratio test (\cref{lemma:ratio-test})
\item Absolute convergence (\cref{lemma:absolute-convergence})
\end{itemize}

\begin{lemma}[Divergence test]\label{lemma:divergence-test}
If $a_n\not\to0$, then $\sum a_n$ diverges.
\end{lemma}

\begin{proof}
We prove the contrapositive: if $\sum a_n$ converges, then $a_n\to0$.

In the Cauchy criterion, take $m=n$, then $|a_n|\le\epsilon$ for all $n\ge N$.
\end{proof}

\begin{remark}
The converse is not true; a counterexample of the harmonic series.
\end{remark}

\begin{lemma}\label{lemma:bounded-partial-sums}
A series of non-negative terms converges if and only if its partial sums form a bounded sequence.
\end{lemma}

\begin{proof}
Partial sums are monotonically increasing. But bounded monotonic sequences converge.
\end{proof}

\begin{lemma}[Comparison test]\label{lemma:comparison-test}
Consider two sequences $(a_n)$ and $(b_n)$.
\begin{enumerate}[label=(\roman*)]
\item Suppose $|a_n|\le b_n$ for all $n\ge N_0$ (where $N_0$ is some fixed integer). If $\sum b_n$ converges, then $\sum a_n$ converges.
\item Suppose $a_n\ge b_n\ge0$ for all $n\ge N_0$. If $\sum b_n$ diverges, then $\sum a_n$ diverges.
\end{enumerate}
\end{lemma}

\begin{proof} \
\begin{enumerate}[label=(\roman*)]
\item Since $\sum b_n$ converges, by the Cauchy criterion, fix $\epsilon>0$, there exists $N\in\NN$, $N\ge N_0$ such that for $n\ge m\ge N$,
\[\sum_{k=m}^{n}b_k\le\epsilon.\]
By the triangle inequality,
\[\absolute{\sum_{k=m}^{n}a_k}\le\sum_{k=m}^{n}|a_k|\le\sum_{k=m}^{n}b_k\le\epsilon,\]
so $\sum a_n$ converges, by the Cauchy criterion.

\item We prove the contrapositive. If $\sum a_n$ converges, and since $|b_n|\le a_n$ for all $n\ge N_0$, then by (i), $\sum b_n$ converges.
\end{enumerate}
\end{proof}

To employ the comparison test, we need to be familiar with several series whose convergence or divergence is known.

\begin{example}[Geometric series]
A geometric series takes the form
\[\sum_{n=0}^{\infty}x^n.\]

\begin{proposition*} \
\begin{enumerate}[label=(\roman*)]
\item If $|x|<1$, then $\sum x^n$ converges;
\[\sum_{n=0}^{\infty}x^n=\frac{1}{1-x}.\]
\item If $|x|\ge1$, then $\sum x^n$ diverges.
\end{enumerate}
\end{proposition*}

\begin{proof} \
\begin{enumerate}[label=(\roman*)]
\item For $|x|<1$, the $n$-th partial sum is given by
\begin{equation*}\tag{1}
\sum_{k=0}^{n}x^k=1+x+x^2+\cdots+x^n.
\end{equation*}
Multiplying both sides of (1) by $x$ gives
\begin{equation*}\tag{2}
x\sum_{k=0}^{n}x^k=x+x^2+x^3\cdots+x^{n+1}.
\end{equation*}
Taking the difference of (1) and (2),
\[(1-x)\sum_{k=0}^{n}x^k=1-x^{n+1}\]a
nd so
\[\sum_{k=0}^{n}x^k=\frac{1-x^{n+1}}{1-x}.\]
Taking limits $n\to\infty$, the result follows.

\item For $|x|\ge 1$, $x^n\not\to 0$. By the divergence test, $\sum x^n$ diverges.
\end{enumerate}
\end{proof}
\end{example}

\begin{example}[$p$-series]
A $p$-series takes the form
\[\sum_{n=1}^{\infty}\frac{1}{n^p}.\]
To determine the convergence of $p$-series, we first prove the following lemma, which states that a rather ``thin'' subsequence of $(a_n)$ determines the convergence of $\sum a_n$.

\begin{lemma*}[Cauchy condensation test]
Suppose $a_1\ge a_2\ge\cdots\ge0$. Then $\sum a_n$ converges if and only if the series
\[\sum_{k=0}^{\infty}2^ka_{2^k}=a_1+2a_2+4a_4+\cdots\]
converges.
\end{lemma*}

\begin{proof}
Let $s_n$ and $t_k$ denote the $n$-th partial sum of $(a_n)$ and the $k$-th partial sum of $(2^ka_{2^k})$ respectively; that is,
\begin{align*}
s_n&=a_1+a_2+\cdots+a_n,\\
t_k&=a_1+2a_2+\cdots+2^ka_{2^k}.
\end{align*}
We consider two cases:
\begin{itemize}
\item For $n<2^k$, group terms to give
\begin{align*}
s_n&=a_1+a_2+\cdots+a_n\\
&\le a_1+(a_2+a_3)+\cdots+(a_{2^k}+\cdots+a_{2^{k+1}-1})\\
&\le a_1+2a_2+\cdots+2^ka_{2^k}\\
&=t_k.
\end{align*}
By comparison test, if $(t_k)$ converges, then $(s_n)$ converges.
%Thus if $(s_n)$ is unbounded, then $(t_k)$ is unbounded.

\item For $n>2^k$,
\begin{align*}
s_n&\ge a_1+a_2+(a_3+a_4)+\cdots+(a_{2^{k-1}+1}+\cdots+a_{2^k})\\
&\ge\frac{1}{2}a_1+a_2+2a_4+\cdots+2^{k-1}a_{2^k}\\
&=\frac{1}{2}t_k.
\end{align*}
By comparison test, if $(s_n)$ converges, then $(t_k)$ converges.
%Thus if $(t_k)$ is unbounded, then $(s_n)$ is unbounded.
\end{itemize}
\end{proof}

\begin{proposition*}[$p$-test] \
\begin{enumerate}[label=(\roman*)]
\item If $p>1$, $\sum\frac{1}{n^p}$ converges.
\item If $p\le1$, $\sum\frac{1}{n^p}$ diverges.
\end{enumerate}
\end{proposition*}

\begin{proof}
Note that if $p\le0$, then $\frac{1}{n^p}\not\to0$. By the divergence test, $\sum\frac{1}{n^p}$ diverges.

If $p>0$, we want to apply the above lemma. Consider the series
\[\sum_{k=0}^{\infty}2^k\cdot\frac{1}{(2^k)^p}=\sum_{k=0}^{\infty}2^{(1-p)k}=\sum_{k=0}^{\infty}\brac{2^{1-p}}^k,\]
which is a geometric series. Hence the above series converges if and only if $|2^{1-p}|<1$, which holds if and only if $1-p<0$. Then apply the above lemma to conclude the convergence of $\frac{1}{n^p}$.
\end{proof}

\begin{remark}
If $p=1$, the resulting series is known as the \emph{harmonic series} (which diverges). If $p=2$, the resulting series converges, and the sum of this series is $\frac{\pi^2}{6}$ (Basel problem).
\end{remark}
\end{example}

\begin{example}[The number $e$]
Consider the series
\[\sum_{n=0}^{\infty}\frac{1}{n!}.\]
\begin{claim}
The above series converges.
\end{claim}
Consider the $n$-th partial sum:
\begin{align*}
\sum_{k=0}^{n}\frac{1}{k!}
&=\frac{1}{0!}+\frac{1}{1!}+\frac{1}{2!}+\frac{1}{3!}+\cdots+\frac{1}{n!}\\
&\le1+1+\frac{1}{2}+\frac{1}{2^2}+\cdots+\frac{1}{2^{n-1}}\\
&<1+1+\frac{1}{2}+\frac{1}{2^2}+\cdots=3.
\end{align*}
Since the partial sums are bounded (by $3$), and the terms are non-negative, the series converges. Then we can make the following definition for the sum of the series:
\[e\coloneqq\sum_{n=0}^\infty\frac{1}{n!}\]

\begin{proposition*}
$e$ is irrational.
\end{proposition*}

\begin{proof}
Suppose, for a contradiction, that $e$ is rational. Then $e=\frac{p}{q}$, where $p$ and $q$ are positive integers. Let $s_n$ denote the $n$-th partial sum:
\[s_n=\sum_{k=0}^{n}\frac{1}{k!}.\]
Then
\begin{align*}
e-s_n
&=\frac{1}{(n+1)!}+\frac{1}{(n+2)!}+\frac{1}{(n+3)!}+\cdots\\
&<\frac{1}{(n+1)!}\brac{1+\frac{1}{n+1}+\frac{1}{(n+1)^2}+\cdots}\\
&=\frac{1}{(n+1)!}\cdot\frac{n+1}{n}=\frac{1}{n!n}
\end{align*}
and thus
\[0<e-s_n<\frac{1}{n!n}.\]
Taking $n=q$ and multiplying both sides by $q!$ gives
\[0<q!(e-s_q)<\frac{1}{q}.\]
Note that $q!e$ is an integer (by assumption), and
\[q!s_q=q!\brac{1+1+\frac{1}{2!}+\cdots+\frac{1}{q!}}\]
is an integer, so $q!(e-s_n)$ is an integer. Since $q\ge1$, this implies the existence of an integer between $0$ and $1$, which is absurd. Hence we have reached a contradiction.
\end{proof}

\begin{lemma*}
$e$ is equivalent to the following:
\[\lim_{n\to\infty}\brac{1+\frac{1}{n}}^n=e.\]
\end{lemma*}

\begin{proof}
Let
\[s_n=\sum_{k=0}^n\frac{1}{k!},\quad t_n=\brac{1+\frac{1}{n}}^n.\]
By the binomial theorem,
\[t_n=1+1+\frac{1}{2!}\brac{1-\frac{1}{n}}+\frac{1}{3!}\brac{1-\frac{1}{n}}\brac{1-\frac{2}{n}}+\cdots+\frac{1}{n!}\brac{1-\frac{1}{n}}\brac{1-\frac{2}{n}}\cdots\brac{1-\frac{n-1}{n}}.\]
Comparing term by term, we see that $t_n\le s_n$. By \cref{prop:limsup-liminf-comp}, we have that
\[\limsup_{n\to\infty}t_n\le\limsup_{n\to\infty}s_n=e.\]

Next, if $n\ge m$,
\[t_n\ge1+1+\frac{1}{2!}\brac{1-\frac{1}{n}}+\cdots+\frac{1}{m!}\brac{1-\frac{1}{n}}\cdots\brac{1-\frac{m-1}{n}}.\]
Let $n\to\infty$, keeping $m$ fixed. We get
\[\liminf_{n\to\infty}t_n\ge1+1+\frac{1}{2!}+\cdots+\frac{1}{m!},\]
so that
\[s_m\le\liminf_{n\to\infty}t_n.\]
Letting $m\to\infty$, we finally get
\[e\le\liminf_{n\to\infty}t_n.\]

\end{proof}
\end{example}

\begin{lemma}[Root test]\label{lemma:root-test}
Given $\sum a_n$, put $\displaystyle\alpha=\limsup_{n\to\infty}\sqrt[n]{|a_n|}$. Then
\begin{enumerate}[label=(\roman*)]
\item if $\alpha<1$, $\sum a_n$ converges;
\item if $\alpha>1$, $\sum a_n$ diverges;
\item if $\alpha=1$, the test gives no information.
\end{enumerate}
\end{lemma}

\begin{proof} \
\begin{enumerate}[label=(\roman*)]
\item If $\alpha>1$, we can choose $\beta$ so that $\alpha<\beta<1$, and $n\in\NN$ such that for all $n\ge N$,
\[\sqrt[n]{|a_n|}<\beta.\]
by Theorem 3.17(b). Since $0<\beta<1$, $\sum\beta^n$ converges. Hence by the comparison test, $\sum a_n$ converges.

\item If $\alpha>1$, by Theorem 3.17, there is a sequence $(n_k)$ such that
\[\sqrt[n_k]{|a_{n_k}|}\to\alpha.\]
Hence $|a_n|>1$ for infinitely many values of $n$ so that the condition $a_n\to0$, necessary for convergence of $\sum a_n$, does not hold (Theorem 3.23).

\item Consider the series $\sum\frac{1}{n}$ and $\sum\frac{1}{n^2}$. For each of these series $\alpha=1$, but the first diverges, the second converges. Hence the condition that $\alpha=1$ does not give us information on the convergence of a series.
\end{enumerate}
\end{proof}

\begin{lemma}[Ratio test]\label{lemma:ratio-test}
The series $\sum a_n$
\begin{enumerate}[label=(\roman*)]
\item converges if $\displaystyle\limsup_{n\to\infty}\absolute{\frac{a_{n+1}}{a_n}}<1$;
\item diverges if $\displaystyle\absolute{\frac{a_{n+1}}{a_n}}\ge1$ for all $n\ge n_0$, where $n_0$ is some fixed integer.
\end{enumerate}
\end{lemma}

\begin{proof} \
\begin{enumerate}[label=(\roman*)]
\item If $\displaystyle\limsup_{n\to\infty}\absolute{\frac{a_{n+1}}{a_n}}<1$, there exists $\beta<1$ and $N\in\NN$ such tht for all $n\ge N$,
\[\absolute{\frac{a_{n+1}}{a_n}}<\beta.\]
In particular, from $n=N$ to $n=N+p$,
\begin{align*}
|a_{N+1}|&<\beta|a_N|\\
|a_{N+2}|&<\beta|a_{N+1}|<\beta^2|a_N|\\
&\vdots\\
|a_{N+p}|&<\beta^p|a_N|
\end{align*}
Hence for all $n\ge N$,
\[|a_n|<|a_N|\beta^{-N}\cdot\beta^n.\]
Since $\sum\beta^n$ converges, by the comparison test, $\sum a_n$ converges.
\item Suppose $\displaystyle\absolute{\frac{a_{n+1}}{a_n}}\ge1$ for all $n\ge n_0$, where $n_0$ is some fixed integer. Then $|a_{n+1}|\ge|a_n|$ for $n\ge n_0$, and it is easily seen that $a_n\not\to0$, so $\sum a_n$ diverges.
\end{enumerate}
\end{proof}

The series $\sum a_n$ is said to \emph{converge absolutely} if the series $\sum|a_n|$ converges.

\begin{lemma}[Absolute convergence]\label{lemma:absolute-convergence}
If $\sum a_n$ converges absolutely, then $\sum a_n$ converges.
\end{lemma}

\begin{proof}

\end{proof}

\begin{example}[Power series] \
Given a sequence $(c_n)$ of complex numbers, the series
\[\sum_{n=0}^{\infty}c_nz^n\]
is called a \vocab{power series}. The numbers $c_n$ are called the \emph{coefficients} of the series.

In general, the series will converge or diverge, depending on the choice of $z$. More specifically, with every power series there is associated a circle, the circle of convergence, such that $\sum c_nz^n$ converges if $z$ is in the interior of the circle and diverges if $z$ is in the exterior.

\begin{proposition*}
Given the power series $\sum c_nz^n$, let
\[\alpha=\limsup_{n\to\infty}\sqrt[n]{|c_n|},\quad R=\frac{1}{\alpha}.\]
(If $\alpha=0$, $R=+\infty$; if $\alpha=+\infty$, $R=0$.) Then $\sum c_nz^n$
\begin{enumerate}[label=(\roman*)]
\item converges if $|z|<R$,
\item diverges if $|z|>R$.
\end{enumerate}
\end{proposition*}

$R$ is called the \emph{radius of convergence} of $\sum c_nz^n$.

\begin{proof}
Put $a_n=c_nz^n$, then apply the root test:
\begin{align*}
\limsup_{n\to\infty}\sqrt[n]{|a_n|}
&=\limsup_{n\to\infty}\sqrt[n]{|c_nz^n|}\\
&=|z|\limsup_{n\to\infty}\sqrt[n]{|c_n|}\\
&=|z|\alpha\\
&=\frac{|z|}{R}.
\end{align*}
\begin{enumerate}[label=(\roman*)]
\item If $|z|<R$, then $\displaystyle\limsup_{n\to\infty}\sqrt[n]{|a_n|}<1$. By the root test, $\sum c_nz^n$ converges.
\item If $|z|>R$, then $\displaystyle\limsup_{n\to\infty}\sqrt[n]{|a_n|}>1$. By the root test, $\sum c_nz^n$ diverges.
\end{enumerate}
\end{proof}

Further properties of power series will be discussed in \cref{chap:special-functions}.
\end{example}

\subsection{Summation by Parts}
\begin{proposition}[Partial summation formula]
Given two sequences $(a_n)$ and $(b_n)$, put
\[A_n=\sum_{k=0}^{n}a_k\]
if $n\ge0$; put $A_{-1}=0$. Then, if $0\le p\le q$, we have
\[\sum_{n=p}^{q}a_nb_n=\sum_{n=p}^{q-1}A_n(b_n-b_{n+1})+A_qb_q-A_{p-1}b_p.\]
\end{proposition}

\begin{proof}
The RHS can be written as
\begin{align*}
&\sum_{n=p}^{q-1}A_nb_n+A_qb_q-\sum_{n=p}^{q-1}A_nb_{n+1}-A_{p-1}b_p\\
&=\sum_{n=p}^{q}A_nb_n-\sum_{n=p-1}^{q-1}A_nb_{n+1}\\
&=\sum_{n=p}^{q}A_nb_n-\sum_{n=p}^{q}A_{n-1}b_n\\
&=\sum_{n=p}^{q}\brac{A_n-A_{n-1}}b_n\\
&=\sum_{n=p}^{q}a_nb_n
\end{align*}
which is equal to the LHS.
\end{proof}

\begin{proposition}
Suppose the partial sums $A_n$ of $\sum a_n$ form a bounded sequence, $b_0\ge b_1\ge b_2\ge\cdots$, and $\displaystyle\lim_{n\to\infty}b_n=0$. Then $\sum a_nb_n=0$.
\end{proposition}

\begin{proof}

\end{proof}

\begin{proposition}
Suppose $|c_1|\ge|c_2|\ge|c_3|\ge\cdots$, $c_{2m-1}\ge0,c_{2m}\le0$ for $m=1,2,3,\dots$, and $\displaystyle\lim_{n\to\infty}c_n=0$. Then $\sum c_n$ converges.
\end{proposition}

\subsection{Addition and Multiplication of Series}
\begin{proposition}
If $\sum a_n=A$ and $\sum b_n=B$, then
\begin{enumerate}[label=(\roman*)]
\item $\sum(a_n+b_n)=A+B$,
\item $\sum ca_n=cA$ for any fixed $c$.
\end{enumerate}
\end{proposition}

\begin{proof} \
\begin{enumerate}[label=(\roman*)]
\item Let $A_n=\sum_{k=0}^{n}a_k$, $B_n=\sum_{k=0}^{n}b_k$. Then
\[A_n+B_n=\sum_{k=0}^{n}(a_k+b_k).\]
Since $\lim_{n\to\infty}A_n=A$ and $\lim_{n\to\infty}B_n=B$, we see that
\[\lim_{n\to\infty}(A_n+B_n)=A+B.\]
\item 
\end{enumerate}
\end{proof}

Thus two convergent series may be added term by term, and the resulting series converges to the sum of the two series. The situation becomes more complicated when we consider multiplication of two series. To begin with, we have to define the product. This can be done in several ways; we shall consider the so-called ``Cauchy product''. 

\begin{definition}[Cauchy product]
Given $\sum a_n$ and $\sum b_n$, let
\[c_n=\sum_{k=0}^{n}a_k b_{n-k}\quad(n=0,1,2,\dots)\]
We call $\sum c_n$ the \emph{product} of the two given series.
\end{definition}

This definition may be motivated as follows. If we take two power series $\sum a_nz^n$ and $\sum b_nz^n$, multiply them term by term, and collect terms containing the same power of $z$, we get
\begin{align*}
\sum_{n=0}^{\infty}a_nz^n\cdot\sum_{n=0}^{\infty}b_nz^n
&=\brac{a_0+a_1z+a_2z^2+\cdots}\brac{b_0+b_1z+b_2z^2+\cdots}\\
&=a_0b_0+(a_0b_1+a_1b_0)z+(a_0b_2+a_1b_1+a_2b_0)z^2+\cdots\\
&=c_0+c_1z+c_2z^2.
\end{align*}
Setting $z=1$, we arrive at the above definition. 

\begin{theorem}[Mertens]
Suppose $\sum a_n=A$, $\sum b_n=B$, $\sum a_n$ converges absolutely. Then their Cauchy product converges to $AB$.
\end{theorem}

That is, the product of two convergent series converges, and to the right value, if at least one of the two series converges absolutely.

\begin{proof}
Let $\displaystyle A_n=\sum_{k=0}^{n}a_k$, $\displaystyle B_n=\sum_{k=0}^{n}b_k$, $\displaystyle C_n=\sum_{k=0}^{n}c_k$. Also let $\beta_n=B_n-B$. Then 
\begin{align*}
C_n&=a_0b_0+(a_0b_1+a_1b_0)+\cdots+(a_0b_n+a_1b_{n-1}+\cdots+a_nb_0)\\
&=a_0B_n+a_1B_{n-1}+\cdots+a_nB_0\\
&=a_0(B+\beta_n)+a_1(B+\beta_{n-1})+\cdots+a_n(B+\beta_0)\\
&=A_nB+a_0\beta_n+a_1\beta_{n-1}+\cdots+a_n\beta_0.
\end{align*}
Let
\[\gamma_n=a_0\beta_n+a_1\beta_{n-1}+\cdots+a_n\beta_0.\]
We wish to show that $C_n\to AB$. Since $A_nB\to AB$, it suffices to show that $\displaystyle\lim_{n\to\infty}\gamma_n=0$.

Let 
\[\alpha=\sum_{n=0}^{\infty}|a_n|.\]
Let $\epsilon>0$. Since $B_n\to B$, $\beta_n\to0$. Hence we can choose $N\in\NN$ such that for all $n\ge N$, $|\beta_n|\le\epsilon$, in which case
\begin{align*}
|\gamma_n|&=|\beta_0a_n+\cdots+\beta_Na_{n-N}|+|\beta_{N+1}a_{n-N}a_{n-N-1}+\cdots+\beta_na_0|\\
&\le|\beta_0a_n+\cdots+\beta_Na_{n-N}|+\epsilon\alpha.
\end{align*}
Keeping $N$ fixed, and letting $n\to\infty$, we get
\[\limsup_{n\to\infty}|\gamma_n|\le\epsilon\alpha,\]
sine $a_k\to0$ as $k\to\infty$. Since $\epsilon$ is arbitrary, we have $\displaystyle\lim_{n\to\infty}\gamma_n=0$, as desired.
\end{proof}

\begin{theorem}[Abel]
Let the series $\sum a_n$, $\sum b_n$, $\sum c_n$ converge to $A$, $B$, $C$ respectively, and $\sum c_n$ is the Cauchy product of $\sum a_n$ and $\sum b_n$. Then $C=AB$.
\end{theorem}

\subsection{Rearrangements}
\begin{definition}[Rearrangement]
Let $(k_n)$ be a sequence in which every positive integer appears once and only once. Putting
\[a_n^\prime=a_{k_n}\quad(\forall n\in\NN)\]
we say that $\sum a_n^\prime$ is a \emph{rearrangement} of $\sum a_n$.
\end{definition}

\begin{proposition}
Let $\sum a_n$ be a series of real numbers which converges, but not absolutely. Suppose $-\infty\le\alpha\le\beta\le\infty$. Then there exists a rearrangement $\sum a_n^\prime$ with partial sums $s_n^\prime$ such that
\[\liminf_{n\to\infty}s_n^\prime=\alpha,\quad\limsup_{n\to\infty}s_n^\prime=\beta.\]
\end{proposition}

\begin{proposition}
If $\sum a_n$ is a series of complex numbers which converges absolutely, then every rearrangement of $\sum a_n$ converges, and they all converge to the same sum.
\end{proposition}
\pagebreak

\section*{Exercises}
\begin{prbm}
Show the following:
\begin{enumerate}[label=(\roman*)]
\item $\displaystyle\lim_{n\to\infty}\frac{1}{n^p}=0$ ($p>0$)
\item $\displaystyle\lim_{n\to\infty}\sqrt[n]{p}=1$ ($p>0$)
\item $\displaystyle\lim_{n\to\infty}\sqrt[n]{n}=1$
\item $\displaystyle\lim_{n\to\infty}\frac{n^\alpha}{(1+p)^n}=0$ ($p>0$, $\alpha\in\RR$)
\item $\displaystyle\lim_{n\to\infty}x^n=0$ ($|x|<1$)
\end{enumerate}
\end{prbm}

\begin{solution} \
\begin{enumerate}[label=(\roman*)]
\item Let $\epsilon>0$ be given. Take $N=\floor{\brac{\frac{1}{\epsilon}}^\frac{1}{p}}+1$. Then $n\ge N$ implies
\[\absolute{\frac{1}{n^p}-0}=\frac{1}{n^p}\le\frac{1}{N^p}<\frac{1}{\brac{\brac{\frac{1}{\epsilon}}^\frac{1}{p}}^p}=\epsilon.\]
\item We need to consider the cases when $p>1$, $p=1$, and $0<p<1$.

If $p>1$, 

\item 
\item 
\item 
\end{enumerate}
\end{solution}

\begin{prbm}
Let $(x_n)$ be a sequence in $\RR$, let $\alpha\ge2$ be a constant. Define the sequence $(y_n)$ as follows:
\[y_n=x_n+\alpha x_{n+1}\quad(n=1,2,\dots)\]
Show that if $(y_n)$ is convergent, then $(x_n)$ is also convergent.
\end{prbm}

\begin{prbm}[Contractive sequence]
A sequence $(x_n)$ in $\RR$ is \emph{contractive} if there exists $k\in[0,1)$ such that
\[|x_{n+2}-x_{n+1}|\le k|x_{n+1}-x_n|\quad(\forall n\in\NN)\]
Show that every contractive sequence is convergent.
\end{prbm}

\begin{solution}
By induction on $n$, we have
\[|a_{n+1}-a_n|\le k^{n-1}|a_2-a_1|\quad(\forall n\in\NN)\]
Thus
\begin{align*}
|a_{n+p}-a_{n}|&\le|a_{n+1}-a_{n}|+|a_{n+2}-a_{n+1}|+\cdots+|a_{n+p}-a_{n+p-1}|\\ 
&\le\brac{k^{n-1}+k^{n}+\cdots+k^{n+p-2}}|a_{2}-a_{1}|\\ 
&\le k^{n-1}\brac{1+k+k^{2}+\cdots+k^{p-1}}|a_{2}-a_{1}|\\ 
&\le\frac{k^{n-1}}{1-k}|a_{2}-a_{1}|
\end{align*}
for all $n,p\in\NN$. Since $k^{n-1}\to0$ as $n\to\infty$ (independently of $p$), this implies $(a_n)$ is a Cauchy sequence (in $\RR$) and, hence, it is convergent.
\end{solution}

\begin{prbm}
The sequence $(x_n)$ is recursively defined by
\[\begin{cases}
x_0=\sqrt{2},\\
x_{n+1}=\sqrt{2+x_n}\quad n\ge0.
\end{cases}\]
Show that $(x_n)$ converges.
\end{prbm}

\begin{proof}
We first prove by induction that $x_n\le x_{n+1}\le 2$ for all $n\in\NN$. For $n=0$,
\[x_0=\sqrt{2}\le\sqrt{2+\sqrt{2}}=x_1\le\sqrt{2+\sqrt{4}}=2.\]
If $x_{n-1}\le x_n\le 2$, then
\[x_n=\sqrt{2+x_{n-1}}\le\sqrt{2+x_n}=x_{n+1}\le\sqrt{2+2}=2.\]
Hence $(x_n)$ is monotonically increasing and bounded above by $2$. By the monotone convergence theorem, $(x_n)$ converges; let $x_n\to x$. Applying the limit on both sides of $x_{n+1}=\sqrt{2+x_n}$,
\begin{align*}
\lim_{n\to\infty}x_{n+1}&=\lim_{n\to\infty}\sqrt{2+x_n}\\
x&=\sqrt{2+x}\\
x&=2\text{ or }1
\end{align*}
Since all $x_n\ge0$, we must have $x=2$.
\end{proof}
%    \chapter{Continuity}\label{chap:real-analysis_continuity}
\section{Limit of Functions}
Let $(X,d_X)$ be a metric space, let $E\subseteq X$. Then the metric $d_X$ induces a metric on $E$. Now consider a mapping $f$ (or function) from $E$ into another metric space $(Y,d_Y)$.

In particular, if $Y=\RR$, $f$ is called a \textbf{real-valued function}; and if $Y=\CC$, $f$ is called a \textbf{complex-valued function}.

\begin{definition}[Limit of function]\label{defn:limit-function}
Consider a limit point $p\in E$. We say $\displaystyle\lim_{x\to p}f(x)=q$\index{limit of function} if there exists a point $q\in Y$ such that
\[\forall\epsilon>0,\quad\exists\delta>0,\quad\forall x\in E,\quad0<d_X(x,p)<\delta\implies d_Y\brac{f(x),q}<\epsilon.\]
\end{definition}

In words, this means no matter what $B_\epsilon(q)$ we are given, we can always find a $B_\delta(p)$ succh that $f\brac{\overline{B}_\delta(p)\cap E}\subset B_\epsilon(q)$.

We can recast this definition in terms of limits of sequences:
\begin{theorem}\label{limit-func-seq}
$\displaystyle\lim_{x\to p}f(x)=q$ if and only if $\displaystyle\lim_{n\to\infty}f(p_n)=q$ for every sequence $(p_n)$ in $E$ such that $p_n \neq p$, $\displaystyle\lim_{n\to\infty}p_n=p$.
\end{theorem}

\begin{proof} \

\fbox{$\implies$} Suppose $\displaystyle\lim_{x\to p}f(x)=q$. Choose $(p_n)$ in $E$ satisfying $p_n \neq p$ and $\displaystyle\lim_{n\to\infty}p_n=p$. We now want to show that $\displaystyle\lim_{n\to\infty}f(p_n)=q$.

Let $\epsilon>0$ be given. Since $\displaystyle\lim_{x\to p}f(x)=q$, there exists $\delta>0$ such that 
\[\forall x\in E,\quad0<d_X(x,p)<\delta\implies d_Y\brac{f(x),q}<\epsilon.\]
Also, since $\displaystyle\lim_{n\to\infty}p_n=p$, there exists $N\in\NN$ such that
\[\forall n\ge N,\quad 0<d_X(p_n,p)<\delta.\]
Thus for $n\ge N$, we have $d_Y\brac{f(p_n),q}<\epsilon$, which shows that $\displaystyle\lim_{n\to\infty}f(p_n)=q$.

\fbox{$\impliedby$} We now prove the reverse direction by contrapositive. Suppose $\displaystyle\lim_{x\to p}f(x)\neq q$. Then
\[\exists\epsilon>0,\quad\forall\delta>0,\quad\exists x\in E,\quad d_Y\brac{f(x),q}\ge\epsilon\quad\text{and}\quad0<d_X(x,p)<\delta.\]
Taking $\delta_n=\frac{1}{n}$ ($n=1,2,\dots$), we thus find a sequence in $E$ satisfying $p_n \neq p$ and $\displaystyle\lim_{n\to\infty}p_n=p$ for which $\displaystyle\lim_{n\to\infty}f(p_n)\neq q$.
\end{proof}

\begin{corollary}
If $f$ has a limit at $p$, this limit is unique.
\end{corollary}

\begin{proof}
This follows from  and \cref{limit-func-seq}.
\end{proof}

\begin{proposition}
Suppose $E\subseteq X$, limit point $p\in E$, $f,g:E\to\RR$. Let $\displaystyle\lim_{x\to p}f(x)=A$ and $\displaystyle\lim_{x\to p}g(x)=B$. Then
\begin{enumerate}[label=(\roman*)]
\item $\displaystyle\lim_{x\to p}(f+g)(x)=A+B$
\item $\displaystyle\lim_{x\to p}(fg)(x)=AB$
\item $\displaystyle\lim_{x\to p}\brac{\frac{p}{q}}(x)=\frac{A}{B}$ ($B\neq0$)
\end{enumerate}
\end{proposition}

\begin{proof}
By the same proofs as for sequences, limits are unique, and in $\RR$ they add/multiply/divide as expected.
\end{proof}

\section{Continuous Functions}
Consider metric spaces $(X,d_X)$ and $(Y,d_Y)$, let $E\subseteq X$.

\begin{definition}[Continuity]
We say that $f:E\to Y$ is \vocab{continuous}\index{continuity} at $p\in E$ if 
\[\forall\epsilon>0,\quad\exists\delta>0,\quad\forall x\in X,\quad d_X(x,p)<\delta\implies d_Y\brac{f(x),f(p)}.\]
We say $f$ is continuous in $E$ if it is continuous at every point of $E$.
\end{definition}

\begin{lemma}
Assume $p$ is a limit point of $E$. Then $f$ is continuous at $p$ if and only if $\displaystyle\lim_{x\to p}f(x)=f(p)$.
\end{lemma}

\begin{proof}
Compare Definitions 4.1 and 4.5.
\end{proof}

\begin{theorem}[Sequential criterion for continuity]
$f:E\subseteq X\to Y$ is continuous at $p\in E$ if and only if for every sequence $(x_n)$ in $E$ that converges to $p$, the sequence $\brac{f(x_n)}$ converges to $f(p)$.
\end{theorem}

\begin{proof}
The sequential definition of continuity follows almost directly from the sequential definition of limits.
\end{proof}

As for real-valued functions, the definition of continuity can be phrased in terms of limits.

\begin{corollary}
$f:X\to\RR$ is continuous at $p\in X$ if and only if for any sequence $(x_n)$ with $\displaystyle\lim_{n\to\infty}x_n=p$, we have $\displaystyle\lim_{n\to\infty}f(x_n)=f(p)$.
\end{corollary}

We now consider the composition of functions. The following result shows that a continuous function of a continuous function is continous.

\begin{proposition}
Suppose $X,Y,Z$ are metric spaces, $E\subseteq X$, $f:E\to Y$, $g$ maps the range of $f(E)$ into $Z$, $h:E\to Z$ defined by
\[h(x)=g\circ f(x)\quad(x\in E)\]
If $f$ is continuous at $p\in E$, and $g$ is continuous at $f(p)$, then $h$ is continuous at $p$.
\end{proposition}

\begin{proof}
Let $\epsilon>0$ be given. Since $g$ is continous at $f(p)$, there exists $\eta>0$ such that for all $y\in f(E)$,
\[d_Y\brac{y,f(p)}<\eta\implies d_Z\brac{g(y),g\brac{f(p)}}<\epsilon\]
Since $f$ is continuous at $p$, there exists $\delta>0$ such that for all $x\in E$,
\[d_X\brac{x,p}<\delta\implies d_Y\brac{f(x),f(p)}<\eta\]
It follows that for all $x\in E$,
\[d_X(x,p)<\delta\implies d_Z\brac{h(x),h(p)}=d_Z\brac{g\brac{f(x)},g\brac{f(p)}}<\epsilon\]
Thus $h$ is continuous at $p$. 
\end{proof}

\begin{proposition}
$f:X\to Y$ is continuous on $X$ if and only if $f^{-1}(V)$ is open in $X$ for every open set $V\subseteq Y$.
\end{proposition}

\begin{proof} \

\fbox{$\implies$} Suppose $f$ is continuous on $X$, $V\subseteq Y$ is open. We have to show that every point of $f^{-1}(V)$ is an interior point of $f^{-1}(V)$.

So, suppose $p\in X$ and $f(p)\in V$. Since $V$ is open, there exists $\epsilon>0$ such that $y\in V$ if $d_Y\brac{f(p),y}<\epsilon$; and since $f$ is continuous at $p$, there exists $\delta>0$ such that $d_Y\brac{f(x),f(p)}<\epsilon$ if $d_X(x,p)<\delta$. Thus $x\in f^{-1}(V)$ as soon as $d_X(x,p)<\delta$.

\fbox{$\impliedby$} Conversely, suppose $f^{-1}(V)$ is open in $X$ for every open set $V\subseteq Y$. Fix $p\in X$ and $\epsilon>0$, let $V=\{y\in Y\mid d_Y\brac{y,f(p)}\}<\epsilon$. Then $V$ is open; hence $f^{-1}(V)$ as soon as $d_X(p,x)<\delta$. But if $x\in f^{-1}(V)$, then $f(x)\in V$, so that $d_Y\brac{f(x),f(p)}<\epsilon$.
\end{proof}

\begin{corollary}
$f:X\to Y$ is continuous if and only if $f^{-1}(C)$ is closed in $X$ for every closed set $C\subseteq Y$.
\end{corollary}

\begin{proof}
This follows from the above result, since a set is closed if and only if its complement is open, and since $f^{-1}(E^c)=[f^{-1}(E)]^c$ for every $E\subseteq Y$.
\end{proof}

\begin{proposition}
Let $f,g:X\to\RR$. Then $f+g$, $fg$, and $\frac{f}{g}$ ($g(x)\neq0$ for all $x\in X$) are continuous on $X$.
\end{proposition}

\begin{proof}
At isolated points of X there is nothing to prove. At limit points, the statement follows from Theorems 4.4 and 4.6
\end{proof}




\subsection{Continuity of linear functions in normed spaces}
A great deal of power comes from considering the set of all functions on a space satisfying some property, such as continuity, as a metric space in its own right. In this section we consider some important examples of such spaces.

We begin with the space of bounded real-valued functions on a set $X$. At this stage we assume nothing about $X$.

\begin{definition}[Space of bounded real-valued functions]
If $X$ is any set, we define $B(X)$ to be the space of functions $f:X\to\RR$ for which $f(X)=\{f(x)\mid x\in X\}$ is bounded. If $f\in B(X)$, define $\norm{f}_\infty=\sup_{x\in X}|f(x)|$.
\end{definition}

\begin{lemma}
For any set $X$, $B(X)$ is a vector space, and $\norm{\cdot}_\infty$ is a norm.
\end{lemma}

\begin{proof}

\end{proof}

Now we turn to the space of continuous real-valued functions, $C(X)$. To make sense of what this means we now need $X$ to be a metric space.

\begin{definition}
Let $X$ be a metric space. We write $C(X)$ for the space of all continuous functions $f:X\to\RR$.
\end{definition}



\section{Continuity and Compactness}
Assume $(X,d_X)$ and $(Y,d_Y)$ are metric spaces.

\begin{definition}[Bounded]
$f:E\to\RR^n$ is said to be \vocab{bounded} if there exists $M\in\RR$ such that $|f(x)|\le M$ for all $x\in E$.
\end{definition}

\begin{theorem}
Suppose $f:X\to Y$ is continuous. Then for any compact subset $K\subseteq X$, the image set $f(K)$ is a compact subset of $Y$.
\end{theorem}

\begin{proof}
We prove it by definition. Assume $\{V_i\mid i\in I\}$ is an open cover of $f(K)$. By the continuity of $f$ and 
\end{proof}

\begin{theorem}[Extreme Value Theorem]
A continuous function on a compact set attains its maximum and minimum values.
\end{theorem}

\begin{definition}[Uniform continuity]
Let $(X,d_X)$ and $(Y,d_Y)$ be metric spaces, let $E\subseteq X$. We say that $f:E\to Y$ is \vocab{uniformly continuous} if
\[\forall\epsilon>0,\quad\exists\delta>0,\quad\forall x,y\in E,\quad d_X(x,y)<\delta\implies d_Y\brac{f(x),f(y)}<\epsilon.\]
\end{definition}

Let us consider the differences between the concepts of continuity and of uniform continuity. First, uniform continuity is a property of a function on a set, whereas continuity can be defined at a single point. To ask whether a given function is uniformly continuous at a certain point is meaningless. Second, if $f$ is continuous on $X$, then it is possible to find, for each $\epsilon>0$ and for each point $p\in X$, a number $\delta>0$ having the property specified in Definition 4.5. This $\delta$ depends $\epsilon$ \emph{and} on $p$. If $f$ is, however, uniformly continuous on $X$, then it is possible, for each $\epsilon>0$, to find \emph{one} number $\delta>0$ which will do for \emph{all} points $p\in X$.

Evidently, every uniformly continuous function is continuous. That the two concepts are equivalent on compact sets follows from the next theorem. 

\begin{proposition}
Let $f:E\subseteq X\to Y$ be continuous. Then $f$ is uniformly continous.
\end{proposition}

\begin{proof}

\end{proof}

\section{Continuity and Connectedness}
\begin{proposition}
If $f:X\to Y$ is continous, and if $E\subseteq X$ is connected, then $f(E)$ is connected.
\end{proposition}

\begin{proof}

\end{proof}

\begin{theorem}[Intermediate Value Theorem]
Let $f:[a,b]\to\RR$ be continuous. If $f(a)<f(b)$ and $f(a)<c<f(b)$, then there exists $x\in(a,b)$ such that $f(x)=c$.
\end{theorem}

\begin{proof}

\end{proof}

\section{Discontinuities}
Let $f:X\to Y$. If $f$ is not continuous at $x\in X$, we say that $f$ is discontinuous at $x$, or that $f$ has a discontinuity at $x$.

If $f$ is defined on an interval or a segment, it is customary to divide discontinuities into two types. Before giving this classification, we have to define the \vocab{right-hand} and the \vocab{left-hand limits} of $f$ at $x$, denoted by $f(x+)$ and $f(x-)$ respectively.

\begin{definition}[Right-hand and left-hand limits]
Let $f:(a,b)\to\RR$. Consider any point $x$ such that $a\le x<b$. 
\end{definition}

\begin{definition}[Discontinuities]
Let $f:[a,b]\to\RR$. If $f$ is discontinuous at $x$, and if $f(x+)$ and $f(x-)$ exist, then $f$ is said to have a \vocab{discontinuity of the first kind}, or a \vocab{simple discontinuity}, at $x$. Otherwise the discontinuity is said to be of the \vocab{second kind}.
\end{definition}

There are two ways in which a function can have a simple discontinuity: either  

\section{Monotonic Functions}
\begin{proposition}
Let $f:[a,b]\to\RR$ be monotonically increasing. Then $f(x+)$ and $f(x-)$ exist for all $x\in(a,b)$; more precisely,
\[\sup_{t\in(a,x)}f(t)=f(x-)\le f(x)\le f(x+)=\inf_{t\in(x,b)}f(t).\]
Furthermore, if $a<x<y<b$, then
\[f(x+)\le f(y-).\]
\end{proposition}

Analogous results evidently hold for monotically decreasing functions.

\section{Infinite Limits and Limits at Infinity}
\begin{definition}
For $c\in\RR$, the set $\{x\in\RR\mid x>c\}$ is called a neighbourhood of $+\infty$ and is written $(c,+\infty)$. Similarly, the set $(-\infty,c)$ is a neighbourhood of $-\infty$.
\end{definition}

\begin{definition}
Let $f:E\subset\RR\to\RR$. We say that $\displaystyle\lim_{t\to x}f(t)=A$ where $A$ and $x$ are in the extended real number system, if for every neighbourhood of $U$ of $A$ there is a neighbourhood $V$ of $x$ such that $V\cap E$ is not empty, and such that $f(t)\in U$ for all $t\in V\cap E$, $t\neq x$.
\end{definition}
%    \chapter{Differentiation}\label{chap:differentiation}
\section{The Derivative of A Real Function}
\begin{definition}[Derivative]
Suppose $f:[a,b]\to\RR$. For any $x\in[a,b]$, if the limit
\[\lim_{t\to x}\frac{f(t)-f(x)}{t-x}\quad(a<t<b,t\neq x)\]
exists, we call it $f^\prime$, known as the \vocab{derivative} of $f$.

If $f^\prime$ is defined at a point $x$, we say that $f$ is \vocab{differentiable} at $x$; If $f^\prime$ is defined at every point of a set $E\subseteq[a,b]$, we say that $f$ is differentiable on $E$.
\end{definition}

\begin{lemma}[Differentiability implies continuity]\label{lemma:diff-cont}
If $f:[a,b]\to\RR$ is differentiable at $x\in[a,b]$, then $f$ is continuous at $x$.
\end{lemma}

\begin{proof}
\begin{align*}
\lim_{t\to x}[f(t)-f(x)]
&=\lim_{t\to x}\sqbrac{\frac{f(t)-f(x)}{t-x}\cdot(t-x)}\\
&=\lim_{t\to x}\frac{f(t)-f(x)}{t-x}\cdot\lim_{t\to x}(t-x)\\
&=f^\prime(x)\cdot0=0.
\end{align*}
Since $\displaystyle\lim_{t\to x}f(t)=f(x)$, $f$ is continuous at $x$.
\end{proof}

\begin{remark}
The converse of \cref{lemma:diff-cont} is not true; it is easy to construct continuous functions which fail to be differentiable at isolated points.
\end{remark}

\begin{example}[Weierstrass function]
Let $0<a<1$, let $b>1$ be an odd integer, and $ab>1+\frac{3}{2}\pi$. Then the function
\[W(x)=\sum_{n=0}^{\infty}a^n\cos(b^n\pi x)\]
is continuous and nowhere differentiable on $\RR$.
\end{example}

\begin{notation}
If $f$ has a derivative $f^\prime$ on an interval, and if $f^\prime$ is itself differentiable, we denote the derivative of $f^\prime$ by $f^{\prime\prime}$, and call $f^{\prime\prime}$ the \textbf{second derivative} of $f$. Continuing in this manner, we obtain functions
\[f,f^\prime,f^{\prime\prime},f^{(3)},f^{(4)},\dots,f^{(n)},\]
each of which is the derivative of the preceding one. $f^\prime$ is called the $n$-th derivative (or the derivative or order $n$) of $f$.
\end{notation}

\begin{remark}
In order for $f^{(n)}(x)$ to exist at a point $x$, $f^{(n-1)}(t)$ must exist in a neighbourhood of $x$ (or a one-sided neighbourhood, if $x$ is an endpoint of the interval on which $f$ is defined), and $f^{(n-1)}(x)$ must be differentiable at $x$.
\end{remark}

\begin{notation}
$C_1[a,b]$ denotes the set of differentiable functions over $[a,b]$ whose derivative is continuous. More generally, $C_n[a,b]$ denotes the set of functions whose $n$-th derivative is continuous. In particular, $C_0[a,b]$ is the set of continuous functions over $[a,b]$.
\end{notation}

Later on when we talk about properties of differentiation such as the intermediate value theorems, we usually have the following requirement on the function:
\begin{quote}
$f$ is a continuous function on $[a,b]$ which is differentiable in $(a,b)$.
\end{quote}

\begin{lemma}[Differentiation rules]
Suppose $f,g:[a,b]\to\RR$ are differentiable at $x\in[a,b]$. Then
\begin{enumerate}[label=(\roman*)]
\item Scalar multiplication: for $\alpha\in\RR$, $\alpha f$ is differentiable at $x$, and
\[(\alpha f)^\prime(x)=\alpha f^\prime(x).\]
\item Addition: $f\pm g$ is differentiable at $x$, and
\[(f\pm g)^\prime(x)=f^\prime(x)\pm g^\prime(x).\]
\item Product rule: $fg$ is differentiable at $x$, and
\[(fg)^\prime(x)=f^\prime(x)g(x)+f(x)g^\prime(x).\]
\item Quotient rule: $f/g$ (when $g(x)\neq0$) is differentiable at $x$, and
\[\brac{\frac{f}{g}}^\prime(x)=\frac{f^\prime(x)g(x)-f(x)g^\prime(x)}{g(x)^2}.\]
\end{enumerate}
\end{lemma}

\begin{proof} \
\begin{enumerate}[label=(\roman*)]
\item 
\item \begin{align*}
\frac{(f+g)(t)-(f+g)(x)}{t-x}
&=\frac{f(t)+g(t)-f(x)-g(x)}{t-x}\\
&=\frac{f(t)-f(x)}{t-x}+\frac{g(t)-f(x)}{t-x}
\end{align*}
Taking limits $t\to x$, the first term equals to $f^\prime(x)$, and the second term equals to $g^\prime(x)$. The case for subtraction is analogous.

\item \begin{align*}
\frac{(fg)(t)-(fg)(x)}{t-x}
&=\frac{f(t)g(t)-f(x)g(x)}{t-x}\\
&=\frac{\sqbrac{f(t)-f(x)}g(t)+f(x)\sqbrac{g(t)-g(x)}}{t-x}\\
&=\frac{f(t)-f(x)}{t-x}\cdot g(t)+f(x)\cdot\frac{g(t)-g(x)}{t-x}\\
&=f^\prime(x)g(x)+f(x)g^\prime(x)
\end{align*}
Taking limits $t\to x$, the first term equals to $f^\prime(x)g(x)$, and the second term equals to $f(x)g^\prime(x)$, so we are done.

\item Similarly,
\[\frac{\brac{\frac{f}{g}}(t)-\brac{\frac{f}{g}}(x)}{t-x}=\frac{1}{g(t)g(x)}\sqbrac{g(x)\cdot\frac{f(t)-f(x)}{t-x}-f(x)\cdot\frac{g(t)-g(x)}{t-x}}\]
Taking limits $t\to x$, the result immediately follows.
\end{enumerate}
\end{proof}

By induction, we can obtain the following extensions of the differentiation rules.

\begin{corollary}
Suppose $f_1,f_2,\dots,f_n:[a,b]\to\RR$ are differentiable at $x\in[a,b]$. Then
\begin{enumerate}[label=(\roman*)]
\item $f_1+f_2+\cdots+f_n$ is differentiable at $x$, and
\[(f_1+f_2+\cdots+f_n)^\prime(x)={f_1}^\prime(x)+{f_2}^\prime(x)+\cdots+{f_n}^\prime(x).\]
\item $f_1f_2\cdots f_n$ is differentiable at $x$, and
\begin{align*}
(f_1f_2\cdots f_n)^\prime(x)
=&{f_1}^\prime(x)f_2(x)\cdots f_n(x)+f_1(x){f_2}^\prime(x)\cdots f_n(x)\\
&+\cdots+f_1(x)f_2(x)\cdots {f_n}^\prime(x).
\end{align*}
\end{enumerate}
\end{corollary}

\begin{theorem}[Chain rule]
Suppose $f$ is continuous on $[a,b]$, $f^\prime(x)$ exists at $x\in[a,b]$, $g$ is defined on $I$ that contains $f([a,b])$, and $g$ is differentiable at $f(x)$. Then the composition
\[h(x)\coloneqq g\circ f(x)=g\brac{f(x)}:[a,b]\to\RR\]
is differentiable at $x$, and the derivative at $x$ can be calculated as
\[h^\prime(x)=g^\prime\brac{f(x)}f^\prime(x).\]
\end{theorem}

\begin{proof}
Let $y=f(x)$. By the definition of the derivative, we have
\begin{equation*}\tag{1}
f(t)-f(x)=(t-x)[f^\prime(x)+u(t)]
\end{equation*}
\begin{equation*}\tag{2}
g(s)-g(y)=(s-y)[g^\prime(y)+v(s)]
\end{equation*}
where $t\in[a,b]$, $s\in I$, $\lim_{t\to x}u(t)=0$, $\lim_{s\to y}v(s)=0$.

Let $s=f(t)$. Using first (2) and then (1), we obtain
\begin{align*}
h(t)-h(x)
&=g\brac{f(t)}-g\brac{f(x)}\\
&=[f(t)-f(x)]\cdot[g^\prime(y)+v(s)]\\
&=(t-x)[f^\prime(x)+u(t)][g^\prime(y)+v(s)],
\end{align*}
or, if $t\neq x$,
\[\frac{h(t)-h(x)}{t-x}=[g^\prime(y)+v(s)][f^\prime(x)+u(t)].\]
Letting $t\to x$, we see that $s\to y$, by the continuity of $f$, so that the RHS of the above equation tends to $g^\prime(y)f^\prime(x)$, thus giving us the desired result.
\end{proof}

\begin{example}
One family of pathological examples in calculus is functions of the form
\[f(x)=x^p\sin\frac{1}{x}.\]
For $p=1$, the function is continuous and differentiable everywhere other than $x=0$; for $p=2$, the function is differentiable everywhere, but the derivative is discontinuous.
\end{example}

\section{Mean Value Theorems}
Let $(X,d)$ be a metric space.

\begin{definition}[Local maximum and minimum]
We say that $f:X\to\RR$ has
\begin{enumerate}[label=(\roman*)]
\item a \vocab{local maximum} at $x_0\in X$ if there exists $\delta>0$ such that $f(x_0)\ge f(x)$ for all $x\in B_\delta(x_0)$;
\item a \vocab{local minimum} at $x_0\in X$ if there exists $\delta>0$ such that $f(x_0)\le f(x)$ for all $x\in B_\delta(x_0)$.
\end{enumerate}
\end{definition}

\begin{lemma}[Fermat's theorem]
Suppose $f:[a,b]\to\RR$. If $f$ has a local maximum or minimum at $x_0\in(a,b)$, and if $f^\prime(x_0)$ exists, then $f^\prime(x_0)=0$.
\end{lemma}

\begin{proof}
If $f$ is not differentiable at $x_0$, we are done. Assume now $f$ is differentiable at $x_0$ and $x_0$ is a local maximum. By definition, there exists $\delta>0$ such that $f(x_0)\le f(x)$, for all $x\in B_\delta(x_0)$. Then
\[ \frac{f(x)-f(x_0)}{x-x_0}\begin{cases}
\ge0 & x_0-\delta<x<x+\delta\\
\le0 & x_0<x<x_0+\delta
\end{cases} \]
Since $f^\prime(x_0)$ exists, we have
\[ f^\prime(x_0-)\ge0, \quad f^\prime(x_0+)\le0, \]
but we know that $f^\prime(x_0-)=f^\prime(x_0+)=f^\prime(x_0)$ since $f$ is differentiable at $x_0$. Hence $f^\prime(x_0)=0$.
\end{proof}

\begin{theorem}[Rolle's theorem]\label{thrm:rolle}
If $f$ is continuous on $[a,b]$, differentiable in $(a,b)$ and $f(a)=f(b)$, then there exists $c\in(a,b)$ such that 
\[ f^\prime(c)=0. \]
\end{theorem}

\begin{proof}
Let $h(x)$ be a function defined on $[a,b]$ where $h(a)=h(b)$.

The idea is to show that $h$ has a local maximum/minimum, then by Fermat's Theorem this will then be the stationary point that we're trying to find.

First note that $h$ is continuous on $[a,b]$, so $h$ must have a maximum $M$ and a minimum $m$.

If $M$ and $m$ were both equal to $h(a)=h(b)$, then $h$ is just a constant function and so $h^\prime(x)=0$ everywhere.

Otherwise, $h$ has a maximum/minimum that is not $h(a)=h(b)$, so this extremal point lies in $(a,b)$.

In particular, this extremal point is also a local extremum.
Since $h$ is differentiable on $(a,b)$, by Fermat's theorem this extremum point is stationary, thus Rolle's Theorem is proven.
\end{proof}

\begin{theorem}[Generalised mean value theorem]\label{thrm:generalised-mvt}
If $f$ and $g$ are continuous on $[a,b]$ and differentiable in $(a,b)$, then there exists $c\in(a,b)$ such that
\[ \frac{f^\prime(c)}{g^\prime(c)}=\frac{f(b)-f(a)}{g(b)-g(a)}. \]
\end{theorem}

\begin{proof}
For $t\in[a,b]$, put
\[h(t)=[f(b)-f(a)]g(t)-[g(b)-g(a)]f(t).\]
Then $h$ is continuous on $[a,b]$, and $h$ is differentiable on $(a,b)$. Moreover,
\[h(a)=f(b)g(a)-f(a)g(b)=h(b)\]
thus by Rolle's Theorem, there exists $c\in(a,b)$ such that $h^\prime(c)=0$, i.e. $[g(b)-g(a)]f^\prime(c)=[f(b)-f(a)]g^\prime(c)$
\end{proof}

\begin{theorem}[Mean value theorem]\label{thrm:mvt}
If $f$ is continuous on $[a,b]$ and differentiable in $(a,b)$, then there exists $c\in(a,b)$ such that
\[ f^\prime(c)=\frac{f(b)-f(a)}{b-a}. \]
\end{theorem}

\begin{proof}
Take $g(x)=x$ in \cref{thrm:generalised-mvt}.
\end{proof}

\begin{proposition}
Suppose $f$ is differentiable in $(a,b)$.
\begin{enumerate}[label=(\roman*)]
\item If $f^\prime(x)\ge0$ for all $x\in(a,b)$, then $f$ is monotonically increasing.
\item If $f^\prime(x)=0$ for all $x\in(a,b)$, then $f$ is constant.
\item If $f^\prime(x)\le0$ for all $x\in(a,b)$, then $f$ is monotonically decreasing.
\end{enumerate}
\end{proposition}

\begin{proof}
All conclusions can be read off from the equation
\[f^\prime(x)=\frac{f(x_2)-f(x_1)}{x_2-x_1},\]
which is valid, for each pair of numbers $x_1,x_2$ in $(a,b)$, for some $x$ between
$x_1$ and $x_2$.
\end{proof}

\begin{exercise}
Let $f$ and $g$ be continuous on $[a,b]$ and differentiable on $(a,b)$. If $f^\prime(x)=g^\prime(x)$, then $f(x)=g(x)+C$.
\end{exercise}

\begin{exercise}
Given that $f(x)=x^\alpha$ where $0<\alpha<1$. Prove that $f$ is uniformly continuous on $[0,+\infty)$.
\end{exercise}

\begin{exercise}
Let $f$ be a function continuous on $[0,1]$ and differentiable on $(0,1)$ where $f(0)=f(1)=0$. Prove that there exists $c\in(0,1)$ such that
\[ f(x)+f^\prime(x)=0. \]
\end{exercise}

\section{Darboux's Theorem}
The following result implies some sort of a ``intermediate value'' property of derivatives that is similar to continuous functions.

\begin{theorem}[Darboux's Theorem]
Suppose $f$ is differentiable on $[a,b]$, and suppose $f^\prime(a)<c<f^\prime(b)$. Then there exists $x\in(a,b)$ such that $f^\prime(x)=c$.
\end{theorem}

\begin{proof}
Put $g(t)=f(t)-ct$. Then $g^\prime(a)<0$, so that $g(t_1)<g(a)$ for some $t_1\in(a,b)$, and $g^\prime(b)>0$, so that $g(t_2)<g(b)$ for some $t_2\in(a,b)$.

Hence $g$ attains its minimum on $[a,b]$ (Theorem 4.16) at some point $x$ such that $a<x<b$. By Theorem 5.8, $g^\prime(x)=0$. Hence $f^\prime(x)=c$.
\end{proof}

\begin{corollary}
If $f$ is differentiable on $[a,b]$, then $f^\prime$ cannot have any simple discontinuities on $[a,b]$.
\end{corollary}

\begin{remark}
But $f^\prime$ may very well have discontinuities of the second kind.
\end{remark}

\section{L'Hopital's Rule}
The following theorem is frequently used in the evaluation of limits.

\begin{theorem}[L'Hopital's Rule]
Suppose $f$ and $g$ are differentiable over $(a,b)$ with $g^\prime(x)\neq0$ for all $x\in(a,b)$, where $-\infty\le a<b\le+\infty$. If either
\begin{enumerate}[label=(\roman*)]
\item $\displaystyle\lim_{x\to a}f(x)=0$ and $\displaystyle\lim_{x\to a}g(x)=0$; or
\item $\displaystyle\lim_{x\to a}|g(x)|=+\infty$,
\end{enumerate}
and
\[\lim_{x\to a}\frac{f^\prime(x)}{g^\prime(x)}=A,\]
then
\[\lim_{x\to a}\frac{f(x)}{g(x)}=A.\]
\end{theorem}

\begin{proof}
The entire proof is rather tedious because we have to many cases.

We first consider the case in which $-\infty\le A<+\infty$. Choose $q\in\RR$ such that $A<q$, and choose $r\in\RR$ such that $A<r<q$.

1. $\frac{0}{0}$ or $\frac{\infty}{\infty}$
2. a is normal or $a=-\infty$
3. A is normal or $A=\pm\infty$



We'll only prove the most basic one here:
0/0, a and A are normal
This is the case which will be required for Taylor series

First we define f(a)=g(a)=0, so that $f$ and $g$ are continuous at $x=a$

Now let $x\in(a,b)$, then $f$ and $g$ are continuous on $[a,x]$ and differentiable in $(a,x)$
:
Thus by Cauchy's Mean Value Theorem, there exists $\xi\in(a,x)$ such that
\[ \frac{f^\prime(\xi)}{g^\prime(\xi)}=\frac{f(x)-f(a)}{g(x)-g(a)}=\frac{f(x)}{g(x)} \]

For each $x$, we pick $\xi$ which satisfies the above, so that $\xi$ may be seen as a function of $x$ satisfying $a<\xi(x)<x$

Then by squeezing we have $\lim_{x\to a^+}\xi(x)=a$.

Since $\frac{f^\prime}{g^\prime}$ is continuous near $a$, the theorem regarding the limit of composite functions give
\[ \lim_{x\to a^+}\frac{f(x)}{g(x)} = \lim_{x\to a^+}\frac{f'(\xi)}{g'(\xi)} = \lim_{x\to a^+}\brac{\frac{f^\prime}{g^\prime}}(\xi(x)) = A \]

Now the same reasoning can be used for $b$ where we will use lim(x→b-) to replace all the $\lim_{x\to a^+}$, and $\xi$ will be a function which maps to $(x,b)$.
\end{proof}

\section{Taylor Expansion}
\begin{theorem}[Taylor's Theorem]
Suppose $f:[a,b]\to\RR$, $f^{(n-1)}$ is continuous on $[a,b]$, $f^{(n)}(t)$ exists for every $t\in(a,b)$. Let $\alpha$ and $\beta$ be distinct points of $[a,b]$, and define
\[P(t)=\sum_{k=0}^{n-1}\frac{f^{(k)}(\alpha)}{k!}(t-\alpha)^k.\]
Then there exists $x\in[\alpha,\beta]$ such that
\[f(\beta)=P(\beta)+\frac{f^{(n)}(x)}{n!}(\beta-\alpha)^n.\]
\end{theorem}



\begin{comment}
Consider a function $f:[a,b]\to\RR$. We first look at the mean value theorem from the viewpoint of approximations for $f(x)$ near a point $x=a$. We can regard the constant function
\[ f_0(x)=f(a) \]
as the \vocab{zero order approximation} of $f(x)$. Then we ask if we can understand the remainder
\[ R_1(x)\coloneqq f(x)-f(a), \quad x\in[a,b] \]
for this approximation. For this, if we assume $f\in C_0[a,b]$ and $f^\prime$ exists over $(a,b)$, then the mean value theorem tells us that there exists some $a<\xi_x<x$ (here $\xi_x$ vocabasises that $\xi$ depends on $x$) so that we can write $R_1$ as
\[ R_1(x)=f^\prime(\xi_x)(x-a). \]
This is saying that the derivative of $f$ can control the remainder $R_1(x)$ as an order 1 monomial.



%%%%%%%%%%%%%%%%

The main expression is as follows:
\begin{equation}
f(x)=f(a)+\frac{f^\prime(a)}{1!}(x-a)+\frac{f^{\prime\prime}(a)}{2!}(x-a)^2+\frac{f^{\prime\prime\prime}(a)}{3!}(x-a)^3+\cdots
\end{equation}

So for example we have the following (we've used the ones for $e^x$ and $\ln x$ for generating functions):
\begin{align*}
e^x &= 1+x+\frac{x^2}{2!}+\frac{x^3}{3!}+\cdots \\
\sin x &= x-\frac{x^3}{3!}+\frac{x^5}{5!}-\frac{x^7}{7!}+\cdots \\
\cos x &= 1-\frac{x^2}{2!}+\frac{x^4}{4!}-\frac{x^6}{6!}+\cdots \\
\ln(1+x) &= x-\frac{x^2}{2}+\frac{x^3}{3}-\frac{x^4}{4}+\cdots
\end{align*}

There's a lot of things to say about these equations, for example the one for $\ln(1+x)$ only works for $|x|<1$

Also, if you want the RHS of the expression to be an infinite power series, $f(x)$ has to be smooth (infinitely differentiable)

Even then, the power series may never converge to $f(x)$ at any interval, no matter how small
The most common example given here is $f(x)=e^\frac{-1}{x^2}$ (f(0)=0); the Taylor series for $f(x)$ is just $0$

Now sometimes we don't actually that nice of a property for f, we're often given that fact that $f$ is only finitely differentiable

Then we will have something along the lines of
\[ f(x)\approx f(a)+\frac{f^\prime(a)}{1!}(x-a)+\frac{f^{\prime\prime}(a)}{2!}(x-a)^2+\cdots+\frac{f^{(n)}(a)}{n!}(x-a)^n \]
where $f^{(n)}$ denotes the $n$-th differential.

There are two main forms of the statement regarding the error between the original function and the Taylor series estimate

The simpler form is what's known as the Peano form: Given that f is n times differentiable at $a$, then
\[ f(x)=f(a)+\frac{f^\prime(a)}{1!}(x-a)+\frac{f^{\prime\prime}(a)}{2!}(x-a)^2+\cdots+\frac{f^{(n)}(a)}{n!}(x-a)^n+o((x-a)^n) \]

To show this, we only need to show that we have the following limit:
\[ \lim_{x\to a}\frac{f(x)-{f(a)+\frac{f^\prime(a)}{1!}(x-a)+\frac{f^{\prime\prime}(a)}{2!}(x-a)^2+\cdots+\frac{f^{(n)}(a)}{n!}(x-a)^n}}{(x-a)^n}=0 \]

The basic idea is to use the L'Hopital Rule n times. The numerator becomes $f^{(n)}(x)-f^{(n)}(a)$ which approaches $0$, whereas the denominater is just $n!$, so the limit exists and is equal to $0$.

However, we need to verify all the necessary conditions for L'Hopital
:
Here the main problem is that we don't know if we have the 0/0 indeterminate at each step, so we'll need to check this for the k-th step where k=1,...,n

Fortunately, the k-th derivative of the numerator is
$f^{(k)}(x)-f^{(k)}(a)-(x-a)F_k(x)$ where $F_k$ is just a bunch of random stuff, so the numerator approaches $0$ as $x\to a$
The $k$-th derivative of the denominator is $n(n-1)\cdots(n-k+1)(x-a)^{n-k}$ so it also approaches $0$, and we're done

The other form is actually a family of similar statements which gives more precise values for the error
The Peano form has a fundamental obstacle when used in approximation, we don't have any control on the size of the final term other than its asymptotic behaviour
:
We'll be talking about the one given in the book, known as the Lagrange form:
:
Given that f is n times differentiable on $(a,b)$ such that $f^{(n-1)}$ is continuous on $[a,b]$, then
\[ f(x)=f(a)+\frac{f^\prime(a)}{1!}(x-a)+\frac{f^{\prime\prime}(a)}{2!}(x-a)^2+\cdots+\frac{f^{(n-1)}(a)}{(n-1)!}(x-a)^(n-1)+\frac{f^{(n)}(\xi)}{n!}(x-a)^n \]

Just like in L'Hopital, we intuitively think of $(a,b)$ as just a very small interval at the right hand side of x=a
:
Here we are giving up on the second final term of Peano by combining it with the infinitesimal (small o) term to give an accurate description of the error

For the proof of this one we'll be using Cauchy's MVT

Fix any $x\in(a,b)$, then we construct the functions
\[ F(t)=f(x)-\brac{f(t)+\frac{f^\prime(t)}{1!}(x-t)+\frac{f^{\prime\prime}(t)}{2!}(x-t)^2+\cdots+\frac{f^{(n-1)}(t)}{(n-1)!}(x-t)^{n-1}} \]
\[ G(t)=(x-t)^n \]

We calculate $F^\prime(t)$ as follows:
\[ -[f^\prime(t)+\frac{f^{\prime\prime}(t)}{1!}-f^\prime(t)+\frac{f^{\prime\prime\prime}(t)}{2!}-\frac{f^{\prime\prime}(t)}{1!}+\cdots+\frac{f^{(n)}(t)}{(n-1)!}(x-t)^{n-1}-\frac{f^{(n-1)}(t)}{(n-2)!}(x-t)^{n-2}]=-\frac{f^{(n)}(t)}{(n-1)!}(x-t)^{n-1} \]

$G^\prime(t)=-n(x-t)^{n-1}$, so we have
\[ \frac{F^\prime(t)}{G^\prime(t)}=\frac{f^{(n)}(t)}{n!} \]

The main reason for why we come up with the strange-looking $F$ and $G$ is that we specifically swap out $a$ for $t$ so that $F(x)=G(x)=0$, in hopes of getting rid of $x$:

We apply Cauchy's MVT to $F$ and $G$ on $[a,x]$, so that we obtain $\xi\in(a,x)$ satisfying
\[ \frac{F^\prime(\xi)}{G^\prime(\xi)}=\frac{F(x)-F(a)}{G(x)-G(a)}=\frac{F(a)}{G(a)}. \]
Thus the Lagrange form of the remainder is given by 
\[ F(a)=\frac{f^{(n)}(\xi)}{n!}G(a). \]
\end{comment}

%    \chapter{Riemann--Stieltjes Integral}\label{chap:rs-integration}
\section{Definition of Riemann--Stieltjes Integral}
A \vocab{partition} $P$ of a closed interval $[a,b]\subset\RR$ is a finite set of points $x_0,x_1,\dots,x_n$ where
\[a=x_0\le x_1\le\cdots\le x_{n-1}\le x_n=b.\]
Let $f:[a,b]\to\RR$ be bounded, and $\alpha$ be an increasing function over $[a,b]$. Denote by
\begin{align*}
M_i&=\sup_{[x_{i-1},x_i]}f(x),\\
m_i&=\inf_{[x_{i-1},x_i]}f(x),
\end{align*}
and by
\[\Delta\alpha_i=\alpha(x_i)-\alpha(x_{i-1}).\]
The \vocab{upper sum} of $f$ with respect to the partition $P$ and $\alpha$ is
\[U(f,\alpha;P)=\sum_{i=1}^n M_i \Delta \alpha_i\]
and the \vocab{lower sum} of $f$ with respect to the partition $P$ and $\alpha$ is
\[ L(f,\alpha;P)=\sum_{i=1}^n m_i \Delta \alpha_i. \]
Define the upper Riemann--Stieltjes integral as
\[\upperint_a^bf(x)\dd{\alpha(x)}\coloneqq\inf_P U(f,\alpha;P)\]
and the lower Riemann--Stieltjes integral as
\[\lowerint_a^bf(x)\dd{\alpha(x)}\coloneqq\sup_P L(f,\alpha;P).\]
It is easy to see from definition that
\[ \lowerint_a^bf(x)\dd{\alpha(x)}\le\upperint_a^bf(x)\dd{\alpha(x)}. \]

\begin{definition}[Riemann--Stieltjes integrability]
A function $f$ is \vocab{Riemann--Stieltjes integrable}\index{Riemann--Stieltjes integrability} with respect to $\alpha$ over $[a,b]$, if
\[\lowerint_a^bf(x)\dd{\alpha(x)}=\upperint_a^bf(x)\dd{\alpha(x)}.\]
\end{definition}

\begin{notation}
$\displaystyle\int_a^bf(x)\dd{\alpha(x)}$ denotes the common value, which is called the Riemann--Stieltjes of $f$ with respect to $\alpha$ over $[a,b]$.
\end{notation}

\begin{notation}
$\mathcal{R}_\alpha[a,b]$ denotes the set of Riemann--Stieljes integrable functions with respect to $\alpha$ over $[a,b]$.
\end{notation}

In particular, when $\alpha(x)=x$, we call the corresponding Riemann--Stieljes integration the \emph{Riemann integration}, and use $\mathcal{R}[a,b]$ to denote the set of Riemann integrable functions.

\begin{definition}[Refinement]
The partition $P^\prime$ is a \vocab{refinement} of $P$ if $P^\prime\supset P$. Given two partitions $P_1$ and $P_2$, we say that $P^\prime$ is their \vocab{common refinement} if $P^\prime=P_1\cup P_2$.
\end{definition}

Intuitively, a refinement will give a better estimation than the original partition, so the upper and lower sums of a refinement should be more restrictive. We will now show this.

\begin{lemma}\label{lemma:int-refinement}
If $P^\prime$ is a refinement of $P$, then
\begin{enumerate}[label=(\roman*)]
\item $L(f,\alpha;P)\le L(f,\alpha;P^\prime)$
\item $U(f,\alpha;P^\prime)\le U(f,\alpha;P)$
\end{enumerate}
\end{lemma}

\begin{proof} \
\begin{enumerate}[label=(\roman*)]
\item Suppose first that $P^\prime$ contains just one point more than $P$ Let this extra point be $x^\prime$, and suppose $x_{i-1}<x^\prime<x_i$ for some $i$, where $x_{i-1},x_i\in P$. Put
\[w_1=\inf_{x\in[x_{i-1},x^\prime]}f(x)\]
and
\[w_2=\inf_{x\in[x^\prime,x_i]}f(x).\]
Let, as before,
\[m_i=\inf_{x\in[x_{i-1},x_i]}f(x).\]
Clearly $w_1\ge m_i$ and $w_2\ge m_i$. Hence
\begin{align*}
&L(f,\alpha;P^\prime)-L(f,\alpha;P)\\
&=w_1[\alpha(x^\prime)-\alpha(x_{i-1})]+w_2[\alpha(x_i)-\alpha(x^\prime)]-m_i[\alpha(x_i)-\alpha(x_{i-1})]\\
&=(w_1-m_i)[\alpha(x^\prime)-\alpha(x_{i-1})]+(w_2-m_i)[\alpha(x_i)-\alpha(x^\prime)]\ge0.
\end{align*}
If $P^\prime$ contains $k$ more points than $P$, we repeat this reasoning $k$ times.

\item Analogous to the proof of (i).
\end{enumerate}
\end{proof}

One would expect the lower RS integral to be less than or equal to the upper RS integral. We now show this.

\begin{lemma}\label{lemma:int-upper-lower}
\[\lowerint_a^bf\dd{\alpha}\le\upperint_a^bf\dd{\alpha}.\]
\end{lemma}

\begin{proof}
Let $P^\prime$ be the common refinement of partitions $P_1$ and $P_2$. By \cref{lemma:int-refinement},
\[L(f,\alpha;P_1)\le L(f,\alpha;P^\prime)\le U(f,\alpha;P^\prime)\le U(f,\alpha;P_2)\]
and so
\[L(f,\alpha;P_1)\le U(f,\alpha;P_2).\]
Fix $P_2$ and take sup over all $P_1$ gives
\[\lowerint f\dd{\alpha}\le U(f,\alpha;P_2).\]
Then take inf over all $P_2$, which gives the desired result.
\end{proof}

Now we discuss integrability conditions for $f$.

\begin{theorem}
$f\in \mathcal{R}_\alpha[a,b]$ if and only if
\[\forall\epsilon>0,\quad\exists P,\quad U(f,\alpha;P)-L(f,\alpha;P)<\epsilon.\]
\end{theorem}

\begin{proof} \

\fbox{$\implies$} Suppose $f\in \mathcal{R}_\alpha[a,b]$. Let $\epsilon>0$ be given. Then there exists partitions $P_1$ and $P_2$ such that


\fbox{$\impliedby$} For every $P$, from \cref{lemma:int-upper-lower} we have
\[L(f,\alpha;P)\le\lowerint f\dd{\alpha}\le\upperint f\dd{\alpha}\le U(f,\alpha;P).\]

\end{proof}

\begin{example}[Dirichlet function]
The Dirichlet function is defined over $\RR$ by
\[f(x)=\begin{cases}
1&x\in\QQ \\
0&x\in\RR\setminus\QQ
\end{cases}\]
We try to calculate the two on the interval $[0,1]$.

The Dirichlet function is pathological because for each subinterval $[x_{i-1},x_i]$, the supremum is always $1$ and the infimum is always $0$.

So no matter what partition we use, $U(f,P)$ is always $1$ whereas $L(f,P)$ is always $0$. This means that $U(f)=1$ and $L(f)=0$, so there are two different values for ``the integral of $f$''.

This is like the case where we try to find the limit of the Dirichlet function where $x$ is approaching any given real number $r$, there exists two sequences approaching $r$ whose image approaches two different values.
\end{example}

\begin{example}[Heaviside step function]
The Heaviside step function $H$ is a real-valued function defined by
\[H(x)=\begin{cases}
0&x\le0\\
1&x>0
\end{cases}\]

\begin{proposition*}
$f$ bounded on $[a,b]$, $f$ continuous at $s\in(a,b)$. Let $\alpha(x)=H(x-s)$, then
\[\int_a^b f\dd{\alpha}=f(s).\]
\end{proposition*}

\begin{proposition*}
Suppose $c_n\ge0$ for $n=1,2,\dots$, $\sum c_n$ converges, $(s_n)$ is a sequence of distinct points in $(a,b)$, and
\[\alpha(x)=\sum_{n=1}^{\infty}c_n I(x-s_n).\]
Let $f$ be continuous on $[a,b]$. Then
\[\int_a^b f\dd{\alpha}=\sum_{n=1}^{\infty}c_n f(s_n).\]
\end{proposition*}
\end{example}

\begin{proposition} \
\begin{enumerate}[label=(\roman*)]
\item For all $\epsilon>0$, if there exists $P$ such that $U(f,\alpha;P)-L(f,\alpha;P)<\epsilon$, then $U(f,\alpha;P)-L(f,\alpha;P^\prime)<\epsilon$ where $P^\prime$ is a refinement of $P$.
\item 
\item 
\end{enumerate}
\end{proposition}

\begin{proof}

\end{proof}

\begin{lemma}[Continuity implies integrability]
If $f$ is continuous on $[a,b]$, then $f\in \mathcal{R}_\alpha[a,b]$.
\end{lemma}

\begin{proof}
Let $\epsilon>0$ be given. Choose $\eta>0$ such that
\[[\alpha(b)-\alpha(a)]\eta<\epsilon.\]
Since $f$ is uniformly continuous on $[a,b]$ (Theorem 4.19), there exists $\delta>0$ such that
\[\absolute{f(x)-f(t)}<\eta\]
if $x\in[a,b]$, $t\in[a,b]$, $|x-t|<\delta$.

If $P$ is any partition of $[a,b]$ such that $\Delta x_i<\delta$ for all $i$, then (16) implies that
\[M_i-m_i\le\eta\]
and therefore
\begin{align*}
U(f,\alpha;P)-L(f,\alpha;P)
&=\sum_{i-1}^{n}(M_i-m_i)\Delta\alpha_i\\
&\le\eta\sum_{i=1}^{n}\Delta\alpha_i\\
&=\eta[\alpha(b)-\alpha(a)]\\
&<\epsilon.
\end{align*}
By Theorem 6.6, $f\in\mathcal{R}_\alpha[a,b]$.
\end{proof}

\begin{proposition}
If $f$ is monotonic on $[a,b]$, and if $\alpha$ is continuous on $[a,b]$, then $f\in\mathcal{R}_\alpha[a,b]$.
\end{proposition}

\begin{proposition}
Suppose $f$ is bounded on $[a,b]$, $f$ has only finitely many points of discontinuity on $[a,b]$, and $\alpha$ is continuous at every point at which $f$ is discontinuous. Then $f\in \mathcal{R}_\alpha[a,b]$.
\end{proposition}

\begin{proposition}
$f\in \mathcal{R}_\alpha[a,b]$, $m\le f\le M$, and $\phi$ is uniformly continuous on $[m,M]$. Then
\[\phi\circ f\in \mathcal{R}_\alpha[a,b].\]
\end{proposition}

\begin{proof}
Choose $\epsilon>0$. Since $\phi$ is uniformly continuous on $[m,M]$, there exists $\delta>0$ such that $\delta<\epsilon$ and $|\phi(s)-\phi(t)|<\epsilon$ if $|s-t|<\le\delta$ and $s,t\in[m,M]$.

Since $f\in \mathcal{R}_\alpha[a,b]$, there exists a partition $P=\{x_0,x_1,\dots,x_n\}$ of $[a,b]$ such that
\[U(f,\alpha;P)-L(f,\alpha;P)<\delta^2.\]

\end{proof}

\section{Properties of the Integral}
\begin{theorem} \
\begin{enumerate}[label=(\roman*)]
\item If $f_1,f_2\in \mathcal{R}_\alpha[a,b]$, then 
\[ f_1+f_2\in \mathcal{R}_\alpha[a,b]; \]
$cf\in \mathcal{R}_\alpha[a,b]$ for every $c\in\RR$, and
\[ \int_a^b(f_1+f_2)\dd{\alpha}=\int_a^bf_1\dd{\alpha}+\int_a^bf_2\dd{\alpha}, \]
\[ \int_a^b(cf)\dd{\alpha}=c\int_a^bf\dd{\alpha}. \]

\item If $f_1,f_2\in \mathcal{R}_\alpha[a,b]$ and $f_1\le f_2$, then
\[ \int_a^bf_1\dd{\alpha}\le\int_a^bf_2\dd{\alpha}. \]

\item If $f\in \mathcal{R}_\alpha[a,b]$ and $c\in[a,b]$, then $f\in \mathcal{R}_\alpha[a,c]$ and $f\in \mathcal{R}_\alpha[c,b]$, and
\[ \int_a^bf\dd{\alpha}=\int_a^c\dd{\alpha}+\int_c^b\dd{\alpha}. \]

\item If $f\in \mathcal{R}_\alpha[a,b]$ and $|f|\le M$, then
\[ \absolute{\int_a^bf\dd{\alpha}}\le M\sqbrac{\alpha(b)-\alpha(a)}. \]

\item If $f\in R_{\alpha_1}[a,b]$ and $f\in R_{\alpha_2}[a,b]$, then $f\in R_{\alpha_1+\alpha_2}[a,b]$ and
\[ \int_a^bf\dd{(\alpha_1+\alpha_2)}=\int_a^bf\dd{\alpha_1}+\int_a^bf\dd{\alpha_2}; \]
if $f\in \mathcal{R}_\alpha[a,b]$ and $c$ is a positive constant, then $f\in R_{c\alpha}[a,b]$ and
\[ \int_a^bf\dd{(c\alpha)}=c\int_a^bf\dd{\alpha}. \]

\item If $f\in \mathcal{R}_\alpha[a,b]$ and $g\in \mathcal{R}_\alpha[a,b]$, then $fg\in \mathcal{R}_\alpha[a,b]$.
\end{enumerate}
\end{theorem}

\begin{proof} \
\begin{enumerate}[label=(\roman*)]
\item If $f=f_1+f_2$ and $P$ is any partition of $[a,b]$, we have
\begin{align*}
L(f_1,\alpha;P)+L(f_2,\alpha;P)&\le L(f,\alpha;P)\\
&\le U(f,\alpha;P)\\
&\le U(f_1,\alpha;P)+U(f_2,\alpha;P).
\end{align*}

If $f_1\in \mathcal{R}_\alpha[a,b]$ and $f_2\in \mathcal{R}_\alpha[a,b]$, let $\epsilon>0$ be given. There are partitions $P_1$ and $P_2$ such that


\item 
\item 
\item 
\item 
\item 
\end{enumerate}
\end{proof}

\begin{theorem}[Triangle inequality]
$f\in \mathcal{R}_\alpha[a,b]$, then $|f|\in \mathcal{R}_\alpha[a,b]$,
\[ \absolute{\int_a^bf\dd{\alpha}}\le\int_a^b|f|\dd{\alpha}. \]
\end{theorem}

\begin{proof}

\end{proof}

6.14 6.15
Heaviside step function

6.16 corollary
for intinite sum, need $\sum c_n$ to converge
(23) comparison test

\begin{proposition}[Integration by substitution]
Assume $\alpha$ increases monotonically, $\alpha^\prime\in R[a,b]$. Let $f$ be a bounded real function on $[a,b]$, then
\[f\in \mathcal{R}_\alpha[a,b]\iff f\alpha^\prime\in R[a,b].\]
\end{proposition}

\begin{proposition}[Change of variables]
Suppose $\phi:[A,B]\to[a,b]$ is a strictly increasing continuous function. Suppose $\alpha$ is monotonically increasing on $[a,b]$, $f\in \mathcal{R}_\alpha[a,b]$. Define $\beta$ and $g$ on $[A,B]$ by
\[\beta(y)=\alpha(\phi(y)),\quad g(y)=f(\phi(y)).\]
Then $g\in R(\beta)$ and
\[\int_A^B g\dd{\beta}=\int_a^b f\dd{\alpha}.\]
\end{proposition}

\section{Integration and Differentiation}
We shall show that integration and differentiation are, in a certain sense, inverse operations.

\begin{lemma}
$f\in \mathcal{R}_\alpha[a,b]$. For $x\in [a,b]$, put
\[F(x)=\int_a^x f(t)\dd{t}.\]
Then $F$ is continuous on $[a,b]$; furthermore, if $f$ is continuous at $x_0\in[a,b]$, then $F$ is differentiable at $x_0$, and
\[F^\prime(x_0)=f(x_0).\]
\end{lemma}

\begin{theorem}[Fundamental Theorem of Calculus]
$f\in \mathcal{R}_\alpha[a,b]$, there is a differentiable function $F$ on $[a,b]$ such that $F^\prime=f$, then
\begin{equation}
\int_a^b f(x)\dd{x}=F(b)-F(a).
\end{equation}
\end{theorem}

\begin{theorem}[Integration by parts]
Suppose $F$ and $G$ are differentiable functions on $[a,b]$, $F^\prime=f\in R$, $G^\prime=g\in R$. Then
\begin{equation}
\int_a^b F(x)g(x)\dd{x}=F(b)G(b)-F(a)G(a)-\int_a^b f(x)G(x)\dd{x}.
\end{equation}
\end{theorem}
%    \chapter{Sequence and Series of Functions}\label{chap:func-seq-series}
\section{Uniform Convergence}
\begin{definition}[Pointwise convergence]
Suppose $(f_n)$ is a sequence of functions defined on a set $E$, and suppose that $\brac{f_n(x)}$ converges for every $x\in E$. We can then define a function $f$ by
\[f(x)=\lim_{n\to\infty}f_n(x)\quad(\forall x\in E)\]
We say that $(f_n)$ \vocab{converges pointwise}\index{pointwise convergence} to $f$ on $E$, denoted by $f_n\to f$, if
\[\forall\epsilon>0,\quad\forall x\in E,\quad\exists N\in\NN,\quad\forall n>N,\quad\absolute{f_n(x)-f(x)}<\epsilon.\]
$f$ is called the \textbf{limit} or limit function of $(f_n)$.

Similarly, if $\sum f_n(x)$ converges for every $x\in E$, and if we define
\[f(x)=\sum_{n=1}^\infty f_n(x)\quad(\forall x\in E)\]
the function $f$ is called the \textbf{sum of the series} $\sum f_n$.
\end{definition}

Most properties are not preserved by pointwise continuity; that is, $f$ does not inherit most properties of $f_n$.

\begin{example}[$f_n$ continuous, $f$ discontinuous]
Let $f_n(x)=x^n$ for $x\in[0,1]$. Then
\[f(x)=\lim_{n\to\infty}f_n(x)=\begin{cases}
0&\text{if }x\in(0,1]\\
1&\text{if }x=1
\end{cases}\]
and so the limit function $f(x)$ is discontinuous.
\end{example}

\begin{example}[$f_n$ integrable, $f$ not integrable]
Recall that the Dirichlet function
\[D(x)=\begin{cases}
1&\text{if }x\in\QQ\\
0&\text{if }x\in\RR\setminus\QQ
\end{cases}\]
is not integrable.

\begin{proof}
Consider the interval $[0,1]$. We partition $P:0=x_0<x_1<\cdots<x_n=1$. The sum is given by $\sum_{i=1}^n D(t_i)\Delta x_i$. Then
\[M_i=\max_{t\in[x_{i-1},x_i]}D(t)=1\implies U(D;P)=1\quad\forall P\]
and
\[m_i=\min_{t\in[x_{i-1},x_i]}D(t)=0\implies L(D;P)=0\quad\forall P.\]
Hence 
\[\upperint_0^1 D(x)\dd{x}=1,\quad\lowerint_0^1 D(x)\dd{x}=0\]
so $\upperint_0^1 D(x)\dd{x}\neq\lowerint_0^1 D(x)\dd{x}$, and thus $D(x)$ is not integrable.
\end{proof}

We define a sequence of functions as follows:
\[D_n(x)=\begin{cases}
1&\text{if }x=\frac{p}{q},p\in\ZZ,q\in\ZZ\setminus\{0\},|q|\le n\\
0&\text{if otherwise}
\end{cases}\]

\end{example}

\begin{definition}[Uniform convergence]
We say that $(f_n)$ \vocab{uniformly converges}\index{uniform convergence} to $f$ over $E$, denoted by $f_n\rightrightarrows f$, if 
\[\forall\epsilon>0,\quad\exists N\in\NN,\quad\forall x\in E,\quad\forall n>N,\quad\absolute{f_n(x)-f(x)}<\epsilon.\]
For series, we say that the series $\sum f_n(x)$ converges uniformly on $E$ if the sequence of partial sums $(S_n)$ defined by
\[S_n(x)=\sum_{i=1}^{n}f_i(x)\]
converges uniformly on $E$.
\end{definition}

Uniform convergence is stronger than pointwise convergence, since $N$ is uniform (or ``fixed'') for all $x\in E$; for pointwise convergence, the choice of $N$ is determined by $x$.

The Cauchy criterion for uniform convergence is as follows.

\begin{lemma}[Cauchy criterion]
$(f_n)\rightrightarrows f$ if and only if
\[\forall\epsilon>0,\quad\exists N\in\NN,\quad\forall x\in E,\quad\forall n,m\ge N,\quad\absolute{f_n(x)-f_m(x)}\le\epsilon.\]
\end{lemma}

\begin{proof} \

\fbox{$\implies$} Suppose $f_n\rightrightarrows f$ on $E$. Let $\epsilon>0$ be given. Then there exists $N\in\NN$ such that for all $x\in E$, for all $n>N$,
\[\absolute{f_n(x)-f(x)}<\frac{\epsilon}{2}.\]
Then for all $n,m>N$,
\begin{align*}
|f_n(x)-f_m(x)|
&=\absolute{\brac{f_n(x)-f(x)}-\brac{f_m(x)-f(x)}}\\
&\le|f_n(x)-f(x)|+|f_m(x)-f(x)|\\
&<\frac{\epsilon}{2}+\frac{\epsilon}{2}=\epsilon
\end{align*}
by triangle inequality.

\fbox{$\impliedby$} Conversely, suppose the Cauchy condition holds. By Theorem 3.11, the sequence $\brac{f_n(x)}$ converges, for every $x$, to a limit which we may call $f(x)$. Thus $(f_n)\to f$ on $E$. We have to prove that the convergence is uniform.

Let $\epsilon>0$ be given. Choose $N\in\NN$ such that (13) holds. Fix $n$, and let $m\to\infty$ in (13). Since $f_m(x)\to f(x)$ as $m\to\infty$, thus for all $n\ge N$ and for all $x\in E$,
\[\absolute{f_n(x)-f(x)}\le\epsilon,\]
which completes the proof.
\end{proof}

The following criterion is sometimes useful.

\begin{proposition}
Suppose $f_n\to f$ on $E$. Let
\[M_n=\sup_{x\in E}\absolute{f_n(x)-f(x)}.\]
Then $f_n\rightrightarrows f$ on $E$ if and only if $M_n\to0$ as $n\to\infty$.
\end{proposition}

For series, there is a very convenient test for uniform convergence, due to Weierstrass.

\begin{lemma}[Weierstrass M-test]
Suppose $(f_n)$ is a sequence of functions defined on $E$, and suppose there exists $(M_n)\in\RR^+$ such that $|f_n(x)|\le M_n$ for all $n\ge1$ and for all $x\in E$.

Then $\sum f_n$ converges uniformly on $E$ if $\sum M_n$ converges.
\end{lemma}

\section{Uniform Convergence and Continuity}
We now consider properties preserved by uniform convergence.

\begin{proposition}
Suppose $f_n\rightrightarrows f$ on $E$. Let $x\in E$ be a limit point, let
\[\lim_{t\to x}f_n(t)=A_n.\]
Then $(A_n)$ converges, and $\displaystyle\lim_{t\to x}f(t)=\lim_{n\to\infty}A_n$.
\end{proposition}

\begin{proposition}
Let $(f_n)$ be a sequence of continuous functions on $E$, $f_n\rightrightarrows f$. Then $f$ is continuous in $E$.
\end{proposition}

\begin{definition}[Supremum norm]
If $X$ is a metric space, we denote the set of all complex-valued, continuous, bounded functions with domain $X$ by $C(X)$.

If $f\in C(X)$, we define 
\[\norm{f}\coloneqq\sup_{x\in X}|f(x)|,\]
known as the \vocab{suprenum norm} of $f$.
\end{definition}

\begin{lemma}
$\norm{f}$ gives a norm on $C(X)$.
\end{lemma}

\begin{proof}
Check that $\norm{f}$ satisfies the conditions for a norm:
\begin{enumerate}[label=(\roman*)]
\item 
\end{enumerate}
\end{proof}

\begin{proposition}
$\brac{C(X),\norm{\cdot}}$ is a metric space.
\end{proposition}

\section{Uniform Convergence and Integration}
\begin{theorem}
Assume $(f_n)$ is a sequence of functions defined over $[a,b]$ and each $f_n\in R_\alpha[a,b]$. If $f_n\to f$, then $f\in R_\alpha[a,b]$, and
\[ \lim_{n\to\infty}\int_a^bf_n\dd{\alpha}=\int_a^bf\dd{\alpha}. \]
\end{theorem}

\begin{proof}
Define
\end{proof}

\begin{corollary}
Assume $a_n\in R_\alpha[a,b]$ and
\[ f(x)\coloneqq\sum_{n=0}^\infty a_n(x) \]
converges uniformly. Then it follows
\[ \int_a^bf\dd{\alpha}=\sum_{n=0}^\infty a_n\dd{\alpha}. \]
\end{corollary}

\begin{proof}
Consider the sequence of partial sums 
\[ f_n(x)\coloneqq\sum_{k=0}^na_k(x), \quad n=0,1,\dots \]
It follows $f_n\in R_\alpha[a,b]$ and $f_n\rightrightarrows f$. Apply above theorem to $(f_n)$ and the conclusion follows.
\end{proof}

\section{Uniform Convergence and Differentiation}
\begin{theorem}
$(f_n)$ differentiable on $[a,b]$, $\exists x_0\in[a,b]\suchthat f_n(x_0)\to y_0=f(x_0)$ and $f_n^\prime\rightrightarrows f^\prime$. Then $f_n\rightrightarrows f$ on $[a,b]$, and $f$ is differentiable, $f^\prime(x)=\lim_{n\to\infty}f_n^\prime(x)$ for any $x\in[a,b]$.
\end{theorem}

\begin{proof}
$f_n(x_0)\to y_0$ thus
\end{proof}

\section{Stone--Weierstrass Approximation Theorem}
\begin{theorem}[Weierstrass approximation theorem]
If $f$ is a continuous complex function on $[a,b]$, there exists a sequence of polynomials $P_n$ such that $P_n\rightrightarrows f$ on $[a,b]$.

If $f$ is real, then $P_n$ may be taken real.
\end{theorem}
%    \chapter{Some Special Functions}\label{chap:special-functions}
\section{Power Series}
\begin{definition}[Analytic function]
An \vocab{analytic function}\index{analytic function} is a function that can be represented by a power series, i.e., functions of the form
\[f(x)=\sum_{n=0}^\infty c_n x^n\]
or, more generally,
\[f(x)=\sum_{n=0}^\infty c_n(x-a)^n.\]
\end{definition}

As a matter of convenience, we shall often take $a=0$ without any loss of generality.

We shall restrict ourselves to real values of $x$. The \vocab{radius of convergence} is the maximum $R$ such that $f(x)$ converges in $(-R,R)$. If $f(x)$ converges for all $x\in(-R,R)$, for some $R>0$, we say that $f$ is expanded in a power series about the point $x=0$.

\begin{proposition}
Suppose the series $\displaystyle\sum_{n=0}^\infty c_nx^n$ converges for $x\in(-R,R)$. Then
\begin{enumerate}[label=(\roman*)]
\item $\displaystyle\sum_{n=0}^\infty c_nx^n$ converges uniformly on $[-R+\epsilon,R-\epsilon]$ for all $\epsilon>0$;
\item $f(x)$ is continuous and differentiable on $(-R,R)$, and 
\[f^\prime(x)=\sum_{n=1}^\infty nc_nx^{n-1}.\]
\end{enumerate}
\end{proposition}

\begin{proof} \
\begin{enumerate}[label=(\roman*)]
\item Let $\epsilon>0$ be given. For $|x|\le R-\epsilon$, we have
\[|c_nx^n|\le|c_n(R-\epsilon)^n|\]
and since
\[\sum c_n(R-\epsilon)^n\]
converges absolutely (every power series converges absolutely in the interior of its internal of convergence, by the root test), Theorem 7.10 show that $\displaystyle\sum_{n=0}^\infty c_nx^n$ uniformly converges on $[-R+\epsilon,R-\epsilon]$.

\item 
\end{enumerate}
\end{proof}

\begin{corollary}
$f$ has derivatives of all orders in $(-R,R)$, which are given by
\[f^{(k)}(x)=\sum_{n=k}^\infty n(n-1)\cdots(n-k+1)c_nx^{n-k}.\]
In particular,
\[f^{(k)}(0)=k!c_k,\quad k=0,1,2,\dots\]
(Here $f^{(0)}$ means $f$, and $f^{(k)}$ is the $k$-th derivative of $f$, for $k=1,2,3,\dots$)
\end{corollary}

\begin{proof}
Apply theorem successively to $f,f^\prime,f^{\prime\prime},\dots$. Put $x=0$.
\end{proof}

\begin{proposition}
Suppose $\sum c_n$ converges. Put
\[f(x)=\sum_{n=0}^\infty c_n x^n\]
for $x\in(-R,R)$
\end{proposition}

\section{Exponential and Logarithmic Functions}
\begin{definition}[Exponential function]
\begin{equation}
\exp(z)\coloneqq\sum_{n=0}^\infty\frac{z^n}{n!}.
\end{equation}
\end{definition}

\begin{proposition}
$\exp(z)$ converges for every $z\in\CC$.
\end{proposition}

\begin{proof}
Ratio test.
\end{proof}

\begin{proposition}
For $z,w\in\CC$,
\[\exp(z+w)=\exp(z)+\exp(w).\]
\end{proposition}

\begin{corollary}
For $z\in\CC$,
\[\exp(z)\exp(-z)=1.\]
\end{corollary}

\begin{proof}
Take $z=z$, $w=-z$ in the previous result.
\end{proof}

\begin{proposition}
$\exp$ is strictly increasing in $\RR$.
\end{proposition}

\begin{proposition}
For $z\in\CC$,
\[\exp^\prime(z)=\exp(z)\]
\end{proposition}

Further,
\[\exp^\prime(z)=\lim_{h\to0}\frac{\exp(z+h)-\exp(z)}{h}=\lim_{h\to0}\frac{\exp(z+h)-1}{h}\exp(z).\]
Let $\exp(1)=e$. So $\exp(n)=\exp(1+\cdots+1)=\exp(1)\cdots\exp(1)=e^n$. This holds for any $n\in\QQ$.

\section{Trigonometric Functions}
Define
\begin{align*}
C(x)&=\frac{\exp(ix)+\exp(-ix)}{2}\\
S(x)&=\frac{\exp(ix)-\exp(-ix)}{2i}
\end{align*}

Our goal here is to show that $C(x)$ and $S(x)$ coincide with the functions $\cos x$ and $\sin x$, whose definition is usually based on geometric considerations.

\begin{proposition}[Euler's identity]
\[\exp(ix)=C(x)+iS(x).\]
\end{proposition}

\begin{proof}

\end{proof}

From definition, it is easy to see that $C(0)=1$, $S(0)=0$, and
\begin{align*}
C^\prime(x)&=S(x)\\
S^\prime(x)&=C(x)
\end{align*}

\begin{proposition} \
\begin{enumerate}[label=(\roman*)]
\item $\exp$ is periodic, with period $2\pi i$.
\item $C$ and $S$ are periodic, with period $2\pi$.
\item If $0<t<2\pi$, then $\exp(it)\neq1$.
\item If $z\in\CC$, $|z|=1$, there exists a unique $t\in[0,2\pi)$ such that $\exp(it)=z$.
\end{enumerate}
\end{proposition}



\section{Algebraic Completeness of the Complex Field}
We now prove that the complex field is \emph{algebraically complete}; that is, every non-constant polynomial with complex coefficients has a complex root.

\begin{theorem}[Fundamental Theorem of Algebra]
Suppose $a_0,\dots,a_n$ are complex numbers, $n\ge1$, $a_n\neq0$,
\[P(z)=\sum_{k=0}^n a_kz^k.\]
Then $P(z)=0$ for some complex number $z$.
\end{theorem}

\begin{proof}
Without loss of generality, assume $a_n=1$. Let $\mu=\inf|P(z)|$.
\end{proof}

\section{Fourier Series}
\begin{definition}[Trigonometric polynomial]
A \vocab{trigonometric polynomial}\index{trigonometric polynomial} is a finite sum of the form
\[f(x)=a_0+\sum_{n=1}^\infty(a_n\cos nx+b_n\sin nx)\quad(x\in\RR)\]
where $a_0,\dots,a_N,b_1,\dots,b_N\in\CC$.
\end{definition}

We can write the above in the form
\[f(x)=\sum_{n=-N}^N c_ne^{inx}.\]
It is clear that every trigonometric polynomial is periodic, with period $2\pi$.

For non-zero integer $n$, $e^{inx}$ is the derivative of $\frac{1}{in}e^{inx}$, which also has period $2\pi$. Hence
\[\frac{1}{2\pi}\int_{-\pi}^{\pi}e^{inx}\dd{x}=\begin{cases}
1&(n=0)\\
0&(n=\pm1,\pm2,\dots)
\end{cases}\]

\begin{definition}[Fourier coefficients]
If $f$ is an integrable function on $[-\pi,\pi]$, the numbers $c_m$ defined by
\[c_m=\frac{1}{2\pi}\int_{-\pi}^{\pi}f(x)e^{inx}\dd{x}\]
for all integers $m$ are called the \vocab{Fourier coefficients}\index{Fourier coefficients} of $f$.
\end{definition}

\begin{definition}[Fourier series]
The series
\[\sum_{n=-\infty}^{\infty}c_ne^{inx}\]
formed with the Fourier coefficients is called the \vocab{Fourier series}\index{Fourier series} of $f$.
\end{definition}

\begin{definition}[Orthogonal system of functions]
Let $(\phi_n)$ be a sequence of complex functionns on $[a,b]$ such that
\[\int_{a}^{b}\phi_n(x)\overline{\phi_m(x)}\dd{x}=0\quad(n\neq m)\]
Then $(\phi_n)$ is said to be an \vocab{orthogonal system of functions}\index{orthogonal system of functions} on $[a,b]$. If in addition
\[\int_{a}^{b}\absolute{\phi_b(x)}^2\dd{x}=1\]
for all $n$, $(\phi_n)$ is said to be \vocab{orthonormal}.
\end{definition}

\section{Gamma Function}
\begin{definition}[Gamma function]
For $0<x<\infty$, the \vocab{Gamma function}\index{Gamma function} is defined as
\begin{equation}
\Gamma(x)\coloneqq\int_0^\infty t^{x-1}e^{-t}\dd{t}.
\end{equation}
The integral converges for these $x$. (When $x<1$, both $0$ and $\infty$ have to be looked at.)
\end{definition}

\begin{lemma} \
\begin{enumerate}[label=(\roman*)]
\item The functional equation
\[\Gamma(x+1)=x\Gamma(x)\]
holds for $0<x<\infty$.
\item $\Gamma(n+1)=n!$ for $n=1,2,3,\dots$
\item $\log\Gamma$ is convex on $(0,\infty)$.
\end{enumerate}
\end{lemma}

\begin{proof} \
\begin{enumerate}[label=(\roman*)]
\item Integrate by parts.
\item Since $\Gamma(1)=1$, (1) implies (2) by induction.
\item 
\end{enumerate}
\end{proof}

In fact, these three properties characterise $\Gamma$ completely.

\begin{lemma}[Characteristic properties of $\Gamma$] \label{lemma:gamma-char}
If $f$ is a positive function on $(0,\infty)$ such that
\begin{enumerate}[label=(\roman*)]
\item $f(x+1)=xf(x)$,
\item $f(1)=1$,
\item $\log f$ is convex,
\end{enumerate}
then $f(x)=\Gamma(x)$.
\end{lemma}

\begin{proof}

\end{proof}

\begin{definition}[Beta function]
For $x>0$ and $y>0$, the \vocab{beta function}\index{beta functions} is defined as
\[B(x,y)\coloneqq\int_0^1 t^{x-1}(1-t)^{y-1}\dd{t}.\]
\end{definition}

\begin{lemma}
\[B(x,y)=\frac{\Gamma(x)\Gamma(y)}{\Gamma(x+y)}.\]
\end{lemma}

\begin{proof}
Let $f(x)=\dfrac{\Gamma(x+y)}{\Gamma(y)}B(x,y)$. We want to prove that $f(x)=\Gamma(x)$, using \cref{lemma:gamma-char}.
\begin{enumerate}[label=(\roman*)]
\item \[B(x+1,y)=\int_0^1 t^x(1-t)^{y-1}\dd{t}.\]
Integrating by parts gives
\begin{align*}
B(x+1,y)&=\underbrace{\sqbrac{t^x\cdot\frac{(1-t)^y}{y}(-1)}_0^1}_{0}+\int_0^1 xt^{x-1}\frac{(1-t)^y}{y}\dd{t}\\
&=\frac{x}{y}\int_0^1 t^{x-1}(1-t)^{y-1}(1-t)\dd{t}\\
&=\frac{x}{y}\brac{\int_0^1 t^{x-1}(1-t)^{y-1}\dd{t}-\int_0^1 t^x(1-t)^{y-1}\dd{t}}\\
&=\frac{x}{y}\brac{B(x,y)-B(x+1,y)}
\end{align*}
which gives $B(x+1,y)=\dfrac{x}{x+y}B(x,y)$. Thus
\begin{align*}
f(x+1)&=\frac{\Gamma(x+1+y)}{\Gamma(y)}B(x+1,y)\\
&=\frac{(x+y)B(x+y)}{\Gamma(y)}\cdot\frac{x}{x+y}B(x,y)\\
&=xf(x).
\end{align*}
\item \[B(1,y)=\int_0^1(1-t)^{y-1}\dd{t}=\sqbrac{-\frac{(1-t)^y}{y}}_0^1=\frac{1}{y}\]
and thus
\[f(1)=\frac{\Gamma(1+y)}{\Gamma(y)}B(1,y)=\frac{y\Gamma(y)}{\Gamma(y)}\frac{1}{y}=1.\]
\item We now show that $\log B(x,y)$ is convex, so that
\[\log f(x)=\underbrace{\log\Gamma(x+y)}_\text{convex}+\log B(x,y)-\underbrace{\log\Gamma(y)}_\text{constant}\]
is convex with respect to $x$.
\[B(x_1,y)^\frac{1}{p}B(x_2,y)^\frac{1}{q}=\brac{\int_0^1 t^{x_1-1}(1-t)^{y-1}\dd{t}}^\frac{1}{p}\brac{\int_0^1 t^{x_2-1}(1-t)^{y-1}\dd{t}}^\frac{1}{q}\]
By H\"{o}lder's inequality,
\begin{align*}
B(x_1,y)^\frac{1}{p}B(x_2,y)^\frac{1}{q}
&=\int_0^1\sqbrac{t^{x_1-1}(1-t)^{y-1}}^\frac{1}{p}\sqbrac{t^{x_2-1}(1-t)^{y-1}}^\frac{1}{q}\dd{t}\\
&=\int_0^1 t^{\frac{x_1}{p}+\frac{x_2}{q}-1}(1-t)^{y-1}\dd{t}\\
&=B\brac{\frac{x_1}{p}+\frac{x_2}{q},y}.
\end{align*}
Taking log on both sides gives
\[\log B(x,y)^\frac{1}{p}B(x_2,y)^\frac{1}{q}\ge\log B\brac{\frac{x_1}{p}+\frac{x_2}{q},y}\]
or
\[\frac{1}{p}\log B(x,y)+\frac{1}{q}\log B(x_2,y)\ge\log B\brac{\frac{x_1}{p}+\frac{x_2}{q},y}.\]
Hence $\log B(x,y)$ is convex, so $\log f(x)$ is convex.
\end{enumerate}
Therefore $f(x)=\Gamma(x)$ which implies $B(x,y)=\dfrac{\Gamma(x)\Gamma(y)}{\Gamma(x+y)}$.
\end{proof}

An alternative form of $\Gamma$ is as follows:
\[\Gamma(x)=2\int_0^{+\infty}t^{2x-1}e^{-t^2}\dd{t}.\]
Using this form of $\Gamma$, we present an alternative proof.

\begin{proof}
\begin{align*}
\Gamma(x)\Gamma(y)
&=\brac{2\int_0^{+\infty}t^{2x-1}e^{-t^2}\dd{t}}\brac{2\int_0^{+\infty}s^{2y-1}e^{-s^2}\dd{s}}\\
&=4\iint_{[0,+\infty)\times[0,+\infty)}t^{2x-1}s^{2y-1}e^{-\brac{t^2+s^2}}\dd{t}\dd{s}
\end{align*}
Using polar coordinates transformation, let $t=r\cos\theta$, $s=r\sin\theta$. Then $\dd{t}\dd{s}=r\dd{r}\dd{\theta}$. Thus
\begin{align*}
\Gamma(x)\Gamma(y)
&=4\int_0^\frac{\pi}{2}\sqbrac{\int_0^{+\infty}r^{2x-1}\cos^{2x-1}\theta\cdot r^{2y-1}\sin^{2y-1}\theta\cdot e^{-r^2}\cdot r\dd{r}}\dd{\theta}\\
&=\underbrace{2\int_0^\frac{\pi}{2}\cos^{2x-1}\theta\sin^{2y-1}\theta\dd{\theta}}_{B(x,y)}\cdot\underbrace{2\int_0^{+\infty}r^{2(x+y)-1}e^{-r^2}\dd{r}}_{\Gamma(x+y)}
\end{align*}
since
\begin{align*}
B(x,y)&=\int_0^1 t^{x-1}(1-t)^{y-1}\dd{t}\quad t=\cos^2\theta\\
&=\int_\frac{\pi}{2}^0 \cos^{2(x-1)}\theta\sin^{2(y-1)}\theta\cdot2\cos\theta(-\sin\theta)\dd{\theta}\\
&=2\int_0^\frac{\pi}{2}\cos^{2x-1}\theta\sin^{2y-1}\theta\dd{\theta}.
\end{align*}
Hence $B(x,y)=\dfrac{\Gamma(x)\Gamma(y)}{\Gamma(x+y)}$.
\end{proof}

More on polar coordinates:
\begin{equation}
I=\int_{-\infty}^{+\infty}e^{-x^2}\dd{x}
\end{equation}

\begin{proof}
\begin{align*}
I^2&=\int_{-\infty}^{+\infty}e^{-x^2}\dd{x}\int_{-\infty}^{+\infty}e^{-y^2}\dd{y}\\
&=\iint_{\RR^2}e^{-\brac{x^2+y^2}}\dd{x}\dd{y}\quad x=r\cos\theta,y=r\sin\theta\\
&=\int_0^{2\pi}\underbrace{\int_0^{+\infty}e^{-r^2}r\dd{r}}_\text{constant w.r.t. $\theta$}\dd{\theta}\quad s=r^2,\dd{s}=2r\dd{r}\\
&=2\pi\int_0^{+\infty}e^{-s}\cdot\frac{1}{2}\dd{s}\\
&=2\pi\sqbrac{\frac{1}{2}e^{-s}(-1)}_0^\infty=\pi
\end{align*}
and thus
\[I=\int_{-\infty}^{+\infty}e^{-x^2}\dd{x}=\sqrt{\pi}.\]
\end{proof}

From this, we have
\[\Gamma\brac{\frac{1}{2}}=2\int_0^\infty e^{-t^2}\dd{t=\sqrt{\pi}.}\]

\begin{lemma}
\[\Gamma(x)=\frac{2^{x-1}}{\sqrt{\pi}}\Gamma\brac{\frac{x}{2}}\Gamma\brac{\frac{x+1}{2}}.\]
\end{lemma}

\begin{proof}
Let $\displaystyle f(x)=\frac{2^{x-1}}{\sqrt{\pi}}\Gamma\brac{\frac{x}{2}}\Gamma\brac{\frac{x+1}{2}}$. We want to prove that $f(x)=\Gamma(x)$.
\begin{enumerate}[label=(\roman*)]
\item \begin{align*}
f(x+1)&=\frac{2^x}{\sqrt{\pi}}\Gamma\brac{\frac{x+1}{2}}\Gamma\brac{\frac{x}{2}+1}\\
&=\frac{2^x}{\sqrt{\pi}}\Gamma\brac{\frac{x+1}{2}}\frac{x}{2}\Gamma\brac{\frac{x}{2}}\\
&=xf(x)
\end{align*}
\item $f(1)=\frac{1}{\sqrt{\pi}}\Gamma\brac{\frac{1}{2}}\Gamma(1)=1$ since $\Gamma\brac{\frac{1}{2}}=\sqrt{\pi}$.
\item \[\log f(x)=\underbrace{(x-1)\log2}_\text{linear}+\underbrace{\log\Gamma\brac{\frac{x}{2}}}_\text{convex}+\underbrace{\log\Gamma\brac{\frac{x+1}{2}}}_\text{convex}-\underbrace{\log\sqrt{\pi}}_\text{constant}\]
and hence $\log f(x)$ is convex.
\end{enumerate}
Therefore $f(x)=\Gamma(x)$.
\end{proof}

\begin{theorem}[Stirling's formula]
This provides a simple approximate expression for $\Gamma(x+1)$ when $x$ is large (hence for $n!$ when $n$ is large). The formula is
\begin{equation}
\lim_{x\to\infty}\frac{\Gamma(x+1)}{(x/e)^x\sqrt{2\pi x}}=1.
\end{equation}
\end{theorem}

\begin{proof}

\end{proof}

\begin{lemma}
\[B(p,1-p)=\Gamma(p)\Gamma(1-p)=\frac{\pi}{\sin p\pi}.\]
\end{lemma}

\begin{proof}

\end{proof}
\fi

%%%%%%%%%%%%%%% COMPLEX ANALYSIS
\ifcanalysis
    \part{Complex Analysis}\label{part:complex-analysis}
    \chapter{Complex Numbers and Functions}\label{chap:complex-functions}
\section{$\CC$ As Metric Space}
We can identify $\CC$ with the plane $\RR^2$ by taking
real and imaginary parts. Thus we have mutually inverse bijections
\[z\mapsto(\Re z,\Im z)\]
from $\CC$ to $\RR^2$, and
\[(x,y)\mapsto x+iy\]
from $\RR^2$ to $\CC$. As we have seen, $\RR^2$ is a metric space with the metric induced from the Euclidean norm
\[\norm{(x,y)}_2=\sqrt{x^2+y^2}.\]
This gives a metric on $\CC$ by the identification $\CC\cong\RR^2$ described above.

If $z=\Re z+i\Im z$ is a complex number we write $|z|$ (called the \vocab{modulus}) for this Euclidean norm; that is,
\[|z|=\sqrt{(\Re z)^2+(\Im z)^2}.\]
The distance between the two points $z,w\in\CC$ is then $|z-w|$.

Let us write down some basic properties of the modulus $|z|$. Recall that $e^{i\theta}=\cos\theta+i\sin\theta$ when $\theta\in\RR$. For now, we will take this as the definition of $e^{i\theta}$. Later on we will define the complex exponential function $e^z$ and link the two concepts.

\begin{lemma}
Let $z,w\in\CC$. Then
\begin{enumerate}[label=(\roman*)]
\item $|z|^2=z\bar{z}$, where $\bar{z}$ is the complex conjugate of $z$;
\item If $z=re^{i\theta}$, where $r\in[0,\infty)$ and $\theta\in\RR$, then $|z|=r$;
\item $|zw| = |z||w|$.
\end{enumerate}
\end{lemma}

\begin{proof} \
\begin{enumerate}[label=(\roman*)]
\item If $z=a+ib$ then $z\bar{z}=(a+ib)(a-ib)=a^2+b^2$.
\item We have $z=r\cos\theta+ir\sin\theta$ and so
\[|z|=\sqrt{r^2\cos^2\theta+r^2\sin^2\theta}=r.\]
\item One can calculate directly, writing $z=a+ib$ and $w=c+id$. Alternatively, write $z=re^{i\theta}$, $w=r^\prime e^{i\alpha}$, and then observe that $zw=rr^\prime e^{i(\theta+\alpha)}$ and use (2).
\end{enumerate}
\end{proof}

\section{Complex Differentiability}
Suppose that $a\in\CC$, and that $U$ is a neighbourhood of $a$. That is, $U$ contains some ball $B_\eta(a)$ for $\eta>0$, but $U$ itself need not be open. Suppose that $f:U\setminus\{a\}\to\CC$ is a function; that is, $f$ is defined on $U$, except not at $a$. Then we say that $\displaystyle\lim_{z\to a}f(z)=L$ if
\[\forall\epsilon>0,\quad\exists\delta>0,\quad 0<|z-a|<\delta\implies|f(z)-L|<\epsilon\]
(and we assume $\delta<\eta$ so that $f$ is defined when $|z-a|<\delta$).

\subsection{Complex Differentiability}
With the relevant notions of limit having been recalled, we can give the definition of (complex) derivative. In fact, it is the same as the definition of real derivative, but with complex numbers in place of reals.

\begin{definition}[Complex differentiability]
Let $a\in\CC$, and suppose that $f:U\to\CC$ is a function, where $U$ is a neighbourhood of $a$. Then we say that $f$ is \vocab{complex differentiable}\index{complex differentiability} at $a$ if the limit
\[\lim_{z\to a}\frac{f(z)-f(a)}{z-a}\]
exists. If the limit exists, we write $f^\prime(a)$ for it and call this the \textbf{derivative} of $f$ at $a$.
\end{definition}

Since we will be talking exclusively about functions on $\CC$, we just use the terms differentiable/derivative and omit the word ``complex''.

\begin{proposition}[Differentiability implies continuity]
If $f$ is differentiable at $z$, then $f$ is continuous at $z$.
\end{proposition}

\begin{proof}
Suppose $f:U\to\CC$ has derivative 
\[f^\prime(z)=\lim_{h\to0}\frac{f(z+h)-f(z)}{h}.\]
Then
\[\lim_{h\to0}\brac{f(z+h)-f(z)}=f^\prime(z)\lim_{h\to0}h=0.\]
\end{proof}

The following lemma collects the basic facts about derivatives. We omit the proof, which is essentially identical to the real case.

\begin{lemma}
Let $a\in\CC$, let $U$ be a neighbourhood of $a$ and let $f,g:U\to\CC$.
\begin{enumerate}[label=(\roman*)]
\item (Sums, products) If $f,g$ are differentiable at $a$, then $f+g$ and $fg$ are differeitnable at $a$, and
\[(f+g)^\prime(a)=f^\prime(a)+g^\prime(a)\]
and
\[(fg)^\prime(a)=f^\prime(a)g(a)+f(a)g^\prime(a).\]
\item (Quotients) If $f,g$ are differentiable at $a$ and $g(a)\neq0$ then $f/g$ is differentiable at $a$ and
\[\brac{\frac{f}{g}}^\prime(a)=\frac{f^\prime(a)g(a)-f(a)g^\prime(a)}{g(a)^2}.\]
\item (Chain rule) If $U$ and $V$ are open subsets of $\CC$ and $f:V\to U$, $g:U\to\CC$, where $f$ is differentiable at $a\in V$ and $g$ is differeitnable at $f(a)\in U$, then $g\circ f$ is differentiable at $a$, with
\[(g\circ f)^\prime(a)=g^\prime(f(a))f^\prime(a).\]
\end{enumerate}
\end{lemma}

Just as in the real case, the basic rules of differentiation stated above allow one to check that polynomial functions are differentiable: using the product rule and induction one sees that $z^n$ has derivative $nz^{n-1}$ for all $n\ge0$ (as a constant obviously has derivative $0$, and $f(z)=z$ has derivative $1$). Then by linearity it follows every polynomial is differentiable.

Just as in the real-variable case, one can formulate complex differentiability in the following form, which is in fact the better form to use in most instances.

\begin{lemma}
Let $a\in\CC$, let $U$ be a neighbourhood of $a$ and let $f:U\to\CC$. Then $f$ is differentiable at $a$, with derivative $f^\prime(a)$, if and only if
\[f(z)=f(a)+f^\prime(a)(z-a)+\epsilon(z)(z-a),\]
where $\epsilon(z)\to0$ as $z\to a$.
\end{lemma}

\begin{proof}
check that this definition is indeed equivalent to (really just a reformulation of) the previous one.
\end{proof}

\begin{definition}[Holomorphic function]
Let $U\subset\CC$ be an open set. If $f:U\to\CC$ is complex differentiable at every $a\in U$, we say that $f$ is \vocab{holomorphic}\index{holomorphic function} on $U$.
\end{definition}

\subsection{Cauchy--Riemann Equations}
A function from $\CC$ to $\CC$ may also be thought of as a function from $\RR^2$ to $\RR^2$, and it is useful to study what differentiability means in this language.

Let $a\in\CC$, and let $U$ be a neighbourhood of $a$. Let $f:U\to\CC$ be a function. We abuse notation and identify $\CC\cong\RR^2$ in the usual way, and identify $a$ with $(a_1,a_2)$ (thus $a=a_1+ia_2$). Then (again with some abuse of notation) we may think of $U$ as an open subset of $\RR^2$ and write $f=(u,v)$, where $u,v:\RR^2\to\RR$ are called the \emph{components} of $f$. Another way to think of this is that
\[f(x+iy)=u(x,y)+iv(x,y).\]

\begin{example}
Consider the function $f(z)=z^2$ (which is holomorphic on all of $\CC$). Since
\[(x+iy)^2=(x^2-y^2)+(2xy)i,\]
the components of $f$ are given by
\begin{align*}
u(x,y)&=x^2-y^2,\\
v(x,y)&=2xy.
\end{align*}
\end{example}

We have the partial derivatives
\[\pdv{u(a)}{x}\coloneqq\lim_{h\to0}\frac{u(a_1+h,a_2)-u(a_1,a_2)}{h}\]
(if the limit exists) and
\[\pdv{u(a)}{y}\coloneqq\lim_{k\to0}\frac{u(a_1,a_2+k)-u(a_1,a_2)}{k},\]
and similarly for $v$. It is important to note that $h, k$ in these limits are real.

An important fact is that if $f$ is differentiable then these partial derivatives do exist, and moreover they are subject to a constraint.

\begin{theorem}[Cauchy--Riemann equations]
Let $a\in\CC$, let $U$ be a neighbourhood of $a$, and let $f:U\to\CC$ be a function which is complex differentiable at $a$. Let $u,v:\RR^2\to\RR$ be the components of $f$. Then the four partial derivatives $\displaystyle\pdv{u}{x}, \pdv{u}{y}, \pdv{v}{x}, \pdv{v}{y}$ exist at $a$. Moreover, we have the Cauchy--Riemann equations
\begin{equation}
\pdv{u}{x}=\pdv{v}{y}\quad\text{and}\quad\pdv{v}{x}=-\pdv{u}{y}
\end{equation}
and $\displaystyle f^\prime(a)=\pdv{u(a)}{x}+i\pdv{v(a)}{x}$.
\end{theorem}

\begin{proof}
We have
\[f(z)=f(a)+f^\prime(a)(z-a)+\epsilon(z)(z-a),\]
where $\epsilon(z)\to0$ as $z\to a$. Identifying $\CC\cong\RR^2$ and writing $a=(a_1,a_2)$, $z=(a_1+h,a_2+k)$, $f=(u,v)$ and $f^\prime(a)=(b_1,b_2)$, this gives
(u(a1 + h, a2 + k), v(a1 + h, a2 + k))
= (u(a1, a2), v(a1, a2)) + (b1, b2) · (h, k) + (ε1(h, k), ε2(h, k)) · (h, k).
The · here means complex multiplication, under the identification of C and
R
2
: thus
(b1, b2) · (h, k) = (b1h − b2k, b1k + b2h)
(because (b1 + ib2)(h + ik) = (b1h − b2k) + i(b2h + b1k)). The functions
ε1(h, k), ε2(h, k) both tend to 0 as k(h, k)k → 0.
Looking at the first component, we have
u(a1 + h, a2 + k) = u(a1, a2) + b1h − b2k + ε1(h, k)h − ε2(h, k)k.
In particular,
u(a1 + h, a2) = u(a1, a2) + b1h + ε1(h, 0)h.

Since $\epsilon_1(h,0)\to0$ as $|h|\to0$, it follows that $\pdv{u(a)}{x}$ exists and equals $b_1$.

Very similar arguments may be used for the other partial derivatives and we see that they all exist, and that
\[∂yu(a) = −b2, ∂xv(a) = b2, ∂yv(a) = b1.\]
Everything stated in the theorem now follows. 
\end{proof}

Assuming for the time being that $u,v$ have continuous partial derivatives of all orders (and in particular the mixed partials are equal), we can show that
\[\Delta u=\pdv[2]{u}{x}+\pdv[2]{u}{y}=0,\quad\Delta v=\pdv[2]{v}{x}+\pdv[2]{v}{y}=0.\]
Such an equation $\Delta u=0$ is called Laplace's equation and its solution is said to be a harmonic function.

Let us pause to give a simple example using the Cauchy-Riemann equations, which shows that complex differentiation is a much more rigid property than one might think at first sight.

\begin{example}
The function $f(z)=\bar{z}$ is not (complex) differentiable anywhere.
\end{example}

\begin{proof}
Let $u,v:\RR^2\to\RR$ be the components of $f$. Then clearly $u(x,y) = x$, $v(x,y)=-y$ and so $\partial_x u=1,\partial_y u=0, \partial_x v=0, \partial_y v=-1$. Thus $\partial_xu$ is never equal to $\partial_yv$, so the Cauchy--Riemann equations are never satisfied.
\end{proof}

$\CC$ is $\RR^2$ with a multiplication. Note that each map $f:\CC\to\CC$ induces a map $f_R:\RR^2\to\RR^2$ (and vice versa).
\begin{example}
Consider $f:\CC\to\CC$, $z\mapsto z^2$.

This is equivalent to $x+iy\mapsto (x+iy)^2=(x^2-y^2)+(2xy)i$.

Thus the mapping is the same as $f_R:\RR^2\to\RR^2$, $(x,y)\mapsto(x^2-y^2,2xy)$.
\end{example}
We want to form a connection between differentiability in $\CC$ and $\RR^2$.
\begin{definition}
A map $f_R:\RR^2\to\RR^2$ is called (totally) differentiable at $\begin{pmatrix}x_0\\ y_0\end{pmatrix}$ if there is a matrix $J\in\RR^{2\times2}$ and a map $\phi:\RR^2\to\RR^2$
\[f_R\brac{\begin{pmatrix}x\\y\end{pmatrix}}=\underbrace{f_R\brac{\begin{pmatrix}x_0\\y_0\end{pmatrix}}+J\brac{\begin{pmatrix}x\\y\end{pmatrix}-\begin{pmatrix}x_0\\ y_0\end{pmatrix}}}_{\text{linear approximation}}+\underbrace{\phi\brac{\begin{pmatrix}x\\ y\end{pmatrix}}}_{\text{error term}}\]
where $\frac{\phi\brac{\begin{pmatrix}x\\ y\end{pmatrix}}}{\norm{\begin{pmatrix}x\\y\end{pmatrix}-\begin{pmatrix}x_0\\ y_0\end{pmatrix}}}\to0$ as $\begin{pmatrix}x\\y\end{pmatrix}\to\begin{pmatrix}x_0\\ y_0\end{pmatrix}$.

$J$ is called the \textbf{Jacobian matrix} of $f_R$ at $\begin{pmatrix}x_0\\y_0\end{pmatrix}\in\RR^2$:
\[J=\begin{pmatrix}
\vdots&\vdots\\
\frac{\partial f_R}{\partial x}&\frac{\partial f_R}{\partial y}\\
\vdots&\vdots
\end{pmatrix}\]
\end{definition}
\begin{example}
Considering the above example, 
\[J=\begin{pmatrix}
2x&-2y\\
2y&2x
\end{pmatrix}.\]
\end{example}



\begin{comment}
\begin{definition}[Derivative]
The \vocab{derivative} of $f(z)$ at $z=a$ is given by the limit
\[f^\prime(a)=\lim_{z\to a}\frac{f(z)-f(a)}{z-a}.\]
\end{definition}

\subsection{Analytic Functions}
The class of \vocab{analytic functions} is formed by the complex functions of a complex variable which possess a derivative wherever the function is defined.

The definition of the derivative can be rewritten in the form
\[f^\prime(z)=\lim_{h\to0}\frac{f(z+h)-f(z)}{h}.\]
\begin{lemma}[Differentiability implies continuity]
$f(z)$ is continuous if it is analytic.
\end{lemma}

\begin{proof}
From $\displaystyle f(z+h)-f(z)=h\cdot\frac{f(z+h)-f(z)}{h}$ we obtain
\[\lim_{h\to0}\sqbrac{f(z+h)-f(z)}=0\cdot f^\prime(z)=0.\]
\end{proof}

If we write $f(z)=u(z)+iv(z)$ it follows that $u(z)$ and $v(z)$ are both continuous.

The limit of the difference quotient must be the same regardless of the way in which $h$ approaches zero. If we choose real values for $h$, then the imaginary part $y$ is kept constant, and the derivative becomes a partial derivative with respect to $x$. We have thus
\[f^\prime(z)=\pdv{f}{x}=\pdv{u}{x}+i\pdv{v}{x}.\]
Similarly, if we substitute purely imaginary values $ik$ for $h$, we obtain 
\[f^\prime(z)=\lim_{k\to0}\frac{f(z+ik)-f(z)}{ik}=-i\pdv{f}{y}=-i\pdv{u}{y}+\pdv{v}{y}.\]
It follows that $f(z)$ must satisfy the partial differential equation
\[\pdv{f}{x}=-i\pdv{f}{y}\]
which resolves into the real equations
\begin{equation}\label{eqn:cauchy-riemann}
\pdv{u}{x}=\pdv{v}{y},\quad\pdv{u}{y}=-\pdv{v}{x}.
\end{equation}
These are the \vocab{Cauchy--Riemann equations} which must be satisfied by the real and imaginary part of any analytic function.

Using \cref{eqn:cauchy-riemann}, we can write down four formally different expressions for $f^\prime(z)$; the simplest is
\[f^\prime(z)=\pdv{u}{x}+i\pdv{v}{x}.\]
For the quantity $|f^\prime(z)|^2$, we have, for instance,
\[\absolute{f^\prime(z)}^2=\brac{\pdv{u}{x}}^2+\brac{\pdv{u}{y}}^2=\brac{\pdv{u}{x}}^2+\brac{\pdv{v}{x}}^2=\pdv{u}{x}\pdv{v}{y}-\pdv{u}{y}\pdv{v}{x}.\]
The last expression shows that $\absolute{f^\prime(z)}^2$ is the Jacobian of $u$ and $v$ with respect to $x$ and $y$.

We shall prove later that the derivative of an analytic function is itself analytic. By this fact $u$ and $v$ will have continuous partial derivatives of all orders, and in particular the mixed derivatives will be equal. Using this information we obtain from \cref{eqn:cauchy-riemann},
\begin{align*}
\Delta u&=\pdv[2]{u}{x}+\pdv[2]{u}{y}=0,\\
\Delta v&=\pdv[2]{v}{x}+\pdv[2]{v}{y}=0.
\end{align*}
A function $u$ which satisfies \textbf{Laplace's equation} $\Delta u=0$ is said to be \vocab{harmonic}. The real and imaginary part of an analytic function are thus harmonic. If two harmonic functions $u$ and $v$ satisfy the Cauchy--Riemann equations, then $v$ is said to be the \textbf{conjugate harmonic function} of $u$.










\end{comment}

\fi

%%%%%%%%%%%%%%% TOPOLOGY
\iftop
    \part{Topology}\label{part:topology}
You have already studied metric spaces in some detail. These are objects where one has a notion of distance between points, satisfying some simple axioms. They have a rich and interesting theory, which leads to such concepts as connectedness, completeness and
compactness.

Two metric spaces are viewed as ``the same'' if there is an isometry between them, which is a bijection that preserves distances. But there is a much more flexible notion of equivalence: two spaces are homeomorphic if there is a continuous bijection between them with continuous inverse. Many properties of metric spaces are preserved by a homeomorphism (for example, connectedness and compactness). Thus homeomorphic metric spaces may have very different metrics, but nevertheless have many properties in common. The conclusion to draw from this is that a metric is, frequently, a somewhat artificial and rigid piece of structure. So, one is led naturally to the study of Topology. The fundamental objects in Topology are topological spaces. Here, there is no metric in general. But one still has a notion of open sets, and so concepts such as connectedness and compactness continue to make sense.

Why study Topology? The reason is that it simultaneously simplifies and generalises the theory of metric spaces. By discarding the metric, and focusing solely on the more basic and fundamental notion of an open set, many arguments and proofs are simplified. And many constructions (such as the important concept of a quotient space) cannot be carried out in the setting of metric spaces: they need the more general framework of topological spaces. But perhaps the most important reason is that the spaces that arise naturally in Topology have a particularly beautiful theory.
    \chapter{Topological Spaces and Continuous Functions}\label{chap:topological-spaces}
\section{Topological Spaces}
\begin{definition}[Topological space]
A \vocab{topology} on a set $X$ is a collection $\mathcal{T}$ of subsets of $X$ satisfying:
\begin{enumerate}[label=(\roman*)]
\item $X,\emptyset\in\mathcal{T}$;
\item if $U_i\in\mathcal{T}$ for all $i\in I$, then $\bigcup_{i\in I}U_i\in\mathcal{T}$;\hfill(closed under arbitrary unions)
\item if $U_1,\dots,U_n\in\mathcal{T}$, then $\bigcap_{i=1}^{n}U_i\in\mathcal{T}$.\hfill(closed under finite intersections)
\end{enumerate}
A set $X$ for which a topology $\mathcal{T}$ has been specified is called a \vocab{topological space}, denoted by $(X,\mathcal{T})$. $U\subset X$ is called an \emph{open set} of $X$ if $U\in\mathcal{T}$.
\end{definition}

\begin{notation}
When $\mathcal{T}$ is understood, we talk about the topological space $X$.
\end{notation}

\begin{example}
Let $X$ be any non-empty set.
\begin{itemize}
\item The \emph{discrete topology} on $X$ is the set of all subsets of $X$; that is, $\mathcal{T}=\mathcal{P}(X)$.
\item The \emph{indiscrete topology} (or \emph{trivial topology}) on $X$ is $\mathcal{T}=\{X,\emptyset\}$.
\item The \emph{co-finite topology} on $X$ consists of the empty set together with every subset $U$ of $X$ such that $X\setminus U$ is finite.
\item Let $\mathcal{T}_c$ be the collection of all subsets $U\subset X$ such that $U^c$ either is countable or is all of $X$. Then $\mathcal{T}_c$ is a topology on $X$.
\end{itemize}
\end{example}

\begin{definition}
Suppose $\mathcal{T}$ and $\mathcal{T}^\prime$ are two topologies on a given set $X$. We say that
\begin{enumerate}[label=(\roman*)]
\item $\mathcal{T}$ is \vocab{finer} than $\mathcal{T}^\prime$ if $\mathcal{T}\supset\mathcal{T}^\prime$;
\item $\mathcal{T}$ is \emph{coarser} than $\mathcal{T}^\prime$ if $\mathcal{T}\subset\mathcal{T}^\prime$;
\item $\mathcal{T}$ is \emph{comparable} with $\mathcal{T}^\prime$ if either $\mathcal{T}\supset\mathcal{T}^\prime$ or $\mathcal{T}\subset\mathcal{T}^\prime$.
\end{enumerate}
\end{definition}

\begin{remark}
The indiscrete topology is the coarsest topology possible, while the discrete topology is the finest topology possible.
\end{remark}

\section{Basis for a Topology}
In linear algebra, every vector space is generated by a basis. In topology, we have a similar notion, as it is usually hard to define a topology by specifying all the open sets.

\begin{definition}[Basis]\label{defn:topology-basis}
A \vocab{basis} for a topology on $X$ is a collection $\mathcal{B}$ of subsets of $X$ (called \emph{basis elements}) if
\begin{enumerate}[label=(\roman*)]
\item for all $x\in X$, there exists $B\in\mathcal{B}$ such that $x\in B$;
\item for all $B_1,B_2\in\mathcal{B}$ and $x\in B_1\cap B_2$, there exists $B_3\in\mathcal{B}$ such that $x\in B_3\subset B_1\cap B_2$.
\end{enumerate}
We define the \emph{topology $\mathcal{T}$ generated by basis $\mathcal{B}$} as
\begin{equation}
U\in\mathcal{T}\iff\forall x\in U,\:\exists B\in\mathcal{B},\:x\in B\subset U.
\end{equation}
\end{definition}

We now check that the collection $\mathcal{T}$ generated by the basis $\mathcal{B}$ is, in fact, a topology on $X$.
\begin{enumerate}[label=(\roman*)]
\item $\emptyset$ satisfies the defining condition of openness vacuously, so $\emptyset\in\mathcal{T}$. $X\in\mathcal{T}$ follows from (i) of \cref{defn:topology-basis}.
\item Consider a collection $\{U_i\mid i\in I\}$ of elements of $\mathcal{T}$. We want to show that $U=\bigcup_{i\in I}U_i\in\mathcal{T}$. 

Given $x\in U$, there exists $i\in I$ such that $x\in U_i$. Since $U_i\in\mathcal{T}$, there exists $B\in\mathcal{B}$ such that $x\in B\subset U_i$. Thus $x\in B\subset U$, so $U\in\mathcal{T}$.

\item Take two elements $U_1,U_2\in\mathcal{T}$, we want to show that $U_1\cap U_2\in\mathcal{T}$.

Given $x\in U_1\cap U_2$, choose $B_1\in\mathcal{B}$ such that $x\in B_1\subset U_1$; choose $B_2\in\mathcal{B}$ such that $x\in B_2\subset U_2$. Then $x\in B_1\cap B_2$.

Since $\mathcal{B}$ is a basis, by (ii) of \cref{defn:topology-basis}, there exists $B_3\in\mathcal{B}$ such that $x\in B_3\subset B_1\cap B_2$. Thus $U_1\cap U_2\in\mathcal{T}$.

Finally, we show by induction that any finite intersection $U_1\cap\cdots\cap U_n\in\mathcal{T}$. This is trivial for $n=1$; suppose it true for $n-1$ and prove it
for $n$. Now
\[(U_1\cap\cdots\cap U_n)=(U_1\cap\cdots\cap U_{n-1})\cap U_n.\]
By hypothesis, $U_1\cap\cdots\cap U_{n-1}\in\mathcal{T}$. Thus by the result just proved, the intersection of $U_1\cap\cdots\cap U_{n-1}$ and $U_n$ also belongs to $\mathcal{T}$.
\end{enumerate}

Another way of describing the topology generated by a basis is given in the following result:

\begin{lemma}\label{lemma:topology-generated-basis-unions}
Let $\mathcal{T}$ be the topology on $X$ generated by basis $\mathcal{B}$. Then $\mathcal{T}$ equals the collection of all unions of elements of $\mathcal{B}$.
\end{lemma}

\begin{proof}
Let $\mathcal{B}=\{B_i\mid i\in I\}$.
\begin{itemize}
\item If $B_i\in\mathcal{B}$, see that
\[\forall x\in B,\:x\in B\subset B\implies B\in\mathcal{T}.\]
Since $\mathcal{T}$ is a topology, the arbitrary unions of $B_i$'s must be in $\mathcal{T}$.
\item Conversely, given $U\in\mathcal{T}$, for each $x\in U$, there exists $B_x\in\mathcal{B}$ such that $x\in B_x\subset U$. Then $U=\bigcup_{x\in U}B_x$, so $U$ is a union of elements of $\mathcal{B}$.
\end{itemize}
\end{proof}

\begin{remark}
The above result states that every $U\in\mathcal{T}$ can be expressed as a union of basis elements.
\end{remark}

We have described in two different ways how to go from a basis to the topology it generates. Sometimes we need to go in the reverse direction, from a topology to a basis generating it. Here is one useful way of obtaining a basis for a given topology.

\begin{lemma}
Let $(X,\mathcal{T})$ be a topological space. Suppose that $\mathcal{C}$ is a collection of open sets of $X$, such that
\[\forall U\in\mathcal{T},\quad\forall x\in U,\quad\exists C\in\mathcal{C},\quad x\in C\subset U.\]
Then $\mathcal{C}$ is a basis for $\mathcal{T}$.
\end{lemma}

\begin{proof}
We first show that $\mathcal{C}$ is a basis.
\begin{enumerate}[label=(\roman*)]
\item For all $x\in X$, since $X\in\mathcal{T}$, by hypothesis, there exists $C\in\mathcal{C}$ such that $x\in C\subset X$.
\item Let $x\in C_1\cap C_2$, where $C_1,C_2\in\mathcal{C}\subset\mathcal{T}$. Thus $C_1,C_2\in\mathcal{T}$, so $C_1\cap C_2\in\mathcal{T}$. Hence by hypothesis, there exists $C_3\in\mathcal{C}$ such that $x\in C_3\subset C_1\cap C_2$.
\end{enumerate}

Let $\mathcal{T}^\prime$ be the topology generated by $\mathcal{C}$. We will show that $\mathcal{T}=\mathcal{T}^\prime$.
\begin{itemize}
\item Let $U\in\mathcal{T}$, $x\in U$. By hypothesis, there exists $C\in\mathcal{C}$ such that $x\in C\subset U$. By definition, $U\in\mathcal{T}^\prime$. Hence $\mathcal{T}\subset\mathcal{T}^\prime$.
\item Conversely, let $W\in\mathcal{T}^\prime$. By \cref{lemma:topology-generated-basis-unions}, $W$ is a union of elements of $\mathcal{C}$. Since each element of $\mathcal{C}$ is an element of $\mathcal{T}$ (and thus open), and a union of open sets is open, so $W\in\mathcal{T}$. Hence $\mathcal{T}^\prime\subset\mathcal{T}$.
\end{itemize}
\end{proof}

When topologies are given by bases, the next result is a criterion to determine whether one topology is finer than another.

\begin{lemma}
Let $\mathcal{B}$ and $\mathcal{B}^\prime$ be bases for the topologies $\mathcal{T}$ and $\mathcal{T}^\prime$ respectively on $X$. Then the following are equivalent:
\begin{enumerate}[label=(\roman*)]
\item $\mathcal{T}^\prime$ is finer than $\mathcal{T}$.
\item For all $x\in X$, and for all $B\in\mathcal{B}$ such that $x\in B$, there exists $B^\prime\in\mathcal{B}^\prime$ such that $x\in B^\prime\subset B$.
\end{enumerate}
\end{lemma}

\begin{proof} \

\fbox{(ii)$\implies$(i)} Let $U\in\mathcal{T}$. To show that $\mathcal{T}\subset\mathcal{T}^\prime$, we want to show that $U\in\mathcal{T}^\prime$.

Let $x\in U$. Since $\mathcal{B}$ generates $\mathcal{T}$, there exists $B\in\mathcal{B}$ such that $x\in B\subset U$. By (ii), there exists $B^\prime\in\mathcal{B}^\prime$ such that $x\in B^\prime\subset B$. Then $x\in B^\prime\subset U$, so $U\in\mathcal{T}^\prime$, by definition.

\fbox{(i)$\implies$(ii)} We are given $x\in X$ and $B\in\mathcal{B}$, with $x\in B$.

Now $B\in\mathcal{T}$ by definition, and $\mathcal{T}\subset\mathcal{T}^\prime$ by (i); therefore, $B\in\mathcal{T}^\prime$. Since $\mathcal{T}^\prime$ is generated by $\mathcal{B}^\prime$, there exists $B^\prime\in\mathcal{B}^\prime$ such that $x\in B^\prime\subset B$.
\end{proof}

We now define three topologies on the real line $\RR$.

\begin{definition} \
\begin{enumerate}[label=(\roman*)]
\item Let $\mathcal{B}$ be the collection of all open intervals in $\RR$. The topology generated by $\mathcal{B}$ is called the \vocab{standard topology} on $\RR$.

Whenever we consider $\RR$, we shall suppose it is given this topology unless stated otherwise. 

\item Let $\mathcal{B}^\prime$ be the collection of all half-open intervals of the form $[a,b)$. The topology generated by $\mathcal{B}^\prime$ is called the \vocab{lower limit topology} on $\RR$. 

When $\RR$ is given the lower limit topology, we denote it by $\RR_\ell$.

\item Let $K=\{\frac{1}{n}\mid n\in\ZZ^+\}$, and let $\mathcal{B}^{\prime\prime}$ be the collection of all open intervals $(a,b)$, along with all sets of the form $K\setminus(a,b)$. The topology generated by $B^{\prime\prime}$ is called the \vocab{$K$-topology} on $\RR$.

When $\RR$ is given this topology, we denote it by $\RR_K$.
\end{enumerate}
\end{definition}

It is easy to see that all three of these collections are bases; in each case, the intersection of two basis elements is either another basis element or is empty. The relation between these topologies is the following:

\begin{lemma}
The topologies of $\RR_\ell$ and $\RR_K$ are strictly finer than the standard topology on $\RR$, but are not comparable with one another.
\end{lemma}

\begin{definition}[Subbasis]
A \vocab{subbasis} $\mathcal{S}$ for a topology on $X$ is a collection of subsets of $X$ whose union equals $X$.

The \emph{topology $\mathcal{T}$ generated by the subbasis} $\mathcal{S}$ is defined as the collection of all unions of finite intersections of elements of $\mathcal{S}$:
\[U\in\mathcal{T}\iff U=\text{union of finite intersections in }\mathcal{S}.\]
\end{definition}

We now check that $\mathcal{T}$ is a topology. Consider the collection
\[\mathcal{B}=\{\text{all finite intersections of elements of }\mathcal{S}\}.\]
It suffices to show that $\mathcal{B}$ is a basis, for then by \cref{lemma:topology-generated-basis-unions}, the collection $\mathcal{T}$ of all unions of elements of $\mathcal{B}$ is a topology.
\begin{enumerate}[label=(\roman*)]
\item Given $x\in X$, it belongs to an element of $\mathcal{S}$ and hence to an element of $\mathcal{B}$.
\item Let
\[B_1=S_1\cap\cdots\cap S_m,\quad B_2=S_1^\prime\cap\cdots\cap S_n^\prime\]
be two elements of $\mathcal{B}$. Their intersection
\[B_1\cap B_2=(S_1\cap\cdots\cap S_m)\cap(S_1^\prime\cap\cdots\cap S_n^\prime)\]
is also a finite intersection of elements of $\mathcal{S}$, so it belongs to $\mathcal{B}$.
\end{enumerate}
\pagebreak

\section{Examples of Topologies}
\subsection{Order Topology}
\begin{definition}[Order topology]
Let $(X,<)$, $|X|>1$. Let $\mathcal{B}$ be the collection of all sets of the following types:
\begin{enumerate}[label=(\roman*)]
\item All open intervals $(a,b)$ in $X$.
\item All intervals of the form $[a_0,b)$, where $a_0$ is the smallest element (if any) of $X$.
\item All intervals of the form $(a,b_0]$, where $b_0$ is the largest element (if any) of $X$.
\end{enumerate}
The topology generated by $\mathcal{B}$ is called the \vocab{order topology}.
\end{definition}

We need to check that $\mathcal{B}$ is a basis of $X$.
\begin{enumerate}[label=(\roman*)]
\item Every $x\in X$ lies in some element of $\mathcal{B}$: the smallest element (if any) lies in all sets of type (ii), the largest element (if any) lies in all sets of type (iii), and every other element lies in a set of type (i).
\item The intersection of any two sets of the preceding types is a set of one of these types, or is empty. Several cases need to be checked; we leave it to you.

For instance, let $x\in(a,b)\cap(c,d)$. Let $p=\max\{a,c\}$, $q=\min\{b,d\}$. Then $x\in(p,q)\subset(a,b)\cap(c,d)$, where $(p,q)\in\mathcal{B}$.
\end{enumerate}

\begin{example}
\begin{itemize}
\item The standard topology on $\RR$ is just the order topology derived from the usual order on $\RR$.
\end{itemize}
\end{example}

\begin{definition}
Let $(X,<)$, $a\in X$. Then the following subsets of $X$ are \vocab{rays} determined by $a$:
\begin{align*}
(a,+\infty)&=\{x\in X\mid x>a\},\\
[a,+\infty)&=\{x\in X\mid x\ge a\},\\
(-\infty,a)&=\{x\in X\mid x<a\},\\
(-\infty,a]&=\{x\in X\mid x\le a\}.
\end{align*}
\end{definition}

$(a,+\infty)$ and $(-\infty,a)$ are called \emph{open rays}, since they are open; for instance, $(a,+\infty)=\bigcup_{x>a}(a,x)$. Similarly, $[a,+\infty)$ and $(-\infty,a]$ are \emph{closed rays}.

\begin{lemma}
The collection of open rays form a subbasis for the order topology.
\end{lemma}

\begin{proof}
Let $\mathcal{T}$ be the order topology on $X$, let $\mathcal{T}^\prime$ be the topology generated by the subbasis of open rays. We will show that $\mathcal{T}=\mathcal{T}^\prime$.
\begin{itemize}
\item Because the open rays are open in the order topology, the topology they generate is contained in the order topology. Hence $\mathcal{T}^\prime\subset\mathcal{T}$.
\item On the other hand, every basis element for the order topology equals a finite intersection of open rays; the interval $(a,b)$ equals the intersection of $(-\infty,b)$ and $(a,+\infty)$, while $[a_0,b)$ and $(a,b_0]$, if they exist, are themselves open rays. Hence the topology generated by the open rays contains the order topology, so $\mathcal{T}\subset\mathcal{T}^\prime$.
\end{itemize}
\end{proof}

\subsection{Product Topology}
\begin{definition}
Let $(X,\mathcal{T}_X)$ and $(Y,\mathcal{T}_Y)$ be topological spaces. The \vocab{product topology} on $X\times Y$ is the topology $\mathcal{T}_{X\times Y}$ with basis
\[\mathcal{B}=\{U\times V\mid U\in\mathcal{T}_X,V\in\mathcal{T}_Y\}.\]
\end{definition}

We first check that $\mathcal{B}$ is a basis.
\begin{enumerate}[label=(\roman*)]
\item $X\times Y$ is a basis element, so every element of $X\times Y$ is contained in $X\times Y$.
\item Let $U_1\times V_1,U_2\times V_2\in\mathcal{B}$. Then their intersection is
\[(U_1\times V_1)\cap(U_2\times V_2)=(U_1\cap U_2)\times(V_1\cap V_2).\]
Since $U_1\cap U_2\in\mathcal{T}_X$, $V_1\cap V_2\in\mathcal{T}_Y$, we have that $(U_1\cap U_2)\times(V_1\cap V_2)\in\mathcal{B}$.
\end{enumerate}

\subsection{Subspace Topology}
\begin{definition}[Subspace]
Let $(X,\mathcal{T})$ be a topological space. If $Y\subset X$, the collection
\[\mathcal{T}_Y\coloneqq\{Y\cap U\mid U\in\mathcal{T}\}\]
is a topology on $Y$, called the \vocab{subspace topology}. With this topology, $Y$ is called a \vocab{subspace} of $X$; its open sets consist of all intersections of open sets of $X$ with $Y$.
\end{definition}

We check that $\mathcal{T}_Y$ is a topology.

\begin{lemma}
If $\mathcal{B}$ is a basis for the topology of $X$, then
\[\mathcal{B}_Y=\{B\cap Y\mid B\in\mathcal{B}\}\]
is a basis for the subspace topology on $Y$.
\end{lemma}

\begin{lemma}
Let $Y$ be a subspace of $X$. If $U$ is open in $Y$, and $Y$ is open in $X$, then $U$ is open in $X$.
\end{lemma}

\begin{proposition}
If $A$ is a subspace of $X$, and $B$ is a subspace of $Y$, then the product topology on $A\times B$ is the same as the topology $A\times B$ inherits as a subspace of $X\times Y$.
\end{proposition}
\pagebreak

\section{Closed Sets and Limit Points}
Let $X$ be a topological space.

Note that if $U$ is an open set containing $x$, we often say that $U$ is a \vocab{neighbourhood} of $x$.

\subsection{Closed Sets}
\begin{definition}[Closed set]
$A\subset X$ is \vocab{closed} if its complement $A^c$ is open.
\end{definition}

The collection of closed subsets of a space $X$ has properties similar to those satisfied by the collection of open subsets of $X$:
\begin{lemma}
Let $X$ be a topological space.
\begin{enumerate}[label=(\roman*)]
\item $\emptyset$ and $X$ are closed.
\item Arbitrary intersections of closed sets are closed.
\item Finite unions of closed sets are closed.
\end{enumerate}
\end{lemma}

\begin{proof} \
\begin{enumerate}[label=(\roman*)]
\item $\emptyset$ and $X$ are closed because they are the complements of the open sets $X$ and $\emptyset$, respectively.
\item Suppose $\{A_i\mid i\in I\}$ is a collection of closed sets. By de Morgan's laws,
\[\brac{\bigcap_{i\in I}A_i}^c=\bigcup_{i\in I}{A_i}^c.\]
Since ${A_i}^c$'s are open, the RHS is open since it is an arbitrary union of open sets. Hence $\bigcap A_i$ is closed.
\item Suppose $A_i$ is closed for $i=1,\dots,n$. Then
\[\brac{\bigcup_{i=1}^{n}A_i}^c=\bigcap_{i=1}^{n}{A_i}^c.\]
The RHS is a finite intersection of open sets and is thus open. Hence $\bigcup A_i$ is closed.
\end{enumerate}
\end{proof}

\begin{remark}
Note that $\emptyset$ and $X$ are both open and closed. This explains the statement ``a door is not a set'': a door must be either open or closed, and cannot be both, while a set can be open, or closed, or both, or neither!
\end{remark}

If $Y$ is a subspace of $X$, we say $A$ is closed in $Y$ if $A\subset Y$ and $A$ is closed in the subspace topology of $Y$ (that is, if $Y\setminus A$ is open in $Y$). We have the following result:

\begin{proposition}
Let $Y$ be a subspace of $X$. Then $A$ is closed in $Y$ if and only if it equals the intersection of a closed set of $X$ with $Y$.
\end{proposition}

\begin{proof} \

\fbox{$\impliedby$} Assume that $A=C\cap Y$, where $C$ is closed in $X$. Then $X\setminus C$ is open in $X$, so that $(X\setminus C)\cap Y$ is open in $Y$, by definition of the subspace topology. But $(X\setminus C)\cap Y=Y\setminus A$. Hence $Y\setminus A$ is open in $Y$, so that $A$ is closed in $Y$.

\fbox{$\implies$} Suppose $A$ is closed in $Y$. Then $Y\setminus A$ is open in $Y$, so that by definition it equals the intersection of an open set $U$ of $X$ with $Y$. The set $X\setminus U$ is closed in $X$, and $A=Y\cap(X\setminus U)$, so that $A$ equals the intersection of a closed set of $X$ with $Y$, as desired.
\end{proof}

\begin{proposition}
Let $Y$ be a subspace of $X$. If $A$ is closed in $Y$, and $Y$ is closed in $X$, then $A$ is closed in $X$.
\end{proposition}

\subsection{Closure and Interior}
\begin{definition}
The \vocab{interior} of $A\subset X$ is the union of all open sets contained in $A$, denoted by $\Int A$.

The \vocab{closure} of $A$ is the intersection of all closed sets contained in $A$, denoted by $\overline{A}$.
\end{definition}

\begin{proposition}
Let $Y$ be a subspace of $X$; let $A\subset Y$, let $\overline{A}$ denote the closure of $A$ in $X$. Then the closure of $A$ in $Y$ equals $\overline{A}\cap Y$.
\end{proposition}



\subsection{Limit Points}
\begin{definition}
Suppose $A\subset X$. $x\in X$ is a limit point of $A$ if every neighbourhood of $x$ intersects $A$ in some point other than $x$ itself.
\end{definition}

$A^\prime$ denotes the set of all limit points of $A$.

\begin{proposition}
Let $A\subset X$. Then $\overline{A}=A\cup A^\prime$.
\end{proposition}

\begin{corollary}
$A\subset X$ is closed if and only if it contains all its limit points.
\end{corollary}

\subsection{Hausdorff Spaces}
\begin{definition}[Hausdorff space]
A \vocab{Hausdorff space} is a topological space $X$ such that for all distinct $x_1,x_2\in X$, there exist neighbourhoods $U_1$ and $U_2$ of $x_1$ and $x_2$ respectively that are disjoint.
\end{definition}

\begin{proposition}
Every finite point set in a Hausdorff space $X$ is closed.
\end{proposition}

The condition that finite point sets be closed is in fact weaker than the Hausdorff condition. For example, $\RR$ in the finite complement topology is not a Hausdorff space, but it is a space in which finite point sets are closed. The condition that finite point sets be closed has been given a name of its own: it is called the \emph{T1 axiom}.

\begin{proposition}
Let $X$ be a space satisfying the T1 axiom; let $A\subset X$. Then $x$ is a limit point of $A$ if and only if every neighborhood of $x$ contains infinitely many points of $A$.
\end{proposition}

\begin{proposition}
If $X$ is a Hausdorff space, then a sequence of points of $X$ converges to at most one point of $X$.
\end{proposition}

\begin{proposition}
Every simply ordered set is a Hausdorff space in the order topology. The product of two Hausdorff spaces is a Hausdorff space. A subspace of a Hausdorff space is a Hausdorff space.
\end{proposition}

\begin{comment}
\begin{definition}
A topological space $(X,\mathcal{T})$ is \vocab{metrisable} if it arises from (at least oe) metric space $(X,d)$, i.e. there is at least one metric $d$ on $X$ such that $\mathcal{T}=\mathcal{T}_d$.
\end{definition}

\begin{definition}
Two metrics on a set are \vocab{topologically equivalent} if they give rise to the same topology.
\end{definition}

\begin{example}
\begin{itemize}
\item The metrics $d_1$, $d_2$, $d_\infty$ on $\RR^n$ are all topologically equivalent. (Recall that $d_1$, $d_2$, $d_\infty$ are the metrics arising from the norms $\norm{\cdot}_1$, $\norm{\cdot}_2$, $\norm{\cdot}_\infty$, respectively.)
We shall call the topology defined by the above metrics the \emph{standard} (or canonical) topology on $\RR^n$.
\item The discrete topology on a non-empty set $X$ is metrisable, using the metric
\[d(x,y)=\begin{cases}
0&\text{if }x=y,\\
1&\text{if }x\neq y.
\end{cases}\]
It is easy to check that this is a metric. To see that is gives the discrete topology, consider any subset $U\subset X$. Then for every $x\in U$, $B_\frac{1}{2}(x)\subset U$.
\end{itemize}
\end{example}

\begin{definition}
Given two topologies $\mathcal{T}_1$ and $\mathcal{T}_2$ on the same set, we say $\mathcal{T}_1$ is \vocab{coarser} than $\mathcal{T}_2$ if $\mathcal{T}_1\subset\mathcal{T}2$.
\end{definition}

\begin{remark}
For any space $(X,\mathcal{T})$, the indiscrete topology on $X$ is coarser than $\mathcal{T}$ which in turn is coarser than the discrete topology on $X$.
\end{remark}

\begin{definition}
Let $(X,\mathcal{T})$ be a topological space. A subset $V$ of $X$ is \vocab{closed} in $X$ if $X\setminus V$ is open in X (i.e. $X\setminus V\in\mathcal{T}$).
\end{definition}

\begin{example} \
\begin{itemize}
\item In the space $[0,1)$ with the usual topology coming from the Euclidean metric, $[1/2,1)$ is closed.
\item In a discrete space, all subsets are closed since their complements are open.
\item In the co-finite topology on a set $X$, a subset is closed if and only if it is finite or all of $X$.
\end{itemize}
\end{example}

\begin{proposition}
Let $X$ be a topological space. Then
\begin{enumerate}[label=(\roman*)]
\item $X$, $\emptyset$ are closed in $X$;
\item if $V_1$, $V_2$ are closed in $X$ then $V_1\cup V_2$ is closed in $X$;
\item if $V_i$ is closed in $X$ for all $i\in I$ then $\bigcap_{i\in I}Vi$ is closed in $X$.
\end{enumerate}
\end{proposition}

\begin{proof}
These properties follow from (i), (ii), (iii) of definition of topological space, and from the De Morgan laws.
\end{proof}

\begin{definition}[Convergent sequence]
A sequence $\{x_n\}_{n\in\NN}$ in a topological space $X$ converges to a point $x\in X$ if given any open set $U$ containing $x$ there exists $N\in\NN$ such that $x_n\in U$ for all $n>N$.
\end{definition}

\begin{example} \
\begin{itemize}
\item In a metric space this is equivalent to the metric definition of convergence.
\item In an indiscrete topological space $X$ any sequence converges to any point $x\in X$.
\item In an infinite space $X$ with the co-finite topology any sequence $\{x_n\}$ of pairwise distinct elements (i.e. such that $x_n\neq x_m$ when $n\neq m$) converges to any point $x\in X$.
\end{itemize}
\end{example}
\end{comment}
\fi

%%%%%%%%%%%%%%% MISC
\ifgraph
    \part{Graph Theory}
    A very early theorem of Graph Theory, perhaps even the first, was proved in $1766$ by Euler, concerning a popular problem of the time called ``the bridges of K\"{o}nigsberg''. K\"{o}nigsberg is divided into 4 districts by the river Pregel and has $7$ bridges. The problem was to decide whether it is possible to take a walk that crosses every bridge exactly once. To formulate this problem mathematically, we construct a graph in which there is a vertex for each district and an edge representing each bridge.

    \chapter{Graph Theory}
% ASO: https://courses.maths.ox.ac.uk/pluginfile.php/93815/mod_resource/content/3/GraphTheoryPartA-notes-2023.pdf
% Part B: https://courses.maths.ox.ac.uk/pluginfile.php/25887/mod_resource/content/3/Graph%20Theory%20Lecture%20Notes.pdf

A very early theorem of Graph Theory, perhaps even the first, was proved in 1766 by Euler, concerning a popular problem of the time called ``the bridges of K\"{o}nigsberg''. K\"{o}nigsberg is divided into 4 districts by the river Pregel and has 7 bridges. The problem was to decide whether it is possible to take a walk that crosses every bridge exactly once. To formulate this problem mathematically, we construct a graph in which there is a vertex for each district and an edge representing each bridge.

\section{Introduction to Graphs}
\begin{notation}
$[n]$ denotes the set $\{1,2,\dots,n\}$.
\end{notation}

\begin{notation}
For any set $S$, $\binom{n}{k}$ denotes the set of subsets of $S$ of size $k$; that is,
\[\binom{n}{k}=\{A\subseteq S\mid|A|=k\}.\]
\end{notation}

\begin{definition}[Graph]
A \vocab{graph} is a pair $G=(V(G),E(G))$. The elements of $V(G)$ are called the \vocab{vertices} of $G$; the elements of $E(G)$ are called the \vocab{edges} of $G$.
\end{definition}

$G$ can be represented visually by drawing a point for each vertex and a line between any pair of points that form an edge.

\begin{notation}
$|G|=|V(G)|$ denotes the number of vertices; $e(G)=|E(G)|$ denotes the number of edges.
\end{notation}

\begin{definition}
The \vocab{order} of a graph $G$ is $|V(G)|$. The \vocab{size} of $G$ is $|E(G)|$.
\end{definition}

\begin{definition}
The \vocab{complement} of $G$, denoted by $\overline{G}$, is a graph with the same vertex set as $G$ and $E(\overline{G}) = \{e \notin E(G)\}$; that is, $\overline{G}$ has edges exactly where there are no edges in $G$.
\end{definition}

\begin{notation}
We write $uv=\{u,v\}=vu$ for the (unordered) pair representing an edge between $u$ and $v$.
\end{notation}

\begin{definition}[Simple graph]
A \vocab{loop} is an edge $vv$ for some $v \in V$. An edge $uv$ is a \vocab{multiple edge} if it appears more than once in $E$. 

A graph is \vocab{simple} if it has no loops or multiple edges.
\end{definition}

\begin{remark}
Unless explicitly stated otherwise, we will only consider simple graphs. General (potentially non-simple) graphs are also called multigraphs.
\end{remark}

\begin{definition}
Vertices $u$ and $v$ are \vocab{neighbours} if $uv\in E(G)$; we also say that $u$ and $v$ are \emph{adjacent}. An edge $e\in E(G)$ is \vocab{incident} to a vertex $v \in V(G)$ if $v\in e$. Edges $e$ and $e^\prime$ are \vocab{incident} if $e\cap e^\prime=\emptyset$.
\end{definition}

\begin{definition}
Given a vertex $v$, the \vocab{degree} of $v$, denoted by $d(v)$, is the number of neighbours of $v$ in $G$. If the degree of each vertex is the same, we can call that the degree of the graph.

A \vocab{leaf} is a vertex of degree one, i.e.\ with a unique neighbour.
\end{definition}

\begin{remark}
A trivial graph is a graph with order $1$. An empty graph is a graph of size $0$. Note that a graph must have at least one vertex by definition. But a graph can certainly have no edges!
\end{remark}

\begin{definition}[Subgraph]
$H$ is a \vocab{subgraph} of $G$ if $V(H) \subseteq V(G)$ and $E(H) \subseteq E(G)$.
\begin{itemize}
\item $H$ is a \vocab{spanning subgraph} if $V(H)=V(G)$.
\item $H$ is an \vocab{induced subgraph} if $uv\in E(H)\iff uv\in E(G)$ for all $u,v\in V(H)$.
\end{itemize}
\end{definition}

\begin{definition}[Walk]
A \vocab{$u-v$ walk} in $G$, denoted by $W$, is a finite sequence of vertices 
\[u=v_0,v_1,\dots,v_k=v\]
such that $v_i v_{i+1} \in E(G)$ for all $0 \le i < k-1$.
\begin{itemize}
\item If the vertices in $W$ are distinct, we call it a \vocab{path}.
\item If $u=v$, we call $W$ a \vocab{closed walk}; otherwise, it is an \vocab{open walk}.
\item A \vocab{trail} is an open walk without repeating vertices.
\item A \vocab{path} is an open walk without repeating edges. Note that paths are trails, but not vice-versa.
\item A \vocab{circuit} is a closed walk without repeating edges, i.e. $u=v$, it begins and ends with the same vertex.
\item A \vocab{cycle} is a closed walk without repeating vertices, other than the initial and terminal vertices, i.e. $u=v$ but the vertices are otherwise distinct and $W$ has at least 3 vertices. If a graph $G$ has no cycle we call it \vocab{acyclic}.
\item A \vocab{trail} is a walk in which no two vertices appear consecutively (in either order) more than once; that is, no edge is used more than once. A \vocab{tour} is a closed trail.
\end{itemize}
\end{definition}

\begin{definition}[Connectedness]
A graph $G$ is \vocab{connected} if there exists a $u-v$ path for all $u,v\in V(G)$.

Vertices $u,v\in V(G)$ \vocab{lie in the same component} if they are joined by a $u-v$ walk. Clearly this forms an equivalence relation and the partition of $V(G)$ into equivalence classes expresses $G$ as a union of disjoint connected graphs called its \vocab{components}.
\end{definition}

\begin{definition}[Diameter]
The \vocab{diameter} of a connected graph $G$, denoted by $\diam G$, is defined as
\[\diam G=\max_{u\neq v}d(u,v).\]
By construction, $d(u,v) \le \diam G$ for all $u,v \in V(G)$.
\end{definition}

\begin{definition}[Complete graph]
$G$ is \vocab{complete} if every pair of vertices in $G$ is joined by an edge. A complete graph on $n$ vertices is denoted by $K_n$.
\end{definition}

\begin{definition}
A graph $G$ is \vocab{planar} if it can be drawn such that a pair of edges can only cross at a vertex.
\end{definition}

\begin{theorem}[Euler's Characteristic Formula]
For any connected planar graph, with $V$ vertices, $E$ edges and $F$ faces (regions bounded by edges, including the outer, infinitely large region), then
\begin{equation}
V-E+F=2.
\end{equation}
\end{theorem}

\begin{proof}
We prove by induction.

If $G$ has zero edges, that is $E=0$, then $V=1$ and $F=1$. Then $V-E+F=1-0+1=2$.

Suppose Euler's formula holds for a graph with $n$ edges.

For a graph $G$ with $n+1$ edges, we now consider two cases:

\textbf{Case 1}: If $G$ is a tree and does not contain any cycle. It can be easily proven by induction for trees with any number of edges.

\textbf{Case 2}: If $G$ is not a tree and contains at least one cycle. Choose an edge $e_1$ in $G$ which divides the given region into two different parts and remove that edge $e_1$ to get another graph $G^\prime$. Note that $F>F^\prime$.

Now, $G$ has $n+1$ edges, then $G^\prime$ has $n$ edges so by the hypothesis $G^\prime$ satisfies the Euler's formula. For $G^\prime$, $V^\prime-E^\prime+F^\prime=2$ where $V^\prime=V$, $E^\prime=E-1$ and $F^\prime=F-1$.

Substituting these values gives
\begin{align*}
V^\prime-E^\prime+F^\prime&=2\\
V-E+1+F-1&=2\\
V-E+F&=2
\end{align*}

Hence, Euler's formula is applicable for $n+1$ edges.

This proves the Euler's formula.
\end{proof}

\subsection{Trees}
\begin{definition}[Tree]
A \vocab{tree} is a connected graph that does not contain any cycles; that is, it is a minimally connected graph.
\end{definition}

\begin{proposition}\label{prop:tree-acyclic}
Any tree is acyclic.
\end{proposition}

\begin{proof}
Let $G$ be a tree, i.e. $G$ is minimally connected.

Suppose, for a contradiction, that $G$ contains a cycle $C$. Let $e \in E(C)$. We will obtain our contradiction by showing that $G-e \coloneqq (V(G),E(G)\setminus\{e\})$ is connected. 

Let $P$ be the path obtained by deleting $e$ from $C$. Consider any $u,v$ in $V(G)$. As $G$ is connected, there is an $u-v$ walk $W$ in $G$. Replacing any use of $e$ in $W$ by $P$ gives an $u-v$ walk in $G-e$. Thus $G-e$ is connected, a contradiction.
\end{proof}

The following are equivalent characterisations of trees.
\begin{lemma}[Characterisation of trees]\label{lemma:tree-char} \
\begin{enumerate}[label=(\roman*)]
\item $G$ is a tree if and only if $G$ is connected and acyclic.
\item Any two vertices in a tree are joined by a unique path.
\end{enumerate}
\end{lemma}

\begin{proof} \
\begin{enumerate}[label=(\roman*)]
\item If $G$ is a tree then $G$ is connected and acyclic.

Conversely, let $G$ be connected and acyclic. Suppose for a contradiction that $G-e$ is connected for some $e = (u,v) \in E(G)$.

Let $W$ be a shortest $u-v$ walk in $G-e$. Then $W$ must be a path, i.e. have no repeated vertices, otherwise we would find a shorter walk by deleting a segment of $W$ between two visits to the same vertex. Combining $W$ with $(u,v)$ gives a cycle, which is a contradiction.

\item Suppose for a contradiction that this fails for some tree $G$. 

Choose $u,v$ in $V(G)$ so that there are distinct $u-v$ paths P1, P2, and P1 is as short as possible over all such choices of $u$ and $v$.

Then P1 and P2 only intersect in $u$ and $v$, so their union is a cycle, contradicting \cref{prop:tree-acyclic}.
\end{enumerate}
\end{proof}

\begin{remark}
The fact that a shortest walk between two points is a path is often useful. More generally, considering an extremal (shortest, longest, minimal, maximal, \dots) object is often a useful proof technique.
\end{remark}

\begin{lemma}\label{tree_leaves}
Any tree with at least two vertices has at least two leaves.
\end{lemma}
\begin{proof}
Consider any tree $G$. Let $P$ be a longest path in $G$. The two ends of $P$ must be leaves. Indeed, an end cannot have a neighbour in $V(G) \setminus V(P)$, or we could make $P$ longer, and cannot have any neighbour in $V(P)$ other than the next in the sequence of $P$, or we would have a cycle.

The existence of leaves in trees is useful for inductive arguments, via the following lemma. Given $v \in V(G)$, let $G-v$ be the graph with $V(G-v) = V(G)\setminus\{v\}$ and $E(G-v) = \{(u,v) \in E(G) \mid v \notin \{u,v\}\}$.
\end{proof}

\begin{lemma}\label{tree_leaf}
If $G$ is a tree and $v$ is a leaf of $G$ then $G-v$ is a tree.
\end{lemma}
\begin{proof}
By \cref{lemma:tree-char} (i), it suffices to show that $G-v$ is connected and acyclic. Acyclicity is immediate from \cref{prop:tree-acyclic}. Connectedness follows by noting for any $u,v \in V(G)\setminus\{v\}$ that the unique $u-v$ path in $G$ is contained in $G-v$.
\end{proof}

\begin{lemma}\label{tree_edges}
Any tree on $n$ vertices has $n-1$ edges.
\end{lemma}
\begin{proof}
By induction. A tree with 1 vertex has 0 edges. Let $G$ be a tree on $n>1$ vertices. By \cref{tree_leaves}, $G$ has a leaf $v$. By \cref{tree_leaf}, $G-v$ is a tree. By induction hypothesis, $G-v$ has $n-2$ edges. Replacing $v$ gives $n-1$ edges in $G$.
\end{proof}

We conclude this section with another characterisation of trees. First we note that any connected graph $G$ contains a minimally connected subgraph (i.e. a tree) with the same vertex set, which we call a \vocab{spanning tree} of $G$.

\begin{lemma}
A graph $G$ is a tree on $n$ vertices if and only if $G$ is connected and has $n-1$ edges.
\end{lemma}

\begin{proof}
If $G$ is a tree then $G$ is connected by definition and has $n-1$ edges by \cref{tree_edges}. 

Conversely, suppose that $G$ is connected and has $n-1$ edges. Let $H$ be a spanning tree of $G$. Then $H$ has $n-1$ edges by \cref{tree_edges}, so $H = G$, so $G$ is a tree.
\end{proof}

\subsection{Bipartite Graphs, Matching}
\begin{definition}[Bipartite graph]
A graph $G$ is \vocab{bipartite} if $V(G)$ can be partitioned into two non-empty disjoint sets $A$ and $B$ such that no edge has both endpoints in the same set.

$G$ is said to be \vocab{complete bipartite} if $G$ is bipartite and all possible edges between the two sets $A$ and $B$ are drawn. In the case where $|A|=m, |B|=n$, such a graph is denoted by $K_{m,n}$.

Let $k \ge 2$. A graph $G$ is said to be \vocab{$k$-partite} if $V(G)$ can be partitioned into $k$ pairwise disjoint sets $A_1, \dots, A_k$ such that no edge has both endpoints in the same set.

A \vocab{complete $k$-partite} graph is defined similarly as a complete bipartite. In the case where $|A_i| = n_i$, such a graph is denoted by $K_{n_1,n_2,\dots,n_k}$.
\end{definition}

The Marriage Problem is as follows:

\begin{quote}
Given $n$ men and $n$ women, under what conditions is it possible to pair each man with a woman such that every pair know each other?
\end{quote}

We now mathematically formulate this problem. Let $G$ be a graph. We say $M\subseteq E(G)$ is a \emph{matching} if the edges in $M$ are pairwise disjoint. We say $M$ is \emph{perfect} if every vertex belongs to some edge of $M$. 
In this terminology, the marriage problem asks when a bipartite graph has a perfect matching.

We will return to this question later. First we consider the algorithmic question of how to find a matching of maximum size.

\begin{theorem}[K\"{o}nig's Theorem]
In any bipartite graph, the size of a maximum matching equals the size of a minimum cover.
\end{theorem}

\begin{theorem}[Hall's Theorem]
Let $G$ be a bipartite graph with parts $A$ and $B$. Then $G$ has a matching covering $A$ if and only if $|N(S)|\ge|S|$ for every $S\subseteq A$.
\end{theorem}

\subsection{Isomorphism}
\begin{definition}[Isomorphism]
Let $G_1 = (V_1, E_1)$ and $G_2 = (V_2, E_2)$ be graphs. An isomorphism $\phi : G_1 \to G_2$ is a bijection (a one-to-one correspondence) from $V_1$ to $V_2$ such that $(u,v) \in E_1$ if and only if $(\phi(u),\phi(v)) \in E_2$. We say $G_1$ is isomorphic to $G_2$ if there is an isomorphism between them.
\end{definition}

\subsection{Minimum Cost Spanning Trees}
$G$ is a connected graph and we have some cost $c(e)>0$ for every edge $e\in E(G)$. For any $S\subseteq E(G)$ we call $c(S)=\sum_{e\in S}c(e)$ the cost of $S$. The problem we wish to solve is:

\begin{quote}
Find $S\subseteq E(G)$ with minimum possible $c(S)$ such that $(V(G),S)$ is a connected graph.
\end{quote}

A solution is necessarily a spanning tree for $G$. Recall that this is a tree $T=(V(G),S)$ where $S\subseteq E(G)$. We say that $T$ is a minimum cost spanning tree of $G$ if any other spanning tree $T^\prime$ satisfies $c(T^\prime)\ge c(T)$. How can we find one efficiently?

One natural method to try is the ``greedy algorithm'': choose edges one at a time, each time choosing the cheapest edge that does not create a cycle. There are various versions of this algorithm; we will describe the one due to Kruskal.

\begin{theorem}[Kruskal's Algorithm]
At step $i\ge0$, keep track of a subset $A_i\subseteq E(G)$. This will have the property that $(V(G),A_i)$ is acyclic. Start with $A_0=\emptyset$. At step $i\ge0$, is there an edge $e\in E(G)\setminus A_i$ such that $(V(G),A_i\cup\{e\})$ is acyclic? If no, then output $A=A_i$ and stop. If yes, then set $A_{i+1}=A_i\cup\{e\}$ for one such $e$ such that $c(e)$ is minimal, and proceed to step $i+1$.
\end{theorem}

You should try a few examples on small graphs to understand the algorithm and check that it does find minimum cost spanning trees in your examples. However, it is not obvious that it will always work, and indeed there are different problems in graph theory for which greedy algorithms don't always work. Fortunately, the greedy algorithm always works for the minimum cost spanning tree problem, as shown by the following theorem.

\begin{theorem}
$(V(G),A)$ is a minimum cost spanning tree of $G$.
\end{theorem}

\begin{proof}

\end{proof}

\subsection{Euler Tours and Trails}
\begin{definition}
Let $W$ be a walk in a graph $G$. We call $W$ an \vocab{Euler trail} if every edge of $G$ appears exactly once in $W$. 

We call $W$ an \vocab{Euler tour} if it is closed, i.e. it starts and ends at the same vertex.
\end{definition}

For a graph $G$ with an Euler tour $W$, clearly $G$ must be connected after we delete all isolated vertices (i.e. vertices of degree zero). We also note that each visit of $W$ to a vertex v uses two edges at v (one to arrive and one to leave). This is also true of the start and end vertex of W if we consider them to be a single visit. (Or we can think of the vertex sequence of W as being written around a circle rather than along a line, so that there is no start or end, and each visit uses two edges.) As every edge is used exactly once, we deduce that every vertex has even degree; we call a graph with this property \vocab{Eulerian}.

\begin{lemma}
In any graph, there are an even number of vertices with odd degree.
\end{lemma}

\begin{proof}
Since every edge has two endpoints,
\[\sum_{v\in V(G)}d(v)=2|E(G)|.\]
Therefore, in the sum, there must be an even number of occurrences of $d(v)$ for which $d(v)$ is odd.
\end{proof}

An Euler tour can be found efficiently using \vocab{Fleury's Algorithm}:
\begin{quote}
Start at any vertex. We will follow a walk, erasing each edge after it is used (erased edges cannot be used again). At each stage, ensure that the following holds:
\begin{enumerate}[label=(\roman*)]
\item when the edge is removed, the resulting graph is connected once isolated vertices are removed, and
\item we do not run along an edge to a leaf, unless this is the only edge of the graph.
\end{enumerate}
\end{quote}

\begin{theorem}[Euler]
Let $G$ be a connected Eulerian graph. Then $G$ has an Euler tour.
\end{theorem}

\subsection{Hamiltonian Cycles}
In the previous section, we investigated graphs that admit an Euler tour, which is a closed walk that traverses every edge exactly once. There is a seemingly related question that one can ask about a graph: does there exists a closed walk that visits every vertex exactly once?

\begin{definition}[Hamiltonian cycle]
A \vocab{Hamiltonian cycle} is a closed walk that visits every vertex of a graph $G$ once.

When a graph $G$ contains such a cycle, it is said to be \vocab{Hamiltonian}.
\end{definition}

\begin{theorem}[Ore's Theorem]
Let $G$ be a connected graph with $n\ge3$ vertices. Suppose that for every pair of non-adjacent vertices $x$ and $y$, $d(x)+d(y)\ge n$. Then $G$ is Hamiltonian.
\end{theorem}

\begin{proof}

\end{proof}

\begin{corollary}[Dirac's Theorem]
If $G$ is connected with $n\ge3$ vertices and for every vertex $v$, $d(v)\ge\frac{n}{2}$, then $G$ is Hamiltonian.
\end{corollary}

\subsection{Shortest Paths}
How do you find the quickest route from A to B? Maybe you ask your satnav, but how does your satnav find the route? It doesn't check all options, as there are too many: it uses an efficient algorithm. We formulate the problem mathematically as follows:

\begin{quote}
Let $G$ be a connected graph. Let $\ell(e)>0$ be the ``length'' of the edge $e$ for $e\in E(G)$. The $\ell$-length of a path $P$ is $\ell(P)=\sum_{e\in E(P)}\ell(e)$.

Given $x,y\in V(G)$, an $\ell$-shortest $xy$-path is an $xy$-path $P$ that minimises $\ell(P)$.
\end{quote}

The \vocab{Dijkstra's Algorithm} is a method of finding a $\ell$-shortest $xy$-path. The idea of the algorithm is to maintain a ``tentative distance from $x$'' called $D(v)$ for each $v\in V(G)$. At each step of the algorithm we finalise $D(u)$ for some vertex $u$. At the end of the algorithm all $D(u)$ will be equal to the correct value, i.e. $D(u)=\ell(P_u^*)$ for some $\ell$-shortest $xu$-path $P_u^*$.

\subsection{Chinese Postman Problem}

\subsection{Colouring}
vertex/edge colouring and Ramsey Theory
graph colouring and the four-colour theorem; Ramsey theory; 

\subsection{Tur\'{a}n graph}


Menger's theorem, network flows; complete subgraphs and Turán's theorem, and the Erdös-Stone theorem; probabilistic methods in graph theory
\pagebreak

\section*{Exercises}
Problems can include tournament, matching, and scheduling problems.
\begin{prbm}[K\"{o}nigsberg Bridge Problem]
K\"{o}nigsberg was a small town in Prussia. There is a river running through the town and there were seven bridges across the river. The inhabitants of K\"{o}nigsberg liked to walk around the town and cross all of the bridges:

Is it possible to walk around the town and cross every bridge, once and once only?
\end{prbm}

\begin{solution}
We replace every landmass by a vertex and every bridge by an edge to give the following graph.
\end{solution}

\begin{prbm}(Moser's circle problem) 
Determine the number of regions into which a circle is divided if $n$ points on its circumference are joined by chords with no three internally concurrent.
\end{prbm}

\begin{proof}[Solution]
Consider the graph which has points on the circumference and intersection points between chords as its vertices.

Let $V, E, F$ denote the number of vertices, edges, regions respectively.

To count the number of intersection points, note that $4$ points on the circumference give one unique intersection point between the two non-parallel chords formed by connecting two pairs of points which intersect inside the circle. Hence, number of intersection points is $\dbinom{n}{4}$. 
\[ V = n + \binom{n}{4} \]
Total number of edges includes $n$ circular arcs, number of original chords formed from connecting pairs of points on the circumference
E = no. of original lines + 2 x no. of intersection points
E = n choose 2 + 2 x n choose 4 + n
since there are n circular arcs

Using Euler's Characteristic Formula, we have 
\begin{align*}
F &= E - V - 1 \\
\Aboxed{F &= 1 + \binom{n}{2} + \binom{n}{4}}
\end{align*}
\end{proof}

\begin{comment}
\chapter{Game Theory}
\textbf{Recommended readings:} \href{https://mathematicalolympiads.files.wordpress.com/2012/08/martin_j-_osborne-an_introduction_to_game_theory-oxford_university_press_usa2003.pdf}{``An Introduction to Game Theory" by Osborne}
%http://www.matt-versaggi.com/mit_open_courseware/GameAI/MathematicalGameTheoryandApplications.pdf

% games of normal form and extensive form, and their applications in economics, relations between game theory and decision making; games of complete information: static games with finite or infinite strategy spaces, Nash equilibrium of pure and mixed strategy, dynamic games, backward induction solutions, information sets, subgame-perfect equilibrium, finitely and infinitely-repeated games; games of incomplete information: Bayesian equilibrium, first price sealed auction, second price sealed auction, and other auctions, dynamic Bayesian games, perfect Bayesian equilibrium, signaling games; cooperative games: bargaining theory, cores of n-person cooperative games, the Shapley value and its applications in voting, cost sharing, etc.

\section{Strict Dominance}
\subsection{Prisoner's Dilemma}
To start off, we will take a look at the \vocab{Prisoner's Dilemma}, which goes as follows:

\begin{quote}
Two thieves plan to rob a store, but the police arrest them for trespassing. The police suspect that they planned to break in but lack the evidence to support such an accusation. They require a confession to charge the suspects. The police offer them the following deal:
\begin{itemize}
\item If no one confesses, both are charged a one month jail sentence each for trespassing.
\item If a rat confesses and the other does not, the rat is not charged but the other is charged a twelve month jail sentence for robbery.
\item If both confess, both are charged an eight month jail sentence each.
\end{itemize}
If both criminals are self-interested and only care about minimising their jail time, should they take the interrogator's deal?
\end{quote}

We condense the above information into a \vocab{payoff matrix} as shown below, where we have two players, A and B. The horizontal rows represent A's choices, while the vertical columns represent B's choices, and each cell contains a combination of their payoffs.

\begin{table}[H]
\centering
\begin{tabular}{rcc}
\multicolumn{1}{l}{}         & quiet                       & confess                     \\ \cline{2-3} 
\multicolumn{1}{r|}{quiet}   & \multicolumn{1}{c|}{$-1$, $-1$} & \multicolumn{1}{c|}{$-12$, $0$} \\ \cline{2-3} 
\multicolumn{1}{r|}{confess} & \multicolumn{1}{c|}{$0$, $-12$} & \multicolumn{1}{c|}{$-8$, $-8$} \\ \cline{2-3} 
\end{tabular}
\end{table}

\subsection{Split or Steal}
The game goes as follows:
\begin{quote}
Each of two players, Sarah and Steve, has to pick one of two balls: inside one ball appears the word ``\textbf{split}'' and inside the other the word ``\textbf{steal}'' (each player is first asked to secretly check which of the two balls in front of him/her is the split ball and which is the steal ball). They make their decisions simultaneously. 
\end{quote}

The possible outcomes are shown in the figure below, where each row is labelled with a possible choice for Sarah and each column with a possible choice for Steven. Each cell in the table thus corresponds to a possible pair of choices and the resulting outcome is written inside the cell.

\begin{figure}[H]
    \centering
    \includegraphics[width=12cm]{images/Split_or_steal.png}
\end{figure}

\section{Nash Equilibrium}
\vocab{Nash Equilibrium} is a set of optimal strategies that work against \textit{all} counter-steategies. This means that if any given player were told the strategies of all their opponents, they still would choose to retain their original strategy. 

\subsection{Matrix games}

\section{Fair Division}
\subsection{Rental harmony problem}
Sperner's lemma

%https://www.cs.cmu.edu/~arielpro/15896/docs/paper19b.pdf
\end{comment}
\fi
\pagebreak

\backmatter
\printbibliography[title={Bibliography}]

\printindex
\end{document}